\documentclass[11pt,a4paper]{article}
\usepackage{amsmath, amssymb, fullpage, mathrsfs, bm, pgf, tikz}
\usepackage{mathrsfs}
\usetikzlibrary{arrows}
\setlength{\textheight}{10in}
%\setlength{\topmargin}{0in}
\setlength{\topmargin}{-0.5in}
\setlength{\parskip}{0.1in}
\setlength{\parindent}{0in}

\begin{document}
\newcommand{\la}{\leftarrow}
\newcommand{\lra}{\leftrightarrow}


\title{Solution to IMO 2016 shortlisted problems.}
\author{Anzo Teh}
\date{25 July 2017}
\maketitle
\section{Algebra}
\begin{itemize}
\item[\textbf{A1}]
Let $a$, $b$, $c$ be positive real numbers such that $\min(ab,bc,ca) \ge 1$. Prove that $$\sqrt[3]{(a^2+1)(b^2+1)(c^2+1)} \le \left(\frac{a+b+c}{3}\right)^2 + 1.$$

\textbf{Thoughts.} 
Non-homogeneous inequality, with the relation $\min(ab,bc,ca) \ge 1$ that we aren't sure how to use it (at its first sight). 
How should we solve it, then? 
The way, when desperate, is to brute force the whole thing by cubing the left-hand side and trying to expand 
$\left(\frac{a+b+c}{3}\right)^6+3\left(\frac{a+b+c}{3}\right)^4+3\left(\frac{a+b+c}{3}\right)^2+1$, 
but let's not make our lives miserable with (probably) hundreds of terms on the right. 

Instead, we start with the following observation: 
\begin{itemize}
\item [1.] What happens when $a=b=c$? Then the equality holds! 
This motivates us to show that whenever $a+b+c$ is fixed, the maximal possible value of $(a^2+1)(b^2+1)(c^2+1)$ can be attained whenever $a=b=c$. With this in mind...
\item [2.] We want to see what happens to $(a^2+1)(b^2+1)$ by ranging all possible pairs of $(a,b)$ such that $ab\ge 1$ and $a+b$ is fixed. 
As it turns out, $(a^2+1)(b^2+1)=a^2b^2+a^2+b^2+1$ (there are only four terms so it doesn't hurt to expand). 
Since $a+b$ is fixed in our context, we can write this in terms of $a+b$, giving 
$a^2b^2+a^2+b^2+1=(a+b)^2+(ab-1)^2$. 
Now it's easy to see that this value increases with $ab$ (as $ab\ge 1$). 
\item [3.] Now with $a+b$ fixed, $ab$ increases when $|a-b|$ decreases. 
The next step is to compare $f(a,b,c)$ with $f(k,k,k)$ where $k$ is the average of $a,b,c$. 
(Here we denote $f(a,b,c)$ as $(a^2+1)(b^2+1)(c^2+1)$). 
Ideally, we want to find $x$ and $y$ with average $k$ such that 
$f(a,b,c)\le f(k,x,y)\le f(k,k,k)$. 
The right inequality is easy to maintain. 
The left inequality, however, requires some cas-by-case analysis (such as $m(a,b,c)\le k$ and $m(a,b,c)\ge k$, where $m$ denotes the median of the three variables here). 
And finally we also need to check that $\min(kx, ky, xy)$ is maintained.
\end{itemize}

\textbf{Solution.} 
We start with a preliminary observation: 
given that $k\ge 2$, and given the set of pairs $K=\{(a,b): a+b=k, ab\ge 1\}$, 
then for any $(a_1,b_1), (a_2, b_2)\in K$, 
$(a_1^2+1)(b_1^2+1)\ge (a_2^2+1)(b_2^2+1)$ iff $|a_1-b_1|\le |a_2-b_2|$. 
Indeed, for $(a,b)\in K$, $(a^2+1)(b^2+1)=(a+b)^2+a^2b^2-2ab+1$
$=k^2+(ab-1)^2$
$=k^2+(\frac{(a+b)^2-(a-b)^2}{2}-1)^2$
$=k^2+(\frac{k^2-(a-b)^2}{2}-1)^2$, 
and given that $ab\ge 1$, this function is increasing in $ab$. 
In addition, with $a+b$ fixed, this function is also dereasing in $(a-b)^2$, which turns out to also be decreasing in $|a-b|$. 

Now let $a+b+c=3k$, and let $f(a,b,c)=(a^2+1)(b^2+1)(c^2+1)$. 
W.l.o.g. assume that $a\le b\le c$. 
Let $b\le k$, then by above, $(b^2+1)(c^2+1)\le (k^2+1)((b+c-k)^2+1)$ because 
$b+c\ge 2k$ ($a\le k$ and $c\ge k$), and 
$(b+c-k)-k=(b+c)-2k\le (b+c)-2b=c-b$ (since $b\le k$), 
which follows that $k(b+c-k)\ge bc\ge 1$. 
Likewise, if $b\ge k$ then by above, 
$(a^2+1)(b^2+1)\le ((a+b-k)^2+1)(k^2+1)$ because 
$b+a\le 2k$ ($a\le k$ and $c\ge k$), and 
$k-(a+b-k)=2k-(b+c)\le 2b-(a-b)=b-a$ (since $b\ge k$), 
which follows that $k(a+b-k)\ge ab=1$. 
Additionally, after the operation, we have $a\le k\le b+c-k$ in the first case, and $a+b-k\le k\le c$ 
(a good question to ask might be: what if $a+b-k < 0$ in the second case? This is impossible because we only do this when $b\ge k$). 
So we only need to verify that the first two has product at least one. 
In case 1, $ak\ge ab=1$ and in case 2 we have already verified that $k(a+b-k)\ge ab=1$. 
Thus $f(a,b,c)\le f(a, k, (b+c-k))$ in case 1, or $f(a,b,c)\le f((a+b-k), k, c)$ in case 2. 
This means we can focus on the case where $b=k$. 
Nevertheless, when $b=k$ we have $a+c=2k$ and by similar procedure we have $(a^2+1)(c^2+1)\le (k^2+1)^2$, 
therefore we have: 
$f(a,b,c)\le f(a, k, (b+c-k))\le f(k,k,k)$ in case 1, 
$f(a,b,c)\le f((a+b-k), k, c)\le f(k,k,k)$ in case 2, 
and $f(k,k,k)=(k^2+1)^3$ is precisely the cube the right hand side. Q.E.D. 

\item[\textbf{A2}]
 Find the smallest constant $C > 0$ for which the following statement holds: among any five positive real numbers $a_1,a_2,a_3,a_4,a_5$ (not necessarily distinct), one can always choose distinct subscripts $i,j,k,l$ such that
\[ \left| \frac{a_i}{a_j} - \frac {a_k}{a_l} \right| \le C. \]

\textbf{Thoughts.} 
By arranging $a_i\le a_j$ and $a_k\le a_l$ we know that $C$ is bounded above by 1. A baby step, but a great start.

The next sensible thing we can do is to sort the numbers in order: 
$a_1\le a_2\le a_3\le a_4\le a_5$. 
Also it's important to realize that $|\frac{a_1}{a_2}-\frac{a_4}{a_5}|=|\frac{a_1a_5-a_2a_4}{a_2a_5}|\ge |\frac{a_1a_5-a_2a_4}{a_4a_5}|=|\frac{a_1}{a_4}-\frac{a_2}{a_5}|$ so we just have to consider the latter. 
In a similar fashion let's consider $|\frac{a_1}{a_3}-\frac{a_2}{a_4}|$, 
$|\frac{a_2}{a_4}-\frac{a_3}{a_5}|$. 
As seen below, this is just a comparison among 
$\frac 1{bc}|\frac 1a-\frac 1d|$, $\frac 1b|\frac 1a-\frac 1c|$, $\frac 1c|\frac 1b-\frac 1d|$ 
(as below  $a=\frac {a_2}{a_1}$, 
$b=\frac {a_3}{a_2}$, 
$c=\frac {a_4}{a_3}$, 
$d=\frac {a_5}{a_4}$). 
It's difficult to see how great the minimum of the three numbers can go, 
but in light of the factors $\frac 1b$ and $\frac 1c$ we can try some simple cases like $b=c=1$, 
giving $|\frac 1a-\frac 1d|$, $|\frac 1a-1|$, $|1-\frac 1d|$. 
Considering $0\le \frac 1a, \frac 1d\le 1$ we have 
$(1-\frac 1a)+(\frac 1a-\frac 1d)=(1-\frac 1d)$ 
so considering that $0\le 1-\frac 1d, 1-\frac 1a\le 1$ and assuming that $\frac 1a-\frac 1d\ge 0$ we have 
$1\ge 1-\frac 1d\ge 2\min\{1-\frac 1a, \frac 1a-\frac 1d\}$, which means one of the elements in the set must be at most $\frac 12$. 
A similar conclusion can be reached for the case $\frac 1a-\frac 1d\le 0$. 
Moreover, this motivates the following equality case: by setting $a=2$ and $d$ approaching infinity (and yeah, this is how the example $1, 2, 2, 2, n$ can be conjured). 
The job now reduces to proving that $\min \{\frac 1{bc}|\frac 1a-\frac 1d|, \frac 1b|\frac 1a-\frac 1c|, \frac 1c|\frac 1b-\frac 1d|\}\le \frac 12$, which is no longer hard by case-by-case analysis as below. 

\textbf{Solution.} The answer is $\frac 12$. 

First, let $a_1=1, a_2=a_3=a_4=2$, $a_5=n$ for some extremely large real number $n$. 
The possible numbers of $\frac {a_i}{a_j}$ are $\frac 1n, \frac 2n, \frac 12, 1, 2, \frac n2, n$. 
Observe that the ratios $\frac 1n, \frac 2n, \frac n2, n$ all involve $a_5$, so there cannot exist distinct $i, j, k, l$ satisfying 
$\frac{a_i}{a_j}, \frac {a_k}{a_l}\in\{\frac 1n, \frac 2n, \frac n2, n\}$. 
In addition, both $\frac 12$ and 2 will involve $a_1$, meaning that there cannot exist distinct $i, j, k, l$ satisfying 
$\frac{a_i}{a_j}, \frac {a_k}{a_l}\in\{\frac 12, 2\}$. 
Since there are three $i$'s satisfying $a_i=2$, there cannot be distinct $i,j,k,l$ satisfying $\frac{a_i}{a_j}=\frac {a_k}{a_l}=1$. 
We therefore know that  $\frac{a_i}{a_j}=\frac {a_k}{a_l}$ is impossible, 
and same goes to $\frac{a_i}{a_j}=\frac 1n, \frac {a_k}{a_l}=\frac 2n$. 
We also have $n-\frac n2> \frac n2 -2 > 2-1>1-\frac 12>\frac 12-\frac 2n$ for sufficiently large real $n$, 
and $\lim_{n\to\infty}\frac 12-\frac 2n=\frac 12$. 
Therefore $C\ge \frac 12$. 

Now, we show that $C=\frac 12$ fits in all situations. 
W.L.O.G. let $a_1\le a_2\le a_3\le a_4\le a_5$, and let $a=\frac {a_2}{a_1}$, 
$b=\frac {a_3}{a_2}$, 
$c=\frac {a_4}{a_3}$, 
$d=\frac {a_5}{a_4}$. 
Observe that $a, b, c, d\ge 1$. 
Suppose that for some $a,b,c,d$, $C=\frac 12$ does not fit for any of the four distinct subscripts. 
Now, considering $|\frac {a_1}{a_4}-\frac {a_2}{a_5}|$
$=\frac 1{bc}|\frac 1a-\frac 1d|$
and from $0\le \frac 1a, \frac 1d\le 1$
we have $\frac bc\ge \frac bc|\frac 1a-\frac 1d|>C=\frac 12$
so $b, c<2$. 
Next, $C<|\frac {a_1}{a_3}-\frac {a_2}{a_4}|$
$=\frac 1b|\frac 1a-\frac 1c|$
$\le |\frac 1a-\frac 1c|$
and from $\frac 1c<\frac 12$ we must have $\frac 1a>\frac 12$. 
Similarly, $C<|\frac {a_2}{a_4}-\frac {a_3}{a_5}|$
$=\frac 1c|\frac 1b-\frac 1d|$
$\le |\frac 1b-\frac 1d|$
and from $\frac 1b<\frac 12$ we must have $\frac 1d>\frac 12$. 
Looking back, we have $\frac 12<|\frac {a_1}{a_4}-\frac {a_2}{a_5}|=\frac bc|\frac 1a-\frac 1d|\le|\frac 1a-\frac 1d|$, 
yet $\frac 12<\frac 1a , \frac 1d\le 1$, contradiction.

\item[\textbf{A3}]
Find all positive integers $n$ such that the following statement holds: Suppose real numbers $a_1$, $a_2$, $\dots$, $a_n$, $b_1$, $b_2$, $\dots$, $b_n$ satisfy $|a_k|+|b_k|=1$ for all $k=1,\dots,n$. Then there exists $\varepsilon_1$, $\varepsilon_2$, $\dots$, $\varepsilon_n$, each of which is either $-1$ or $1$, such that
\[ \left| \sum_{i=1}^n \varepsilon_i a_i \right| + \left| \sum_{i=1}^n \varepsilon_i b_i \right| \le 1. \]

\textbf{Thoughts.} 
One good counterexample construction can be found using the triangle inequality: 
$ \left| \sum_{i=1}^n \varepsilon_i a_i \right| + \left| \sum_{i=1}^n \varepsilon_i b_i \right|\ge  \left| \sum_{i=1}^n \varepsilon_i a_i + \sum_{i=1}^n \varepsilon_i b_i \right|$. 
Therefore, for the case where $a_i, b_i\ge 0$ we have: 
$a_i+b_i=1$ 
so the right hand side now becomes $\left| \sum_{i=1}^n \varepsilon_i \right|$. 
When $n$ is even, we know that the condition holds only when $\sum_{i=1}^n \varepsilon_i$ is 0, which means exactly hald of the $\varepsilon_i$'is is 1 and the other -1. 
This motivates us to think of the counterexample $(1, 0, 0, \cdots , 0)$ on one side $(0, 1, 1, \cdots, 1)$ on the other. 
For odd $n$, we have $\sum_{i=1}^n \varepsilon_i=\pm 1$ to consider. 
Some experimentation yields that we can have $\sum_{i=1}^n \varepsilon_i= 1$ and 
$0\le \sum_{i=1}^n \varepsilon_i a_i\le 1$, and 
$0\le \sum_{i=1}^n \varepsilon_i b_i\le 1$, giving 
$| \sum_{i=1}^n \varepsilon_i a_i|+|\sum_{i=1}^n \varepsilon_i b_i|\le 1$. 
(Well, the proof for the existence of $\varepsilon_i$'s isn't that straightforward, but so does the construction of the counterexamples $a_i$'s and $b_i$'s which is guarateed to fail. The proof of the existence of such $\varepsilon_i$'s is better explained in the solution. 

In view of this we likely know that the answer *might* be positive even extended to the cases that $a_i$'s and $b_i$'s might have opposite signs for odd $n$ (and yes, once the invariant of $|a_i+b_i|=1$ or $|a_i-b_i|=1$ is observed the proof isn't that hard anymore, although still tricky). 

\textbf{Solution.} 
The answer is all odd $n$. 
We first find a counterexample for all $n=2k$ for some $k\ge 1$ an integer. 
Consider $a_i=1$, $b_i=0$ for all $i\in [1, 2k-1]$, 
and $a_{2k}=0, b_{2k}=1$. 
Then $\sum_{i=1}^n \varepsilon_i b_i=\varepsilon_{2k}$, which has absolute value of 1. 
Also $\sum_{i=1}^n \varepsilon_i a_i=\sum_{i=1}^{2k-1} \varepsilon_i$. 
If $x$ of the indices $\varepsilon_1, \varepsilon_2, \cdots , \varepsilon_{2k-1}$ are 1 and the rest -1, 
then the value would be $x-(2k-1-x)=2x-2k+1$, which is an odd integer. Thus it has absolute value at least 1 too. 
Therefore we have 
$\left| \sum_{i=1}^n \varepsilon_i a_i \right| + \left| \sum_{i=1}^n \varepsilon_i b_i \right|\ge 1+1=2$. 

Now let $n$ be odd. 
We start with the following lemma: 
Let $(a_i, b_i), i\in [1,m]$ satisfy $0\le a_i, b_i\le 1$ and $a_i+b_i=1, \forall i\in [1,m]$. 
Then there exists $\varepsilon_1, \varepsilon_2, \cdots, \varepsilon_m\in \{-1, 1\}$ satisfying
\begin{center}
\begin{tabular}{c c c c} 
$0\le \sum_{i=1}^m \varepsilon_i a_i, \sum_{i=1}^m \varepsilon_i b_i\le 1$
& and & $\sum_{i=1}^m \varepsilon_i a_i +\sum_{i=1}^m \varepsilon_i b_i=1$ & if  $m$ is odd, \\ 
$-1\le \sum_{i=1}^m \varepsilon_i a_i, \sum_{i=1}^m \varepsilon_i b_i\le 1$
& and & $\sum_{i=1}^m \varepsilon_i a_i +\sum_{i=1}^m \varepsilon_i b_i=0$ & if  $m$ is even. \\ 
\end{tabular} 
\end{center}
In the first case, we notice that the second condition can be achieved whenever exactly $\frac{m+1}2$ of $\varepsilon_1, \varepsilon_2, \cdots , \varepsilon_m$ are 1 and the rest -1. 
For any combination of $\varepsilon_1, \varepsilon_2, \cdots , \varepsilon_m$ with exactly $\frac{m+1}{2}$ of them as 1 we have 
$ \sum_{i=1}^m \varepsilon_i a_i  + \sum_{i=1}^m \varepsilon_i b_i $
$=\sum_{i=1}^m \varepsilon_i$
$=\frac{m+1}{2}-\frac{m-1}{2}=1$. 
The aim now is to assign $\varepsilon_1, \varepsilon_2, \cdots , \varepsilon_m$ 
in a way that exactly $\frac{m+1}2$ of them are 1, and $0\le \sum_{i=1}^m \varepsilon_i a_i\le 1$ 
(if this is true, $0\le \sum_{i=1}^m \varepsilon_i b_i\le 1$ holds true too). 
W.L.O.G. assume that $a_1\le a_2\le\cdots \le a_m$, and consider the numbers 
$x_0, x_1, \cdots , x_{\frac{m-1}2}$ such that 
\[ x_k=\sum_{i=1}^{k} -a_i+\sum_{i=k+1}^{k+\frac{m+1}{2}} a_i+\sum_{i=k+\frac{m+1}{2}+1}^{m} -a_i\]
Notice, first, that for each $x_i$, exactly $\frac{m+1}{2}$ of the $a_i$'s has coefficient 1 and the rest -1. 
Therefore if $0\le x_k\le 1$ for some $k$ we are done. 
Observe also that, $x_0=\sum_{i=1}^{\frac{m+1}{2}} a_i+\sum_{i=\frac{m+1}{2}+1}^{m} -a_i$
$\le \sum_{i=1}^{\frac{m+1}{2}} a_{\frac{m+1}{2}} +\sum_{i=\frac{m+1}{2}+1}^{m} -a_{\frac{m+1}{2}}$
$=a_{\frac{m+1}{2}}$
$\le 1.$
$x_{\frac{m-1}{2}}=\sum_{i=1}^{\frac{m-1}{2}} -a_i+\sum_{i=\frac{m+1}{2}}^{m} a_i$
$\ge \sum_{i=1}^{\frac{m-1}{2}} -a_{\frac{m+1}{2}}+\sum_{i=\frac{m+1}{2}}^{m} a_{\frac{m+1}{2}}$
$=a_{\frac{m+1}{2}}$
$\ge 0$. 
Moreover, $x_k-x_{k-1}=a_{k+\frac{m+1}{2}}-a_{k}-a_{k}+a_{k+\frac{m+1}{2}}=2(a_{k+\frac{m+1}{2}}-a_{k}), $
so $x_{k-1}\le a_k\le x_{k-1}+2$ (since $|a_i-a_j|\le 1$). 
If $0\le x_0\le 1$ or $0\le x_{\frac{m-1}{2}}\le 1$ then we are done. 
Otherwise we have $x_0<0$ and $x_{\frac{m-1}{2}}>1$, 
which means there exists $k$ such that $x_k>1$ and $x_{k-1}\le 1$. 
If $x_{k-1}\ge 0$ we are done. 
Otherwise, we have $x_k-x_{k-1}>1$ and therefore $a_{k+\frac{m+1}{2}}-a_{k}>\frac 12$, 
meaning that $a_{k+\frac{m+1}{2}}>\frac 12$. 
Now let $y_k=x_k-2a_{k+\frac{m+1}{2}}$ we have $y_k\le x_{k-1}<0$ but $y_k\ge x_k-2>-1$. 
This means $0\le -y_k\le 1$ and 
$y_k=\sum_{i=1}^{k} a_i+\sum_{i=k+1}^{k+\frac{m+1}{2}-1} -a_i+\sum_{i=k+\frac{m+1}{2}}^{m} a_i$. 
It is not hard to see that exactly $\frac{m+1}{2}$ of the $a_i$'s has coefficient 1 and the rest -1, so we are done. 

The second case isn't much different from the first. 
The second condition can be achieved whenever exactly half of $\varepsilon_1, \varepsilon_2, \cdots , \varepsilon_m$ are 1 and the rest -1. 
To see why, we have 
$ \sum_{i=1}^m \varepsilon_i a_i  + \sum_{i=1}^m \varepsilon_i b_i $
$=\sum_{i=1}^m \varepsilon_i$
$=\frac m2-\frac m2=0$. 
Again w.l.o.g. let $a_1\le a_2\le\cdots \le a_m$, and consider the numbers 
$x_0, x_1, \cdots , x_{\frac{m}2}$ such that 
\[ x_k=\sum_{i=1}^{k} -a_i+\sum_{i=k+1}^{k+\frac{m}{2}} a_i+\sum_{i=k+\frac{m}{2}+1}^{m} -a_i\]
Again each $x_k$ satisfies exactly $\frac m2$ have coefficient 1 and the rest -1. 
Thus if $-1\le x_k\le 1$ for some $k$ we are done. 
Observe also that, $x_0=\sum_{i=1}^{\frac{m}{2}} a_i+\sum_{i=\frac{m}{2}+1}^{m} -a_i$
$\le \sum_{i=1}^{\frac{m}{2}} a_{\frac{m}{2}} +\sum_{i=\frac{m}{2}+1}^{m} -a_{\frac{m}{2}}$
$=0$. 
$x_{\frac{m}{2}}=\sum_{i=1}^{\frac{m}{2}} -a_i+\sum_{i=\frac{m}{2}+1}^{m} a_i$
$\ge \sum_{i=1}^{\frac{m}{2}} -a_{\frac{m}{2}} +\sum_{i=\frac{m}{2}+1}^{m} a_{\frac{m}{2}}$
$=0$. 
If $-1\le x_0\le 1$ or $-1\le x_{\frac m2}\le 1$ we are done. 
Otherwise we have $x_0<-1$ and $x_{\frac m2}>1$, 
which allows us to pick a $k$ satisfying $x_k>1$ and $x_{k-1}\le 1$. 
By the similar logic as in case 1 we have $0\le x_k-x_{k-1}\le 2$. 
This means $x_{k-1}>-1$, which in turn gives $-1\le x_{k-1}\le x_k$ and we are done. 

To finish up the solution, 
we split the $a_i$'s and $b_i$'s into two groups: 
one with $a_ib_i\ge 0$, and the other one with $a_ib_i\le 0$ (if $a_ib_i=0$ then we assign them arbitrarily). 
W.L.O.G. let the first group be 
$(a_i, b_i)$; $i\in [1, m]$ and the second group be $(a_i, b_i)$: $i\in [m+1, n]$. 
For $i\in [1, m]$, we can w.l.o.g. assume that $a_i, b_i\ge 0$ (indeed, if $\varepsilon_i$ is a solution for $(a_i, b_i)$ then $-\varepsilon_i$ is a solution for $(-a_i, -b_i)$). 
Similarly we can also assume that $i\in [m+1, n]$ we have $a_i\ge 0$ and $b_i\le 0$. 
Also we invoke another property: if $-1\le x+y\le 1$ and $-1\le x-y\le 1$ then $|x|+|y|\le 1$. 
Indeed, $|x|+|y|\in \{x+y, x-y, y-x, -x-y\}$ so $-1\le x+y\le 1\to -1\le -x-y\le 1$, 
$-1\le x-y\le 1\to -1\le -x+y\le 1$ so $-1\le |x|+|y|\le 1$. 
(Bonus: try to show that this is an if and only if condition: convince yourself that $|x|+|y|=\max \{x+y, x-y, y-x, -x-y\}$.)

We split into two cases: 
\begin{itemize}
\item [Case 1.] $m$ is odd. 
From our lemma, by some careful choices of $\varepsilon_i$'s we have 
$0\le\sum_{i=1}^m \varepsilon_i a_i, \sum_{i=1}^m \varepsilon_i b_i\le 1$ and $\sum_{i=1}^m \varepsilon_i a_i +\sum_{i=1}^m \varepsilon_i b_i=1$. 
In addition, 
$-1\le\sum_{i=m+1}^n \varepsilon_i a_i, \sum_{i=m+1}^n \varepsilon_i (-b_i)\le 1$ and $\sum_{i=m+1}^n \varepsilon_i a_i +\sum_{i=m+1}^n \varepsilon_i (-b_i)=0$. 
This means, $\sum_{i=m+1}^n \varepsilon_i a_i=\sum_{i=m+1}^n \varepsilon_i b_i=c$ for some $c$. 
Also let $\sum_{i=1}^m \varepsilon_i a_i=a$ and $\sum_{i=1}^m \varepsilon_i b_i=b$, from which we know 
$a+b=1$ and $0\le a, b\le 1$. 
Now our term of interest $\left| \sum_{i=1}^n \varepsilon_i a_i \right| + \left| \sum_{i=1}^n \varepsilon_i b_i \right|$ becomes $|a+c|+|b+c|=|a+c|+|1-a+c|$
If $-1\le c\le 0$, then we have 
$a+c+b+c=1+2c$ and from $-1\le c\le 0$ we have $-1\le c\le 1$. 
Also, $a+c-b-c=a-b=a-(1-a)=2a-1$ and from $0\le a\le 1$ we have $-1\le 2a-1\le 1$. 
Thus we are done for the case where $c\le 0$. 
If $c\ge 0$, we 'flip' the signs of $\varepsilon_i$ for $i\in [m+1, n]$ so that 
$\sum_{i=m+1}^n (-\varepsilon_i a_i)=\sum_{i=m+1}^n (-\varepsilon_i) b_i=c$. 
Now our sum becomes $|a-c|+|1-a-c|$, and we get 
$-1=a-1+1-a-1\le a-c+1-a-c\le a-0+1-a-0=1$, and $a+c-b-c=a-b$ is also in the interval $[-1, 1]$ as before. 
This settles our case 1. 

\item [Case 2.] $m$ is even. 
 Again by our lemma we have 
$-1\le\sum_{i=1}^m \varepsilon_i a_i, \sum_{i=1}^m \varepsilon_i b_i\le 1$ and $\sum_{i=1}^m \varepsilon_i a_i +\sum_{i=1}^m \varepsilon_i b_i=0$. 
In addition, 
$0\le\sum_{i=m+1}^n \varepsilon_i a_i, \sum_{i=m+1}^n \varepsilon_i (-b_i)\le 1$ and $\sum_{i=m+1}^n \varepsilon_i a_i +\sum_{i=m+1}^n \varepsilon_i (-b_i)=1$. 
Now, let $\sum_{i=1}^m \varepsilon_i a_i=a$ and we have $\sum_{i=1}^m \varepsilon_i b_i=-a$ ($-1\le a\le 1$), 
let $\sum_{i=m+1}^n \varepsilon_i a_i=c$ and we have $\sum_{i=m+1}^n \varepsilon_i b_i=c-1$ ($0\le c\le 1$). 
Again, we need to consider $|a+c|+|-a+c-1|$. 
Observe that $a+c-a+c-1=2c-1$ and from $0\le c\le 1$ we have $-1\le 2c-1\le 1$. 
$a+c+a-c+1=2a+1$. If $-1\le a\le 0$ we have $-1\le 2a+a\le 1$, which serves our purpose. 
If $0\le a\le 1$, again we can reverse the signs of $a_i$ ($i\in [1,m]$) as of above so that 
$\sum_{i=1}^m (-\varepsilon_i) a_i=-a, \sum_{i=1}^m (-\varepsilon_i) b_i=a$, 
meaning that we need to now consider 
$|-a+c|+|a+c-1|$. 
Again $-a+c+a+c-1=2c-1$, giving $-1\le 2c-1\le 1$, as usual. 
Moreover, $-a+c-a-c+1=1-2a$, which means $-1\le 1-2a\le 1$ for $0\le a\le 1$. 
This completes the proof as well. 
\end{itemize}

\item[\textbf{A4}]
Find all functions $f:(0,\infty)\rightarrow (0,\infty)$ such that for any $x,y\in (0,\infty)$, $$xf(x^2)f(f(y)) + f(yf(x)) = f(xy) \left(f(f(x^2)) + f(f(y^2))\right)\cdots (*).$$

\textbf{Solution.} 
The only function is $f(x)\equiv \frac 1x$, which works because 
$xf(x^2)f(f(y)) + f(yf(x))=x\frac1 {x^2} y+\frac 1 {y\frac 1{x}}$
$=\frac xy+\frac yx$
$=\frac{x^2}{xy}+\frac{y^2}{xy}$
$=f(xy)\left(f(f(x^2)) + f(f(y^2))\right)$. 

For the rest of the solution we proceed with the normal functional algorithmic procedure: 
\begin{itemize}
\item [Step 1.] 
Plugging $x=y=1$ gives $f(1)f(f(1))+f(f(1))=2f(1)f(f(1))$, 
and since $f>0$, we can factorize $f(f(1))$ out to get $f(1)+1=2f(1)$, giving \fbox{$f(1)=1$}. 

\item [Step 2.]
Plugging $x\la 1$ (and substituting $f(1)\la 1$ due to step 1) gives $f(f(y))+f(y)=f(y)(1+f(f(y^2)))$, 
giving \fbox{$f(f(y))=f(y)f(f(y^2))$}. 

\item [Step 3.]
Plugging $y\la 1$, on the other hand, gives 
$xf(x^2)+f(f(x))=f(x)(f(f(x^2))+1)$. 
From step 2, $f(f(x))=f(x)f(f(x^2))$, 
which gives rise to \fbox{$xf(x^2)=f(x)$}. 

\item [Step 4.]
Substitute $xf(x^2)\la f(x)$ (step 3), $f(f(y))\la f(y)f(f(y^2))$ (step 2), and $yf(x)\la xyf(x^2)$ (step 3) into (*) gives: 
$$f(x)f(y)f(f(y^2)) + f(xyf(x^2)) = f(xy) \left(f(f(x^2)) + f(f(y^2))\right)\cdots(**).$$
In the special case where $xy=1$ we have $f(x)f(y)f(f(y^2)) + f(f(x^2)) = 1 \left(f(f(x^2)) + f(f(y^2))\right)$, 
so $f(x)f(y)=1$ whenever $xy=1$. 
In other words, for all $x\in\mathbb{R}^+$, \fbox{$f(\frac 1x)=\frac 1{f(x)}$}. 

\item [Step 5.]
Having the results in Step 4 in mind, we do the following substitution: \\
$\bullet$ Substitute $\frac 1x$ and $\frac 1x$ in place of $x$ and $y$ into (**) to turn
$f(\frac 1x)f(\frac 1x)f(f(\frac 1{x^2})) + f(\frac 1{x^2}f(\frac 1{x^2}))$ 
$= f(\frac 1{x^2}) \left(f(f(\frac 1{x^2})) + f(f(\frac 1{x^2}))\right)$
into 
$\frac 1{f(x)f(x)f(f(x^2))}+ \frac 1{f(x^2f(x^2))}$ 
$= \frac 2{f(x^2)f(f(x^2))}\cdots (5a)$. \\
$\bullet$ Subsitute $x=y$ into (**) we get 
$f(x)f(x)f(f(x^2)) + f(x^2f(x^2)) = 2f(x^2) f(f(x^2))\cdots (5b).$ \\
Comparing 5a and 5b gives 
$\frac 1{f(x)f(x)f(f(x^2))}+ \frac 1{f(x^2f(x^2))}=\frac 2{f(x^2)f(f(x^2))}$
$=\frac 4{f(x)f(x)f(f(x^2))+f(x^2f(x^2))}$, 
which we can cross multiply to get 
$(f(x)f(x)f(f(x^2))+ f(x^2f(x^2)))^2=4(f(x)f(x)f(f(x^2))\cdot f(x^2f(x^2)))$, 
a.k.a. $(f(x)f(x)f(f(x^2))- f(x^2f(x^2)))^2=0$. 
Therefore $f(x)f(x)f(f(x^2))=f(x^2f(x^2))=f(x^2)f(f(x^2))$. 
Factorizing out $f(f(x^2))$ gives $f(x)^2=f(x^2)$, 
and from (2), $f(x)=xf(x^2)=xf(x)^2$, or \hbox{$xf(x)=1$, or $f(x)=\frac 1x$}. 

\end{itemize}

\item[\textbf{A5}]
Consider fractions $\frac{a}{b}$ where $a$ and $b$ are positive integers.
\begin{itemize}
\item[(a)]
Prove that for every positive integer $n$, there exists such a fraction $\frac{a}{b}$ such that $\sqrt{n} \le \frac{a}{b} \le \sqrt{n+1}$ and $b \le \sqrt{n}+1$.
\item[(b)]
Show that there are infinitely many positive integers $n$ such that no such fraction $\frac{a}{b}$ satisfies $\sqrt{n} \le \frac{a}{b} \le \sqrt{n+1}$ and $b \le \sqrt{n}$. 
\end{itemize}

\textbf{Thoughts.} 
Part (b) should very well give a hint on part (a) : $b=\lfloor\sqrt{n}+1\rfloor$ is necessary. 
Together with another trick of arranging the natural numbers $n$ according to their integer square root 
(i.e. $\lfloor\sqrt{n}\rfloor$) the case of $b=\lfloor\sqrt{n}+1\rfloor$ covers half of the required numbers. 
As it turns out, the case $b=\lfloor\sqrt{n}\rfloor$ covers the remaining half, so we are practically done with part (a). 

Part (b) requires some experimentation, which is not too hard is we carry on with the partition of the natural numbers according to their integer square roots. 
Notice, also, that there is no suitable $b$ for $n=k^2+1$ for some $k$, which can be proven using the inequality 
$(k+\frac 1k)^2>k^2+2$. This is precisely what we need to solve the problem. 

\textbf{Solution.} 
For part (a), we partion the set of positive integers according to their integer square roots, 
that is, the sets $S_1=\{1,2,3\}$, $S_2=\{4,5,6,7,8\}$, $S_3=\{9,10,11,12,13,14,15\}$, etc. 
Consider $S_k=\{k^2, k^2+1, \cdots k^2+2k\}$, 
and we claim that $b=k$ and $b=k+1$ alone will jointly work for the sets. 
Indeed, considering $c\in [0, k]$ we have $(k+\frac ak)^2=k^2+2a+(\frac ak)^2$. 
With $(\frac ak)^2\le 1$, we have $\sqrt{k^2+2a}\le k+\frac ak\le \sqrt{k^2+2a+1}$, 
so $b=k$ works for $k^2, k^2+2, \cdots , k^2+2k$. 
Meanwhile for $c\in [0, k+1]$ we have $(k+\frac a{k+1})^2=k^2+\frac{2ak}{k+1}+(\frac a{k+1})^2$
$=k^2+2a-\frac{2a}{k+1}+(\frac a{k+1})^2$. 
Notice that $-\frac{2a}{k+1}+(\frac a{k+1})^2=\frac{a^2-2a(k+1)}{(k+1)^2}$
$=\frac{(a-(k+1))^2-(k+1)^2}{(k+1)^2}$
$=\frac{(a-(k+1))^2}{(k+1)^2}-1$, 
and with $0\le a\le k+1$ we have 
$-1\le \frac{(a-(k+1))^2}{(k+1)^2}-1\le 0$. 
Therefore 
$\sqrt{k^2+2a-1}\le k+\frac a{k+1}\le \sqrt{k^2+2a+1}$, 
and this works for $n=k^2+1$, $k^2+3$, $\cdots$, $k^2+2k-1$. 
Therefore all elements in $S_k$ are covered. 
As for part (b) we show that there's no fraction $\frac ab$ (with $b\le k$) lying in the interval $[\sqrt{k^2+1}, \sqrt{k^2+2}]$. 
Notice that, $k< \sqrt{k^2+1} < \sqrt{k^2+2} < \sqrt{k^2+2k+1} =k+1$. 
Assume that $\frac ab$ satisfies this property, 
then from $\frac ab > k$ and $b\le k$ we have 
$(\frac ab)^2\ge (k+\frac 1k)^2=k^2+2+\frac 1{k^2} > k^2+2$, contradiction. 

\item[\textbf{A6/IMO 5}]
The equation
$$(x-1)(x-2)\cdots(x-2016)=(x-1)(x-2)\cdots (x-2016)$$is written on the board, with $2016$ linear factors on each side. What is the least possible value of $k$ for which it is possible to erase exactly $k$ of these $4032$ linear factors so that at least one factor remains on each side and the resulting equation has no real solutions?

\textbf{Thoughts.} 2016 is the minimum--the first thing that must jump out immediately of your mind. 
The construction should also be intuitive: we need to either have $LHS>RHS$ all the times or vice versa. 
W.L.O.G. let's make $LHS<RHS$. One observation is that, $RHS<0$ must imply $LHS<0$, meaning that on each side there are oddly many factors with roots greater than $x$ (hence contributing to negative factors). 

Thus this gives the "mod 4" construction as detailed below, because whenever $RHS$ is negative we have $x\in (4i-2, 4i-1)$ for some integer $i$, which also guarantees the negativity of $LHS$. 
Next, it is not hard to prove that $|LHS|<|RHS|$ whenever both of them are positive, 
so the only hard part is to prove that $|LHS|>|RHS|$ when both negative. 
Thankfully $|\frac{(x-(4i-2))(x-(4i-1))}{(x-4i)(x-(4i-3))}|\le \frac 19$ whenever $x\in (4i-2, 4i-1)$, and we can finish off by using appropriate sum telescoping, again as detailed below. 

\textbf{Solution.} The answer is 2016. 
Anything fewer doesn't work, because for some $i$, the factor $x-i$ appears on both sides, 
so $i$ is itself a root. 

It remains to show that 2016 is good to go. 
We claim that the equation $\displaystyle\prod_{i=1}^{504} (x-(4i-3))(x-(4i))=\displaystyle\prod_{i=1}^{504} (x-(4i-2))(x-(4i-1))$ has no real solution by showing that the left-hand side is always strictly smaller than the right hand side. 
We first eliminate the obvious cases where $LHS>0$ while $RHS<0$. 
Observe that whenever $x\in (4i+1, 4i+2)$ for some $i\in [0, 503]$, 
there are $2i+1$ negative factors (and the rest $1007-2i$ positive) on the left (hence negative) while $2i$ negative factors (and the rest $1008-2i$ positive) on the right (hence positive). 
Also whenever $x\in (4i-1, 4i)$ for some $i\in [1, 504]$, 
there are $2i-1$ negative factors (and the rest $1009-2i$ positive) on the left (hence negative) while $2i$ negative factors (and the rest $1008-2i$ positive) on the right (hence positive). 
Thus in both of the cases the left is less than 0 while the right is more than 0. 
As for the endpoints $x\in\{1,2,\cdots ,2016\}$, if $x=4i$ or $x=4i+1$ then LHS=0 while RHS has $2i$ negative factors (while the rest positive) hence positive. 
If $x=4i-1$ or $x=4i-2$ then the right is 0 while the left has $2i-1$ negative factors (while the rest positive) hence negative. 

If $x>2016$ then we have LHS and RHS both greater than 0 (since all remaining 2016 factors are positive). 
Nevertheless, in light of the relation 
$(x-(4i-2))(x-(4i-1))-(x-(4i-3))(x-(4i))=(4i-1)(4i-2)-(4i-3)(4i)=2$ we have 
$|(x-(4i-2))(x-(4i-1))|>|(x-(4i-3))(x-(4i))|$, and thus 
$\displaystyle\prod_{i=1}^{504} |(x-(4i-3))(x-(4i))|<\displaystyle\prod_{i=1}^{504} |(x-(4i-2))(x-(4i-1))|$. 
Since each side is positive, $\displaystyle\prod_{i=1}^{504} (x-(4i-3))(x-(4i))<\displaystyle\prod_{i=1}^{504} (x-(4i-2))(x-(4i-1))$. 
The case $x<1$ is symmetrical and hence analogous. 

We are thus left with the trickiest case: $x\in (4i-2, 4i-1)$ for some $i\in [1, 504]$, whereby both sides are negative. 
The goal is therefore to show that $|LHS|>|RHS|$.  
We still want to keep in mind that $(x-(4i-2))(x-(4i-1))-(x-(4i-3))(x-(4i))=(4i-1)(4i-2)-(4i-3)(4i)=2$, 
and that both $(x-(4i-2))(x-(4i-1))$ and $(x-(4i-3))(x-(4i))$ are positive for $x\not\in [4i-3, 4i]$. 
Now, let $x\in (4i-2, 4i-1)$ for some $i\in [1, 504]$, 
then from $(x-(4i-2))(x-(4i-1))=(x-(4i-1.5))-\frac 14\ge -\frac 14$ we get 
$\frac{|(x-(4i-2))(x-(4i-1))|}{|(x-(4i))(x-(4i-3))|}=\frac{c}{c+2}=1-\frac{2}{c+2}\le 1-\frac 2{2+\frac 14}=\frac 19$ where $c=|(x-(4i-2))(x-(4i-1))|$. 
Next, let's investigate $\frac{|(x-(4j-2))(x-(4j-1))|}{|(x-(4j))(x-(4j-3))|}$ 
for some $j<i$. 
We know that $x>4i+1$, so 
$(x-(4j-2))(x-(4j-1))> (4i-4j-1)(4i-4j)$ 
and therefore 
$\frac{|(x-(4j-2))(x-(4j-1))|}{|(x-(4j))(x-(4j-3))|}$
$\frac{(x-(4j-2))(x-(4j-1))}{(x-(4j-2))(x-(4j-1))-2}$
$=1+\frac 2{(x-(4j-2))(x-(4j-1))-2}$
$<1+\frac 2{(4(i-j)-1)(4i-4j)-2}$
$=1+\frac 1{2(4(i-j)-1)(i-j)-1}$. 
It's also not hard to verify that $2(4(i-j)-1)(i-j)-1<(i-j+1)^2-1$ for $j\le i-1$, so we in turn have 
$ 1+\frac 1{2(4(i-j)-1)(i-j)-1}$
$<1+\frac 1{(i-j+1)^2-1}$
Thus $\dfrac{\displaystyle\prod_{j=1}^{i-1} (x-(4j-2))(x-(4j-1))}{\displaystyle\prod_{j=1}^{i-1} (x-(4j-3))(x-(4j))}$
$<\displaystyle\prod_{j=1}^{i-1}(1+\frac 2{(4i-4j-1)(4i-4j)-2})$
$<\displaystyle\prod_{j=-\infty}^{i-1}(1+\frac 2{(4i-4j-1)(4i-4j)-2})$
$<\displaystyle\prod_{j=-\infty}^{i-1}(1+\frac 1{(i-j+1)^2-1})$
$=\frac{2^2}{2^2-1}\cdot\frac{3^2}{3^2-1}\cdot\cdots$
$=\lim_{x\to\infty}2\cdot \frac{x-1}{x}$
$=2$. 
Likewise, $\dfrac{\displaystyle\prod_{j=i+1}^{504} (x-(4j-2))(x-(4j-1))}{\displaystyle\prod_{j=i+1}^{504} (x-(4j-3))(x-(4j))}$
$<2$, 
(we can drop the modulus since they are all greater than 0. 
Thus $\dfrac{\displaystyle\prod_{i=1}^{504} |(x-(4i-2))(x-(4i-1))|}{\displaystyle\prod_{i=1}^{504} |(x-(4i-3))(x-(4i))|}$
$<2\times \frac{1}{9}\times 2$
$<\frac 49$, 
and we are done. (OMG the long proof...)

\end{itemize}

\section{Combinatorics}
\begin{itemize}
\item[\textbf{C1}]
 The leader of an IMO team chooses positive integers $n$ and $k$ with $n > k$, and announces them to the deputy leader and a contestant. The leader then secretly tells the deputy leader an $n$-digit binary string, and the deputy leader writes down all $n$-digit binary strings which differ from the leader’s in exactly $k$ positions. (For example, if $n = 3$ and $k = 1$, and if the leader chooses $101$, the deputy leader would write down $001, 111$ and $100$.) The contestant is allowed to look at the strings written by the deputy leader and guess the leader’s string. What is the minimum number of guesses (in terms of $n$ and $k$) needed to guarantee the correct answer?

\textbf{Thoughts.} 
This is equivalent to finding the number of possible strings said by the leader given the strings said by the deputy leader. 
Intuitively, for each digit, there is $\frac kn$ probability of it being changed by the deputy leader (so there will be $\frac kn$ of the strings with the digit being changed). 
From here, the string can be uniquely determined if $\frac kn\neq \frac 12$: 
for each digit we know that it is 0 or 1.  
In the case where $\frac kn=\frac 12$, 
then for each digit there are equally many strings with 0 on it as those wih 1 on it. 
Nevertheless, if we only look at the strings with leading 0, 
then among those strings, for each of the rest of the digits there are $\frac{k-1}{n-1}$ of the strings with that digit being changed 
(and yeah $\frac{k-1}{n-1}\neq \frac 12$) 
so the string can be uniquely determined like above (same goes for the strings with leading 1). 
Now we have two candidates, and the last step is to prove that it works. 
The verification might sound difficult, but again all we need is to show that if one string works we can find another string that works too (as of below). 

\textbf{Solution.} The answer is 2 for $n=2k$ and 1 otherwise. 

Notice that there are $\binom nk$ strings the deputy leader can write. 
For the $i$-th digit (for any $i\in [0, n-1]$), there are $\binom {n-1}{k-1}$ such strings with $i$-th digit differing from the original, 
$\binom {n-1}k$ such strings with $i$-th digit equal to the original. 
If $\frac{(n-1)!}{(k-1)!(n-k)!}=\binom {n-1}{k-1}\neq \binom {n-1}k=\frac{(n-1)!}{(k)!(n-k-1)!}$, the contestant can determine that digit by counting the number of strings with 0 in it (and the number of strings with 1 in it). 
This happens when $(k-1)!(n-k)!\neq k!(n-k-1)!$, or $n-k\neq k$ (fatorizing factors out) or $n\neq 2k$. 
No further guesses is needed and the contestant can get it in one try. 

If $n=2k$, then for each digit, half of the strings have one's and half have zero's. 
The student then considers the strings with 0 on the leading digit. 
If, the correct string has 0 on that leading digit, then for each of the written strings (with leading 0), among the remaining $2k-1$ digits there are $k-1$ being changed from the original. By the claim above the student can determine the remaining $2k-1$ digits. 
Similar conclusion can be reached for he case with 1 as leading digit. 
This gives the student the correct answer after 2 guesses. 
To see why 2 guesses is necessary, let $a_0a_1\cdots a_{2k-1}$ be the string given by the leader, $b_0b_1\cdots b_{2k-1}$ be a string with $b_i=1-a_i$ for each $i$, $c_0c_1\cdots c_{2k-1}$ be any string written by the deputy leader. 
Now, we have $c_i=a_i$ or $c_i=b_i$ but not both. 
With $c_0c_1\cdots c_{2k-1}$ having $k$ same digits and $k$ different digits as $a_0a_1\cdots a_{2k-1}$, 
it must have $2k-k=k$ same digits and $2k-k=k$ different digits as $b_0b_1\cdots b_{2k-1}$ too. 
Thus $b_0b_1\cdots b_{2k-1}$ is actually another possibility. 

\item[\textbf{C2}]
Find all positive integers $n$ for which all positive divisors of $n$ can be put into the cells of a rectangular table under the following constraints:\\
$\bullet$ each cell contains a distinct divisor;\\
$\bullet$ the sums of all rows are equal; and\\
$\bullet$ the sums of all columns are equal.\\

\textbf{Solution.} 
The answer is $n=1$, which works with 1 being placed in a $1\times 1$ table. 
To show that this fails for other $n$, first prime factorize it into 
$\displaystyle\prod_{i=1}^k p_i^{a_i}$. 
If $r$ is the number of rows and $c$ is the number of columns then $rc=\displaystyle\prod_{i=1}^k (a_i+1)$, 
the number of divisors of $n$. 
W.l.o.g. $r\ge c$ and therefore $r\ge\sqrt{\displaystyle\prod_{i=1}^k (a_i+1)}=\displaystyle\prod_{i=1}^k \sqrt{(a_i+1)}$. 
We have also known that the sum of divisors is 
$\displaystyle\prod_{i=1}^k \frac{p_i^{a_i+1}-1}{p_i-1}$. 
Knowing that one of the cells contains $n$, the sum of each row must be greater than $n$, 
($n$ cannot be the only cell in that row, otherwise all cells would have to contain the same number which is absurd for $n>1$). 
This means that the sum of each column is greater than $rn$, giving the following inequality: 
$$\displaystyle\prod_{i=1}^k \frac{p_i^{a_i+1}-1}{p_i-1}>rn\ge\displaystyle\prod_{i=1}^k \sqrt{(a_i+1)}p_i^{a_i}$$
or equivalently, 
$\displaystyle\prod_{i=1}^k \frac{p_i-\frac{1}{p_i^{a_i}}}{p_i-1}>\displaystyle\prod_{i=1}^k \sqrt{(a_i+1)}$

Now for each prime $p$, we are interested to investigate the ratio
$\frac{1}{p-1}(p-\frac{1}{p^{a_i}}): \sqrt{(a_i+1)}$. 
For $p=2$ we have $(2-\frac{1}{2^{a_i}}): \sqrt{(a_i+1)}$. 
Notice that for $a_i\ge 3$, $(2-\frac{1}{2^{a_i}})<2$ while $\sqrt{(a_i+1)}\ge 2$. 
so the ratio is smaller than 1. 
for $a_i=1$, $(2-\frac{1}{2^{a_i}})=\frac 32$ and $\sqrt{(a_i+1)}=\sqrt{2}$ so the ratio is $\frac 3{2\sqrt{2}}$, 
for $a_i=2$ we have $\frac 74\div \sqrt{3}=\frac 7{4\sqrt{3}}$. 
Knowing that $\frac 3{2\sqrt{2}}=\sqrt{\frac 98}>\sqrt{\frac {49}{16}}=\frac 7{4\sqrt{3}}$ the maximum ratio is 
$\sqrt{\frac 98}$. 
For $p\ge 3$ we have $\frac{1}{p-1}(p-\frac{1}{p^{a_i}})$ decreasing with $p$ with $a_i$ fixed because
$\frac{1}{p-1}(p-\frac{1}{p^{a_i}})$
$=1+\frac 1p+\cdots +\frac 1{p^{a_i}}$
$\le 1+\frac 13+\cdots +\frac 1{3^{a_i}}$
$=\frac{1}{2}(3-\frac{1}{3^{a_i}})$. 
When $a_i=1$ the ratio is $\frac 4{3\sqrt{2}}=\sqrt{\frac 89}$, when $a_i\ge 2$ the ratio is at most 
$\frac{1}{2}(3-\frac{1}{3^{a_i}})\div \sqrt{a_i+1}$
$\le\frac{1}{2}(3)\div \sqrt{2+1}$
$=\frac 3{2\sqrt{3}}$
$=\sqrt{\frac 34}$
$<\sqrt{\frac 89}$. 
Thus the maximum possible ratio is $ \sqrt{\frac 89}$. 

Summing up, for $n$ consisting at least two distinct prime factors the ratio 
$\displaystyle\prod_{i=1}^k \frac{p_i-\frac{1}{p_i^{a_i}}}{p_i-1}\div\displaystyle\prod_{i=1}^k \sqrt{(a_i+1)}$
cannot exceed 
$\sqrt{\frac 98}\times \sqrt{\frac 89}^{i-1}\le 1$, 
contradicting that $\displaystyle\prod_{i=1}^k \frac{p_i-\frac{1}{p_i^{a_i}}}{p_i-1}>\displaystyle\prod_{i=1}^k \sqrt{(a_i+1)}$. 
Hence $i=1$ and from the previous paragraph, $p < 3$ and thus $p=2$. 
However, this implies $n$ is a power of 2 and from $a_i\ge 1$, at least two rows must be used (we assumed $r\ge c$). 
The row containing $n$ must therefore have sum at least $2n$, 
but for $n$ a power of two the sum of divisors is $2n-1$, contradiction. 

\item[\textbf{C3}]
Let $n$ be a positive integer relatively prime to $6$. We paint the vertices of a regular $n$-gon with three colours so that there is an odd number of vertices of each colour. Show that there exists an isosceles triangle whose three vertices are of different colours.

\textbf{Thoughts.} 
This problem is hard for a C3: how are we going to use the fact that an odd number of vertices is painted for each colour? 
Nevertheless, the fact that $3\neq n$ generated some insight for us: for each fixed segment $AB$ with endpoint of different colours there are three vertices $C_1, C_2, C_3$ such that the resulting triangle is isoceles. 
This should motivate the double-counting solution: assuming a contradiction and we consider those isoceles triangles with exactly two colours being used, then each of these triangles has two polychromatic lines while each polychromatic line belongs to three such triangles, so we can play around with the parity to produce contradiction!

\textbf{Solution.} 
Suppose that the conclusion is false, and let $a, b, c$ be the three colours used. 
Let $\Gamma$ be the circumcircle of the regular $n$-gon. 
Let $N$ be the set of isoceles triangles such that each $a$ and $b$ is used at least once and the colour $c$ is not used, 
and let $M$ be the set of unordered pairs of vertices $\{A, B\}$ such that $A$ is of colour $a$ and $B$ is of colour $b$ (vice versa) 

We start with an observation: each member of $M$ is one side of exactly three triangles in $N$. 
To see why, let $(A,B)$ be one member in $M$. From $2\nmid n$ we know that $AB$ cannot be a diameter of $\Gamma$. 
This implies that there exists one such vertex $C_1$ satisfying $AB=AC_1$, and another vertex $C_2$ satisfying $AB=BC_2$. 
In addition, since $n$ is odd, the perpendicular bisector of $AB$ will hit exactly one vertex in the $n$-gon, 
so there is exactly one such $C_3$ with $AC_3=BC_3$. 
$C_1, C_2, C_3$ are also pairwise different; otherwise $C_1=C_2=C_3$ and $ABC_1$ is equilateral, 
contradicting the fact that $3\nmid n$. 
Also notice that $ABC_1$, $ABC_2$, $ABC_3\in N$ because the colour $c$ is not used at $C$ (otherwise we are done since $A, B$ are of colour $a$ and $b$), and each colour $a$ and $b$ is used at least once. 
This gives us the required three triangles. 

Next, also notice that for each triangle in $N$, 
since no colour of $c$ is used, among each $a$ and $b$, one colour is used twice and the other once. 
This implies that each triangle in $N$ has two sides in $M$. 
If we consider all such pairs $(x, y)$ where $x\in M, y\in N$ and $x$ is a side of $y$ then we count each element in $M$ for 3 times and each element in $N$ for 2 times. 
Thus $2|N|=3|M|$. 
This means $M$ is even. On the other hand we also have $|M|=|a|\cdot |b|$ where $|a|$ and $|b|$ are the number of sides with colour $a$ and $b$, respectively. This means $|a|$ or $|b|$ is even, contradiction. 

\item[\textbf{C4/IMO 2}]
Find all integers $n$ for which each cell of $n \times n$ table can be filled with one of the letters $I,M$ and $O$ in such a way that:\\
$\bullet$ in each row and each column, one third of the entries are $I$, one third are $M$ and one third are $O$; and \\
$\bullet$ in any diagonal, if the number of entries on the diagonal is a multiple of three, then one third of the entries are $I$, one third are $M$ and one third are $O$. 

\textbf{Thoughts.} 
The first condition already says that $3|n$, but sadly $n=3$ doesn't work (and it's extremely easy to prove that it doesn't). 
One way to start with is, therefore, to check the smallest $n$ that works. 
How about...making sure that if we partition the table into $3\times 3$ grids, each diagonal of the grids have exactly one of each $I, M, O$? 
Wait a minute...we have to make sure that among the center cells of the $3\times 3$ grids, there are equally many $I$, $M$, and $O$. 
Best to have that condition for the columns in the $3\times 3$ grids too, and we can sort out the rows later. 
This gives $n=9$ works, and for $n=9k$ all we need to do is to replicate the table. 

The proof of $9|n$ being necessary seems a little bit difficult, but the insights can be seen as we try to prove why $n=3$ fails: 
it seems like the fault lies on the center cell. 
As it turns out, the center cells of each $3\times 3$ grid lie on exactly two diagonals with size a multiple of 3 
(the corner: 1 and the sides: 0). 
This motivates us to consider just the columns and rows with indices congruent to 2 modulo 3, which works. 
(Alternatively one can also consider the columns and rows with indices not congruent to 2 modulo 3). 

\textbf{Solution.} The answer is all $n$ divisible by 9. 
We start by showing an example for $n=9$, given below: 
\begin{center}
\begin{tabular}{|c|c|c||c|c|c||c|c|c|}
\hline
I & M & O & M & O & I & O & I & M\\
\hline
M & M & M & O & O & O & I & I & I\\
\hline
I & M & O & M & O & I & O & I & M\\
\hline
\hline
O & I & M & I & M & O & M & O & I\\
\hline
I & I & I & M & M & M & O & O & O\\
\hline
O & I & M & I & M & O & M & O & I\\
\hline
\hline
M & O & I & O & I & M & I & M & O\\
\hline
O & O & O & I & I & I & M & M & M\\
\hline
M & O & I & O & I & M & I & M & O\\
\hline
\end{tabular}
\end{center}
For $n=9k$ for some $k$ we just have to split the grid into $k^2$ $9\times 9$ grids, and fill each one with the letters above. 
For sake of verification, observe that there are exactly 3 $I$'s, 3 $M$'s and 3 $O$'s in each column or each row of a single $9\times 9$ grid. 
Also, each diagonal is in the form of either $R_m=\{(i, j): i+j=m\}$, or $L_m=\{(i, j): i-j=m\}$, 
for some $m$ satisfying $1\le (i, j)\le n$. 
Now for $R_m$, the size $|R_m|$ is $m-1$ for $m\le n+1$, and $2n+1-m$ for $m\ge n+1$. 
Notice that 3 divides $|R_m|$ iff $m\equiv 1\pmod{n}$ (first case), or iff  $m\equiv 1\pmod{n}$ (second case). 
Thus it is not hard to see that the diagonals are in the form of 
$(1, m-1), (2, m-2), \cdots (m-1, 1)$ in the first case, and 
$(m-n, n), (m-n+1, n-1), \cdots (n, m-n)$ in the second case. 
In each of the cases we can group them into groups of three, such that, if we further split each $9\times 9$ grids into $3\times 3$ grids, each group contains three cells along the main diagonal. 
Nevertheless, from the construction above we see that each main diagonal in the $3\times 3$ grids have one $I$, one $M$ and one $O$. 
Thus this set of diagonal works too. 
A similar conclusion can be yielded for diagonals in the form of $L_m$. 

To show that $9|n$ is necessary, observe from the first condition that $3|n$. 
Let $n=3k$ and let's split the table into $k^2$ $3\times 3$ cells. 
Notice from the logic (of diagonals characterization) as of above, 
the center of each $3\times 3 $ cell ($(i, j)$ where $i, j\equiv 2\pmod{3}$) lie on both $R_m$ and $L_m$ with both size divisible by 3; 
the four corners (($(i, j)$ where $i, j\not\equiv 2\pmod{3}$) lie on exactly one of the sets satifying the properties; 
the four sides (($(i, j)$ where exactly one of $i$ and $j$ is congruent to 2 mod 3) lie on none of them. 
Thus, when we mark the cells in each column, each row, and each diagonal with size divisible by 3, 
the center cells are marked 4 times, the corners thrice, and the sides twice (as illustrated below). 
\begin{center}
\begin{tabular}{|c|c|c|}
\hline
3 & 2 & 3\\
\hline
2 & 4 & 2\\
\hline
3 & 2 & 3\\
\hline
\end{tabular}
\end{center}
Let $c$ be the number of $M$'s on the center cells. 
Considering just the $3i-1$-th column for $i\in [1, k]$ and the $3j-1$-th row for $j\in [1, k]$ yields $2k^2$ $M$'s being counted. 
Each cell on the "side" is being counted once, each cell on the "center" twice, an each cell on the "corner" none. 
This gives the number of $M$'s on the side as $2k^2-c$, which follows that there must be $k^2+c$ $M$'s at the corner. 
Now let's see what happens as we consider all such markings (all columns, all rows, and all diagonals of size divisible by 3). 
Observe that for each $3\times 3$ cells we have $3+2+3+2+4+2+3+2+3=24$ markings, so each letter ($M$, in particular) has $8k^2$ markings. 
This means $8k^2=4c+2(2k^2-c)+3(k^2+c)=3c+7k^2$, or $c=\frac{k^2}{3}$. 
Hence $3|k^2$, or $3|k$, or $9|n$. 

\item[\textbf{C5}]
Let $n \geq 3$ be a positive integer. Find the maximum number of diagonals in a regular $n$-gon one can select, so that any two of them do not intersect in the interior or they are perpendicular to each other. 

\textbf{Solution.} The answer is $n-3$ for $n$ odd, and $n-2$ for $n$ even. 
For simplicity we denote the circumcircle of the $n$-gon as $\Gamma$, and our $n$-gon be $A_1A_2\cdots A_n$. 

We first shot that no two diagonals can be perpendicular for $n$ odd. 
Indeed, in terms of cordinate geometry, let one diagonal joining $(\sin a, \cos a)$ and $(\sin b, \cos b)$ we have the gradient as 
$\frac{\cos b-\cos a}{\sin b-\sin a}$. 
Similarly let another diagonal with gradient 
$\frac{\cos d-\cos c}{\sin d-\sin c}$. 
If they are perpendicular then the gradient has product -1 (one of them might be infinity, but this means that the other one has gradient 0). 
This gives $(\cos b-\cos a)(\cos d-\cos c)+(\sin b-\sin a)(\sin d-\sin c)=0$. 
Using the identity of $\cos b-\cos a=-2\sin\frac{b-a}{2}\sin\frac{b+a}{2}$ and 
$\sin b-\sin a=2\sin\frac{b-a}{2}\cos\frac{b+a}{2}$, we have 
$\sin\frac{b-a}{2}\sin\frac{d-c}{2}(\sin\frac{b+a}{2}\sin\frac{d+c}{2}+\cos\frac{b+a}{2}\cos\frac{d+c}{2})=0$. 
Since $0\le a,b,c,d<2\pi$, and $a\neq b, c\neq d$, 
$\sin\frac{b-a}{2}, \sin\frac{d-c}{2}\neq 0$ so 
$\sin\frac{b+a}{2}\sin\frac{d+c}{2}+\cos\frac{b+a}{2}\cos\frac{d+c}{2}=0$, or 
$\cos (\frac{(b+a)-(d+c)}2)=0$, which forces 
$\frac{(b+a)-(d+c)}2=\frac{k\pi}2$ for some odd $k$. 
Converting $a, b, c, d$ into $\frac{2a\pi}{n}$, $\frac{2b\pi}{n}$, $\frac{2c\pi}{n}$, $\frac{2d\pi}{n}$ ($0\le a, b, c, d < n$) we get 
$kn=2(a+b)-2(c+d)$. 
Knowing that $k$ is odd, $n$ has to be even. 
Thus the case where $n$ is odd reduces to finding the number of diagonals without any two of them intersecting in the interior. 
Denoting $f(n)$ be this number and we show that $f(n)=n-3, \forall n\ge 3$. 
Base case when $n=3$, where $f(3)=0$ as there is no diagonal in a triangle. 
Next, let $f(n)=n-3$ for $n=3, 4, \cdots k$ for some $k\ge 3$. 
For $n=k+1$, let one diagonal split our $n$-gon into an $x$-gon and a $y$-gon, where $x+y=n+2$ and $x, y\ge 3$. 
Then there are extra $f(x)+f(y)=x-3+y-3=x+y-6=n-4$ diagonals that can be constructed (each diagonal must belong to only one of the polygons becaus it cannot intersect our first diagonal int he interior). 
This gives $n-4+1=n-3$ diagonals, at most. 
Equality can be achieved by taking diagonals $A_1A_k$, $k=3, 4, \cdots , n-1$.

Now $n$ be even. Let $F$ be the set of diagonals intersecting at least one other diagonals. We show that: 
\begin{itemize}
\item [1.]
Each two lines in $F$ are perpendicular or parallel to each other. 

\item [2.]
There are two lines in $F$ such that each of them intersects every other lines in $F$ that are perpendicular to itself.

\item [3.]
Let $A$ and $B$ be two consecutive endpoints of lines in $F$. 
Then for any selected diagonal with an endpoint on the minor arc $AB$, the other the other endpoint must also lie on this minor arc too.

\end{itemize}
For the first claim, choose two lines $l_1$ and $l_2$ that intersect in the interior (they are perpendicular). 
Now, suppose that there is a line $l_3\in F$ that is perpendicular to neither of them. 
Then it cannot intersect either of $l_1$ and $l_2$, so there must be a line $l_4$ perpendicular to itself. 
Now, let the endpoints of $l_1$ be $A, C$, and $l_2$ be $B$, $D$. 
Consider $AB$: the fact that the lines intersect means that $\angle ACB=\angle ADB <90^{\circ}$, so $C$ and $D$ are on the major arc $AB$. 
Let the endpoints of $l_3$ be $E, G$ and the endpoints of $l_4$ be $F, H$. 
W.L.O.G. let $E, F, G, H$ be in that order (since they intersect) on our circle $\Gamma$. 
Now, since $l_3$ and $l_4$ intersect, they must be both on the minor arc of one of $AB, BC, CD, DA$, 
implying that $F$ and $G$ are both on the major arc of $EH$. Contradiction. 

For the second claim, we notice that if we have a line joining points with polar angle $a$ and $b$, and another one $c$ and $d$, then 
$\frac{(b+a)-(d+c)}2=\frac{k\pi}2$ for some $k$ odd. 
Again assuming that $A_i$ has angle $\frac{2i\pi}{n}$ we get that $A_aA_b$ and $A_cA_d$ are perpendicular iff 
$kn=2(b+a)-2(c+d)$ for some odd $k$. 
This simply means $(b+a)-(c+d)\equiv \frac n2\pmod{n}$. 
In a similar fashion we can also derive $(b+a)-(c+d)\equiv 0\pmod{n}$ iff $A_aA_b$ and $A_cA_d$ are parallel. 

Now split the $F$ into two sets: $F_1=\{(A_aA_b): a+b\equiv 0\pmod{n}\}$ and $F_2=\{(A_aA_b): a+b\equiv \frac n2\pmod{n}\}$. 
Not hard to see that the diagonals in the same sets are parallel to one another and any two diagonals from different sets are perpendicular to each other. 
We claim that the two lines in $F$ are precisely the longest line in $F_1$ and the longest line in $F_2$. 
Indeed, let $AB$ and $CD$ be parallel lines on $\Gamma$ satisfying $CD\ge AB$, 
then $ABCD$ is an isoceles trapezoid satisfying $\angle A=\angle B\ge 90^{\circ}$ and $\angle C=angle D\le 90^{\circ}$. 
Denote also $E$ and $F$ as the perpendiculars from $A$ and $B$ on $CD$. It follows that $E$ and $F$ are on the closed segment $CD$ itself. 
Let $l$ be a line perpendicular to both $AB$ and $CD$, intersecting open segment $AB$ at $G$ and line $CD$ at $H$. 
With $AE\parallel BF\parallel GH$, the fact that $G$ lies on segment $AB$ means that $H$ lies on open segment $EF$ too, 
therefore lying on open segment $CD$ as well. 
This means, any perpendicular line intersecting $AB$ in its interior will intersect $CD$ in its interior too. 
Now let $l$ be the longest line in $F_1$, and $m$ be any line in $F_2$. Since there exists $n$ in $F_1$ such that $m$ intersects $n$ in its interior, and $n$ is no longer than $n$, $m$ must intersect $l$ in its interior too. This proves the claim that every line in $F_2$ intersects $l$ in its interior, and similarly every line in $F_1$ intersects the longest line in $F_2$ in its interior too. 

As for the third claim, 
denote the endpoints of the longest line in $F_1$ as $C$ and $D$, and in $F_2$ as $E$ and $F$ (which might or might not be completely distinct from $A$ and $B$). 
Let $G$ be a point on $AB$ and presumably it's an endpoint of a selected diagonal (with $G\neq A$ and $G\neq B$). 
W.L.O.G. we assume that $AB$ lies on minor arc $CE$ (so same goes for $G$, and the other endpoint, obviously, must also be on the minor arc $CE$). W.L.O.G. also that $C, A, G, B, E$ are on $\Gamma$ in that order.
By the choice of $A$ and $B$ (consecutive endpoints in $F$) we know this diagonal cannot be in $F_1$ or $F_2$, so it intersects 
none of the segment $CD$ or $EF$. 
Suppose $H$ is a point on the minor arc $BE$, and $H\neq B$. If the line in $F$ with endpoint $B$ does not intersect $GH$, 
then (by drawing) it intersects neither $EF$ nor $CD$, contradicting that every line in $F$ intersects either of them. 
Therefore $H$ cannot be the endpoint of the selected diagonal, and similarly, any point on the minor arc $CA$ that is not $A$ cannot be the endpoint of the selected diagonal. Thus that point must be on minor arc $AB$. 

In a similar way we can show that if $A, B, C$ are consecutive points of any diagonal in $F$ in that order, any diagonal with $B$ as endpoint must have another endpoint either in minor arc $AB$ or in minor arc $BC$. 
Thus, if $A_{t_1}A_{t_2}\cdots A_{t_k}$ are consecutive endpoints of $F$ then any remaining diagonals must have both endpoints lying on $A_{t_i}A_{t_{i+1}}$ for some $i\in [1,k]$, indices taken modulo $k$. 
Moreover, each endpoints belong to at most two diagonals, with at least four of them belonging exclusively to $l_1$ and $l_2$, the longest lines in $F_1$ and $F_2$ (they intersect every other lines in the opposite family of $F$ so they cannot share endpoint with other lines in $F$). 
Since each diagonal has two endpoints, the number of elements in $F$ cannot exceed $\frac 12(1+1+1+1+2+\cdots +2)$
$=\frac 12 (4+2(k-4))$
$=k-2$. 
Considering the polygon $A_{t_i}\cdots A_{t_{i+1}}$, which is a $t_{i+1}-t_{i}+1$-gon, 
we know that $t_{i+1}-t_{i}-2$ diagonals not intersecting each other can be drawn, plus one line $A_{t_i}A_{t_{i+1}}$ to be drawn. 
This gives $t_{i+1}-t_{i}-1$ lines, resulting in $\displaystyle\sum_{i=1}^k (t_{i+1}-t_i-1)$
$=t_{k+1}-t_1-k$
$=n-k$ 
extra diagonals, 
thus the upper bound is $n-k+k-2=n-2$. 

To achieve this bound, select $A_0A_k$, where $k=\frac n2$, and $A_1A_{n-1}$. 
Now take $A_1A_i$ with $i=3, 4, \cdots , k$ and $A_{n-1}A_i$ with $i=n-3, n-4, \cdots k+1$. 
This gives $2+2(k-2)=2k-2=n-2$. Q.E.D.

%play around with c7 characters
\newcommand{\person}{Feridun }
\newcommand{\pronoun}{he }
\newcommand{\possesive}{his }
\newcommand{\animal}{goose }
\newcommand{\animals}{geese }
\newcommand{\Animal}{Goose }
\newcommand{\Animals}{Geese }
\newcommand{\move}{jump }
\newcommand{\moves}{jumps }

\item[\textbf{C7/IMO 6}]
There are $n\ge 2$ line segments in the plane such that every two segments cross and no three segments meet at a point. \person  has to choose an endpoint of each segment and place a \animal on it facing the other endpoint. Then \pronoun will clap \possesive hands $n-1$ times. Every time \pronoun claps, each \animal will immediately jump forward to the next intersection point on its segment. \Animals never change the direction of their jumps. \person wishes to place the \animals in such a way that no two of them will ever occupy the same intersection point at the same time.

(a) Prove that \person can always fulfill \possesive wish if $n$ is odd.

(b) Prove that \person can never fulfill \possesive wish if $n$ is even.

\textbf{Thoughts.} 
We wish to play parity game to prove that the \animals cannot coincide in part (a). 
Intuitively, this makes sense: 
for each point segment $\ell$ there are even number of intersection points, 
which means that for each intersection point we can chacterize one side of having even number of points and the other odd 
(excluding the point itself), giving rise of the notion "even sides" and "odd sides" as below (inspired by the case $n=3$). 
This means for each two lines we can arrange pick one endpoints such that one endpoint is on even side and the other endpoint is on odd side (w.r.t. the intersection). 
What if these selections conflict with each other when $n$ lines are considered together? 
They won't, based on the Menelaus' logic below. 

In part (b), things are trickier: unfortunately we cannot say that the \animals will meet just because both of them come from even side or odd side: the number of intersections must be the same. 
This motivates another version of the solution: 
to find cases where the other lines such that for selected rays of some two selected lines, any other lines either intersect both rays or none of them. 
This means the \animals coming from both rays will intersect (same for the case when the \animals coming not from the rays). 
Denoting the angles of the line to $x$-axis as $\theta_1$ and $\theta_2$, then this happens whenever either all lines have angles in $(\theta_1, \theta_2)$ or all lines have angles not in this range. 
This shows that, if we rank the lines according to their angles, then any two lines neighbouring in ranking must have endpoints at different directions being chosen. 
Don't forget to include the comparison between the first and the last line to arrive at contradiction!

\textbf{Solution.} (a) Let the segments be $\ell_1, \ell_2, \cdots , \ell_n$. 
Let $P_{ij}$ be the intersection of line $ij$. For each segment $\ell_i$ we aim to investigate the number of points on each side of $P_{ij}$ (other than $P_{ij}$). 
Since there are $n-2$ such points (which is odd), 
one side has even number of points and the other side odd. 
We call this odd side of $\ell_i$ w.r.t. point $P_{ij}$. 

Now place the first \animal arbitrarily on $\ell_1$. For $i\in [2, n]$ we do the following: 
if the \animal corresponding to $\ell_1$ is place on the odd side of $\ell_1$ w.r.t. $P_{1i}$, 
\person places one \animal at the even side of $\ell_i$ w.r.t. $P_{1i}$ (and vice versa). 
We now proceed to the following claim: 
using the procedure detailed above, for each two distinct integers $i, j\in [1, n]$, 
the \animals corresponding to $\l_i$ and $\l_j$ lie on different parity of $l_i$ and $l_j$, respectively, 
both w.r.t. $P_{ij}$. 
Indeed, consider the triangle formed by lines $\ell_1$, $\ell_i$ and $\ell_j$. 
Menelaus' theorem says that any line either intersects none or two of the segments 
$P_{ij}P_{1i}$, $P_{1j}P_{1i}$, $P_{ij}P_{1j}$. 
Thus considering lines $l_k$ with $k\not\in\{1,i,j\}$ 
we know that it has even number of total intersection points with segments $P_{ij}P_{1i}$, $P_{1j}P_{1i}$, $P_{ij}P_{1j}$. 
If this number is even on $P_{1j}P_{1i}$, 
then each endpoint is on the odd side of $l_1$ w.r.t. one of $P_{1j}$ and $P_{1i}$, and even on the other. 
Thus according of our choice of placing the \animals, 
either one \animal come from the odd side of $l_i$ w.r.t. $P_{1i}$ and the other from even side of $l_j$ w.r.t. $P_{1j}$, 
or vice versa. 
The intersection with $P_{ij}P_{1i}$ and $P_{ij}P_{1j}$ will be both odd or both even. 
If it's both odd and in the first case (one \animal come from the odd side of $l_i$ w.r.t. $P_{1i}$ and the other from even side of $l_j$ w.r.t. $P_{1j}$), 
then the \animal corresponding to $i$ come from the odd side of $l_i$ w.r.t. $P_{1i}$ and the other from even side of $l_j$ w.r.t. $P_{1j}$, which works for this pair of $(i,j)$. 
The other three subcases can be treated equally. 
If this number is odd on $P_{ij}P_{1i}$, 
then each endpoint is on the odd side of $l_i$ w.r.t. both $P_{1i}$ and $P_{1j}$, or vice versa (both even). 
According to our choice again, 
both \animals come from the odd side of $l_i$ w.r.t. $P_{1i}$ and the and of $l_j$ w.r.t. $P_{1j}$, or both from the even side of their respective lines. 
The intersection with $P_{ij}P_{1i}$ and $P_{ij}P_{1j}$ will be one odd and one even, 
for the same endpoint w.r.t the lines $l_i$ and $l_j$, exactly one of them will change sign when switching from $P_{1i}$ to $P_{ij}$ and from $P_{1j}$ to $P_{ij}$. 
Again this $(i, j)$ works. 

Finally, to see why the \animals won't intersect at the same time, observe that  if this happens for some of $(i, j)$, then the \animals must have encountered the same number of points before. This implies that they have to come both from the odd side or the even side of the line, contradiction. 

(b) Let $\ell_1, \ell_2, \cdots , \ell_n$ be the segments, and let $\theta_i\in [0, \pi)$ be the angle need for $x$-axis to rotate counterclockwise to reach $\ell_i$. 
W.L.O.G let $0=\theta_1 < \theta_2< \cdots < \theta_n$. For convinience we introduce $\ell_{n+1}=\ell_1$ with $\theta_{n+1}=\pi$. 

For each segment \person has the choice of placing the \animal in the direction of $\theta_i$ or $\theta_i +\pi$ compared to the positive $x$-direction. Let's investigate $\ell_i$ and $\ell_{i+1}$ together. 
Now, suppose that $\ell_i$ and $\ell_j$ intersect at $P_{ij}$ for some $j$; define $\ell_{(i+1)j}$ similarly. 
We know that for each $j$ different from $i$, $i+1$, $\ell_i$, $\ell_{i+1}$ and $\ell_j$ are in that order (in anticlockwise angle, cycles allowed). 
This means that $P_{ij}P_{i(i+1)}, P_{i(i+1)}P_{(i+1)j}, P_{(i+1)j}P_{ij}$ must also be in that order, 
forcing $P_{ij}, P_{i(i+1)}, P_{(i+1)j}$ to be in clockwise order. 
From here we infer that the vectors $P_{i(i+1)}P_{ij}$ and $P_{i(i+1)}P_{(i+1)j}$ either have directions ($\theta_i$, $\theta_{i+1}$) or ($\theta_i +\pi$, $\theta_{i+1} +\pi$), 
and considering all such $j$'s, we know that there are equal number of intersection points lying on the half-line starting from $P_{i(i+1)}$ and extending in the $\theta_i$ direction, and on the half-line starting from $P_{i(i+1)}$ and extending in the $\theta_{i+1}$ direction. 
This means that the \animals will collide when both placed in the $\theta_i, theta_{i+1}$ direction, or $\theta_i +\pi, theta_{i+1}+\pi$ direction, 
forcing the directions to be  $\theta_i, \theta_{i+1}+\pi$ or  $\theta_i +\pi, \theta_{i+1}$. 

Summarizing above, if we let $\theta_1$ to be the direction headed by the first \animal
the directions must be $\theta_1, \theta_2+\pi, \theta_3, \theta_4+\pi, \cdots , \theta_n+\pi$ ($n$ is even). 
Recall that we can also compare it with the "$n+1$-th" line (which is the first) and it has to have direction $\theta_{n+1}=\theta_1 +\pi$, 
which contradicts our choice of making the first $\animal$ facing the direction $\theta_1$. 
(The case where $\theta_1+\pi$ is chosen as the direction is completely analogous: if one configuration works, then the completely opposite configuration works too). 
\end{itemize}

\section{Geometry}
\begin{itemize}
\item[\textbf{G1/IMO 1}]
Triangle $BCF$ has a right angle at $B$. Let $A$ be the point on line $CF$ such that $FA=FB$ and $F$ lies between $A$ and $C$. Point $D$ is chosen so that $DA=DC$ and $AC$ is the bisector of $\angle{DAB}$. Point $E$ is chosen so that $EA=ED$ and $AD$ is the bisector of $\angle{EAC}$. Let $M$ be the midpoint of $CF$. Let $X$ be the point such that $AMXE$ is a parallelogram. Prove that $BD,FX$ and $ME$ are concurrent.

\textbf{Solution.} 
The fact that $\angle CBF=90^{\circ}$ and $M$ being the midpoint of $CF$ should very well suggest us to draw the circumcircle of $\triangle BCF$. 
As it turns out, $D$ and $X$ seems to lie on this circle (and that's almost everything we need). 
Now $\angle DCF=\angle DCA=\angle DAC=\angle BAF=\angle ABF=90^{\circ}-\frac 12\angle BFC$. 
$DC=DA=\frac{CA}{2\cos\angle DCA}$
                            $=\frac{CF-AF}{2\cos(90^{\circ}-\frac 12\angle BFC)}$
                            $=\frac{CF-BF}{2\sin\frac 12\angle BFC}$
                            $=\frac{CF-CF\cos\angle BFC}{2\sin\frac 12\angle BFC}$
                            $=CF\cdot\frac{1-(1-2\sin ^2\frac 12\angle BFC)}{2\sin\frac 12\angle BFC}$
                            $=CF\cdot\sin \frac 12\angle BFC$
                            $=CF\cdot\cos (90^{\circ}-\frac 12\angle BFC)$
                            $=CF\cdot\cos\angle DCF$.
                            If $D'$ is on ray CD satisfying $\angle CD'F=90^{\circ}$ we have $CD'=CF\cos\angle D'CF=CF\cos\angle DCF=CD$, so
                            $D=D'$ and $D$ lies on the circumcircle of $BCF$. 
                            Moreover, $\angle DFC=90^{\circ}-\angle DCF=\frac 12\angle BFC=\angle BFD$, 
                            so $BD$ and $DC$ subtend the same angle and $BD=DC$. 

Now $EA=ED$ and $\angle CAD=\angle EAD$ so $\angle EDA=\angle EAD=\angle CAD$, so $ED\parallel CM$. 
                            With $EX\parallel AM$ we have $E, X, D$ collinear, too. 
                            Moreover, $DE=\frac{DA}{2\cos\angle EDA}$
                            $=\frac{DA}{2\cos\angle CAD}$
                            $=\frac{DC}{2\cos\angle DCA}$
                            $=\frac{CF\cos\angle DCF}{2\cos\angle DCF}$
                            $=\frac{CF}{2}$
                            $=CM$,
                            so $CMED$ is a parallelogram. 
                            Thus $\angle DEM=\angle DCM=\angle DCA=\angle CAD$, meaning that $DEMA$ is an isoceles trapezoid (so $\angle EAM=\angle EDM$). 
                            With $AMXE$ parallelogram we also have $\angle DXM=\angle EXM=angle EAM=\angle EDM=\angle XDM$, so $MX=MD$ and $X$ lies on the circumcircle of $BCF$, too. 
                            With $DX\parallel CF$ we conclude that $DXFC$ is an isoceles trapezoid so $CD=FX$, from previous identity we have $BD=DC$ so $FX=BD$.

Finally, we already had $F,E,X$ collinear and $\angle DBA=\angle DBC+\angle CBA$
                            $=\angle DFC+(90^{\circ}-\angle FBA)$
                            $=\angle DFC+(\frac 12\angle BFC)$
                            $=\angle DFC+(\angle DFC)$
                            $=\angle DMC$, 
                            so $B$ lies on the circle containing $D,E,A,M$ too. This means $\angle BED=\angle BMD$, 
                            and since $BD$ and $FX$ subtend the same angle on circle $BCF$ we have 
                            $\angle BMD=\angle FMX=\angle FEX$ (the last equality follows from that $EXFM$ is isoceles trapezoid, hence cyclic). 
                            Therefore, $B, E, F$ are in fact collinear, and $E$ is the intersection of $DX$ and $BF$.
                            Hence $ME$ is the common perpendicular bisector of segments $BX$ and $DF$ (since $BXFD$ is an isoceles trapezoid), 
                            and the intersection of $BD$ and $FX$ will lie on this perpendicular bisector too.

\item[\textbf{G2}]
Let $ABC$ be a triangle with circumcircle $\Gamma$ and incenter $I$ and let $M$ be the midpoint of $\overline{BC}$. The points $D$, $E$, $F$ are selected on sides $\overline{BC}$, $\overline{CA}$, $\overline{AB}$ such that $\overline{ID} \perp \overline{BC}$, $\overline{IE}\perp \overline{AI}$, and $\overline{IF}\perp \overline{AI}$. Suppose that the circumcircle of $\triangle AEF$ intersects $\Gamma$ at a point $X$ other than $A$. Prove that lines $XD$ and $AM$ meet on $\Gamma$.

\definecolor{zzttqq}{rgb}{0.6,0.2,0.}
\definecolor{uuuuuu}{rgb}{0.26666666666666666,0.26666666666666666,0.26666666666666666}
\begin{tikzpicture}[line cap=round,line join=round,>=triangle 45,x=1.0cm,y=1.0cm]
\clip(-2.270656607672633,-3.796329905018268) rectangle (18.226000561693667,5.8638337348048095);
\fill[line width=1.2pt,color=zzttqq,fill=zzttqq,fill opacity=0.10000000149011612] (2.6740633796648576,3.4476258129586164) -- (2.42,-1.08) -- (3.0332828475238647,0.5873608084337415) -- cycle;
\fill[line width=1.2pt,color=zzttqq,fill=zzttqq,fill opacity=0.10000000149011612] (2.6740633796648576,3.4476258129586164) -- (7.861779119227031,1.5355684211039) -- (10.16,-0.98) -- cycle;
\draw [line width=1.2pt] (6.264092589154242,0.9752335994617323) circle (4.359017433131806cm);
\draw [line width=1.2pt] (5.201874941490228,2.312402981551107) circle (2.771021863689802cm);
\draw [line width=1.2pt] (6.907124929188343,-3.3360936198545272)-- (2.6740633796648576,3.4476258129586164);
\draw [line width=1.2pt] (4.667907423661546,5.031491221927057)-- (6.907124929188343,-3.3360936198545272);
\draw [line width=1.2pt] (4.667907423661546,5.031491221927057)-- (2.42,-1.08);
\draw [line width=1.2pt] (2.42,-1.08)-- (10.16,-0.98);
\draw [line width=1.2pt] (4.667907423661546,5.031491221927057)-- (10.16,-0.98);
\draw [line width=1.2pt] (3.0332828475238647,0.5873608084337415)-- (7.861779119227031,1.5355684211039);
\draw [line width=1.2pt] (5.447530983375448,1.061464614768821)-- (5.474688584848092,-1.040533739213849);
\draw [line width=1.2pt] (4.667907423661546,5.031491221927057)-- (5.447530983375448,1.061464614768821);
\draw [line width=1.2pt] (5.447530983375448,1.061464614768821)-- (2.42,-1.08);
\draw [line width=1.2pt] (5.447530983375448,1.061464614768821)-- (10.16,-0.98);
\draw [line width=1.2pt] (2.6740633796648576,3.4476258129586164)-- (2.42,-1.08);
\draw [line width=1.2pt] (2.6740633796648576,3.4476258129586164)-- (3.0332828475238647,0.5873608084337415);
\draw [line width=1.2pt] (2.6740633796648576,3.4476258129586164)-- (7.861779119227031,1.5355684211039);
\draw [line width=1.2pt] (2.6740633796648576,3.4476258129586164)-- (10.16,-0.98);
\draw [line width=1.2pt,color=zzttqq] (2.6740633796648576,3.4476258129586164)-- (2.42,-1.08);
\draw [line width=1.2pt,color=zzttqq] (2.42,-1.08)-- (3.0332828475238647,0.5873608084337415);
\draw [line width=1.2pt,color=zzttqq] (3.0332828475238647,0.5873608084337415)-- (2.6740633796648576,3.4476258129586164);
\draw [line width=1.2pt,color=zzttqq] (2.6740633796648576,3.4476258129586164)-- (7.861779119227031,1.5355684211039);
\draw [line width=1.2pt,color=zzttqq] (7.861779119227031,1.5355684211039)-- (10.16,-0.98);
\draw [line width=1.2pt,color=zzttqq] (10.16,-0.98)-- (2.6740633796648576,3.4476258129586164);
\begin{scriptsize}
\draw [fill=uuuuuu] (4.667907423661546,5.031491221927057) circle (2.0pt);
\draw[color=uuuuuu] (4.537798208586435,5.46281218149112) node {$A$};
\draw [fill=uuuuuu] (2.42,-1.08) circle (2.0pt);
\draw[color=uuuuuu] (2.024729807820653,-1.2921730943261232) node {$B$};
\draw [fill=uuuuuu] (10.16,-0.98) circle (2.0pt);
\draw[color=uuuuuu] (10.383801296892647,-1.0782949325588227) node {$C$};
\draw [fill=uuuuuu] (5.447530983375448,1.061464614768821) circle (2.0pt);
\draw[color=uuuuuu] (5.571542657128387,1.3634807476178594) node {$I$};
\draw [fill=uuuuuu] (5.474688584848092,-1.040533739213849) circle (2.0pt);
\draw[color=uuuuuu] (5.357664495361086,-1.2921730943261232) node {$D$};
\draw [fill=uuuuuu] (7.861779119227031,1.5355684211039) circle (2.0pt);
\draw[color=uuuuuu] (8.155903778483268,1.6308284498269852) node {$E$};
\draw [fill=uuuuuu] (3.0332828475238647,0.5873608084337415) circle (2.0pt);
\draw[color=uuuuuu] (2.648541112975279,0.7931389829050577) node {$F$};
\draw [fill=uuuuuu] (2.6740633796648576,3.4476258129586164) circle (2.0pt);
\draw[color=uuuuuu] (2.1851384291461278,3.698317346910891) node {$X$};
\draw [fill=uuuuuu] (6.29,-1.03) circle (2.0pt);
\draw[color=uuuuuu] (6.409232124050314,-0.7396545097605967) node {$M$};
\draw [fill=uuuuuu] (6.907124929188343,-3.3360936198545272) circle (2.0pt);
\draw[color=uuuuuu] (7.050866609352216,-2.8071434068445025) node {$N$};
\end{scriptsize}
\end{tikzpicture}

\textbf{Solution.}
W.L.O.G. let $AB<AC$. 
First, well-known spiral similarity property should dictate the similarity of triangles $BXF$ an $CXE$, so $\frac{CX}{CE}=\frac{BX}{BF}$.
Also, let's also invoke an identity for triangles (feel free to verify it; I'm not gonna do this):
$$\frac{BX}{XC}\cdot \frac{\sin\angle BXD}{\sin\angle CXD}=\frac{BD}{DC}.$$Denoting $N_1$ as the other intersection of $XD$ and $\Gamma$ gives $\frac{\sin\angle BXD}{\sin\angle CXD}=\frac{BN_1}{CN_1}.$
Similarly we have $\frac{AB}{AC}\cdot \frac{\sin\angle ABM}{\sin\angle ACM}=\frac{BM}{CM}=1$.
ALso let $N_2$ as the other intersection of $AM$ and $\Gamma$ and we have $\frac{\sin\angle ABM}{\sin\angle ACM}=\frac{BN_2}{CN_2}.$
Therefore all we need is $\frac{\sin\angle ABM}{\sin\angle ACM}=\frac{\sin\angle BXD}{\sin\angle CXD}$,
and it's not hard to see that $\frac{\sin\angle ABM}{\sin\angle ACM}=\frac{AC}{AB}$,
so we are left with proving the fact
$\frac{BF}{EC}\cdot\frac{AC}{AB}=\frac{BD}{DC}$.

Now, $\frac{BD}{DC}=\frac{\tan\frac12\angle C}{\tan\frac12\angle B}$,
$\frac{AC}{AB}=\frac{\sin\angle B}{\sin\angle C}=\frac{2\sin\frac 12\angle B\cos\frac 12\angle B}{2\sin\frac 12\angle C\cos\frac 12\angle C}$.
Also $IE=IF$, and by angle chasing we have $\angle FIB=\angle ICE=\frac12\angle C$,
$\angle EIC=\angle IBF=\frac12\angle B$.
Therefore $BIF$ and $ICE$ similar, yielding $\frac{BF}{EC}=(\frac{BF}{FI})^2=(\frac{\sin\frac12\angle C}{\sin\frac12\angle B})^2$,
now it's no longer difficult to prove that $(\frac{\sin\frac12\angle C}{\sin\frac12\angle B})^2\cdot \frac{2\sin\frac 12\angle B\cos\frac 12\angle B}{2\sin\frac 12\angle C\cos\frac 12\angle C}=\frac{\tan\frac12\angle C}{\tan\frac12\angle B}$.

\item[\textbf{G3}]
Let $B = (-1, 0)$ and $C = (1, 0)$ be fixed points on the coordinate plane. A nonempty, bounded subset $S$ of the plane is said to be nice if

$\text{(i)}$ there is a point $T$ in $S$ such that for every point $Q$ in $S$, the segment $TQ$ lies entirely in $S$; and

$\text{(ii)}$ for any triangle $P_1P_2P_3$, there exists a unique point $A$ in $S$ and a permutation $\sigma$ of the indices $\{1, 2, 3\}$ for which triangles $ABC$ and $P_{\sigma(1)}P_{\sigma(2)}P_{\sigma(3)}$ are similar.

Prove that there exist two distinct nice subsets $S$ and $S'$ of the set $\{(x, y) : x \geq 0, y \geq 0\}$ such that if $A \in S$ and $A' \in S'$ are the unique choices of points in $\text{(ii)}$, then the product $BA \cdot BA'$ is a constant independent of the triangle $P_1P_2P_3$.

\definecolor{uuuuuu}{rgb}{0.26666666666666666,0.26666666666666666,0.26666666666666666}
\begin{tikzpicture}[line cap=round,line join=round,>=triangle 45,x=2.0cm,y=2.0cm]
\clip(-1.7992213964115553,-0.5528722288194026) rectangle (4.3785964850163905,2.3587601987753266);
\draw [line width=1.2pt] (0.,1.7320508075688772)-- (0.,0.);
\draw [line width=1.2pt] (0.,0.)-- (3.,0.);
\draw [shift={(-1.,0.)},line width=1.2pt]  plot[domain=0.:1.0471975511965976,variable=\t]({1.*2.*cos(\t r)+0.*2.*sin(\t r)},{0.*2.*cos(\t r)+1.*2.*sin(\t r)});
\draw [shift={(1.,0.)},line width=1.2pt]  plot[domain=0.:2.0943951023931957,variable=\t]({1.*2.*cos(\t r)+0.*2.*sin(\t r)},{0.*2.*cos(\t r)+1.*2.*sin(\t r)});
\draw [line width=1.2pt] (-1.,0.)-- (0.,0.);
\draw [line width=1.2pt,dash pattern=on 3pt off 3pt] (-1.,0.)-- (2.2421749535956774,1.567482498996266);
\draw [line width=1.2pt] (0.4643237490478057,0.70795126183782)-- (1.,0.);
\draw [line width=1.2pt] (1.2141107495551722,1.0704480481283867)-- (1.,0.);
\draw (0.23140048113606526,1.2521249956673668) node[anchor=north west] {$S$};
\draw (1.0103427357508932,1.6281660841021104) node[anchor=north west] {$S'$};
\begin{scriptsize}
\draw [fill=uuuuuu] (-1.,0.) circle (2.0pt);
\draw[color=uuuuuu] (-1.127719452778083,0.008503396058179014) node {$B$};
\draw [fill=uuuuuu] (1.,0.) circle (2.0pt);
\draw[color=uuuuuu] (0.967366611358351,-0.13654102376665067) node {$C$};
\draw [fill=uuuuuu] (3.,0.) circle (2.0pt);
\draw[color=uuuuuu] (3.1376608931817334,-0.10968094602131184) node {$T'$};
\draw [fill=uuuuuu] (0.,0.) circle (2.0pt);
\draw[color=uuuuuu] (-0.010340218571984802,-0.11505296157037961) node {$F$};
\draw [fill=uuuuuu] (0.4643237490478057,0.70795126183782) circle (2.0pt);
\draw[color=uuuuuu] (0.44090908754970864,0.8196777439674117) node {$A$};
\draw [fill=uuuuuu] (1.2141107495551722,1.0704480481283867) circle (2.0pt);
\draw[color=uuuuuu] (1.209107311066401,1.1957188324021553) node {$A'$};
\end{scriptsize}
\end{tikzpicture}

\textbf{Solution.} We show that the following works: $S=\{(x,y): x\ge 0, y\ge 0, (x+1)^2+y^2\le 4\}$ and $S'=\{x\ge 0, y\ge 0, (x+1)^2+y^2\ge 4, (x-1)^2+y^2\le 4\}$. We claim that $BA\cdot BA'=4$ for those choices. 
Denote triangles $ABC$ and $DEF$ as quasi-similar if there exists a permutation $\sigma$ of $\{D,E,F\}$ with $ABC$ and $\sigma(D)\sigma(E)\sigma(F)$ as similar. 

We first start with the following claim: 
for every point $A$ above $x$-axis, $A\in S$ iff $AC\le AB\le BC$ and $A'\in S'$ iff $A'C\le BC\le A'B$. \\
Proof: let's first investigate all points $A\in S$ and $A'\in S'$. 
Let $A=(x,y)$ be arbitrary. 
Now, $AC\le AB\lra(x-1)^2+y^2\le  (x+1)^2+y^2\lra 0\le x$. 
$BC\ge AB\lra 2^2\ge (x+1)^2+y^2\lra (x+1)^2+y^2\le 4$. 
Therefore $AC\le AB\le BC$ iff $(x,y)$ satisfies both $x\ge 0$ and $(x+1)^2+y^2\le 4$ iff $(x,y)\in S$. 
Next, $AC\le BC$ iff 
$AC\le 2$ or $(x-1)^2+y^2\le 4$ 
and $BC\le AB$ iff
$(x-1)^2+y^2\ge 4$, 
therefore $AC\le BC\le AB$ iff the two conditions are satisfied (and from here $(x-1)^2+y^2\le (x-1)^2+y^2$ so $x\ge 0$ is implied), 
and this is equivalent to $A\in S'$. 

We are now ready to justify our selection: 
\begin{itemize}
\item[(i)]
In $S$ we can simply take any point as $T$, since the boundaries $x$-axis with $x\in [0, \sqrt{3}]$, $y$-axis with $y\in [0, 1], $ and the arc of the circle $(x-1)^2+y^2=4$ with $x\ge 0$ are concave. 
In $S'$ we take $T=(3,0)$. 

\item[(ii)] Observe that the objective is equivalent to: 
for each triangle $P_1P_2P_3$ there is a unique point $A$ with triangle $ABC$ quasisimilar to $P_1P_2P_3$ satisfying $AC\le AB\le BC$, 
and another unique point $A'$ with $A'BC$ quasisimilar to $P_1P_2P_3$ and $A'C\le BC\le A'B$. 
Moreover we want to prove that $BA\cdot BA'=BC^2=4$. 
We will also use the fact that for each triangle $DEF$ there is a unique $A$ above $x$-axis that is similar to $DEF$ (with $B, C$ fixed). 
We split into the following cases: 
\begin{itemize}
\item [Case 1.] $P_1P_2P_3$ equilateral. The only point $A$ with $y\ge 0$ satisfying this is 
$A=(0, \sqrt{3})$. 
Since we have $(x+1)^2+y^2=(x-1)^2+y^2=1+3=4$, $A$ lies in both $S$ and $S'$ and we have $BA\cdot BA=2\times 2=4$. 
\item [Case 2.] $P_1P_2P_3$ is isoceles, with the two equal sides longer than the other. 
Now, let $P_1P_2=P_1P_3>P_2P_3$. 
This means, if $AC$ is the shortest and if triangle $ABC$ is quasisimilar to $P_1P_2P_3$ then $AC$ corresponds to $P_2P_3$, 
and $AB, BC$ correspond to $P_1P_2$ and $P_1P_3$, which implies that 
$AB=BC=2$. 
Such $A$ can be uniquely constructed, and with $AC < AB=BC=2$ we have $A$ lies in $S$. 
Similarly, if $A'C$ is the shortest side of $A'BC$ and if triangle $A'BC$ is quasisimilar to $P_1P_2P_3$ then $A'C$ corresponds to $P_2P_3$, and $A'B, BC$ corresponds to $P_1P_2$ and $P_1P_3$, so 
$A'$ can also be uniquely constructed (which turns out to be equal to $A$ in this case). 
Therefore, $A'C < BC=A'B=2$, which implies that $A'$ is in $S'$, and moreover $BA\cdot BA'=2\times 2=4$. 

\item [Case 3.] $P_1P_2P_3$ is isoceles with the two equal sides longer than the other. 
Now, let $P_1P_2=P_1P_3<P_2P_3$. 
This means $AC$ and $A'C$ corresponds to $P_1P_2$. 
In $ABC$, we know that $AB\le BC$ implies $AB$ corresponds to $P_1P_3$, $BC$ corresponds to $P_2P_3$;
In $ABC$, we know that $BC\le A'B$ implies $BC$ corresponds to $P_1P_3$, $A'B$ corresponds to $P_2P_3$. 
Therefore $\frac{AB}{BC}=\frac{P_1P_3}{P_2P_3}=\frac{BC}{A'B}$ and $BC^2=AB\cdot A'B$. 
\end{itemize}

\item [Case 4.] $P_1P_2P_3$ scalene, and let $P_1P_2<P_1P_3<P_2P_3$. 
This means $AC$ and $A'C$ corresponds to $P_1P_2$. 
In $ABC$, we know that $AB\le BC$ implies $AB$ corresponds to $P_1P_3$, $BC$ corresponds to $P_2P_3$;
In $ABC$, we know that $BC\le A'B$ implies $BC$ corresponds to $P_1P_3$, $A'B$ corresponds to $P_2P_3$. 
Therefore $\frac{AB}{BC}=\frac{P_1P_3}{P_2P_3}=\frac{BC}{A'B}$ and $BC^2=AB\cdot A'B$. 
\end{itemize}

\item[\textbf{G4}]
Let $ABC$ be a triangle with $AB = AC \neq BC$ and let $I$ be its incentre. The line $BI$ meets $AC$ at $D$, and the line through $D$ perpendicular to $AC$ meets $AI$ at $E$. Prove that the reflection of $I$ in $AC$ lies on the circumcircle of triangle $BDE$.

\definecolor{uuuuuu}{rgb}{0.26666666666666666,0.26666666666666666,0.26666666666666666}
\begin{tikzpicture}[line cap=round,line join=round,>=triangle 45,x=1.0cm,y=1.0cm]
\clip(2.508533677869763,-3.670896102057754) rectangle (20.155064472682568,4.645990585584426);
\draw [line width=1.2pt] (6.3,0.64) circle (3.7402481873228917cm);
\draw [line width=1.2pt] (6.33659146518181,4.380069192869033)-- (3.9870541679262246,-2.299343137618459);
\draw [line width=1.2pt] (6.33659146518181,4.380069192869033)-- (8.554993680980937,-2.3440341823625643);
\draw [line width=1.2pt] (3.9870541679262246,-2.299343137618459)-- (8.554993680980937,-2.3440341823625643);
\draw [line width=1.2pt] (3.9870541679262246,-2.299343137618459)-- (9.083041002455232,1.2728647515940277);
\draw [line width=1.2pt] (6.33659146518181,4.380069192869033)-- (6.287016087198279,-0.6871054337761034);
\draw [shift={(8.67788250632608,-5.057302928688166)},line width=1.2pt,dash pattern=on 4pt off 4pt]  plot[domain=1.4392311051423707:2.610082583237681,variable=\t]({1.*5.441526689219892*cos(\t r)+0.*5.441526689219892*sin(\t r)},{0.*5.441526689219892*cos(\t r)+1.*5.441526689219892*sin(\t r)});
\draw [line width=1.2pt] (6.292107838093549,-0.16666984677409122)-- (6.53504758519073,-0.5132391930122152);
\begin{scriptsize}
\draw [fill=uuuuuu] (3.9870541679262246,-2.299343137618459) circle (2.0pt);
\draw[color=uuuuuu] (3.7873931802595675,-2.4783712991910054) node {$B$};
\draw [fill=uuuuuu] (6.33659146518181,4.380069192869033) circle (2.0pt);
\draw[color=uuuuuu] (6.446073025680614,4.644264829899929) node {$A$};
\draw [fill=uuuuuu] (8.554993680980937,-2.3440341823625643) circle (2.0pt);
\draw[color=uuuuuu] (8.891402019308613,-2.4291364872387637) node {$C$};
\draw [fill=uuuuuu] (6.287016087198279,-0.6871054337761034) circle (2.0pt);
\draw[color=uuuuuu] (6.183487361935326,-0.9192689207033586) node {$I$};
\draw [fill=uuuuuu] (7.685028544826755,0.2928796589089621) circle (2.0pt);
\draw[color=uuuuuu] (7.7918245523752185,0.557775437863886) node {$D$};
\draw [fill=uuuuuu] (6.292107838093549,-0.16666984677409122) circle (2.0pt);
\draw[color=uuuuuu] (6.150664153967165,0.14748533826187354) node {$E$};
\draw [fill=uuuuuu] (9.39173461207809,0.3371968516998012) circle (2.0pt);
\draw[color=uuuuuu] (9.547866178671836,0.6070102498161274) node {$I'$};
\draw [fill=uuuuuu] (9.083041002455232,1.2728647515940277) circle (2.0pt);
\draw[color=uuuuuu] (9.301692118910626,1.5424716769087157) node {$I''$};
\draw [fill=uuuuuu] (6.53504758519073,-0.5132391930122152) circle (2.0pt);
\draw[color=uuuuuu] (6.823539917314466,-0.5418020290695073) node {$P$};
\end{scriptsize}
\end{tikzpicture}

\textbf{Solution.} 
Let $I'$ be the reflection of $I$ in $AC$. 
Observe that $AC$ is an angle bisector of $\angle BDI'$ by the definition of $I'$, 
and since $DE\perp AC$, 
$DE$ is another angle bisector of this angle. 
This implies that the intersection of $DE$ and the circumcircle of $BDI'$ (other than $D$) is equidistant from $B$ and $I'$, 
i.e. on the perpendicular bisector of $BI'$. 
It therefore suffices to prove that $E$ lies on this perpendicular bisector, or $BE=EI'$. 

Let $I''$ be the image of $I$ when reflected in $D$, then $DI=DI'=DI''$. 
Moreover, $I''$ lies on line $BD$, which entails that $I'I''$ is parallel to $AC$ and perpendicular to $DE$. 
Therefore, $DE$ is the perpendicular bisector of $I'I''$ and $EI''=EI'$. 
The problem is now reduced to proving $BE=EI''$. 
Let $P$ be the foot of perpendicular from $E$ to $BD$, 
then the problem is now equivalent to proving that $P$ is the midpoint of $BI''$. 
Knowing that $BP=BI+ID-PD$ and $PI''=PD+DI''=PD+DI$ it suffices to prove that $BI=2PD$. 

Denote the common angles $\angle ABI, \angle IBC, \angle ICB, \angle ACI$ as $\alpha$. 
Then, $\angle ADB=3\alpha$, and $\angle IDE=|90^{\circ}-3\alpha|$ (as we will see, we are only interested in the cosine of this angle so don't worry about the sign). 
So $\frac{PD}{BI}$
$=\frac{DE\cos\angle IDE}{BI}$
$=\frac{AD\tan\angle DAI\cos\angle IDE}{BI}$
$=\frac{AB\tan\angle DAI\cos\angle IDE\sin\angle ABD}{BI\sin\angle ADB}$
$=\frac{BI\tan\angle DAI\cos\angle IDE\sin\angle ABD\sin\angle AIB}{BI\sin\angle ADB\sin\angle BAI}$
$=\frac{\tan(90^{\circ}-2\alpha)\cos|90^{\circ}-3\alpha|\sin\alpha\sin(90^{\circ}+\alpha)}{\sin(3\alpha)\sin(90^{\circ}-2\alpha)}$
$=\frac{\sin(90^{\circ}-2\alpha)\sin(3\alpha)\sin\alpha\cos\alpha}{\sin(3\alpha)\sin(90^{\circ}-2\alpha)\cos(90^{\circ}-2\alpha)}$
$=\frac{\sin\alpha\cos\alpha}{\sin (2\alpha)}$
$=\frac{\sin\alpha\cos\alpha}{2\sin\alpha\cos\alpha}$
$=\frac 12$, 
as $\cos|90^{\circ}-x|=\cos(90^{\circ}-x)=\sin x$, $\sin(90^{\circ}+\alpha)=\cos\alpha$ and $\tan x=\frac{\sin x}{\cos x}$. 

\item[\textbf{G5}]
Let $D$ be the foot of perpendicular from $A$ to the Euler line (the line passing through the circumcentre and the orthocentre) of an acute scalene triangle $ABC$. A circle $\omega$ with centre $S$ passes through $A$ and $D$, and it intersects sides $AB$ and $AC$ at $X$ and $Y$ respectively. Let $P$ be the foot of altitude from $A$ to $BC$, and let $M$ be the midpoint of $BC$. Prove that the circumcentre of triangle $XSY$ is equidistant from $P$ and $M$.

\definecolor{uuuuuu}{rgb}{0.26666666666666666,0.26666666666666666,0.26666666666666666}
\begin{tikzpicture}[line cap=round,line join=round,>=triangle 45,x=1.0cm,y=1.0cm]
\clip(-0.780894732325327,-3.0144161105989133) rectangle (15.499927939805652,4.658806400961947);
\draw [line width=1.2pt] (3.968556783697015,2.1726396229121065) circle (1.8545634959840431cm);
\draw [line width=1.2pt] (4.565154539765928,-0.2774755660589251) circle (2.5217044477442268cm);
\draw [line width=1.2pt] (3.1345454545454556,3.8290909090909078)-- (1.86,-1.28);
\draw [line width=1.2pt] (1.86,-1.28)-- (9.94,-1.04);
\draw [line width=1.2pt] (3.1345454545454556,3.8290909090909078)-- (9.94,-1.04);
\draw [line width=1.2pt] (3.1345454545454556,3.8290909090909078)-- (3.2850433621226975,-1.2376719793428905);
\draw [line width=1.2pt] (3.1345454545454556,3.8290909090909078)-- (5.852548986261354,0.4375174625344114);
\draw [line width=1.2pt] (5.852548986261354,0.4375174625344114)-- (5.9,-1.16);
\draw [line width=1.2pt] (2.454233092884584,1.1020185321051048)-- (5.805563901771507,1.9180603449254363);
\draw [line width=1.2pt] (3.968556783697015,2.1726396229121065)-- (4.565154539765928,-0.2774755660589251);
\draw [line width=1.2pt] (3.1819964682841015,2.2315734465564963)-- (4.493547220403405,2.13330418581266);
\draw [line width=1.2pt] (3.1345454545454556,3.8290909090909078)-- (3.968556783697015,2.1726396229121065);
\draw [line width=1.2pt] (4.493547220403405,2.13330418581266)-- (4.565154539765928,-0.2774755660589251);
\begin{scriptsize}
\draw [fill=uuuuuu] (3.1345454545454556,3.8290909090909078) circle (2.0pt);
\draw[color=uuuuuu] (3.2397606058357145,4.0571238239484115) node {$A$};
\draw [fill=uuuuuu] (1.86,-1.28) circle (2.0pt);
\draw[color=uuuuuu] (1.9656092662776379,-1.0394815342838948) node {$B$};
\draw [fill=uuuuuu] (9.94,-1.04) circle (2.0pt);
\draw[color=uuuuuu] (10.035234416812123,-0.8129657405846811) node {$C$};
\draw [fill=uuuuuu] (5.852548986261354,0.4375174625344114) circle (2.0pt);
\draw[color=uuuuuu] (5.957950130226277,0.6735441555664081) node {$O$};
\draw [fill=uuuuuu] (5.9,-1.16) circle (2.0pt);
\draw[color=uuuuuu] (6.000421841544879,-0.9262236374342878) node {$M$};
\draw [fill=uuuuuu] (3.2850433621226975,-1.2376719793428905) circle (2.0pt);
\draw[color=uuuuuu] (3.381332976897723,-0.9970098229652922) node {$P$};
\draw [fill=uuuuuu] (3.2294474820227475,0.6340559840220845) circle (2.0pt);
\draw[color=uuuuuu] (3.3247040284729197,0.8717454750532201) node {$H$};
\draw [fill=uuuuuu] (3.1819964682841015,2.2315734465564963) circle (2.0pt);
\draw[color=uuuuuu] (3.282232317154317,2.4715132680539162) node {$F$};
\draw [fill=uuuuuu] (4.493547220403405,2.13330418581266) circle (2.0pt);
\draw[color=uuuuuu] (4.697956027774401,2.499827742266318) node {$E$};
\draw [fill=uuuuuu] (3.968556783697015,2.1726396229121065) circle (2.0pt);
\draw[color=uuuuuu] (3.919307986933355,2.570613927797322) node {$S$};
\draw [fill=uuuuuu] (2.454233092884584,1.1020185321051048) circle (2.0pt);
\draw[color=uuuuuu] (2.1071816373396466,1.1407329800710362) node {$X$};
\draw [fill=uuuuuu] (5.805563901771507,1.9180603449254363) circle (2.0pt);
\draw[color=uuuuuu] (6.028736315757281,2.089267866186493) node {$Y$};
\draw [fill=uuuuuu] (4.565154539765928,-0.2774755660589251) circle (2.0pt);
\draw[color=uuuuuu] (4.669641553561999,-0.04847493684983516) node {$J$};
\draw [fill=uuuuuu] (4.540998234142051,0.535786723278248) circle (2.0pt);
\draw[color=uuuuuu] (4.6413270793495975,0.7726448153098141) node {$N$};
\end{scriptsize}
\end{tikzpicture}

\textbf{Solution.}
We first investigate the locus of $S$. 
Denote by $O$ the circumcenter and $H$ the orthocenter of triangle $ABC$
Denote also by $E$ the midpoint of $AO$, $F$ the midpoint of $AH$, and $N$ the midpoint of $OH$ (the nine-point-center). 
Obviously $S$ passes through the perpendicular bisector of $AD$, so this locus is a line. 
In the case there the circle passes through $H$, from the fact that $\angle ADH=90^{\circ}$ we know that $S=F$.
Similarly, if the circle passes through $O$, $S=E$ in this case with $\angle ADO=90^{\circ}$. 
Thus the locus is actually $EF$, i.e. parallel to $OH$. 

We proceed to prove that the circumcenter of triangle $XSY$ (namely $J$) lies on the perpendicular bisector of $PM$. 
That is, the line passing through the midpoint of $OH$ (a.k.a. nine-point-center) and perpendicular to $BC$ (or parallel to $AH$). 
We first show this in the special cases that $S$ is the midpoint of $AO$ or $AH$. 
In the first case ($E=S$), $X$ is the midpoint of $AB$ and $Y$ is the midpoint of $AC$ (imagine the homothety centered at $A$ with factor $\frac 12$ which brings $ABC$ to $AXY$ and point $O$ to the midpoint of $AO$). 
$J$ lies on the perpendicular bisector of $XY$
Notice that, with $XY\parallel BC$, this perpendicular bisector of $XY$ is also perpendicular to $BC$. 
Moreover, nine-point-circle passes through the midpoints of $AB$ and $AC$, so this perpendicular bisector passes through the nine-point center. 
Therefore the perpendicualr bisector of $XY$ is the perpendicular bisector of $PM$ itself, 
and with $SX=SY$, $S$ (the midpoint of $AO$, a.k.a. $E$ in this case) lies on this perpendicular bisector too. 
In the second case, $X$ and $Y$ are going to be the altitude from $C$ to $AB$, and $B$ to $AC$, respectively. 
Since the nine-point circle passes through the midpoint of $AH$ ($S$ in this case), $X$, $Y$, $P$, $M$, 
$J$, the circumcenter of $XSYPM$ is the nine-point center itself. 

Now let's do the general case. Observe that with $E$ (midpoint of $AO$) and $N$ (midpoint of $OH$) both equidistant from $PM$ the conclusion now becomes $J$ lies on $EN$. We invoke the following two lemmas:\\ 
$\bullet$ $\frac {SJ}{AS}=\frac {AE}{EN}$.\\
Proof: we have $AE=\frac 12 AO$ and $EN=AF=\frac 12 AH$, and $AS=SX=SY$, 
so we just have to prove that 
$\frac {SJ}{SX}=\frac {AO}{AH}$.
First, it is well notice that $SJ$ is the circumradius of $SXY$, so knowing that $\angle SXY=\angle XSY=90^{\circ}-\frac 12\angle XSY$
$= 90^{\circ}-\angle XAY=90^{\circ}-\angle BAC$ we have 
$SX=2SJ\sin\angle SXY=2SJ\cos\angle BAC$, yielding $\frac {SJ}{SX}=\frac {1}{2\cos\angle BAC}$.
Let $T$ be the reflection of $H$ in $M$, then $HBMT$ is a parallelogram with $\angle ABT=\angle ABC+\angle CBT$
$\angle ABC+\angle HCB=90^{\circ}$, and similarly $\angle ACT=90^{\circ}$. 
Therefore $A, O, T$ collinear and with $AH\parallel OM$ we have $\frac{OM}{AH}=\frac {TH}{MT}=\frac 12$. 
We also know that $\frac {OM}{AO}=\frac {OM}{BO}=\cos\angle BOM=\cos\angle BAC$
so $\frac {AO}{AH}=\frac {AO}{2OM}=\frac 1{2\cos\angle BAC}=\frac {SJ}{AS}$.\\ 
$\bullet$ $\angle (HA, AS)=\angle (SJ, AO)$. (This would also imply $\angle (AO, AS)=\angle (SJ, AH)$).\\
Proof: we use the well-known fact that the circumcenter and orthocenter of each triangle are the isogonal conjugates of each other. 
In particular, if $\ell$ is the perpendicular from $A$ to $XY$ then $AS$ and $\ell$ are the images of each other in the reflection of the internal angle bisector of $\angle AXY$. 
This gives $\angle (AB, AS)=\angle (\ell, AC)$. 
Same goes for the relation between $AH$ and $AO$, and therefore $\angle (AB, AH)=\angle (AO, AC)$. 
Moreover, $SJ\perp XY$ (since $SX=SY$ and $JX\perp JY$ we know that $SJ$ must be the perpendicular bisector of $XY$). 
Therefore $SJ\parallel \ell$. 
Now we have $\angle (SJ, AO)=\angle (\ell, AO)=\angle (\ell, AC)+\angle (AC, AO)=\angle (AB, AS)+\angle (AH, AB)=\angle (AH, AS)$. 

To complete the proof denote $J'$ by the intersection of $SJ$ and $EN$ and we shall prove that $J=J'$ by proving that $SJ=SJ'$. 
From the first lemma it suffices to prove that  $\frac {SJ'}{AS}=\frac {AE}{EN}$. 
Now $\frac {AS}{SE}=\frac{\sin\angle AES}{\sin\angle SAE}=\frac{\sin\angle AOH}{\sin\angle SAO}$ and 
$\frac{SJ'}{SE}=\frac{\sin\angle SEJ'}{\sin\angle EJ'S}=\frac{\sin\angle AFE}{\sin\angle EJ'S}=\frac{\sin\angle AHO}{\sin\angle EJ'S}$. 
Now, $\angle (EJ', J'S)=\angle (EN, SJ)=\angle (AH, SJ)=\angle (AO, AS)$ so the angles $\angle SEJ'$ and $\angle SAO$ are either equal or supplementary, hence $\sin\angle SEJ'=\sin\angle SAO$. 
Therefore, $\frac {SJ'}{AS}=\frac {SJ'}{SE}\div \frac {AS}{SE}= \frac{\sin\angle AHO}{\sin\angle EJ'S}\div\frac{\sin\angle AOH}{\sin\angle SAO}=\frac{\sin\angle AHO}{\sin\angle AOH}=\frac{AO}{AH}=\frac {EN}{AE}$, Q.E.D.

\item[\textbf{G6}]
Let $ABCD$ be a convex quadrilateral with $\angle ABC = \angle ADC < 90^{\circ}$. The internal angle bisectors of $\angle ABC$ and $\angle ADC$ meet $AC$ at $E$ and $F$ respectively, and meet each other at point $P$. Let $M$ be the midpoint of $AC$ and let $\omega$ be the circumcircle of triangle $BPD$. Segments $BM$ and $DM$ intersect $\omega$ again at $X$ and $Y$ respectively. Denote by $Q$ the intersection point of lines $XE$ and $YF$. Prove that $PQ \perp AC$.

\definecolor{uuuuuu}{rgb}{0.26666666666666666,0.26666666666666666,0.26666666666666666}
\begin{tikzpicture}[line cap=round,line join=round,>=triangle 45,x=1.0cm,y=1.0cm]
\clip(-0.06077115315761548,-3.6584705525177825) rectangle (17.29579728541572,4.521755615922858);
\draw [line width=1.2pt] (5.233712087730585,0.15993392341275056) circle (3.52235927554948cm);
\draw [line width=1.2pt] (8.344455105709537,0.24031487994967485) circle (3.52235927554948cm);
\draw [line width=1.2pt] (6.707454566010934,3.3591678901243993)-- (6.870712627429188,-2.958919086761974);
\draw [line width=1.2pt] (6.789083596720062,0.2001244016812127)-- (3.393315919220401,3.1632582164224963);
\draw [line width=1.2pt] (6.789083596720062,0.2001244016812127)-- (10.923409101776997,2.639483726074752);
\draw [line width=1.2pt] (4.823271172307094,0.14932821500387372)-- (5.119534491301556,1.6569665297266845);
\draw [line width=1.2pt] (8.754896021133032,0.25092058835854925)-- (3.393315919220401,3.1632582164224963);
\draw [line width=1.2pt] (10.923409101776997,2.639483726074752)-- (4.823271172307094,0.14932821500387372);
\draw [line width=1.2pt] (8.754896021133032,0.25092058835854925)-- (7.174752186710168,1.1092332982817523);
\draw [line width=1.2pt,dash pattern=on 4pt off 4pt] (1.712528154328142,0.06894725846694982)-- (11.865639039111985,0.33130154489547325);
\draw [line width=1.2pt] (5.119534491301556,1.6569665297266845)-- (11.865639039111985,0.33130154489547325);
\draw [line width=1.2pt] (1.712528154328142,0.06894725846694982)-- (8.580780113524357,1.2572718370740135);
\draw [shift={(7.3725916469831345,5.981256284249301)},line width=1.2pt,dash pattern=on 4pt off 4pt]  plot[domain=3.7577798782786243:5.528106782022175,variable=\t]({1.*4.876038189743517*cos(\t r)+0.*4.876038189743517*sin(\t r)},{0.*4.876038189743517*cos(\t r)+1.*4.876038189743517*sin(\t r)});
\begin{scriptsize}
\draw [fill=uuuuuu] (6.707454566010934,3.3591678901243993) circle (2.0pt);
\draw[color=uuuuuu] (6.685651535574803,3.6690198622016474) node {$A$};
\draw [fill=uuuuuu] (3.393315919220401,3.1632582164224963) circle (2.0pt);
\draw[color=uuuuuu] (3.199245179652681,3.4577225072972766) node {$B$};
\draw [fill=uuuuuu] (6.870712627429188,-2.958919086761974) circle (2.0pt);
\draw[color=uuuuuu] (6.97241223151645,-2.7151787895518416) node {$C$};
\draw[color=black] (2.6936407947029357,2.8464694448953467) node {$\omega_1$};
\draw[color=black] (7.70440663957802,3.993512228661931) node {$\omega_2$};
\draw [fill=uuuuuu] (10.923409101776997,2.639483726074752) circle (2.0pt);
\draw[color=uuuuuu] (11.122895988566595,2.6879964287170686) node {$D$};
\draw [fill=uuuuuu] (6.75976776182244,1.3346472122191515) circle (2.0pt);
\draw[color=uuuuuu] (6.896948890479175,1.5258609767430293) node {$E$};
\draw [fill=uuuuuu] (6.769863235751817,0.9439523711522477) circle (2.0pt);
\draw[color=uuuuuu] (6.610188194537528,1.0730809305193776) node {$F$};
\draw [fill=uuuuuu] (6.789083596720062,0.2001244016812127) circle (2.0pt);
\draw[color=uuuuuu] (6.564910189915163,0.0015014877900685064) node {$M$};
\draw [fill=uuuuuu] (8.754896021133032,0.25092058835854925) circle (2.0pt);
\draw[color=uuuuuu] (9.032561441834067,0.3109011860428972) node {$M_1$};
\draw [fill=uuuuuu] (4.823271172307094,0.14932821500387372) circle (2.0pt);
\draw[color=uuuuuu] (4.42929763856027,0.2618278681174477) node {$M_2$};
\draw [fill=uuuuuu] (5.119534491301556,1.6569665297266845) circle (2.0pt);
\draw[color=uuuuuu] (5.0405507009622,2.023919027589046) node {$X$};
\draw [fill=uuuuuu] (8.580780113524357,1.2572718370740135) circle (2.0pt);
\draw[color=uuuuuu] (8.692976407166327,1.480582972120664) node {$Y$};
\draw [fill=uuuuuu] (7.174752186710168,1.1092332982817523) circle (2.0pt);
\draw[color=uuuuuu] (7.168616918213366,1.420212299290844) node {$P$};
\draw [fill=uuuuuu] (1.712528154328142,0.06894725846694982) circle (2.0pt);
\draw[color=uuuuuu] (1.8484513750854517,0.3561791906652624) node {$N_1$};
\draw [fill=uuuuuu] (11.865639039111985,0.33130154489547325) circle (2.0pt);
\draw[color=uuuuuu] (12.205817078702718,0.6127545501919983) node {$N_2$};
\draw [fill=uuuuuu] (7.821783221564824,1.1259524464692119) circle (2.0pt);
\draw[color=uuuuuu] (7.9232503285861196,1.3447489582535685) node {$Q$};
\end{scriptsize}
\end{tikzpicture}

\textbf{Solution.} 
Let $\omega_1$ be the circumcircle of $ABC$ and $\omega_2$ the circumcircle of $ADC$, then these two circles are symmetric w.r.t. $AC$.
Also notice that $BP$ passes through $M_1$, the midpoint of arc $AC$ of $\omega_1$ not containing $B$, and $DP$ passes through $M_2$, the midpoint of arc $AC$ of $\omega_2$ not containing $D$.

We first start with a preliminary observation: $X$ lies on $\omega_2$ and $Y$ lies on $\omega_1$. W.L.O.G. for this section we assume that $AB\le AC$. Indeed, let $X'$ be on $BM$ satisfying $MX'\cdot MB=MA^2=MC^2$. Then $\angle X'AC=\angle MBA$ and $\angle X'CA=\angle MBC$. Thus $\angle ADC=\angle ABC=\angle MBA+\angle MBC=\angle X'AC+\angle X'CA=\pi - \angle AX'C$, so $X'$ lie on $\omega_2$. In addition, let $BM$ intersect $\omega_1$ again at $X''$, then $X'$ and $X''$ are symmetrical w.r.t. $AC$. Combining with the fact that $M_1$ and $M_2$ are also symmetrical w.r.t. $AC$ (being the midpoint of arc) we have $X'M_2=X''M_1$. Knowing that the two circles have the same radius further allows us to assert $\angle X'BP=\angle X''BM_1=\angle X'DM_2=\angle X'DP$, showing that $D, B, P, X'$ cyclic hence $X'=X$. Similarly, $Y$ lies on $\omega_1$.

Next, let $N_1$ be diametrically opposite $M_1$ w.r.t. $\omega_1$ and define similarly for $N_2$. We claim that $XE$ passes through $N_2$ by claiming that $XE$ is the internal angle bisector of $\angle AXC$. Indeed, by angle bisector theorem we have $\frac{AE}{EC}=\frac{AB}{BC}$. Invoking our $X''$ from the previous section (i.e. the other intersection of $BM$ and $\omega_1$) gives $AXCX''$ parallelogram. Now invoking a little bit more trigonometric bashing we have $1=\frac{AM}{CM}=\frac{AB}{BC}\cdot\frac{\sin\angle ABM}{\sin\angle CBM}=\frac{AB}{BC}\cdot\frac{AX''}{CX''}=\frac{AB}{BC}\cdot\frac{CX}{AX},$ so $\frac{AX}{CX}=\frac{AB}{BC}=\frac{AE}{EC}$, and the conclusion follows by the angle bisector theorem. Analogously, $YF$ passes through $N_1$.

Finally, considering triangle $PEF$, and letting the perpendicular from $P$ to reach $AC$ at $P_1$ we have (considering signed length) $\frac{EP_1}{FP_1}=\frac{\cot\angle FEP}{\cot\angle EFP}$. Similarly if letting perpendcular from $Q$ to reach AC at $Q_1$ we have $\frac{EQ_1}{FQ_1}=\frac{\cot\angle FEQ}{\cot\angle EFQ}$. Now $\cot\angle FEP=\cot\angle MEM_1=\frac{MM_1}{EM}$, $\cot\angle EFP=\cot\angle MFM_2=\frac{MM_2}{FM}$. Considering $MM_2=MM_1$ we have $\frac{\cot\angle FEP}{\cot\angle EFP}=\frac{FM}{EM}$. Analogously, $\cot\angle FEQ=\cot\angle FEQ=\cot\angle MEN_2=\frac{MN_2}{EM}$, and $\cot\angle EFQ=\cot\angle N_1FM=\frac{MN_1}{FM}$. Therefore we have $\frac{\cot\angle FEQ}{\cot\angle EFQ}=\frac{FM}{EM}$ since again it is not hard to verify that $MN_2=MN_1$. (For signed convention we can say that $ME<0$ if it's nearer to $A$ than $B$, and $>0$ otherwise). Therefore, $\frac{EP_1}{FP_1}=\frac{EQ_1}{FQ_1}$, so $P_1\equiv Q_1$ and the two perpendicular lines coincide.

\item[\textbf{G7}]
Let $I$ be the incentre of a non-equilateral triangle $ABC$, $I_A$ be the $A$-excentre, $I'_A$ be the reflection of $I_A$ in $BC$, and $l_A$ be the reflection of line $AI'_A$ in $AI$. Define points $I_B$, $I'_B$ and line $l_B$ analogously. Let $P$ be the intersection point of $l_A$ and $l_B$.

\begin{itemize}
\item [(a)] Prove that $P$ lies on line $OI$ where $O$ is the circumcentre of triangle $ABC$.
\item [(b)] Let one of the tangents from $P$ to the incircle of triangle $ABC$ meet the circumcircle at points $X$ and $Y$. Show that $\angle XIY = 120^{\circ}$.
\end{itemize}

\textbf{Solution.} 
\definecolor{uuuuuu}{rgb}{0.26666666666666666,0.26666666666666666,0.26666666666666666}
\begin{tikzpicture}[line cap=round,line join=round,>=triangle 45,x=0.5cm,y=0.5cm]
\clip(-9.784068425578555,-10.140965804211529) rectangle (28.496495861557182,7.900830581551528);
\draw [line width=1.2pt] (7.3936864077333855,0.543416871511203) circle (2.3435633554098465cm);
\draw [line width=1.2pt] (4.163371953821335,3.9396243640131448)-- (3.095957911065971,-1.3270607605957092);
\draw [line width=1.2pt] (11.575141965729667,-1.5742672928016477)-- (3.095957911065971,-1.3270607605957092);
\draw [line width=1.2pt] (4.163371953821335,3.9396243640131448)-- (11.575141965729667,-1.5742672928016477);
\draw [line width=1.2pt] (4.163371953821335,3.9396243640131448)-- (9.053160130826601,-8.833360183928688);
\draw [line width=1.2pt] (4.163371953821335,3.9396243640131448)-- (7.3936864077333855,0.543416871511203);
\draw [line width=1.2pt] (-3.9735037128951296,0.5816773980697412)-- (7.3936864077333855,0.543416871511203);
\draw [line width=1.2pt] (4.163371953821335,3.9396243640131448)-- (-3.9735037128951296,0.5816773980697412);
\draw [line width=1.2pt] (4.163371953821335,3.9396243640131448)-- (9.3618077011955,1.7532514797246423);
\draw [line width=1.2pt] (4.163371953821335,3.9396243640131448)-- (9.48035504203699,5.81942527058766);
\draw [line width=1.2pt] (4.163371953821335,3.9396243640131448)-- (4.047099015150206,-0.048537432406617);
\draw [line width=1.2pt] (4.099415080528914,1.7459036100830427) -- (3.8556856648354945,1.952818970749336);
\draw [line width=1.2pt] (4.099415080528914,1.7459036100830427) -- (4.354785304136048,1.9382679608571916);
\draw [line width=1.2pt] (9.48035504203699,5.81942527058766)-- (9.053160130826601,-8.833360183928688);
\draw [line width=1.2pt] (9.260937182474935,-1.706607312390734) -- (9.017207766781517,-1.4996919517244416);
\draw [line width=1.2pt] (9.260937182474935,-1.706607312390734) -- (9.51630740608207,-1.514242961616586);
\draw [line width=1.2pt] (4.163371953821335,3.9396243640131448)-- (9.421081371616246,3.7863383751561503);
\begin{scriptsize}
\draw [fill=uuuuuu] (4.163371953821335,3.9396243640131448) circle (2.0pt);
\draw[color=uuuuuu] (4.396384084264771,4.488867242915525) node {$A$};
\draw [fill=uuuuuu] (3.095957911065971,-1.3270607605957092) circle (2.0pt);
\draw[color=uuuuuu] (3.3311857736662107,-0.7705494156648485) node {$B$};
\draw [fill=uuuuuu] (11.575141965729667,-1.5742672928016477) circle (2.0pt);
\draw[color=uuuuuu] (11.819484811248481,-1.0368489933144878) node {$C$};
\draw [fill=uuuuuu] (7.3936864077333855,0.543416871511203) circle (2.0pt);
\draw[color=uuuuuu] (7.725128804885269,0.727385708614372) node {$O$};
\draw [fill=uuuuuu] (9.053160130826601,-8.833360183928688) circle (2.0pt);
\draw[color=uuuuuu] (9.60586957203585,-8.376731102282667) node {$I_A$};
\draw [fill=uuuuuu] (5.461026912989256,0.5499219593246881) circle (2.0pt);
\draw[color=uuuuuu] (5.694594525306765,1.093547627882626) node {$I$};
\draw [fill=uuuuuu] (9.48035504203699,5.81942527058766) circle (2.0pt);
\draw[color=uuuuuu] (9.905456596891696,6.436182904478511) node {$I_A'$};
\draw [fill=uuuuuu] (-3.9735037128951296,0.5816773980697412) circle (2.0pt);
\draw[color=uuuuuu] (-3.6425344160337345,1.210053693104343) node {$P_1=P$};
\draw [fill=uuuuuu] (9.3618077011955,1.7532514797246423) circle (2.0pt);
\draw[color=uuuuuu] (9.87216914968549,2.3751143453215144) node {$I_A''$};
\draw [fill=uuuuuu] (7.257093521907928,-4.141719112301999) circle (2.0pt);
\draw[color=uuuuuu] (7.492116674441834,-3.5999824281922645) node {$M$};
\draw [fill=uuuuuu] (4.047099015150206,-0.048537432406617) circle (2.0pt);
\draw[color=uuuuuu] (4.2965217426461555,-0.3711000491903897) node {$H$};
\draw [fill=uuuuuu] (9.266757586431796,-1.5069674566705147) circle (2.0pt);
\draw[color=uuuuuu] (9.489363506814133,-0.9702740989020779) node {$T$};
\draw [fill=uuuuuu] (9.421081371616246,3.7863383751561503) circle (2.0pt);
\draw[color=uuuuuu] (9.922100320494797,3.723255957172813) node {$U$};
\end{scriptsize}
\end{tikzpicture}

(a) We first prove the following lemma: let $I''_A$ the point on $I_AI_A'$ satisfying $AI''_A=AI'_A$, and $I''_A\neq I'_A$ unless $AI'_A$ and $I_AI_A'$ are perpendicular to each other. Then triangles $AOI$ and $I_AAI''_A$ are similar. 
Indeed, let $H$ be the orthocenter of triangle $ABC$, then $AH\parallel I_AI_A''$. 
Also, $\angle OAI=\angle IAH=\angle AI_AI''_A$, so it suffices to prove that $\frac{AI}{AO}=\frac{I_AI''_A}{I_AA}$. 
To do so, define $I_C$ has how $I_A$ and $I_B$ are defined. 
We first notice that $I$ is the orthocenter of $I_AI_BI_C$, so $I_AA$ is an altitude of triangle $I_AI_BI_C$. 
In addition, we know that triangles $I_AI_BI_C$ are $I_ACB$ are similar with a similitude of $\frac {I_AB}{I_AI_B}=\cos\angle BI_AC$.
Denote by $T$ the altitude from $I_A$ to $BC$, then by the similarities of the triangles mentioned above we have 
$\frac {I_AI'_A}{I_AA}=\frac{2I_AT}{I_AA}=2\cos\angle BI_AC$. 
In addition, if we let $U$ to be the perpendicular from $A$ to $I_AI_A''$ we have $I'_A$ and $I''_A$ symmetric to each other w.r.t. $U$. 
Therefore, $I_AI_A''+I_AI_A'=2I_AU=2I_AA\cos\angle AI_AI_A'=2I_AA\cos\angle AOI$. 
So $\frac{I_AI''_A}{I_AA}=\frac{2I_AA\cos\angle AOI-I_AI_A'}{I_AA}$
$=\frac{2I_AA\cos\angle AOI-2I_AA\cos\angle BI_AC}{I_AA}$
$=2(\cos\angle AOI-\cos\angle BI_AC)$. 
On the other hand, denoting $M$ as the other intersection of $AI$ and $BC$ we have 
$MB=MC=MI=2AO\cos BAM$ (because $AO$ is the circumradius of triangle $ABC$), 
and $AM=AO\sin\angle ABM$, 
so $\frac{AI}{AO}=\frac{AM-MI}{AO}=\frac{2AO\cos\angle ABM-2AO\cos BAM}{AO}$
$=2(\cos\angle ABM-\cos\angle BAM)$
$=2(\cos\angle AOI-\cos\angle BI_AC)$
(the equality $\angle ABM=\angle AOI$ and $\angle BAM=\angle BI_AC$ can be established via angle chasing). 
This establishes the desired equality. 

Now denote $P_1$ on $OI$ such that $I$ lies between $O$ and $P_1$ and $OI\cdot OP_1=AO$. 
Then, triangles $AOI$ anf $AP_1O$ are similar. 
Therefore, by the fact that $\triangle AOI\sim \triangle I_AAI''_A$ $\angle IAP_1=\angle OAP_1-\angle OAI$
$=\angle AIO-\angle OAI$
$=\angle I_AI''_AA-\angle OAI$
$=180^{\circ}-\angle I_AI'_AA-\angle HAI$
$=180^{\circ}-\angle I_AI'_AA-\angle AI_AI_A'$
$=\angle I_A'AI_A$. 
This means that $l_A$ passes through $P_1$ and similarly, $l_B$ passes through $P_1$. Hence $P_1=P$ and lies on $OI$. 

(b) 

\definecolor{uuuuuu}{rgb}{0.26666666666666666,0.26666666666666666,0.26666666666666666}
\begin{tikzpicture}[line cap=round,line join=round,>=triangle 45,x=1.0cm,y=1.0cm]
\clip(-0.8138091660405693,-3.0651104432757283) rectangle (16.466431254695724,5.079141998497374);
\draw [line width=1.2pt] (6.582731062360519,1.069813550044211) circle (3.637122258760492cm);
\draw [line width=1.2pt] (4.515703038946392,0.407310575525049) circle (1.170862934814784cm);
\draw [line width=1.2pt] (0.779076137488415,-0.7903153922135896)-- (9.745500983360207,-0.7261666146543425);
\draw [line width=1.2pt] (6.582731062360519,1.069813550044211)-- (6.608751575538458,-2.567215630216099);
\draw [line width=1.2pt] (3.44598165453877,-0.7712354655175488)-- (6.608751575538458,-2.567215630216099);
\draw [line width=1.2pt] (5.064465903538294,-1.603892691711937) -- (4.990267326538933,-1.734558404021711);
\draw [line width=1.2pt] (3.44598165453877,-0.7712354655175488)-- (6.582731062360519,1.069813550044211);
\draw [line width=1.2pt] (4.976326089567558,0.21408439664382672) -- (5.052386627331731,0.08449368788283629);
\draw [line width=1.2pt] (6.582731062360519,1.069813550044211)-- (9.745500983360207,-0.7261666146543425);
\draw [line width=1.2pt] (8.201215311360043,0.23715632384982185) -- (8.12701673436068,0.10649061154004773);
\draw [line width=1.2pt] (6.608751575538458,-2.567215630216099)-- (9.745500983360207,-0.7261666146543425);
\draw [line width=1.2pt] (8.139096010567245,-1.5818957680547254) -- (8.215156548331418,-1.7114864768157159);
\draw [shift={(6.6087515755384585,-2.5672156302160922)},line width=1.2pt]  plot[domain=0.5307529852439434:2.6251480876371422,variable=\t]({1.*3.637122258760486*cos(\t r)+0.*3.637122258760486*sin(\t r)},{0.*3.637122258760486*cos(\t r)+1.*3.637122258760486*sin(\t r)});
\draw [line width=1.2pt] (0.779076137488415,-0.7903153922135896)-- (6.582731062360519,1.069813550044211);
\begin{scriptsize}
\draw [fill=uuuuuu] (6.582731062360519,1.069813550044211) circle (2.0pt);
\draw[color=uuuuuu] (6.534049586776862,1.4052126220886587) node {$O$};
\draw [fill=uuuuuu] (4.515703038946392,0.407310575525049) circle (2.0pt);
\draw[color=uuuuuu] (4.445394440270476,0.6839504132231442) node {$I$};
\draw [fill=uuuuuu] (0.779076137488415,-0.7903153922135896) circle (2.0pt);
\draw[color=uuuuuu] (0.5085048835462079,-0.6984688204357588) node {$P$};
\draw [fill=uuuuuu] (3.44598165453877,-0.7712354655175488) circle (2.0pt);
\draw[color=uuuuuu] (3.333448534936141,-0.41296919609315924) node {$X$};
\draw [fill=uuuuuu] (9.745500983360207,-0.7261666146543425) circle (2.0pt);
\draw[color=uuuuuu] (9.99009767092412,-0.6834425244177271) node {$Y$};
\draw [fill=uuuuuu] (6.5957413189494885,-0.7487010400859457) circle (2.0pt);
\draw[color=uuuuuu] (6.729391435011273,-0.4430217881292224) node {$N$};
\draw [fill=uuuuuu] (6.608751575538458,-2.567215630216099) circle (2.0pt);
\draw[color=uuuuuu] (6.609181066867021,-2.772097670924113) node {$S$};
\draw [fill=uuuuuu] (4.524079566392375,-0.7635223954452082) circle (2.0pt);
\draw[color=uuuuuu] (4.625709992486855,-0.5181532682193801) node {$T$};
\end{scriptsize}
\end{tikzpicture}

We first establish a relation between the length $OI$ and the inradius $r$. 
Let the circumradius be $R=AO=BO=CO$. Using fresh new label than (a), we denote $M$ as the second intersection of $AI$ and the circumcircle of $ABC$. 
This $M$ is also the midpoint of arc $BC$ not containing $A$. 
Now, $MI=MB=MC=2R\sin \angle BAI=2R\sin\frac{\angle A}{2}$, 
$AM=AB\frac{\sin\angle ABI}{\angle AIB}$
$=2R\sin\angle C \frac{\sin\frac 12 B}{\sin (90^{\circ}+\frac 12 C)}$
$=2R(2\sin\angle \frac{\angle C}{2} \cos\angle \frac{\angle C}{2}) \frac{\sin\frac 12 \angle B}{\cos\frac 12 \angle C}$
$=4R\sin\frac 12\angle B\sin\frac 12\angle C$
so $MI\cdot AI=8R^2\sin\frac 12 \angle A\sin\frac 12\angle B\sin\frac 12\angle C$. 
Meanwhile, letting $D$ be the point of tangency of the incircle to $BC$ we have 
$r=ID=IB\sin\angle IBC=BC\frac{\sin\angle ICB}{\sin\angle CIB}\sin\frac 12\angle B$
$=2R\sin A\frac{\sin\frac 12 C}{\sin (90^{\circ}+\frac 12 A)}\sin\frac 12\angle B$
$=2R(2\sin\angle \frac{\angle A}{2} \cos\angle \frac{\angle A}{2})\frac{\sin\frac 12 C}{\cos\frac 12 A)}\sin\frac 12\angle B$
$=4R\sin\frac 12 \angle A\sin\frac 12\angle B\sin\frac 12\angle C$. 
Considering the fact that $-MI\cdot AI$ is the power of point of $I$ w.r.t. the circumcircle we have 
$OI^2=R^2-MI\cdot AI$=$R^2-8R^2\sin\frac 12 \angle A\sin\frac 12\angle B\sin\frac 12\angle C$, 
or $\frac{OI^2}{R^2}=1-8\sin\frac 12 \angle A\sin\frac 12\angle B\sin\frac 12\angle C$ 
while $\frac {r}{R}=4\sin\frac 12 \angle A\sin\frac 12\angle B\sin\frac 12\angle C$ 
so $\frac{OI^2}{R^2}=1-\frac {2r}{R}$, 
or $\frac{r}{R}=\frac 12-\frac{OI^2}{2R^2}$, 
or $r=\frac 12R(1-\frac{OI^2}{R^2})$ 
(this is actually a well-known identity, the purpose of including the proof is to show the power of trigonometry in solving problems). 

Now, let $T$ be the tangency point of the incircle to $XY$, and $N$ be the midpoint of $XY$. 
Keeping in mind that $OP\cdot OI=R^2$, 
we now have $\frac{PI}{OP}$
$=1-\frac{OI}{OP}$
$=1-\frac{OI}{R^2\div OI}$
$=1-\frac{OI^2}{R^2}$. 
Therefore $ON=IT\frac{PO}{PI}$
$=r\frac{1}{1-\frac{OI^2}{R^2}}$
$=\frac 12 R(1-\frac{OI^2}{R^2})\frac{1}{1-\frac{OI^2}{R^2}}$
$=\frac 12 R$. 
Moreover, letting $S$ be the midpoint of arc $XY$ lying on the opposite side as $I$ w.r.t. $XY$ we have 
$O, N, S$ collinear, $ON=NS$, and $ON\perp XY$. 
Therefore, $OX=OY=OS=XS=YS$, 
yielding $OXS$ and $OYS$ both equilateral and 
$\angle XOY=60^{\circ}+60^{\circ}=120^{\circ}$. 
Additionally, $PI\cdot PO=PO^2-(IO\cdot OP)=PO^2-R^2$, 
which is the power of point of $P$ w.r.t. the circumcircle. 
This, in turn, is equal to the value $PX\cdot PY$, so $IOYX$ is cyclic. 
Thus $\angle XIY=\angle XOY=120^{\circ}$. 

\end{itemize}

\section{Number Theory}
\begin{itemize}
\item[\textbf{N1}]
For any positive integer $k$, denote the sum of digits of $k$ in its decimal representation by $S(k)$. Find all polynomials $P(x)$ with integer coefficients such that for any positive integer $n \geq 2016$, the integer $P(n)$ is positive and $$S(P(n)) = P(S(n)).$$

\textbf{Thoughts.} 
The first thing to try is $c\cdot 10^k$ for $c\le 9$, of course, 
because $S(c\cdot 10^k)=c$ and it's easier to manipulate. 
(Well I first tried $n=10^k$ that gives $S(P(10^k))=P(1)$, a constant). 
The next thing is that, if we space $k$ big enough, the numbers are likely to be in the form 
$(a_kc^k)(0\cdots 0)(a_{k-1}c^{k-1}(0\cdots 0)\cdots (0\cdots 0)a_0c^0$, 
provided $a_i\ge 0$ for all $i$. 
But we cannot simply make that assumption! 
Fortunately, the fact that $a_i< 0$ will cause tons of trailing 9's, which will be good for a contradiction. 
Having that in mind, we know that $P(c)$ is the sum of $a_kc^k, \cdots , a_0c^0$ and the sum of digits (as shown above) is 
$S(a_kc^k)+\cdots +S(a_0c^0)$. 
With the fact that $S(i)\le i$ with equality iff $i\le 9$, it's no longer difficult to complete the solution. 

\textbf{Solution.} 
The answer is the constant polynomial $P(x)=c$ where $c\in\{1,2,\cdots ,9\}$, 
or the identity polynomial $P(x)=x$. In the first case we have $S(P(n))=S(c)=c=P(\text{anything})=P(S(n))$, 
and in the second case $S(P(n))=S(n)=P(S(n))$. 

Now let $P(x)=\displaystyle\sum_{i=0}^k a_ix^i$. 
The first thing to do is to prove that $a_i\ge 0$, $\forall i\ge 0$. 
Indeed, let $n=c\cdot 10^m$ ($1\le c\le 9$) then we have 
$P(c)=P(S(n))=S(P(n))=S(P(c\cdot 10^m))$. 
Lt $d$ be such that 
$10^{d}>max\{|a_i(9^i)| : i\in [0, k]\}$
For $m>d$ satisfying we have 
$P(c\cdot 10^m)=\displaystyle\sum_{i=0}^k a_i (c^i)(10^{mi})$. 
Let $a_j<0$ for some $j$. 
Now notice that 
$\displaystyle\sum_{i=0}^{j-1} a_i (c^i)(10^{mi})$
$<\displaystyle\sum_{i=0}^{j-1} 10^{d}(10^{mi})$
$<\displaystyle\sum_{i=0}^{mj-m+d} 10^i$
$<10^{mj-m+d+1}$
$\le 10^{mj}$
so
$P(c\cdot 10^m)=\displaystyle\sum_{i=0}^k a_i (c^i)(10^{mi})$
$=\displaystyle\sum_{i=0}^{j-1} a_i (c^i)(10^{mi})$
$+a_j(c^j)(10^{mi})$
$+\displaystyle\sum_{i=j+1}^{k} a_i (c^i)(10^{mi})$
$<10^{mj}$
$+a_j(c^j)(10^{mj})$
$+\displaystyle\sum_{i=j+1}^{k} a_i (c^i)(10^{mi})$
$\le\displaystyle\sum_{i=j+1}^{k} a_i (c^i)(10^{mi})$
(the first inequality is due to our choice of $m$). 
As $P(x)>0$ for all $x\ge 2016$, the leading coefficient is positive so we can choose $j$ such that there exists an $l\ge 1$ satisfying 
$c_{j+l}>0$ and $c_{j+1}, c_{j+2}, \cdots c_{j+l-1}=0$. 
In a similar way we can also deduce that 
$P(c\cdot 10^m)$
$>\displaystyle\sum_{i=j+l}^{k} a_i (c^i)(10^{mi})$
$-10^{m(j+l)-m+d+1}$
Combining the inequalities and by assuming that $c_{j+1}, c_{j+2}, \cdots c_{j+l-1}=0$, $c_j<0$ and $c_{j+l}>0$ we have 
\[
\displaystyle\sum_{i=j+l}^{k} a_i (c^i)(10^{mi})
-10^{m(j+l)-m+d+1}
<P(c\cdot 10^m)
<\displaystyle\sum_{i=j+1}^{k} a_i (c^i)(10^{mi})
=\displaystyle\sum_{i=j+l}^{k} a_i (c^i)(10^{mi})\]
This means that there will be at least $m-d$ consecutive 9's as digit, meaning that $S(P(c\cdot 10^m))$ is at least $9(m-d)$. 
It follows that $P(S(c\cdot 10^m))=P(c)\ge 9(m-d)$ for all sufficiently large $m$. 
However, this is contradicted by the fact that $\lim_{m\to\infty}9(m-d)\to\infty$. 
Hence $c_i\ge 0$ for all $i$. 

Since $a_i (c^i)(10^{ni}) < 10^{(n+1)i}$ (because $a_i(c^i)<10^n$ by our choice of $n$), 
the number $P(c\cdot 10^n)$ are in the form of $(a_kc^k)(0\cdots 0) (a_{k-1}c^{k-1})(0\cdots 0)\cdots(0\cdots 0)(a_0c^0)$ when laid in decimal form. 
Therefore 
$S(P(c\cdot 10^n))$
$=\displaystyle\sum_{i=0}^k S(a_i (c^i))$, 
and 
$P(S(c\cdot 10^n))=P(c)=\displaystyle\sum_{i=0}^k a_i (c^i)$. 
Knowing that $S(x)\le x$ with equality holds if and only if $0\le x\le 9$ 
(indeed, if $k=\displaystyle\sum_{i=0}^k b_i(10^i)$ then $S(k)=\displaystyle\sum_{i=0}^k b_i$, 
so $k-S(k)=\displaystyle\sum_{i=0}^k b_i(10^i -1)\ge 0$, 
with equality holds iff $b_i=0$ for $i\ge 1$, )
we have $a_i(c^i)\le 9$ for all $c\in \{0,1,\cdots 9\}$. 
This means $k\le 1$ (if we assume that $a_k > 0$). 
If $k=0$ then we get $a_0\le 9$, yielding the constant solution. 
If $k=1$, then $9a_1\le 9$ (when $c=9$) and $a_1=1$, yielding $P(x)=x+c$ for some constant $c$ 
(and since $c=a_0$ we have $c=a_0\le 9$ too). 
This entails $S(P(n))=S(n+c)$ and $P(S(n))=S(n)+c$ for all $n\ge 2016$, 
and letting $n=10^d-1$ we have $S(n)=9d$, and for $c\ge 1$, $S(n+c)=S(10^k-1+c)=c$, which doesn't hold for $d=5$. 
Therefore $c=0$ and we get the identity polynomial. 

\item[\textbf{N2}]
 Let $\tau(n)$ be the number of positive divisors of $n$. Let $\tau_1(n)$ be the number of positive divisors of $n$ which have remainders $1$ when divided by $3$. Find all positive integral values of the fraction $\frac{\tau(10n)}{\tau_1(10n)}$.

\textbf{Solution.} The answer is 2 and all composite numbers. 
Let $m=10n$, with $m=3^y\cdot\displaystyle\prod_{i=1}^{k}p_i^{a_i}\cdot\displaystyle\prod_{i=1}^{l}q_i^{b_i}$ with $p_i\equiv 1\pmod{3}$ and $q_i\equiv 2\pmod{3}$. 
Notice that $\tau(m)=(y+1)\cdot\displaystyle\prod_{i=1}^{k}(a_i+1)\cdot\displaystyle\prod_{i=1}^{l}(b_i+1)$. 

Now we want to investigate all the divisors that is congruent to 1 mod 3, 
observe that such divisors fulfill 
$\displaystyle\prod_{i=1}^{k}p_i^{c_i}\cdot\displaystyle\prod_{i=1}^{l}q_i^{d_i}$ 
with $c_i\le a_i$, $d_i\le b_i$ and $\displaystyle\sum_{i=1}^{l}d_i$ even. 
We proceed with the following claim: 
the number of combinations $(d_1, d_2, \cdots , d_l)$ satisfying $\displaystyle\sum_{i=1}^{l}d_i$ even and $d_i\le b_i$ is 
$\lfloor\frac{\prod_{i=1}^{l}(b_i+1)}{2} \rfloor$. 
\begin{itemize}
\item[Case 1.] $b_i$ odd for some $i$, and w.l.o.g. let this $i$ be $l$. 
Now, let $x$ be the number of combinations $(d_1, d_2, \cdots , d_{l-1})$ ($d_i\le b_i$) satisfying $\displaystyle\sum_{i=1}^{l-1}d_i$ even, and $z$ be the number of combinations with corresponding odd sums. 
Considering $d_i=\{0, 2, \cdots, b_l-1\}$ and $d_i=\{1,3,\dots , b_l\}$ we have: 
the number of combinations $(d_1, d_2, \cdots , d_{l-1})$ ($d_i\le b_i$) satisfying $\displaystyle\sum_{i=1}^{l}d_i$ even
is $x+z+x+z+\cdots + x+z=(x+z)\cdot\frac{b_i+1}{2}$, 
and similarly $z+x++\cdots +z+x=(x+z)\cdot\frac{b_i+1}{2}$ for odd-sum combinations. 
Therefore there is equally many odd and even sum combinations, and we are done.

\item[Case 2.] Now let $b_i$ even for all $i$. Let $O$ be number of combinations with $\displaystyle\sum_{i=1}^{l}d_i$ odd and $E$ be combinations with $\displaystyle\sum_{i=1}^{l}d_i$ even. The claim is $E-O=1$. 
We induct on $l$. 
Base case $l=0$ yield 1 combination for even sum and 0 combination for odd sum, vacuously. 
Now let $l=k$ for some $k$ and we have $O'$ as the number of combinations $(d_1, d_2, \cdots , d_k)$ with $\displaystyle\sum_{i=1}^{k}d_i$ odd, and $E'$ as the number of combinations with $\displaystyle\sum_{i=1}^{l}d_i$ even. 
Now that $b_{k+1}$ is even, using the logic above the number of even combination is 
$E'+O'+E'+O'+\cdots +E'=E'(\frac{b_{k+1}}{2}+1)+O'(\frac{b_{k+1}}{2})$, 
and similarly the number of combinations yielding odd sum is 
$O'(\frac{b_{k+1}}{2}+1)+E'(\frac{b_{k+1}}{2})$.
This yields $E-O=E'-O'$ and by induction hypothesis this number is 1, so we are done. 
\end{itemize}

Summing above, $\tau_1 (m)=\displaystyle\prod_{i=1}^{k}(a_i+1)\lfloor\frac{\prod_{i=1}^{l}(b_i+1)}{2} \rfloor$, 
so the ratio now becomes 
$(y+1)\dfrac{\displaystyle\prod_{i=1}^{l}(b_i+1)}{\lfloor\frac{\prod_{i=1}^{l}(b_i+1)}{2} \rfloor}$. 
Equivalently, $2(y+1)$ when $b_i$ odd for some $b_i$, or $(y+1)\frac{2k+1}{k}$ otherwise (where $2k+1=\displaystyle\prod_{i=1}^{l}(b_i+1)$ here). 
The first case yields that the ratio must be even; in the second case, we have $\gcd(2k+1, k)=1$ so $k|y+1$. 
In other words, the ratio must be divisible by $2k+1$. 
Notice, also, that $l\ge 2$ ($m=10n$ contains prime factors 2 and 5) so $2k+1=\displaystyle\prod_{i=1}^{l}(b_i+1)$ must be composite. 
So our integer ratio cannot be an odd prime. 

It remains to show that any even or composite numbers work. 
For even numbers $2k$, simply take $10\cdot 3^{k-1}$ and by our proof the ratio is $2k$. 
For odd composite number $xz$ with $x, z\ge 3$, take $m=2^{x-1}5^{z-1}$. 

\item[\textbf{N3/IMO 4}]
A set of postive integers is called fragrant if it contains at least two elements and each of its elements has a prime factor in common with at least one of the other elements. Let $P(n)=n^2+n+1$. What is the least possible positive integer value of $b$ such that there exists a non-negative integer $a$ for which the set $$\{P(a+1),P(a+2),\ldots,P(a+b)\}$$is fragrant?

\textbf{Thoughts.} This is  problem requiring no more than number experimentation, 
and it's immediate to see why $P(n)$ and $P(n+1)$ are relatively prime (otherwise the answer is 2 and it's too trivial to be on the IMO), which directly gives away the fact that $b=3$ doesn't work either. 
How about $P(n)$ and $P(n+2)$? The investigation of the first case also shows that $b=4$ fails, and further checking on $P(n)$ and $P(n+3)$ shows how $b=5$ fails too. 
Finally, for $b=6$ the relation between $\gcd(P(n), P(n+2))$, $\gcd(P(n), P(n+3))$ and $\gcd(P(n), P(n+4))$ will be good to construct an example (which cannot be determined by brute force since it's rather big!!!)
Ps: basic combinatorics skill needed to construct an example. 

\textbf{Solution.} The answer is $b=6.$ 
Observe that this solution works because the set $\{P(197), P(198), P(199), P(200), P(201), P(202)\}$ has 
$P(199)\equiv P(202)\equiv P(1)=3\equiv 0\pmod{3}$, 
$P(198)\equiv P(2)=7\equiv 0\equiv 21=P(4)\equiv P(200)\pmod {7}$, 
$P(197)\equiv P(7)=57\equiv 0\equiv 133=P(11)\equiv P(201)\pmod{19}$. 

First, notice that $P(n)-P(n-1)=n^2+n+1-(n^2-n-1)=2n$, 
and knowing that $n^2+n+1\equiv n+n+1=2n+1\equiv 1\pmod{2}$, 
we know that if $p|P(n)$ and $p|2n$ then $p|n$ (since $P(n)$ is relatively prime to 2), 
and consequently $p|n^2+n$ and $p|1$, showing that $P(n)$ and $P(n-1)$ are relatively prime. 
This means, $b=2$ fails, and $b=3$ fails too sine $P(a+1)$ and $P(a+3)$ are both relatively prime to $P(a+2)$. 
(We will use profusely the fact that $P(a)$ and $P(a+1)$ cannot have any common prime factor throughout the solution). 

Now, for $b=4$ and $b=5$ our strategy is to determine an upper bound for $\gcd(P(n), P(n+c))$ for $c=2, 3$. 
Observe that $P(n+c)-P(n)=2cn+c^2+c=c(2n+c+1)$. 
For $c=2$ this is the same as $2(2n+3)$. 
If $p|P(n+2)$ and $p|P(n)$ then $p|2(2n+3)$, and therefore $p|2n+3$ with $P$ being odd at all times. 
This entails $2n\equiv -3\pmod{p}$, 
and $0\equiv 4P(n)=4n^+4n+1=(2n)^2+2(2n)+1\equiv (-3)^2-3+1=7\pmod{7}$. 
Hence $p=7$ and $n\equiv 2\pmod{7}$. 
Now for $b=4$, knowing that $P(a+2)$ is relatively prime with $P(a+1)$ and $P(a+3)$ it must have a common prime factor with $P(a+4)$, 
and by the previous step this prime factor has to be 7. 
Similarly $P(a+1)$ and $P(a+3)$ must both be divisible by 7. 
This means $P(a+1), P(a+2), P(a+3), P(a+4)$ are all divisible by 7 for some $a$, contradicting that any two neighbouring elements are coprime. 

Finally for $b=5$ we investigate $c=3$ as in the previous paragraph. 
Now $3(2n+3+1)=3(2n+4)=3(2)(n+2)$. 
If a prime $p$ satisfies $p|P(n)$ and $p|P(n+3)$ simultaneously then either $p=3$ or $p|n+2$ 
(again $p$ must be relatively prime to 2 so this can be easily factored out). 
In the second case we have $n\equiv =2\pmod{p}$, 
so $P(n)\equiv P(-2)=4-2+1=3\equiv\pmod{p}$, forcing $p=3$ (no choice!) 
Thus viewing the set $\{P(a+1), \cdots , P(a+5)\}$ 
we know that $P(a+3)$ must have a common factor with $P(a+1)$ or $P(a+5)$, and by previous paragraph this common factor has to be 7. 
Thus neither of $P(a+2)$ nor $P(a+4)$ can be divisible by 7, and they cannot have common prime factor (again by previous paragraph). 
This entails $P(a+1)$ and $P(a+4)$ must have common factor, and by what we established earlier this factor must be 3. 
Similarly, $P(a+2)$ and $P(a+5)$ must both be divisible by 3. 
However, $P(a+1)$ and $P(a+2)$ are both divisible by 3, contradiction. 

\item[\textbf{N4}]
Let $n, m, k$ and $l$ be positive integers with $n \neq 1$ such that $n^k + mn^l + 1$ divides $n^{k+l} - 1$. Prove that
\begin{itemize}
\item[$\bullet$] $m = 1$ and $l = 2k$; or
\item[$\bullet$] $l|k$ and $m = \frac{n^{k-l}-1}{n^l-1}$.
\end{itemize}

\textbf{Solution.} We split our solution into two cases: 
\begin{itemize}
\item[Case 1.] $l\le k$. 
Now from $n^k + mn^l + 1|n^{k+l} - 1$, and from the fact that $(n^l-1)(n^k + mn^l + 1)$
$=n^{k+l}+mn^{2l}+n^l-n^k-mn^l-1$ we have 
$(n^{k+l}+mn^{2l}+n^l-n^k-mn^l-1)-(n^{k+l} - 1)$
$=mn^{2l}+n^l-n^k-mn^l$
$=n^l(mn^{l}+1-n^{k-l}-m)$
$=n^l(m(n^l-1)-(n^{k-l}-1))$
is divisible by $n^k + mn^l + 1$. 
Knowing that $\gcd(n, n^k + mn^l + 1)=\gcd(n, 1)=1$ we have 
$\gcd(n^l, n^k + mn^l + 1)=1$ 
so $m(n^l-1)-(n^{k-l}-1)$ is itself divisible by $n^k + mn^l + 1$. 
Now, $m(n^l-1)<mn^l<n^k + mn^l + 1$ and $n^{k-l}-1\le n^k-1<n^k + mn^l + 1$, meaning that 
$0<\frac{m(n^l-1)}{n^k + mn^l + 1}, \frac{(n^{k-l}-1)}{n^k + mn^l + 1}<1$. 
Therefore $|\frac{m(n^l-1)-(n^{k-l}-1)}{n^k + mn^l + 1}|<1$, and therefore has to be 0. 
We thus have 
$m(n^l-1)=(n^{k-l}-1)$ and since $n>1$, 
$m=\frac{n^{k-l}-1}{n^l-1}$. 
Let $k-l=cl+d$ with $0\le d<l$, then $n^{k-l}=n^{cl}\cdot n^d\equiv(n^l)^c\cdot n^d\equiv 1\cdot n^d=n^d\pmod{n^l-1}$, 
and from $n^d<n^l$ we have $n^d\not\equiv 1\pmod{n^l-1}$ unless $d=0$. 
Therefore $l|k-l$, or $l|k$. 

\item[Case 2.] $l\ge k$. 
Similar to above we have 
$(n^k-1)(n^k + mn^l + 1)-(n^{k+l}-1)$
$=n^{2k}+mn^{k+l}+n^k-n^k-mn^l-1-(n^{k+l}-1)$
$=n^{2k}+mn^{k+l}-mn^l-n^{k+l}$
$=n^k(n^k-mn^{l-k}+(m-1)n^l)$ 
is divisible by $n^k + mn^l + 1$. 
Again by the logic above, 
$\gcd (n^k, n^k + mn^l + 1)=1$, 
which very well means that $n^k + mn^l + 1|n^k-mn^{l-k}+(m-1)n^l$. 
Again we have $n^k+(m-1)n^l<n^k+mn^l+1$ and $mn^{l-k}<n^k+mn^l+1$ so by the logic above, again, 
$n^k-mn^{l-k}+(m-1)n^l=0$. 
Rearranging the terms give: 
$m=\frac{n^l-n^k}{n^l-n^{l-k}}$. 
Now, if $m\ge 2$, then we have 
$n^l-n^k\ge 2n^l-2n^{l-k}$, or 
$2n^{l-k}\ge n^l+n^k > n^l=n^k(n^{l-k})$, 
or $2>n^k$, forcing $n=1$ (contradiction since $k\ge 1$). 
Thus $m=1$ ($m$ must be positive) and we have $k=l-k$, or $l=2k$. 
\end{itemize}

\item[\textbf{N5}]
Let $a$ be a positive integer which is not a perfect square, and consider the equation \[k = \frac{x^2-a}{x^2-y^2}.\]Let $A$ be the set of positive integers $k$ for which the equation admits a solution with $x>\sqrt{a}$, and let $B$ be the set of positive integers for which the equation admits a solution with $0\leq x<\sqrt{a}$. Show that $A=B$.

\textbf{Solution.} 
To show that $k\in A\to k\in B$, let $x>\sqrt{a}$ for some $x$ satisfying the equation. 
It follows that $|y|<|x|$. 
Denote $y=x-c$ and we have $a=x^2-k(x^2-y^2)=x^2-kc(2x-c)=x^2-2kcx+kc^2$. 
Let $x_1=x-2kc$ and $y_1=x_1+c$ and we have 
$\dfrac{x_1^2-a}{x_1^2-y_1^2}$
$=\dfrac{(x-2kc)^2-(x^2-kc(2x-c))}{(x_1-y_1)(x_1+y_1)}$
$=\dfrac{-2kc(2x-2kc)+kc(2x-c)}{-c(2(x-2kc)+c)}$
$=\dfrac{-kc(4x-4kc-2x+c)}{-c(2x-4kc)+c)}$
$=k$. 
This means $k$ admits $(x_1, y_1)$ as well, and from $x_1=y_1+c<y_1$ we have $x_1<\sqrt{a}$. 
Also notice that $x\ge 2kc$ because... so $x_1\ge 0$. Therefore
$k\in B$ too. 

Conversely, we want to show that $k\in B\to k\in A$. 
Let $x<\sqrt{a}$ for some $x$ satisfying the equation. 
It follows that $|y|>|x|$. 
Denote $y=x+c$ and we have $a=x^2-k(x^2-y^2)=x^2-k(-c)(2x+c)=x^2+2kcx-kc^2$. 
Let $x_2=x+2kc$ and $y_2=x_10c$ and we have 
$\dfrac{x_2^2-a}{x_1^2-y_2^2}$
$=\dfrac{(x+2kc)^2-(x^2+kc(2x+c))}{(x_2-y_2)(x_2+y_2)}$
$=\dfrac{2kc(2x+2kc)-kc(2x+c)}{c(2(x+2kc)-c)}$
$=\dfrac{kc(4x+4kc-2x-c)}{c(2x+4kc)-c)}$
$=k$. 
This means $k$ admits $(x_2, y_2)$ as well, and from $x_2=y_2+c>y_2$ we have $x_1>\sqrt{a}$. Therefore
$k\in A$ too. 

\item[\textbf{N6}]
Denote by $\mathbb{N}$ the set of all positive integers. Find all functions $f:\mathbb{N}\rightarrow \mathbb{N}$ such that for all positive integers $m$ and $n$, the integer $f(m)+f(n)-mn$ is nonzero and divides $mf(m)+nf(n)$.

\textbf{Solution.}
The only function is $f(x)=x^2, \forall x\in\mathbb{N}$. 
In fact, $mf(m)+nf(n)=m^3+n^3=(m+n)(m^2-mn+n^2)=(m+n)(f(m)-mn+n^2)$. 

Substituting $m=n=1$ gives $2f(1)-1|2f(1)$, so $2f(1)-1=\pm 1$. 
Since $f(1) > 0$ ($f(1)\in\mathbb{N}$) we have $f(1)=1$. 
Next, letting $n=1$ gives $f(m)-(m-1)|mf(m)+1=m(f(m)-(m-1))+m^2-m+1$, 
so with $f(m)-(m-1)|m^2-m+1$ and $m^2-m+1=(m-\frac 12)^2+\frac 34 > 0$ we have 
$|f(m)-(m-1)|\le m^2-m+1$ and $f(m)\le m^2$. 

The next step is to show that $f(p)=p^2$ for all sufficiently large prime $p$. 
Substituting $m=n=p$ gives $2f(p)-p^2|2pf(p)=p(2f(p)-p^2)+p^3$, so 
$2f(p)-p^2|p^3$ and from $f(p)\le p^2$ we have 
$2f(p)-p^2\in\{p^2, p, 1, -1, -p\}$ (again it this value cannot be $-p^2$ or lower because $f(p)>0$. 
Therefore $f(p)\in\{p^2, \frac{p^2+p}2, \frac{p^2+1}2, \frac{p^2-1}2, \frac{p^2-p}2\}$. 
Now we check $n=1, m=p$ again and we have (from above) $f(p)-(p-1)|p^2-p+1$. 
We investigate the following cases: 
\begin{itemize}
\item [(a)] $f(p)=\frac{p^2+p}2$, 
then $\frac{p^2+p}2-(p-1)|p^2-p+1=2(\frac{p^2+p}2-(p-1))-1$, 
so $\frac{p^2+p}2-(p-1)\le 1, $
which doesn't hold for $p\ge 2$. 

\item [(b)] $f(p)=\frac{p^2+1}2$, 
then $\frac{p^2+1}2-(p-1)|p^2-p+1=2(\frac{p^2+1}2-(p-1))+p-2$, 
which means $\frac{p^2+1}2-(p-1)\le p-2$, not true for $p\ge 3$. 

\item [(c)] $f(p)=\frac{p^2-1}2$, 
then $\frac{p^2-1}2-(p-1)|p^2-p+1=2(\frac{p^2-1}2-(p-1))+p$, 
meaning  $\frac{p^2+1}2-(p-1)\le p$, not true for $p\ge 3$. 

\item [(d)] $f(p)=\frac{p^2-p}2$, 
then $\frac{p^2-p}2-(p-1)|p^2-p+1$. 
Observe that $2(\frac{p^2-p}2-(p-1))=p(p-1)-2(p-1)=(p-1)(p-2)$, 
so $p-1|2(p^2-p+1)$. Now $2(p^2-p+1)\equiv 2(1^2-1+1)=2\pmod{p-1}$, 
so $p-1\le 2$< or $p\le 3$. 
\end{itemize}
We therefore know that all four cases cannot hold for $p\ge 5$, so $f(p)=p^2$ for $p\ge 5$. 

Now, let $m$ be arbitrary and let $n=p$ for some prime $p$ we have 
$f(m)+p^2-mp|mf(m)+p^3=m(f(m)+p^2-mp)+p^3-mp^2+m^2p=m(f(m)+p^2-mp)+p(p^2-pm+m^2)$. 
Consider the ratio $\dfrac{p(p^2-pm+m^2)}{f(m)+p^2-mp}$
$=p(1+\dfrac{m^2-f(m)}{f(m)+p^2-mp})$, 
and therefore $\dfrac{p(m^2-f(m))}{f(m)+p^2-mp}$
must be an integer. 
Choosing any $p>f(m)$ gives 
$p\nmid f(m)$, and hence $p\nmid f(m)+p^2-mp$, hence $p$ and $f(m)+p^2-mp$ are relatively prime. 
Therefore $\dfrac{m^2-f(m)}{f(m)+p^2-mp}$ is itself an integer, 
and with $f(m)+p^2-mp$ approaching infinity as $p$ approaching infinity we know that $f(m)-m^2$ must be zero. 
(Remember, there are infinitely many primes). 

\end{itemize}



\end{document}