\documentclass[11pt,a4paper]{article}
\usepackage{amsmath, amssymb, fullpage, mathrsfs, bm, pgf, tikz}
\usepackage{mathrsfs}
\usetikzlibrary{arrows}
\setlength{\textheight}{10in}
%\setlength{\topmargin}{0in}
\setlength{\topmargin}{-0.5in}
\setlength{\parskip}{0.1in}
\setlength{\parindent}{0in}

\begin{document}
\newcommand{\la}{\leftarrow}
\newcommand{\lra}{\leftrightarrow}


\title{Solution to IMO 2016 shortlisted problems.}
\author{Anzo Teh}
\date{\today}
\maketitle

Preface: 
The IMO 2016 shortlisted problems were published on July 19, 2017: after the second day of IMO 2017. As I am not involved in any IMO training, I have no access to the shortlisted problems until that day. Thereafter, I worked on the problems to the best ability possible, and decided to compile my solution to the problems here. 

A shortcoming of providing only the solutions to the readers is that, it does not show the process of generating insights needed to solve the problems. 
For problems involving functional equations (A4, N6, and the currently unsolved A7), the solutions are usually intuitive because the majortiy of it is to substitute variables. 
However, for problems like C7 (Problem 6: lines and endpoints) it may not be suggestive at all on how do we come out with the notion of "bearings" in the solutions; for N5 (set $A$ = set $B$) you'd never understand how can we come out with the alternative pairs without trying it yourself. 
In view of this, I have a writeup on my own thought proess before each solution, in effort of making it easy for readers to understand how did I arrive at the ideas needed for the solution. 

To conclude: the thought process contains the intuitive part of the writing, while the solution contains the part where mathematical rigor is maintained at the highest possible level. 

\newpage
\section{Algebra}
\begin{itemize}
\item[\textbf{A1}]
Let $a$, $b$, $c$ be positive real numbers such that $\min(ab,bc,ca) \ge 1$. Prove that $$\sqrt[3]{(a^2+1)(b^2+1)(c^2+1)} \le \left(\frac{a+b+c}{3}\right)^2 + 1.$$

\textbf{Thoughts.} 
Non-homogeneous inequality, with the relation $\min(ab,bc,ca) \ge 1$ that we aren't sure how to use it (at its first sight). 
How should we solve it, then? 
The way, when desperate, is to brute force the whole thing by cubing the left-hand side and trying to expand 
$\left(\frac{a+b+c}{3}\right)^6+3\left(\frac{a+b+c}{3}\right)^4+3\left(\frac{a+b+c}{3}\right)^2+1$, 
but let's not make our lives miserable with (probably) hundreds of terms on the right. 

Instead, we start with the following observations: 
\begin{itemize}
\item [1.] What happens when $a=b=c$? Then the equality holds! 
This motivates us to show that whenever $a+b+c$ is fixed, the maximal possible value of $(a^2+1)(b^2+1)(c^2+1)$ can be attained whenever $a=b=c$. With this in mind...
\item [2.] We want to see what happens to $(a^2+1)(b^2+1)$ by ranging all possible pairs of $(a,b)$ such that $ab\ge 1$ and $a+b$ is fixed. 
As it turns out, $(a^2+1)(b^2+1)=a^2b^2+a^2+b^2+1$ (there are only four terms so it doesn't hurt to expand). 
Since $a+b$ is fixed in our context, we can write this in terms of $a+b$, giving 
$a^2b^2+a^2+b^2+1=(a+b)^2+(ab-1)^2$. 
Now it's easy to see that this value increases with $ab$ (as $ab\ge 1$). 
\item [3.] Now with $a+b$ fixed, $ab$ increases when $|a-b|$ decreases. 
The next step is to compare $f(a,b,c)$ with $f(k,k,k)$ where $k$ is the average of $a,b,c$. 
(Here we denote $f(a,b,c)$ as $(a^2+1)(b^2+1)(c^2+1)$). 
Ideally, we want to find $x$ and $y$ with average $k$ such that 
$f(a,b,c)\le f(k,x,y)\le f(k,k,k)$. 
The right inequality is easy to establish for any $x, y$ with sum $2k$, given that $\min(kx, ky, xy)\ge 1$. 
This, however, requires us to maintain this invariant for the right inequality. 
Additionally, for left inequality to work we need one of $k, x, y$ to be the same as one of $a, b, c$, 
which requires some case-by-case analysis (i.e. $m(a,b,c)\le k$ and $m(a,b,c)\ge k$, where $m$ denotes the median of the three variables here). 
These aren't hard, just some work needed. 
\end{itemize}

\textbf{Solution.} 
We start with a preliminary observation: 
given that $k\ge 2$, and given the set of pairs $K=\{(a,b): a+b=k, ab\ge 1\}$, 
then for any $(a_1,b_1), (a_2, b_2)\in K$ we have
\[(a_1^2+1)(b_1^2+1)\ge (a_2^2+1)(b_2^2+1) \text{ iff } |a_1-b_1|\le |a_2-b_2|.\] 
Indeed, for $(a,b)\in K$, $(a^2+1)(b^2+1)=(a+b)^2+a^2b^2-2ab+1$
$=k^2+(ab-1)^2$. 
In addition, $ab=\frac{(a+b)^2-(a-b)^2}{2}=\frac{k^2-(a-b)^2}{2}$
so we have 
\[(a_1^2+1)(b_1^2+1)-(a_2^2+1)(b_2^2+1)
=(a_1b_1-1)^2-(a_2b_2-1)^2
=(a_1b_1+a_2b_2-2)(a_1b_1-a_2b_2)\]
\[=(a_1b_1+a_2b_2-2)\left(\frac{(k^2-(a_1-b_1)^2)-(k^2-(a_2-b_2)^2)}{2}\right)\]
\[=(a_1b_1+a_2b_2-2)\left(\frac{(a_2-b_2)^2-(a_1-b_1)^2)}{2}\right).\]
Now that $a_1b_1, a_2b_2\ge 1$, 
$(a_1^2+1)(b_1^2+1)-(a_2^2+1)(b_2^2+1)\ge 0$ iff 
$a_1b_1-a_2b_2\ge 0$. 
(Technically we need to consider the case where $a_1b_1=a_2b_2=1$, which gives $a_1b_1+a_2b_2-2=0$. 
However this will also give $a_1b_1-a_2b_2=0$ so the claim is valid.)
This means $(a_1^2+1)(b_1^2+1)-(a_2^2+1)(b_2^2+1)\ge 0$ iff $(a_2-b_2)^2-(a_1-b_1)^2\ge 0$ 
iff $|a_2-b_2|\ge a_1b_1|$. 
In other words, 
In addition, with $a+b$ fixed, this function is also dereasing in $(a-b)^2$, which turns out to also be decreasing in $|a-b|$. 

Now let $a+b+c=3k$, and let $f(a,b,c)=(a^2+1)(b^2+1)(c^2+1)$. 
Notice that the left hand side is $\sqrt[3]{f(a,b,c)}$ while the right hand side is $(k^2+1)=\sqrt[3]{f(k,k,k)}$. 
W.l.o.g. assume that $a\le b\le c$. We want to show that 
\begin{itemize}
\item [1.] If $b\le k$ then $f(a,b,c)\le f(a, k, b+c-k)\le f(k, k, k)$. 
\item [2.] If $b\ge k$ then $f(a,b,c)\le f(a+b-k, k, c)\le f(k,k,k)$. 
\end{itemize}

In the first case, we have $a\le k$ so $b+c\ge 2k$, meaning that 
$b+c-k\ge k$. 
Moreover, $b+c-k+k=b+c$, and $(b+c-k)-k=(b+c)-2k\ge b+c-2b=c-b$. 
Therefore $k(b+c-k)\ge bc\ge 1$ and by above, $(b^2+1)(c^2+1)\le (k^2+1)((b+c-k)^2)+1$, thus 
$f(a,b,c)\le f(a, k, b+c-k)$. 
In addition, $a\le k\le b+c-k$, so $\min\{ak, a(b+c-k), k(b+c-k)\}=ak\ge ab\ge 1$. 
We also have $a+(b+c-k)=3k=k=k+k$, 
so by the first paragraph again 
$(a^2+1)((b+c-k)^2+1)\le (k^2+1)^2$, 
which gives 
$f(a, k, b+c-k)\le f(k, k, k)$. 

In the second case, we have (similarly)
$b+a\le 2k$ ($a\le k$ and $c\ge k$), which means $2k-(a+b)\ge 0$, or $k\ge a+b-k$. In addition, 
$k-(a+b-k)=2k-(b+c)\le 2b-(a-b)=b-a$ (since $b\ge k$), 
and 
$ab\le k(a+b-k)$ for this reason. 
Therefore, 
$(a^2+1)(b^2+1)\le ((a+b-k)^2+1)(k^2+1)$ 
and we have $f(a,b,c)\le f(a+b-k, k, c)$. 
Additionally, $a+b-k\le k\le c$ so
$\min\{(a+b-k)k, kc, (a+b-k)c\}=(a+b-k)k\ge ab\ge 1$ (as proven above). 
By the fact that $(a+b-k)+c=k+k$ we have, by first paragraph, 
$((a+b-k)^2+1)(c^2+1)\le (k^2+1)^2$ so 
$f(a+b-k, k, c)\le f(k,k,k)$. 

\newpage
\item[\textbf{A2}]
 Find the smallest constant $C > 0$ for which the following statement holds: among any five positive real numbers $a_1,a_2,a_3,a_4,a_5$ (not necessarily distinct), one can always choose distinct subscripts $i,j,k,l$ such that
\[ \left| \frac{a_i}{a_j} - \frac {a_k}{a_l} \right| \le C. \]

\textbf{Thoughts.} 
By arranging $a_i\le a_j$ and $a_k\le a_l$ we know that $C$ is bounded above by 1. A baby step, but a great start.

The next sensible thing we can do is to sort the numbers in order: 
$a_1\le a_2\le a_3\le a_4\le a_5$. 
Also it's important to realize that $|\frac{a_1}{a_2}-\frac{a_4}{a_5}|=|\frac{a_1a_5-a_2a_4}{a_2a_5}|\ge |\frac{a_1a_5-a_2a_4}{a_4a_5}|=|\frac{a_1}{a_4}-\frac{a_2}{a_5}|$ so we just have to consider the latter. 
In a similar fashion let's consider $|\frac{a_1}{a_3}-\frac{a_2}{a_4}|$, 
$|\frac{a_2}{a_4}-\frac{a_3}{a_5}|$. 
As seen below, this is just a comparison among 
$\frac 1{bc}|\frac 1a-\frac 1d|$, $\frac 1b|\frac 1a-\frac 1c|$, $\frac 1c|\frac 1b-\frac 1d|$ 
(as below  $a=\frac {a_2}{a_1}$, 
$b=\frac {a_3}{a_2}$, 
$c=\frac {a_4}{a_3}$, 
$d=\frac {a_5}{a_4}$). 
It's difficult to see how great the minimum of the three numbers can go, 
but in light of the factors $\frac 1b$ and $\frac 1c$ we can try some simple cases like $b=c=1$, 
giving $|\frac 1a-\frac 1d|$, $|\frac 1a-1|$, $|1-\frac 1d|$. 
Considering $0\le \frac 1a, \frac 1d\le 1$ we have 
$(1-\frac 1a)+(\frac 1a-\frac 1d)=(1-\frac 1d)$ 
so considering that $0\le 1-\frac 1d, 1-\frac 1a\le 1$ and assuming that $\frac 1a-\frac 1d\ge 0$ we have 
$1\ge 1-\frac 1d\ge 2\min\{1-\frac 1a, \frac 1a-\frac 1d\}$, which means one of the elements in the set must be at most $\frac 12$. 
A similar conclusion can be reached for the case $\frac 1a-\frac 1d\le 0$. 
Moreover, this motivates the following equality case: by setting $a=2$ and $d$ approaching infinity (and yeah, this is how the example $1, 2, 2, 2, n$ can be conjured). 
The job now reduces to proving that $\min \{\frac 1{bc}|\frac 1a-\frac 1d|, \frac 1b|\frac 1a-\frac 1c|, \frac 1c|\frac 1b-\frac 1d|\}\le \frac 12$, which is no longer hard by case-by-case analysis as below. 

\textbf{Answer.} $C=\frac 12$. 

\textbf{Solution.} 
First, we show that $C\ge\frac 12$ is necessary. 
Suppose that there exists $\epsilon >0$ such that for each $a_1,a_2,a_3,a_4,a_5$ we can choose $i, j, k, l$ with 
$\left| \frac{a_i}{a_j} - \frac {a_k}{a_l} \right| \le \frac 12-\epsilon $. 

First, let $a_1=1, a_2=a_3=a_4=2$, $a_5=n$ for some extremely large real number $n$. 
The possible numbers of $\frac {a_i}{a_j}$ are $\frac 1n, \frac 2n, \frac 12, 1, 2, \frac n2, n$. 
Observe that the ratios $\frac 1n, \frac 2n, \frac n2, n$ all involve $a_5$, so there cannot exist distinct $i, j, k, l$ satisfying 
$\frac{a_i}{a_j}, \frac {a_k}{a_l}\in\{\frac 1n, \frac 2n, \frac n2, n\}$. 
In addition, both $\frac 12$ and 2 will involve $a_1$, meaning that there cannot exist distinct $i, j, k, l$ satisfying 
$\frac{a_i}{a_j}, \frac {a_k}{a_l}\in\{\frac 12, 2\}$. 
Since there are three $i$'s satisfying $a_i=2$, there cannot be distinct $i,j,k,l$ satisfying $\frac{a_i}{a_j}=\frac {a_k}{a_l}=1$. 
We therefore know that  $\frac{a_i}{a_j}=\frac {a_k}{a_l}$ is impossible, 
and same goes to $\frac{a_i}{a_j}=\frac 1n, \frac {a_k}{a_l}=\frac 2n$. 
We also have $n-\frac n2> \frac n2 -2 > 2-1>1-\frac 12>\frac 12-\frac 2n$ for sufficiently large real $n$, 
and if $\frac 12-\frac 2n > C=\frac 12-\epsilon$ then $C<n-\frac n2, \frac n2 -2, 2-1, 1-\frac 12$, 
so $\frac 12-\frac 2n\le \frac 12-\epsilon$ for all $n$< which does not hold for $n > 2\epsilon$. 
Therefore $C\ge \frac 12$. 

Now, we show that $C=\frac 12$ fits in all situations. 
W.L.O.G. let $a_1\le a_2\le a_3\le a_4\le a_5$, and let $a=\frac {a_2}{a_1}$, 
$b=\frac {a_3}{a_2}$, 
$c=\frac {a_4}{a_3}$, 
$d=\frac {a_5}{a_4}$. 
Observe that $a, b, c, d\ge 1$. 
Suppose that for some $a,b,c,d$, $C=\frac 12$ does not fit for any of the four distinct subscripts. 
Now, considering $|\frac {a_1}{a_4}-\frac {a_2}{a_5}|$
$=\frac 1{bc}|\frac 1a-\frac 1d|$
and from $0\le \frac 1a, \frac 1d\le 1$
we have $\frac bc\ge \frac bc|\frac 1a-\frac 1d|>C=\frac 12$
so $b, c<2$. 
Next, $C<|\frac {a_1}{a_3}-\frac {a_2}{a_4}|$
$=\frac 1b|\frac 1a-\frac 1c|$
$\le |\frac 1a-\frac 1c|$
and from $\frac 1c<\frac 12$ we must have $\frac 1a>\frac 12$. 
Similarly, $C<|\frac {a_2}{a_4}-\frac {a_3}{a_5}|$
$=\frac 1c|\frac 1b-\frac 1d|$
$\le |\frac 1b-\frac 1d|$
and from $\frac 1b<\frac 12$ we must have $\frac 1d>\frac 12$. 
Looking back, we have $\frac 12<|\frac {a_1}{a_4}-\frac {a_2}{a_5}|=\frac bc|\frac 1a-\frac 1d|\le|\frac 1a-\frac 1d|$, 
yet $\frac 12<\frac 1a , \frac 1d\le 1$, contradiction.

\newpage
\item[\textbf{A3}]
Find all positive integers $n$ such that the following statement holds: Suppose real numbers $a_1$, $a_2$, $\dots$, $a_n$, $b_1$, $b_2$, $\dots$, $b_n$ satisfy $|a_k|+|b_k|=1$ for all $k=1,\dots,n$. Then there exists $\varepsilon_1$, $\varepsilon_2$, $\dots$, $\varepsilon_n$, each of which is either $-1$ or $1$, such that
\[ \left| \sum_{i=1}^n \varepsilon_i a_i \right| + \left| \sum_{i=1}^n \varepsilon_i b_i \right| \le 1. \]

\textbf{Thoughts.} 
One good counterexample construction can be found using the triangle inequality: 
$ \left| \sum_{i=1}^n \varepsilon_i a_i \right| + \left| \sum_{i=1}^n \varepsilon_i b_i \right|\ge  \left| \sum_{i=1}^n \varepsilon_i a_i + \sum_{i=1}^n \varepsilon_i b_i \right|$. 
Therefore, for the case where $a_i, b_i\ge 0$ we have: 
$a_i+b_i=1$ 
so the right hand side now becomes $\left| \sum_{i=1}^n \varepsilon_i \right|$. 
When $n$ is even, we know that the condition holds only when $\sum_{i=1}^n \varepsilon_i$ is 0, which means exactly half of the $\varepsilon_i$'is is 1 and the other -1. 
This motivates us to think of the counterexample $(1, 0, 0, \cdots , 0)$ on one side $(0, 1, 1, \cdots, 1)$ on the other (and yes this counterexample is valid). 
For odd $n$, we have $\sum_{i=1}^n \varepsilon_i=\pm 1$ to consider. 
Some experimentation yields that we can have $\sum_{i=1}^n \varepsilon_i= 1$ and 
$0\le \sum_{i=1}^n \varepsilon_i a_i\le 1$, and 
$0\le \sum_{i=1}^n \varepsilon_i b_i\le 1$, giving 
$| \sum_{i=1}^n \varepsilon_i a_i|+|\sum_{i=1}^n \varepsilon_i b_i|= 1$. 
(Well, the proof for the existence of $\varepsilon_i$'s isn't that straightforward, but so does the construction of the counterexamples $a_i$'s and $b_i$'s which is guarateed to fail). The proof of the existence of such $\varepsilon_i$'s is better explained in the solution. 

In view of this we likely know that the answer *might* be positive even extends to the cases that $a_i$'s and $b_i$'s might have opposite signs for odd $n$. And yes, the invariant of $|a_i-b_i|=1$ or $|a_i+b_i|=1$ further motivates us to split the pairs $(a_i, b_i)$ into those with same signs and different signs. The simplified case of $a_i, b_i\ge 0$ for all $i$ also motivates us to find such $\varepsilon_i$'s such that $\sum\varepsilon_i a_i$ and $\sum\varepsilon_i b_i$ are both in $[-1, 1]$ for $a_i, b_i\ge 0, $ which we can further narrow the interval down to $[0, 1]$ for both when $m$ (the total number of summands) is odd, and $[0, 1]$ for sum of $a_i$'s and $[-1, 0]$ for sum of $b_i$'s when $m$ is even. This is all we need to prove the answer is "yes". 

\textbf{Answer.} All odd $n$. 

\textbf{Solution.} 
We first find a counterexample for all $n=2k$ for some $k\ge 1$ an integer. 
Consider $a_i=1$, $b_i=0$ for all $i\in [1, 2k-1]$, 
and $a_{2k}=0, b_{2k}=1$. 
Then $\sum_{i=1}^n \varepsilon_i b_i=\varepsilon_{2k}$, which has absolute value of 1. 
Also $\sum_{i=1}^n \varepsilon_i a_i=\sum_{i=1}^{2k-1} \varepsilon_i$. 
If $x$ of the indices $\varepsilon_1, \varepsilon_2, \cdots , \varepsilon_{2k-1}$ are 1 and the rest $-$1, 
then the value would be $x-(2k-1-x)=2x-2k+1$, which is an odd integer. Thus it has absolute value at least 1 too. 
Therefore we have 
$\left| \sum_{i=1}^n \varepsilon_i a_i \right| + \left| \sum_{i=1}^n \varepsilon_i b_i \right|\ge 1+1=2$. 

Now let $n$ be odd. 
We start with the following lemma: 
Let $(a_i, b_i), i\in [1,m]$ satisfy $0\le a_i, b_i\le 1$ and $a_i+b_i=1, \forall i\in [1,m]$. 
Then there exists $\varepsilon_1, \varepsilon_2, \cdots, \varepsilon_m\in \{-1, 1\}$ satisfying
\begin{center}
\begin{tabular}{c c c c} 
$0\le \sum_{i=1}^m \varepsilon_i a_i, \sum_{i=1}^m \varepsilon_i b_i\le 1$
& and & $\sum_{i=1}^m \varepsilon_i a_i +\sum_{i=1}^m \varepsilon_i b_i=1$ & if  $m$ is odd, \\ 
$-1\le \sum_{i=1}^m \varepsilon_i a_i\le 0\le\sum_{i=1}^m \varepsilon_i b_i\le 1$
& and & $\sum_{i=1}^m \varepsilon_i a_i +\sum_{i=1}^m \varepsilon_i b_i=0$ & if  $m$ is even. \\ 
\end{tabular} 
\end{center}
In the first case, we notice that the second condition can be achieved whenever exactly $\frac{m+1}2$ of $\varepsilon_1, \varepsilon_2, \cdots , \varepsilon_m$ are 1 and the rest $-1$. 
For any combination of $\varepsilon_1, \varepsilon_2, \cdots , \varepsilon_m$ with exactly $\frac{m+1}{2}$ of them as 1 we have 
$ \sum_{i=1}^m \varepsilon_i a_i  + \sum_{i=1}^m \varepsilon_i b_i $
$=\sum_{i=1}^m \varepsilon_i$
$=\frac{m+1}{2}-\frac{m-1}{2}=1$. 
The aim now is to assign $\varepsilon_1, \varepsilon_2, \cdots , \varepsilon_m$ 
in a way that exactly $\frac{m+1}2$ of them are 1, and $0\le \sum_{i=1}^m \varepsilon_i a_i\le 1$ 
(if this is true, $0\le \sum_{i=1}^m \varepsilon_i b_i\le 1$ holds true too). 
W.L.O.G. assume that $a_1\le a_2\le\cdots \le a_m$, and consider the numbers 
$x_0, x_1, \cdots , x_{\frac{m-1}2}$ such that 
\[ x_k=\sum_{i=1}^{k} -a_i+\sum_{i=k+1}^{k+\frac{m+1}{2}} a_i+\sum_{i=k+\frac{m+1}{2}+1}^{m} -a_i\]
Notice, first, that for each $x_i$, exactly $\frac{m+1}{2}$ of the $a_i$'s has coefficient 1 and the rest -1. 
Therefore if $0\le x_k\le 1$ for some $k$ we are done. 
Observe also that, 
\[x_0=\sum_{i=1}^{\frac{m+1}{2}} a_i+\sum_{i=\frac{m+1}{2}+1}^{m} -a_i
\le \sum_{i=1}^{\frac{m+1}{2}} a_{\frac{m+1}{2}} +\sum_{i=\frac{m+1}{2}+1}^{m} -a_{\frac{m+1}{2}}
=a_{\frac{m+1}{2}}
\le 1;\]
\[x_{\frac{m-1}{2}}=\sum_{i=1}^{\frac{m-1}{2}} -a_i+\sum_{i=\frac{m+1}{2}}^{m} a_i
\ge \sum_{i=1}^{\frac{m-1}{2}} -a_{\frac{m+1}{2}}+\sum_{i=\frac{m+1}{2}}^{m} a_{\frac{m+1}{2}}
=a_{\frac{m+1}{2}}
\ge 0.\] 
If $x_0, x_{\frac{m-1}{2}}\not\in [0, 1]$ (observe that we are done if either of them is in the interval), then $x_0<0$ and $x_{\frac{m-1}{2}}>1$. 
This allows us to choose a $k$ such that $x_k>1$ and $x_{k-1}\le 1$. 
Moreover, $x_k-x_{k-1}=a_{k+\frac{m+1}{2}}-a_{k}-a_{k}+a_{k+\frac{m+1}{2}}=2(a_{k+\frac{m+1}{2}}-a_{k}), $
so $x_{k-1}\le a_k\le x_{k-1}+2$ (since $|a_i-a_j|\le 1$). 
If $x_{k-1}\ge 0$ we are done. 
Otherwise, we have $x_k-x_{k-1}>1$ and therefore $a_{k+\frac{m+1}{2}}-a_{k}>\frac 12$, 
meaning that $a_{k+\frac{m+1}{2}}>\frac 12$. 
Now let $y_k=x_k-2a_{k+\frac{m+1}{2}}$ we have $y_k\le x_{k-1}<0$ but $y_k\ge x_k-2>-1$. 
This means $0\le -y_k\le 1$ and 
\[-y_k=\sum_{i=1}^{k} a_i+\sum_{i=k+1}^{k+\frac{m+1}{2}-1} -a_i+\sum_{i=k+\frac{m+1}{2}}^{m} a_i.\] 
Among them, $\varepsilon_1$, $\varepsilon_2$, $\cdots$, $\varepsilon_k$, $\varepsilon_{k+\frac{m+1}2}, \cdots, \varepsilon_m$ are 1 $(k+m-(k+\frac{m+1}2-1)=\frac{m+1}2$ of them) with the rest $-1$. Hence we are done. 

The second case isn't much different from the first. 
The second condition can be achieved whenever exactly half of $\varepsilon_1, \varepsilon_2, \cdots , \varepsilon_m$ are 1 and the rest $-1$. 
To see why, we have 
$ \sum_{i=1}^m \varepsilon_i a_i  + \sum_{i=1}^m \varepsilon_i b_i $
$=\sum_{i=1}^m \varepsilon_i$
$=\frac m2-\frac m2=0$. 
With this fixed, all we need to make sure is that 
$ \sum_{i=1}^m \varepsilon_i a_i \in [0,1]$. 
Again w.l.o.g. let $a_1\le a_2\le\cdots \le a_m$, and consider the numbers 
$x_0, x_1, \cdots , x_{\frac{m}2}$ such that 
\[ x_k=\sum_{i=1}^{k} -a_i+\sum_{i=k+1}^{k+\frac{m}{2}} a_i+\sum_{i=k+\frac{m}{2}+1}^{m} -a_i\]
Again each $x_k$ satisfies exactly $\frac m2$ have coefficient 1 and the rest -1. 
Thus if $-1\le x_k\le 1$ for some $k$ we are done. 
Otherwise, notice that
\[x_0=\sum_{i=1}^{\frac{m}{2}} a_i+\sum_{i=\frac{m}{2}+1}^{m} -a_i
\le \sum_{i=1}^{\frac{m}{2}} a_{\frac{m}{2}} +\sum_{i=\frac{m}{2}+1}^{m} -a_{\frac{m}{2}}
=0;\]
\[x_{\frac{m}{2}}=\sum_{i=1}^{\frac{m}{2}} -a_i+\sum_{i=\frac{m}{2}+1}^{m} a_i
\ge \sum_{i=1}^{\frac{m}{2}} -a_{\frac{m}{2}} +\sum_{i=\frac{m}{2}+1}^{m} a_{\frac{m}{2}}
=0.\]
This forces $x_0<-1$ and $x_{\frac m2}>1$, 
which allows us to pick a $k$ satisfying $x_k>1$ and $x_{k-1}\le 1$. 
By the similar logic as in case 1 we have $0\le x_k-x_{k-1}\le 2$. 
This means $x_{k-1}>-1$, which gives $-1\le x_{k-1}\le 1$. 
If $x_{k-1}\le 0$ we are done. Otherwise, $-1\le -x_{k-1}\le 0$ and we have 
\[-x_{k-1}=\sum_{i=1}^{k-1} a_i+\sum_{i=k}^{k+\frac{m}{2}-1} -a_i+\sum_{i=k+\frac{m}{2}}^{m} a_i.\]
Again $\varepsilon_i=1$ for $i=1, 2, \cdots k-1$ and $k+\frac{m}{2}, \cdots, m$, which isn't that hard to verify that there are exactly $\frac m2$ of them. 

To finish up the solution, 
we split the $a_i$'s and $b_i$'s into two groups: 
one with $a_ib_i\ge 0$, and the other one with $a_ib_i\le 0$ (if $a_ib_i=0$ then we assign them arbitrarily). 
W.L.O.G. let the first group be 
$(a_i, b_i)$; $i\in [1, m]$ and the second group be $(a_i, b_i)$: $i\in [m+1, n]$. 
For $i\in [1, m]$, we can w.l.o.g. assume that $a_i, b_i\ge 0$ (indeed, if $\varepsilon_i$ is a solution for $(a_i, b_i)$ then $-\varepsilon_i$ is a solution for $(-a_i, -b_i)$). 
Similarly we can also assume that $i\in [m+1, n]$ we have $a_i\ge 0$ and $b_i\le 0$. 
Also notice that $|x|+|y|\in \{x+y, x-y, y-x, -x-y\}$ 
so to prove that $|x|+|y|\le 1$ all w need to do is to prove that $-1\le x+y\le 1$ and $-1\le x-y\le 1$. 
(Bonus: try to show that this is an if and only if condition: convince yourself that $|x|+|y|=\max \{x+y, x-y, y-x, -x-y\}$.)

We split into two cases: 
\begin{itemize}
\item [Case 1.] $m$ is odd. 
From our lemma, by some careful choices of $\varepsilon_i$'s we have 
\begin{center}
\begin{tabular}{c c c} 
$0\le \sum_{i=1}^m \varepsilon_i a_i, \sum_{i=1}^m \varepsilon_i b_i\le 1$
& and & $\sum_{i=1}^m \varepsilon_i a_i +\sum_{i=1}^m \varepsilon_i b_i=1$ ; \\ 
$-1\le \sum_{i=m+1}^n \varepsilon_i a_i\le 0\le \sum_{i=m+1}^n \varepsilon_i (-b_i)\le 1$
& and & $\sum_{i=m+1}^n \varepsilon_i a_i +\sum_{i=m+1}^n \varepsilon_i (-b_i)=0$. \\ 
\end{tabular} 
\end{center}
This means, $\sum_{i=m+1}^n \varepsilon_i a_i=\sum_{i=m+1}^n \varepsilon_i b_i=c$ for some $c\in [-1, 0]$. 
Also let $\sum_{i=1}^m \varepsilon_i a_i=a$ and $\sum_{i=1}^m \varepsilon_i b_i=b$, from which we know 
$a+b=1$ and $0\le a, b\le 1$. 
Now our term of interest $\left| \sum_{i=1}^n \varepsilon_i a_i \right| + \left| \sum_{i=1}^n \varepsilon_i b_i \right|$ becomes $|a+c|+|b+c|=|a+c|+|1-a+c|$
Now we have 
$a+c+b+c=1+2c$ and from $-1\le c\le 0$ we have $-1\le c\le 1$. 
Also, $a+c-b-c=a-b=a-(1-a)=2a-1$ and from $0\le a\le 1$ we have $-1\le 2a-1\le 1$. 
This settles our case 1. 

\item [Case 2.] $m$ is even. 
 Again by our lemma we have 
\begin{center}
\begin{tabular}{c c c} 
$-1\le \sum_{i=1}^m \varepsilon_i a_i\le 0\le \sum_{i=1}^m \varepsilon_i b_i\le 1$
& and & $\sum_{i=1}^m \varepsilon_i a_i +\sum_{i=1}^m \varepsilon_i b_i=0$ ; \\ 
$0\le \sum_{i=m+1}^n \varepsilon_i a_i,  \sum_{i=m+1}^n \varepsilon_i (-b_i)\le 1$
& and & $\sum_{i=m+1}^n \varepsilon_i a_i +\sum_{i=m+1}^n \varepsilon_i (-b_i)=1$. \\ 
\end{tabular} 
\end{center}
Now, let $\sum_{i=1}^m \varepsilon_i a_i=a$ and we have $\sum_{i=1}^m \varepsilon_i b_i=-a$ ($-1\le a\le 0$), 
let $\sum_{i=m+1}^n \varepsilon_i a_i=c$ and we have $\sum_{i=m+1}^n \varepsilon_i b_i=c-1$ ($0\le c\le 1$). 
Again, we need to consider $|a+c|+|-a+c-1|$. 
Observe that $a+c-a+c-1=2c-1$ and from $0\le c\le 1$ we have $-1\le 2c-1\le 1$. 
$a+c+a-c+1=2a+1$, and from $-1\le a\le 0$ we have $-1\le 2a+a\le 1$, which serves our purpose. 
\end{itemize}

\newpage
\item[\textbf{A4}]
Find all functions $f:(0,\infty)\rightarrow (0,\infty)$ such that for any $x,y\in (0,\infty)$, $$xf(x^2)f(f(y)) + f(yf(x)) = f(xy) \left(f(f(x^2)) + f(f(y^2))\right)\cdots (*).$$

\textbf{Thoughts.} Naturally, plugging $x=y=1$, $x=1$ and $y=1$ are the first hings we need to do (since $x=0$ is disallowed here). 
This gives us $xf(x^2)=f(x)$, after which we can (kind of) conjecture that the answer is $f(x)=\frac 1x$. 
A few more experimentations will give an intuition that there isn't any other solution, but it's not obvious why it is so. 
Therefore, in light of the results we have proven in steps 2 and 3 we should convert $xf(x^2)$ to $f(x)$ (nicer equation) and $f(f(y))$ to $f(y)f(f(y^2))$ (matches more closely with right-hand-side), and now we have (**). 
It's true that $f(xy)$ looks nasty (where we do not know how to proceed), but since we know that $f(1)=1$ we should consider cases where $xy=1$, which easily gives us $f(\frac 1x)=\frac 1{f(x)}$. 
Finally, we can compare (*) when $(x, y)$ is replaced with $(\frac 1x, \frac 1y)$ in light of what we got in step 4 (undeniably, this is a clever trick but somehow necessary because any further substitution seems hopeless at this point). 

\textbf{Answer.} The only function is $f(x)\equiv \frac 1x$

\textbf{Solution.} 
Our function works because
$xf(x^2)f(f(y)) + f(yf(x))=x\frac1 {x^2} y+\frac 1 {y\frac 1{x}}$
$=\frac xy+\frac yx$
$=\frac{x^2}{xy}+\frac{y^2}{xy}$
$=f(xy)\left(f(f(x^2)) + f(f(y^2))\right)$. 

For the rest of the solution we proceed with the normal functional algorithmic procedure to show that $f(x)\equiv \frac 1x$ is the only function: 
\begin{itemize}
\item [Step 1.] 
Plugging $x=y=1$ gives $f(1)f(f(1))+f(f(1))=2f(1)f(f(1))$, 
and since $f>0$, we can factorize $f(f(1))$ out to get $f(1)+1=2f(1)$, giving \fbox{$f(1)=1$}. 

\item [Step 2.]
Plugging $x\la 1$ (and substituting $f(1)\la 1$ due to step 1) gives $f(f(y))+f(y)=f(y)(1+f(f(y^2)))$, 
giving \fbox{$f(f(y))=f(y)f(f(y^2))$}. 

\item [Step 3.]
Plugging $y\la 1$, on the other hand, gives 
$xf(x^2)+f(f(x))=f(x)(f(f(x^2))+1)$. 
From step 2, $f(f(x))=f(x)f(f(x^2))$, 
which gives rise to \fbox{$xf(x^2)=f(x)$}. 

\item [Step 4.]
Substitute $xf(x^2)\la f(x)$ (step 3), $f(f(y))\la f(y)f(f(y^2))$ (step 2), and $yf(x)\la xyf(x^2)$ (step 3) into (*) gives: 
$$f(x)f(y)f(f(y^2)) + f(xyf(x^2)) = f(xy) \left(f(f(x^2)) + f(f(y^2))\right)\cdots(**).$$
In the special case where $xy=1$ we have \[f(x)f(y)f(f(y^2)) + f(f(x^2)) = 1 \left(f(f(x^2)) + f(f(y^2))\right),\] 
so $f(x)f(y)=1$ whenever $xy=1$. 
In other words, for all $x\in\mathbb{R}^+$, \fbox{$f(\frac 1x)=\frac 1{f(x)}$}. 

\item [Step 5.]
Having the results in Step 4 in mind, we do the following substitution: 
\begin{itemize}
\item [5a.]
Substitute $\frac 1x$ and $\frac 1x$ in place of $x$ and $y$ into (**) (step 4) we have
$$f\left(\frac 1x\right)f\left(\frac 1x\right)f\left(f\left(\frac 1{x^2}\right)\right) + f\left(\frac 1{x^2}f\left(\frac 1{x^2}\right)\right)
= f\left(\frac 1{x^2}\right) \left(f\left(f\left(\frac 1{x^2}\right)\right) + f\left(f\left(\frac 1{x^2}\right)\right)\right)$$
\[\frac1{f(x)}\cdot \frac1{f(x)}\cdot f\left(\frac 1{f(x^2)}\right)+f\left(\frac 1{x^2f(x^2)}\right)
=\frac1{f(x^2)}\left(2f\left(\frac 1{f(x^2)}\right)\right)\]
\[\frac1{f(x)}\cdot \frac1{f(x)}\cdot \frac 1{f\left(f(x^2)\right)}+\frac 1{f\left(x^2f(x^2)\right)}
=\frac1{f(x^2)}\cdot \frac 2{\left(f\left(f(x^2)\right)\right)}\]
\[\frac 1{f(x)^2 f(f(x^2))}+ \frac 1{f(x^2f(x^2))}
= \frac 2{f(x^2)f(f(x^2))}\]. 
\item [5b.]
Subsitute $x=y$ into (**) we get 
$f(x)^2f(f(x^2)) + f(x^2f(x^2)) = 2f(x^2) f(f(x^2)).$
\item [5c.]
From now on denote $a=f(x)^2f(f(x^2))$ and $b=f(x^2f(x^2))$. 
Substituting $f(x^2) f(f(x^2))\la \frac{a + b} 2$ gives 
$\frac 1{a}+ \frac 1{b}=\frac 2{f(x^2)f(f(x^2))}=\frac 4{a+b}.$
which we can cross multiply to get 
$(a+b)^2=4ab$, or $(a-b)^2=a^2+b^2-ab=a^2+b^2+2ab-4ab=(a+b)^2-4ab=0$. 
This yields $a-b=0$, and hence 
$f(x)^2f(f(x^2))=a=b=f(x^2f(x^2))$. 
Now looking back to 5(b) again we have 
$2f(x^2) f(f(x^2)) = f(x)^2f(f(x^2)) + f(x^2f(x^2)) = f(x)^2f(f(x^2)) + f(x)^2f(f(x^2))=2f(x)^2f(f(x^2)),$ 
so $f(x^2) f(f(x^2))=f(x)^2 f(f(x^2))$, or simply 
$f(x^2)=f(x)^2$ (after factorizing $f(f(x^2))$ out which we assumed its non-zero. 
Finally from step 3 again we have 
$f(x)=xf(x^2)=xf(x)^2$, so $xf(x)=1$ and $f(x)=\frac 1x$. 

\end{itemize}
\end{itemize}

\newpage
\item[\textbf{A5}]
Consider fractions $\frac{a}{b}$ where $a$ and $b$ are positive integers.
\begin{itemize}
\item[(a)]
Prove that for every positive integer $n$, there exists such a fraction $\frac{a}{b}$ such that $\sqrt{n} \le \frac{a}{b} \le \sqrt{n+1}$ and $b \le \sqrt{n}+1$.
\item[(b)]
Show that there are infinitely many positive integers $n$ such that no such fraction $\frac{a}{b}$ satisfies $\sqrt{n} \le \frac{a}{b} \le \sqrt{n+1}$ and $b \le \sqrt{n}$. 
\end{itemize}

\textbf{Thoughts.} 
Part (b) should very well give a hint on part (a) : $b=\lfloor\sqrt{n}+1\rfloor$ is necessary. 
Together with another trick of arranging the natural numbers $n$ according to their integer square root 
(i.e. $\lfloor\sqrt{n}\rfloor$) the case of $b=\lfloor\sqrt{n}+1\rfloor$ covers half of the required numbers 
(i.e. those numbers in the form $n=k^2+1, k^2+3, \cdots k^2+2k-1$). 
Therefore we only need to worry about the case $n=k^2, k^2+2, \cdots , k^2+2k$, 
which turns out all to be comfortably settled by the case $b=\lfloor\sqrt{n}\rfloor$
(and how do we know that? Experimenting with small numbers!). 

Part (b) requires some experimentation, which is not too hard if we carry on with the partition of the natural numbers according to their integer square roots. 
Notice, also, that there is no suitable $b$ for $n=k^2+1$ for some $k$, which can be proven using the inequality 
$(k+\frac 1k)^2>k^2+2$. This is precisely what we need to solve the problem. 

\textbf{Solution.} 
For part (a), we partion the set of positive integers according to their integer square roots, 
that is, the sets $S_1=\{1,2,3\}$, $S_2=\{4,5,6,7,8\}$, $S_3=\{9,10,11,12,13,14,15\}$, etc. 
Consider $S_k=\{k^2, k^2+1, \cdots k^2+2k\}$, 
and we claim that $b=k$ and $b=k+1$ alone will jointly work for the sets. 
(That is, for every positive integer $n\in S_k$, there exists such a fraction $\frac{a}{b}$ such that $\sqrt{n} \le \frac{a}{b} \le \sqrt{n+1}$ and $b \le \{k, k+1\}$). 
Indeed, let $n=k^2+a$ with $0\le a\le 2k$. 
If $a=2x$ for some integer $x$ then 
$(k+\frac xk)^2=k^2+ 2k\left(\frac xk\right)+\left(\frac xk\right)^2$
$=k^2+2x+\frac{x^2}{k^2}$
and since $x=\frac a2\le k$ we have $\frac{x^2}{k^2}\le 1$. 
Therefore, 
$\sqrt{n}=\sqrt{k^2+a}\le k+\frac xk\le\sqrt{k^2+a+1}=\sqrt{n+1}$. 
On the other hand, if $a=2x-1$ for some integer $x\in [1, k]$ we have 
$(k+\frac x{k+1})^2=k^2+\frac{2xk}{k+1}+(\frac x{k+1})^2$
$=k^2+2x-\frac{2x}{k+1}+(\frac x{k+1})^2$. 
Notice that $-\frac{2x}{k+1}+(\frac x{k+1})^2=\frac{x^2-2x(k+1)}{(k+1)^2}$
$=\frac{(x-(k+1))^2-(k+1)^2}{(k+1)^2}$
$=\frac{(x-(k+1))^2}{(k+1)^2}-1$, 
and with $0\le x\le k+1$ we have 
$-1\le \frac{(x-(k+1))^2}{(k+1)^2}-1\le 0$. 
Therefore 
$\sqrt{n}=\sqrt{k^2+2x-1}\le k+\frac x{k+1}\le \sqrt{k^2+2x}=\sqrt{n+1}$. 

As for part (b) we show that there's no fraction $\frac ab$ (with $b\le k$) lying in the interval $[\sqrt{k^2+1}, \sqrt{k^2+2}]$. 
Notice that, $k< \sqrt{k^2+1} < \sqrt{k^2+2} < \sqrt{k^2+2k+1} =k+1$. 
Assume that $\frac ab$ satisfies this property, 
then from $\frac ab > k$ and $b\le k$ we have 
$(\frac ab)^2\ge (k+\frac 1k)^2=k^2+2+\frac 1{k^2} > k^2+2$, contradiction. 

\newpage
\item[\textbf{A6/IMO 5}]
The equation
$$(x-1)(x-2)\cdots(x-2016)=(x-1)(x-2)\cdots (x-2016)$$is written on the board, with $2016$ linear factors on each side. What is the least possible value of $k$ for which it is possible to erase exactly $k$ of these $4032$ linear factors so that at least one factor remains on each side and the resulting equation has no real solutions?

\textbf{Thoughts.} 2016 is the minimum--the first thing that must jump out immediately of your mind. 
The construction should also be intuitive: we need to either have $LHS>RHS$ all the times or vice versa. 
W.L.O.G. let's make $LHS<RHS$. One observation is that, $RHS<0$ must imply $LHS<0$, meaning that on each side there are oddly many factors with roots greater than $x$ (hence contributing to negative factors). We therefore want to find a construction that satisfies the following bare minimum: as we travel along the positive $x$-axis, $LHS(x)$ must dip into negative area before $RHS(x)$ does, and $RHS(x)$ must go into positive area before $LHS(x)$ does. 

Thus this gives the "mod 4" construction as detailed below, because whenever $RHS$ is negative we have $x\in (4i-2, 4i-1)$ for some integer $i$, which also guarantees the negativity of $LHS$. (This is also the simplest construction fulfilling the bare minimum, which turns out to work too!)
Next, it is not hard to prove that $|LHS|<|RHS|$ whenever both of them are positive, 
so the only hard part is to prove that $|LHS|>|RHS|$ when both negative. 
Thankfully $|\frac{(x-(4i-2))(x-(4i-1))}{(x-4i)(x-(4i-3))}|\le \frac 19$ whenever $x\in (4i-2, 4i-1)$, and we can finish off by using appropriate sum telescoping, again as detailed below. 

\textbf{Answer.} 2016.

\textbf{Solution.}  
For each $i$, the factor $x-i$ cannt appear on both sides (otherwise $i$ will itself be a root), 
so $x-i$ must be erased on on of the sides for each $i\in \{1, 2, \cdots , 2016\}$, forcing at least 2016 factors to be erased. 
It remains to show that 2016 is good to go. 

We claim that the equation 
$$\displaystyle\prod_{i=1}^{504} (x-(4i-3))(x-(4i))=\displaystyle\prod_{i=1}^{504} (x-(4i-2))(x-(4i-1))$$ 
has no real solution by showing that the left-hand side is always strictly smaller than the right hand side, realized by the following simpler cases: 
\begin{itemize}
\item [Case 1.] 
$x\in\{1,2,\cdots ,2016\}$. 
Now, if $x=4i$ or $x=4i+1$ then $LHS=0$ while $RHS$ has $2i$ negative factors (while the rest positive) hence positive, 
so $LHS=0<RHS$. 
If $x=4i-1$ or $x=4i-2$ then the right is 0 while the left has $2i-1$ negative factors (while the rest positive) hence negative, 
giving $LHS<0=RHS$. 

\item [Case 2.] $x\in (4i+1, 4i+2)$ for some integer $i\in [0, 503]$. 
Now there are $2i+1$ negative factors (and the rest $1007-2i$ positive) on the left (hence negative) while $2i$ negative factors (and the rest $1008-2i$ positive) on the right (hence positive). This gives $LHS<0<RHS$.

\item [Case 3.] $x\in (4i-1, 4i)$ for some integer $i\in [1, 504]$. 
Similar to case 2, there are $2i-1$ negative factors (and the rest $1009-2i$ positive) on the left (hence negative) while $2i$ negative factors (and the rest $1008-2i$ positive) on the right (hence positive). 
Again $LHS<0<RHS$. 

\item [Case 4.] $x>2016$, $x<1$, or $x\in (4j, 4j+1)$ for some integer $j\in [1, 503]$. 
Observe the following relation:
$$(x-(4i-2))(x-(4i-1))-(x-(4i-3))(x-(4i))=(4i-1)(4i-2)-(4i-3)(4i)=2\cdots (*)$$ 
We claim that for each integer $i\in [1, 504]$ we have 
$(x-(4i-2))(x-(4i-1)), (x-(4i-3))(x-(4i))>0$. 
If $x>2016$ we have $x-(4i-2), x-(4i-1), x-(4i-3), x-(4i) >0$. 
If $x<1$ we have $x-(4i-2), x-(4i-1), x-(4i-3), x-(4i) <0$ (and recall that the product of two negative numbers are positive). 
If $x\in (4j, 4j+1)$ for some integer $j\in [1, 503]$ we have: 
$$\begin{cases} 
x-(4i-2), x-(4i-1), x-(4i-3), x-(4i) >0 & \text{if } i\le j \\ 
x-(4i-2), x-(4i-1), x-(4i-3), x-(4i) <0 & \text{if } i > j 
\end{cases} $$
Therefore we always have 
$$|(x-(4i-2))(x-(4i-1))|>|(x-(4i-3))(x-(4i))|.$$ 
This implies 
$$\displaystyle\prod_{i=1}^{504} |(x-(4i-3))(x-(4i))|<\displaystyle\prod_{i=1}^{504} |(x-(4i-2))(x-(4i-1))|$$ 
and since each side is positive, 
$$\displaystyle\prod_{i=1}^{504} (x-(4i-3))(x-(4i))<\displaystyle\prod_{i=1}^{504} (x-(4i-2))(x-(4i-1)).$$ 
\end{itemize}

We are now left with the trickiest case: 
\begin{itemize}

\item [Case 5.]
$x\in (4i-2, 4i-1)$ for some $i\in [1, 504]$, whereby both sides are negative. 
The goal is therefore to show that $|LHS|>|RHS|$.  
By (*) in case 4 we always have 
$(x-(4j-2))(x-(4j-1))-(x-(4j-3))(x-(4j))=2$, 
and since $x\in (4i-2, 4i-1)$, for $j > i$ we have $x-(4j-2), x-(4j-1), x-(4j-3), x-(4j)<0$ and 
for $j<i$ we have $x-(4j-2), x-(4j-1), x-(4j-3), x-(4j)>0$. 
This allows us to conclude that whenever $j\neq i$, we have 
$(x-(4j-2))(x-(4j-1)), (x-(4j-3))(x-(4j))>0$. 
First, from $(x-(4i-2))(x-(4i-1))=(x-(4i-1.5))-\frac 14\ge -\frac 14$ we get 
\[\frac{|(x-(4i-2))(x-(4i-1))|}{|(x-(4i))(x-(4i-3))|}=\frac{c}{c+2}=1-\frac{2}{c+2}\le 1-\frac 2{2+\frac 14}=\frac 19\] 
where $c=|(x-(4i-2))(x-(4i-1))|.$
Next, let's investigate $\dfrac{|(x-(4j-2))(x-(4j-1))|}{|(x-(4j))(x-(4j-3))|}$ 
for some $j<i$. 
We know that $x>4i+1$, so 
$(x-(4j-2))(x-(4j-1))> (4i-4j-1)(4i-4j)=4(i-j)(4(i-j)-1)$. 
Again letting $c= (x-(4j-2))(x-(4j-1))$ we have 
\[\frac{|(x-(4j-2))(x-(4j-1))|}{|(x-(4j))(x-(4j-3))|}
=\frac{c}{c-2}
=1+\frac 2{c-2}
<1+\frac 2{4(i-j)(4(i-j)-1)-2}\]
\[=1+\frac 1{2(i-j)(4(i-j)-1)-1}
<1+\frac 1{(i-j+1)^2-1},\]
the last inequality holds since for $i-j\ge 1$ we have 
\[2(i-j)(4(i-j)-1)-1-((i-j+1)^2-1)
=8(i-j)^2-2(i-j)-((i-j)^2+2(i-j)+1)\]
\[=(i-j)(7(i-j)-4)-1
\ge 1(7-4)-1=2\]
and therefore $2(4(i-j)-1)(i-j)-1>(i-j+1)^2-1$ for $j\le i-1$.
Thus 
\[\dfrac{\displaystyle\prod_{j=1}^{i-1} (x-(4j-2))(x-(4j-1))}{\displaystyle\prod_{j=1}^{i-1} (x-(4j-3))(x-(4j))}
<\displaystyle\prod_{j=1}^{i-1}\left(1+\frac 1{(i-j+1)^2-1}\right)
=\frac{2^2}{2^2-1}\cdot\frac{3^2}{3^2-1}\cdot\cdots\cdot \frac{i^2}{i^2-1}\]
\[=\frac{2(2)}{1(3)}\cdot\cdots\cdot \frac{i(i)}{(i-1)(i+1)}
=2\times\frac{i}{i+1}<2\]
(notice that we dropped the modulus sign since each product is positive, as proven before).  
Likewise, $\dfrac{\displaystyle\prod_{j=i+1}^{504} (x-(4j-2))(x-(4j-1))}{\displaystyle\prod_{j=i+1}^{504} (x-(4j-3))(x-(4j))}$
$<2$. 
Thus $\dfrac{\displaystyle\prod_{i=1}^{504} |(x-(4i-2))(x-(4i-1))|}{\displaystyle\prod_{i=1}^{504} |(x-(4i-3))(x-(4i))|}$
$<2\times \frac{1}{9}\times 2$
$=\frac 49<1$, 

\end{itemize}
and now we are done (omg the long proof...)

\end{itemize}

\newpage
\section{Combinatorics}
\begin{itemize}
\item[\textbf{C1}]
 The leader of an IMO team chooses positive integers $n$ and $k$ with $n > k$, and announces them to the deputy leader and a contestant. The leader then secretly tells the deputy leader an $n$-digit binary string, and the deputy leader writes down all $n$-digit binary strings which differ from the leader’s in exactly $k$ positions. (For example, if $n = 3$ and $k = 1$, and if the leader chooses $101$, the deputy leader would write down $001, 111$ and $100$.) The contestant is allowed to look at the strings written by the deputy leader and guess the leader’s string. What is the minimum number of guesses (in terms of $n$ and $k$) needed to guarantee the correct answer?

\textbf{Thoughts.} 
This is equivalent to finding the number of possible strings said by the leader given the strings said by the deputy leader. 
Intuitively, for each digit, there is $\frac kn$ probability of it being changed by the deputy leader (so there will be $\frac kn$ of the strings with the digit being changed). 
From here, the string can be uniquely determined if $\frac kn\neq \frac 12$: 
for each digit we know that it is 0 or 1.  
In the case where $\frac kn=\frac 12$, 
then for each digit there are equally many strings with 0 on it as those wih 1 on it. 
Nevertheless, if we only look at the strings with leading 0, 
then among those strings, for each of the rest of the digits there are $\frac{k-1}{n-1}$ of the strings with that digit being changed 
(and yeah $\frac{k-1}{n-1}\neq \frac 12$) 
so the string can be uniquely determined like above (same goes for the strings with leading 1). 
Now we have two candidates, and the last step is to prove that it works. 
The verification might sound difficult, but again all we need is to show that if one string works, the string determined by flipping all digits works too.

\textbf{Answer.} The answer is 2 for $n=2k$ and 1 otherwise. 

\textbf{Solution.}
Notice that there are $\binom nk$ strings the deputy leader can write. 
For the $i$-th digit (for any $i\in [0, n-1]$), there are $\binom {n-1}{k-1}$ such strings with $i$-th digit differing from the original, 
$\binom {n-1}k$ such strings with $i$-th digit equal to the original. 
If $\frac{(n-1)!}{(k-1)!(n-k)!}=\binom {n-1}{k-1}\neq \binom {n-1}k=\frac{(n-1)!}{(k)!(n-k-1)!}$, the contestant can determine that digit by counting the number of strings with 0 in it (and the number of strings with 1 in it). 
This happens when $(k-1)!(n-k)!\neq k!(n-k-1)!$, or $n-k\neq k$ (factorizing factors out) or $n\neq 2k$. 
Since for each digit, the contestant can determine what is that digit (0 or 1) of the string, no further guess is needed and the answer is 1.

If $n=2k$, then for each digit, half of the strings written by deputy leader have one's and half have zero's. 
The student then considers the strings with 0 as the leading digit. 
If, the correct string has 0 on that leading digit, then for each of the written strings (with leading 0), among the remaining $2k-1$ digits there are $k-1$ being changed from the original. By the claim above (by considering just the substrings containing the last $2k-1$ digits of the original) the student can determine the remaining $2k-1$ digits. 
Similar conclusion can be reached for the case with 1 as leading digit: $k$ digits being changed from the remaining $2k-1$ digits. 
This gives the student the correct answer after 2 guesses. 
To see why 2 guesses is necessary, let $a_0a_1\cdots a_{2k-1}$ be the string given by the leader, $b_0b_1\cdots b_{2k-1}$ be a string with $b_i=1-a_i$ for each $i$, $c_0c_1\cdots c_{2k-1}$ be any string written by the deputy leader. 
Now, we have $c_i=a_i$ or $c_i=b_i$ but not both. 
With $c_0c_1\cdots c_{2k-1}$ having $k$ same digits and $k$ different digits as $a_0a_1\cdots a_{2k-1}$, 
it must have $2k-k=k$ same digits and $2k-k=k$ different digits as $b_0b_1\cdots b_{2k-1}$ too. 
Thus $b_0b_1\cdots b_{2k-1}$ is actually another possibility and the contestant cannot distinguish between $a_0\cdots a_{2k-1}$ and $b_0\cdots b_{2k-1}$ by just looking at the strings written by the deputy leader. 

\newpage
\item[\textbf{C2}]
Find all positive integers $n$ for which all positive divisors of $n$ can be put into the cells of a rectangular table under the following constraints:\\
$\bullet$ each cell contains a distinct divisor;\\
$\bullet$ the sums of all rows are equal; and\\
$\bullet$ the sums of all columns are equal.\\

\textbf{Thoughts.} 
Prime numbers can't work (too obvious to explain), but how to generalize this idea to the general case? 
Since the divisors are all distinct, there cannot be exactly one row or one column, which means the sum of each column and each row must be greater than $n$. 
It might jump directly out of you that the sum of divisors of $n$ (namely $\sigma_1(n)$) in this case must be greater $2n$ (meaning that the sum of divisors must be greater than $2n$); we can prove even more: $\sigma_1(n)$ must be greater than both $rn$ and $cn$ where $r, c$ are the row count and column count of the rectangle, respectively. 
Given also that $rc=\sigma_0(n)$ is the number of divisors of $n$, we know that $\sigma_1(n)>n\sqrt{\sigma_0(n)}$. 
This turned out to be enough to produce a contradiction for those $n$ which are not powers of 2. 

\textbf{Solution.} 
The answer is $n=1$, which works with 1 being placed in a $1\times 1$ table. 
To show that this fails for other $n$, first prime factorize it into 
$\displaystyle\prod_{i=1}^k p_i^{a_i}$. 
If $r$ is the number of rows and $c$ is the number of columns then $rc=\displaystyle\prod_{i=1}^k (a_i+1)$, 
the number of divisors of $n$. 
W.l.o.g. $r\ge c$ and therefore $r\ge\sqrt{\displaystyle\prod_{i=1}^k (a_i+1)}=\displaystyle\prod_{i=1}^k \sqrt{(a_i+1)}$. 
We have also known that the sum of divisors is 
$\displaystyle\prod_{i=1}^k \frac{p_i^{a_i+1}-1}{p_i-1}$. 
Knowing that one of the cells contains $n$, the sum of each row must be greater than $n$, 
($n$ cannot be the only cell in that row, otherwise all cells would have to contain the same number which is absurd for $n>1$). 
This means that the sum of each column is greater than $rn$, giving the following inequality: 
$$\displaystyle\prod_{i=1}^k \frac{p_i^{a_i+1}-1}{p_i-1}>rn\ge\displaystyle\prod_{i=1}^k \sqrt{(a_i+1)}p_i^{a_i}$$
or equivalently, 
$\displaystyle\prod_{i=1}^k \frac{p_i-\frac{1}{p_i^{a_i}}}{(p_i-1)\sqrt{(a_i+1)}}>1$. 
Now for each prime $p$ and positive integer $a$ we let $f(p, a)=\frac{p-\frac{1}{p^{a}}}{(p-1)\sqrt{(a+1)}}=\frac1{\sqrt{(a+1)}}\left(1+p^{-1}+\cdots +p^{-a}\right)$, and we show that 
\begin{itemize}
\item[1.] $f(p,a)<f(q, a)$ whenever $p>q$. 
\item[2.] $f(2,a)\le f(2,1)=\sqrt{\frac 98}$ and $f(p,a)\le f(3,1)=\sqrt{\frac 89}$ for all $p\ge 3$. 
\end{itemize} 

(1) is easy: for $p>q$ we have 
\[f(p,a)=\frac1{\sqrt{(a+1)}}\left(1+p^{-1}+\cdots +p^{-a}\right)<\frac1{\sqrt{(a+1)}}\left(1+q^{-1}+\cdots +q^{-a}\right)=f(q,a).\]
Now for (2), we notice that: 
$f(2, 1)=\frac{2-\frac{1}{2}}{\sqrt{(1+1)}}=\frac3{2\sqrt{2}}=\sqrt{\frac 98}, $
$f(2, 2)=\frac{2-\frac{1}{2^2}}{\sqrt{(2+1)}}=\frac7{4\sqrt{3}}=\sqrt{\frac {49}{48}}<\sqrt{\frac 98}, $
and for all $a\ge 3$ we have 
$f(2, a)=\frac{2-\frac{1}{2^a}}{\sqrt{(a+1)}}<\frac{2}{\sqrt{(3+1)}}=1<\sqrt{\frac 98}$. 
$f(3, 1)=\frac{3-\frac{1}{3}}{2\sqrt{(1+1)}}=\frac8{6\sqrt{2}}=\sqrt{\frac 89}$, 
and for all $a\ge 2$ we have 
$f(3, a)=\frac{3-\frac{1}{3^a}}{2\sqrt{(a+1)}}<\frac 3{2\sqrt{2+1}}=\sqrt{\frac 34}<\sqrt{\frac 89}$. 
By (1) we have $f(p,a)\le f(3, a)\le f(3, 1)=\sqrt{\frac 89}$ whenever $a\ge 1$ and $p\ge 3$. 

Summing up, recall that we always have $\displaystyle\prod_{i=1}^k f(p_i, a_i)> 1$. 
If $p_1<p_2<\cdots <p_k$ then we have $p_i\ge 3$ for $i\ge 2$. 
If $k\ge 2$ we have $\displaystyle\prod_{i=1}^k f(p_i, a_i)\le f(2, a_1)\cdot \displaystyle\prod_{i=2}^k f(3, a_i)\le\sqrt{\frac 98}\times \sqrt{\frac 89}^{i-1}\le 1$, 
which is a contradiction. 
Therefore we must have $k=1$, with $f(p_1, a_1)>1$. By (2) we have $p=2$. 
However, this implies $n$ is a power of 2 and from $a_i\ge 1$, at least two rows must be used (we assumed $r\ge c$). 
The row containing $n$ must therefore have sum at least $2n$, 
but for $n$ a power of two the sum of divisors is $2n-1$, contradiction. 

\newpage
\item[\textbf{C3}]
Let $n$ be a positive integer relatively prime to $6$. We paint the vertices of a regular $n$-gon with three colours so that there is an odd number of vertices of each colour. Show that there exists an isosceles triangle whose three vertices are of different colours.

\textbf{Thoughts.} 
This problem is hard for a C3: how are we going to use the fact that an odd number of vertices is painted for each colour? 
Nevertheless, the fact that $3\nmid n$ generated some insight for us: for each fixed segment $AB$ with endpoint of different colours there are three vertices $C_1, C_2, C_3$ such that the resulting triangle is isoceles. 
This should motivate the double-counting solution: assuming the problem conclusion doesn't hold and we consider those isoceles triangles with exactly two colours being used, then each of these triangles has two polychromatic lines while each polychromatic line belongs to three such triangles, so we can play around with the parity to produce contradiction!

\textbf{Solution.} 
Suppose that the conclusion is false, and let $a, b, c$ be the three colours used. 
Let $\Gamma$ be the circumcircle of the regular $n$-gon. 
Let $N$ be the set of isoceles triangles such that each $a$ and $b$ is used at least once and the colour $c$ is not used, 
and let $M$ be the set of unordered pairs of vertices $\{A, B\}$ such that $A$ is of colour $a$ and $B$ is of colour $b$ (and vice versa). 

We start with an observation: each member of $M$ is one side of exactly three triangles in $N$. 
To see why, let $(A,B)$ be one member in $M$. From $2\nmid n$ we know that $AB$ cannot be a diameter of $\Gamma$. 
This implies that there exists one such vertex $C_1$ satisfying $AB=AC_1$, and another vertex $C_2$ satisfying $AB=BC_2$. 
In addition, since $n$ is odd, the perpendicular bisector of $AB$ will hit exactly one vertex in the $n$-gon, 
so there is exactly one such $C_3$ with $AC_3=BC_3$. 
$C_1, C_2, C_3$ are also pairwise different; otherwise $C_1=C_2=C_3$ and $ABC_1$ is equilateral, 
contradicting the fact that $3\nmid n$. 
Also notice that $ABC_1$, $ABC_2$, $ABC_3\in N$ because the colour $c$ is not used at $C$ (otherwise we are done since $A, B$ are of colour $a$ and $b$), and each colour $a$ and $b$ is used at least once. 
This gives us the required three triangles. 

Next, also notice that for each triangle in $N$, 
since no colour of $c$ is used, among each $a$ and $b$, one colour is used twice and the other once. 
This implies that each triangle in $N$ has two sides in $M$. 
If we consider all such pairs $(x, y)$ where $x\in M, y\in N$ and $x$ is a side of $y$ then we count each element in $M$ for 3 times and each element in $N$ for 2 times. 
Thus $2|N|=3|M|$. 
This means $M$ is even. On the other hand we also have $|M|=|a|\cdot |b|$ where $|a|$ and $|b|$ are the number of sides with colour $a$ and $b$, respectively. This means $|a|$ or $|b|$ is even, contradiction. 

\newpage
\item[\textbf{C4/IMO 2}]
Find all integers $n$ for which each cell of $n \times n$ table can be filled with one of the letters $I,M$ and $O$ in such a way that:\\
$\bullet$ in each row and each column, one third of the entries are $I$, one third are $M$ and one third are $O$; and \\
$\bullet$ in any diagonal, if the number of entries on the diagonal is a multiple of three, then one third of the entries are $I$, one third are $M$ and one third are $O$. 

\textbf{Thoughts.} 
The first condition already says that $3|n$, but sadly $n=3$ doesn't work (and it's extremely easy to prove that it doesn't). 
One way to start with is, therefore, to check the smallest $n$ that works. 
How about...making sure that if we partition the table into $3\times 3$ grids, each diagonal of the grids have exactly one of each $I, M, O$? 
Wait a minute...we have to make sure that among the center cells of the $3\times 3$ grids, there are equally many $I$, $M$, and $O$. 
Best to have that condition for the columns in the $3\times 3$ grids too, and we can sort out the rows later. 
This gives $n=9$ works, and for $n=9k$ all we need to do is to replicate the table. 

The proof of $9|n$ being necessary seems a little bit difficult, but the insights can be seen as we try to prove why $n=3$ fails: 
it seems like the fault lies on the center cell. 
As it turns out, the center cells of each $3\times 3$ grid lie on exactly two diagonals with size a multiple of 3 
(the corner: 1 and the sides: 0). 
This motivates us to consider just the columns and rows with indices congruent to 2 modulo 3, which works. 
(Alternatively one can also consider the columns and rows with indices not congruent to 2 modulo 3). 

\textbf{Answer.} Any $n$ divisible by 9. 

\textbf{Solution.} 
Throughout the solution we denote the coordinates of each cell as $(i, j):1\le i, j\le n$. 
We start by showing an example for $n=9$, given below: 
\begin{center}
\begin{tabular}{|c|c|c||c|c|c||c|c|c|}
\hline
I & M & O & M & O & I & O & I & M\\
\hline
M & M & M & O & O & O & I & I & I\\
\hline
I & M & O & M & O & I & O & I & M\\
\hline
\hline
O & I & M & I & M & O & M & O & I\\
\hline
I & I & I & M & M & M & O & O & O\\
\hline
O & I & M & I & M & O & M & O & I\\
\hline
\hline
M & O & I & O & I & M & I & M & O\\
\hline
O & O & O & I & I & I & M & M & M\\
\hline
M & O & I & O & I & M & I & M & O\\
\hline
\end{tabular}
\end{center}
For $n=9k$ for some $k$ we just have to split the grid into $k^2$ $9\times 9$ grids, and fill each one with the letters above. 
(Formally, if we let $(i, j): 1\le i\le j\le n$ be the table coordinates then for each $1\le i, j\le 9$ and $0\le a, b\le k-1$, 
$(i, j)$ and $(9a+i, 9b+j)$ contain the same letter). 
For sake of verification, observe that there are exactly 3 $I$'s, 3 $M$'s and 3 $O$'s in each column or each row of a single $9\times 9$ grid. 
Also, each diagonal is in the form of either $R_m=\{(i, j): i+j=m\}$, or $L_m=\{(i, j): i-j=m\}$, 
for some $m$ satisfying $1\le (i, j)\le n$. 
Now for $R_m$, the size $|R_m|$ is $m-1$ for $m\le n+1$, and $2n+1-m$ for $m\ge n+1$. 
Notice that 3 divides $|R_m|$ iff $m\equiv 1\pmod{n}$ for both of the cases ($m\le n+1$ vs $m\ge n+1$). 
Thus it is not hard to see that the diagonals are in the form of 
$(1, m-1), (2, m-2), \cdots (m-1, 1)$ in the first case, and 
$(m-n, n), (m-n+1, n-1), \cdots (n, m-n)$ in the second case. 
In each of the cases we can group them into groups of three, such that, if we further split each $9\times 9$ grids into $3\times 3$ grids, each group contains three cells along the main diagonal. 
Nevertheless, from the construction above we see that each main diagonal in the $3\times 3$ grids has one $I$, one $M$ and one $O$. 
(Formally speaking, among $(k, m-k), (k+1, m-k-1), (k+2, m-k-2)$ we have an $I$, an $M$ and an $O$ for $k\equiv 1\pmod{3}$ because they lie on the main diagonal of the same $3\times 3$ square.)
Thus this set of diagonal works. 
A similar conclusion can be yielded for diagonals in the form of $L_m$. 

To show that $9|n$ is necessary, observe from the first condition that $3|n$. 
Let $n=3k$ and let's split the table into $k^2$ $3\times 3$ cells. 
Notice from the logic (of diagonals characterization) as of above, 
the center of each $3\times 3 $ cell ($(i, j)$ where $i, j\equiv 2\pmod{3}$) lie on both $R_m$ and $L_m$ with both size divisible by 3; 
the four corners (($(i, j)$ where $i, j\not\equiv 2\pmod{3}$) lie on exactly one of the sets satifying the properties; 
the four sides (($(i, j)$ where exactly one of $i$ and $j$ is congruent to 2 mod 3) lie on none of them. 
Thus, when we mark the cells in each column, each row, and each diagonal with size divisible by 3, 
the center cells are marked 4 times, the corners thrice, and the sides twice (as illustrated below). 
\begin{center}
\begin{tabular}{|c|c|c|}
\hline
3 & 2 & 3\\
\hline
2 & 4 & 2\\
\hline
3 & 2 & 3\\
\hline
\end{tabular}
\end{center}
Let $c$ be the number of $M$'s on the center cells. 
Considering just the $3i-1$-th column for $i\in [1, k]$ and the $3j-1$-th row for $j\in [1, k]$ yields $2k^2$ $M$'s being counted. 
Each cell on the "side" is being counted once, each cell on the "center" twice, an each cell on the "corner" none. 
This gives the number of $M$'s on the side as $2k^2-c$, which follows that there must be $k^2+c$ $M$'s at the corner. 
Now let's see what happens as we consider all such markings (all columns, all rows, and all diagonals of size divisible by 3). 
Observe that for each $3\times 3$ cells we have $3+2+3+2+4+2+3+2+3=24$ markings, so each letter ($M$, in particular) has $8k^2$ markings. 
This means $8k^2=4c+2(2k^2-c)+3(k^2+c)=3c+7k^2$, or $c=\frac{k^2}{3}$. 
Hence $3|k^2$, or $3|k$, or $9|n$. 

\newpage
\item[\textbf{C5}]
Let $n \geq 3$ be a positive integer. Find the maximum number of diagonals in a regular $n$-gon one can select, so that any two of them do not intersect in the interior or they are perpendicular to each other. 

\textbf{Thoughts.} A few things must come into mind: no perpendicular diagonals for odd $n$ (proving this needs some work but not hard to think of), and for even case we will probably need perpendicular and intersecting diagonals to maximize our selection. It's also worth noting that, the more regions the diagonals split the circular arc into, the fewer available diagonals there are for us to use. 
There are intuitively true, and it's not difficult for us to arrive at the answer (or rather, conjecture) of $n-2$. The difficult part is to putting these arguments down to make our proofs rigorous. 

So how do we proceed? The first thing is to notice that those intersecting diagonals belong to just one pair of opposite classes: those in the same class will be parallel to each other and perpendicular to everyone in the opposite class (not hard to see---just draw it yourself). Next, we want to find those diagonals that will intersect the most diagonals that are of the opposite class--which we can show that they will intersect all diagonals of the opposite class. And that's basically it! Simply consider the relationship between the number of diagonals and the number of regions they split the circle into, and show that the rest of the diagonals must have diagonals lying entirely in the regions (which reduces to our simplified case of triangulation). 

\textbf {Answer. }$n-3$ for $n$ odd, and $n-2$ for $n$ even. (Alternatively, $2\lfloor \frac n2\rfloor-2$). 

\textbf{Solution.}
For simplicity we denote the circumcircle of the $n$-gon as $\Gamma$, and our $n$-gon be $A_0A_1\cdots A_{n-1}$, 
with $A_i$ having coordinates $C\left(\frac{2i\pi}{n}\right)=\left(\cos\frac{2i\pi}{n}, \sin\frac{2i\pi}{n}\right)$. 

We first show that no two diagonals can be perpendicular for $n$ odd. 
Now on the circle $\Gamma$, let one diagonal joining $C(a)$ and $C(b)$, and the other joining $C(c), C(d)$, 
with $0\le a, b, c, d<2\pi$. 
Then the diagonals have gradients $\frac{\sin b-\sin a}{\cos b-\cos a}$ 
and $\frac{\sin d-\sin c}{\cos d-\cos c}$. 
If they are perpendicular then the gradient has product $-1$ (one of them might be infinity, but this means that the other one has gradient 0). 
Therefore, 
$\frac{(\sin b-\sin a)(\sin d-\sin c)}{(\cos b-\cos a)(\cos d-\cos c)}=-1$, or $(\cos b-\cos a)(\cos d-\cos c)+(\sin b-\sin a)(\sin d-\sin c)=0$. 
Using the identity of $\cos b-\cos a=-2\sin\frac{b-a}{2}\sin\frac{b+a}{2}$ and 
$\sin b-\sin a=2\sin\frac{b-a}{2}\cos\frac{b+a}{2}$, we have 
$\sin\frac{b-a}{2}\sin\frac{d-c}{2}(\sin\frac{b+a}{2}\sin\frac{d+c}{2}+\cos\frac{b+a}{2}\cos\frac{d+c}{2})=0$. 
Since $0\le a,b,c,d<2\pi$, and $a\neq b, c\neq d$, 
we have $0< \frac{|a-b|}2, \frac{|c-d|}2 < \pi$, and therefore 
$\sin\frac{b-a}{2}, \sin\frac{d-c}{2}\neq 0$. 
We necessarily have 
$\sin\frac{b+a}{2}\sin\frac{d+c}{2}+\cos\frac{b+a}{2}\cos\frac{d+c}{2}=0$, or 
$\cos (\frac{(b+a)-(d+c)}2)=0$, which forces 
$\frac{(b+a)-(d+c)}2=\frac{k\pi}2$ for some odd $k$. 
Now, knowing that $a, b, c, d=\frac{2w\pi}{n}, \frac{2x\pi}{n}, \frac{2y\pi}{n}, \frac{2z\pi}{n}$ for some integers $0\le w, x, y, z<n$ we have 
$kn=2(w+x)-2(y+z)$. 
Knowing that $k$ is odd, $n$ has to be even. 
Thus the case where $n$ is odd reduces to finding the number of diagonals without any two of them intersecting in the interior. 
(The fact that a triangulation has exactly $n-3$ diagonals is well-known, but let's prove it anyway). 
Denoting $f(n)$ be this number and we show that $f(n)=n-3, \forall n\ge 3$. 
Base case when $n=3$, where $f(3)=0$ as there is no diagonal in a triangle. 
Next, let $f(n)=n-3$ for $n=3, 4, \cdots k$ for some $k\ge 3$. 
For $n=k+1$, let one diagonal split our $n$-gon into an $x$-gon and a $y$-gon, where $x+y=n+2$ and $x, y\ge 3$. 
Then there are extra $f(x)+f(y)=x-3+y-3=x+y-6=n-4$ diagonals that can be constructed (each diagonal must belong to only one of the polygons because it cannot intersect our first diagonal in the interior). 
This gives $n-4+1=n-3$ diagonals, at most. 
Equality can be achieved by taking diagonals $A_1A_k$, $k=3, 4, \cdots , n-1$.

Now $n$ be even. Let $F$ be the set of diagonals intersecting at least one other diagonal. We show that: 
\begin{itemize}
\item [1.]
Each two lines in $F$ are perpendicular or parallel to each other. 

\item [2.]
There are two lines in $F$ such that each of them intersects every other lines in $F$ that are perpendicular to itself.

\item [3.]
Let $A$ and $B$ be two consecutive endpoints of lines in $F$. 
Then for any selected diagonal not in $F$ with an endpoint on the minor arc $AB$ (that is different from $A$ and $B$), the other endpoint must also lie on this minor arc too (possibly equal to $A$ or $B$).

\end{itemize}
\emph{First claim:}
Now let $AC$ and $BD$ be two intersecting perpendicular chords on $\Gamma$. We claim that the four endpoints split the circles into four minor arcs (that is, arcs that are less than a semicircle). 
Let the intersection be $P$, then $\angle ACB=\angle ADB<\angle APB=90^{\circ}$, since $P$ lies on both segment $AC$ and $BD$. 
This means that the arc $AB$ not containing $C$ and $D$ ais minor, and a similar conclusion can be achieved for three other arcs. 
Suppose that $AC$ and $BD$ are in $F$, and let another diagonal with endpoints $E, G$ be in $F$ and perperndicular to neither of $AC$ nor $BD$. 
This means that there necessarily exists another diagonal (with endpoints $H, J$) that intersects $EG$ in its interior (and perpendicular to it). 
Since they do not intersect chords $AC$ and $BD$, they must lie on a minor arc (say, $AB$). 
This instantly contradicts the fact that $E, F, G, H$ must split $\Gamma$ into four minor arcs. 

\emph{Second claim:}
Claim 1 allows us to split them into two sets $F_1$ and $F_2$ such that any two diagonals are parallel to each other if and only if they belong to the same partition. 
An example would be $F_1=\{(A_aA_b): a+b\equiv 0\pmod{n}\}$ and $F_2=\{(A_aA_b): a+b\equiv \frac n2\pmod{n}\}$, 
valid by the identity derived in the first paragraph. 
We claim that the longest line in $F_1$ intersects every line in $F_2$ in its interior, and similarly the longest line in $F_2$ intersects every line in $F_1$ in its interior. 
Indeed, let $AB$ and $CD$ be parallel lines on in $F_1$ satisfying $CD\ge AB$, 
then $ABCD$ is an isoceles trapezoid satisfying $\angle A=\angle B\ge 90^{\circ}$ and $\angle C=\angle D\le 90^{\circ}$. 
Denote also $E$ and $F$ as the perpendiculars from $A$ and $B$ on $CD$. It follows that $E$ and $F$ are on the closed segment $CD$ itself. 
Let $l$ be a line perpendicular to both $AB$ and $CD$, intersecting open segment $AB$ at $G$ and line $CD$ at $H$. 
With $AE\parallel BF\parallel GH$, the fact that $G$ lies on segment $AB$ means that $H$ lies on open segment $EF$ too, 
therefore lying on open segment $CD$ as well. 
This means, any perpendicular line intersecting $AB$ in its interior will intersect $CD$ in its interior too. 
Now let $l$ be the longest line in $F_1$, and $m$ be any line in $F_2$. Since there exists $n$ in $F_1$ such that $m$ intersects $n$ in its interior, and $n$ is no longer than $l$, $m$ must intersect $l$ in its interior too. This proves the claim that every line in $F_2$ intersects $l$ in its interior, and similarly every line in $F_1$ intersects the longest line in $F_2$ in its interior too. 

\emph{Third claim:} 
denote the endpoints of the longest line in $F_1$ as $C$ and $D$, and in $F_2$ as $E$ and $F$ (which might or might not be completely distinct from $A$ and $B$). 
Let $G$ be a point on minor arc $AB$ and presumably it's an endpoint of a selected diagonal (with $G\neq A$ and $G\neq B$). 
W.L.O.G. we assume that $AB$ lies on minor arc $CE$ (so same goes for $G$, and the other endpoint, obviously, must also be on the minor arc $CE$). W.L.O.G. also that $C, A, G, B, E$ are on $\Gamma$ in that order.
By the choice of $A$ and $B$ (consecutive endpoints in $F$) we know this diagonal cannot be in $F_1$ or $F_2$, so it intersects 
none of the segment $CD$ or $EF$. 
Suppose $H$ is a point on the minor arc $BE$, and $H\neq B$. If the diagonal $\ell_B$ in $F$ with endpoint $B$ does not intersect $GH$, 
then (by drawing) it intersects neither $EF$ nor $CD$, contradicting that every line in $F$ intersects either of them. 
Therefore if $GH$ is a selected diagonal then $\ell_B$ must intersect $GH$, meaning that $GH$ is in $F$ (contradicting that $A$ and $B$ are consecutive in $F$). 
Therefore $H$ cannot be the endpoint of the selected diagonal, and similarly, any point on the minor arc $CA$ that is not $A$ cannot be the endpoint of the selected diagonal. Thus that point must be on minor arc $AB$. 
Notice that the case would be trickier if $G$ coincides with $A$ or $B$. 
However, in this case if $A, B, C$ are consecutive points of any diagonal in $F$ in that order, $B$ can be treated as on arc $AC$, thus any diagonal with $B$ as endpoint must have another endpoint either in minor arc $AB$ or in minor arc $BC$. 

To conclude the proof, if $A_{t_1}A_{t_2}\cdots A_{t_k}$ are consecutive endpoints of $F$ then any remaining diagonals must have both endpoints lying on $A_{t_i}A_{t_{i+1}}$ for some $i\in [1,k]$, indices taken modulo $k$. 
Moreover, each endpoints belong to at most two diagonals, with at least four of them belonging exclusively to $l_1$ and $l_2$, the longest lines in $F_1$ and $F_2$ (they intersect every other lines in the opposite family of $F$ so they cannot share endpoint with other lines in $F$). 
Since each diagonal has two endpoints, the number of elements in $F$ cannot exceed $\frac 12(1+1+1+1+2+\cdots +2)$
$=\frac 12 (4+2(k-4))$
$=k-2$. 
Considering the polygon $A_{t_i}\cdots A_{t_{i+1}}$, which is a $t_{i+1}-t_{i}+1$-gon, 
we know that $t_{i+1}-t_{i}-2$ diagonals not intersecting each other can be drawn, plus one line $A_{t_i}A_{t_{i+1}}$ to be drawn. 
This gives $t_{i+1}-t_{i}-1$ lines, resulting in $\displaystyle\sum_{i=1}^k (t_{i+1}-t_i-1)$
$=t_{k+1}-t_1-k$
$=n-k$ 
extra diagonals, 
thus the upper bound is $n-k+k-2=n-2$. 

To achieve this bound, select $A_0A_k$, where $k=\frac n2$, and $A_1A_{n-1}$. 
Now take $A_1A_i$ with $i=3, 4, \cdots , k$ and $A_{n-1}A_i$ with $i=n-3, n-4, \cdots k+1$. 
This gives $2+2(k-2)=2k-2=n-2$. Q.E.D.

%play around with c7 characters
\newcommand{\person}{Feridun }
\newcommand{\pronoun}{he }
\newcommand{\possesive}{his }
\newcommand{\animal}{goose }
\newcommand{\animals}{geese }
\newcommand{\Animal}{Goose }
\newcommand{\Animals}{Geese }
\newcommand{\move}{jump }
\newcommand{\moves}{jumps }

\newpage
\item[\textbf{C7/IMO 6}]
There are $n\ge 2$ line segments in the plane such that every two segments cross and no three segments meet at a point. \person  has to choose an endpoint of each segment and place a \animal on it facing the other endpoint. Then \pronoun will clap \possesive hands $n-1$ times. Every time \pronoun claps, each \animal will immediately \move forward to the next intersection point on its segment. \Animals never change the direction of their jumps. \person wishes to place the \animals in such a way that no two of them will ever occupy the same intersection point at the same time.

(a) Prove that \person can always fulfill \possesive wish if $n$ is odd.

(b) Prove that \person can never fulfill \possesive wish if $n$ is even.

\textbf{Thoughts.} 
We wish to play parity game to prove that the \animals cannot coincide in part (a). 
Intuitively, this makes sense: 
for each point segment $\ell$ there are even number of intersection points, 
which means that for each intersection point we can chacterize one side of having even number of points and the other odd 
(excluding the point itself), giving rise of the notion "even sides" and "odd sides" as below (inspired by the case $n=3$). 
This means for each two lines we can arrange pick one endpoints such that one endpoint is on even side and the other endpoint is on odd side (w.r.t. the intersection). 
What if these selections conflict with each other when $n$ lines are considered together? 
They won't, based on the Menelaus' logic below. 

In part (b), things are trickier: unfortunately we cannot say that the \animals will meet just because both of them come from even side or odd side: the number of intersections must be the same. 
This motivates another version of the solution: 
to find cases where the other lines such that for selected rays of some two selected lines, any other lines either intersect both rays or none of them. 
This means the \animals coming from both rays will intersect (same for the case when the \animals coming not from the rays). 
Denoting the angles of the line to $x$-axis as $\theta_1$ and $\theta_2$, then this happens whenever either all lines have angles in $(\theta_1, \theta_2)$ or all lines have angles not in this range. 
This shows that, if we rank the lines according to their angles, then any two lines neighbouring in ranking must have endpoints at different directions being chosen. 
Don't forget to include the comparison between the first and the last line to arrive at contradiction!
(Notice that this approach works for part (a) too; the intention is to show another soltuion that works uniquely for part (a) which gives an opporunity for some partial marks). 

\textbf{Solution.} 
(a) Let the segments be $\ell_1, \ell_2, \cdots , \ell_n$. 
Let $P_{ij}$ be the intersection of lines $\ell_i$ and $\ell_j$. 
For each $i$ and $j$, consider line $\ell_i$ and point $P_{ij}$; this point separates segment $\ell_i$ into two half-open segment (open at $P_{ij}$, closed at the respective endpoints). 
We wish to investigate the number of points $P_{ix}$ on each of the two open segments (with $x\neq j$). 
Since there are $n-2$ such points (which is odd), 
one segment has even number of points and the other segment odd. 
We call these segments even and odd side of $\ell_i$ w.r.t. point $P_{ij}$, respectively. 
For clarity (and brevity) we even name this attribute as the parity of the open segment of $\ell_i$ w.r.t. point $P_{ij}$. 

Now place the first \animal  on any endpoint on $\ell_1$. For $i\in [2, n]$ we do the following: 
if the \animal corresponding to $\ell_1$ is placed on the odd side of $\ell_1$ w.r.t. $P_{1i}$, 
\person places one \animal at the even side of $\ell_i$ w.r.t. $P_{1i}$ (and vice versa) 
(in other words, the \animals on segments $\ell_i$ and $\ell_1$ are on sides of opposite parity w.r.t. $P_{1i}$). 
Given our choice of placing \animals we proceed to the following claim: 
for each two distinct integers $i, j\in [1, n]$, 
the \animals corresponding to $\ell_i$ and $\ell_j$ lie on different parity w.r.t. $P_{ij}$. 
Before proceeding to the claim, let's note the following observations: 
\begin{itemize}
\item[1.] For each $i, j, k$ with $i\neq j, k$, an endpoint on line $i$ is on a side of different parity w.r.t. $P_{ij}$ and $P_{ik}$ if and only if there are even number of intersection points $P_{il}$ ($l\neq i, j, k$) on the open segment $P_{ij}P_{ik}$.
\item[2.] By Menelaus' theorem, any line intersects none or two of the open segments 
$P_{ij}P_{ki}$, $P_{kj}P_{ki}$, $P_{ij}P_{kj}$ for each $i, j, k$.
\item[3.] Consider all intersection points (with other $\ell$'s) on the triangle segments mentioned in (2.). The total quanity is even.
\end{itemize}
Now let's get back to the claim and for each $(i, j)$, consider the lines $\ell_1$, $\ell_i$, $\ell_j$, the intersection points $P_{1i}$, $P_{1j}$, $P_{ij}$ and the triangle determined by them. We separate in two cases:\\
$\bullet$ Case 1: There are even number of intersection points on open segment $P_{1j}P_{1i}$.\\ 
Each endpoint is on the odd side of $\ell_1$ w.r.t. one of $P_{1j}$ and $P_{1i}$, and even on the other (by observation 1). 
Thus according of our choice of placing the \animals, 
the \animals on $\ell_i$ and $\ell_j$ will come from sides with opposite parity w.r.t. $P_{1i}$ and $P_{1j}$, respectively.
In addition, the number of intersection points $P_{ik}$ on segment $P_{ij}P_{1i}$ and the number of intersection points $P_{jk}$ on segment $P_{ij}P_{1j}$ will be both odd or both even. 
W.L.O.G. assume that this quantity is both odd.
Then by observation 1 the \animal on $\ell_i$ is on the endpoint of a side with same parity w.r.t. $P_{1i}$ and $P_{ij}$, and similarly claim for our \animal on $\ell_j$. 
Nevertheless, from the fact that the selected endpoints on $\ell_i$ and $\ell_j$ are of opposite parity w.r.t. $P_{1i}$ and $P_{1j}$ it follows that the parities w.r.t. $P_{ij}$ will be opposite too. 
(Similar case for the "both even" case, except the parity of the \animal 's side corresponding to $\ell_i$ and $\ell_j$ are both flipped when moved from $P_{1i}$ and $P_{1j}$ to $P_{ij}$). \\
$\bullet$ Case 2: There are odd number of intersection points on open segment $P_{1j}P_{1i}$.\\ 
Now each endpoint of $\ell_1$ is on the side with same parity w.r.t. $P_{1i}$ and $P_{1j}$. 
According to our choice again, 
the \animals on $\ell_i$ and $\ell_j$ will be placed on the sides with same parities w.r.t. $P_{1i}$ and $P_{1j}$, respectively (because the \animal on $\ell_i$ and \animal on $\ell_1$ have different parity w.r.t. $P_{1i}$, and similar to \animal on $\ell_j$).
By observation 3, the number of intersection points on each of segment $P_{1i}P_{ij}$ and segment $P_{1j}P_{ij}$ have opposite parity. 
Thus when comparing the endpoint containing \animal of $\ell_i$ w.r.t. $P_{1i}$ vs $P_{ij}$, and comparing the same for $\ell_j$ w.r.t. $P_{1j}$ vs $P_{ij}$, one pair of comparison has parity swapped and the other has parity preserved.  
It thus follows that the endpoints of $\ell_i$ and $\ell_j$ are on the sides with opposite parities w.r.t. $P_{ij}$.  

Finally, suppose that two \animals intersect on $P_{ij}$ for some $i, j$, then they must have come from the sides of same parity of $\ell_i$ and $\ell_j$ w.r.t. $P_{ij}$, contradicting our mega-claim above. 

(b) Let $\ell_1, \ell_2, \cdots , \ell_n$ be the segments, and let $\theta_i\in [0, \pi)$ be the angle need for $x$-axis to rotate counterclockwise to reach $\ell_i$. 
W.L.O.G let $0=\theta_1 < \theta_2< \cdots < \theta_n$. For convinience we introduce $\ell_{n+1}=\ell_1$ with $\theta_{n+1}=\pi$. 

For each segment \person has the choice of placing the \animal in the direction of $\theta_i$ or $\theta_i +\pi$ compared to the positive $x$-direction. Let's investigate $\ell_i$ and $\ell_{i+1}$ together. 
Let's reuse the notation $P_{ij}$ from part (a). 
We know that for each $j\not\in\{i, i+1\}$, $\ell_i$, $\ell_{i+1}$ and $\ell_j$ are in that order (in anticlockwise angle, cycles allowed). 
This means that $P_{ij}P_{i(i+1)}, P_{i(i+1)}P_{(i+1)j}, P_{(i+1)j}P_{ij}$ must also be in that order, 
forcing $P_{ij}, P_{i(i+1)}, P_{(i+1)j}$ to be in clockwise order. 
From here we infer that the vectors $P_{i(i+1)}P_{ij}$ and $P_{i(i+1)}P_{(i+1)j}$ either have directions ($\theta_i$, $\theta_{i+1}$) or ($\theta_i +\pi$, $\theta_{i+1} +\pi$), 
and considering all such $j$'s, we know that there are equal number of intersection points lying on the half-line starting from $P_{i(i+1)}$ and extending in the $\theta_i$ direction, and on the half-line starting from $P_{i(i+1)}$ and extending in the $\theta_{i+1}$ direction. 
This means that the \animals will collide when both placed in the $\theta_i, \theta_{i+1}$ direction, or $\theta_i +\pi, \theta_{i+1}+\pi$ direction, 
forcing the directions to be  $\theta_i, \theta_{i+1}+\pi$ or  $\theta_i +\pi, \theta_{i+1}$. 

Summarizing above, if we let $\theta_1$ to be the direction headed by the first \animal
the directions must be $\theta_1, \theta_2+\pi, \theta_3, \theta_4+\pi, \cdots , \theta_n+\pi$ ($n$ is even). 
Recall that we can also compare it with the "$n+1$-th" line (which is the first) and it has to have direction $\theta_{n+1}=\theta_1 +\pi$, 
which contradicts our choice of making the first \animal facing the direction $\theta_1$. 
(The case where $\theta_1+\pi$ is chosen as the direction is completely analogous: if one configuration works, then the completely opposite configuration works too). 
\end{itemize}

\newpage
\section{Geometry}
\begin{itemize}
\item[\textbf{G1/IMO 1}]
Triangle $BCF$ has a right angle at $B$. Let $A$ be the point on line $CF$ such that $FA=FB$ and $F$ lies between $A$ and $C$. Point $D$ is chosen so that $DA=DC$ and $AC$ is the bisector of $\angle{DAB}$. Point $E$ is chosen so that $EA=ED$ and $AD$ is the bisector of $\angle{EAC}$. Let $M$ be the midpoint of $CF$. Let $X$ be the point such that $AMXE$ is a parallelogram. Prove that $BD,FX$ and $ME$ are concurrent.

\textbf{Thoughts.} Right angle at $B$, and midpoint of $CF$ is given in the problem. What else could be better than drawing the circumcircle of $BCF$? And that's pretty much all we need for idea generation: as you draw the diagram you will soon realize tha $D$ and $X$ both lie on this circumcircle, and that $ED$ is parallel to $CF$ (which practically means that $M, D, E$ are collinear). The rest of the job is to prove the claims above (and there are many ways to do it; I have found two--one as detailed below, the other one using trigonometric bashing). And finally the use of the collinearity of $B, F, E$ and the fact that $BFDX$ is isoceles trapezoid is just one of the many possible ways to finish the solution. 

\definecolor{uuuuuu}{rgb}{0.26666666666666666,0.26666666666666666,0.26666666666666666}
\definecolor{xdxdff}{rgb}{0.6588235294117647,0.6588235294117647,0.6588235294117647}
\definecolor{qqqqff}{rgb}{0.3333333333333333,0.3333333333333333,0.3333333333333333}
\begin{tikzpicture}[line cap=round,line join=round,>=triangle 45,x=1.0cm,y=1.0cm]
\clip(-3.0588481536904415,-4.087465088698609) rectangle (15.949416309119467,4.871212597251791);
\draw [line width=1.2pt] (3.23,-0.59) circle (3.0104152537482274cm);
\draw [line width=1.2pt,dash pattern=on 4pt off 4pt] (6.837019243486866,0.6946592960642324) circle (3.8289603458182753cm);
\draw [line width=1.2pt] (1.1141635269103165,-2.7314565181538035)-- (8.283544019011412,-2.850549217358141);
\draw [line width=1.2pt] (4.798013646776459,-2.79265004173962) -- (4.696794805481305,-2.914952710025465);
\draw [line width=1.2pt] (4.798013646776459,-2.79265004173962) -- (4.700912740440424,-2.6670530254864793);
\draw [line width=1.2pt] (0.22,-0.54)-- (10.399380492101093,-0.709092699204337);
\draw [line width=1.2pt] (5.408850119866141,-0.6261935235858166) -- (5.307631278570987,-0.7484961918716619);
\draw [line width=1.2pt] (5.408850119866141,-0.6261935235858166) -- (5.311749213530106,-0.5005965073326759);
\draw [line width=1.2pt] (10.399380492101093,-0.709092699204337)-- (8.283544019011412,-2.850549217358141);
\draw [line width=1.2pt] (9.423485785234028,-1.8143911283912397) -- (9.305907403106831,-1.6982194353203615);
\draw [line width=1.2pt] (9.377017108005676,-1.8614224812421178) -- (9.25943872587848,-1.7452507881712396);
\draw [line width=1.2pt] (3.4161205556724505,2.414656243026177)-- (10.399380492101093,-0.709092699204337);
\draw [line width=1.2pt] (3.23,-0.59)-- (1.1141635269103165,-2.7314565181538035);
\draw [line width=1.2pt] (0.22,-0.54)-- (1.1141635269103165,-2.7314565181538035);
\draw [line width=1.2pt] (3.4161205556724505,2.414656243026177)-- (6.24,-0.64);
\draw [line width=1.2pt] (4.88874626384298,0.9434293305556646) -- (4.767374291829472,0.8312269124705128);
\draw [line width=1.2pt] (6.24,-0.64)-- (1.1141635269103165,-2.7314565181538035);
\draw [line width=1.2pt] (3.4161205556724505,2.414656243026177)-- (5.273544019011411,-2.800549217358141);
\draw [line width=1.2pt,dash pattern=on 4pt off 4pt] (3.23,-0.59)-- (8.283544019011412,-2.850549217358141);
\draw [line width=1.2pt] (6.24,-0.64)-- (10.399380492101093,-0.709092699204337);
\draw [line width=1.2pt] (8.321062891036922,-0.5919131214225067) -- (8.318317601064173,-0.7571795777818301);
\draw [line width=1.2pt] (3.23,-0.59)-- (6.24,-0.64);
\draw [line width=1.2pt] (0.22,-0.54)-- (3.23,-0.59);
\draw [line width=1.2pt] (5.273544019011411,-2.800549217358141)-- (8.283544019011412,-2.850549217358141);
\draw [line width=1.2pt] (6.746863372725922,-2.74236693118393) -- (6.744118082753174,-2.907633387543253);
\draw [line width=1.2pt] (6.812969955269651,-2.743465047173029) -- (6.810224665296903,-2.908731503532352);
\draw [line width=1.2pt] (6.24,-0.64)-- (8.283544019011412,-2.850549217358141);
\begin{scriptsize}
\draw [fill=uuuuuu] (0.22,-0.54) circle (2.0pt);
\draw[color=uuuuuu] (0.32958159837567247,-0.2610188077068794) node {$C$};
\draw [fill=uuuuuu] (6.24,-0.64) circle (2.0pt);
\draw[color=uuuuuu] (6.36263944961534,-0.36019236142588756) node {$F$};
\draw [fill=uuuuuu] (3.4161205556724505,2.414656243026177) circle (2.0pt);
\draw[color=uuuuuu] (3.4361931686236048,2.681129952623695) node {$B$};
\draw [fill=uuuuuu] (10.399380492101093,-0.709092699204337) circle (2.0pt);
\draw[color=uuuuuu] (10.51139978019385,-0.44283698952506095) node {$A$};
\draw [fill=uuuuuu] (5.273544019011411,-2.800549217358141) circle (2.0pt);
\draw[color=uuuuuu] (5.387432838045092,-2.5254816176242314) node {$D$};
\draw [fill=uuuuuu] (3.23,-0.59) circle (2.0pt);
\draw[color=uuuuuu] (3.3378460611855885,-0.3106055845663835) node {$M$};
\draw [fill=uuuuuu] (8.283544019011412,-2.850549217358141) circle (2.0pt);
\draw[color=uuuuuu] (8.395697300855009,-2.575068394483736) node {$E$};
\draw [fill=uuuuuu] (1.1141635269103165,-2.7314565181538035) circle (2.0pt);
\draw[color=uuuuuu] (1.2221435818467465,-2.459365915144893) node {$X$};
\draw [fill=uuuuuu] (5.345836473089689,1.5514565181538016) circle (2.0pt);
\draw[color=uuuuuu] (5.453548540524431,1.821625820392291) node {$G$};
\end{scriptsize}
\end{tikzpicture}

\textbf{Solution.} 
Let $\Gamma$ be the circumcircle of triangle $BCF$, which we know that $M$ is its center and $CF$ its diameter. 
Denote $G$ as the intersection of $AB$ and $\Gamma$ other than $B$ (with $G\neq B$ unless $AB$ is tangent to $\Gamma$). 
Since $FA=FB$, we have 
$\angle GCA=\angle GCF=\angle GBF=\angle ABF=\angle BAF=\angle GAC$, hence $GA=GC$. 
Now, denote $D'$ as the reflection of $G$ in $AC$, and we know that $D'$ lies on $\Gamma$. 
Notice also that $AD'=AG=CG=CD'$, and $\angle BAC=\angle GAC=\angle D'AC$, so $D'$ fulfills $D'A=D'B$ and $AC$ is the bisector of $\angle D'AB$. 
We therefore have $D=D'$ since there is only one such point fulfilling such property (i.e. the intersection of perpendicular bisector of $AC$ and the reflection of $AB$ in $CF$, which cannot be the same unless $AB$ is parallel or perpendicular to $AC$, which forces triangle $FAB$ and $BCF$ to be degenerate because line $AC$ will coincide with $FB$). 
Now that we established that $D$ is on $\Gamma$, we claim that $MDEA$ is an isoceles trapezoid. 
Indeed, $\angle AMD=\angle FMD=\angle FMG=2\angle GCF=2\angle GAF=2\angle CAD=\angle CAE=\angle MAE$, 
and $\angle DEA=180^{\circ}-\angle DAE-\angle ADE=180^{\circ}-2\angle DAE=180^{\circ}-\angle MAE=180^{\circ}-\angle AMD$ 
(proven). This also gives $AC\parallel ED$, and $D, E, X$ collinear. Moreover $\angle MXD=\angle MXE=\angle MAE=\angle MDX$ so $MD=MX$ and $X$ is on $\Gamma$. Therefore $CFDX$ is also an isoceles trapezoid. 

To finish up the solution, we claim that $EM$ is the perpendicular bisector of both $DF$ and $BX$. 
Indeed, $M$ is on the perpendicular bisector of the two lines because $MD=MF=MB=MX$ (all four points lie on $\Gamma$). 
Now, notice that $\angle BMD+\angle BAD=2\angle BCD+2\angle BAC=2(\angle BCF+\angle DCF+\angle BAC)$
$=2(\angle BCF+\angle GCF+\angle GAC)=2(\angle BCF+\angle GAF+\angle GAC)=2(\angle BCF+ \angle BFC)=2(90^{\circ})=180^{\circ}$, 
meaning that $B, D,  M, A$ are concylic. 
Nevertheless, knowing that $AMDE$ is isoceles trapezoid, $E$ lies on this circle too. 
Thus $\angle BED=\angle BAD=2\angle BAM=\angle BAF+\angle ABF=\angle BFM$, and coupled with the fact that $AC\parallel EX$ we have $B, F, E$ collinear. Finally, $\angle BFD=\angle BFC+\angle CFD=\angle BFC+\angle CFG=2\angle GAC+90^{\circ}-\angle GCF$
$=2\angle GAC+90^{\circ}-\angle GAF$
$=90^{\circ}+\angle GAC$
$=90^{\circ}+\angle GCF$
$=90^{\circ}+(90^{\circ}-\angle CFG)$
$=180^{\circ}-\angle CFG$
$=180^{\circ}-\angle CFD$
$=\angle FDX$, 
showing that $BFDX$ is also an isoceles trapezoid. With $BF$ and $DX$ intersecting at $E$, we conclude that $EM$ is the perpendicular bisector of both $DF$ and $BX$, and $DB$ and $FX$ will intersect on this perpendicular bisector too. 

\newpage
\item[\textbf{G2}]
Let $ABC$ be a triangle with circumcircle $\Gamma$ and incenter $I$ and let $M$ be the midpoint of $\overline{BC}$. The points $D$, $E$, $F$ are selected on sides $\overline{BC}$, $\overline{CA}$, $\overline{AB}$ such that $\overline{ID} \perp \overline{BC}$, $\overline{IE}\perp \overline{AI}$, and $\overline{IF}\perp \overline{AI}$. Suppose that the circumcircle of $\triangle AEF$ intersects $\Gamma$ at a point $X$ other than $A$. Prove that lines $XD$ and $AM$ meet on $\Gamma$.

\textbf{Thoughts. }The first thing that must naturally jump out of your mind is the spiral similarity between $XFB$ and $XEC$ (come on, two circles intersecting at $A$ and $X$). All's remaining is to trigonometric-bash the problem. 

\definecolor{zzttqq}{rgb}{0.6,0.2,0.}
\definecolor{uuuuuu}{rgb}{0.26666666666666666,0.26666666666666666,0.26666666666666666}
\begin{tikzpicture}[line cap=round,line join=round,>=triangle 45,x=1.0cm,y=1.0cm]
\clip(-2.270656607672633,-3.796329905018268) rectangle (18.226000561693667,5.8638337348048095);
\fill[line width=1.2pt,color=zzttqq,fill=zzttqq,fill opacity=0.10000000149011612] (2.6740633796648576,3.4476258129586164) -- (2.42,-1.08) -- (3.0332828475238647,0.5873608084337415) -- cycle;
\fill[line width=1.2pt,color=zzttqq,fill=zzttqq,fill opacity=0.10000000149011612] (2.6740633796648576,3.4476258129586164) -- (7.861779119227031,1.5355684211039) -- (10.16,-0.98) -- cycle;
\draw [line width=1.2pt] (6.264092589154242,0.9752335994617323) circle (4.359017433131806cm);
\draw [line width=1.2pt] (5.201874941490228,2.312402981551107) circle (2.771021863689802cm);
\draw [line width=1.2pt] (6.907124929188343,-3.3360936198545272)-- (2.6740633796648576,3.4476258129586164);
\draw [line width=1.2pt] (4.667907423661546,5.031491221927057)-- (6.907124929188343,-3.3360936198545272);
\draw [line width=1.2pt] (4.667907423661546,5.031491221927057)-- (2.42,-1.08);
\draw [line width=1.2pt] (2.42,-1.08)-- (10.16,-0.98);
\draw [line width=1.2pt] (4.667907423661546,5.031491221927057)-- (10.16,-0.98);
\draw [line width=1.2pt] (3.0332828475238647,0.5873608084337415)-- (7.861779119227031,1.5355684211039);
\draw [line width=1.2pt] (5.447530983375448,1.061464614768821)-- (5.474688584848092,-1.040533739213849);
\draw [line width=1.2pt] (4.667907423661546,5.031491221927057)-- (5.447530983375448,1.061464614768821);
\draw [line width=1.2pt] (5.447530983375448,1.061464614768821)-- (2.42,-1.08);
\draw [line width=1.2pt] (5.447530983375448,1.061464614768821)-- (10.16,-0.98);
\draw [line width=1.2pt] (2.6740633796648576,3.4476258129586164)-- (2.42,-1.08);
\draw [line width=1.2pt] (2.6740633796648576,3.4476258129586164)-- (3.0332828475238647,0.5873608084337415);
\draw [line width=1.2pt] (2.6740633796648576,3.4476258129586164)-- (7.861779119227031,1.5355684211039);
\draw [line width=1.2pt] (2.6740633796648576,3.4476258129586164)-- (10.16,-0.98);
\draw [line width=1.2pt,color=zzttqq] (2.6740633796648576,3.4476258129586164)-- (2.42,-1.08);
\draw [line width=1.2pt,color=zzttqq] (2.42,-1.08)-- (3.0332828475238647,0.5873608084337415);
\draw [line width=1.2pt,color=zzttqq] (3.0332828475238647,0.5873608084337415)-- (2.6740633796648576,3.4476258129586164);
\draw [line width=1.2pt,color=zzttqq] (2.6740633796648576,3.4476258129586164)-- (7.861779119227031,1.5355684211039);
\draw [line width=1.2pt,color=zzttqq] (7.861779119227031,1.5355684211039)-- (10.16,-0.98);
\draw [line width=1.2pt,color=zzttqq] (10.16,-0.98)-- (2.6740633796648576,3.4476258129586164);
\begin{scriptsize}
\draw [fill=uuuuuu] (4.667907423661546,5.031491221927057) circle (2.0pt);
\draw[color=uuuuuu] (4.537798208586435,5.46281218149112) node {$A$};
\draw [fill=uuuuuu] (2.42,-1.08) circle (2.0pt);
\draw[color=uuuuuu] (2.024729807820653,-1.2921730943261232) node {$B$};
\draw [fill=uuuuuu] (10.16,-0.98) circle (2.0pt);
\draw[color=uuuuuu] (10.383801296892647,-1.0782949325588227) node {$C$};
\draw [fill=uuuuuu] (5.447530983375448,1.061464614768821) circle (2.0pt);
\draw[color=uuuuuu] (5.571542657128387,1.3634807476178594) node {$I$};
\draw [fill=uuuuuu] (5.474688584848092,-1.040533739213849) circle (2.0pt);
\draw[color=uuuuuu] (5.357664495361086,-1.2921730943261232) node {$D$};
\draw [fill=uuuuuu] (7.861779119227031,1.5355684211039) circle (2.0pt);
\draw[color=uuuuuu] (8.155903778483268,1.6308284498269852) node {$E$};
\draw [fill=uuuuuu] (3.0332828475238647,0.5873608084337415) circle (2.0pt);
\draw[color=uuuuuu] (2.648541112975279,0.7931389829050577) node {$F$};
\draw [fill=uuuuuu] (2.6740633796648576,3.4476258129586164) circle (2.0pt);
\draw[color=uuuuuu] (2.1851384291461278,3.698317346910891) node {$X$};
\draw [fill=uuuuuu] (6.29,-1.03) circle (2.0pt);
\draw[color=uuuuuu] (6.409232124050314,-0.7396545097605967) node {$M$};
\draw [fill=uuuuuu] (6.907124929188343,-3.3360936198545272) circle (2.0pt);
\draw[color=uuuuuu] (7.050866609352216,-2.8071434068445025) node {$N$};
\end{scriptsize}
\end{tikzpicture}

\textbf{Solution.}
W.L.O.G. let $AB<AC$. 
First, that $\angle XBF=\angle XCE$ and $\angle BXF=\angle BXC-\angle FXC=\angle BAC-\angle FXC=\angle FXE-\angle FXC=\angle CXE$ should dictate the similarity of triangles $BXF$ an $CXE$, so $\frac{CX}{CE}=\frac{BX}{BF}$. 
In addition we have following: 
\[\frac{BX}{XC}\cdot \frac{\sin\angle BXD}{\sin\angle CXD}
=\frac{BX}{XC}\cdot \frac{\sin\angle BXD}{\sin\angle CXD}\cdot\frac{DX}{DX}
=\frac{2|\triangle BXD|}{2|\triangle CXD|}
=\frac{BD}{DC}\cdots (*).\]
Similarly we have $\frac{AB}{AC}\cdot \frac{\sin\angle ABM}{\sin\angle ACM}=\frac{BM}{CM}=1\to \frac{\sin\angle ABM}{\sin\angle ACM}=\frac{AC}{AB}$.

Denoting $N_1$ as the other intersection of $XD$ and $\Gamma$ gives $\frac{\sin\angle BXD}{\sin\angle CXD}=\frac{BN_1}{CN_1}.$
Also let $N_2$ as the other intersection of $AM$ and $\Gamma$ and we have $\frac{\sin\angle ABM}{\sin\angle ACM}=\frac{BN_2}{CN_2}=\frac{AC}{AB}.$
Therefore all we need is $\frac{\sin\angle ABM}{\sin\angle ACM}=\frac{\sin\angle BXD}{\sin\angle CXD}$. 
By (*), 
this is equivalent to
$\frac{BF}{EC}\cdot\frac{AC}{AB}=\frac{BD}{DC}$.

Now, $\frac{BD}{DC}=\frac{\tan\frac12\angle C}{\tan\frac12\angle B}$,
$\frac{AC}{AB}=\frac{\sin\angle B}{\sin\angle C}=\frac{2\sin\frac 12\angle B\cos\frac 12\angle B}{2\sin\frac 12\angle C\cos\frac 12\angle C}$.
Also $IE=IF$, and by angle chasing we have $\angle FIB=\angle ICE=\frac12\angle C$,
$\angle EIC=\angle IBF=\frac12\angle B$.
Therefore $BIF$ and $ICE$ similar, yielding $\frac{BF}{EC}=\left(\frac{BF}{FI}\right)^2=\left(\frac{\sin\frac12\angle C}{\sin\frac12\angle B}\right)^2$,
now it's no longer difficult to prove that $\left(\frac{\sin\frac12\angle C}{\sin\frac12\angle B}\right)^2\cdot \frac{2\sin\frac 12\angle B\cos\frac 12\angle B}{2\sin\frac 12\angle C\cos\frac 12\angle C}=\frac{\tan\frac12\angle C}{\tan\frac12\angle B}$.

\newpage

\item[\textbf{G3}]
Let $B = (-1, 0)$ and $C = (1, 0)$ be fixed points on the coordinate plane. A nonempty, bounded subset $S$ of the plane is said to be nice if

$\text{(i)}$ there is a point $T$ in $S$ such that for every point $Q$ in $S$, the segment $TQ$ lies entirely in $S$; and

$\text{(ii)}$ for any triangle $P_1P_2P_3$, there exists a unique point $A$ in $S$ and a permutation $\sigma$ of the indices $\{1, 2, 3\}$ for which triangles $ABC$ and $P_{\sigma(1)}P_{\sigma(2)}P_{\sigma(3)}$ are similar.

Prove that there exist two distinct nice subsets $S$ and $S'$ of the set $\{(x, y) : x \geq 0, y \geq 0\}$ such that if $A \in S$ and $A' \in S'$ are the unique choices of points in $\text{(ii)}$, then the product $BA \cdot BA'$ is a constant independent of the triangle $P_1P_2P_3$.

\textbf{Thoughts.} 
By letting $P_1P_2P_3$ equilateral we know that the only viable point $A$ with $ABC$ equilateral and with non-negative coordinates for $A$ is $(0, \sqrt{3})$. This also forces $BA=2$, and if for set $S$ and $S'$, the product $BA\cdot BA'$ is fixed across all such triangles $P_1P_2P_3$, then we must have $BA\cdot BA'=2\times 2=4=BC^2$. 

Now this gives us an intuition on how $S$ and $S'$ *might* be constructed: if $A, A', B$ are collinear, $A$ and $A'$ are lying on the same side as $B$, and $BA\cdot BA'=BC^2$, then triangles $BCA$ and $BA'C$ are similar. What if...we try out $S$ containing all points $A$ with $BA\le 2$ (and having non-negative coordinates), and $S'$ containing the points $A'$ such that for each point $A$ in $S$ we have $BA\cdot BA'=4$. That would work, but again we need to verify it. Notice that a sensible question to ask would be: what exactly is $S'$ (given the condition above)? Now that $BA\le 2$ for $A\in S$, we know that it is bounded by a circle, and the line $y=0$, which means that $S'$ has something to do with the inversion of $y$-axis w.r.t. the circular arc $X: BX=2$ and radius 2. And yeah, $S\cup S'$ is precisely the part of this image of inversion that has nonnegative coordinates.

\definecolor{uuuuuu}{rgb}{0.26666666666666666,0.26666666666666666,0.26666666666666666}
\begin{tikzpicture}[line cap=round,line join=round,>=triangle 45,x=2.0cm,y=2.0cm]
\clip(-1.7992213964115553,-0.5528722288194026) rectangle (4.3785964850163905,2.3587601987753266);
\draw [line width=1.2pt] (0.,1.7320508075688772)-- (0.,0.);
\draw [line width=1.2pt] (0.,0.)-- (3.,0.);
\draw [shift={(-1.,0.)},line width=1.2pt]  plot[domain=0.:1.0471975511965976,variable=\t]({1.*2.*cos(\t r)+0.*2.*sin(\t r)},{0.*2.*cos(\t r)+1.*2.*sin(\t r)});
\draw [shift={(1.,0.)},line width=1.2pt]  plot[domain=0.:2.0943951023931957,variable=\t]({1.*2.*cos(\t r)+0.*2.*sin(\t r)},{0.*2.*cos(\t r)+1.*2.*sin(\t r)});
\draw [line width=1.2pt] (-1.,0.)-- (0.,0.);
\draw [line width=1.2pt,dash pattern=on 3pt off 3pt] (-1.,0.)-- (2.2421749535956774,1.567482498996266);
\draw [line width=1.2pt] (0.4643237490478057,0.70795126183782)-- (1.,0.);
\draw [line width=1.2pt] (1.2141107495551722,1.0704480481283867)-- (1.,0.);
\draw (0.23140048113606526,1.2521249956673668) node[anchor=north west] {$S$};
\draw (1.0103427357508932,1.6281660841021104) node[anchor=north west] {$S'$};
\begin{scriptsize}
\draw [fill=uuuuuu] (-1.,0.) circle (2.0pt);
\draw[color=uuuuuu] (-1.127719452778083,0.008503396058179014) node {$B$};
\draw [fill=uuuuuu] (1.,0.) circle (2.0pt);
\draw[color=uuuuuu] (0.967366611358351,-0.13654102376665067) node {$C$};
\draw [fill=uuuuuu] (3.,0.) circle (2.0pt);
\draw[color=uuuuuu] (3.1376608931817334,-0.10968094602131184) node {$T'$};
\draw [fill=uuuuuu] (0.,0.) circle (2.0pt);
\draw[color=uuuuuu] (-0.010340218571984802,-0.11505296157037961) node {$F$};
\draw [fill=uuuuuu] (0.4643237490478057,0.70795126183782) circle (2.0pt);
\draw[color=uuuuuu] (0.44090908754970864,0.8196777439674117) node {$A$};
\draw [fill=uuuuuu] (1.2141107495551722,1.0704480481283867) circle (2.0pt);
\draw[color=uuuuuu] (1.209107311066401,1.1957188324021553) node {$A'$};
\end{scriptsize}
\end{tikzpicture}

\textbf{Solution.} We show that the following works: $S=\{(x,y): x\ge 0, y\ge 0, (x+1)^2+y^2\le 4\}$ and $S'=\{x\ge 0, y\ge 0, (x+1)^2+y^2\ge 4, (x-1)^2+y^2\le 4\}$. We claim that $BA\cdot BA'=4$ for those choices. 
Denote triangles $ABC$ and $DEF$ as \emph{quasi-similar} if there exists a permutation $\sigma$ of $\{D,E,F\}$ with $ABC$ and $\sigma(D)\sigma(E)\sigma(F)$ as similar. 

We first start with the following claim: 
for every point $A$ above $x$-axis, $A\in S$ iff $AC\le AB\le BC$ and $A'\in S'$ iff $A'C\le BC\le A'B$. \\
Proof: let's first investigate all points $A\in S$ and $A'\in S'$. 
Let $A=(x,y)$ be arbitrary. 
Now, $AC\le AB\lra(x-1)^2+y^2\le  (x+1)^2+y^2\lra 0\le x$. 
$BC\ge AB\lra 2^2\ge (x+1)^2+y^2\lra (x+1)^2+y^2\le 4$. 
Therefore \[AC\le AB\le BC \lra (x,y) \text{ satisfies both } x\ge 0 \text{ and } (x+1)^2+y^2\le 4 \lra (x,y)\in S.\] 
Next, $AC\le BC$ iff 
$AC\le 2$ or $(x-1)^2+y^2\le 4$ 
and $BC\le AB$ iff
$(x-1)^2+y^2\ge 4$, 
therefore $AC\le BC\le AB$ iff the two conditions are satisfied (and from here $(x-1)^2+y^2\le (x+1)^2+y^2$ so $x\ge 0$ is implied), 
and this is equivalent to $A\in S'$. 

We are now ready to justify our selection: 
\begin{itemize}
\item[(i)]
In $S$ we can simply take any point as $T$, since the boundaries $x$-axis with $x\in [0, \sqrt{3}]$, $y$-axis with $y\in [0, 1], $ and the arc of the circle $(x-1)^2+y^2=4$ with $x\ge 0$ are convex. 
In $S'$ we take $T'=(3,0)$. 
Let $Q'$ be in $S'$ and we want to show that the whole segment $T'Q'$ lies on $S'$. 
Consider the region $S\cup S'$, i.e. the region bounded by $x-$ and $y-$ axes, together with the circular arc of the circle $(x-1)^2+y^2=4$. 
This region is also convex, so with $T', Q'$ lying in the region $S\cup S'$, the entire segment is in this region too. 
The aim is therefore to show that no point of the segment lies in the interior of $S$ or on the $y$-axis. 
Suppose on the contrary we have $R\in S$ belonging in the segment $T'Q'$. 
From the convexity of the region $S\cup S'$ we only need to consider the part of the line $T'Q'$ lying in this region. 
With $R$ lying in $S$, we know that segment $T'Q'$ intersects the boundary separating $S$ and $S'$ (i.e. the circular arc) at least once, 
and since $Q'$ is outside (or on the boundary of) the region $S$, it must intersect the boundary for another time, 
entailing the fact that this segment has to intersect the circular arc for exactly twice. 
Let $X$ and $X'$ be the intersections, and we have 
$\angle BX'T'=180^{\circ}-\angle BXT'$. 
Denoting $X'$ as the further point from $T'$ and we have $\angle BX'T'<90^{\circ}$. 
Now consider the point $A=(0, \sqrt{3})$, and $A$ is on both $S$ and $S'$, which we have $\angle BAT'=90^{\circ}$. 
Therefore all $X$ on the arc must sastisfying $\angle BXT'\ge \angle BAT'=90^{\circ}$ since $X$ lies inside the triangle $BAT'$ 
(contradiction).

\item[(ii)] Observe that the objective is equivalent to: 
for each triangle $P_1P_2P_3$ there is a unique point $A$ with triangle $ABC$ quasisimilar to $P_1P_2P_3$ satisfying $AC\le AB\le BC$, 
and another unique point $A'$ with $A'BC$ quasisimilar to $P_1P_2P_3$ and $A'C\le BC\le A'B$. 
Moreover we want to prove that $BA\cdot BA'=BC^2=4$. 
We will also use the fact that for each triangle $DEF$ there is a unique $A$ above $x$-axis that is similar to $DEF$ (with $B, C$ fixed). 
We split into the following cases: 
\begin{itemize}
\item [Case 1.] $P_1P_2P_3$ equilateral. The only point $A$ with $y\ge 0$ satisfying this is 
$A=(0, \sqrt{3})$. 
Since we have $(x+1)^2+y^2=(x-1)^2+y^2=1+3=4$, $A$ lies in both $S$ and $S'$ and we have $BA\cdot BA=2\times 2=4$. 
\item [Case 2.] $P_1P_2P_3$ is isoceles, with the two equal sides longer than the other. 
Now, let $P_1P_2=P_1P_3>P_2P_3$. 
This means, if $AC$ is the shortest and if triangle $ABC$ is quasisimilar to $P_1P_2P_3$ then $AC$ corresponds to $P_2P_3$, 
and $AB, BC$ correspond to $P_1P_2$ and $P_1P_3$, which implies that 
$AB=BC=2$. 
Such $A$ can be uniquely constructed, and with $AC < AB=BC=2$ we have $A$ lies in $S$. 
Similarly, if $A'C$ is the shortest side of $A'BC$ and if triangle $A'BC$ is quasisimilar to $P_1P_2P_3$ then $A'C$ corresponds to $P_2P_3$, and $A'B, BC$ corresponds to $P_1P_2$ and $P_1P_3$, so 
$A'$ can also be uniquely constructed (which turns out to be equal to $A$ in this case). 
Therefore, $A'C < BC=A'B=2$, which implies that $A'$ is in $S'$, and moreover $BA\cdot BA'=2\times 2=4$. 

\item [Case 3.] $P_1P_2P_3$ is isoceles with the two equal sides longer than the other. 
Now, let $P_1P_2=P_1P_3<P_2P_3$. 
This means $AC$ and $A'C$ corresponds to $P_1P_2$. 
In $ABC$, we know that $AB\le BC$ implies $AB$ corresponds to $P_1P_3$, $BC$ corresponds to $P_2P_3$;
In $ABC$, we know that $BC\le A'B$ implies $BC$ corresponds to $P_1P_3$, $A'B$ corresponds to $P_2P_3$. 
Therefore $\frac{AB}{BC}=\frac{P_1P_3}{P_2P_3}=\frac{BC}{A'B}$ and $BC^2=AB\cdot A'B$. 

\item [Case 4.] $P_1P_2P_3$ scalene, and let $P_1P_2<P_1P_3<P_2P_3$. 
This means $AC$ and $A'C$ corresponds to $P_1P_2$. 
In $ABC$, we know that $AB\le BC$ implies $AB$ corresponds to $P_1P_3$, $BC$ corresponds to $P_2P_3$;
In $ABC$, we know that $BC\le A'B$ implies $BC$ corresponds to $P_1P_3$, $A'B$ corresponds to $P_2P_3$. 
Therefore $\frac{AB}{BC}=\frac{P_1P_3}{P_2P_3}=\frac{BC}{A'B}$ and $BC^2=AB\cdot A'B$. 
\end{itemize}
\end{itemize}

\newpage

\item[\textbf{G4}]
Let $ABC$ be a triangle with $AB = AC \neq BC$ and let $I$ be its incentre. The line $BI$ meets $AC$ at $D$, and the line through $D$ perpendicular to $AC$ meets $AI$ at $E$. Prove that the reflection of $I$ in $AC$ lies on the circumcircle of triangle $BDE$.

\textbf{Thoughts.} Focusing on the circumcircle of $BDE$ (as per suggested by the problem) is itself a pain because there isn't much information we can take home regarding the point $I'$. Instead, the notion of reflection should immediately let us think of angle bisection: that $AC$ bisects $\angle BDI'$. As $DE$ is perpendicular to $AC$, line $DE$ bisects the same angle too. This realization should let us focus on the circumcircle of $BDI'$ instead: we know that $E$ lies on this circumcircle if and only if $BE=EI'$. Now the equality $BE=EI'$ is something easier to establish; we chose the method of trigonometry and introducing the point $I''$ (although direct computation *should* work too). 

\definecolor{uuuuuu}{rgb}{0.26666666666666666,0.26666666666666666,0.26666666666666666}
\begin{tikzpicture}[line cap=round,line join=round,>=triangle 45,x=1.0cm,y=1.0cm]
\clip(2.508533677869763,-3.670896102057754) rectangle (20.155064472682568,4.645990585584426);
\draw [line width=1.2pt] (6.3,0.64) circle (3.7402481873228917cm);
\draw [line width=1.2pt] (6.33659146518181,4.380069192869033)-- (3.9870541679262246,-2.299343137618459);
\draw [line width=1.2pt] (6.33659146518181,4.380069192869033)-- (8.554993680980937,-2.3440341823625643);
\draw [line width=1.2pt] (3.9870541679262246,-2.299343137618459)-- (8.554993680980937,-2.3440341823625643);
\draw [line width=1.2pt] (3.9870541679262246,-2.299343137618459)-- (9.083041002455232,1.2728647515940277);
\draw [line width=1.2pt] (6.33659146518181,4.380069192869033)-- (6.287016087198279,-0.6871054337761034);
\draw [shift={(8.67788250632608,-5.057302928688166)},line width=1.2pt,dash pattern=on 4pt off 4pt]  plot[domain=1.4392311051423707:2.610082583237681,variable=\t]({1.*5.441526689219892*cos(\t r)+0.*5.441526689219892*sin(\t r)},{0.*5.441526689219892*cos(\t r)+1.*5.441526689219892*sin(\t r)});
\draw [line width=1.2pt] (6.292107838093549,-0.16666984677409122)-- (6.53504758519073,-0.5132391930122152);
\begin{scriptsize}
\draw [fill=uuuuuu] (3.9870541679262246,-2.299343137618459) circle (2.0pt);
\draw[color=uuuuuu] (3.7873931802595675,-2.4783712991910054) node {$B$};
\draw [fill=uuuuuu] (6.33659146518181,4.380069192869033) circle (2.0pt);
\draw[color=uuuuuu] (6.446073025680614,4.644264829899929) node {$A$};
\draw [fill=uuuuuu] (8.554993680980937,-2.3440341823625643) circle (2.0pt);
\draw[color=uuuuuu] (8.891402019308613,-2.4291364872387637) node {$C$};
\draw [fill=uuuuuu] (6.287016087198279,-0.6871054337761034) circle (2.0pt);
\draw[color=uuuuuu] (6.183487361935326,-0.9192689207033586) node {$I$};
\draw [fill=uuuuuu] (7.685028544826755,0.2928796589089621) circle (2.0pt);
\draw[color=uuuuuu] (7.7918245523752185,0.557775437863886) node {$D$};
\draw [fill=uuuuuu] (6.292107838093549,-0.16666984677409122) circle (2.0pt);
\draw[color=uuuuuu] (6.150664153967165,0.14748533826187354) node {$E$};
\draw [fill=uuuuuu] (9.39173461207809,0.3371968516998012) circle (2.0pt);
\draw[color=uuuuuu] (9.547866178671836,0.6070102498161274) node {$I'$};
\draw [fill=uuuuuu] (9.083041002455232,1.2728647515940277) circle (2.0pt);
\draw[color=uuuuuu] (9.301692118910626,1.5424716769087157) node {$I''$};
\draw [fill=uuuuuu] (6.53504758519073,-0.5132391930122152) circle (2.0pt);
\draw[color=uuuuuu] (6.823539917314466,-0.5418020290695073) node {$P$};
\end{scriptsize}
\end{tikzpicture}

\textbf{Solution.} 
Let $I'$ be the reflection of $I$ in $AC$. 
Observe that $AC$ is an angle bisector of $\angle BDI'$ by the definition of $I'$, 
and since $DE\perp AC$, 
$DE$ is another angle bisector of this angle. 
This implies that the intersection of $DE$ and the circumcircle of $BDI'$ (other than $D$) is equidistant from $B$ and $I'$, 
i.e. on the perpendicular bisector of $BI'$. 
It therefore suffices to prove that $E$ lies on this perpendicular bisector, or $BE=EI'$. 

Let $I''$ be the image of $I$ when reflected in $D$, then $DI=DI'=DI''$. 
Moreover, $I''$ lies on line $BD$, which entails that $I'I''$ is parallel to $AC$ and perpendicular to $DE$. 
Therefore, $DE$ is the perpendicular bisector of $I'I''$ and $EI''=EI'$. 
The problem is now reduced to proving $BE=EI''$. 
Let $P$ be the foot of perpendicular from $E$ to $BD$, 
then the problem is now equivalent to proving that $P$ is the midpoint of $BI''$. 
Knowing that $BP=BI+ID-PD$ and $PI''=PD+DI''=PD+DI$ it suffices to prove that $BI=2PD$. 

Denote the common angles $\angle ABI, \angle IBC, \angle ICB, \angle ACI$ as $\alpha$. 
Then, $\angle ADB=3\alpha$, and $\angle IDE=|90^{\circ}-3\alpha|$ (as we will see, we are only interested in the cosine of this angle so don't worry about the sign). 
So \[\frac{PD}{BI}
=\frac{DE\cos\angle IDE}{BI}
=\frac{(AD\tan\angle DAI)\cos\angle IDE}{BI}
=\frac{AB\sin\angle ABD}{\sin\angle ADB}\cdot\frac{\tan\angle DAI\cos\angle IDE}{BI}\]
\[=\frac{BI\sin\angle AIB}{\sin\angle BAI}\cdot\frac{\tan\angle DAI\cos\angle IDE\sin\angle ABD}{BI\sin\angle ADB}
=\frac{\tan(90^{\circ}-2\alpha)\cos|90^{\circ}-3\alpha|\sin\alpha\sin(90^{\circ}+\alpha)}{\sin(3\alpha)\sin(90^{\circ}-2\alpha)}\]
\[=\frac{\sin(90^{\circ}-2\alpha)\sin(3\alpha)\sin\alpha\cos\alpha}{\sin(3\alpha)\sin(90^{\circ}-2\alpha)\cos(90^{\circ}-2\alpha)}
=\frac{\sin\alpha\cos\alpha}{\sin (2\alpha)}
=\frac{\sin\alpha\cos\alpha}{2\sin\alpha\cos\alpha}
=\frac 12,\] 
as $\cos|90^{\circ}-x|=\cos(90^{\circ}-x)=\sin x$, $\sin(90^{\circ}+\alpha)=\cos\alpha$ and $\tan x=\frac{\sin x}{\cos x}$. 

\newpage

\item[\textbf{G5}]
Let $D$ be the foot of perpendicular from $A$ to the Euler line (the line passing through the circumcentre and the orthocentre) of an acute scalene triangle $ABC$. A circle $\omega$ with centre $S$ passes through $A$ and $D$, and it intersects sides $AB$ and $AC$ at $X$ and $Y$ respectively. Let $P$ be the foot of altitude from $A$ to $BC$, and let $M$ be the midpoint of $BC$. Prove that the circumcentre of triangle $XSY$ is equidistant from $P$ and $M$.

\textbf{Thoughts.} Solution 1 is more intuitive if you are comfortable with using trigonometry to solve geometry problems, so I will write the thought process of the approach using spiral similarity as in solution 2. First, we should see how do $X$, $Y$, $S$, and $J$ (the circumcenter of $XSY$) vary as $S$ varies. The fact that $X$ and $Y$ vary along two respective straight lines and different versions of triangles $DXY$ are similar suggests to us that $S$ and $J$ should also vary along a straight line. More importantly, $DX_1X_2$ will be similar to $DJ_1J_2$ where $X_1, X_2, J_1, J_2$ are different versions of $X$ and $J$, respectively, such that $X_i$ and $J_i$ correspond to the same $i$. 

Knowing that, a strategy is to prove the problem statement for one special case of pair $(X, Y)$, and show that the other $J$ lie on the perpendicular from our $J$ in this special case to $BC$. The easiest $(X, Y)$ is the altitude from $B$ and $C$ to opposite side; the corresponding $J$ is the nine-point-center (which works). Now all we need to do is to prove that for any two $J$ (namely $J_1,, J_2$) we have $J_1J_2\perp BC$, which can be established by the similarity of $DJ_1J_2$ and $DX_1X_2$ as mentioned before (with a bit of angle chasing). 

\definecolor{uuuuuu}{rgb}{0.26666666666666666,0.26666666666666666,0.26666666666666666}
\begin{tikzpicture}[line cap=round,line join=round,>=triangle 45,x=1.0cm,y=1.0cm]
\clip(-0.780894732325327,-3.0144161105989133) rectangle (15.499927939805652,4.658806400961947);
\draw [line width=1.2pt] (3.968556783697015,2.1726396229121065) circle (1.8545634959840431cm);
\draw [line width=1.2pt] (4.565154539765928,-0.2774755660589251) circle (2.5217044477442268cm);
\draw [line width=1.2pt] (3.1345454545454556,3.8290909090909078)-- (1.86,-1.28);
\draw [line width=1.2pt] (1.86,-1.28)-- (9.94,-1.04);
\draw [line width=1.2pt] (3.1345454545454556,3.8290909090909078)-- (9.94,-1.04);
\draw [line width=1.2pt] (3.1345454545454556,3.8290909090909078)-- (3.2850433621226975,-1.2376719793428905);
\draw [line width=1.2pt] (3.1345454545454556,3.8290909090909078)-- (5.852548986261354,0.4375174625344114);
\draw [line width=1.2pt] (5.852548986261354,0.4375174625344114)-- (5.9,-1.16);
\draw [line width=1.2pt] (2.454233092884584,1.1020185321051048)-- (5.805563901771507,1.9180603449254363);
\draw [line width=1.2pt] (3.968556783697015,2.1726396229121065)-- (4.565154539765928,-0.2774755660589251);
\draw [line width=1.2pt] (3.1819964682841015,2.2315734465564963)-- (4.493547220403405,2.13330418581266);
\draw [line width=1.2pt] (3.1345454545454556,3.8290909090909078)-- (3.968556783697015,2.1726396229121065);
\draw [line width=1.2pt] (4.493547220403405,2.13330418581266)-- (4.565154539765928,-0.2774755660589251);
\begin{scriptsize}
\draw [fill=uuuuuu] (3.1345454545454556,3.8290909090909078) circle (2.0pt);
\draw[color=uuuuuu] (3.2397606058357145,4.0571238239484115) node {$A$};
\draw [fill=uuuuuu] (1.86,-1.28) circle (2.0pt);
\draw[color=uuuuuu] (1.9656092662776379,-1.0394815342838948) node {$B$};
\draw [fill=uuuuuu] (9.94,-1.04) circle (2.0pt);
\draw[color=uuuuuu] (10.035234416812123,-0.8129657405846811) node {$C$};
\draw [fill=uuuuuu] (5.852548986261354,0.4375174625344114) circle (2.0pt);
\draw[color=uuuuuu] (5.957950130226277,0.6735441555664081) node {$O$};
\draw [fill=uuuuuu] (5.9,-1.16) circle (2.0pt);
\draw[color=uuuuuu] (6.000421841544879,-0.9262236374342878) node {$M$};
\draw [fill=uuuuuu] (3.2850433621226975,-1.2376719793428905) circle (2.0pt);
\draw[color=uuuuuu] (3.381332976897723,-0.9970098229652922) node {$P$};
\draw [fill=uuuuuu] (3.2294474820227475,0.6340559840220845) circle (2.0pt);
\draw[color=uuuuuu] (3.3247040284729197,0.8717454750532201) node {$H$};
\draw [fill=uuuuuu] (3.1819964682841015,2.2315734465564963) circle (2.0pt);
\draw[color=uuuuuu] (3.282232317154317,2.4715132680539162) node {$F$};
\draw [fill=uuuuuu] (4.493547220403405,2.13330418581266) circle (2.0pt);
\draw[color=uuuuuu] (4.697956027774401,2.499827742266318) node {$E$};
\draw [fill=uuuuuu] (3.968556783697015,2.1726396229121065) circle (2.0pt);
\draw[color=uuuuuu] (3.919307986933355,2.570613927797322) node {$S$};
\draw [fill=uuuuuu] (2.454233092884584,1.1020185321051048) circle (2.0pt);
\draw[color=uuuuuu] (2.1071816373396466,1.1407329800710362) node {$X$};
\draw [fill=uuuuuu] (5.805563901771507,1.9180603449254363) circle (2.0pt);
\draw[color=uuuuuu] (6.028736315757281,2.089267866186493) node {$Y$};
\draw [fill=uuuuuu] (4.565154539765928,-0.2774755660589251) circle (2.0pt);
\draw[color=uuuuuu] (4.669641553561999,-0.04847493684983516) node {$J$};
\draw [fill=uuuuuu] (4.540998234142051,0.535786723278248) circle (2.0pt);
\draw[color=uuuuuu] (4.6413270793495975,0.7726448153098141) node {$N$};
\end{scriptsize}
\end{tikzpicture}

\textbf{Solution 1.}
Denote by $O$ the circumcenter and $H$ the orthocenter of triangle $ABC$
Denote also by $E$ the midpoint of $AO$, $F$ the midpoint of $AH$, $N$ the midpoint of $OH$ (the nine-point-center), and $J$ the circumcenter of $XSY$. Observe that our goal is to prove that the circumcenter of triangle $XSY$ (namely $J$) lies on the perpendicular bisector of $PM$. Considering the 9-point circle we know that $P$ and $M$ pass through the circle, so the center $N$ is on the perpendicular bisector of $PM$. Thus it suffices to prove that $J$ lies on the perpendicular from $N$ to $BC$. 

Now let's break the solution into the following:
\begin{itemize}
\item [Lemma 1.]  $S$ lies on $EF$.\\
Proof: 
Obviously $S$ passes through the perpendicular bisector of $AD$, so this locus is a line. 
In the case there the circle passes through $H$, from the fact that $\angle ADH=90^{\circ}$ we know that $S=F$.
Similarly, if the circle passes through $O$, $S=E$ in this case with $\angle ADO=90^{\circ}$. 
Thus the locus is actually $EF$, i.e. parallel to $OH$. 
\item [Lemma 2.] The problem statement holds in the special cases where $S\equiv E$ or $S\equiv F$.\\
Proof:
When $S\equiv E$, $X$ is the midpoint of $AB$ and $Y$ is the midpoint of $AC$ (imagine the homothety centered at $A$ with factor $\frac 12$ which brings $ABC$ to $AXY$ and point $O$ to the midpoint of $AO$). 
$J$ lies on the perpendicular bisector of $XY$
Notice that, with $XY\parallel BC$, this perpendicular bisector of $XY$ is also perpendicular to $BC$. 
Moreover, nine-point-circle passes through the midpoints of $AB$ and $AC$, so this perpendicular bisector passes through the nine-point center. 
Therefore the perpendicualr bisector of $XY$ is the perpendicular bisector of $PM$ itself, 
and with $SX=SY$, $S$ (the midpoint of $AO$, a.k.a. $E$ in this case) lies on this perpendicular bisector too. 
When $S\equiv F$, $X$ and $Y$ are going to be the altitude from $C$ to $AB$, and $B$ to $AC$, respectively. 
Since the nine-point circle passes through the midpoint of $AH$ ($S$ in this case), $X$, $Y$, $P$, $M$, 
 the circumcenter of $XSYPM$ (i.e. $J$) is the nine-point center itself. 
\end{itemize}

Now let's do the general case. Observe that with $E$ (midpoint of $AO$) and $N$ (midpoint of $OH$) both equidistant from $PM$ the conclusion now becomes $J$ lies on $EN$. We first need the following: 
\begin{itemize}
\item [Lemma 3.] $\frac {SJ}{AS}=\frac {AE}{EN}$.\\
Proof: we have $AE=\frac 12 AO$ and $EN=AF=\frac 12 AH$, and $AS=SX=SY$, 
so we just have to prove that 
$\frac {SJ}{SX}=\frac {AO}{AH}$.
First, it is well notice that $SJ$ is the circumradius of $SXY$, so knowing that $\angle SXY=\angle XSY=90^{\circ}-\frac 12\angle XSY$
$= 90^{\circ}-\angle XAY=90^{\circ}-\angle BAC$ we have 
$SX=2SJ\sin\angle SXY=2SJ\cos\angle BAC$, yielding $\frac {SJ}{SX}=\frac {1}{2\cos\angle BAC}$.
Let $T$ be the reflection of $H$ in $M$, then $HBMT$ is a parallelogram with $\angle ABT=\angle ABC+\angle CBT$
$\angle ABC+\angle HCB=90^{\circ}$, and similarly $\angle ACT=90^{\circ}$. 
Therefore $A, O, T$ collinear and with $AH\parallel OM$ we have $\frac{OM}{AH}=\frac {TH}{MT}=\frac 12$. 
We also know that $\frac {OM}{AO}=\frac {OM}{BO}=\cos\angle BOM=\cos\angle BAC$
so $\frac {AO}{AH}=\frac {AO}{2OM}=\frac 1{2\cos\angle BAC}=\frac {SJ}{AS}$.
\item [Lemma 4.] $\angle (HA, AS)=\angle (SJ, AO)$. (This would also imply $\angle (AO, AS)=\angle (SJ, AH)$).\\
Proof: we use the well-known fact that the circumcenter and orthocenter of each triangle are the isogonal conjugates of each other. 
In particular, if $\ell$ is the perpendicular from $A$ to $XY$ then $AS$ and $\ell$ are the images of each other in the reflection of the internal angle bisector of $\angle AXY$. 
This gives $\angle (AB, AS)=\angle (\ell, AC)$. 
Same goes for the relation between $AH$ and $AO$, and therefore $\angle (AB, AH)=\angle (AO, AC)$. 
Moreover, $SJ\perp XY$ (since $SX=SY$ and $JX\perp JY$ we know that $SJ$ must be the perpendicular bisector of $XY$). 
Therefore $SJ\parallel \ell$. 
Now we have $\angle (SJ, AO)=\angle (\ell, AO)=\angle (\ell, AC)+\angle (AC, AO)=\angle (AB, AS)+\angle (AH, AB)=\angle (AH, AS)$. 
\end{itemize}

To complete the proof, denote by $J'$ the intersection of $SJ$ and $EN$ and we shall prove that $J=J'$ by proving that $SJ=SJ'$. 
From the first lemma it suffices to prove that  $\frac {SJ'}{AS}=\frac {AE}{EN}$. 
Now $\frac {AS}{SE}=\frac{\sin\angle AES}{\sin\angle SAE}=\frac{\sin\angle AOH}{\sin\angle SAO}$ and 
$\frac{SJ'}{SE}=\frac{\sin\angle SEJ'}{\sin\angle EJ'S}=\frac{\sin\angle AFE}{\sin\angle EJ'S}=\frac{\sin\angle AHO}{\sin\angle EJ'S}$. 
Now, $\angle (EJ', J'S)=\angle (EN, SJ)=\angle (AH, SJ)=\angle (AO, AS)$ so the angles $\angle SEJ'$ and $\angle SAO$ are either equal or supplementary, hence $\sin\angle SEJ'=\sin\angle SAO$. 
Therefore, $\frac {SJ'}{AS}=\frac {SJ'}{SE}\div \frac {AS}{SE}= \frac{\sin\angle AHO}{\sin\angle EJ'S}\div\frac{\sin\angle AOH}{\sin\angle SAO}=\frac{\sin\angle AHO}{\sin\angle AOH}=\frac{AO}{AH}=\frac {EN}{AE}$, Q.E.D.

\textbf{Solution 2.} We present another proof of $NJ\perp BC$, this time involving the point $D$ itself. 
Denote $F$ as above (the midpoint of $AH$), $X_H$ as the foot of perpendicular from $C$ to $AB$, and $Y_H$ the foot of perpendicular from $B$ to $AC$. 

We firstproe that triangles $DXY$ and $DX_HY_H$ are similar. Observe that with $ADXY$ cyclic, we have $\angle (DX, XY)=\angle (DA, AY)=\angle (DA, AC)$, $\angle (DY, XY)=\angle (DA, AX)=\angle (DA, AB)$ and finally $\angle (DX, DY)=\angle (AB, AC)$ for the similar reason. 
Now considering $X_H$ and $Y_H$, we know that from $\angle AX_HH=\angle AY_HH=\angle ADH=90^{\circ}$ we know that $A, X_H, Y_H, H, D$ concyclic. By the similar logic above, 
$\angle (DX_H, X_HY_H)=\angle (DA, AY_H)=\angle (DA, AC)$, $\angle (DY_H, X_HY_H)=\angle (DA, AX_H)=\angle (DA, AB)$ and finally $\angle (D_HX, DY_H)=\angle (AB, AC)$ for the similar reason. 
This gives $\angle (DX, XY)=\angle (DX_H, X_HY_H)$, $\angle (DY, XY)=\angle (DY_H, X_HY_H)$ and $\angle (DX_H, DY_H)=\angle (DX, DY)$, proving that the two triangles are similar. 
(A plausible question would be, since we are using directed angles, what if some two angles are not equal but supplementary? 
This wouldn't happen with the constraint that sum of angles of a triangle is $180^{\circ}$). 

Now denote $\Phi$ as the spiral similarity sending triangle $DX_HY_H$ to $DXY$, so $\Phi(X_H)=X$ and $\Phi(Y_H)=Y$. Notice that, $\Phi$ sends the corresponding circumcenters as well, thus $\Phi(F)=S$. 
This means that $\Phi(FX_HY_H)=SXY$, and this spiral similarity, again, maps their corresponding circumcenters too. 
But we have proven in solution 1 that the circumcenter of $FX_HY_H$ is $N$, so $\Phi(N)=J$. 
With $\Phi(D)=D$, we have $\Phi(DX_HN)=\Phi(DXJ)$. This could, in turn, be manipulated to get the fact that not only $DX_HN$ and $DXJ$ are similar, $DX_HX$ and $DNJ$ are similar too (in a similar orientation). So $\angle (DN, NJ)=\angle (DX_H, X_HX)=\angle (DX_H, X_HA)=\angle (DH, HA)=\angle (OH, HA)$, the last two equality following from the fact that $D$ lies on $OH$ and $D, X_H, A, H$ are concyclic. However we also have $\angle (DN, NJ)=\angle (OH, NJ)$ because $D$ also lies on $OH$. Thus $\angle (OH, NJ)=\angle (OH, HA)$, meaning that $NJ\parallel HA$, i.e. $HA$ and $NJ$ are both perpendicular to $BC$. 


\newpage 

\item[\textbf{G6}]
Let $ABCD$ be a convex quadrilateral with $\angle ABC = \angle ADC < 90^{\circ}$. The internal angle bisectors of $\angle ABC$ and $\angle ADC$ meet $AC$ at $E$ and $F$ respectively, and meet each other at point $P$. Let $M$ be the midpoint of $AC$ and let $\omega$ be the circumcircle of triangle $BPD$. Segments $BM$ and $DM$ intersect $\omega$ again at $X$ and $Y$ respectively. Denote by $Q$ the intersection point of lines $XE$ and $YF$. Prove that $PQ \perp AC$.

\textbf{Thoughts.} Using the notation below, the problem would have been much easier if $P$ and $Q$ are defined in terms of $E, F, M_1, M_2, N_1, N_2$ only. But this is not the case. Now let's see how can we arrive here. 

We wish that $ABCD$ is cyclic to aid angle chasing, but it isn't. Nevertheless, we should immediately see that the circumcirles of $ABC$ and $ADC$ are the reflections of each other in $AC$, which, if we draw the two circles, give rise to the points $M_1$ and $M_2$. Next, drawing circle $BPD$ lets us guess that $X$ and $Y$ might lie on the two circumcircles aforementioned (which isn't that hard to prove either), and finally with angle-bisector theorem we know that $XE$ passes through $N_2$ and $YF$ passes through $N_1$. Now it's no longer difficult to finish the proof. 

Each of the subproblems mentioned above are not hard to prove, but realizing them (i.e. seeing that we can break the problem into these subproblems) requires some insight.

\definecolor{uuuuuu}{rgb}{0.26666666666666666,0.26666666666666666,0.26666666666666666}
\begin{tikzpicture}[line cap=round,line join=round,>=triangle 45,x=1.0cm,y=1.0cm]
\clip(-0.06077115315761548,-3.6584705525177825) rectangle (17.29579728541572,4.521755615922858);
\draw [line width=1.2pt] (5.233712087730585,0.15993392341275056) circle (3.52235927554948cm);
\draw [line width=1.2pt] (8.344455105709537,0.24031487994967485) circle (3.52235927554948cm);
\draw [line width=1.2pt] (6.707454566010934,3.3591678901243993)-- (6.870712627429188,-2.958919086761974);
\draw [line width=1.2pt] (6.789083596720062,0.2001244016812127)-- (3.393315919220401,3.1632582164224963);
\draw [line width=1.2pt] (6.789083596720062,0.2001244016812127)-- (10.923409101776997,2.639483726074752);
\draw [line width=1.2pt] (4.823271172307094,0.14932821500387372)-- (5.119534491301556,1.6569665297266845);
\draw [line width=1.2pt] (8.754896021133032,0.25092058835854925)-- (3.393315919220401,3.1632582164224963);
\draw [line width=1.2pt] (10.923409101776997,2.639483726074752)-- (4.823271172307094,0.14932821500387372);
\draw [line width=1.2pt] (8.754896021133032,0.25092058835854925)-- (7.174752186710168,1.1092332982817523);
\draw [line width=1.2pt,dash pattern=on 4pt off 4pt] (1.712528154328142,0.06894725846694982)-- (11.865639039111985,0.33130154489547325);
\draw [line width=1.2pt] (5.119534491301556,1.6569665297266845)-- (11.865639039111985,0.33130154489547325);
\draw [line width=1.2pt] (1.712528154328142,0.06894725846694982)-- (8.580780113524357,1.2572718370740135);
\draw [shift={(7.3725916469831345,5.981256284249301)},line width=1.2pt,dash pattern=on 4pt off 4pt]  plot[domain=3.7577798782786243:5.528106782022175,variable=\t]({1.*4.876038189743517*cos(\t r)+0.*4.876038189743517*sin(\t r)},{0.*4.876038189743517*cos(\t r)+1.*4.876038189743517*sin(\t r)});
\begin{scriptsize}
\draw [fill=uuuuuu] (6.707454566010934,3.3591678901243993) circle (2.0pt);
\draw[color=uuuuuu] (6.685651535574803,3.6690198622016474) node {$A$};
\draw [fill=uuuuuu] (3.393315919220401,3.1632582164224963) circle (2.0pt);
\draw[color=uuuuuu] (3.199245179652681,3.4577225072972766) node {$B$};
\draw [fill=uuuuuu] (6.870712627429188,-2.958919086761974) circle (2.0pt);
\draw[color=uuuuuu] (6.97241223151645,-2.7151787895518416) node {$C$};
\draw[color=black] (2.6936407947029357,2.8464694448953467) node {$\omega_1$};
\draw[color=black] (7.70440663957802,3.993512228661931) node {$\omega_2$};
\draw [fill=uuuuuu] (10.923409101776997,2.639483726074752) circle (2.0pt);
\draw[color=uuuuuu] (11.122895988566595,2.6879964287170686) node {$D$};
\draw [fill=uuuuuu] (6.75976776182244,1.3346472122191515) circle (2.0pt);
\draw[color=uuuuuu] (6.896948890479175,1.5258609767430293) node {$E$};
\draw [fill=uuuuuu] (6.769863235751817,0.9439523711522477) circle (2.0pt);
\draw[color=uuuuuu] (6.610188194537528,1.0730809305193776) node {$F$};
\draw [fill=uuuuuu] (6.789083596720062,0.2001244016812127) circle (2.0pt);
\draw[color=uuuuuu] (6.564910189915163,0.0015014877900685064) node {$M$};
\draw [fill=uuuuuu] (8.754896021133032,0.25092058835854925) circle (2.0pt);
\draw[color=uuuuuu] (9.032561441834067,0.3109011860428972) node {$M_1$};
\draw [fill=uuuuuu] (4.823271172307094,0.14932821500387372) circle (2.0pt);
\draw[color=uuuuuu] (4.42929763856027,0.2618278681174477) node {$M_2$};
\draw [fill=uuuuuu] (5.119534491301556,1.6569665297266845) circle (2.0pt);
\draw[color=uuuuuu] (5.0405507009622,2.023919027589046) node {$X$};
\draw [fill=uuuuuu] (8.580780113524357,1.2572718370740135) circle (2.0pt);
\draw[color=uuuuuu] (8.692976407166327,1.480582972120664) node {$Y$};
\draw [fill=uuuuuu] (7.174752186710168,1.1092332982817523) circle (2.0pt);
\draw[color=uuuuuu] (7.168616918213366,1.420212299290844) node {$P$};
\draw [fill=uuuuuu] (1.712528154328142,0.06894725846694982) circle (2.0pt);
\draw[color=uuuuuu] (1.8484513750854517,0.3561791906652624) node {$N_1$};
\draw [fill=uuuuuu] (11.865639039111985,0.33130154489547325) circle (2.0pt);
\draw[color=uuuuuu] (12.205817078702718,0.6127545501919983) node {$N_2$};
\draw [fill=uuuuuu] (7.821783221564824,1.1259524464692119) circle (2.0pt);
\draw[color=uuuuuu] (7.9232503285861196,1.3447489582535685) node {$Q$};
\end{scriptsize}
\end{tikzpicture}

\textbf{Solution.} 
Let $\omega_1$ be the circumcircle of $ABC$ and $\omega_2$ the circumcircle of $ADC$, then these two circles are symmetric w.r.t. $AC$.
Also notice that $BP$ passes through $M_1$, the midpoint of arc $AC$ of $\omega_1$ not containing $B$, and $DP$ passes through $M_2$, the midpoint of arc $AC$ of $\omega_2$ not containing $D$.

We first start with a preliminary observation: $X$ lies on $\omega_2$ and $Y$ lies on $\omega_1$. W.L.O.G. for this section we assume that $AB\le AC$. Indeed, let $X'$ be on $BM$ satisfying $MX'\cdot MB=MA^2=MC^2$. Then $\angle X'AC=\angle MBA$ and $\angle X'CA=\angle MBC$. Thus $\angle ADC=\angle ABC=\angle MBA+\angle MBC=\angle X'AC+\angle X'CA=\pi - \angle AX'C$, so $X'$ lie on $\omega_2$. In addition, let $BM$ intersect $\omega_1$ again at $X''$, then $X'$ and $X''$ are symmetrical w.r.t. $AC$. Combining with the fact that $M_1$ and $M_2$ are also symmetrical w.r.t. $AC$ (being the midpoint of arc) we have $X'M_2=X''M_1$. Knowing that the two circles have the same radius further allows us to assert $\angle X'BP=\angle X''BM_1=\angle X'DM_2=\angle X'DP$, showing that $D, B, P, X'$ cyclic hence $X'=X$. Similarly, $Y$ lies on $\omega_1$.

Next, let $N_1$ be diametrically opposite $M_1$ w.r.t. $\omega_1$ and define similarly for $N_2$. We claim that $XE$ passes through $N_2$ by claiming that $XE$ is the internal angle bisector of $\angle AXC$. Indeed, by angle bisector theorem we have $\frac{AE}{EC}=\frac{AB}{BC}$. Invoking our $X''$ from the previous section (i.e. the other intersection of $BM$ and $\omega_1$) gives $AXCX''$ parallelogram. Now invoking a little bit more trigonometric bashing we have $1=\frac{AM}{CM}=\frac{AB}{BC}\cdot\frac{\sin\angle ABM}{\sin\angle CBM}=\frac{AB}{BC}\cdot\frac{AX''}{CX''}=\frac{AB}{BC}\cdot\frac{CX}{AX},$ so $\frac{AX}{CX}=\frac{AB}{BC}=\frac{AE}{EC}$, and the conclusion follows by the angle bisector theorem. Analogously, $YF$ passes through $N_1$.

Finally, considering triangle $PEF$, and letting the perpendicular from $P$ to reach $AC$ at $P_1$ we have (considering signed length) $\frac{EP_1}{FP_1}=\frac{\cot\angle FEP}{\cot\angle EFP}$. Similarly if letting perpendcular from $Q$ to reach AC at $Q_1$ we have $\frac{EQ_1}{FQ_1}=\frac{\cot\angle FEQ}{\cot\angle EFQ}$. Now $\cot\angle FEP=\cot\angle MEM_1=\frac{MM_1}{EM}$, $\cot\angle EFP=\cot\angle MFM_2=\frac{MM_2}{FM}$. Considering $MM_2=MM_1$ we have $\frac{\cot\angle FEP}{\cot\angle EFP}=\frac{FM}{EM}$. Analogously, $\cot\angle FEQ=\cot\angle FEQ=\cot\angle MEN_2=\frac{MN_2}{EM}$, and $\cot\angle EFQ=\cot\angle N_1FM=\frac{MN_1}{FM}$. Therefore we have $\frac{\cot\angle FEQ}{\cot\angle EFQ}=\frac{FM}{EM}$ since again it is not hard to verify that $MN_2=MN_1$. (For signed convention we can say that $ME<0$ if it's nearer to $A$ than $B$, and $>0$ otherwise). Therefore, $\frac{EP_1}{FP_1}=\frac{EQ_1}{FQ_1}$, so $P_1\equiv Q_1$ and the two perpendicular lines coincide.

\textbf{Note:} We present another proof of $PQ\perp AC$ after establishing the claims that $X$ on $\omega_2$, $Y$ on $\omega_1$, $XE$ passes through $N_2$ and $YF$ passes through $N_1$. Now, consider the circle containing $B, P, D, X, Y$. 
Denoting $Q_1$ as the intersection of $XE$ with this circle we have $\angle (PQ_1, XE)=\angle (PQ_1, XQ_1)=\angle (PD, XD)=\angle (M_2D, XD)=\angle (M_2N_2, XN_2)=\angle (M_2N_2, XE)$ (because $P, D, M_2$ collinear and $X, M_2, D, N_2$ lie on the same circle $\omega_2$). Thus $PQ_1\parallel M_2N_2$. In a similar way if $Q_2$ is the intersection of $Y_F$ with the circle containing $B, P, D, X, Y$ we also have $PQ_2\parallel M_1N_1$. But with $M_1, N_2, M_2, N_2$ collinear we have: $P, Q_1, Q_2$ lying on the same line, and since all three lie on the circle $BPD$ we have $Q_1$ coincides with $Q_2$ ($P$ cannot be equal to either of them; otherwise $M_1\equiv N_2$ or $M_2\equiv N_1$ which is impossible). This gives $Q_1\equiv Q_2\equiv Q$ and since $PQ\parallel M_1N_1$ and $M_1N_1\perp AC$, we have $PQ\perp AC$. 

\newpage

\item[\textbf{G7}]
Let $I$ be the incentre of a non-equilateral triangle $ABC$, $I_A$ be the $A$-excentre, $I'_A$ be the reflection of $I_A$ in $BC$, and $l_A$ be the reflection of line $AI'_A$ in $AI$. Define points $I_B$, $I'_B$ and line $l_B$ analogously. Let $P$ be the intersection point of $l_A$ and $l_B$.

\begin{itemize}
\item [(a)] Prove that $P$ lies on line $OI$ where $O$ is the circumcentre of triangle $ABC$.
\item [(b)] Let one of the tangents from $P$ to the incircle of triangle $ABC$ meet the circumcircle at points $X$ and $Y$. Show that $\angle XIY = 120^{\circ}$.
\end{itemize}

\textbf{Thoughts.} \#smh the problems concerning the concurrency of two lines on some not-so-well-blended line $OI$ are not that easy: we either manually establish the concurrency point itself or we show homothety of two triangles. Neither is easy. 

In my case, the problem statement of part (b) gave me a hint on where should $P$ be. Using knowledge from Poncelet Porism, we know that there exists a $Z$ on the circumcircle of $ABC$ such that the incircle of $ABC$ and $ZXY$ coincide. 
Therefore if $\angle XIY=120^{\circ}$ we must have $\angle XOY=120^{\circ}$ too. 
This means, $X, O, I, Y$ lie on a circle, and we can invert it with radius $OX$ w.r.t. the circumcircle of $ABC$ to obtain line $XY$. 
Since $P$ must lie on $OI$ and $XY$, we have the image of $P$ lying on both $OI$ and circle $IXY$, hence the image is $I$. 
This gives us $OP\cdot OI=OX^2$. 

Now that we locate the targetted $P$, it remains to prove that this $P$ lies on $l_A$. There are a few ways to establish this; one way is by trigonometry as written below (and with this, part (b) comes for free!)

\textbf{Solution.} 
\definecolor{uuuuuu}{rgb}{0.26666666666666666,0.26666666666666666,0.26666666666666666}
\begin{tikzpicture}[line cap=round,line join=round,>=triangle 45,x=0.5cm,y=0.5cm]
\clip(-9.784068425578555,-10.140965804211529) rectangle (28.496495861557182,7.900830581551528);
\draw [line width=1.2pt] (7.3936864077333855,0.543416871511203) circle (2.3435633554098465cm);
\draw [line width=1.2pt] (4.163371953821335,3.9396243640131448)-- (3.095957911065971,-1.3270607605957092);
\draw [line width=1.2pt] (11.575141965729667,-1.5742672928016477)-- (3.095957911065971,-1.3270607605957092);
\draw [line width=1.2pt] (4.163371953821335,3.9396243640131448)-- (11.575141965729667,-1.5742672928016477);
\draw [line width=1.2pt] (4.163371953821335,3.9396243640131448)-- (9.053160130826601,-8.833360183928688);
\draw [line width=1.2pt] (4.163371953821335,3.9396243640131448)-- (7.3936864077333855,0.543416871511203);
\draw [line width=1.2pt] (-3.9735037128951296,0.5816773980697412)-- (7.3936864077333855,0.543416871511203);
\draw [line width=1.2pt] (4.163371953821335,3.9396243640131448)-- (-3.9735037128951296,0.5816773980697412);
\draw [line width=1.2pt] (4.163371953821335,3.9396243640131448)-- (9.3618077011955,1.7532514797246423);
\draw [line width=1.2pt] (4.163371953821335,3.9396243640131448)-- (9.48035504203699,5.81942527058766);
\draw [line width=1.2pt] (4.163371953821335,3.9396243640131448)-- (4.047099015150206,-0.048537432406617);
\draw [line width=1.2pt] (4.099415080528914,1.7459036100830427) -- (3.8556856648354945,1.952818970749336);
\draw [line width=1.2pt] (4.099415080528914,1.7459036100830427) -- (4.354785304136048,1.9382679608571916);
\draw [line width=1.2pt] (9.48035504203699,5.81942527058766)-- (9.053160130826601,-8.833360183928688);
\draw [line width=1.2pt] (9.260937182474935,-1.706607312390734) -- (9.017207766781517,-1.4996919517244416);
\draw [line width=1.2pt] (9.260937182474935,-1.706607312390734) -- (9.51630740608207,-1.514242961616586);
\draw [line width=1.2pt] (4.163371953821335,3.9396243640131448)-- (9.421081371616246,3.7863383751561503);
\begin{scriptsize}
\draw [fill=uuuuuu] (4.163371953821335,3.9396243640131448) circle (2.0pt);
\draw[color=uuuuuu] (4.396384084264771,4.488867242915525) node {$A$};
\draw [fill=uuuuuu] (3.095957911065971,-1.3270607605957092) circle (2.0pt);
\draw[color=uuuuuu] (3.3311857736662107,-0.7705494156648485) node {$B$};
\draw [fill=uuuuuu] (11.575141965729667,-1.5742672928016477) circle (2.0pt);
\draw[color=uuuuuu] (11.819484811248481,-1.0368489933144878) node {$C$};
\draw [fill=uuuuuu] (7.3936864077333855,0.543416871511203) circle (2.0pt);
\draw[color=uuuuuu] (7.725128804885269,0.727385708614372) node {$O$};
\draw [fill=uuuuuu] (9.053160130826601,-8.833360183928688) circle (2.0pt);
\draw[color=uuuuuu] (9.60586957203585,-8.376731102282667) node {$I_A$};
\draw [fill=uuuuuu] (5.461026912989256,0.5499219593246881) circle (2.0pt);
\draw[color=uuuuuu] (5.694594525306765,1.093547627882626) node {$I$};
\draw [fill=uuuuuu] (9.48035504203699,5.81942527058766) circle (2.0pt);
\draw[color=uuuuuu] (9.905456596891696,6.436182904478511) node {$I_A'$};
\draw [fill=uuuuuu] (-3.9735037128951296,0.5816773980697412) circle (2.0pt);
\draw[color=uuuuuu] (-3.6425344160337345,1.210053693104343) node {$P_1=P$};
\draw [fill=uuuuuu] (9.3618077011955,1.7532514797246423) circle (2.0pt);
\draw[color=uuuuuu] (9.87216914968549,2.3751143453215144) node {$I_A''$};
\draw [fill=uuuuuu] (7.257093521907928,-4.141719112301999) circle (2.0pt);
\draw[color=uuuuuu] (7.492116674441834,-3.5999824281922645) node {$M$};
\draw [fill=uuuuuu] (4.047099015150206,-0.048537432406617) circle (2.0pt);
\draw[color=uuuuuu] (4.2965217426461555,-0.3711000491903897) node {$H$};
\draw [fill=uuuuuu] (9.266757586431796,-1.5069674566705147) circle (2.0pt);
\draw[color=uuuuuu] (9.489363506814133,-0.9702740989020779) node {$T$};
\draw [fill=uuuuuu] (9.421081371616246,3.7863383751561503) circle (2.0pt);
\draw[color=uuuuuu] (9.922100320494797,3.723255957172813) node {$U$};
\end{scriptsize}
\end{tikzpicture}

(a) We first prove the following lemma: let $I''_A$ the point on $I_AI_A'$ satisfying $AI''_A=AI'_A$, and $I''_A\neq I'_A$ unless $AI'_A$ and $I_AI_A'$ are perpendicular to each other. Then triangles $AOI$ and $I_AAI''_A$ are similar. 
Indeed, let $H$ be the orthocenter of triangle $ABC$, then $AH\parallel I_AI_A''$. 
Also, $\angle OAI=\angle IAH=\angle AI_AI''_A$, so it suffices to prove that $\frac{AI}{AO}=\frac{I_AI''_A}{I_AA}$. 
To do so, define $I_C$ like how $I_A$ and $I_B$ are defined. 
We first notice that $I$ is the orthocenter of $I_AI_BI_C$, so $I_AA$ is an altitude of triangle $I_AI_BI_C$. 
In addition, we know that triangles $I_AI_BI_C$ are $I_ACB$ are similar with a similitude of $\frac {I_AB}{I_AI_B}=\cos\angle BI_AC$.
Denote by $T$ the altitude from $I_A$ to $BC$, then by the similarities of the triangles mentioned above we have 
$\frac {I_AI'_A}{I_AA}=\frac{2I_AT}{I_AA}=2\cos\angle BI_AC$. 
In addition, if we let $U$ to be the perpendicular from $A$ to $I_AI_A''$ we have $I'_A$ and $I''_A$ symmetric to each other w.r.t. $U$. 
Therefore, $I_AI_A''+I_AI_A'=2I_AU=2I_AA\cos\angle AI_AI_A'=2I_AA\cos\angle AOI$. 
So $\frac{I_AI''_A}{I_AA}=\frac{2I_AA\cos\angle AOI-I_AI_A'}{I_AA}$
$=\frac{2I_AA\cos\angle AOI-2I_AA\cos\angle BI_AC}{I_AA}$
$=2(\cos\angle AOI-\cos\angle BI_AC)$. 
On the other hand, denoting $M$ as the other intersection of $AI$ and $BC$ we have 
$MB=MC=MI=2AO\cos BAM$ (because $AO$ is the circumradius of triangle $ABC$), 
and $AM=AO\sin\angle ABM$, 
so $\frac{AI}{AO}=\frac{AM-MI}{AO}=\frac{2AO\cos\angle ABM-2AO\cos BAM}{AO}$
$=2(\cos\angle ABM-\cos\angle BAM)$
$=2(\cos\angle AOI-\cos\angle BI_AC)$
(the equality $\angle ABM=\angle AOI$ and $\angle BAM=\angle BI_AC$ can be established via angle chasing). 
This establishes the desired equality. 

Now denote $P_1$ on $OI$ such that $I$ lies between $O$ and $P_1$ and $OI\cdot OP_1=AO$. 
Then, triangles $AOI$ anf $AP_1O$ are similar. 
Therefore, by the fact that $\triangle AOI\sim \triangle I_AAI''_A$ $\angle IAP_1=\angle OAP_1-\angle OAI$
$=\angle AIO-\angle OAI$
$=\angle I_AI''_AA-\angle OAI$
$=180^{\circ}-\angle I_AI'_AA-\angle HAI$
$=180^{\circ}-\angle I_AI'_AA-\angle AI_AI_A'$
$=\angle I_A'AI_A$. 
This means that $l_A$ passes through $P_1$ and similarly, $l_B$ passes through $P_1$. Hence $P_1=P$ and lies on $OI$. 

(b) 

\definecolor{uuuuuu}{rgb}{0.26666666666666666,0.26666666666666666,0.26666666666666666}
\begin{tikzpicture}[line cap=round,line join=round,>=triangle 45,x=1.0cm,y=1.0cm]
\clip(-0.8138091660405693,-3.0651104432757283) rectangle (16.466431254695724,5.079141998497374);
\draw [line width=1.2pt] (6.582731062360519,1.069813550044211) circle (3.637122258760492cm);
\draw [line width=1.2pt] (4.515703038946392,0.407310575525049) circle (1.170862934814784cm);
\draw [line width=1.2pt] (0.779076137488415,-0.7903153922135896)-- (9.745500983360207,-0.7261666146543425);
\draw [line width=1.2pt] (6.582731062360519,1.069813550044211)-- (6.608751575538458,-2.567215630216099);
\draw [line width=1.2pt] (3.44598165453877,-0.7712354655175488)-- (6.608751575538458,-2.567215630216099);
\draw [line width=1.2pt] (5.064465903538294,-1.603892691711937) -- (4.990267326538933,-1.734558404021711);
\draw [line width=1.2pt] (3.44598165453877,-0.7712354655175488)-- (6.582731062360519,1.069813550044211);
\draw [line width=1.2pt] (4.976326089567558,0.21408439664382672) -- (5.052386627331731,0.08449368788283629);
\draw [line width=1.2pt] (6.582731062360519,1.069813550044211)-- (9.745500983360207,-0.7261666146543425);
\draw [line width=1.2pt] (8.201215311360043,0.23715632384982185) -- (8.12701673436068,0.10649061154004773);
\draw [line width=1.2pt] (6.608751575538458,-2.567215630216099)-- (9.745500983360207,-0.7261666146543425);
\draw [line width=1.2pt] (8.139096010567245,-1.5818957680547254) -- (8.215156548331418,-1.7114864768157159);
\draw [shift={(6.6087515755384585,-2.5672156302160922)},line width=1.2pt]  plot[domain=0.5307529852439434:2.6251480876371422,variable=\t]({1.*3.637122258760486*cos(\t r)+0.*3.637122258760486*sin(\t r)},{0.*3.637122258760486*cos(\t r)+1.*3.637122258760486*sin(\t r)});
\draw [line width=1.2pt] (0.779076137488415,-0.7903153922135896)-- (6.582731062360519,1.069813550044211);
\begin{scriptsize}
\draw [fill=uuuuuu] (6.582731062360519,1.069813550044211) circle (2.0pt);
\draw[color=uuuuuu] (6.534049586776862,1.4052126220886587) node {$O$};
\draw [fill=uuuuuu] (4.515703038946392,0.407310575525049) circle (2.0pt);
\draw[color=uuuuuu] (4.445394440270476,0.6839504132231442) node {$I$};
\draw [fill=uuuuuu] (0.779076137488415,-0.7903153922135896) circle (2.0pt);
\draw[color=uuuuuu] (0.5085048835462079,-0.6984688204357588) node {$P$};
\draw [fill=uuuuuu] (3.44598165453877,-0.7712354655175488) circle (2.0pt);
\draw[color=uuuuuu] (3.333448534936141,-0.41296919609315924) node {$X$};
\draw [fill=uuuuuu] (9.745500983360207,-0.7261666146543425) circle (2.0pt);
\draw[color=uuuuuu] (9.99009767092412,-0.6834425244177271) node {$Y$};
\draw [fill=uuuuuu] (6.5957413189494885,-0.7487010400859457) circle (2.0pt);
\draw[color=uuuuuu] (6.729391435011273,-0.4430217881292224) node {$N$};
\draw [fill=uuuuuu] (6.608751575538458,-2.567215630216099) circle (2.0pt);
\draw[color=uuuuuu] (6.609181066867021,-2.772097670924113) node {$S$};
\draw [fill=uuuuuu] (4.524079566392375,-0.7635223954452082) circle (2.0pt);
\draw[color=uuuuuu] (4.625709992486855,-0.5181532682193801) node {$T$};
\end{scriptsize}
\end{tikzpicture}

We first establish a relation between the length $OI$ and the inradius $r$. 
Let the circumradius be $R=AO=BO=CO$. Using fresh new label than (a), we denote $M$ as the second intersection of $AI$ and the circumcircle of $ABC$. 
This $M$ is also the midpoint of arc $BC$ not containing $A$. 
Now, \[MI=MB=MC=2R\sin \angle BAI=2R\sin\frac{\angle A}{2}, 
AM=AB\frac{\sin\angle ABI}{\angle AIB}
=2R\sin\angle C \frac{\sin\frac 12 B}{\sin (90^{\circ}+\frac 12 C)}\]
$=2R(2\sin\angle \frac{\angle C}{2} \cos\angle \frac{\angle C}{2}) \frac{\sin\frac 12 \angle B}{\cos\frac 12 \angle C}$
$=4R\sin\frac 12\angle B\sin\frac 12\angle C$
so $MI\cdot AI=8R^2\sin\frac 12 \angle A\sin\frac 12\angle B\sin\frac 12\angle C$. 
Meanwhile, letting $D$ be the point of tangency of the incircle to $BC$ we have 
\[r=ID=IB\sin\angle IBC=BC\cdot\frac{\sin\angle ICB}{\sin\angle CIB}\cdot\sin\frac 12\angle B
=2R\sin A\cdot\frac{\sin\frac 12 C}{\sin (90^{\circ}+\frac 12 A)}\cdot\sin\frac 12\angle B\]
\[=2R\left(2\sin\frac{\angle A}{2} \cos\frac{\angle A}{2}\right)\cdot\frac{\sin\frac 12\angle C}{\cos\frac 12\angle A}\cdot\sin\frac 12\angle B
=4R\sin\frac 12 \angle A\sin\frac 12\angle B\sin\frac 12\angle C.\]
Considering the fact that $-MI\cdot AI$ is the power of point of $I$ w.r.t. the circumcircle we have 
$OI^2=R^2-MI\cdot AI$=$R^2-8R^2\sin\frac 12 \angle A\sin\frac 12\angle B\sin\frac 12\angle C$, 
or $\frac{OI^2}{R^2}=1-8\sin\frac 12 \angle A\sin\frac 12\angle B\sin\frac 12\angle C$ 
while $\frac {r}{R}=4\sin\frac 12 \angle A\sin\frac 12\angle B\sin\frac 12\angle C$ 
so $\frac{OI^2}{R^2}=1-\frac {2r}{R}$, 
or $\frac{r}{R}=\frac 12-\frac{OI^2}{2R^2}$, 
or $r=\frac 12R\left(1-\frac{OI^2}{R^2}\right)$ 
(this is actually a well-known identity, the purpose of including the proof is to show the power of trigonometry in solving problems). 

Now, let $T$ be the tangency point of the incircle to $XY$, and $N$ be the midpoint of $XY$. 
Keeping in mind that $OP\cdot OI=R^2$, 
we now have $\frac{PI}{OP}$
$=1-\frac{OI}{OP}$
$=1-\frac{OI}{R^2\div OI}$
$=1-\frac{OI^2}{R^2}$. 
Therefore $ON=IT\cdot\frac{PO}{PI}$
$=r\frac{1}{1-\frac{OI^2}{R^2}}$
$=\frac 12 R(1-\frac{OI^2}{R^2})\frac{1}{1-\frac{OI^2}{R^2}}$
$=\frac 12 R$. 
Moreover, letting $S$ be the midpoint of arc $XY$ lying on the opposite side as $I$ w.r.t. $XY$ we have 
$O, N, S$ collinear, $ON=NS$, and $ON\perp XY$. 
Therefore, $OX=OY=OS=XS=YS$, 
yielding $OXS$ and $OYS$ both equilateral and 
$\angle XOY=60^{\circ}+60^{\circ}=120^{\circ}$. 
Additionally, $PI\cdot PO=PO^2-(IO\cdot OP)=PO^2-R^2$, 
which is the power of point of $P$ w.r.t. the circumcircle. 
This, in turn, is equal to the value $PX\cdot PY$, so $IOYX$ is cyclic. 
Thus $\angle XIY=\angle XOY=120^{\circ}$. 

\end{itemize}


\newpage 
\section{Number Theory}
\begin{itemize}
\item[\textbf{N1}]
For any positive integer $k$, denote the sum of digits of $k$ in its decimal representation by $S(k)$. Find all polynomials $P(x)$ with integer coefficients such that for any positive integer $n \geq 2016$, the integer $P(n)$ is positive and $$S(P(n)) = P(S(n)).$$

\textbf{Thoughts.} 
The first thing to try is $c\cdot 10^k$ for $c\le 9$, of course, 
because $S(c\cdot 10^k)=c$ and it's easier to manipulate--provided the notion of expanding the polynomial in terms of its coefficients is intuitive enough for you. 
(Well I first tried $n=10^k$ that gives $S(P(10^k))=P(1)$, a constant). 
The next thing is that, if we space $k$ big enough, the numbers are likely to be in the form 
$(a_kc^k)(0\cdots 0)(a_{k-1}c^{k-1}(0\cdots 0)\cdots (0\cdots 0)a_0c^0$, 
provided $a_i\ge 0$ for all $i$. 
But we cannot simply make that assumption! 
Fortunately, the fact that $a_i< 0$ will cause unbounded amount of trailing 9's (when $k$ is large enough), which will be good for a contradiction. 
Having that in mind, we know that $P(c)$ is the sum of $a_kc^k, \cdots , a_0c^0$ and the sum of digits (as shown above) is 
$S(a_kc^k)+\cdots +S(a_0c^0)$. 
With the fact that $S(i)\le i$ with equality iff $i\le 9$, it's no longer difficult to complete the solution. 

\textbf{Answer.} Either the constant polynomial $P(x)\equiv c$ where $c\in\{1,2,\cdots ,9\}$, or the identity polynomial $P(x)\equiv x$.

\textbf{Solution.} 
Let's first show that they work. In the first case we have $S(P(n))=S(c)=c=P(\text{anything})=P(S(n))$, 
and in the second case $S(P(n))=S(n)=P(S(n))$. 

Now let $P(x)=\displaystyle\sum_{i=0}^k a_ix^i$. 
The first thing to do is to prove that $a_i\ge 0$, $\forall i\ge 0$. 
Indeed, let $n=c\cdot 10^m$ ($1\le c\le 9$) then we have 
$P(c)=P(S(n))=S(P(n))=S(P(c\cdot 10^m))$. 
Lt $d$ be such that 
$10^{d}>\max\{|a_i(9^i)| : i\in [0, k]\}$
For $m>d$ satisfying we have 
$P(c\cdot 10^m)=\displaystyle\sum_{i=0}^k a_i (c^i)(10^{mi})$. 
Let $a_j<0$ for some $j$. 
Now that 
\[\displaystyle\sum_{i=0}^{j-1} a_i (c^i)(10^{mi})
<\displaystyle\sum_{i=0}^{j-1} 10^{d}(10^{mi})
<\displaystyle\sum_{i=0}^{mj-m+d} 10^i
<10^{mj-m+d+1}
\le 10^{mj}\]
\[\Downarrow\]
\[P(c\cdot 10^m)=\displaystyle\sum_{i=0}^k a_i (c^i)(10^{mi})
=\displaystyle\sum_{i=0}^{j-1} a_i (c^i)(10^{mi})
+a_j(c^j)(10^{mi})
+\displaystyle\sum_{i=j+1}^{k} a_i (c^i)(10^{mi})\]
\[<10^{mj}
+a_j(c^j)(10^{mj})
+\displaystyle\sum_{i=j+1}^{k} a_i (c^i)(10^{mi})
\le\displaystyle\sum_{i=j+1}^{k} a_i (c^i)(10^{mi})\]
(the first inequality is due to our choice of $m$). 
As $P(x)>0$ for all $x\ge 2016$, the leading coefficient is positive so we can choose $j$ such that there exists an $l\ge 1$ satisfying 
$c_{j+l}>0$ and $c_{j+1}, c_{j+2}, \cdots c_{j+l-1}=0$. 
In a similar way we can also deduce that 
$P(c\cdot 10^m)$
$>\displaystyle\sum_{i=j+l}^{k} a_i (c^i)(10^{mi})$
$-10^{m(j+l)-m+d+1}$
Combining the inequalities and by assuming that $c_{j+1}, c_{j+2}, \cdots c_{j+l-1}=0$, $c_j<0$ and $c_{j+l}>0$ we have 
\[
\displaystyle\sum_{i=j+l}^{k} a_i (c^i)(10^{mi})
-10^{m(j+l)-m+d+1}
<P(c\cdot 10^m)
<\displaystyle\sum_{i=j+1}^{k} a_i (c^i)(10^{mi})
=\displaystyle\sum_{i=j+l}^{k} a_i (c^i)(10^{mi})\]
This means that there will be at least $m-d$ consecutive 9's as digit, meaning that $S(P(c\cdot 10^m))$ is at least $9(m-d)$. 
It follows that $P(S(c\cdot 10^m))=P(c)\ge 9(m-d)$ for all sufficiently large $m$. 
However, this is contradicted by the fact that $\lim_{m\to\infty}9(m-d)\to\infty$. 
Hence $c_i\ge 0$ for all $i$. 

Since $a_i (c^i)(10^{ni}) < 10^{(n+1)i}$ (because $a_i(c^i)<10^n$ by our choice of $n$), 
the numbers $P(c\cdot 10^n)$ are in the form of $(a_kc^k)(0\cdots 0) (a_{k-1}c^{k-1})(0\cdots 0)\cdots(0\cdots 0)(a_0c^0)$ when laid in decimal form. 
Therefore 
$S(P(c\cdot 10^n))$
$=\displaystyle\sum_{i=0}^k S(a_i (c^i))$, 
and 
$P(S(c\cdot 10^n))=P(c)=\displaystyle\sum_{i=0}^k a_i (c^i)$. 
Knowing that $S(x)\le x$ with equality holds if and only if $0\le x\le 9$ 
(indeed, if $k=\displaystyle\sum_{i=0}^k b_i(10^i)$ then $S(k)=\displaystyle\sum_{i=0}^k b_i$, 
so $k-S(k)=\displaystyle\sum_{i=0}^k b_i(10^i -1)\ge 0$, 
with equality holds iff $b_i=0$ for $i\ge 1$, )
we have $a_i(c^i)\le 9$ for all $c\in \{0,1,\cdots 9\}$. 
This means $k\le 1$ (if we assume that $a_k > 0$). 
If $k=0$ then we get $a_0\le 9$, yielding the constant solution. 
If $k=1$, then $9a_1\le 9$ (when $c=9$) and $a_1=1$, yielding $P(x)=x+c$ for some constant $c$ 
(and since $c=a_0$ we have $c=a_0\le 9$ too). 
This entails $S(P(n))=S(n+c)$ and $P(S(n))=S(n)+c$ for all $n\ge 2016$, 
and letting $n=10^d-1$ we have $S(n)=9d$, and for $c\ge 1$, $S(n+c)=S(10^k-1+c)=c$, which doesn't hold for $d=5$. 
Therefore $c=0$ and we get the identity polynomial. 


\newpage 

\item[\textbf{N2}]
 Let $\tau(n)$ be the number of positive divisors of $n$. Let $\tau_1(n)$ be the number of positive divisors of $n$ which have remainders $1$ when divided by $3$. Find all positive integral values of the fraction $\frac{\tau(10n)}{\tau_1(10n)}$.

\textbf{Thoughts.} The notion of number of divisors should immediately lead to the thoughts of prime factorizing $m=10n$, which immediately gives away the formula for $\tau(m)$. For $\tau_1(m)$ we need to realize that each such divisor has zero exponent of 3 and even sum of exponents across those prime divisors with remainder 2 mod 3. This is obviously a combinatorics problem: intuitively the number of such combinations must be about half of the original allowed combinations, but how to prove it? Knowing that "even+even=odd+odd=even" and "even+odd=odd+even=odd", one can simply use induction to finish the proof (and the construction showing that 2 and composite numbers work comes for free if we used this way!)

\textbf{Answer.} All composite numbers and 2. 

\textbf{Solution.} 
Let $m=10n$, with $m=3^y\cdot\displaystyle\prod_{i=1}^{k}p_i^{a_i}\cdot\displaystyle\prod_{i=1}^{l}q_i^{b_i}$ with $p_i\equiv 1\pmod{3}$ and $q_i\equiv 2\pmod{3}$. 
Now each divisor $d$ of $m$ must be in the form of $3^z\cdot\displaystyle\prod_{i=1}^{k}p_i^{c_i}\cdot\displaystyle\prod_{i=1}^{l}q_i^{d_i}$ satisfying\\ 
$\bullet$ $0\le z\le y$\\
$\bullet$ $0\le c_i\le a_i$\\
$\bullet$ $0\le d_i\le b_i$\\
from which we know 
\[\tau(m)=(y+1)\cdot\displaystyle\prod_{i=1}^{k}(a_i+1)\cdot\displaystyle\prod_{i=1}^{l}(b_i+1).\]
Now $d\equiv 0\pmod{3}$ if $z\ge 1$, and $(1)^{c_i}\cdot(-1)^{d_i}=(-1)^{d_i}\pmod{3}$ if $z=0$. 
So each divisor $d$ counts into $\tau_1(m)$ iff the additional constraint $z=0$ and $\displaystyle\sum_{i=1}^l d_i\equiv 0\pmod{2}$ is satisfied.  

We proceed with the following claim: 
the number of combinations $(d_1, d_2, \cdots , d_l)$ satisfying $\displaystyle\sum_{i=1}^{l}d_i$ even and $d_i\le b_i$ is 
$\lceil\frac{\prod_{i=1}^{l}(b_i+1)}{2} \rceil$. 
\begin{itemize}
\item[Case 1.] $b_i$ odd for some $i$, and w.l.o.g. let this $i$ be $l$. 
Now, let $x$ be the number of combinations $(d_1, d_2, \cdots , d_{l-1})$ ($d_i\le b_i$) satisfying $\displaystyle\sum_{i=1}^{l-1}d_i$ even, and $z$ be the number of combinations with corresponding odd sums. 
Considering $d_i=\{0, 2, \cdots, b_l-1\}$ and $d_i=\{1,3,\dots , b_l\}$ we have: 
the number of combinations $(d_1, d_2, \cdots , d_{l-1})$ ($d_i\le b_i$) satisfying $\displaystyle\sum_{i=1}^{l}d_i$ even
is $x+z+x+z+\cdots + x+z=(x+z)\cdot\frac{b_i+1}{2}$, 
and similarly $z+x++\cdots +z+x=(x+z)\cdot\frac{b_i+1}{2}$ for odd-sum combinations. 
Therefore there are equally many odd and even sum combinations, and we are done.

\item[Case 2.] Now let $b_i$ even for all $i$. Let $O$ be number of combinations with $\displaystyle\sum_{i=1}^{l}d_i$ odd and $E$ be combinations with $\displaystyle\sum_{i=1}^{l}d_i$ even. The claim is $E-O=1$. 
We induct on $l$. 
Base case $l=0$ yields 1 combination for even sum and 0 combination for odd sum, vacuously. 
Now let $l=k$ for some $k$ and we have $O'$ as the number of combinations $(d_1, d_2, \cdots , d_k)$ with $\displaystyle\sum_{i=1}^{k}d_i$ odd, and $E'$ as the number of combinations with $\displaystyle\sum_{i=1}^{l}d_i$ even. 
Now that $b_{k+1}$ is even, using the logic above the number of even combination is 
$E'+O'+E'+O'+\cdots +E'=E'(\frac{b_{k+1}}{2}+1)+O'(\frac{b_{k+1}}{2})$, 
and similarly the number of combinations yielding odd sum is 
$O'(\frac{b_{k+1}}{2}+1)+E'(\frac{b_{k+1}}{2})$.
This yields $E-O=E'-O'$ and by induction hypothesis this number is 1, so we are done. 
\end{itemize}

Summing above, $\tau_1 (m)=\displaystyle\prod_{i=1}^{k}(a_i+1)\lceil\frac{\prod_{i=1}^{l}(b_i+1)}{2} \rceil$, 
so the ratio now becomes 
$(y+1)\dfrac{\displaystyle\prod_{i=1}^{l}(b_i+1)}{\lceil\frac{\prod_{i=1}^{l}(b_i+1)}{2} \rceil}$. 
Equivalently, $2(y+1)$ when $b_i$ odd for some $b_i$, or $(y+1)\frac{2k+1}{k+1}$ otherwise (where $2k+1=\displaystyle\prod_{i=1}^{l}(b_i+1)$ here). 
The first case yields that the ratio $2(y+1)$ must be even; in the second case, we have ratio as $(y+1)\left(2-\frac 1{k+1}\right)=2(y+1)-\frac{y+1}{k+1}$ so $k+1|y+1$ and the ratio $(y+1)\frac{2k+1}{k+1}$ is divisible by $2k+1$.  
Notice, also, that $l\ge 2$ ($m=10n$ contains prime factors 2 and 5) so $2k+1=\displaystyle\prod_{i=1}^{l}(b_i+1)$ must be composite. 
So our integer ratio cannot be an odd prime. 

It remains to show that any even or composite numbers work. 
For even numbers $2k$, simply take $m=10\cdot 3^{k-1}$ (i.e. $n=3^{k-1}$) we have $\tau(m)=2\times 2\times k=4k$ and $\tau_1(m)=2$ (because the only suitable divisors for $\tau_1$ are 1 and 10). 
For odd composite number $xz$ with $x, z\ge 3$, take $m=3^{d}\cdot 2^{x-1}\cdot 5^{z-1}$ (or $n=3^d\cdot 2^{x-2}\cdot 5^{x-2}$), where $d=\lceil \frac{xz}2\rceil -1$. Now $\tau(m)=dxz$ and  $\tau_1(m)=d$, as proven above. 


\newpage

\item[\textbf{N3/IMO 4}]
A set of postive integers is called fragrant if it contains at least two elements and each of its elements has a prime factor in common with at least one of the other elements. Let $P(n)=n^2+n+1$. What is the least possible positive integer value of $b$ such that there exists a non-negative integer $a$ for which the set $$\{P(a+1),P(a+2),\ldots,P(a+b)\}$$is fragrant?

\textbf{Thoughts.} This is a problem requiring no more than number experimentation, in the form of "for each (small) $k$, how large can $\gcd(P(n), P(n+k))$ be? 
It's now immediate to see why $P(n)$ and $P(n+1)$ are relatively prime (otherwise the answer is 2 and it's too trivial to be on the IMO), which directly gives away the fact that $b=3$ doesn't work either. 
How about $P(n)$ and $P(n+2)$? The investigation of the first case also shows that $b=4$ fails, and further checking on $P(n)$ and $P(n+3)$ shows how $b=5$ fails too. 
Finally, for $b=6$ the relation between $\gcd(P(n), P(n+2))$, $\gcd(P(n), P(n+3))$ and $\gcd(P(n), P(n+4))$ will be good to construct an example (which cannot be determined by brute force since it's rather big!!!)
Ps: basic combinatorics skill needed to construct an example. 

\textbf{Answer.} $b=6.$

\textbf{Solution.}
Now $\{P(197), P(198), P(199), P(200), P(201), P(202)\}$ has 
$P(199)\equiv P(202)\equiv P(1)=3\equiv 0\pmod{3}$, 
$P(198)\equiv P(2)=7\equiv 0\equiv 21=P(4)\equiv P(200)\pmod {7}$, 
$P(197)\equiv P(7)=57\equiv 0\equiv 133=P(11)\equiv P(201)\pmod{19}$, so this 6-element set is fragrant. 

First, notice that $P(n)-P(n-1)=n^2+n+1-(n^2-n-1)=2n$, 
and knowing that $n^2+n+1\equiv n+n+1=2n+1\equiv 1\pmod{2}$, 
we know that if $p|P(n)$ and $p|2n$ then $p|n$ (since $P(n)$ is relatively prime to 2), 
and consequently $p|n^2+n$ and $p|1$, showing that $P(n)$ and $P(n-1)$ are relatively prime. 
This means, $b=2$ fails, and $b=3$ fails too sine $P(a+1)$ and $P(a+3)$ are both relatively prime to $P(a+2)$. 
(We will use profusely the fact that $P(a)$ and $P(a+1)$ cannot have any common prime factor throughout the solution). 

Now, for $b=4$ and $b=5$ our strategy is to determine an upper bound for $\gcd(P(n), P(n+c))$ for $c=2, 3$. 
Observe that $P(n+c)-P(n)=2cn+c^2+c=c(2n+c+1)$. 
For $c=2$ this is the same as $2(2n+3)$. 
If $p|P(n+2)$ and $p|P(n)$ then $p|2(2n+3)$, and therefore $p|2n+3$ with $P(n)$ being odd at all times. 
This entails $2n\equiv -3\pmod{p}$, 
and $0\equiv 4P(n)=4n^+4n+1=(2n)^2+2(2n)+1\equiv (-3)^2-3+1=7\pmod{7}$. 
Hence $p=7$ and $n\equiv 2\pmod{7}$. 
Now for $b=4$, knowing that $P(a+2)$ is relatively prime with $P(a+1)$ and $P(a+3)$, $P(a+2)$ must have a common prime factor with $P(a+4)$, 
and by the previous step this prime factor has to be 7. 
Similarly $P(a+1)$ and $P(a+3)$ must both be divisible by 7. 
This means $P(a+1), P(a+2), P(a+3), P(a+4)$ are all divisible by 7 for some $a$, contradicting that any two neighbouring elements are coprime. 

Finally for $b=5$ we investigate $c=3$ as in the previous paragraph. 
Now $3(2n+3+1)=3(2n+4)=3(2)(n+2)$. 
If a prime $p$ satisfies $p|P(n)$ and $p|P(n+3)$ simultaneously then either $p=3$ or $p|n+2$ 
(again $p$ must be relatively prime to 2 so this can be easily factored out). 
In the second case we have $n\equiv -2\pmod{p}$, 
so $P(n)\equiv P(-2)=4-2+1=3\equiv\pmod{p}$, forcing $p=3$ (no choice!). 
Thus viewing the set $\{P(a+1), \cdots , P(a+5)\}$ 
we know that $P(a+3)$ must have a common factor with $P(a+1)$ or $P(a+5)$, and by previous paragraph this common factor has to be 7. 
Thus neither of $P(a+2)$ nor $P(a+4)$ can be divisible by 7, and they cannot have common prime factor (again by previous paragraph). 
This entails $P(a+1)$ and $P(a+4)$ must have common factor, and by what we established earlier this factor must be 3. 
Similarly, $P(a+2)$ and $P(a+5)$ must both be divisible by 3. 
However, $P(a+1)$ and $P(a+2)$ are both divisible by 3, contradiction. 


\newpage

\item[\textbf{N4}]
Let $n, m, k$ and $l$ be positive integers with $n \neq 1$ such that $n^k + mn^l + 1$ divides $n^{k+l} - 1$. Prove that
\begin{itemize}
\item[$\bullet$] $m = 1$ and $l = 2k$; or
\item[$\bullet$] $l|k$ and $m = \frac{n^{k-l}-1}{n^l-1}$.
\end{itemize}

\textbf{Thoughts.} If we were to find all possible $n, m , k , l$, the problem would be way harder to solve. Nevertheless, the fact that the answer is 'provided' to us should give a hint on how to proceed. Bullet point 1 happens when we have $n^k+n^{2k}+1|n^{3k}-1$ and bullet point 2 happens when we have $n^k+n^{k-l}+\cdots +n^l+1|n^{k+l}-1$. Therefore, a natural way is to investigate both $n^{k+l}-1-(n^k-1)(n^k+mn^l+1)$ and $n^{k+l}-1-(n^l-1)(n^k+mn^l+1)$, and we can simplify the solution (by using the fact that $n$ and $n^k+mn^l+1$ are relatively prime) to $n^k+mn^l+1|mn^{l}+1-n^{k-l}-m$ and $n^k+mn^l+1|n^k-mn^{l-k}+(m-1)n^l$, depending on the cases (and it's not hard to see that the right hand size must be zero by comparing the sizes!)

\textbf{Solution.} We split our solution into two cases: 
\begin{itemize}
\item[Case 1.] $l\le k$. 
Now from $n^k + mn^l + 1|n^{k+l} - 1$, and from the fact that $(n^l-1)(n^k + mn^l + 1)$
$=n^{k+l}+mn^{2l}+n^l-n^k-mn^l-1$ we have 
$(n^{k+l}+mn^{2l}+n^l-n^k-mn^l-1)-(n^{k+l} - 1)$
$=mn^{2l}+n^l-n^k-mn^l$
$=n^l(mn^{l}+1-n^{k-l}-m)$
$=n^l(m(n^l-1)-(n^{k-l}-1))$
is divisible by $n^k + mn^l + 1$. 
Knowing that $\gcd(n, n^k + mn^l + 1)=\gcd(n, 1)=1$ we have 
$\gcd(n^l, n^k + mn^l + 1)=1$ 
so $m(n^l-1)-(n^{k-l}-1)$ is itself divisible by $n^k + mn^l + 1$. 
Now, $m(n^l-1)<mn^l<n^k + mn^l + 1$ and $n^{k-l}-1\le n^k-1<n^k + mn^l + 1$, meaning that 
$0<\frac{m(n^l-1)}{n^k + mn^l + 1}, \frac{(n^{k-l}-1)}{n^k + mn^l + 1}<1$. 
Therefore $|\frac{m(n^l-1)-(n^{k-l}-1)}{n^k + mn^l + 1}|<1$, and has to be 0 (since it is an integer). 
We thus have 
$m(n^l-1)=(n^{k-l}-1)$ and since $n>1$, 
$m=\frac{n^{k-l}-1}{n^l-1}$. 
Let $k-l=cl+d$ with $0\le d<l$, then $n^{k-l}=n^{cl}\cdot n^d\equiv(n^l)^c\cdot n^d\equiv 1\cdot n^d=n^d\pmod{n^l-1}$, 
and from $n^d<n^l$ we have $n^d\not\equiv 1\pmod{n^l-1}$ unless $d=0$. 
Therefore $l|k-l$, or $l|k$. 

\item[Case 2.] $l\ge k$. 
Similar to above we have 
$(n^k-1)(n^k + mn^l + 1)-(n^{k+l}-1)$
$=n^{2k}+mn^{k+l}+n^k-n^k-mn^l-1-(n^{k+l}-1)$
$=n^{2k}+mn^{k+l}-mn^l-n^{k+l}$
$=n^k(n^k-mn^{l-k}+(m-1)n^l)$ 
is divisible by $n^k + mn^l + 1$. 
Again by the logic above, 
$\gcd (n^k, n^k + mn^l + 1)=1$, 
which very well means that $n^k + mn^l + 1|n^k-mn^{l-k}+(m-1)n^l$. 
Again we have $n^k+(m-1)n^l<n^k+mn^l+1$ and $mn^{l-k}<n^k+mn^l+1$ so by the logic above, again, 
$n^k-mn^{l-k}+(m-1)n^l=0$. 
Rearranging the terms give: 
$m=\frac{n^l-n^k}{n^l-n^{l-k}}$. 
Now, if $m\ge 2$, then we have 
$n^l-n^k\ge 2n^l-2n^{l-k}$, or 
$2n^{l-k}\ge n^l+n^k > n^l=n^k(n^{l-k})$, 
or $2>n^k$, forcing $n=1$ (contradiction). 
Thus $m=1$ ($m$ must be positive) and we have $k=l-k$, or $l=2k$. 
\end{itemize}

\newpage

\item[\textbf{N5}]
Let $a$ be a positive integer which is not a perfect square, and consider the equation \[k = \frac{x^2-a}{x^2-y^2}.\]Let $A$ be the set of positive integers $k$ for which the equation admits a solution with $x>\sqrt{a}$, and let $B$ be the set of positive integers for which the equation admits a solution with $0\leq x<\sqrt{a}$. Show that $A=B$.

\textbf{Thoughts.} Despite the short solution, it isn't that obvious to think of how should the 'complementary' pairs $(x_1, y_1)$ be 'generated' from the existing pair $(x, y)$. In view of this, let me demonstrate a real small-case example (to see how things can be suggestive in the end). 

Start experimenting with numbers like $x=7, y=6$. Then $a=10$ and $k=3$ are viable solutions. As it turns out, $x=1, y=2$ is also a solution. We notice that: each pair $(x, y)$ have $|x-y|=1$ (i.e. same distance), the product of $x$'s is $7\times 1=7=10-3=a-k$ and the product of $y$'s is $2\times 6=12=10+2$. This thing doesn't seem to hold true when $|x-y|>1$, but experimenting the differences yield that the product of $x$'s is $a-k(|x-y|)^2$ (the product of $y$'s is, in general, messier and less illuminating. Nevertheless, knowing the supposed value of product of $x$'s suffices). 

Edit: as it turns out, we cannot take $0\le x-2kc<\sqrt{a}$ for granted: much more needs to be done. This doesn't seem to hold true for all $(x, y)$ with $x>\sqrt{a}$, but it holds for minimal such $a$. Some tedious inequality is needed as below, but this is not too hard to think of. 

\textbf{Solution.} 
To show that $k\in A\to k\in B$, let $x>\sqrt{a}$ for some $x$ satisfying the equation, such that $x$ is the minimal possible. 
It follows that $|y|<|x|$, and w.l.o.g. we can let $y\ge 0$. 
Denote $y=x-c$ and we have $a=x^2-k(x^2-y^2)=x^2-kc(2x-c)=x^2-2kcx+kc^2$. 
Let $x_1=x-2kc$ and $y_1=x_1+c$ and we have 
$\dfrac{x_1^2-a}{x_1^2-y_1^2}$
$=\dfrac{(x-2kc)^2-(x^2-kc(2x-c))}{(x_1-y_1)(x_1+y_1)}$
$=\dfrac{-2kc(2x-2kc)+kc(2x-c)}{-c(2(x-2kc)+c)}$
$=\dfrac{-kc(4x-4kc-2x+c)}{-c(2x-4kc+c)}$
$=k$. 
This means $k$ admits $(x_1, y_1)$ as well, and from $x_1=y_1+c<y_1$ we have $x_1<\sqrt{a}$. 
Now, $x_1<\sqrt{a}$ by the minimality of $x$; if $-\sqrt{a}<x_1\le 0$ then $k$ admits $(-x_1, y_1)$ and $k$ is in $B$. 
Obviously this conclusion is also true if $0\le x_1<\sqrt{a}$. 
If $x_1<-\sqrt{a}$, by the minimality of $x$ again, $|x_1|\ge x$, so $x_1\le -x$ and so $x-2kc\le -x$, or $x\le kc$. 
Now $a=x^2-k(x^2-y^2)>0$, or $ky^2>(k-1)x^2$, or $x-c=y>x\sqrt{\frac{k-1}{k}}>x\cdot \frac{k-1}{k}$, so $c<\frac xk$, or $ck<x$, contradiction. Hence we must have $k\in B$. 

Conversely, we want to show that $k\in B\to k\in A$. 
Let $0\le x<\sqrt{a}$ for some $x$ satisfying the equation; choose the maximal such $x<\sqrt{a}$ if there are many. 
It follows that $|y|>|x|$, and again we can assume . 
Denote $y=x+c$ and we have $a=x^2-k(x^2-y^2)=x^2-k(-c)(2x+c)=x^2+2kcx-kc^2$. 
Let $x_2=x+2kc$ and $y_2=x_1-c$ and we have 
$\dfrac{x_2^2-a}{x_1^2-y_2^2}$
$=\dfrac{(x+2kc)^2-(x^2+kc(2x+c))}{(x_2-y_2)(x_2+y_2)}$
$=\dfrac{2kc(2x+2kc)-kc(2x+c)}{c(2(x+2kc)-c)}$
$=\dfrac{kc(4x+4kc-2x-c)}{c(2x+4kc-c)}$
$=k$. 
This means $k$ admits $(x_2, y_2)$ as well, and from the maximality of $x$ and that $x_2=x+c>x$ we have $x_2>\sqrt{a}$. Therefore
$k\in A$ too (proving the inequality here is surprisingly easier!)

\newpage

\item[\textbf{N6}]
Denote by $\mathbb{N}$ the set of all positive integers. Find all functions $f:\mathbb{N}\rightarrow \mathbb{N}$ such that for all positive integers $m$ and $n$, the integer $f(m)+f(n)-mn$ is nonzero and divides $mf(m)+nf(n)$.

\textbf{Thoughts.} Again the most intuitive way is to substitute $m=n$, which gives $2f(m)-m^2|2mf(m)$, and can be translated into $2f(m)-m^2|m^3$ directly (notice that how things are easier if right hand side is defined in terms of only $m$ instead of $m$ and $f(m)$). Things are easy for $m=1$ (where $f(m)=1$). Having established that, we note the following: 
\begin{itemize}
\item [1.] We want to make good use of what we have, which is obviously, $f(1)=1$. With $n=1$ we have $f(m)-(m-1)|m^2-m+1$, forcing $f(m)\le m^2$. 
\item [2.] It's tricky to locate the divisors of $m^3$ in general, but not quite when $m$ is prime. This gives rise to $|2f(p)-p^2|=p^k$ where $k=0, 1, 2, 3$. All we need to do is to use case-by-case analysis on $m=p, n=1\to$ again something we have already known. 
\end{itemize}
Having known that $f(p)=p^2$ for all sufficiently large primes $p$, one just need to experiment what happens when we have arbitrary $m$ and $n=$prime number (and the solution can be completed with a little bit of gimmick from here!)

\textbf{Answer.} $f(x)=x^2$, $\forall x\in\mathbb{N}$. 

\textbf{Solution.}
It is easy to verify that $f(x)=x^2$ fulfills the condition because for every positive integers $m$ and $n$ we have 
$mf(m)+nf(n)=m^3+n^3=(m+n)(m^2-mn+n^2)=(m+n)(f(m)-mn+f(n))$. For the rest of the solution we will use profusely that $f(x)>0, \forall x\in\mathbb{N}$ and that $f(m)+f(n)-mn\neq 0$. 

We start with the two mini-observations: 
\begin{itemize}
\item [1.] Substituting $m, n\la 1$ we have $2f(1)-1|2f(1)$, or $2f(1)-1|1$. With $2f(1)-1=\pm 1$ and $f(1)>0$ we have \fbox{$f(1)=1$}. 
\item [2.] Substituting $n\la 1$ we have $f(m)-(m-1)|mf(m)+1$. Knowing that $mf(m)+1=m(f(m)-(m-1))+m^2-m+1$ we have 
$f(m)-(m-1)|m^2-m+1$. Now from the fact that $m^2-m+1=\left(m-\frac 12\right)^2+\frac 34$ and that $|f(m)-(m-1)|\le m^2-m+1$ we have \fbox{$f(m)\le m^2$, $\forall m\in\mathbb{N}$}. 
\end{itemize}

In view of (2), we aim to show that the equality holds for all positive integers $m$. Let's now focus on proving that $f(p)=p^2$ for all sufficiently large prime $p$'s. 
Substituting $m=n=p$ gives $2f(p)-p^2|2pf(p)=p(2f(p)-p^2)+p^3$, so 
$2f(p)-p^2|p^3$ and from $f(p)\le p^2$ we have 
$2f(p)-p^2\in\{p^2, p, 1, -1, -p\}$ (again it this value cannot be $-p^2$ or lower because $f(p)>0$. 
Therefore $f(p)\in\{p^2, \frac{p^2+p}2, \frac{p^2+1}2, \frac{p^2-1}2, \frac{p^2-p}2\}$. 
Now we check $n=1, m=p$ again and we have (from above) $f(p)-(p-1)|p^2-p+1$. 
We investigate the following cases: 
\begin{itemize}
\item [(a)] $f(p)=\frac{p^2+p}2$, 
then $\frac{p^2+p}2-(p-1)|p^2-p+1=2(\frac{p^2+p}2-(p-1))-1$, 
so $\frac{p^2+p}2-(p-1)\le 1, $
which doesn't hold for $p\ge 2$. 

\item [(b)] $f(p)=\frac{p^2+1}2$, 
then $\frac{p^2+1}2-(p-1)|p^2-p+1=2(\frac{p^2+1}2-(p-1))+p-2$, 
which means $\frac{p^2+1}2-(p-1)\le p-2$, not true for $p\ge 3$. 

\item [(c)] $f(p)=\frac{p^2-1}2$, 
then $\frac{p^2-1}2-(p-1)|p^2-p+1=2(\frac{p^2-1}2-(p-1))+p$, 
meaning  $\frac{p^2+1}2-(p-1)\le p$, not true for $p\ge 3$. 

\item [(d)] $f(p)=\frac{p^2-p}2$, 
then $\frac{p^2-p}2-(p-1)|p^2-p+1$. 
Observe that $2(\frac{p^2-p}2-(p-1))=p(p-1)-2(p-1)=(p-1)(p-2)$, 
so $p-1|2(p^2-p+1)$. Now $2(p^2-p+1)\equiv 2(1^2-1+1)=2\pmod{p-1}$, 
so $p-1\le 2$ or $p\le 3$. 
\end{itemize}
We therefore know that all four cases cannot hold for $p\ge 5$, so $f(p)=p^2$ for $p\ge 5$. 

Now, let $m$ be arbitrary and let $n=p$ for some prime $p$ we have 
$f(m)+p^2-mp|mf(m)+p^3=m(f(m)+p^2-mp)+p^3-mp^2+m^2p=m(f(m)+p^2-mp)+p(p^2-pm+m^2)$. 
Consider the ratio $\dfrac{p(p^2-pm+m^2)}{f(m)+p^2-mp}$
$=p\left(1+\dfrac{m^2-f(m)}{f(m)+p^2-mp}\right)$, 
and therefore $\dfrac{p(m^2-f(m))}{f(m)+p^2-mp}$
must be an integer. 
Choosing any $p>f(m)$ gives 
$p\nmid f(m)$, and hence $p\nmid f(m)+p^2-mp$, hence $p$ and $f(m)+p^2-mp$ are relatively prime. 
Therefore $\dfrac{m^2-f(m)}{f(m)+p^2-mp}$ is itself an integer. 
Fixing $m$ and choosing $p$ sufficiently large (remembering that there are infinitely many primes) and we have $p^2-mp+f(m)>|m^2-f(m)|$, so $f(m)-m^2$ must be zero. 

\end{itemize}



\end{document}