\documentclass[11pt,a4paper]{article}
\usepackage{amsmath, amssymb, fullpage, mathrsfs, bm, pgf, tikz}
\usepackage{mathrsfs}
\usetikzlibrary{arrows}
\setlength{\textheight}{10in}
%\setlength{\topmargin}{0in}
\setlength{\topmargin}{-0.5in}
\setlength{\parskip}{0.1in}
\setlength{\parindent}{0in}

\begin{document}
\newcommand{\la}{\leftarrow}
\newcommand{\lra}{\leftrightarrow}
\newcommand{\bbN}{\mathbb{N}}
\newcommand{\bbZ}{\mathbb{Z}}
\newcommand{\dsum}{\displaystyle\sum}
\newcommand{\dprod}{\displaystyle\prod}

\section*{Algebra}
\begin{enumerate}
	\item [\textbf{A1}] (IMO 4)
	Find all functions $f:\mathbb Z\rightarrow \mathbb Z$ such that, for all integers $a,b,c$ that satisfy $a+b+c=0$, the following equality holds:
	\[f(a)^2+f(b)^2+f(c)^2=2f(a)f(b)+2f(b)f(c)+2f(c)f(a).\]
	
	\textbf{Answer.} There are three families of functions, with $c$ denoting any integer constant: 
	\begin{itemize}
		\item $f(n) = cn^2$. 
		\item $f(n)=0$ if $n$ even, or $c$ if $n$ odd. 
		\item $f(n)=c$ if $n$ odd, $4c$ if $4\mid n-2$, $0$ if $4\mid n$. 
	\end{itemize}
	We can verify that those work (although in the actual IMO a detail verification is needed...erm screw that).\\
	\textbf{Solution.} Plugging $a=b=c=0$ gives $3f(0)^2=6f(0)^2$ which forces $f(0)=0$. 
	Plugging $(a, b, c)=(0, n, -n)$ gives $f(n)^2+f(-n)^2=2f(n)f(-n)$, i.e. $(f(n)-f(-n))^2=0$. This forces $f(n)=f(-n)$, so $f$ is an even function. 
	
	Now we proceed with the following by considering $c=-(a+b)$, bearing in mind that $f(c)=f(-c)$, too. We now have the following: 
	\[
	f(a+b)^2-2f(a+b)(f(a)+f(b))+(f(a)^2+f(b)^2 - 2f(a)f(b))=0
	\]
	which is essentially solving the quadratic equation on $f(a+b)$. Using the quadratic formula (and bearing in mind that $f(a)^2+f(b)^2 - 2f(a)f(b)=(f(a)-f(b))^2$) we have: 
	\begin{flalign*}
		f(a+b)&=\frac{2(f(a)+f(b))\pm \sqrt{4(f(a)+f(b))^2 - 4(f(a)-f(b))^2}}{2}
		\\&=(f(a)+f(b))\pm \sqrt{4f(a)f(b)}
		\\&=(f(a)+f(b))\pm 2\sqrt{f(a)f(b)}
	\end{flalign*}
	In particular, if $f(b)=0$ then $f(a+b)=f(a)$ so $f$ is of period $b$. Hence $f(1)=0$ implies $f\equiv 0$ (which can be put into any category above). We thus assume that $f(1)\neq 0$ in the future. 
	
	Now we see what happens when we fix $f(1)$. We have $f(2)=2f(1)\pm 2\sqrt{f(1)^2}=2f(1)\pm 2f(1)$, i.e. $f(2)=0$ or $f(2)=4f(1)$. 
	In the first case, by setting $b=2$ we have $f(a+2)=f(a)\pm 2\sqrt{0}=f(a)$, so $f(a+2)=f(a)$. 
	We now infer that $f(a)=0$ if $a$ even, and $f(a)=f(1)$ if $a$ odd. 
	
	Otherwise, we assume that $f(2)=4f(1)$. Then $f(3)=f(1)+f(2)\pm\sqrt{f(1)f(2)}$, and setting $f(2)=4f(1)$ we have $5f(1)\pm\sqrt2{4f(1)^2}=5f(1)\pm 4f(1)$, so we have two cases, depending whether $f(3)=f(1)$ or $f(3)=9f(1)$. 
	
	In the first case, $f(3)=f(1)$. We have $f(4)=f(1)+f(3)\pm 2\sqrt{f(1)f(3)}=2f(1)\pm 2f(1)$, which is equal to either $0$ or $4f(1)$. Also $f(4)=f(2)+f(2)\pm 2\sqrt{f(2)^2}=2f(2)\pm 2f(2)$ which is either $0$ or $4f(2)=16f(1)$. Since both $f(4)\in \{0, 4f(1)\}$ and $f(4)\in \{0, 16f(1)\}$, we have $f(4)=0$ and therefore $f$ is of period 4, in the form $c, 4c, c, 0$, etc. 
	
	Otherwise, $f(3)=9f(1)$. Now $f(k)=k^2f(1)$ for $k=0, 1, 2, 3$. We now show that this is true for all integer $k$. Since $f$ is even, we only need to prove this for all $k>0$. 
	We use induction: suppose that $f(n)=n^2f(1)$ for all $n=0, 1, \cdots , k$, with $k\ge 3$. We now consider $f(k+1)$: 
	\begin{itemize}
		\item considering $f(1)$ and $f(k)$, we have $f(k+1)=f(1)+f(k)\pm 2\sqrt{f(1)f(k)}=f(1)+k^2f(1)\pm 2\sqrt{k^2f(1)}=f(1)(1+k^2\pm 2k)=f(1)(k\pm 1)^2$, i.e. either $f(1)(k+1)^2$ or $f(1)(k-1)^2$. 
		
		\item considering $f(2)$ and $f(k-1)$, we have $f(k+1)=f(2)+f(k-1)\pm 2\sqrt{f(2)f(k-1)}=f(1)(2^2+(k-1)^2\pm 2(2)(k-1))=f(1)((k-1)\pm 2)^2$, i.e. $f(1)(k+1)^2$ or $f(1)(k-3)^2$. 
		
	\end{itemize}
	We thus infer than $f(k+1)=f(1)(k+1)^2$, done. 
	
	\item[\textbf{A3}] (IMO 2) Let $n\ge 3$ be an integer, and let $a_2,a_3,\ldots ,a_n$ be positive real numbers such that $a_{2}a_{3}\cdots a_{n}=1$. Prove that
	\[(1 + a_2)^2 (1 + a_3)^3 \dotsm (1 + a_n)^n > n^n.\]
	
	\textbf{Solution.} As per official solution, we will show that $f_k(x)=\frac{(x+1)^k}{x}\ge \frac{k^k}{(k-1)^{k-1}}$ with equality if and only if $x=\frac{1}{k-1}$. 
	Consider the derivative w.r.t. $x$: \[f_k'(x)=\frac{xk(x+1)^{k-1}-(x+1)^k}{x^2}=\frac{(x+1)^{k-1}}{x^2}(xk-(x+1))=\frac{(x+1)^{k-1}}{x^2}((k-1)x-1)\] 
	Since $\frac{(x+1)^{k-1}}{x^2}>0$, we have $f_k'(x)>0$ iff $(k-1)x-1>0$, i.e. $x>\frac{1}{k-1}$. Also $f_k'(x)<0$ iff $(k-1)x-1<0$, i.e. $x<\frac{1}{k-1}$. Thus $f_k$ is increasing when $x>\frac{1}{k-1}$ and $f_k$ is decreasing when $x<\frac{1}{k-1}$. We conclude that $f_k$ attains it minimum point at $x=\frac{1}{k-1}$, with $f_k(\frac{1}{k-1})=\frac{(1+1/(k-1))^k}{1/(k-1)}=\frac{k^k}{(k-1)^{k-1}}$. 
	
	Now, having established this, we have 
	\[\dprod_{k=2}^n (1+a_k)^k
	=\dprod_{k=2}^n \frac{(1+a_k)^k}{a_k}
	\ge \dprod_{k=2}^n\frac{k^k}{(k-1)^{k-1}}
	=\frac{n^n}{1^1}
	=n^n
	\]
	(since $\prod a_k=1$). However, if the equality were to hold, we have $a_k=\frac{1}{k-1}$, so $\prod a_k = \frac{1}{(k-1)!}<1$, which is a contradiction. Hence the equality cannot hold, so the inequality must be strict. 
	
	\item[\textbf{A4}] Let $f$ and $g$ be two nonzero polynomials with integer coefficients and $\deg f>\deg g$. Suppose that for infinitely many primes $p$ the polynomial $pf+g$ has a rational root. Prove that $f$ has a rational root.
	
	\textbf{Solution.} Let $p_1<p_2<\cdots$ be a sequence of prime numbers and $\{r_k\}_{k\ge 1}$ be the sequence of rational numbers, such that $(p_nf+g)(r_n)=0$. 
	
	We first show that the sequence $\{r_n\}$ must be bounded. Notice that $\{p_k\}_{k\ge 0}\to\infty$, and $p_n|f(r_n)|=|g(r_n)|$. Let $M_1$ be a real number such that $f(x), g(x)\neq 0$ for all $x>M_1$, and let $M_2$ be a real number such that $|f(x)|>|g(x)|$ for all $x>M_2$ (this is true since $\deg(f)>\deg(g)$ and therefore $\frac{|f|}{|g|}\to\infty$). 
	From $p_n|f(r_n)|=|g(r_n)|$ (since $p_n\ge 1$ for primes $p_n$) we have $|f(r_n)|\le |g(r_n)|$ since $|\cdot|$ is nonnegative. 
	We therefore know that $r_n\le\max\{M_1, M_2\}$ for all $n\ge 1$, and therefore $\{r_n\}$ is bounded. 
	
	Now, since $\{r_n\}$ is bounded, it has a convergent subsequence. By extracting the convergent subsequence and the corresponding $p_n$ responsible for them, may as well assume that $\{r_n\}$ is itself convergent, and let $\{r_n\}\to r_0$. We show that $r_0$ is rational and $f(r_0)=0$. Now that $f$ and $g$ are polynomials, they are continuous functions. Suppose that $f(r_0)>0$. Since $\{r_n\}\to r_0$, there exists an $N$ such that $f(r_n)>\frac{f(r_0)}{2}$. Since $\{r_n\}$ is also convergent (and bounded), $\{g(r_n)\}$ is also bounded: let $N$ be such that $|g(r_n)|<N$ for all $n\ge 1$. Since $\{p_n\}$ is increasing sequences of primes, it's also unbounded. Thus, there exists $p_n$ that's greater than $\frac{2N}{f(r_0)}$. Now, $p_n|f_n(r_n)|=|g_n(r_n)|$. However, 
	\[p_n|f_n(r_n)|>p_n|\frac{f(r_0)}{2}| > \frac{2N}{f(r_0)} \cdot \left|\frac{f(r_0)}{2}\right| = N > |g_n(r_n)|
	\]
	which is a contradiction. Hence we cannot have $f(r_0)>0$. Similarly we cannot have $f(r_0)<0$. Therefore $f(r_0)=0$. 
	
	Now since $\{r_n\}$ is a rational sequence, let's write $r_n=\frac{u_n}{v_n}$ with $p_n, q_n$ integers. By the rational root theorem, we have $v_n\mid p_na_k$ and $u_n\mid p_na_0+b_0$ where $a_k$ is the leading coefficient of $f$ (recall that $\deg f > \deg g$) and $a_0, b_0$ the constant coefficient of $f$ and $g$, respectively. 
	For clarity, let's focus on all $n$ such that $p_n>a_k$. Since $p_n$ is increasing sequence of primes, there's a $N$ such that $p_n>a_k$ for all $n\ge N$. We may now assume that $N=1$ (by truncating all the first $N-1$ terms) and that $p_n>a_k$ for all $n\ge 1$. 
	
	If $p_n\nmid v_n$ for infinitely many $v_n$, then $v_n\mid a_k$ for those all those $n$'s. Since there are infinitely many of those $n$'s, and $a_k$ has only finitely many divisors, there must be infinitely many $v_n$'s that are the same. Let $\{r_{n_k}\}$ be the subsequence such that $v_{n_k}$ are the same, say $v_0$. Since this subsequence also converges to $r_0$, $u_{n_k}$ must be the same (say $u_0$) for sufficiently large $k$, too. This forces $r_0=u_0/v_0$, and is thus rational. 
	
	Thus $p_n\mid v_n$ for all but finitely many $v_n$. Again by extracting the appropriate subsequence we may assume $p\mid v_n$ for all $n$. Let $v_n=p_nw_n$, then from $v_n\mid pa_k$ we have $w_n\mid a_k$. Again by the logic above, there are infinitely $w_n$ that are the same, say, $w_0$. Since $u_n\mid p_na_0+b_0$, we can write $u_n= \frac{p_na_0+b_0}{d_n}$, and therefore we have $r_n=\frac{u_n}{v_n}=\frac{p_na_0+b_0}{d_np_nw_n}$, with infinitely many of the $w_n$ equal to $w_0$. In other words we have yet another subsequence $r_n=\frac{p_na_0+b_0}{d_np_nw_0}$ converging to $r_0$, i.e. $\frac{p_na_0+b_0}{d_np_n}=\frac{a_0}{d_n}+\frac{b_0}{d_np_n}$ converging to $r_0w_0$. 
	If $d_n$ is unbounded, this gives a subsequence of $\frac{p_na_0+b_0}{d_np_n}$ converging to 0, so $r_0=0$. Otherwise, there exists infinitely many $d_n$ that's the same, say $d_0$, so we have $\frac{p_na_0+b_0}{d_np_n}=\frac{p_na_0+b_0}{d_0p_n}$ which converges to $\frac{a_0}{d_0}$ and therefore $r_0=\frac{a_0}{d_0w_0}$, establishing the claim. 
	\end{enumerate}
	
	\section*{Combinatorics}
	\begin{enumerate}
		\item [\textbf{C1}] Several positive integers are written in a row. Iteratively, Alice chooses two adjacent numbers $x$ and $y$ such that $x>y$ and $x$ is to the left of $y$, and replaces the pair $(x,y)$ by either $(y+1,x)$ or $(x-1,x)$. Prove that she can perform only finitely many such iterations.
		
		\textbf{Solution.} Let $M$ be the maximum of the $n$ numbers. We first notice that at each step, the two numbers $y<x\le M$ is changed to either $y+1$ and $x$ or $x-1$ and $x$. Since $y<x$, $y+1\le x\le M$, so neither $y+1, x, x-1$ can exceed $M$. Thus after any iterations, no number can exceed $M$, and thus the sum of the $n$ numbers is bounded by $Mn$. 
		On the other hand, after each iteration, the sum of the $n$ numbers either increases by 1 (in the case of $(y+1, x)$), or $x-y-1$ (in the case of $(x-1, x)$). Since $x-y-1\ge 0$ (as $x>y$), the sum of the numbers never decrease. But since the sum of the numbers cannot exceed $Mn$ either, it can only increase finitely many times (each increase adds to the sum by at least 1 since the numbers are all positive integers). 
		
		Thus if Alice were to do it infinitely many times, after one point the sum of the numbers remain the same. This falls into the case of $(x, y)\to (x-1, x)$ and when $x-y-1=0$, i.e. $y=x-1$. This is basically swapping the two adjacent numbers $(x, y)\to (y, x)$, subject to the constraint $y=x-1$. 
		We see that the set of the numbers (along with their frequency) do not change, so all those iterations are basically permutations of the numbers. 
		
		Let's now order the permutations of $n$ numbers in the following way: if $\sigma$ and $\gamma$ are two permutations, let $k_0$ be the leftmost index such that $\sigma(k_0)\neq \gamma(k_0)$. We say that $\sigma<\gamma$ iff $\sigma(k_0)<\gamma(k_0)$. Since $y<x$, each iteration changes the arrangement of the $n$ numbers to a smaller permutation. But since there are only $n!$ permutation, there's only finitely many possible such iterations too. This proves the problem. 
		
		\item [\textbf{C2}] Let $n \geq 1$ be an integer. What is the maximum number of disjoint pairs of elements of the set $\{ 1,2,\ldots , n \}$ such that the sums of the different pairs are different integers not exceeding $n$?
		
		\textbf{Answer.} $\lfloor \frac{2n-1}{5}\rfloor$\\
		\textbf{Solution.} Suppose that we have $k$ such pairs of numbers, then the $2k$ numbers must have sum at least $1+2+\cdots + 2k=k(2k+1)$. 
		Since these $k$ pairs have sum that are different integers not exceeding $n$, the sum is at most $n+(n-1)+\cdots + (n-k+1)=\frac{k}{2}\cdot (2n-k+1)$. 
		Thus, $k(2k+1)\le \frac{k}{2}\cdot (2n-k+1)$, which is the same as 
		$2(2k+1)\le 2n-k+1$. 
		Rearranging yields $5k\le 2n-1$, i.e. $k\le \frac{2n-1}{5}$. Since $k$ must be an integers, 
		$k\le \frac{2n-1}{5}$. 
		
		Now we show that this is an attainable bound, which we split into two cases: 
		\begin{itemize}
			\item For $n=5m+1$ and $5m+2$, we have (in both cases) $\lfloor \frac{2n-1}{5}\rfloor=2m$. 
			Consider the first $4m$ numbers and we pair them in the following fashion: 
			\[
			(1, 4m-1), (2, 4m-3), \cdots , (m, 2m+1), (m+1, 4m), (m+2, 4m-2), \cdots , (2m, 2m+2)
			\]
			these give the sum as following: 
			\[4m, 4m-1, \cdots , 3m+1, 5m+1, 5m, \cdots , 4m+2
			\]
			so these are $2m$ pairs of sum $3m+1, 3m+2, \cdots , 4m, 4m+2, 4m+3, \cdots , 5m, 5m+1$ which are all distinct and at most $5m+1\le n$. 
			
			\item Now for $n=5m+3, 5m+4, 5m+5$, $\lfloor \frac{2n-1}{5}\rfloor=2m+1$. 
			We now need the first $4m+2$ numbers and consider the following pairing: 
			\[
			(1, 3m+2), (2, 3m+3), \cdots , (m+1, 4m+2), 
			(m+2, 2m+2), (m+3, 2m+3), \cdots , (2m+1, 3m+1)
			\]
			these give the sum as the following:
			\[(3m+3), (4m+5), \cdots , (5m+1), (5m+3), 
			(3m+4), (3m+6), \cdots , (5m+2)
			\]
			which are the distinct $2m+1$ sums of all integers in the range $3m+3, 3m+4, \cdots , 5m+3$. 
		\end{itemize}
		
		\item [\textbf{C4}]
		Players $A$ and $B$ play a game with $N \geq 2012$ coins and $2012$ boxes arranged around a circle. Initially $A$ distributes the coins among the boxes so that there is at least $1$ coin in each box. Then the two of them make moves in the order $B,A,B,A,\ldots $ by the following rules:\\
		(a) On every move of his $B$ passes $1$ coin from every box to an adjacent box.\\
		(b) On every move of hers $A$ chooses several coins that were not involved in $B$'s previous move and are in different boxes. She passes every coin to and adjacent box.
		Player $A$'s goal is to ensure at least $1$ coin in each box after every move of hers, regardless of how $B$ plays and how many moves are made. Find the least $N$ that enables her to succeed.
	\end{enumerate}
	
	\section*{Geomtery}
	\begin{enumerate}
	\item[\textbf{G1}] (IMO 1) Given triangle $ABC$ the point $J$ is the centre of the excircle opposite the vertex $A.$ This excircle is tangent to the side $BC$ at $M$, and to the lines $AB$ and $AC$ at $K$ and $L$, respectively. The lines $LM$ and $BJ$ meet at $F$, and the lines $KM$ and $CJ$ meet at $G.$ Let $S$ be the point of intersection of the lines $AF$ and $BC$, and let $T$ be the point of intersection of the lines $AG$ and $BC.$ Prove that $M$ is the midpoint of $ST.$
	
	\textbf{Solution.} We first show that $A, F, L, J$ lie on a circle. Now that $BK$ and $BM$ are both tangent to the excircle $\omega$ of $ABC$, we have $\angle BKJ=BMJ=90^{\circ}$, and moreover $MJ=MK$, so $M$ and $K$ are symmetric w.r.t. $BJ$. 
	This means, $BJ\perp MK$. By some angle chasing we have 
	\begin{flalign*}
		\angle JFL &= \angle JFM
		\\ &=90^{\circ} - \angle KMF 
		\\ &= 90^{\circ} - (180^{\circ} - \angle KML) 
		\\ &= 90^{\circ} - 180^{\circ} + (180^{\circ} - \frac 12 \angle KJL)
		\\ &= 90^{\circ} - \frac 12 (180^{\circ} - \angle KAL)
		\\ &=\frac 12\angle KAL
		\\ &= \angle LAJ
	\end{flalign*}
	so $A, F, L, J$ are indeed concyclic. Since $\angle ALJ=90^{\circ}$, $\angle AFJ=90^{\circ}$, too, and thus $AS\perp BJ$. Combined with $MK\perp BJ$ we have $AS\parallel MK$. 
	But then $BK=BM$ so $BA=BS$, too. This gives $MS=MB+BS=BK+BA=AK$. 
	
	Similarly, $MT=AL$. Since $AK$ and $AL$ are both tangents to $\omega$, we have $AK=AL$, so $MS=MT$. Since $M, S, T$ are all on $BC$, they are collinear, so $M$ is indeed the midpoint of $ST$. 
	
	\item[\textbf{G2}] Let $ABCD$ be a cyclic quadrilateral whose diagonals $AC$ and $BD$ meet at $E$. The extensions of the sides $AD$ and $BC$ beyond $A$ and $B$ meet at $F$. Let $G$ be the point such that $ECGD$ is a parallelogram, and let $H$ be the image of $E$ under reflection in $AD$. Prove that $D,H,F,G$ are concyclic.
	
	\textbf{Solution.} Since $G$ and $H$ lie on the different side of $DF$, we are proving that $\angle DHF+\angle DGF=180^{\circ}$, which is the same as proving $\angle DEF+\angle DGF=180^{\circ}$. 
	
	There are a few ways to prove this (one of which is the trigonometric bash I submitted as my homework), but one way is to explore the similarities arising from the fact that $ABCD$ is cyclic. 
	Now, by some angle chasing we have (note the use of equal angle at parallelograms, and the use of exterior angle $\angle EDC + \angle ECD = \angle AED$ and $\angle FDE+\angle AED=\angle FAE$)
	\[
	\angle FDG = \angle FDE+\angle EDC + \angle CDG
	= \angle FDE+\angle EDC + \angle ECD
	= \angle FDE+\angle AED
	= \angle FAE
	= \angle FBE
	\]
	with the last equality following from the $ABCD$ is cyclic. 
	In addition, 
	\begin{flalign*}
		\frac{FD}{DG}
		&=\frac{FD}{CE}
		\\ & =\frac{FB\sin \angle FBD / \sin \angle FDB}{BE \sin\angle EBC / \sin\angle BCE} \text{ (sin rule on triangle} FBD)
		\\ & =\frac{FB\sin \angle EBC / \sin \angle BCE}{BE \sin\angle EBC / \sin\angle BCE} (\angle BCE=\angle FDB)
		\\ &=\frac{FB}{BE}
	\end{flalign*}
	so together with $\angle FDG=\angle FBE$, we can conclude that triangles $FDG$ and $FBE$ are similar. 
	In particular, $\angle DGF=\angle BEF$. But since $B, E, D$ are on a straight line in that order, we have $\angle BEF + \angle FED=180^{\circ}$, and therefore $\angle DGF + \angle FED=180^{\circ}$. 
	
	\item[\textbf{G3}] In an acute triangle $ABC$ the points $D,E$ and $F$ are the feet of the altitudes through $A,B$ and $C$ respectively. The incenters of the triangles $AEF$ and $BDF$ are $I_1$ and $I_2$ respectively; the circumcenters of the triangles $ACI_1$ and $BCI_2$ are $O_1$ and $O_2$ respectively. Prove that $I_1I_2$ and $O_1O_2$ are parallel.
	
	\textbf{Solution.} There are two things we need to prove: 
	\begin{itemize}
		\item Let $I$ be the incenter of triangle $ABC$. Then $CI\perp I_1I_2$. 
		
		\item $CI$ is also the radical axis of the circumcircles of $ACI_1$ and $BCI_2$. 
	\end{itemize}
	Once we establish the two, it will follow that $CI\perp O_1O_2$, and therefore with $CO\perp I_1I_2$ we have $O_1O_2\parallel I_1I_2$. 
	
	We will in fact show an overarching claim: $ABI_2I_1$ is cyclic. We have learned so many times that $DFCB$ is cyclic because $\angle CFB=\angle CDB=90^{\circ}$, therefore $\angle ADF=\angle ACB$, meaning that triangles $ADF$ and $ACB$ are similar. Moreover, their similitude is $\frac{AF}{AB}=\cos\angle CAB$. Thus, $\frac{AI_1}{AI}=\cos\angle CAB$, too. But then $I$ and $I_1$ both lie on the internal angle bisector of $\angle CAB$, so we have 
	\[II_1=IA-I_1A=IA(1-\cos\angle CAB)=IA(2\sin^2\frac{\angle CAB}{2})
	=2IA\sin^2 \angle IAB
	\]
	(notice the use of double angle formula: $\cos 2x=1-2\sin^2 x$)
	and similarly $II_2=2IB\sin^2\angle IBA$. Thus we now have 
	\[
	\frac{II_1\cdot IA}{II_2\cdot IB}
	=\frac{2IA\sin^2\angle IAB\cdot\angle IA}{2IB\sin^2\angle IBA\cdot IB}
	=\frac{IA^2}{IB^2}\cdot\frac{\sin^2\angle IAB}{\sin^2\angle IBA}
	=\frac{\sin^2\angle IBA}{\sin^2\angle IAB}\cdot\frac{\sin^2\angle IAB}{\sin^2\angle IBA}
	=1
	\]
	(notice the use of sine rule on triangle $IAB$ in the second last equality). 
	Thus $II_1\cdot IA=II_2\cdot IB$, and so by power of point theorem $I_1ABI_2$ is cyclic. 
	Moreover, the power of point from $I$ to the circumcircle of $ACI_1$ and $BCI_2$ are $II_1\cdot IA$ and $II_2\cdot IB$, respectively. 
	Since these two are equal, $I$ has equal power from the two circles, and hence lie on the radical axis of the two circles. Since $C$ lies on both the circles, $CI$ is the radical axis of these two circles. 
	Finally, if $G$ is the intersection from $CI$ to $I_1I_2$ then (the second equality is the exterior angle)
	\[\angle I_1IG+\angle II_1G
	=(\angle ICA+\angle IAC)+\angle IBA
	=\frac{\angle BCA +\angle BAC + \angle CBA}{2}
	=\frac{180^{\circ}}{2}
	=90^{\circ}
	\]
	which shows that $CI$ and $I_1I_2$ are indeed perpendicular. 
	
	\item[\textbf{G4}] Let $ABC$ be a triangle with $AB \neq AC$ and circumcenter $O$. The bisector of $\angle BAC$ intersects $BC$ at $D$. Let $E$ be the reflection of $D$ with respect to the midpoint of $BC$. The lines through $D$ and $E$ perpendicular to $BC$ intersect the lines $AO$ and $AD$ at $X$ and $Y$ respectively. Prove that the quadrilateral $BXCY$ is cyclic.
	
	\textbf{Solution.} W.L.O.G. assume $AB<AC$. 
	We first notice that, if $M$ is the second intersection of the angle bisector $AD$ of $\angle BAC$ and the circumcircle of $ABC$, then $BM=MC$, and therefore the perpendicular from $M$ to $BC$ (say, $M_1$) will be the midpoint of $BC$. 
	Since $DM_1=M_1E$, we also have $DM=MY$. 
	
	We now show that $AD\cdot DM=XD\cdot EY$. Now, $EY=DY\cos \angle DYE=2DM\cos\angle DYE$. 
	Moreover, we can use the well-known fact that the perpendicular from $A$ to $BC$ and $AO$ are the isogonal conjugate (i.e. the reflection in $AD$) and since $DX\perp BC$, we have $\angle XAD=\angle XDA$. Since $XD\perp EY$, too, $\angle XDA=\angle EYD$. 
	This also means $DX=XA$, and that $AD=2DX\cos\angle EYD$, and thus $DX=\frac{AD}{2\cos\angle EYD}$. Therefore we have 
	\[XD\cdot EY=\frac{AD}{2\cos\angle EYD}\cdot 2DM\cos\angle DYE=AD\cdot DM
	\]
	but by the power of point theorem (since $ABMC$ is cyclic) we have $AD\cdot DM=BD\cdot DC$. It follows that $XD\cdot EY=BD\cdot DC$ too. 
	
	Now reflect $Y$ in the perpendicular bisector of $BC$ to get $Y'$ and we have $XD\cdot DY'=BD\cdot DC$ since $X, D, Y'$ would be collinear. It follows that $BXCY'$ is cyclic. 
	But then $BCYY'$ is also cyclic (isoceles trapezoid), so the conclusion follows. 
	
	\item[\textbf{G5}] (IMO 5) Let $ABC$ be a triangle with $\angle BCA=90^{\circ}$, and let $D$ be the foot of the altitude from $C$. Let $X$ be a point in the interior of the segment $CD$. Let $K$ be the point on the segment $AX$ such that $BK=BC$. Similarly, let $L$ be the point on the segment $BX$ such that $AL=AC$. Let $M$ be the point of intersection of $AL$ and $BK$.
	
	\textbf{Solution.} Let $Y$ to be the orthocenter of triangle $AXB$. Since $YX$ is perpendicular to $AB$, $Y, X, C, D$ are collinear. Now, consider the circle $\omega_A$ centered at $A$ with radius $AC$, and circle $\omega_B$ centered at $B$ with radius $BC$. We now consider the pole $P'$ of $BX$ with respect to $\omega_A$. This pole $P'$ lies on the polar of $B$. Since $\angle BCA=90^{\circ}$, $BC$ is tangent to $\omega_A$ and since $CD$ is perpendicular to $AB$, $CD$ is indeed to polar of $B$ and $P'$ is on $CD$. 
	Moreover, by the definition of pole, $P'$ also lies on perpendicular line from $A$ to $BX$. Thus $P'A\perp BX$ and $P'X\perp AB$, showing that $P'$ is the orthocenter of $AXB$. Thus $P'=Y$, too, i.e. $Y$ is the pole of $BX$ w.r.t. $\omega_A$. Similarly $Y$ is also the pole of $AX$ w.r.t. $\omega_B$. But since $L$ is on both $BX$ and $\omega_A$, $YL$ is tangent to $\omega_A$, and similarly $YK$ is tangent to $\omega_B$. But since $Y$ is on $CD$, the radical axis of $\omega_A$ and $\omega_B$, we have $YK=YL$. Finally, $MK^2=YM^2-YK^2=YM^2-YL^2=ML^2$ so $MK=ML$ (we use the fact that $\angle YKM=\angle YLM=90^{\circ}$). 
	
	\item[\textbf{G6}] Let $ABC$ be a triangle with circumcenter $O$ and incenter $I$. The points $D,E$ and $F$ on the sides $BC,CA$ and $AB$ respectively are such that $BD+BF=CA$ and $CD+CE=AB$. The circumcircles of the triangles $BFD$ and $CDE$ intersect at $P \neq D$. Prove that $OP=OI$.
	
\end{enumerate}

\section*{Number Theory}
\begin{enumerate}
	\item [\textbf{N1}] Call admissible a set $A$ of integers that has the following property:
	If $x,y \in A$ (possibly $x=y$) then $x^2+kxy+y^2 \in A$ for every integer $k$.
	Determine all pairs $m,n$ of nonzero integers such that the only admissible set containing both $m$ and $n$ is the set of all integers.
	
	\textbf{Answer.} All $(m, n)$ satisfying $\gcd(m, n)=1$. \\
	\textbf{Solution.} First, let $d=\gcd(m, n)$. The set $d\bbZ=\{dn: n\in\bbZ\}$ contains $m$ and $n$, and is admissible since $d\mid x$ and $d\mid y$ implies that for all $k\in\bbZ$ we have $x^2, kxy, y^2$ are all divisible by $d$, so $d\mid x^2+kxy+y^2$. 
	So we need $d=1$ for $d\bbZ=\bbZ$. 
	
	On the other hand, suppose $\gcd(m, n)=1$. Letting $x=y=m$ we have $m^2+km^2+m^2=(k+2)m^2\in A$, so $m^2\bbZ\subseteq A$, and similarly $n^2\bbZ\subseteq A$. Since $\gcd(m^2, n^2)=1$ too, we can find $a$ and $b$ such that $am^2+bn^2=1$, by Euclidean theorem. Also $am^2, bn^2$ both $\in A$. 
	Now let $x=am^2$ and $y=bn^2$ and $k=2$ we have $(x+y)^2=x^2+2xy+y^2\in A$ but since $x+y=1$ we have $1\in A$. Finally, letting $x=y=1$ we have $1+k+1=k+2\in A$ for every integer $k$, so $A=\bbZ$. 
	
	\item[\textbf{N2}] Find all triples $(x,y,z)$ of positive integers such that $x \leq y \leq z$ and
	\[x^3(y^3+z^3)=2012(xyz+2).\]
	
	\textbf{Answer.} The only such triple is $(2, 251, 252)$. \\
	\textbf{Solution.} The left hand side is divisible by $x$ while the right hand side is congruent to $2012(2)=4024$ modulo $x$. We therefore have $x\mid 4024=503\times 2^3$, with 503 being a prime. 
	Next, we show that $x$ cannot be divisible by 4. Suppose it is, then the left-hand-side is divisible by $4^3=64$, and $4\mid x$ implies $503\times 2^3\times 2^2=2012(4)\mid 2012xyz$ so $2012xyz$ is divisible by $2^5=32$. It then follows that $4024$ is divisible by 32 (since $32\mid 64$) too, which is a contradiction. 
	Finally, we show that $x$ cannot be greater than $\sqrt[3]{2012}$. 
	For all $x\ge 2$ we have $xyz+2\le z^3+2<z^3+y^3$ since $x\le y\le z$, so either $x=1$ or $x^3<2012$, as claimed. 
	Thus the maximal possible $x$ is 11, but since it also has to be a divisor of $8\times 503$ and cannot be divisible by 4, we have either $x=1$ or $x=2$. 
	
	If $x=1$, we essentially have $y^3+z^3=2012(yz+2)$, and notice that $y^3+z^3$ is divisible by 2012. Since $503$ is a prime that's $\equiv 2\pmod{3}$, we have $y^3\equiv z^3\pmod{503}\to y\equiv z\pmod{503}$, so here $y^3\equiv(-z)^3\to y\equiv -z$, meaning $503\mid y+z$. 
	In addition, $y^3+z^3$ is even, so $y+z$ must also be even ($y$ and $z$ must be of the same parity). 
	It then follows that $y+z$ is divisible by 1006. 
	Let $y+z=1006k$, and $y^3+z^3=(y+z)(y^2-yz+z^2)=1006k(y^2-y(1006k-y)+(1006k-y)^2)$ while $2012(yz+2)=2012(y(1006k-y)+2)$. This gives $k(y^2-y(1006k-y)+(1006k-y)^2)=2(y(1006k-y)+2)$. 
	\begin{itemize}
		\item If $k=1$, we have 
		$(y^2-y(1006-y)+(1006-y)^2)=2(y(1006-y)+2)$, i.e. 
		$3y^2-3018y+1006^2=-2y^2+2012y+4$, i.e. 
		$5y^2-5030y+(1006^2-4)=0$. 
		This reduces to a quadratic equation in $y$ with discriminant 
		$5030^2-4(5)(1006^2-4)=5\times 1006^2+80=4(5)(503^2+4)$. 
		We see that $503^2+4\equiv 4+4\equiv 3$ modulo 5, so this discriminant is divisible by 5 but not $5^2$, and thus not a perfect square. It follows that this quadratic equation has no integer solution. 
		
		\item If $k=2$ we have 
		$y^2-y(2012-y)+(2012-y)^2 = y(2012-y)+2$, so 
		$3y^2-3(2012)y+2012^2=2012y-y^2+2$, i.e. 
		$4y^2-4(2012)y+(2012^2-2)=0$. 
		Again we treat it as a quadratic equation with discriminant 
		$4^2(2012^2)-4(2012^2-2)=4(4(2012^2)-(2012^2-2))$. Since $4\mid 2012$, we see the term $4(2012^2)-(2012^2-2)$ is divisible by 2 but not 4, hence cannot be a perfect square. This also means that $4(4(2012^2)-(2012^2-2))=2^2(4(2012^2)-(2012^2-2))$ cannot be a perfect square, showing that once again this qudratic equation cannot have solution. 
		
		\item Let's see what happens as $k\ge 3$. 
		We have $y+z=1006k$ and therefore by the power-mean inequality, 
		$y^3+z^3\ge 2(\frac{y+z}{2})^3=\frac{(1006k)^3}{4}=2\times 503^3\times k^3$. 
		On the other hand, on the right hand side we have 
		$2012(yz+2)\le 2012(\frac{(y+z)^2}{4}+2)=2012(\frac{(1006k)^2}{4}+2)=2012(503^2k^2+2)$. Combining these two inequalities give 
		\[2\times 503^3\times k^3\le 2012(503^2k^2+2)\stackrel{\div 1006}{\to}
		503^2\times k^3\le 2(503^2k^2+2)
		\]
		This means, $503^2k^2(k-2)\le 4$, and 
		we see that this inequality fails when $k\ge 3$, so no solution for $k\ge 3$. 
		
	\end{itemize}
	We now proceed to the second case: $x=2$, i.e. $8(y^3+z^3)=2012(2yz+2)$, or $y^3+z^3=503(yz+1)$. By the same logic as above, we need $503\mid y+z$. 
	If $y+z=503k$ then by the power-mean inequality (again) we have 
	$y^3+z^3\ge \frac{503^3k^3}{4}$ and 
	$503(yz+1)\le 503(\frac{503^2k^2}{4}+1)$, so 
	\[\frac{503^3k^3}{4}\le 503(\frac{503^2k^2}{4}+1)
	\stackrel{\times 4\div 503}{\to}
	503^2k^3\le 503^2k^2+4
	\]
	which implies $503^2k^2(k-1)\le 4$, as well. Again from here we can see $k\le 1$ so we only need to care about this case. Knowing this, we have $x+y=503$ and therefore $y^2-yz+z^2=yz+1$ becomes $(y-z)^2=1$. Since $z\ge y$, we have $z-y=1$ and $z+y=503$. This implies $y=251$ and $z=252$. We can check that this triple $(x, y, z)=(2, 251, 252)$ fulfills the problem condition, and is thus the only triple of our interest. 
	
	\item [\textbf{N3}] Determine all integers $m \geq 2$ such that every $n$ with $\frac{m}{3} \leq n \leq \frac{m}{2}$ divides the binomial coefficient $\binom{n}{m-2n}$.
	
	\textbf{Answer.} All $m$'s that are prime. \\
	\textbf{Solution.} We use the following formula to calculate the highest power of a prime $p$ dividing $n!$ : 
	\[
	v_p(n!)=\dsum_{k=1}^{\infty}\left\lfloor \frac{n}{p^k}\right\rfloor
	\]
	and notice that $\binom{n}{m-2n}=\frac{n!}{(m-2n)!(3n-m)!}$. 
	
	We first see what happens when $m$ is prime. Then for all $n$ within the stipulated condition we have $n<m$ and so $\gcd(n, m)=1$. Choose any $n$ arbitrary, subject to the inequality constraint. 
	Let $p$ to be any prime dividing $n$ and let $k_0=v_p(n)$. We now have 
	\[v_p\left(\frac{n!}{(m-2n)!(3n-m)!}\right)
	= v_p(n!) - v_p((m-2n)!) - v_p((3n-m)!)
	= \dsum_{k=1}^{\infty}\left\lfloor \frac{n}{p^k}\right\rfloor - \left\lfloor \frac{m-2n}{p^k}\right\rfloor - \left\lfloor \frac{3n-m}{p^k}\right\rfloor
	\]
	We first notice that for each $k$, $\left\lfloor \frac{n}{p^k}\right\rfloor - \left\lfloor \frac{m-2n}{p^k}\right\rfloor - \left\lfloor \frac{3n-m}{p^k}\right\rfloor\ge 0$ since $\lfloor a+b\rfloor \ge \lfloor a\rfloor + \lfloor b\rfloor$ for all real numbers $a$ and $b$. 
	It now remains to show that for all $k\le k_0$ we have $\left\lfloor \frac{n}{p^k}\right\rfloor - \left\lfloor \frac{m-2n}{p^k}\right\rfloor - \left\lfloor \frac{3n-m}{p^k}\right\rfloor \ge 1$. 
	Since $p^{k_0}\mid v_p(n)$, $\frac{n}{p^k}$ is an integer and therefore $\left\lfloor \frac{n}{p^k}\right\rfloor  = \frac{n}{p^k}$. 
	However, as $\gcd(n, m)=1$, $m$ is not divisible by $p$ and so neither is $m-2n$ nor $3n-m$. 
	We therefore have $\left\lfloor \frac{m-2n}{p^k}\right\rfloor < \frac{m-2n}{p^k}$ and 
	$\left\lfloor \frac{3n-m}{p^k}\right\rfloor < \frac{3n-m}{p^k}$. Therefore, 
	\begin{flalign*}
	\left\lfloor \frac{n}{p^k}\right\rfloor - \left\lfloor \frac{m-2n}{p^k}\right\rfloor - \left\lfloor \frac{3n-m}{p^k}\right\rfloor
	&= \frac{n}{p^k} - \left\lfloor \frac{m-2n}{p^k}\right\rfloor - \left\lfloor \frac{3n-m}{p^k}\right\rfloor
	\\& > \frac{n}{p^k} - \frac{m-2n}{p^k} -  \frac{3n-m}{p^k}
	\\& > 0
	\end{flalign*}
	and since the floor functions are integers, so is the differences and sums among them. Therefore $\left\lfloor \frac{n}{p^k}\right\rfloor - \left\lfloor \frac{m-2n}{p^k}\right\rfloor - \left\lfloor \frac{3n-m}{p^k}\right\rfloor\ge 1$ for all $k\le k_0$, establishing the claim. 
	Therefore this identity holds when $m$ is prime. 
	
	Next, let's see what happens if $m$ is composite. If $m$ is divisible by 2, choosing $n=\frac{m}{2}$ gives $\dbinom{n}{m-2n}=\dbinom{n}{0}=1$ so we have $n\nmid 1$. If $m$ is divisible by 3, choosing $m=\frac{m}{3}$ gives $\dbinom{n}{m-2n}=\dbinom{n}{n}=1$ so we have $n\nmid 1$, too. We can therefore assume that the smallest prime dividing $m$ is at least 5. 
	
	\item[\textbf{N6}] Let $x$ and $y$ be positive integers. If ${x^{2^n}}-1$ is divisible by $2^ny+1$ for every positive integer $n$, prove that $x=1$.
	
\end{enumerate}

\end{document}