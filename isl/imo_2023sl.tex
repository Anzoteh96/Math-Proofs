\documentclass[11pt,a4paper]{article}
\usepackage{amsmath, amssymb, fullpage, mathrsfs, bm, pgf, tikz, float}
\usepackage{mathrsfs,amsthm}
\usetikzlibrary{arrows}
\setlength{\textheight}{10in}
%\setlength{\topmargin}{0in}
\setlength{\topmargin}{-0.5in}
\setlength{\parskip}{0.1in}
\setlength{\parindent}{0in}

\begin{document}
	\newcommand{\la}{\leftarrow}
	\newcommand{\lra}{\leftrightarrow}
	\newcommand{\bbN}{{\mathbb N}}
	\newcommand{\bbZ}{{\mathbb Z}}
	\newcommand{\bbQ}{{\mathbb Q}}
	\newcommand{\bbR}{{\mathbb R}}
	\newcommand{\bbC}{{\mathbb C}}
	\newcommand{\bbH}{{\mathbb H}}
	\newcommand{\dfeq}{\stackrel{\mathrm{def}}{=}}
	\newcommand{\ra}{\rightarrow}
	\newcommand{\Span}{\mathrm{span}}
	\newcommand{\scrP}{\mathscr{P}}
	\newcommand{\rank}{\mathrm{rank}}
	\newcommand{\nullity}{\mathrm{nullity}}
	\newcommand{\Col}{\mathrm{Col}}
	\newcommand{\Row}{\mathrm{Row}}
	\newcommand{\tr}{\mathrm{tr}}
	\newcommand{\ol}{\overline}
	\newcommand{\norm}[1]{||#1||}
	\newcommand{\doubleline}[1]{\underline{\underline{#1}}}
	\newcommand{\elemop}[1]{\stackrel{#1}{\longrightarrow}}
	\newcommand{\Ind}{\mathrm{Ind}}
	\newcommand{\Res}{\mathrm{Res}}
	\newcommand{\End}{\mathrm{End}}
	\newcommand{\cl}{\mathrm{cl}}
	\newcommand{\code}[1]{\texttt{#1}}
	\newcommand\tab[1][0.5cm]{\hspace*{#1}}
	\newcommand{\<}{\langle}
	\renewcommand{\>}{\rangle}
	\newcommand{\qubits}[1]{|{#1}\rangle}
	\newcommand{\powset}{\mathcal{P}}
	\newcommand{\dsum}{\displaystyle\sum}
	\newcommand{\dprod}{\displaystyle\prod}
	
	\newtheorem{lemma}{Lemma}
	
	\title{Solution to IMO 2023 Shortlist}
	\date{}
	\maketitle
	\section*{Algebra}
	\begin{enumerate}
		\item [A1.] 
		Professor Oak is feeding his $100$ Pokémon. Each Pokémon has a bowl whose capacity is a positive real number of kilograms. These capacities are known to Professor Oak. The total capacity of all the bowls is $100$ kilograms. Professor Oak distributes $100$ kilograms of food in such a way that each Pokémon receives a non-negative integer number of kilograms of food (which may be larger than the capacity of the bowl). The dissatisfaction level of a Pokémon who received $N$ kilograms of food and whose bowl has a capacity of $C$ kilograms is equal to $\lvert N-C\rvert$.
		
		Find the smallest real number $D$ such that, regardless of the capacities of the bowls, Professor Oak can distribute food in a way that the sum of the dissatisfaction levels over all the $100$ Pokémon is at most $D$.
		
		\textbf{Answer.} 50. 
		
		\textbf{Solution.} 
		We consider in general that there are $n$ pokemons, and the total capacity of bowls (and amount of food distributed) is also $n$. 
		That way we show that the answer is $\frac{n}{2}$ if $n$ is even. 
		
		To show that we cannot improve from $\frac n2$, consider the scenario where the first $\frac n2$ bowls have capacity $\frac{1}{2}$ and the remaining $\frac n2$ bowls having capacity $\frac 32$. 
		It then follows that the dissatisfaction score for each bowl must be at least $\frac 12$ (since the food given must be an integer). 
		
		Now to show that $\frac n2$ is attainable, let $x_1, \cdots, x_n$ be the bowl capacity 
		(so $\sum x_i = n$), 
		and sort them according to their fractional part, 
		i.e. 
		$\{x_1\}\le \cdots \{x_n\}$. 
		Let $m < n$ be such that 
		$\sum \{x_i\} = m$, 
		which is itself an integer. 
		(Note that if $m = 0$ then $x_i$ are all integer, in which case a distribution of food with quantity $x_i$ is valid and giving 0 dissatisfaction score). 
		Distribute the food in quantities $y_1, \cdots, y_n$ such that 
		\[
		y_i = 
		\begin{cases}
			\lfloor x_i\rfloor & i\le n - m\\
			\lceil x_i\rceil & i\ge n - m\\
		\end{cases}
		\]
		We note that $\{x_{n - m + 1}\} > 0$. 
		Note that the average of 
		$\{x_i\}$ for $i\le n - m$ is at most $\frac{m}{n}$, 
		and for $i > n - m$ is at least $\frac{m}{n}$. 
		Therefore, we have the dissatisfaction score here as 
		\[
		\sum_{i\le n - m} \{x_i\}
		+\sum_{i > n - m} 1 - \{x_i\}
		\le \sum_{i\le n - m} \frac{m}{n}
		+\sum_{i > n - m} \frac{n-m}{n}
		=\frac{2m(n - m)}{n}
		\le\frac{(m + n - m)^2}{2n}
		=\frac{n}{2}
		\]
		as desired. 
		
		\item [A2.] 
		Let $\mathbb{R}$ be the set of real numbers. Let $f:\mathbb{R}\rightarrow\mathbb{R}$ be a function such that\[f(x+y)f(x-y)\geqslant f(x)^2-f(y)^2\]for every $x,y\in\mathbb{R}$. Assume that the inequality is strict for some $x_0,y_0\in\mathbb{R}$.
		
		Prove that either $f(x)\geqslant 0$ for every $x\in\mathbb{R}$ or $f(x)\leqslant 0$ for every $x\in\mathbb{R}$.
		
		\textbf{Solution.}
		Suppose $x_0, y_0$ are such that 
		$f(x_0 + y_0)f(x_0 - y_0) > f(x_0)^2 - f(y_0)^2$. 
		Reversing their role, we have 
		$f(x_0 + y_0)f(y_0 - x_0) \ge f(y_0)^2 - f(x_0)^2$. 
		Thus adding them up, we have 
		\[
		f(x_0 + y_0)(f(x_0 - y_0) + f(y_0 - x_0)) > 0
		\]
		In particular, setting $b_0 = x_0 - y_0$ we have 
		$f(b_0) + f(-b_0)\neq 0$. 
		
		Now for any $a$,
		let $x = \frac{a + b_0}{2}$ and $y = \frac{a - b_0}{2}$. 
		Then $x + y = a$ and $x - y = b_0$. 
		Plugging this back into the original equation we have 
		$f(a)f(b_0)\ge f(x)^2 - f(y)^2$ and 
		$f(a)f(-b_0)\ge f(y)^2 - f(x)^2$, and therefore 
		\[
		f(a)(f(b_0) + f(-b_0))\ge 0
		\]
		Given also that $f(b_0) + f(-b_0)\neq 0$, 
		we have $f(a)$ and $f(b_0) + f(-b_0)$ have the same sign 
		for all a$a$, as desired. 
		
		\item [A3.] (IMO 4)
		Let $x_1,x_2,\dots,x_{2023}$ be pairwise different positive real numbers such that
		\[a_n=\sqrt{(x_1+x_2+\dots+x_n)\left(\frac{1}{x_1}+\frac{1}{x_2}+\dots+\frac{1}{x_n}\right)}\]is an integer for every $n=1,2,\dots,2023.$ Prove that $a_{2023} \geqslant 3034.$
		
		\textbf{Solution.} 
		We show that $a_{n+2} - a_n\ge 3$ for all $n\ge 1$. 
		This way, $a_{2023}\ge a_1 + 3\cdot 1011 = 3034$, 
		since $a_1 = 1$. 
		
		To see why, we see that $\{a_n\}$ is strictly increasing, 
		so it suffices to show that for any $n$, we cannot have $a_{n+1} - a_n = a_{n+2}-a_{n+1}=1$. 
		Denote, now, $S_n = x_1 + \cdots + x_n$ and $T_n = \frac{1}{x_1} + \cdots + \frac{1}{x_n}$. 
		Now we have 
		\[
		a_{n+1}^2 = (S_n + x_{n+1})(T_n + \frac{1}{x_{n+1}})
		=a_n^2 + x_{n+1}T_n + \frac{S_n}{x_{n+1}} + 1
		\]
		using $S_nT_n = a_n$. 
		By AM-GM inequality, $x_{n+1}T_n + \frac{S_n}{x_{n+1}} \ge 2\sqrt{T_nS_n} = 2a_n$, 
		so $a_{n+1}^2 \ge a_n^2 + 2a_n + 1 = (a_n+1)^2$, 
		with equality if and only if $\frac{S_n}{x_{n+1}} = x_{n+1}T_n$.
		
		Therefore, if 
		$a_{n+1} - a_n = a_{n+2}-a_{n+1}=1$ for some $n$, 
		then 
		\[x_{n+1} = \sqrt{\frac{S_n}{T_n}}
		\qquad 
		x_{n+2} = \sqrt{\frac{S_{n+1}}{T_{n+1}}}
		= \sqrt{\frac{S_n + x_{n+1}}{T_n + \frac{1}{x_{n+1}}}}
		=\sqrt{\frac{S_n + \sqrt{S_n/T_n}}{T_n + \sqrt{T_n/S_n}}}
		=\sqrt{\frac{S_n}{T_n}}\]
		I.e. $x_{n+1}=x_{n+2}$. 
		This contradicts that $x_n$'s are pairwise distinct. 
		
		\item [A4.]
		Let $\mathbb R_{>0}$ be the set of positive real numbers. Determine all functions $f \colon \mathbb R_{>0} \to \mathbb R_{>0}$ such that\[x \cdot \left(f(x) + f(y)\right) \geq \left(f(f(x)) + y\right) \cdot f(y)\]for every $x, y \in \mathbb R_{>0}$.
		
		\textbf{Answer.} 
		The family of functions satisfying the condition is $f(x) = \frac{c}{x}$ for some constant $c > 0$. 
		
		\textbf{Solution}. 
		The given function works since both sides give 
		$c + \frac{cx}{y}$. 
		
		By plugging $x=y$, we obtain $f(f(x))\le x$. 
		Plugging $x = f(y)$ and using that $f$ takes positive values, 
		we get $f(f(y)) + f(y)\ge f(f(f(y))) + y$, 
		or $f(y) - f(f(f(y))) \ge y - f(f(y))$. 
		
		We show that the inequalities above are equality, i.e. 
		$f(f(x)) = x$ for all $x > 0$. 
		Suppose otherwise, and consider the sequence $\{x_n\}$ such that $x_{n+1} = f(x_n)$, 
		and $x_0$ any number satisfying $f(f(x_0)) < x_0$. 
		Then the inequality $f(y) - f(f(f(y))) \ge y - f(f(y))$ directly implies that 
		$x_{n+1} - x_{n+3} \ge x_n - x_{n+2}$ for all $n$, and therefore 
		\[
		x_0 - x_2 \le x_2 - x_4\le\cdots
		\]
		In other words, if $d\triangleq x_0-x_2 > 0$, 
		then $x_{2n}\le x_0 - 2nd$ which will eventually become negative as $n > \frac{x_0}{2d}$, and hence impossible. 
		
		Having established $f(f(x))$, 
		the given inequality now becomes $xf(x) \ge yf(y)$ for all pairs $(x, y)$. 
		By taking symmetry, we in fact have $xf(x) = yf(y)$, 
		meaning that the function $xf(x)$ is a constant and therefore in the form $\frac{c}{x}$. 
		
		\item [A6.] (IMO 3)
		For each integer $k\geq 2$, determine all infinite sequences of positive integers $a_1$, $a_2$, $\ldots$ for which there exists a polynomial $P$ of the form\[ P(x)=x^k+c_{k-1}x^{k-1}+\dots + c_1 x+c_0, \]where $c_0$, $c_1$, \dots, $c_{k-1}$ are non-negative integers, such that\[ P(a_n)=a_{n+1}a_{n+2}\cdots a_{n+k} \]for every integer $n\geq 1$.
	\end{enumerate}
    
    \section*{Combinatorics}
    \begin{enumerate}
    	\item [C1.] 
    	Let $m$ and $n$ be positive integers greater than $1$. In each unit square of an $m\times n$ grid lies a coin with its tail side up. A move consists of the following steps.
    	\begin{itemize}
    		\item select a $2\times 2$ square in the grid;
    		\item flip the coins in the top-left and bottom-right unit squares;
    		\item flip the coin in either the top-right or bottom-left unit square.
    	\end{itemize}
    	
    	
    	\textbf{Answer.} 
    	All $m, n$ such that $3\mid mn$. 
    	
    	\textbf{Solution.}
    	For necessity, we label the coordinates of the grids as $(i, j)$, 
    	where $i=1,2, \cdots, m$ from left to right and $j=1, \cdots, n$ from top to down. 
    	We categorize the grids into $C_0, C_1, C_2$ according to the remainder of $i + j$ of the coordinates $(i, j)$. 
    	Consider $N(i)$ as the number of coins in $C_i$ heads up; 
    	we have $N(i)=0$ for all $i$ to start with. 
    	Now, the grids in top-right and bottom-left squares have the same $i + j$ 
    	(hence in the same $C_k$); 
    	the top-left and bottom-right squares each occupy the remaining two different $C_k$s. 
    	It then follows that $N(i)$ changes (in either direction) by exactly 1, 
    	and hence changes in parity. 
    	This means that $N(0), N(1), N(2)$ have the same parity at all times. 
    	For all coins to be heads-up, 
    	the number of coins in $C_0, C_1$ and $C_2$ must have the same parity. 
    	Let $m'$ and $n'$ be the biggest integers such that 
    	$m'\le m, n'\le n$ and $3\mid m', 3\le n'$. 
    	Among the grids $(i, j)$ with either $i\le m'$ or $j\le n'$ (or both), 
    	there are equal number of squares in each $C_0, C_1, C_2$. 
    	If $3\nmid mn$, 
    	then the leftover $(m'-m)(n'-n)$ squares are nonempty, 
    	giving us the cases 
    	$1\times 1, 1\times 2, 2\times 1, 2\times 2$. 
    	In the first case, 
    	there's an extra $C_2$; 
    	in the second and third case, 
    	there's an extra $C_2$ and $C_3$; 
    	in the fourth case there's an extra $C_1$ and $C_2$, but two extra $C_3$'s. 
    	Hence the conditions cannot be fulfilled. 
    	
    	Next, for the case that $3\mid m$ or $3\mid n$, 
    	we show that we may flip three consecutive coins in the following manner: 
    	by relying solely on the three consecutive coins on the right, 
    	we have: 
    	\[
    	\begin{pmatrix}
    		0 & 0\\
    		0 & 0\\
    		0 & 0\\
    	\end{pmatrix}
    	\to 
    	\begin{pmatrix}
    		0 & 0\\
    		1 & 0\\
    		1 & 1\\
    	\end{pmatrix}
    	\to 
    	\begin{pmatrix}
    		1 & 0\\
    		0 & 1\\
    		1 & 1\\
    	\end{pmatrix}
    	\to 
    	\begin{pmatrix}
    		1 & 0\\
    		1 & 0\\
    		1 & 0\\
    	\end{pmatrix}
    	\]
    	and by relying solely on the coins on the left, we have: 
    	\[
    	\begin{pmatrix}
    		0 & 0\\
    		0 & 0\\
    		0 & 0\\
    	\end{pmatrix}
    	\to 
    	\begin{pmatrix}
    		1 & 1\\
    		0 & 1\\
    		0 & 0\\
    	\end{pmatrix}
    	\to 
    	\begin{pmatrix}
    		1 & 1\\
    		1 & 0\\
    		0 & 1\\
    	\end{pmatrix}
    	\to 
    	\begin{pmatrix}
    		0 & 1\\
    		0 & 1\\
    		0 & 1\\
    	\end{pmatrix}
    	\]
    	The algorithm is similar if we were to flip three coins in a row instead of in a column. 
    	
    	Thus if $3\mid m$, we may split the grid into $3\times 1$ subgrids and do each of those operations iteratively, 
    	depending whether each subgrids have neighbours on the left or on the right. 
    	The case where $3\mid n$ is similar. 
    	
    	\item [C2.]
    	Determine the maximal length $L$ of a sequence $a_1,\dots,a_L$ of positive integers satisfying both the following properties:
    	\begin{itemize}
    		\item every term in the sequence is less than or equal to $2^{2023}$, and
    		\item there does not exist a consecutive subsequence $a_i,a_{i+1},\dots,a_j$ (where $1\le i\le j\le L$) with a choice of signs $s_i,s_{i+1},\dots,s_j\in\{1,-1\}$ for which\[s_ia_i+s_{i+1}a_{i+1}+\dots+s_ja_j=0.\]
    	\end{itemize}
    	\textbf{Answer.} $2^{2024} - 1$. 
    	
    	\textbf{Solution. }
    	We first determine such a construction. 
    	Consider, now, the sequence $A_n$ defined in the following manner: 
    	$A_0=(1)$, and $A_{n+1}$ is concatenating $2 \times A_n$, 1, $2\times A_n$ 
    	where $\times$ means element-wise multiplication. 
    	For example, we have 
    	\[
    	A_1 = (2, 1, 2)\qquad A_2 = (4, 2, 4, 1, 4, 2, 4)
    	\qquad A_3 = (8, 4, 8, 2, 8, 4, 8, 1,8, 4, 8, 2, 8, 4, 8)
    	\]
    	It's not hard to see that $A_n$ has length $2^{n + 1} - 1$ by induction. 
    	To show that the second bullet point also apply, we induct on $n$, where base case $n = 0$ is clear. 
    	Thereafter, in $A_{n + 1}$, 
    	suppose $i\le j$ are such that 
    	$s_ia_i + \cdots+ s_ja_j = 0$, 
    	then $a_i + \cdots + a_j$ is even since $s_i, \cdots, s_j$ are $\{-1, 1\}$. 
    	Since $A_{n + 1}$ contains only a single odd number 1, 
    	it then follows that $(a_i, \cdots, a_j)$ must be in the left part of 
    	$2 \times A_n$ or right part of $2\times A_n$. 
    	In either case the second bullet point applies to $A_n$, 
    	so $s_ia_i + \cdots+ s_ja_j = 0$ can't hold in this case either. 
    	
    	To show that $2^{n + 1} - 1$ is the best possible, 
    	consider any sequence $(a_1, \cdots, a_k)$. 
    	We first assign $s_1, \cdots, s_k$ according to the following iterative protocol, 
    	keeping track of the partial sum: 
    	set $t_0 = 0$, $s_1 = 1$ and so $t_1 = a_1$; 
    	for all $i\le k$, 
    	\[
    	s_i = 
    	\begin{cases}
    		-1 & t_{i-1} > 0\\
    		1 & t_{i-1}\le 0
    	\end{cases}
    	\]
    	and we let $t_i = t_{i-1} + s_ia_i$. 
    	This way, we have $|t_i|\le a_i\le 2^n$. 
    	
    	We note that if for some $i > 0$, $t_i = 0$, then 
    	$s_1a_1 + \cdots + s_ia_i = 0$; 
    	if $t_i = t_j$ for some $i < j$, 
    	then $s_{i+1}a_{i+1} + \cdots + s_ja_j = 0$. 
    	Finally, if $|t_m| = 2^n$ for some $k=m\ge 2$, 
    	then with $a_m\le 2^n$ we have $t_{m-1}$ and $t_m$ the same sign. 
    	However, if $t_{m-1}\neq 0$, then $s_m$ has different sign than $t_m$, 
    	so $|t_m| = |t_{m-1} + s_ma_m| = ||t_{m - 1}| - a_m| < 2^n$, contradiction. 
    	Thus $t_{m -1} = 0$, 
    	i.e. we have $s_1a_1 + \cdots + s_ma_m = 0$. 
    	It then follows that $t_1, \cdots, t_k$ must all be distinct, 
    	taking values $\{-2^n, \cdots, -1, 1, \cdots, 2^n\}$, 
    	and cannot take both $-2^n$ and $2^n$ (since only $|t_1|$ is allowed to be $2^n$). 
    	It then follows that $k\le 2\cdot 2^n - 1 = 2^{n + 1} - 1$. 
    	
    	\item [C3.] 
    	Let $n$ be a positive integer. A Japanese triangle consists of $1 + 2 + \dots + n$ circles arranged in an equilateral triangular shape such that for each $i = 1$, $2$, $\dots$, $n$, the $i^{th}$ row contains exactly $i$ circles, exactly one of which is coloured red. A ninja path in a Japanese triangle is a sequence of $n$ circles obtained by starting in the top row, then repeatedly going from a circle to one of the two circles immediately below it and finishing in the bottom row. 
    	
    	In terms of $n$, find the greatest $k$ such that in each Japanese triangle there is a ninja path containing at least $k$ red circles.	
    	
    	
    	\textbf{Answer.} $1 + \lfloor \log_2 n\rfloor$. 
    	
    	\textbf{Solution.} Denote $f(n)$ as the answer when there are $n$ such rows. 
    	We note that $f$ is monotone in $n$, 
    	so it suffices to show that $f(2^k) \ge k + 1$ and $f(2^{k - 1}) \le k$.
    	
    	We first show that $f(2^{k} - 1) \le k$ by giving a Japanese triangle such that for each $m=0, \cdots, k - 1$, any ninja path cannot contain more than one red circle among rows $2^m, 2^m + 1, \cdots, 2^{m + 1} - 1$. 
    	Indeed, for each $\ell=0, \cdots, 2^m - 1$, 
    	on row $2^m + \ell$, we colour the $2\ell + 1$-th circle from the left red. 
    	Note that this is valid since $2^m + \ell\ge 2\ell + 1$ for all $\ell\le 2^m - 1$. 
    	Notice also that for each ninja path, as we traverse from row $j$ to row $j + 1$, 
    	the number of circles on the same row to the left of the path cannot decrease, 
    	and similarly so for the circles on the right (in fact one of the counts stay the same, 
    	the other one increases by 1). 
    	Now on our example, the number of circles on the right of red circle at row is $2^m + \ell$ is 
    	$2^m - \ell - 1$, which decreases with $\ell$. 
    	It then follows that no ninja path can pass through two of the red circles. 
    	
    	Now to show that $f(2^k)\ge k + 1$, we consider any Japanese triangle. 
    	For each $k, \ell$ with $1\le \ell \le k$, 
    	denote $a_{k, \ell}$ as the maximum number of red circles a ninja path can pass through, 
    	and also passing through the $\ell$-th circle of $k$-th row. 
    	We note that $a_{1, 1} = 1$, and for all $k\ge 2$ we have 
    	\[
    	a_{k, \ell}
    	=\begin{cases}
    		\max \{a_{k - 1, \ell}, a_{k - 1, \ell - 1}\} + 1 & \text{circle $(k, \ell)$ is red}\\
    		\max \{a_{k - 1, \ell}, a_{k - 1, \ell - 1}\} & \text{circle $(k, \ell)$ is not red}\\
    	\end{cases}
    	\]
    	where by convention $a_{k, 0} = a_{k, k + 1} = 0$. 
    	Denote $S_n = \sum_{m=1}^n a_{n, m}$. We show that for each $k$ and $\ell = 0, \cdots, 2^k - 1$, 
    	$S_{2^k + \ell}\ge k\cdot (2^k + \ell) + 2\ell + 1$. 
    	To this end, we show by induction that 
    	$\max_{m\le 2^k + \ell} \{a_{2^k + \ell, m}\}\ge k + 1$, 
    	and also $S_{2^k + \ell + 1} - S_{2^k + \ell}\ge k + 2$. 
    	
    	We first note that in a sequence 
    	$a_1, a_2, \cdots, a_n$, $M = \max \{a_1, \cdots, a_n\}$, 
    	and $b_1, \cdots, b_{n + 1}$ are such that $b_1 = 1, b_{n + 1} = a_n$, 
    	and $b_k = \max\{a_k, a_{k - 1}\}$, 
    	we have $\sum_{i=1}^{n + 1} b_i\ge M + \sum_{i=1}^n a_i$. 
    	To this end, let $b_m = M$, 
    	and we have $b_i\ge a_i$ and $b_{i + 1}\ge a_i$, 
    	giving 
    	\[\sum_{i=1}^{n + 1} b_i\ge \sum_{i=1}^{m} a_i + \sum_{j=m}^n a_i = a_m + \sum_{i=1}^n a_i = M + \sum_{i=1}^n a_i
    	\]
    	as desired. 
    	
    	It then follows that $S_{n + 1} - S_n\ge 1 + \max_{m\le n} a_{n, m}$, 
    	the extra $+1$ for the circle coloured red. 
    	We now show that $f(2^k)\ge k + 1$ and $S_{2^k + \ell}\ge k\cdot (2^k + \ell)+2\ell + 1$ by induction on $k$. 
    	Base case with $k=0$ is clear with $f(1)=S_1 = 1$. 
    	Now suppose that this is true for some $k$, where $f(2^k)\ge k + 1$, 
    	then for each $n\ge 2^k$ we have 
    	$\max_{m\le n} a_{n, m}\ge k + 1$ for $n\ge 2^k$ and so $S_{n + 1} - S_n\ge k + 2$, 
    	so $S_{2^k + \ell}\ge k2^k + 1 + \ell(k + 2) = k(2^k+\ell) + 2\ell + 1$ for each such $\ell$, 
    	and in particular, 
    	$S_{2^{k+1}}\ge k2^{k+1}+2^{k+1}+1=(k + 1)2^{k+1} + 1$. 
    	By pigeonhole principle, then, 
    	we have $\max_m \{a_{2^{k+1}, m}\}\ge \frac{(k + 1)2^{k+1} + 1}{2^{k+1}} > k + 1$, 
    	giving $f(2^{k + 1})\ge k + 2$, as desired. 
    	
    	\item [C4.] 
    	Let $n\geqslant 2$ be a positive integer. Paul has a $1\times n^2$ rectangular strip consisting of $n^2$ unit squares, where the $i^{\text{th}}$ square is labelled with $i$ for all $1\leqslant i\leqslant n^2$. He wishes to cut the strip into several pieces, where each piece consists of a number of consecutive unit squares, and then translate (without rotating or flipping) the pieces to obtain an $n\times n$ square satisfying the following property: if the unit square in the $i^{\text{th}}$ row and $j^{\text{th}}$ column is labelled with $a_{ij}$, then $a_{ij}-(i+j-1)$ is divisible by $n$.
    	
    	Determine the smallest number of pieces Paul needs to make in order to accomplish this.
    	
    	\textbf{Answer.} $2n - 1$. 
    	
    	\textbf{Solution.} To construct such a partition, 
    	we first split the $1\times n^2$ strip into $n$ equal pieces (of length $n$ each). 
    	For the $k$-th strip 
    	($k=1, \cdots, n - 1$),
    	we further divide into $(1, \cdots, k)$, 
    	and then $(k+1, \cdots, n)$, 
    	where the $k+1$-th row of the square can be filled with 
    	$(k+1, \cdots, n), (1, \cdots, k)$. 
    	The last $(1, \cdots, n)$ can itself be filled for the first row, $(1, \cdots, n)$, 
    	giving a total of $2n-1$ strips. 
    	
    	Now to show this is necessary, 
    	consider a set $S \subseteq \{0, 1, \cdots, n - 1\}$. 
    	Let $S = s_1, \cdots, s_k$ (all distinct). 
    	Let $k = |S|$, 
    	and consider the task of tiling an $1\times kn$ numbered $1, \cdots, kn$ onto 
    	a $k\times n$ square such that on the numbers on the $j$-th row 
    	are congruent to $s_j + 1, \cdots, s_j + n$ in that order. 
    	In this partition, denote a cut at $x$ if $x$ and $x + 1$ belong to different strips. 
    	We first consider the following claim. 
    	\begin{lemma}
    		Given an optimal partition of the strip fulfilling the congruence of set $S$. 
    		Then each cut at $x$ must have $x\equiv y\pmod{n}$ for some $ y\in\{0\}\cup S$. 
    	\end{lemma}
    	
    	\begin{proof}
    		Suppose $x\equiv y\pmod{n}$ with $1\le y\le n - 1$ and $y\not\in S$. 
    		Consider all cuts at $jn + y$ for some $j\in \{0, 1, \cdots, k - 1\}$. 
    		Note that since $y\not\in S$, 
    		$y$ and $y + 1$ must be adjacent squares of each of the $k$ rows. 
    		
    		Let $C = \{j: jn + y\text{ has a cut}\}$. 
    		If for some $j\in C$ the strip ending at $jn + y$ and the strip starting at $jn + y + 1$ are in the same row, 
    		then they must be adjacent within the row, 
    		and therefore can be merged. 
    		Otherwise, for each $j\in S$ we denote $s(j)$ be such that the strip starting at $jn + y + 1$ ends at $jn + y + s(j)$; 
    		denote $s_0$ as $\min_{j\in S} \{s(j)\}$. 
    		Then we can redo the partition such that 
    		for each $j\in C$, 
    		the strip ending at $jn + y$ would end at $jn + y + s_0$ instead, 
    		and the the strip starting at $jn + y + 1$ would start at $jn + y + s_0 + 1$ instead. 
    		Notice that the tiling onto the $k\times n$ board would still be valid, 
    		but we eliminated the strip starting at $jn + y + 1$ for those $j$'s with $s(j) = s_0$. 
    		Thus in either of the cases we can obtain a tiling with one fewer strips. 
    	\end{proof}
    	
    	Now we go back to the problem. We show that given this $S$, 
    	we need at least the following number of strips: 
    	\[
    	\begin{cases}
    		2k - 1 & 0\in S\\
    		2k & 0\not\in S\\
    	\end{cases}
    	\]
    	The original problem asks for $S = \{0, 1, \cdots, n - 1\}$, hence solving the problem. 
    	To show this claim, we use induction on $k$. 
    	For $k = 1$ we note that we need at least 1 strip and this equality holds only when we tile $1, 2, \cdots, n$ in that order. 
    	Now given a target set $S\subseteq \{0, \cdots, n - 1\}$ of size $\ge 2$ and suppose that our claim holds for all all proper subset of $S$. 
    	Consider any such partition / tiling. 
    	If each of the $k$ rows consists of at least 2 strips, 
    	then $\ge 2k$ strips are required, as claimed. 
    	Otherwise, one row has a single strip of length $n$; 
    	let the strip $s_a$ correspond to $a + 1, \cdots, a + n\pmod{n}$ in that order. 
    	Consider $S' = S\backslash \{a\}$, 
    	and consider our original $1 \times kn$ strip with this strip $s_a$ removed. 
    	Now the remaining $1\times (k - 1)n$ strip with the original cuts is a valid tiling of $S'$. 
    	If $a = 0$, we see that going from $S'$ to $S$ requires this extra strip 
    	(i.e. at $(a+1, \cdots, a+n)$). 
    	Otherwise, we see that $a\not\in S'$ but the resulting tiling with $s_a$ removed has a cut at $a$. 
    	By our lemma, such partition cannot be optimal. 
    	It then follows that going from $S'$ to $S$ requires at least 2 additional strips 
    	(to account for $s_a$ and the resulting suboptimality). 
    	This finishes the induction. 
    \end{enumerate}
    
    \section*{Geometry}
    \begin{enumerate}
    	\item [G1.]
    	Let $ABCDE$ be a convex pentagon such that $\angle ABC = \angle AED = 90^\circ$. Suppose that the midpoint of $CD$ is the circumcenter of triangle $ABE$. Let $O$ be the circumcenter of triangle $ACD$.
    	
    	Prove that line $AO$ passes through the midpoint of segment $BE$.
    	
    	\textbf{Solution.}
    	Let $M$ be the midpoint of $CD$, 
    	and $F, G$ the midpoints of $AC$ and $AD$, respectively. 
    	We have $MF\parallel AD$ and is the perpendicular bisector of $AB$ (since $MA=MB=ME$, and also $AF=CF=AB$), 
    	so in particular $AB\perp MF$ and so $BC\parallel AD$. 
    	Similarly, $DE\perp AC$. 
    	
    	Now let $BC$ and $DE$ intersect again at $N$, 
    	and $K$ the midpoint of $BE$. 
    	Then $A, B, E, N$ are concyclic with center $M$, 
    	and in addition $ACND$ is a parallelogram. 
    	Note that 
    	\[
    	\angle ABE = \angle AND
    	\qquad 
    	\angle AEB = \angle ANC = \angle NAD
    	\]
    	so triangles $ABE$ and $DNA$ are similar. 
    	Since $M$ is the midpoint of $AN$, we also have 
    	$ABK$ and $DNM$ similar. 
    	In particular, $\angle BAK = \angle NDM=\angle ACD$. 
    	Finally, given that $AD\perp AB$, 
    	$\angle BAO = 90^{\circ} - \angle DAO = \angle ACD = \angle BAK$, 
    	$A, O, K$ are collinear, as desired. 
    	
    	\item [G2.] 
    	Let $ABC$ be a triangle with $AC > BC,$ let $\omega$ be the circumcircle of $\triangle ABC,$ and let $r$ be its radius. Point $P$ is chosen on $\overline{AC}$ such taht $BC=CP,$ and point $S$ is the foot of the perpendicular from $P$ to $\overline{AB}$. Ray $BP$ mets $\omega$ again at $D$. Point $Q$ is chosen on line $SP$ such that $PQ = r$ and $S,P,Q$ lie on a line in that order. Finally, let $E$ be a point satisfying $\overline{AE} \perp \overline{CQ}$ and $\overline{BE} \perp \overline{DQ}$. Prove that $E$ lies on $\omega$.
    	
    	\textbf{Solution.}
    	Let us now show that $P$ is the incenter of triangle $EAB$. 
    	
    	Denote $O$ the circumcenter of triangle $ABC$. 
    	We first show that $QP=QC$. 
    	Indeed, notices that 
    	$\angle QPC = \angle APC = 90^{\circ} - \angle CAB 
    	= \angle OCB$, so with $CB = CP$ and $CO = QP$ we have triangles $CBO$ and $CPQ$ congruent.
    	Since $AE\perp CQ$, we have 
    	$\angle EAC = 90^{\circ} - \angle QCP = 90^{\circ} - \angle OCB = \angle CAB$. 
    	
    	Next we show also that $QD = QP$. 
    	By the properties of cyclic quadrilateral $DABC$, 
    	$CP=CB$ implies $DA=DP$. 
    	In addition, $\angle QPD = \angle SPB = 90^{\circ} - \angle DBA
    	= \angle ODA$, 
    	and therefore triangles $DOA$ and $DQP$ are congruent too 
    	(since $QP = DO$). 
    	This gives 
    	$\angle EBP = 90^{\circ} - \angle QDP = 90^{\circ} - \angle SPB = \angle DBA$. 
    	
    	Finally, we compute 
    	\[
    	\angle AEB
    	=180^{\circ} - \angle EAB  -\angle EBA
    	=2(90^{\circ} - \angle PAB - \angle PBA)
    	\]\[
    	=2(90^{\circ} - \angle CPB)
    	=180^{\circ} - \angle CPB - \angle CBP
    	=\angle ACB
    	\]
    	so $ECAB$ is indeed cyclic. 
    	
    	\item [G3. ] 
    	Let $ABCD$ be a cyclic quadrilateral with $\angle BAD < \angle ADC$. Let $M$ be the midpoint of the arc $CD$ not containing $A$. Suppose there is a point $P$ inside $ABCD$ such that $\angle ADB = \angle CPD$ and $\angle ADP = \angle PCB$.
    	
    	Prove that lines $AD, PM$, and $BC$ are concurrent.
    	
    	\textbf{Solution.}
    	Let $CP$ intersect $AD$ again at $A'$, 
    	and $DP$ intersect $BC$ again at $B'$. 
    	Note that $\angle A'DP = \angle ADP = \angle PCB = \angle PCB'$, 
    	so $A'DCB'$ is cyclic. 
    	
    	We now claim that $MC$ and $MD$ are both tangent to the circle $A'B'CD$ at $C$ and $D$, 
    	respectively. 
    	Observe that $\angle PA'D + \angle A'DP = \angle CPD = \angle ADB$. 
    	Denote $N$ as the intersection between $AD$ and $BC$. 
    	Then $\angle ADC - \angle BAD = \angle DNC$. 
    	Similarly we also get 
    	$\angle PDA' - \angle PA'D = \angle DNC$, 
    	and also $\angle BDA - \angle DAC = \angle DNC$ 
    	(since both $ABCD$ and $A'B'CD$ are cyclic). 
    	giving 
    	\[\angle PA'D = \frac{\angle ADB - \angle DNC}{2} = \frac{\angle DAC}{2}
    	=\frac{180^{\circ} - \angle DMC}{2}
    	=\angle MDC
    	\]
    	and similarly $\angle PA'D = \angle MCD$, 
    	so $MC$ and $MD$ are indeed tangent to this new circle. 
    	
    	To finish the solution, 
    	notice that by Brokard's theorem, $P, N$ and the point of intersection between $A'B'$ and $CD$ are self-polar. 
    	This means $PN$ passes through the pole of $CD$, 
    	which is just the point $M$. 
    	
    	\item [G4.] (IMO 2)
    	Let $ABC$ be an acute-angled triangle with $AB < AC$. Let $\Omega$ be the circumcircle of $ABC$. Let $S$ be the midpoint of the arc $CB$ of $\Omega$ containing $A$. The perpendicular from $A$ to $BC$ meets $BS$ at $D$ and meets $\Omega$ again at $E \neq A$. The line through $D$ parallel to $BC$ meets line $BE$ at $L$. Denote the circumcircle of triangle $BDL$ by $\omega$. Let $\omega$ meet $\Omega$ again at $P \neq B$. 
    	
    	Prove that the line tangent to $\omega$ at $P$ meets line $BS$ on the internal angle bisector of $\angle BAC$.
    	
    	\textbf{Solution.} 
    	Let $F$ be such that $AF$ is the diameter of $\Omega$, $Q$ on $BS$ with $PQ$ tangent to $\omega$ at $P$, and $T$ such that $PQ$ intersects $\Omega$ again at $T$.
    	By angle chasing, $P, D, F$ are collinear, and $ST\parallel PF$.
    	This immediately gives
    	$$
    	\frac{DP}{ST} = \frac{DQ}{QS}
    	$$Now, $\frac{DQ}{QS} = \frac{AD\sin\angle DAQ}{AS\sin\angle SAQ}$.
    	Since $SE=SF$, we also have $SPE$ and $FTS$ congruent, so $ST = PE$.
    	In addition, $\frac{DP}{PE} = \frac{DA}{AF}$, giving the identity
    	$$
    	\frac{AS}{AF}=\frac{\sin\angle DAQ}{\sin\angle SAQ}
    	$$Now if $Q'$ on $AS$ (i.e. $DS$) is such that $AQ'$ is internal angle bisector of $\angle BAC$,
    	then $\angle DAQ'=\frac{\angle B - \angle C}{2}$ while $\angle SAQ'=90^{\circ}$.
    	Meanwhile, $AF$ is the diameter and $AS$ subtends on the circumference at angle $\frac{\angle B - \angle C}{2}$
    	(both referring to $\Omega$),
    	so $\frac{AS}{AF}=\frac{\sin\angle DAQ}{\sin\angle SAQ}=\frac{\sin\angle DAQ'}{\sin\angle SAQ'}$.
    	This would imply $Q=Q'$, as desired.
    	
    	\item [G5.] 
    	Let $ABC$ be an acute-angled triangle with circumcircle $\omega$ and circumcentre $O$. Points $D\neq B$ and $E\neq C$ lie on $\omega$ such that $BD\perp AC$ and $CE\perp AB$. Let $CO$ meet $AB$ at $X$, and $BO$ meet $AC$ at $Y$.
    	
    	Prove that the circumcircles of triangles $BXD$ and $CYE$ have an intersection lie on line $AO$.
    	
    	\item [G6.] 
    	Let $ABC$ be an acute-angled triangle with circumcircle $\omega$. A circle $\Gamma$ is internally tangent to $\omega$ at $A$ and also tangent to $BC$ at $D$. Let $AB$ and $AC$ intersect $\Gamma$ at $P$ and $Q$ respectively. Let $M$ and $N$ be points on line $BC$ such that $B$ is the midpoint of $DM$ and $C$ is the midpoint of $DN$. Lines $MP$ and $NQ$ meet at $K$ and intersect $\Gamma$ again at $I$ and $J$ respectively. The ray $KA$ meets the circumcircle of triangle $IJK$ again at $X\neq K$.
    	
    	Prove that $\angle BXP = \angle CXQ$.
    	
    	
    	\item [G7.] 
    	Let $ABC$ be an acute, scalene triangle with orthocentre $H$. Let $\ell_a$ be the line through the reflection of $B$ with respect to $CH$ and the reflection of $C$ with respect to $BH$. Lines $\ell_b$ and $\ell_c$ are defined similarly. Suppose lines $\ell_a$, $\ell_b$, and $\ell_c$ determine a triangle $\mathcal T$.
    	
    	Prove that the orthocentre of $\mathcal T$, the circumcentre of $\mathcal T$, and $H$ are collinear.
    	
    	\textbf{Solution.} 
    	We introduce the following points, starting with $B_C$ as the reflection of $B$ in $CH$ 
    	and $C_B$ the reflection of $C$ in $BH$ 
    	(Hence $\ell_a$ is essentially $B_CC_B$). 
    	Note that $A_C, C_A$ and $B_A, A_B$ can be defined similarly. 
    	Let $\ell_a$ and $\ell_b$ intersect at $C'$; define $A'$ and $B'$ similarly. 
    	Also for convenience let $AH, BH, CH$ intersect $BC, CA, AB$ at $D, E, F$, respectively. 
    	
    	We first show that $ABC$ and $A'B'C'$ (i.e. $\mathcal T$) are similar, 
    	and that $AH=A'H, BH=B'H, CH=C'H$. 
    	Indeed, by te definition of $C_A$, $AH$ is the perpendicular bisector of $CC_A$, 
    	and also perpendicular bisector of $BB_A$. 
    	Therefore we have $\angle C_AAB_A = \angle BAC$. 
    	Next, we see that $AA_C=2AF$ and $AA_B = 2AE$ so $EF\parallel A_BA_C$. 
    	This gives the following sequence of similarity: 
    	\[
    	\triangle AA_BA_C\sim \triangle AEF\sim \triangle ABC\sim \triangle AB_AC_A
    	\]
    	which gives $\frac{AA_B}{AA_C}=\frac{AB_A}{AC_A}$, 
    	which also means triangles $AA_BB_A$ and $AA_CC_A$ are similar. 
    	Thus there's a spiral similarity bring these two triangles, giving, 
    	in directed angle, 
    	\[
    	\angle(AA_C, AA_B)=\angle (AB, AC) = \angle (C_AA_C, A_BB_A)
    	=\angle(A'A_C, A'A_B)
    	=\angle(\ell_b, \ell_c)
    	\]
    	thus giving $A, A', A_C, A_B$ concyclic. 
    	Since $AH=AA_C=AA_B$ due to the definition of $A_C$ and $A_B$, 
    	$H$ is the center of the circle and therefore $AH=A'H$. 
    	The equality $BH=B'H$ and $CH=C'H$ can be shown similarly, 
    	and also $\angle(\ell_a, \ell_c) = \angle(BA, BC)$, 
    	$\angle(\ell_a, \ell_b)=\angle(AC, BC)$, 
    	which finishes the claim 
    	(in fact, $A'B'C'$ is rotation of $ABC$, enlarged by some scale, then reflected). 
    	
    	Now let $H', O'$ be the orthocenter and circumcenter of $A'B'C'$, respectively,  
    	we have $A'H':B'H':C'H'=A'H:B'H:C'H$, 
    	so $H$ and $H'$ are both on the appolonius circles of $A'B'H', B'C'H', C'A'H'$ all through $H'$. 
    	Assume $H\neq H'$, otherwise we're done. 
    	This will mean that the three circles are coaxial (they can't all be the same circle since the centers are on $A'B', B'C', C'A'$ which are not concurrent). 
    	Thus the goal now becomes showing that $O'$ lies on the radical axis $HH'$. 
    	
    	Indeed, we note that the reflection $H'_A$ of $H'$ in $B'C'$ also lies on the appolonius circle of 
    	$B'C'H'$ through $H'$. 
    	Let $P_A$ be the center of this appolonius circle, 
    	we have $P_AH'_A$ tangent to circumcircle of $B'C'H'_A$. 
    	Given also that $H'_A$ also lies on circle $A'B'C'$, 
    	this means that the appolonius circle and the circle $A'B'C'$ are orthogonal to each other. 
    	We thus get that the power of point of $O'$ w.r.t. this appolonius circle is $r^2$, the square of the circumradius of $A'B'C'$. 
    	In the same way we also deduce that the power of point of $O'$ w.r.t. the other two appolonius circles are also $r^2$, 
    	showing that $O'$ indeed lies on the given radical axis. 
    \end{enumerate}
    
    \section*{Number Theory}
    \begin{enumerate}
    	\item [N1.] (IMO 1) Determine all composite integers $n>1$ that satisfy the following property: if $d_1$, $d_2$, $\ldots$, $d_k$ are all the positive divisors of $n$ with $1 = d_1 < d_2 < \cdots < d_k = n$, then $d_i$ divides $d_{i+1} + d_{i+2}$ for every $1 \leq i \leq k - 2$.
    	
    	\textbf{Answer.} All prime powers $n = p^m$ for $m\ge 2$. 
    	
    	\textbf{Solution.} 
    	To show prime powers work, 
    	notice that the divisors are $1, p, \cdots, p^m$, 
    	and we naturally have $p^u\mid p^{u+1} + p^{u+2}$ for all $u\ge 0$. 
    	
    	To show such $n$ must be prime powers, 
    	consider $d_{k-2} \mid d_{k-1} + d_k = d_{k-1} + n$, 
    	so $d_{k-2} \mid d_{k-1}$. 
    	By the way the divisors are ordered, we have $d_ud_{k - u + 1} = n$, 
    	so this means $d_2\mid d_3$. 
    	Notice also that $d_2$ must also be the smallest prime divisor of $n$. 
    	If $n$ is not a prime power, then there exists another prime divisor $q$ of $n$. 
    	Choose such $q > d_2$ that is the smallest possible 
    	(among the prime divisors of $n$ that's not $d_2$), 
    	and let $q=d_u$, where $u\ge 4$ (since $d_2\mid d_3$).  
    	Then $d_2, \cdots, d_{u-1}$ are divisible by $d_2$ but $d_u$ is not, 
    	so $d_{u-2}\nmid d_{u-1}+d_u$. 
    	
    	\item [N2.] 
    	Determine all ordered pairs $(a,p)$ of positive integers, with $p$ prime, such that $p^a+a^4$ is a perfect square.
    	
    	\textbf{Answer}. $(a, p) = (1, 3), (2, 3), (6, 3), (9, 3)$. The corresponding numbers $p^a+a^4$ are $2^2, 5^2, 45^2$ and $162^2$. 
    	
    	\textbf{Solution.} Let $p^a + a^4 = b^2$, 
    	then $p^a = (b - a^2)(b + a^2)$. 
    	Since $a > 0$, 
    	this means there are integers $x < y$ such that $x + y = a$, 
    	$b - a^2 = p^x$ and $b + a^2 = p^y$. 
    	It then follows that $2(x + y)^2 = 2a^2 = p^y - p^x$. 
    	
    	We separate this into $p = 2$ and $p\ge 3$ odd. 
    	If $p = 2$, then $2(x + y)^2 = 2^x(2^{y - x} - 1)$. 
    	Notice that $2^{y - x} - 1$ is odd, $2^x$ is even, 
    	and the power of each odd prime dividing $2a^2$ is even. 
    	It then follows that $2^{y - x} - 1$ is a perfect square. 
    	Since $2^{y - x} - 1\equiv 3\pmod{4}$ for $y - x\ge 2$ and 
    	cannot be a perfect square, 
    	we have $y - x = 1$ so $a = 2x + 1$ and $2^y - 2^x = 2^x$. 
    	Now $2x + 1$ is a power of 2, 
    	so $x = 0$ is necessary. On the other hand this forces RHS to be $2^x = 1$ which is not $2\times$ a perfect square. 
    	
    	Hence $p$ is now odd, 
    	and we have $2(x + y)^2 = p^y - p^x\ge 3^y - 3^x$. 
    	The highest power of $p$ dividing $p^y - p^x$ is $x$ and the highest power of $p$ dividing $2(x+y)^2$ is even, 
    	so $x$ must be even. 
    	
    	We first show that $x\le 4$. 
    	To start with, we first note that $\frac{3^{y + 1} - 3^x}{3^y - 3^x} > \frac{3^{y + 1} - 3^{x + 1}}{3^y - 3^x} = 3$, 
    	while for $a \ge 2$ we have $2(a + 1)^2 \le 2a^2\frac{9}{4} < 2a^2\cdot 3$. 
    	Thus, for $x\ge 2$ (and therefore $x + y\ge 2x+1\ge 5$), 
    	the fact that $2(x + y)^2\ge 3^y - 3^x$ means 
    	$2(2x + 1)^2 \ge 3^{x + 1} - 3^x = 2\cdot 3^x$, 
    	i.e. $(2x + 1)^2\ge 3^x$. 
    	We see that equality holds as $x = 4$, 
    	and for $x\ge 4$, $\frac{(2(x+1)+1)^2}{(2x + 1)^2}\le \frac{11^2}{9^2} < 3 = 3^x$, 
    	and therefore $x\le 4$ must hold. 
    	From our example, since $(2x + 1)^2= 3^x$ at $x=4$, 
    	for $x=4$ we have $2(x + y)^2 = p^y - p^x$ implies $p=3$, $y=5$, 
    	giving the $a=4 + 5 = 9$. 
    	
    	Thus we may now focus on $x = 0$ and $x = 2$. 
    	When $x = 0$, we have $2y^2 = p^y - 1$, 
    	which $y = 1, p = 3$ and $y=2, p = 3$ both work. 
    	for $y \ge 2$, we similarly have $\frac{p^{y + 1} - 1}{p^y - 1} > p$ and also 
    	$\frac{(y + 1)^2}{y^2}\le \frac{9}{4} < 3$, 
    	and therefore we have $y = 1, 2$ (i.e. $a=1, 2$) here. 
    	
    	Finally, we consider $x = 2$. Now if for some $y\ge 3$, 
    	$2(2 + (y + 1))^2 \ge 3^{y + 1} - 3^2$ holds then 
    	we also have $2(2 + y)^2 \ge 3^{y} - 3^2$. 
    	However, when $y = 5$, 
    	$2(2 + 5)^2 = 98$ while $3^5 - 3^2 = 234$, 
    	forcing $y\le 4$. 
    	When $y = 3$, we have $2(2 + 3)^2 = 50$, 
    	and $p^3 - p^2 = p^2(p - 1)$, 
    	where $p=3$ gives 18 while $p\ge 5$ gives $\ge 100$. 
    	When $y = 4$, $2(2 + 4)^2 = 72$, 
    	$p^4 - p^2 = p^2(p^2 - 1)$. 
    	Here, $p = 3$ works, giving the solution as $a = 2 + 4 = 6$. 
    	This gives us all the solutions. 
    	
    	\item [N3.] For positive integers $n$ and $k \geq 2$, define $E_k(n)$ as the greatest exponent $r$ such that $k^r$ divides $n!$. Prove that there are infinitely many $n$ such that $E_{10}(n) > E_9(n)$ and infinitely many $m$ such that $E_{10}(m) < E_9(m)$.
    	
    	\textbf{Solution.} 
    	Note that $E_{10}(m) = \min\{E_2(m), E_5(m)\}$. 
    	Also, for each prime $p$, 
    	we have $E_p(m) = \sum_{i = 1}^{\infty}\lfloor \frac{m}{p^i}\rfloor$. 
    	In other words, $E_5(m)\le E_2(m)$, and therefore 
    	$E_{10}(m)  = E_5(m)$. 
    	Also, $E_9(m) = \lfloor \frac{E_3(m)}{2}\rfloor$. 
    	
    	We first show that for $k \equiv 3\pmod{6}$, we have 
    	$E_{10}(5^k) > E_9(5^k)$. 
    	Now, 
    	\[
    	E_{10}(5^k) = E_{5}(5^k) = 
    	\sum_{i = 1}^{k}\frac{5^k}{5^i}
    	=\frac{5^k - 1}{4}
    	\]
    	Given that $k\equiv 3\pmod{6}$, we have $5^k\equiv 8\pmod{9}$, 
    	so $\lfloor \frac{5^k}{3}\rfloor = \frac{5^k}{3} - \frac{2}{3}$ and 
    	$\lfloor \frac{5^k}{9}\rfloor = \frac{5^k}{9} - \frac{8}{9}$
    	\[
    	E_{3}(5^k) = 
    	\sum_{i = 1}^{\infty}\lfloor \frac{5^k}{3^i}\rfloor
    	\le (\sum_{i = 1}^{\infty}\frac{5^k}{3^i}) - \frac{2}{3} - \frac{8}{9}
    	=\frac{5^k}{2} - \frac{14}{9}
    	< 2\left(\frac{5^k - 1}{4}\right) - 1
    	\]
    	which shows $E_{10}(5^k) > E_9(5^k)$. 
    	
    	Next, we show that for $k\equiv 10\pmod{20}$ we have 
    	$E_{10}(3^k) < E_9(3^k)$. 
    	Indeed, 
    	\[
    	E_{3}(3^k) = 
    	\sum_{i = 1}^{k}\frac{3^k}{3^i}
    	=\frac{3^k - 1}{2}
    	\]
    	while by our condition we have $3^k\equiv 24\pmod{25}$, 
    	so $\lfloor \frac{3^k}{5}\rfloor = \frac{3^k}{5} - \frac{4}{5}$ and 
    	$\lfloor \frac{3^k}{25}\rfloor = \frac{3^k}{5} - \frac{24}{25}$. 
    	Therefore, 
    	\[
    	E_{5}(3^k) = 
    	\sum_{i = 1}^{\infty}\lfloor \frac{3^k}{5^i}\rfloor
    	\le (\sum_{i = 1}^{\infty}\frac{3^k}{5^i}) - \frac{4}{5} - \frac{24}{25}
    	=\frac{3^k }{4} - \frac{44}{25}
    	\]
    	It then follows that 
    	$E_{9}(3^k)=\lfloor \frac{3^k - 1}{4}\rfloor
    	\ge  \frac{3^k - 3}{4} > \frac{3^k }{4} - \frac{44}{25} \ge E_{5}(3^k)$, 
    	as claimed. 
    	
    	\item [N4.]
    	Let $a_1, \dots, a_n, b_1, \dots, b_n$ be $2n$ positive integers such that the $n+1$ products
    	\[a_1 a_2 a_3 \cdots a_n, b_1 a_2 a_3 \cdots a_n, b_1 b_2 a_3 \cdots a_n, \dots, b_1 b_2 b_3 \cdots, b_n\]form a strictly increasing arithmetic progression in that order. Determine the smallest possible integer that could be the common difference of such an arithmetic progression.
    	
    	\textbf{Answer.} $n!$
    	
    	\textbf{Solution.} 
    	To show that $n!$ is attainable, let $a_k = k$ and $b_k = k + 1$, 
    	then the $k$-th sequence is $b_1\cdots b_{k-1}a_k\cdots a_n = 2\cdot 3\cdots \cdot k \cdot k \cdots \cdot n = k\cdot n!$. 
    	
    	To show $n!$ is the smallest possible, 
    	let the $n + 1$ numbers, in that order, be 
    	$C + kd$ for $k = 0, 1, \cdots, d$. 
    	Let $g = \gcd(C, d)$. 
    	Then $\gcd(C + kd, C + (k + 1)d) = \gcd(d, C + kd) = \gcd(d, C) = g$. 
    	Notice, also, that for $k=1, 2, \cdots, d$ we have 
    	\[
    	\frac{b_k}{a_k} = \frac{C + kd}{C + (k - 1)d}
    	\]
    	suggesting that $a_k$ must be divisible by $\frac{C + (k - 1)d}{g}$. 
    	Considering that $d = (b_1 - a_1)a_2\cdots a_n$ we have 
    	\[
    	d\ge a_2\cdots a_n \ge \prod_{i=1}^{n - 1} \left(\frac{C + kd}{g}\right)
    	\]
    	Note that $\frac{C + kd}{g}$ with $k=0, \cdots, n - 1$ forms an arithmetic progression of integers, 
    	so $\prod_{i=1}^{n - 1} \left(\frac{C + kd}{g}\right) \ge 2\cdot 3\cdot \cdots \cdot n = n!$. 
    	
    	\item [N5.]
    	Let $a_1<a_2<a_3<\dots$ be positive integers such that $a_{k+1}$ divides $2(a_1+a_2+\dots+a_k)$ for every $k\geqslant 1$. Suppose that for infinitely many primes $p$, there exists $k$ such that $p$ divides $a_k$. Prove that for every positive integer $n$, there exists $k$ such that $n$ divides $a_k$.
    	
    	\textbf{Solution.} 
    	Let $S_k = 2\sum_{i=1}^k a_i$ and $T_k = \frac{S_k}{a_{k + 1}}$. 
    	
    	We first show that $\{T_k\}_{k\ge 1}$ is unbounded. 
    	To this end, we note that $T_k$ is an integer, 
    	and also for each $k$ we have 
    	\[
    	\frac{a_{k+2}}{a_{k+1}}
    	=\frac{S_{k + 1}}{S_k}\cdot \frac{T_k}{T_{k+1}}
    	=\frac{S_{k} + 2a_{k+1}}{S_k}\cdot \frac{T_k}{T_{k+1}}
    	=(1 + \frac{2}{T_k})\cdot \frac{T_k}{T_{k+1}}
    	=\frac{T_k + 2}{T_{k+1}}
    	\]
    	Now consider all $p$ such that $p\nmid a_1a_2$, 
    	but $p\mid a_k$ for some $k$; 
    	note that there are infinitely many such $p$. 
    	For this $p$, choose the minimum such $k$, 
    	then $p\mid a_k =
    	\frac{a_{k-1}(T_{k-2} + 2)}{T_{k-1}}$. 
    	By the minimality of this $k$, 
    	we do have $p\mid T_{k-2}+2$. 
    	Given that there are infinitely many such $p$, 
    	we deduce that $\{T_k\}$ must be unbounded. 
    	
    	Note also that 
    	$\frac{T_k+2}{T_{k+1}} = \frac{a_{k+2}}{a_{k+1}} > 1$, 
    	so $T_{k + 1}\le T_k + 1$. 
    	Given also that $\{T_k\}$ is also unbounded, 
    	for each $n$ we can identify the minimal $k$ such that $T_k\ge n$. 
    	For all such $n$ such that $n > T_1$, 
    	$k\ge 2$, and $n\le T_k\le T_{k - 1} + 1\le (n - 1) + 1$, 
    	we must have $T_{k-1} = n - 1$ and $T_k = n$. 
    	But this now means that 
    	\[
    	\frac{a_{k+2}}{a_{k+1}}
    	=\frac{n+2}{n + 1}
    	\]
    	and so $n + 2\mid a_{k+2}$ (given that $\gcd(n + 2, n + 1) = 1$). 
    	Thus we get that given all $n > T_1$, 
    	there exists $k$ for which $n + 2\mid a_{k + 2}$. 
    	This completes the solution. 
    	
    	\item [N6.]
    	A sequence of integers $a_0, a_1 …$ is called kawaii if $a_0 =0, a_1=1,$ and$$(a_{n+2}-3a_{n+1}+2a_n)(a_{n+2}-4a_{n+1}+3a_n)=0$$for all integers $n \geq 0$. An integer is called kawaii if it belongs to some kawaii sequence.
    	Suppose that two consecutive integers $m$ and $m+1$ are both kawaii (not necessarily belonging to the same kawaii sequence). Prove that $m$ is divisible by $3,$ and that $m/3$ is also kawaii.
    	
    	\textbf{Solution}. 
    	We show that any kawaii number $m$ cannot have $3\mid m - 2$. 
    	To see why, if $a_n, a_{n-1}\in \{0, 1\}\pmod{3}$ then $a_{n+1}$ will also have this property. 
    	Thus the claim follows from $a_0=0$ and $a_1=1$. 
    	This means that if $m$ and $m+1$ are both kawaii then 
    	$m$ is divisible by 3. 
    	
    	It now remains to show the second part. 
    	Observe that $a_N\equiv a_{N+1}\pmod C$ for some $C$ would mean 
    	$a_{n+1}\equiv a_n\pmod{C}$ for all $n\ge N$. 
    	Given that $a_1=1$ and $a_2\in \{3, 4\}$, 
    	we either have $a_n$ odd for all $n\ge 2$ (if $a_2=3$), 
    	or $a_n\equiv 1\pmod{3}$ for all $n\ge 2$ (if $a_2=4$). 
    	In particular, the only kawaii number divisible by 6 is 0. 
    	
    	Now suppose that $m$ and $m + 1$ are both kawaii, 
    	and $m > 0$. 
    	Notice that $3\pmod m$, and by our analysis above, $m$ is odd 
    	(hence $m + 1$ even). 
    	Thus $m + 1$ must come from a sequence where $a_2 = 4$. 
    	Let $m + 1 = a_k$ for some $k\ge 2$, with $a_0, a_1, \cdots$ a kawaii sequence. 
    	Denote $\{b_n\}$ as another sequence with $b_0 = 0$, $b_1 = 1$, 
    	and 
    	\[
    	b_{n+1} = 
    	\begin{cases}
    		3b_{n} - 2b_{n - 1} & a_{n+2} = 3a_{n + 1} - 2a_{n}\\
    		4b_{n} - 3b_{n - 1} & a_{n+2} = 4a_{n + 1} - 3a_{n}\\
    	\end{cases}
    	\]
    	Note also that $\{b_n\}$ is kawaii. 
    	We now show that $a_{n+1} = 3b_n + 1$ for all $n$, 
    	thereby showing that $m / 3 = b_{k - 1}$ and therefore kawaii. 
    	Indeed, this holds for $n = 0, 1$ since $a_2 = 4$. 
    	Then all we need is the following:
    	\[
    	a_{n+2} = 3a_{n + 1} - 2a_{n}
    	\to a_{n + 2} = 3(3b_n + 1) - 2(3b_{n-1}+ 1) = 3(3b_n - 2b_{n-1}) + 1
    	= 3b_{n+1} + 1
    	\]
    	\[
    	a_{n+2} = 4a_{n + 1} - 3a_{n}
    	\to a_{n + 2} = 4(3b_n + 1) - 3(3b_{n-1}+ 1) = 3(4b_n - 3b_{n-1}) + 1
    	= 3b_{n+1} + 1
    	\]
    	completing the indution step. 
    	
    	\item [N7.]
    	Let $a,b,c,d$ be positive integers satisfying\[\frac{ab}{a+b}+\frac{cd}{c+d}=\frac{(a+b)(c+d)}{a+b+c+d}.\]Determine all possible values of $a+b+c+d$.
    	
    	\textbf{Answer.} All integers $n$ such that there exists an integer $m\ge 2$ and $m^2\mid n$. 
    	
    	\textbf{Solution.}
    	We first notice that the following assignment works: 
    	\[
    	a = kx^2\qquad b = c = kxy \qquad d = ky^2
    	\]
    	for some positive $k, x, y$. 
    	This gives both sides as $kxy$ and $a+b+c+d=k(x+y)^2$. 
    	Thus if $n = m^2k$, where $m\ge 1$, 
    	then we may pick $x=1, y=m - 1$. 
    	
    	Conversely, to show that this condition is necessary, 
    	note that any $n$ without this property is square free 
    	(i.e. it is a product of different primes, $p_1p_2\cdots p_u$). 
    	Notice that $n$ is also the smallest integer that is simultaneously divisible by all the primes $p_1, p_2, \cdots, p_u$. 
    	It then follows that there is a prime $p_{\ell}$ such that 
    	neither $a+b$ nor $c+d$ is divisible by $p_{\ell}$. 
    	Consequently, when the fractions are written in lowest terms (i.e. gcd of numerator and denominator is 1), 
    	the LHS has denominator not divisible by $p_{\ell}$, 
    	while RHS has denominator divisible by $p_{\ell}$. 
    	It follows that they cannot be equal. 
    	
    	\item [N8.]
    	Determine all functions $f\colon\mathbb{Z}_{>0}\to\mathbb{Z}_{>0}$ such that, for all positive integers $a$ and $b$,
    	\[
    	f^{bf(a)}(a+1)=(a+1)f(b).
    	\]
    	
    	\textbf{Answer.} 
    	$f(n) = n + 1$ (where LHS = RHS = $(a + 1)(b + 1)$. 
    	
    	\textbf{Solution. }
    	We break the solution into the following. 
    	\begin{lemma}
    		$f$ is injective. 
    	\end{lemma}
    	\begin{proof}
    		Denote $N = \{f(n): n = 1, 2, \cdots, \}$. 
    		Note that for all $a, b$, 
    		$(a + 1)f(b)\in N$, 
    		so $N$ is unbounded since $f(b) > 0$. 
    		Next, consider the sequence 
    		$\{a_n\}$ given by $a_n = f^{nf(1)}(2) = 2f(n)$. 
    		Note that $a_{m_1}=a_{m_2}$ for some $m_1<m_2$ implies that $a_{k} = a_{k + (m_2-m_1)}$ for all $k\ge m_1$, 
    		i.e. $a_k$ is eventually periodic (hence bounded). 
    		This would then mean that $\{2f(n)\}$ is also bounded, 
    		which contradicts $N$ is unbounded. 
    	\end{proof}
    	
    	\begin{lemma}
    		$N = \{2, 3, 4, \cdots\}$, 
    		and that for all $k\ge 1$ we have $k = f^m(1)$ for some $m\ge 0$. 
    	\end{lemma}
    	
    	\begin{proof}
    		Next, we show that $1\not\in N$. 
    		If $f(n)=1$ for some $n$ then $f^{nf(1)}(2)=2$, 
    		i.e. $f^{knf(1)}(2)=2$, and $\{a_n\}$ is periodic, contradiction. 
    		
    		To show the other part, 
    		Consider the relation $\sim$ given by $a\sim b$ if and only if there exists a $k\ge 0$ such that either $f^k(a)=b$ or $f^k(b) = a$; 
    		it's clear that this $\sim$ is reflective and symmetric, 
    		i.e. $a\sim a$, and also $a\sim b\to b\sim a$. 
    		We now claim that $\sim$ is also transitive. 
    		If there is $k, m\ge 0$ such that $f^k(a)=b, f^m(b)=c$, 
    		then $f^{k+m}(a)=c$; 
    		if $f^k(a)=f^m(c)=b$, 
    		w.l.o.g. suppose $k\le m$. 
    		Then by transitivity, $b = f^m(c)=f^{k}(f^{m-k}(c))$, 
    		so by injectivity, $f^{m-k}(c) = a$. 
    		Thus $\sim$ does partition things into equivalene relations. 
    		
    		With this, we note that $k\in N\to \exists b: k=f(b) 
    		\to \forall a: (a+1)f(b)=(a+1)k\in N$. 
    		Let $a_1, a_2\ge 1$ be arbitrary, and let $c_1, c_2$ be such that 
    		$f(c_1)=(a_1+1)f(1)$ and $f(c_2)=(a_2+1)f(1)$. 
    		Then: 
    		\[
    		f^{c_2f(a_1)}(a_1+1) = (a_1+1)f(c_2)=(a_1+1)(a_2+1)f(1)
    		=(a_2+1)f(c_1)=f^{c_1f(a_2)}(a_2+1)
    		\]
    		showing that either there exists $k := k(a_1, a_2)$ such that either $f^k(a_1+1)=a_2+1$ or $f^k(a_2+1)=a_1+1$. 
    		Thus $2, 3, 4, \cdots$ are all in the same group w.r.t. $\sim$. 
    		Now $1, f(1), f(f(1)), \cdots$ are also in the same group w.r.t. $\sim$, 
    		but with $f(1) > 1$, 
    		we see that there is only one equivalence class under $\sim$ here. 
    		Since 1 is not a value of $f$, 
    		it follows that each $k > 1$ can be written as $k = f^m(1)$ for some $m$. 
    	\end{proof}
    	
    	\begin{lemma}
    		$f(1) = 2$. 
    	\end{lemma}
    	
    	\begin{proof}
    		We claim that there exists a function $g(k)$ such that for all $a\ge 2$ we have $f(ak - 1) = g(k)f(a - 1)$. 
    		Now consider any $a\ge 2, c\ge 1$, we have 
    		\[
    		f^{bf(af(c) - 1)}(af(c))
    		=af(c)f(b)
    		\]
    		Since $f(b)f(c)\ge 4$, 
    		there exists $d$ with $f(d)=f(b)f(c)$, 
    		and satisfies 
    		$f^{df(a - 1)}(a) = af(d)=af(b)f(c)$. 
    		Comparing this with $f^{cf(a-1)}(a)=af(c)$, 
    		and the injectivity of $f$, we have 
    		\[
    		df(a - 1) = cf(a - 1) + bf(af(c) - 1)
    		\]
    		i.e. $f(af(c) - 1) = \frac{(d - c)f(a - 1)}{b}$. 
    		If $k=1$ there's nothing to prove (and $g(1)=1$), 
    		otherwise if $k > 1$, 
    		choose $c$ with $f(c)=k$, 
    		Choose $b = 1$ and $d$ such that $f(d) = f(1)k$, 
    		get $f(ak - 1) = (d - c)f(a - 1)$, 
    		so this $g(k) = d - c$ (which is also valid since $d, c$ depends only on $k$ in this case since $f(c)=k$ and $f(d) = f(1)k$).
    		
    		On the other side, we show that $f(1) = 2$. 
    		Given that each number $n\ge 2$ can be written as $f^k(1)=n$, let $c(n)$ be the number such that $f^{c(n)}(1)=n$, 
    		then $c(f(1))=1$, and by the equation given, $c((a+1)f(b)) - c(a + 1) = bf(a) > 0$ for all $a, b\ge 1$. 
    		Thus $f(1)$ cannot be written in the form $(a+1)f(b)$. 
    		Since $f(b)$ and $a + 1$ each takes all integer values of $\ge 2$, 
    		$(a+1)f(b)$ takes all composite values, $f(1)$ will then have to be a prime.
    		If $f(1) = p$ is odd, 
    		then we from $f(2k-1) = g(k)f(1)$ we have $f(p)=f^2(1)$ divisible by $p$. 
    		Now take $a = p - 1$, and $b$ such that $p f(b) = f(p)$ (such $p$ exists since $f(p) \neq p$), 
    		and we have $(a + 1)f(b) = f(p) = f^{bf(p - 1)}(p)$, 
    		so $bf(p-1)=1$. This means $b=f(p-1)=1$, 
    		contradicting that $1\not\in N$. 
    		It then follows that $f(1)$ is even and prime, so $f(1) = 2$. 
    	\end{proof}
    	
    	
    	With this, setting $a=b=1$ gives $f^2(2)=2f(1)=4$. 
    	Let $p = f(2)$ with $f(p) = 4$. 
    	Note that $p$ is odd; 
    	otherwise, given that $p\neq 2$ we can find $b$ with $f(b) = \frac{p}{2}$ and setting $a=1$, get $f^{2b}(2)=2f(b) = p$, 
    	which contradicts that $f^n(2)$ is different for each $n\ge 0$ 
    	(we note that $2b\neq 1$). 
    	We also have $4 = f(p) = g(2)f(\frac{p - 1}{2})$, 
    	and since $\frac{p-1}{2} < p$ and $f(\frac{p - 1}{2}) > 1$, 
    	we must have $f(\frac{p - 1}{2}) = 2$. 
    	Thus by injectivity of $f$, 
    	$\frac{p - 1}{2} = 1$, giving $p=3$. 
    	Thus in summary we have shown $f(n) = n + 1$ for $n = 1, 2, 3$. 
    	
    	We can now finish off the solution. 
    	Recall that $f(ak-1)=g(k)f(a-1)$; setting $a=k=2$ gives $4=f(3)=g(2)f(1)=2g(2)$, 
    	so $g(2)=2$ and $f(2a-1) = 2f(a - 1)$. 
    	Thus, if, for some $k\ge 2$, we have 
    	$f(n) = n + 1$ for $n=1, \cdots, 2k-1$, then: 
    	\[
    	f(2k + 1) = 2f(k) = 2(k + 1)\qquad 
    	a = 1; b = k: f^{2k}(2) = f^{kf(1)}(2) = 2f(k) = 2(k + 1)
    	\]
    	and given our premise, $f^{2k - 2}(2) = 2 + 2k - 2 = 2k$, 
    	so $f^2(2k) = f(2k + 1) = 2(k + 1)$. 
    	It then follows by injectivity of $f$ that $f(2k) = 2k + 1$ and $f(2k + 1) = 2(k + 1)$. 
    	Thus by doing this induction on $k$, we do have $f(n) = n + 1$ for all $n \ge 1$. 
    \end{enumerate}
    
\end{document}