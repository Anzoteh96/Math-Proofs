\documentclass[11pt,a4paper]{article}
\usepackage{amsmath, amssymb, fullpage, mathrsfs, bm, pgf, tikz}
\usepackage{mathrsfs,amsthm}
\usetikzlibrary{arrows}
\setlength{\textheight}{10in}
%\setlength{\topmargin}{0in}
\setlength{\topmargin}{-0.5in}
\setlength{\parskip}{0.1in}
\setlength{\parindent}{0in}

\begin{document}
	\newcommand{\la}{\leftarrow}
	\newcommand{\lra}{\leftrightarrow}
	\newcommand{\bbN}{{\mathbb N}}
	\newcommand{\bbZ}{{\mathbb Z}}
	\newcommand{\bbQ}{{\mathbb Q}}
	\newcommand{\bbR}{{\mathbb R}}
	\newcommand{\bbC}{{\mathbb C}}
	\newcommand{\bbH}{{\mathbb H}}
	\newcommand{\dfeq}{\stackrel{\mathrm{def}}{=}}
	\newcommand{\ra}{\rightarrow}
	\newcommand{\Span}{\mathrm{span}}
	\newcommand{\scrP}{\mathscr{P}}
	\newcommand{\rank}{\mathrm{rank}}
	\newcommand{\nullity}{\mathrm{nullity}}
	\newcommand{\Col}{\mathrm{Col}}
	\newcommand{\Row}{\mathrm{Row}}
	\newcommand{\tr}{\mathrm{tr}}
	\newcommand{\ol}{\overline}
	\newcommand{\norm}[1]{||#1||}
	\newcommand{\doubleline}[1]{\underline{\underline{#1}}}
	\newcommand{\elemop}[1]{\stackrel{#1}{\longrightarrow}}
	\newcommand{\Ind}{\mathrm{Ind}}
	\newcommand{\Res}{\mathrm{Res}}
	\newcommand{\End}{\mathrm{End}}
	\newcommand{\cl}{\mathrm{cl}}
	\newcommand{\code}[1]{\texttt{#1}}
	\newcommand\tab[1][0.5cm]{\hspace*{#1}}
	\newcommand{\<}{\langle}
	\renewcommand{\>}{\rangle}
	\newcommand{\qubits}[1]{|{#1}\rangle}
	\newcommand{\powset}{\mathcal{P}}
	\newcommand{\dsum}{\displaystyle\sum}
	\newcommand{\dprod}{\displaystyle\prod}
	
	\newtheorem{lemma}{Lemma}
	
	\section*{Algebra}
	\begin{enumerate}
		\item [A1.] 
		Let $n$ be an integer, and let $A$ be a subset of $\{0, 1, \cdots, 5^n\}$ consisting of $4n + 2$ numbers. Prove that there exist $a,b,c\in A$ such that $a < b < c$ and $c + 2a > 3b$.
		
		\textbf{Solution.} 
		Suppose $A$ is a counterexample of the condition above. 
		Here, we order the numbers as $a_1 < \cdots < a_m$ with $m=4n+2$. 
		Then we get the following identity: 
		for all $i\le 4n = m - 2$ we have 
		\[
		2a_i + a_m \le 3a_{i + 1} \Rightarrow (a_m - a_{i + 1}) \le 2(a_{i + 1} - a_i)
		\Rightarrow 
		\frac{a_m - a_{i}}{a_m - a_{i + 1}}
		= 1 + \frac{a_{i + 1} - a_{i}}{a_m - a_{i + 1}}
		\ge 1 + \frac 12
		=\frac 32
		\]
		
		Repeating this procedure, we get 
		\[
		\frac{a_m - a_1}{a_m - a_{m - 1}} \ge \left(\frac 32\right)^{m - 2}
		=\left(\frac 32\right)^{4n}
		=\left(\frac {81}{16}\right)^{n}
		> 5^n
		\]
		which contradicts $0\le a_1$ and $a_m\le 5^n$. 
		
		\item [A2.] 
		For every integer $n\ge 1$ consider the $n\times n$ table with entry $\lfloor \frac {ij}{n + 1}$ at the intersection of row $i$ and column $j$, for every $i=1, \cdots, n$ and $j=1, \cdots, n$. 
		
		Determine all integers $n\ge 1$ for which the sum of the $n^2$ entries in the table is equal to $\frac 14 n^2(n-1)$. 
		
		\textbf{Answer.} All $n$ such that $n + 1$ is prime. 
		
		\textbf{Solution.} Let $r(x)$ be the remainder of an integer $x$ when divided by $n + 1$, 
		thus $0\le x\le n$. 
		$\lfloor \frac{ij}{n + 1}\rfloor = \frac {1}{n + 1} (ij - r(ij))$. 
		This gives the sum as 
		\[
		\frac{1}{n + 1}\sum_{i=1}^n\sum_{j=1}^n (ij - r(ij))
		=\frac{1}{n + 1}(\sum_{i=1}^n i)^2 - \frac{1}{n + 1} \sum_{i=1}^n\sum_{j=1}^n r(ij)
		=\frac{n^2(n+1)}{4} - \frac{1}{n + 1} \sum_{i=1}^n\sum_{j=1}^n r(ij)
		\]
		It follows that $\sum_{i=1}^n\sum_{j=1}^n r(ij) = 
		(n + 1)(\frac{n^2(n+1)}{4} - \frac{n^2(n-1)}{4})=\frac{n^2(n+1)}{2}$. 
		
		Now we consider each row $k$. Suppose that $\gcd(k, n + 1) = \ell$. 
		Then $r(ki)$ for $i=1, \cdots, n$ is just $\ell$ copies of 
		$\ell, 2\ell, \cdots, \ell(\frac{n + 1}{\ell} - 1)$. 
		It then follows the sum of $r(ki)$ here is given by 
		\[
		\frac 12 \ell^2 (\frac{n + 1}{\ell})(\frac{n + 1}{\ell} - 1)
		=\frac 12 \cdot (n + 1)(n + 1 - \ell)
		\]
		This is upper bounded by $\frac {n(n+1)}{2}$, which means the sum of all the remainders cannot exceed $\frac{n^2(n+1)}{2}$. 
		Equality holds if and only if $\ell=1$ for all such $k$, 
		i.e. $\gcd(k, n + 1)=1$ for $k=1, \cdots, n$. 
		This is precisely when $n + 1$ is prime. 
		
		\item [A3.] Given a positive integer $n$, find the smallest value of 
		\[
		\lfloor \frac {a_1}{1}\rfloor + \cdots + \lfloor \frac {a_n}{n}\rfloor
		\]
		over all permutations $(a_1, \cdots, a_n)$ of $(1, \cdots, n)$. 
		
		\textbf{Answer.} $\lfloor \log_2 n\rfloor + 1$. 
		
		\textbf{Solution.} Let's first do the easier part: 
		finding such a construction. 
		Indeed, let $2^k\le n\le 2^{k+1}$. 
		Consider the following construction: 
		\[
		a_\ell = 
		\begin{cases}
			\min(2\ell - 1, n) & \exists m\ge 0: \ell = 2^m\\
			\ell - 1 & \text{otherwise}\\
		\end{cases}
		\]
		Then $\lfloor \frac{a_{\ell}}{\ell}$ is 1 when $\ell=2^m$ and 0 otherwise, so the sum is indeed $k + 1 = \lfloor \log_2 n \rfloor + 1$. 
		One can also verify that the construction above is a permutation: 
		indeed for each $m\le k$, $a_{2^m}, \cdots, a_{\min(2^{m+1}-1, n)}$ is 
		$\min(2^{m+1}-1, n), 2^m, \cdots, \min(2^{m+1}-1, n) - 1$, 
		i.e. permutation within this interval. 
		
		To establish the lower bound, we denote $f(n)$ as our desired answer, and it suffices to show that $f(n)\ge f(\lfloor \frac n2\rfloor) + 1$ for all $n\ge 2$. 
		Observe that $f(1) = 1$. 
		We now consider this lemma: 
		\begin{lemma}\label{lemma_a3}
			If $a_1, \cdots, a_m$ are distinct positive integers, then
			then 
			$\lfloor \frac {a_1}{1}\rfloor + \cdots + \lfloor \frac {a_m}{m}\rfloor\ge f(m)$: 
		\end{lemma}
	    \begin{proof}
	    	If $b_i$ is the position of $a_i$ when arranged in sorted order then $b_i\le a_i$ and 
	    	$b_1, \cdots, b_m$ is a permutation of $1, \cdots, m$, 
	    	so 
	    	\[
	    	\lfloor \frac {a_1}{1}\rfloor + \cdots + \lfloor \frac {a_m}{m}\rfloor
	    	\ge 
	    	\lfloor \frac {b_1}{1}\rfloor + \cdots + \lfloor \frac {b_m}{m}\rfloor
	    	\ge f(m)
	    	\]
	    \end{proof}
		
		In particular, this would be the case when $(a_1, \cdots, a_n)$ is a permutation of $1, \cdots, n$. 
		(I.e. we look at the ``prefix'' of the first $m$ entries). 
		
		Now fix $m=f(\lfloor \frac n2\rfloor)$, and we have the two cases: 
		\begin{itemize}
			\item $a_k\ge k$ for some $k > m$. 
			Then $\lfloor \frac {a_{m+1}}{m+1}\rfloor + \cdots + \lfloor \frac {a_n}{n}\rfloor\ge 1$, 
			so 
			\[
			\lfloor \frac {a_1}{1}\rfloor + \cdots + \lfloor \frac {a_n}{n}\rfloor
			\ge f(m) + 1
			\]
			$f(m)$ for the first $m$ terms; 1 for the rest. 
			
			\item $a_k < k$ for some $k > m$. 
			It then follows that $a_{\ell}=n$ for some $\ell\le m$. 
			Now let $n'$ be the smallest number not in $a_1, \cdots, a_m$; we have $n'\le m\le \frac n2$. 
			If we define $a'_i = \begin{cases}
				n' & i=\ell\\
				a_i & \text{otherwise}\\
			\end{cases}$
		    then 
		    \[
		    \lfloor \frac {a'_1}{1}\rfloor + \cdots + \lfloor \frac {a'_m}{m}\rfloor\ge f(m)
		    \]
		    by Lemma \ref{lemma_a3}. In addition, $n - n'\ge \frac{n}{2}\ge \ell$ so 
		    $\lfloor \frac{n}{\ell}\rfloor - \lfloor \frac{n'}{\ell}\rfloor \ge 1$. 
		    Therefore 
		    \[
		    \lfloor \frac {a_1}{1}\rfloor + \cdots + \lfloor \frac {a_m}{m}\rfloor
		    \ge 
		    1 + \lfloor \frac {a'_1}{1}\rfloor + \cdots + \lfloor \frac {a'_m}{m}\rfloor
		    \ge f(m) + 1
		    \]
		    
		\end{itemize}
	\end{enumerate}
	
	\section*{Combinatorics}
	\begin{enumerate}
		\item [C1.] 
		Let $S$ be an infinite set of positive integers, such that 
		there exists four pairwise distinct $a, b, c, d\in S$ with $\gcd(a, b)\neq \gcd(c, d)$. 
		Prove that there exist three pairwise distinct $x, y, z\in S$ such that 
		\[
		\gcd(x, y)=\gcd(y, z)\neq \gcd(z, x)
		\]
		
		\textbf{Solution.} 
		By dividing each number by the same common divisor $d$ as necessary, we may assume $\gcd(S)=1$ (i.e. for each prime $p$ there's $s\in S$ with $p\nmid s$). 
		We now consider the following two cases: 
		
		\emph{Case 1.} There exists a prime $p$ such that $p\mid s$ for infinitely many primes $p$. 
		By the $\gcd(S)=1$ assumption we may assume that there exists $y$ such that $p\nmid y$. 
		Let $S_p = \{s\in S: p\mid s\}$, 
		and we see that: 
		\[
		\{\gcd(y, x): x\in S_p\}\subseteq \{d: \text{positive divisors of }y\}
		\]
		Since $|S_p|$ is infinite, while the set of positive divisors of $y$ is finite, 
		by pigeonhole principle we have $\gcd(x, y)=\gcd(z, y)$ for some $x\neq z\in S_p$. 
		Here we have $p\mid x, z$ while $p\nmid y$, so $\gcd(x, y)\neq \gcd(x, z)$, as desired. 
		
		\emph{Case 2.} 
		Now we have each prime dividing only finitely many members in $S$. 
		The $\gcd(a, b)\neq \gcd(c, d)$ condition means $\gcd(x, z) > 1$ for some $x\neq z\in S$. 
		We note that the set of primes $p$ dividing $xz$ is finite, so in this case, we have 
		\[
		|\{s\in S: \gcd(s, xz) > 1\}| < \infty
		\]
		Thus we may take $y\in S$ such that $\gcd(y, xz)=1$. 
		It then follows that $\gcd(y, x)=\gcd(y, z)=1$. 
		
		\item [C2.] Let $n \ge 3$ be an integer. 
		An integer $m \ge n + 1$ is called $n$-colorful if, given infinitely many marbles in each of $n$ colours $C_1, C_2, \cdots , C_n$, 
		it is possible to place $m$ of them around a circle so that in any group of $n + 1$ consecutive marbles there is at least one marble of colour $C_i$ for each $i = 1, \cdots , n$.
		
		Prove that there are only finitely many positive integers which are not $n$-colourful, and find the largest among them.
		
		\textbf{Answer.} The largest number that's not $n$-colorful is $n^2-n-1$. 
		
		\textbf{Solution.} 
		We first show that $n^2-n-1$ is not $n$-colorful. 
		Indeed, one of the color, say $C_j$, is used at most $\lfloor \frac{n^2-n-1}{n}\rfloor = n-2$ times. 
		Since $n^2-n-1=(n-2)(n+1)+1$, it follows that if we iterate through the marbles in clockwise fashion, the gap of some two of them (possibly cyclic repetition) is at least $n+2$. 
		This means the $\ge n+1$ marbles in between them has no colour $C_j$. 
		
		Conversely, we show that any $m\ge n^2-n$ is colorful. 
		Let $g$ be the remainder of $m$ when divided by $n$ (i.e. $0\le g\le n - 1$). 
		We consider the following arrangement: 
		\[
		\underbrace{(C_1C_2\cdots C_nC_1)}_{g \text{ copies}}
		\underbrace{(C_1C_2\cdots C_n)}_{(m - g(n + 1))/n\text{ copies}}
		\]
		
		For all $n^2-n$, we have $g(n+1)\le m$ since $g(n+1) < n^2-n$ for all $g=0, \cdots, n-2$, 
		and when $g=n-1$, $m\ge n^2-1=g(n+1)$. 
		Next, the $i$-th color of each group (either in $(C_1C_2\cdots C_nC_1)$ or $(C_1C_2\cdots C_n)$) are of color $C_i$ for $i=1, 2, \cdots, n$, 
		and are either spaced $n$ or $n+1$ apart, 
		which guarantees that any $n+1$ consecutive marbles would cover this $C_i$. 
		
		\item [C4.] 
		The kingdom of Anisotropy consists of $n$ cities. For every two cities there exists exactly one direct one-way road between them. We say that a path from $X$ to $Y$ is a sequence of roads such that one can move from $X$ to $Y$ along this sequence without returning to an already visited city. A collection of paths is called diverse if no road belongs to two or more paths in the collection.
		
		Let $A$ and $B$ be two distinct cities in Anisotropy. Let $N_{AB}$ denote the maximal number of paths in a diverse collection of paths from $A$ to $B$. Similarly, let $N_{BA}$ denote the maximal number of paths in a diverse collection of paths from $B$ to $A$. Prove that the equality $N_{AB} = N_{BA}$ holds if and only if the number of roads going out from $A$ is the same as the number of roads going out from $B$.
		
		\textbf{Solution.} 
		Let's first show the following: 
		\begin{lemma}
			\label{lemma_c4a}
			There exists a diverse collection of path from $A$ to $B$ with maximal number, 
			such that all paths in the form of $A\to v\to B$ are included (here $\to$ means directed edge). 
		\end{lemma}
		
		\begin{proof}
			Let's consider any such diverse collection $\mathcal{C}$ with $N_{AB}$ such paths. 
			Consider any $v$ such that $A\to v\to B$ is not included. We consider these cases: 
			
			\textbf{Case 0.} Neither $A\to v$ nor $v\to B$ are in any path contained in $\mathcal{C}$. 
			Then we may even add $A\to v\to B$ directly. 
			
			\textbf{Case 1a.} $A\to v$ belongs to some path contained in $\mathcal{C}$ but not $v\to B$. 
			Now, if the path is $A\to v\Rightarrow B$ where $\Rightarrow$ denotes a path from $v$ to $B$, then we may replace the paths in $v\Rightarrow B$ to $v\to B$, that gives $A\to v\to B$.  
			
			\textbf{Case 1b.} $v\to B$ belongs to some path contained in $\mathcal{C}$ but not $A\to v$. 
			Similar case: if this is $A\Rightarrow v\to B$ we may replace $A\Rightarrow v$ with $A\to v$. 
			
			\textbf{Case 2.} Both $A\to v$ and $v\to B$ are part of the collection but in different paths. 
			This means there exists the two paths 
			\[
			A\to v\Rightarrow B\qquad A\Rightarrow v\to B
			\]
			such that $A\Rightarrow v$ and $v\Rightarrow B$ are two paths with disjoint roads, and also disjoint from edges $A\to v$ and $v\to B$. 
			Call these two paths $U$ and $V$. 
			Thus $A\Rightarrow v\Rightarrow B$ does contain a path that's subset of $U\cup V$ (basically, take the union, and remove all cycles), 
			so we may replace the two original paths with this path and $A\to v\to B$. 
			
			So considering all cases above it means we can indeed include $A\to v\to B$ into the collection, for all such eligible $v$. 
		\end{proof}
	    
	    Now, referencing to towns $A$ and $B$, each vertex can be categorized into exactly one of the following: $\vec{AB}$-type ($A\to v\to B$), $\vec{BA}$-type ($B\to v\to A$), 
	    in-type ($A\to v\la B$) and out-type ($A\la v\to B$). 
	    Regardless of the members of the collection, either $A\to B$ or $B\to A$ must also be in a maximally-constructed diverse collection. 
	    
	    By above we may assume that we can construct a maximally constructed path by including all paths of the form $A\to v\to B$ for all $\vec{AB}$-type vertices $v$. 
	    Thereafter, any new path starting with $A\to v'$ must have $v'$ an in-type and any new path ending with $v''\to B$ must have $v''$ an out-type. 
	    Now we claim the following:
	    
	    \begin{lemma}
	    	\label{lemma_c4b}
	        A maximally diverse collection of paths from $A$ to $B$ containing $A\to v\to B$ for all $\vec{AB}$-type vertices $v$. Then the remaining paths are the maximal possible-sized set of paths in the form $A\to v_1\Rightarrow v_2\to B$, 
	        where $v_1$ is in-type, $v_2$ is out-type, 
	        and any two paths of the form $v_1\Rightarrow v_2$ have disjoint paths and starting and ending vertices. 
	    \end{lemma}
        \begin{proof}
        	The disjoint edges condition follows from definition; the disjoint starting and ending vertices follow from $A\to v_1$ and $B\to v_2$ condition. 
        	
        	Conversely, if we have such a collection of $A\to v_1\Rightarrow v_2\to B$, then these collections have edges disjoint from $A\to v\to B$, so such addition is valid. It therefore means we can pick the collection with maximum number of paths. 
        \end{proof}
        To finish, denote $C_{AB}$ as the quantity described in Lemma \ref{lemma_c4b}. 
        Such paths do not depend on $A$ and $B$ other than that we have in- and out-types of edges, so $C_{AB}=C_{BA}$. 
        Therefore, 
        \[
        N_{AB} = C_{AB} + 1\{A\to B\} + |\{v: \vec{AB}-\text{type}\}|
        \qquad 
        N_{BA} = C_{BA} + 1\{B\to A\} + |\{v: \vec{BA}-\text{type}\}|
        \]
        where $1\{A\to B\}$ means there's a directed edge $A\to B$. 
	    so $N_{AB}-N_{BA} = 1\{A\to B\} + |\{v: \vec{AB}-\text{type}\}| - (1\{B\to A\} + |\{v: \vec{BA}-\text{type}\}|)$. 
	    Given also that the out degree of $A$, $\text{out}(A)$ is given by 
	    $1\{A\to B\} + |\{v: \vec{AB}-\text{type}\}| + |\{v: \text{out-type}\}|$, 
	    we have 
	    \[
	    N_{AB} - N_{BA} = \text{out}(A) - \text{out}(B)
	    \]
	    as desired. 
	\end{enumerate}
	
	
	\section*{Geometry}
	\begin{enumerate}
		\item [G1.]  Let $ABCD$ be a paralleogram such that $AC = BC$. A point $P$ is chosen on the extension of the segment $AB$ beyond $B$. 
		The circumcircle of the triangle $ACD$ meets the segment $PD$ again at $Q$, and the circumcircle of the triangle $APQ$ meets the segment $PC$ again at $R$.
		
		Prove that the lines $CD, AQ$, and $BR$ are concurrent.
		
		\textbf{Solution.} 
		Denote $T$ as $CD\cap AQ$ and $R'=CP\cap BT$. Our goal is to show that $R'$ is on circle $APQ$. 
		 
		Here we have $AC=BC=AD$ (from the definition of parallelogram), and vy some angle chasing we have 
		\[
		\angle QAC = \angle TAC = \angle TDP = \angle CDP = \angle DPA
		\qquad 
		\angle CTA = \angle TAB
		\]
		so triangles $CTA$ and $ADP$ are similar. It then follows that $CT\cdot AP = AD\cdot AC = AC^2=BC^2$. 
		Given also that $\angle TCA = \angle CBA=\angle CAB$, we have triangles $CBT$ and $APC$ similar, so $\angle TBC = \angle CPA$, which in turn becomes 
		$\angle TR'C=\angle TQC$. Therefore $TCQR'$ is cyclic. 
		This means: 
		\[
		\angle QAP = \angle QTC = \angle QR'C
		\]
		and so $APR'Q$ is indeed cyclic. 
		
		\item [G4.] 
		Let $ABCD$ be a quadrilateral inscribed in a circle $\Omega$. 
		Let the tangent to $\Omega$ at $D$ intersect the rays $BA$ and $BC$ at points $E$ and $F$, respectively. 
		A point $T$ is chosen inside the triangle $ABC$ so that $TE \parallel CD$ and $TF \parallel AD$. Let $K \neq D$ be a point on the segment $DF$ such that $TD = TK$.
		
		Prove that the lines $AC, DT$ and $BK$ intersect at one point.
		
		\item [G5.]
		Let $ABCD$ be a cyclic quadrilateral whose sides have pairwise different lengths. 
		Let $O$ be the circumcenter of $ABCD$. 
		The internal angle bisectors of $\angle ABC$ and $\angle ADC$ meet $AC$ at $B_1$ and $D_1$ respectively. 
		Let $O_B$ be the centre of the circle which passes through $B$ and is tangent to $AC$ at $D_1$. 
		Similarly, let $O_D$ be the centre of the circle which passes through $D$ and is tangent to $AC$ at $B_1$.
		
		Assume that $BD_1 \parallel DB_1$. Prove that $O$ lies on the line $O_BO_D$.
		
	\end{enumerate}
	
	\section*{Number Theory}
	\begin{enumerate}
		\item [N1.] 
		Determine all integers $n\ge 1$ for which there exists a pair of positive integers $(a, b)$ such that 
		no cube of a prime divides $a^2+b+3$ and 
		\[
		\frac{ab+3b+8}{a^2+b+3} = n
		\]
		
		\textbf{Answer.} $n=2$ is the only solution, realized by $a=2, b=2$. 
		
		\textbf{Solution.} By solving equations on both sides we have $(a + 3 - n)b = na^2+3n - 8$. 
		If both sides are 0 then $a+3-n=na^2+3n-8=0$. 
		This means we either have $n=1$ or $n=2$ (since we need $8\ge 3n$). 
		$n=2$ has been shown to be possible, so we consider $n=1$, 
		i.e. $a^2=5$, which is impossible. 
		We therefore have $a+3-n\neq 0$, and $b=\frac{na^2-5}{a+3-n}$. 
		Using this, we have 
		\[
		a^2+b+3 = a^2 + \frac{na^2-5}{a+3-n} + 3 = \frac{(a+1)^3}{a + 3 - n}
		\]
		Now consider any prime $p$ dividing $a+1$. 
		By the ``no cube'' condition we need $3v_p(a+1) - (a+3-n)\le 2$, 
		so $(a+3-n)\ge 3v_p(a+1)-2\ge v_p(a+1)$ since $v_p(a+1)\ge 1$. 
		It then follows that $a+3-n$ is divisible by $a+1$, and since $a+3-n>0$, 
		$a+3-n\ge a+1$. In particular, $n\le 2$. 
		To see why we cannot have $n=1$, we have $\frac{(a+1)^3}{a+2}$ an integer, 
		but since $a+1\equiv -1\pmod{a+2}$ we have $a+2\mid -1$. 
		This is impossible since $a\ge 1$. 
		
		\item [N3.] 
		Find all positive integers $n$ with the following property: the $k$ positive divisors of $n$ have a permutation $(d_1,d_2, \cdots ,d_k)$ such that for every $i = 1,2,\cdots ,k$, 
		the number $d_1 + \cdots + d_i$ is a perfect square.
		
		\textbf{Answer.} $n=1, 3$. We have $d_1=1$ for the former, and $d_1=1, d_2=3$ for latter. 
		
		\textbf{Solution.} 
		By our condition, $d_i$ is a difference between two squares, 
		which follows that $d_i\not \equiv 2\pmod{4}$. 
		In particular, $d_i\neq 2$ for any $d_i$, so $n$ cannot be even.
		
		We now proceed with the following: 
		
		\begin{lemma}\label{lemma:n3}
			Let $x_i$ be the positive integer such that $x_i^2 = d_1 + \cdots + d_i$. 
			Then $x_{i-1}=\frac{d_i-1}{2}$ and $x_i=\frac{d_i + 1}{2}$ must hold. 
		\end{lemma}
		
		
		\begin{proof}[Proof of Lemma \ref{lemma:n3}]
			We perform induction on the divisors $d_i$ of $n$ on the size of $d_i$ itself: 
			when $d_i=1$, we use the fact that the only way to write $1=x^2-y^2$ is when $x=1$ and $y=0$ to conclude. 
			This also means $d_1=1$. 
			
			Now for some $d_i$, suppose we have that for all $d_j$'s with $j < i$, 
		    the lemma holds. 
		    Consider, now, writing $d_i = (x_i - x_{i - 1})(x_i + x_{i - 1})$, 
		    If $a = x_i - x_{i - 1}$ and $b = x_i + x_{i - 1}$, 
		    then $a, b$ are both divisors of $d_i$, and hence $n$. 
		    Suppose that $1<a, b<d_i$. 
		    It then follows that $a=d_j$ and $b=d_k$ for some $j, k$. 
		    By our assumption, $x_{j - 1} = \frac{a - 1}{2}$ and $x_{j} = \frac{a + 1}{2}$, 
		    given the monotonicity of $x_1, x_2, \cdots, x_k$, 
		    we have $x_{i-1}$ and $x_i$ either both $\le x_{j - 1} = \frac{a - 1}{2}$, 
		    or $\ge x_{j - 1} = \frac{a + 1}{2}$. 
		    Similarly, either $x_i\le \frac{b-1}{2}$, or $x_{i-1}\ge \frac{b+1}{2}$. 
		    On ther other hand, we can solve: 
		    \[
		    x_{i-1} = \frac{b-a}{2}
		    \qquad 
		    x_i = \frac{b+a}{2}
		    \]
		    So $x_i$ is $\ge$ both $\frac{a+1}{2}$ and $\frac{b+1}{2}$, 
		    which then follows that 
		    $\frac{b-a}{2}\ge \frac{b+1}{2}$ too. 
		    This is absurd as it implies $a\le 1$. 
		    
		    Hence we have $a=1, b=d_i$. This readily implies $x_{i-1}=\frac{d_i-1}{2}$ and $x_i = \frac{d_i+1}{2}$. 
		    
		\end{proof}
	    To finish off the problem, we have $x_i-x_{i-1}=1$ for all $i$, which then follows that $x_i=i$ and $d_i=2i-1$. 
	    This means the divisors of $n$ are all the odd numbers leading to $n$. 
	    In particular we have $n-2\mid n$ for $n>1$, which only makes sense when $n=3$. 
	    
	    \item [N4.] 
	    Alice is given a rational number $r > 1$ and a line with two points $B \neq R$, 
	    where the point $R$ contains a red bead and point $B$ contains a blue bead. 
	    Alice plays a solitaire game by performing a sequence of moves. 
	    In every move, she chooses a (not necessarily positive) integer $k$,
	    and a bead to move. 
	    If that bead is placed at point $X$, and the other bead is placed at $Y$, 
	    then Alice moves the chosen bead to point $X'$ with $\vec{YX'} = r^k\vec{YX}$. 
	    
	    Alice’s goal is to move the red bead to the point $B$. 
	    Find all rational numbers $r > 1$ such that Alice can reach her goal in at most 2021 moves.
	    
	    \textbf{Answer.} $r = 1+\frac 1{\ell}, \ell=1, 2, \cdots, 1010$. 
	    
	    \textbf{Solution.} 
	    W.l.o.g. let $B$ be at 1 and $R$ be at 0. 
	    To show that the $r$ given above is feasible, 
	    Alice can move in the following manner: 
	    in alternate fashion, Alice moves blue bead with $k=1$ and then red bead with $k=-1$. 
	    Then after each pair of moves the red bead is moved by $\frac{1}{\ell}$ to the right. 
	    Thus it will reach $1$ after $2\ell\le 2020$ moves. 
	    
	    Now let's consider $r=\frac{x}{y}$, with $\gcd(x, y)=1$. 
	    Suppose that $d\triangleq x - y\ge 2$. 
	    We now consider modulo $d$ on all rational numbers with both numerator and denominator relatively prime to $d$, 
	    such that $\frac{x'}{y'}\equiv k$ for integer $k$ if $x'k\equiv y'$. 
	    Then $r\equiv 1\pmod{d}$, and so is all its power. 
	    Since $r^k-r^{\ell}\equiv 0\pmod{d}$ for any pairs of integers $k, \ell$, 
	    the positions of the two beads modulo $d$ won't change regardless of the moves by Alice (given that the distances of the beads will always be $r^{x}$). 
	    In particular, the position of red bead will always be congruent to 0 $\pmod{d}$, 
	    i.e. cannot reach 1. 
	    
	    We're left to consider $r = 1 + \frac 1{\ell}$ for some $\ell$, so our goal now becomes showing that $\ell\le 1010$. (TODO)
	    
	    \item [N8.]
	    For a polynomial $P(x)$ with integer coefficients let $P^{1}(x) = P(x)$ and $P^{k+1}(x) = P(P^k(x))$ for $k \ge 1$. 
	    Find all positive integers $n$ for which there exist a polynomial $P(x)$ with integer coefficients such that for every integer $m \ge 1$, 
	    the numbers $P^m(1), \cdots, P^ m(n)$ leave exactly $\lceil n/2m\rceil$ distinct remainders when divided by $n$. 
	    
	    \textbf{Answer.} All the prime powers $p^k$ for some prime $p$ and $k\ge 0$. 
	    
	    \textbf{Solution.} 
	    Throughout we assume $P$ is a polynomial with integer coefficients. 
	    The following identity will be used profusely: for any $m\ge 0$ and integers $x\neq y$ we have 
	    \[
	    x-y\mid P^m(x)-P^m(y)
	    \]
	    Thus we may (sometimes) consider $P^m(\cdot \pmod{n})$. 
	    
	    We define $\sigma(P, m, n)$ as the number of distinct reimainders of $P^m(1), \cdots, P^m(n)$ when devided by $n$. 
	    A preliminary observation would be that for any polynomials $P$ and integers $m$ and $n$, 
	    $\sigma(P, m + 1, n)\le \sigma(P, m, n)$, with equality if and only if $\sigma(P, M, n)=\sigma(P, m, n)$ for all $M\ge m$. 
	    
	    We break down our solution into the following. 
	    
	    \begin{lemma}
	    	\label{lemma_n8a}
	    	Let $p$ and $q$ be any integers such that $\gcd(p, q)=1$. 
	    	Then for any polynomial $P$, $\sigma(P, m, pq)=\sigma(P, m, p)\cdot \sigma(P, m, q)$. 
	    \end{lemma}
        
        \begin{proof}
        	Consider the mapping $r: \bbN_{pq}\to \bbN_{p}\times \bbN_{q}$ by the following: 
        	for each $x$ with $0\le x\le pq-1$, if $x\equiv y\pmod{p}$ and $x\equiv z\pmod{q}$ 
        	then $r(x)=(y, z)$. Since $\gcd(p, q)=1$, this mapping is bijective via Chinese Remainder Theorem. 
        	
        	Now if $r(x)=(y, z)$, then $r(P^m(x)\pmod{pq})=(P^m(y)\pmod{p}, P^m(z)\pmod{q})$, since 
        	$p\mid x-y$ implies $p\mid P^m(x) - P^m(y)$ (and similarly for $q$ and $z$). It then follows that 
        	\[
        	|\{P^m(x)\pmod{pq}: x\in\bbN\}|
        	=|\{(P^m(y)\pmod{p}, P^m(z)\pmod{q}): y, z\in\bbN\}|
        	\]
        	\[
        	=|\{P^m(y)\pmod{p}: y\in\bbN\}|\cdot |\{ P^m(z)\pmod{q}: z\in\bbN\}|
        	\]
        	as desired. 
        \end{proof}
    
        Now if $n$ is divisible by at least two primes, it can be written as $n=pq$ with $1<p, q<n$. 
        The problem condition implies that $\sigma(P, m, n)=1$ for some (sufficiently large) $n$, 
        so $\sigma(P, m, p)=1=\sigma(P, m, q)=1$ for sufficiently large $m$. 
        Let $m_p$ (respectively $m_q$) be the minimum index such that $\sigma(P, m, p)$ (respectively $\sigma(P, m, q)$) is 1; 
        we have $m_p, m_q\ge 1$. 
        W.l.o.g, also, that $m_p\le m_q$. 
        Then by the lemma \ref{lemma_n8a} we have 
        \[
        \sigma(P, m_p, q) = \sigma(P, m_p, p)\cdot \sigma(P, m_p, q)
        =\sigma(P, m_p, n)
        =\lceil\frac{\sigma(P, m_p - 1, n)}{2}\rceil
        \]\[
        =\lceil\frac{\sigma(P, m_p - 1, p) \sigma(P, m_p - 1, q)}{2}\rceil
        \]
        By the minimality of $m_p$, $\sigma(P, m_p - 1, p)\ge 2$, 
        and $\sigma(P, m_p - 1, q) > \sigma(P, m_p, q)$, so 
        \[
        \lceil\frac{\sigma(P, m_p - 1, p) \sigma(P, m_p - 1, q)}{2}\rceil
        \ge \lceil\frac{2( \sigma(P, m_p, q) + 1)}{2}\rceil
        \ge \sigma(P, m_p, q) + 1
        \]
        i.e. a contradiction. 
        This effectively reduces $n$ to the cases of prime powers, which we consider in the following: 
        
        \begin{lemma}
        	\label{lemma_n8b}
        	Let $n=p^d$ for some prime $p$, and $a_1, \cdots, a_n$ be such that for each $k\le d$, 
        	$p^k \mid x-y$ implies $p^k\mid a_x-a_y$. 
        	Then there exists a polynomial $P$ with integer coefficients such that $P(x)\equiv a_x\pmod{n}$ for $x=1, \cdots, n$. 
        \end{lemma}
        \begin{proof}
        	We do induction on $d$. For base case $d=0$ there's nothing to prove since any polynomial would work. 
        	
        	Now suppose that for some $d\ge 1$, the conclusion above holds for $d-1$, 
        	i.e. there exists $P(x)$ such that $P(x)\equiv a_x\pmod{p^{d-1}}$ for all $x=1, \cdots, p^{d-1}$. 
        	Consider the number $r_x = P(x)-a_x$, which is divisible by $p^{d-1}$ for $x=1, \cdots, p^{d-1}$ (and therefore for all $x\in\bbZ$). 
        	
        	(LOL this lemma is wrong :( will fix stuff later)
        \end{proof}
        
        Now we can finish the proof. Here, consider $n=p^d$. 
        Consider the function $s(\cdot)$ that maps $\{0, 1, \cdots, p - 1\}$ to itself such that, 
        \[
        0\le a_i\le p - 1, 
          \ell = \sum_{i=0}^{d-1} a_ip^i \Rightarrow s(\ell) = \sum_{i=0}^{d-1} a_{d-i-1}p^i 
        \] 
        I.e. $s(\ell)$ is the number obtained by reversing the digits of $\ell$ written in base $p$ (with leading zeros until there are $d$ digits in total). 
        By Lemma \ref{lemma_n8b}, there exists a polynomial $P$ satisying 
        \[
        P(x) \equiv s(\lfloor \frac{s(x)}{2} \rfloor)\pmod{n}, \forall x=0, 1, \cdots, n-1
        \]
        Indeed, if $p^k\mid x-y$, then:
        \begin{itemize}
        	\item $x$ and $y$ have identical last $k$ digits;
        	\item $s(x)$ and $s(y)$ have identical first $k$ digits;
        	\item $\lfloor \frac{s(x)}{2} \rfloor$ and $\lfloor \frac{s(y)}{2} \rfloor$ have identical first $k$ digits
        	\item $s(\lfloor \frac{s(x)}{2} \rfloor)$ and $s(\lfloor \frac{s(y)}{2} \rfloor)$ have identical last $k$ digits
        \end{itemize}
        which then means $p^k \mid P(x)-P(y)$. 
        Finally, given that $s$ is bijective with $s(s(x))=x$, the distinct values of $P^m(x)$ mod $n$ are 
        $s(\lfloor \frac{x}{2^m}\rfloor), \forall x=0, 1, \cdots, n-1$, thus giving us 
        $1 + \lfloor \frac{n - 1}{2^m}\rfloor=\lceil \frac{n}{2^m}\rceil$ distinct values, as desired. 
         
	\end{enumerate}
\end{document}