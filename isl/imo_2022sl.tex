\documentclass[11pt,a4paper]{article}
\usepackage{amsmath, amssymb, fullpage, mathrsfs, bm, pgf, tikz, float}
\usepackage{mathrsfs,amsthm}
\usetikzlibrary{arrows}
\setlength{\textheight}{10in}
%\setlength{\topmargin}{0in}
\setlength{\topmargin}{-0.5in}
\setlength{\parskip}{0.1in}
\setlength{\parindent}{0in}

\begin{document}
	\newcommand{\la}{\leftarrow}
	\newcommand{\lra}{\leftrightarrow}
	\newcommand{\bbN}{{\mathbb N}}
	\newcommand{\bbZ}{{\mathbb Z}}
	\newcommand{\bbQ}{{\mathbb Q}}
	\newcommand{\bbR}{{\mathbb R}}
	\newcommand{\bbC}{{\mathbb C}}
	\newcommand{\bbH}{{\mathbb H}}
	\newcommand{\dfeq}{\stackrel{\mathrm{def}}{=}}
	\newcommand{\ra}{\rightarrow}
	\newcommand{\Span}{\mathrm{span}}
	\newcommand{\scrP}{\mathscr{P}}
	\newcommand{\rank}{\mathrm{rank}}
	\newcommand{\nullity}{\mathrm{nullity}}
	\newcommand{\Col}{\mathrm{Col}}
	\newcommand{\Row}{\mathrm{Row}}
	\newcommand{\tr}{\mathrm{tr}}
	\newcommand{\ol}{\overline}
	\newcommand{\norm}[1]{||#1||}
	\newcommand{\doubleline}[1]{\underline{\underline{#1}}}
	\newcommand{\elemop}[1]{\stackrel{#1}{\longrightarrow}}
	\newcommand{\Ind}{\mathrm{Ind}}
	\newcommand{\Res}{\mathrm{Res}}
	\newcommand{\End}{\mathrm{End}}
	\newcommand{\cl}{\mathrm{cl}}
	\newcommand{\code}[1]{\texttt{#1}}
	\newcommand\tab[1][0.5cm]{\hspace*{#1}}
	\newcommand{\<}{\langle}
	\renewcommand{\>}{\rangle}
	\newcommand{\qubits}[1]{|{#1}\rangle}
	\newcommand{\powset}{\mathcal{P}}
	\newcommand{\dsum}{\displaystyle\sum}
	\newcommand{\dprod}{\displaystyle\prod}
	
	\newtheorem{lemma}{Lemma}
	
	\section*{Algebra}
	\begin{enumerate}
		\item [A1.] 
		Let $(a_n)_{n\ge 1}$ be a sequence of positive real numbers with the property that 
		\[(a_{n+1})^2 + a_na_{n+2} \le a_n + a_{n+2}\]
		for all positive integers $n$. Show that $a_{2022}\le 1$.
		
		\textbf{Solution.} 
		We show that if $a_n > 1$, then $n \le 1$. 
		
		Suppose that $a_n > 1$ for some $n > 0$. 
		Then $(a_{n-1} - 1)(a_{n + 1} - 1) < 0$ so exactly one of the following scenarios happen: 
		$a_{n - 1} > 1$ or $a_{n + 1} > 1$. In either case, this implies consecutive numbers $a_m, a_{m + 1}$ that are both greater than 1. 
		On one hand we have 
		\[
		0 \ge (a_{m + 1}^2 - 1) + (a_m - 1)(a_{m + 2} - 1) \ge   (a_{m + 1}^2 - 1) + (a_m - 1)(-1) = a_{m + 1}^2 - a_m
		\]
		since $a_{m + 2}\ge 0$ and $a_m > 1$. This gives $a_{m + 1} < a_{m + 1}^2 \le a_m$. 
		In a similar way if $m \ge 1$ then $a_m < a_{m + 1}$. 
		But since $a_{m} < a_{m + 1}$ and $a_{m + 1} < a_m$ cannot happen simultaneously, 
		we have $m=0$, which proves our claim. 
		
		\item[A2.]
		Let $k \ge 2$ be an integer. 
		Find the smallest integer $n \ge k + 1$ with the property that there exist a set of $n$ distinct real numbers such that each of its elements can be written as a sum of $k$ other distinct elements of the set.
		
		\textbf{Answer.} $n = k + 4$. 
		
		\textbf{Solution.} 
		We sort the numbers $x_1 < \cdots < x_n$. 
		To show that $n\le k + 3$ is not enough, by considering the sum of $k$ numbers excluding $x_1$ (and respectively, $x_n$), 
		we have 
		\[
		x_1 \ge x_2 + \cdots + x_{k + 1}
		\qquad 
		x_n \le x_{n - k} + \cdots + x_{n - 1}
		\]
		i.e. $0 < x_n - x_1 \le (x_{n - k} + \cdots + x_{n - 1}) - (x_2 + \cdots + x_{k + 1})$. 
		The last quantity is nonpositive if $n\le k + 2$, 
		and if $n = k + 3$ then it becomes $x_{k + 2} - x_2 < x_n - x_1$, which is also impossible. 
		
		To give such a construction, we consider the following: 
		for $k$ even, take $(\pm 1, \cdots, \pm \frac{k}{2} + 2)$, 
		and for $k$ odd, take $(0, \pm 1, \cdots, \pm \frac{k - 1}{2} + 2)$. 
		To see why these work, we note that the case where $k=2$ and $n=6$ is obvious; 
		to go from $k$ to $k+2$ for $k$ even, 
		for all numbers not in the form $\pm \frac{k}{2} + 3$ we just have to add $( \frac{k}{2} + 3, -( \frac{k}{2} + 3))$ to the summands on top of the existing ones; 
		for those in the form $\frac{k}{2} + 3$, we can do 
		$(1, \cdots, \frac{k}{2} + 2, -2, \cdots, -(\frac{k}{2} + 1))$, 
		and proced similarly for $-(\frac{k}{2} + 3)$. 
		Finally, to go from $k$ to $k+1$ to $k$ even, 
		for all nonzero numbers simply add the newly added member 0 to the summand; 
		for 0, take $1, 2, -3, (4, -4), \cdots, (\frac{k+3}{2}, \frac{-k-3}{2})$. 
		
		\item[A3.] (IMO 2)
		Let $\mathbb{R}^+$ denote the set of positive real numbers. Find all functions $f: \mathbb{R}^+ \to \mathbb{R}^+$ such that for each $x \in \mathbb{R}^+$, there is exactly one $y \in \mathbb{R}^+$ satisfying$$xf(y)+yf(x) \leq 2$$
		
		\textbf{Answer.} The only such function is $f(x) = \frac 1x$, where the `matching' $y$ for each $x$ is $y = x$. 
		
		\textbf{Solution.} 
		A quick note that the function works: when $x=y$ we get LHS=2 exactly; 
		when $x\neq y$ we have $\frac{x}{y}+\frac{y}{x} = 2 + \frac{(x-y)^2}{xy} > 2$. 
		
		We now show that for each $x$, the $y$ that satisfies the desired inequality must have $y = x$. 
		Indeed, if $x\neq y$ are such that $xf(y)+yf(x) \leq 2$, 
		then $f(x) > \frac 1x$ and $f(y) > \frac 1y$. 
		However, 
		\[
		xf(y)+yf(x) > \frac xy + \frac yx > 2
		\]
		whenever $x\neq y$, a contradiction. 
		
		Thus we have $xf(x)\le 1$ for all $x$, or $f(x)\le \frac 1x$. 
		We show that equality has to hold here. 
		Indeed, fix an $x$ and for each $y\neq x$ we have 
		\[
		2 < xf(y) + yf(x) \le \frac xy + yf(x)
		\]
		I.e. $f(x) > \frac{2y - x}{y^2}$ for each $y > x$. 
		The threshold $\frac{2y - x}{y^2}$ is continuous in $y$ (in the neighbourhood of $x$) and we consider the limit 
		$\lim_{y\to x^+} \frac{2y - x}{y^2} = \frac 1x$, 
		showing that $f(x)\ge \frac 1x$ must hold. 
		
		\item[A4.]
		Let $n \ge 3$ be an integer, and let $x_1, \cdots , x_n$ be real numbers in the interval $[0, 1]$. 
		Let $s=x_1+ \cdots +x_n$, and assume that $s\ge 3$. 
		Prove that there exist integers $i$ and $j$ with $1\ge i < j\le n$ such that
		$2^{j - i}x_ix_j > 2^{s - 3}$.
		
		\item[A6.]
		Let $\bbR$ be the set of all real numbers. 
		Denote by $\mathcal{F}$ the set of all functions $f: \bbR\to\bbR$ such that 
		\[
		f(x+f(y)) = f(x) + f(y)
		\]
		for every $x, y\in\bbR$. 
		Find all rational numbers $q$ such that for every function $f\in\mathcal{F}$, 
		there exists some $z\in\bbR$ satisfying $f(z)=qz$. 
		
		\textbf{Answer.}
		All $q$ in the form $\{\frac{k+1}{k}, \forall k\in\bbZ\backslash \{0\}\}$. 
		
		\textbf{Solution.}
		Consider any fixed $f$. If $f(0)=0$, then any rational $q$ works for that $f$. 
		Therefore we assume that $f(0)\triangleq m \neq 0$, 
		and plugging $x=km$ and $y=0$ gives us $f((k + 1)m) = f(km) + m$, for any integer $k$. 
		This relation gives us that $f(km) = f(0) + km = (k + 1)m$, 
		which means $q = \frac{k + 1}{k}$ is fulfilled by this $f$ whenever $k\neq 0$. 
		It then follows that $\{\frac{k+1}{k}, \forall k\in\bbZ\backslash \{0\}\}$ works. 
		
		To show that there is no other solution, 
		it suffices to pick any $q$ that does not have the form $\frac{k+1}{k}$ 
		and a function $f$ such that $f(z)\neq qz$ for any real $z$. 
		Consider the following (rather pathologically constructed) function template $f$: 
		\[
		f(0) = 1;\qquad  f(x)\in \bbZ, \forall x\in (0, 1); \qquad 
		f(x + 1) = f(x) + 1, \forall x\in\bbR
		\]
		Then since $f(y)$ takes only integer values and our construction implies $f(x + k)=f(x) + k$ for all integers $k$, 
		the condition for $f$ is fulfilled. 
		In other words, it then suffices to consider all $z$ and $f(z)$ such that $z$ is a rational. 
		Notice also that $f(n) = n + 1$ for all integers $n$ and $\frac{n+1}{n}\neq q$ by our assumption. 
		
		To continue with the construction, we note that we are free to determine the values of $f(x)$ for all $x\in (0, 1)$, 
		as long as it is an integer. Now let $q=\frac{r}{s}$ with $s > 0$ and $\gcd(r, s)=1$. 
		Since $q$ is not in our claimed form, $|r - s|\neq 1$. 
		Now we have $f(x + n) = f(x) + n$, so we need the ratio $\frac{f(x) + n}{x + n}\neq \frac{r}{s}$ 
		for all integers $n$. 
		In other words we are looking at $f(x)\neq \frac{r}{s}\cdot x + n(\frac{r}{s} - 1)$. 
		It then suffices to show that $\mathcal{F}\triangleq \{\frac{r}{s}\cdot x + n(\frac{r}{s} - 1): n\in\bbZ\}$ cannot contain all integers. 
		Indeed, by partitioning the numbers in $\bbR$ according to their fractional part 
		(so that two numbers are in the same class if and only if their difference is an integer), 
		the numbers $\frac{r}{s}\cdot x + n(\frac{r}{s} - 1)$ that are an integer must have 
		$n(\frac{r}{s} - 1)$ in the same class. 
		Since $\gcd(r, s) = 1$, it then follows that the corresponding $n$'s are congruent mod $s$, 
		and the attainable integers in $\mathcal{F}$ has to be congruent modulo $|r - s|$. 
		Thus such example of $f(x)$ can be constructed since $|r-s|\neq 1$. 
		Since this works for all $x\in (0, 1)$, it follows that such $f$ can be constructed. 
		
	\end{enumerate}
    
    \section*{Combinatorics}
    \begin{enumerate}
    	\item [C1.]
    	A $\pm 1$-sequence is a sequence of $2022$ numbers $a_1, \ldots, a_{2022},$ each equal to either $+1$ or $-1$. Determine the largest $C$ so that, for any $\pm 1$-sequence, there exists an integer $k$ and indices $1 \le t_1 < \ldots < t_k \le 2022$ so that $t_{i+1} - t_i \le 2$ for all $i$, and$$\left| \sum_{i=1}^{k} a_{t_i} \right| \ge C.$$
    	
    	\textbf{Answer.} 506. 
    	
    	\textbf{Solution.} 
    	In general we show that a sequence with $4k+2$ numbers will yield the answer $k + 1$; here $k=505$. 
    	
    	We first partition the numbers into blocks of consecutive $+1$ and $-1$; 
    	w.l.o.g. suppose there are at least as many $+1$'s as $-1$'s. Then we choose all the $+1$'s, 
    	and for each block of $-1$'s we choose them in alternating order, skipping the first and last entry in the case of odd sized block 
    	(which then means for a block of sized $m$ we would pick $\lfloor \frac{m}{2}\rfloor$ $-1$'s). 
    	Then we will choose at least $2k+1$ of $+1$'s, and at most $k$ of $-1$'s, giving the answer $k + 1$. 
    	
    	To show that $k+1$ cannot be improved, consider the following example: 
    	\[
    	+1, +1, -1, -1, \cdots, +1, -1
    	\]
    	i.e. $(+1, +1, -1, -1)$ repeated for $k$ times, and finally $+1, -1$. Suppose $\ell\le k$ blocks of  $+1$ with length 2 are chosen, then if the last $+1$ was chosen, then there's at least $\ell$ blocks of $-1$ with length 2 in between; we're required to pick at least one $-1$'s from each block. 
    	This gives the sum $\le 2\ell+1-\ell=\ell+1\le k+1$. 
    	Conversely, if the last $+1$ was not chosen, then there's at least $\ell - 1$ blocks of $-1$ with length 2 in between; again we're required to pick at least one $-1$'s from each block. 
    	This gives the sum $\le 2\ell-\ell + 1=\ell+1\le k+1$. 
    	
    	\item[C2.] (IMO 1)
    	The Bank of Oslo issues two types of coin: aluminum (denoted A) and bronze (denoted B). Marianne has $n$ aluminum coins and $n$ bronze coins arranged in a row in some arbitrary initial order. A chain is any subsequence of consecutive coins of the same type. Given a fixed positive integer $k \leq 2n$, Gilberty repeatedly performs the following operation: he identifies the longest chain containing the $k^{th}$ coin from the left and moves all coins in that chain to the left end of the row. For example, if $n=4$ and $k=4$, the process starting from the ordering $AABBBABA$ would be $AABBBABA \to BBBAAABA \to AAABBBBA \to BBBBAAAA \to ...$
    	
    	Find all pairs $(n,k)$ with $1 \leq k \leq 2n$ such that for every initial ordering, at some moment during the process, the leftmost $n$ coins will all be of the same type.
    	
    	\textbf{Answer.} 
    	$n\le k\le \lceil \frac {3n}{2}\rceil$. 
    	
    	\textbf{Solution.} 
    	If $k < n$, a counter-configuration will be $\underbrace{A\cdots A}_{n - 1} \underbrace{B\cdots B}_{n} \underbrace{A\cdots A}_{1}$, 
    	and no change will be made after each move (since the $k$-th coin from the left is always on the first block). 
    	If $k > \lceil \frac {3n}{2}\rceil$, 
    	consider $\underbrace{A\cdots A}_{\lceil \frac n2\rceil}
    	\underbrace{B\cdots B}_{\lfloor \frac n2\rfloor}
    	\underbrace{A\cdots A}_{\lfloor \frac n2\rfloor}
    	\underbrace{B\cdots B}_{\lceil \frac n2\rceil}$. 
    	Then the $k$-th coin will always be in the last block 
    	and the configuration will alternative between $ABAB$ and $BABA$ (each letter here indicates a block). 
    	
    	We now concentrate on the cases $n\le k\le \lceil \frac {3n}{2}\rceil$. 
    	To do this, we consider the number of blocks of consecutive equal coin; 
    	the minimum possible number is 2 which gives us the desired configuration. 
    	We now have several observations: 
    	\begin{itemize}
    		\item Since $n\ge k$, as long as there are at least 3 blocks the $k$-th coin cannot be in the first block; 
    		
    		\item If the $k$-th coin is in the middle (i.e. neither first nor last) then the number of blocks decreases by at least one; 
    		
    		\item If the $k$-th coin is in the last block, then number of blocks decreases by 1 if we have odd number of blocks to start with, 
    		and stays the same otherwise. 
    	\end{itemize}
        
        We now consider the moment where the number of blocks stabilizes, i.e. the number of blocks is even and the $k$-th coin is in the last (rightmost) block. 
        If there are $\ge 4$ blocks, 
        then the operations are basically just rotating each block of consecutive coins. 
        The shortest of these blocks has at most $\lfloor \frac n2\rfloor$ coins so this will force $k > \lceil \frac {3n}{2}\rceil$, 
        a contradiction. 
    	
    	\item[C3.]
    	In each square of a garden shaped like a 2022 $\times$ 2022 board, 
    	there is initially a tree of height 0. 
    	A gardener and a lumberjack alternate turns playing the following game, 
    	with the gardener taking the first turn:
    	\begin{itemize}
    		\item The gardener chooses a square in the garden. Each tree on that square and all the surrounding squares (of which there are at most eight) then becomes one unit taller. 
    		
    		\item The lumberjack then chooses four different squares on the board. Each tree of positive height on those squares then becomes one unit shorter.
    	\end{itemize}
    	We say that a tree is majestic if its height is at least $10^6$. 
    	Determine the largest number $K$ such that the gardener can ensure there are eventually $K$ majestic trees on the board, 
    	no matter how the lumberjack plays.
    	
    	\textbf{Answer.} $K=2022^2\times \frac 59$. 
    	
    	\textbf{Solution.} Name the positions of the grid in terms as coordinates $(x, y)$ with $1\le x, y\le 2022$. 
        Colour a square $(x, y)$ black if both $x$ and $y$ are congruent to 1 or 2 modulo 3, 
        and the rest white. 
        Then there are exactly $2022^2\times \frac 59$ white squares. 
        For any square and its surrounding squares (i.e. a $3\times 3$ subgrid), 
        there are at most 4 black squares (based on our construction). 
        It then follows that after gardener makes their move of increasing the heights of some trees, 
        the lumberjack can choose the black squares that the gardener picks in the previous round
        (if there is fewer than 4 such black squares, choose other such squares arbitrarily). 
        It then follows that the Lumberjack can force a height of 0 for all the black squares after however many steps, 
        establishing an upper bound on $K$. 
        
        To show a matching lower bound, gardener can partition the garden into $674^2$ subgrid of $3\times 3$ squares each, 
        and keep on choosing each subgrid to increase the height (each subgrid being chosen exactly once until all are chosen, then repeat). 
        Consider doing this for $M = (4\cdot 674^2 + 1)\cdot 10^{20}$ rounds, 
        Suppose that there are $L = 2022^2 - K - 1 = 4\cdot 674^2 + 1$ grids where lumberjack can prevent the tree in each grid from growing over a height of $10^6$. 
        This means the lumberjack must have chosen each of the $L$ squares for $M - 10^6$ times. 
        Given that each player does $M\cdot 674^2$ moves (each allowing the lumberjack to choose 4 victim grids), this gives 
        \[
        4M\cdot 674^2 \ge (M - 10^6)L = (4\cdot 674^2 + 1)M
        \]
        This only holds when $M \le (4\cdot 674^2 + 1)\cdot 10^6$, a contradiction. 
        
        \item [C6.]
        Let $n$ be a positive integer. We start with $n$ piles of pebbles, each initially containing a single pebble. One can perform moves of the following form: choose two piles, take an equal number of pebbles from each pile and form a new pile out of these pebbles. Find (in terms of $n$) the smallest number of nonempty piles that one can obtain by performing a finite sequence of moves of this form.
        
        \textbf{Answer.} 1 if $n$ is power of 2; 2 otherwise. 
        
        \textbf{Answer.} If $n=2^k$, we can transform the piles into $2^{k-1}$ piles of 2, 
        $2^{k-2}$ piles of 4, etc, until a single pile is left. 
        
        Now consider the case where $n$ is not a power of two. Notice that there is a unique integer sequence 
        $0\le a_1 < a_2 < \cdots < a_{\ell}$ such that $n = \sum_{i=1}^{\ell} 2^{a_{i}}$. 
        This means we may partition the initial piles of pebbles into groups of $2^{a_i}$ piles, 
        merge each group until a single pile of $2^{a_i}$ pebbles is left. 
        Notice that $n < 2^{a_{\ell} + 1}$. We now do the following operations: 
        \begin{itemize}
        	\item If there are two piles of the same size, merge them into one; 
        	
        	\item otherwise, choose the smallest one (say, size $a$), remove $a$ pebbles from the biggest and smallest piles, 
        	and form a pile of size $2a$. 
        \end{itemize}
        We stop when there are only two piles left. 
        
        Here is an example when $n = 23$: 
        \[
        (16, 4, 2, 1) \to (15, 4, 2, 2) \to (15, 4, 4)\to (15, 8)\to (7, 16)
        \]
        
        To show that this would reduce the number of piles to two, the invariant we will always maintain (up until the step before two piles left) is that the $a$ in the second case is always power of 2, 
        and the biggest pile is always the same and has strictly more than $\frac{n}{2}$ pebbles. 
        This way, at each step the total number of pebbles of the piles excluding the biggest one is always in the form of $x\to x+2^k$ where $k$ is the biggest exponent of 2 dividing $x$, 
        so eventually this $x$ will reach $2^{a_{\ell}}$ as a single pile. 
        
        Finally, we show that for such $n$, we cannot have a single pile left. 
        The fact that $n$ is not a power of 2 means that there is an odd $d > 1$ such that $d\mid n$; 
        we show that at any time not all piles have size divisble by $d$. 
        Indeed, since $d > 1$, this is true in the beginning. 
        Now suppose that this holds at some stage, 
        and consider a move with $q$ pebbles taken from each source pile into a new pile of $2q$ pebbles. 
        If $d\nmid q$ then $d\nmid 2q$ since $d$ is odd; 
        if $d\mid q$ then the number of pebbles modulo $d$ of the source piles does not change, 
        so the invariant remains. 
        Thus we conclude that one pile is impossible. 
    	
    \end{enumerate}
    
    \section*{Geometry}
    \begin{enumerate}
    	\item [G1.] (IMO 4)
    	Let $ABCDE$ be a convex pentagon such that $BC=DE$. Assume that there is a point $T$ inside $ABCDE$ with $TB=TD,TC=TE$ and $\angle ABT = \angle TEA$. Let line $AB$ intersect lines $CD$ and $CT$ at points $P$ and $Q$, respectively. Assume that the points $P,B,A,Q$ occur on their line in that order. Let line $AE$ intersect $CD$ and $DT$ at points $R$ and $S$, respectively. Assume that the points $R,E,A,S$ occur on their line in that order. Prove that the points $P,S,Q,R$ lie on a circle.
    	
    	\textbf{Solution.} 
    	Here, $\angle STE = \angle QTD$, and together with $\angle ABT = \angle AET$ we have 
    	$\angle AST = \angle AQV$. 
    	Therefore if $TD$ intersects $AB$ at $U$ and $CT$ intersects $AE$ at $V$ we have $SQVU$ cyclic, 
    	and $AS\cdot AV = AU\cdot AQ$. 
    	Notice that the problem is equivalent to showing that $AS\cdot AR = AP\cdot AQ$, 
    	so it suffices to show that $UV\parallel PR$, or equivalently $UV\parallel CD$. 
    	
    	Now, note that triangles $TBC$ and $TDE$ are congruent, 
    	so in particular $\angle BTC = \angle DTE$. 
    	This yields $\angle BTU = \angle ETV$ and so triangles $BTU$ and $ETV$ are in fact similar, 
    	with similitude $\frac{TU}{TV}=\frac{TB}{TE} = \frac{TD}{TC}$. 
    	Together with that $U, T, D$ collinear and $V, T, C$ collinear, we indeed have $UV\parallel CD$, 
    	as desired. 
    	
    	\item [G2.]
    	In the acute-angled triangle $ABC$, 
    	the point $F$ is the foot of the altitude from $A$, 
    	and $P$ is a point on the segment $AF$. 
    	The lines through $P$ parallel to $AC$ and $AB$ meet $BC$ at $D$ and $E$ respectively. 
    	Points $X \neq A$ and $Y \neq A$ lie on circles $ABD$ and $ACE$ respectively, 
    	such that $DA = DX$ and $EA = EY$. 
    	
    	Prove that $B, C, X$ and $Y$ are concyclic.
    	
    	\textbf{Solution.} 
    	We have $FB\cdot FD = FE\cdot FC = FB\cdot FC\cdot \frac{FP}{FA}$, 
    	and therefore $F$ has equal power of point on circles $ABD$ and $ACE$. 
    	It therefore follows that $AF$ is the radical axis of circles $ABD$ and $ACE$. 
    	
    	Next, by angle chasing and the identity $DA=DX$, 
    	we have $BC$ the external angle bisector of $\angle XBA$, 
    	and also the external angle bisector of $\angle YCA$. 
    	This means that $BX$ and $CA$ will intersect at $A'$, the reflection of $A$ in $BC$, 
    	which lies on $AF$. 
    	
    	Thus $BX, CA$ and the radical axis of circles $ABD$ and $ACE$ are concurrent, 
    	so $B, C, X, Y$ are concyclic. 
    	
    	\item [G3.]
    	Let $ABCD$ be a cyclic quadrilateral. 
    	Assume that the points $Q, A, B, P$ are collinear in this order, 
    	in such a way that the line $AC$ is tangent to the circle $ADQ$, and the line $BD$ is tangent to the circle $BCP$. 
    	Let $M$ and $N$ be the midpoints of $BC$ and $AD$, respectively. 
    	Prove that the following three lines are concurrent: line $CD$, 
    	the tangent of circle $ANQ$ at point $A$, and the tangent to circle $BMP$ at point $B$.
    	
    	
    	\item [G4.]
    	Let $ABC$ be an acute-angled triangle with $AC > AB$, let $O$ be its circumcentre, and let $D$ be a point on the segment $BC$. The line through $D$ perpendicular to $BC$ intersects the lines $AO, AC,$ and $AB$ at $W, X,$ and $Y,$ respectively. The circumcircles of triangles $AXY$ and $ABC$ intersect again at $Z \ne A$.
    	Prove that if $W \ne D$ and $OW = OD,$ then $DZ$ is tangent to the circle $AXY.$
    	
    	\textbf{Solution.} We first show that $AZ\parallel BC$, and that $D, Z, O$ is collinear. 
    	Let ray $DO$ intersect the circle $ABC$ again at $Z'$ (which we will show that $Z=Z'$ later). 
    	By angle chasing, using $OW=OD$ and $DW\perp BC$, 
    	we get $\angle BOD = 180^{\circ} - 2\angle ABC$ and consequently $\angle Z'BO=\angle BZO=90^{\circ}-\angle ABC$. 
    	Thus $\angle Z'BC = \angle ACB$ and similarly $\angle Z'CB = \angle ABC$, 
    	and $ZACB$ is indeed an isoceles trapezoid. 
    	To show that $Z'=Z$, the identity of spiral similarity suggests that 
    	\[
    	\frac{ZB}{ZC} = \frac{BY}{CX} = \frac{BD/\cos\angle ABC}{CD/\cos\angle ACB}
    	\]
    	and given that $\frac{BD}{CD} = \frac{OB\cdot\sin\angle BOD}{OC\cdot\sin\angle COD}
    	=\frac{\sin 2\angle ABC}{\sin 2\angle ACB}$, we have $\frac{ZB}{ZC}  = \frac{\sin\angle ABC}{\sin\angle ACB} = \frac{AC}{AB} = \frac{Z'B}{Z'C}$, so $Z'=Z$.  
    	
    	To finish off the solution, we note that $\angle BAO = 90^{\circ} - \angle ACB = \angle CXD = \angle AXY$, 
    	so $OA$ is tangent to circle $AXY$. Since $OA=OZ$, $OZ$ is also tangent to this circle $AXY$. 
    	Thus the conclusion follows by $D, Z, O$ collinear. 
    	
    	\item [G5.]
    	Let $ABC$ be a triangle, and let $\ell_1$ and $\ell_2$ be two parallel lines. 
    	For $i = 1,2$, let $\ell_i$ meet the lines $BC, CA, AB$ at $X_i, Y_i$ and $Z_i$ respectively. 
    	Suppose that the line through $X_i$ perpendicular to $BC$, 
    	the line through $Y_i$ perpendicular to $CA$, the line through $Z_i$ perpendicular to $AB$, 
    	determine a non-degenerate triangle $\triangle_i$. 
    	
    	Show that the circumcircles of $\triangle_1$ and $\triangle_2$ are tangent to each other.
    	
    	\textbf{Solution.} Let $A_1$ be the point such that $A_1Y_1\perp CA$ and $A_1Z_1\perp AB$, 
    	and define $B_1, C_1$ similarly (and similarly define $A_2B_2C_2$). 
    	Then we have $A_1B_1\perp AB$ and $A_2B_2\perp AB$ so $A_1B_1\parallel A_2B_2$ and similarly for the other two sides too. 
    	This gives triangles $A_1B_1C_1$ (i.e. $\triangle_1$) and $A_2B_2C_2$ (i.e. $\triangle_2$) homothetic, 
    	so it suffices to show that the center of homothety, namely $Z$, lie on both circumcircles to establish their tangency. 
    	
    	We first show that $C, C_1, C_2$ are collinear. 
    	Indeed, $\angle CX_2C_2 = \angle CY_2C_2 = 90^{\circ}$, 
    	so they the four points $C, X_2, C_2, Y_2$ lie on a circle. 
    	Using angle modulo $180^{\circ}$, we can use directed angle to get 
    	\[
    	\angle(CC_2, AC) = \angle(CC_2, CY_2) = \angle(X_2C_2, X_2Y_2) = \angle(CX_2, \ell_2) + 90^{\circ}
    	=\angle(BC, \ell_2) + 90^{\circ}
    	\]
    	and similarly, 
    	$\angle(CC_1, AC) = \angle(BC, \ell_1) + 90^{\circ}$. The claim then follows as $\ell_1\parallel \ell_2$. 
    	This means $C, C_1, C_2, Z$ are collinear, and similarly so for $A, A_1, A_2, Z$ and $B, B_1, B_2, Z$. 
    	
    	Thus now we have 
    	\[
    	\angle(A_2Z, B_2Z)
    	=\angle(A_2A, B_2B)
    	=\angle(A_2A, AB) + \angle(AB, B_2B)
    	=\angle(A_2A, AZ_2) + \angle(BZ_2, B_2B)
    	\]\[
    	=\angle(A_2Y_2, Y_2Z_2) + \angle(X_2Z_2, B_2X_2)
    	=\angle(A_2C_2, \ell_2) + \angle(\ell_2, B_2C_2)
    	=\angle(A_2C_2, B_2C_2)
    	\]
    	showing that $A_2, B_2, C_2, Z$ are indeed concyclic. In a similar manner we can show that $Z$ lies on the circumcircle of triangle $A_1B_1C_1$ too. 
    	
    	\item [G6.]
    	In an acute-angled triangle $ABC$, 
    	point $H$ is the foot of altitude from $A$. 
    	Let $P$ be a moving point such that the bisectors $k$ and $\ell$ of angles $PBC$ and $PCB$, respectively, 
    	intersect each other on the line segment $AH$. Let $k$ and $AC$ meet at $E$, let $\ell$ and $AB$ meet at $F$, 
    	and let $EF$ and $AH$ meet at $Q$.
    	
    	Prove that as $P$ varies, the line $PQ$ passes through a fixed point.
    	
    	\textbf{Solution.} 
    	Let $O$ be the circumcenter of $ABC$, 
    	$K$ the reflection of $H$ in the midpoint of $BC$, 
    	and $J$ the reflection of $H$ in $A$. 
    	We will show that $PQ, AO$ and $KJ$ are concurrent. 
    	
    	\definecolor{xdxdff}{rgb}{0.49019607843137253,0.49019607843137253,1}
    	\definecolor{uuuuuu}{rgb}{0.26666666666666666,0.26666666666666666,0.26666666666666666}
    	\definecolor{ududff}{rgb}{0.30196078431372547,0.30196078431372547,1}
    	\begin{tikzpicture}[line cap=round,line join=round,>=triangle 45,x=1cm,y=1cm, scale=0.5]
    		\clip(-18.34832110193102,-9.606052304842978) rectangle (15.903337388959855,12.371473182187397);
    		\draw [line width=2pt] (-1.5692965301365722,1.1599583545330783) circle (5.021804530224832cm);
    		\draw [line width=2pt] (-1.707983777112049,-3.037642320591351) circle (4.987245102603137cm);
    		\draw [line width=2pt] (-4.620431348934434,5.14857930133988)-- (-6.18,-0.83);
    		\draw [line width=2pt] (-6.18,-0.83)-- (2.9,-1.13);
    		\draw [line width=2pt] (2.9,-1.13)-- (-4.620431348934434,5.14857930133988);
    		\draw [line width=2pt] (-4.421415990917295,11.172110803991934)-- (-4.819446706951573,-0.8749522013121727);
    		\draw [line width=2pt] (-5.664417635925985,1.1464760261644888)-- (-2.642430810787526,3.497206838694108);
    		\draw [line width=2pt] (-2.642430810787526,3.497206838694108)-- (-6.18,-0.83);
    		\draw [line width=2pt] (-5.664417635925985,1.1464760261644888)-- (2.9,-1.13);
    		\draw [line width=2pt] (-4.421415990917295,11.172110803991934)-- (3.96786975558755,-6.078547560421324);
    		\draw [line width=2pt] (-4.620431348934434,5.14857930133988)-- (3.96786975558755,-6.078547560421324);
    		\draw [line width=2pt,dash pattern=on 1pt off 1pt] (-7.4030750508543814,4.320230802138943)-- (3.96786975558755,-6.078547560421324);
    		\draw [line width=2pt] (-7.4030750508543814,4.320230802138943)-- (1.3446293619591614,-6.981519440458144);
    		\draw [line width=2pt] (1.3446293619591614,-6.981519440458144)-- (1.5394467069515727,-1.0850477986878273);
    		\begin{scriptsize}
    			\draw [fill=ududff] (-4.620431348934434,5.14857930133988) circle (2.5pt);
    			\draw[color=ududff] (-5.062090584112141,6.011363154749128) node {$A$};
    			\draw [fill=ududff] (-6.18,-0.83) circle (2.5pt);
    			\draw[color=ududff] (-6.82952491905135,-0.7536441272595498) node {$B$};
    			\draw [fill=ududff] (2.9,-1.13) circle (2.5pt);
    			\draw[color=ududff] (3.3789320155113223,-0.8145901388091774) node {$C$};
    			\draw [fill=uuuuuu] (-4.819446706951573,-0.8749522013121727) circle (2pt);
    			\draw[color=uuuuuu] (-4.940198561012886,-1.119320196557316) node {$H$};
    			\draw [fill=uuuuuu] (-4.421415990917295,11.172110803991934) circle (2pt);
    			\draw[color=uuuuuu] (-4.544049485940304,12.045018298162274) node {$J$};
    			\draw [fill=uuuuuu] (-1.5692965301365722,1.1599583545330783) circle (2pt);
    			\draw[color=uuuuuu] (-1.3139108738100258,1.7451423462751874) node {$O$};
    			\draw [fill=uuuuuu] (1.4818382886612897,-2.828662592273724) circle (2pt);
    			\draw[color=uuuuuu] (1.8248087209958108,-2.521078462198754) node {$X$};
    			\draw [fill=uuuuuu] (1.5394467069515727,-1.0850477986878273) circle (2pt);
    			\draw[color=uuuuuu] (1.0934565824002762,-0.6012790983854803) node {$K$};
    			\draw [fill=xdxdff] (-4.7605969161832595,0.9062347992754418) circle (2.5pt);
    			\draw[color=xdxdff] (-4.51357648016549,1.5623043116263042) node {$I$};
    			\draw [fill=uuuuuu] (-5.664417635925985,1.1464760261644888) circle (2pt);
    			\draw[color=uuuuuu] (-6.402902838203954,1.5013583000766764) node {$E$};
    			\draw [fill=uuuuuu] (-2.642430810787526,3.497206838694108) circle (2pt);
    			\draw[color=uuuuuu] (-2.4109390817033276,4.091563790935855) node {$F$};
    			\draw [fill=uuuuuu] (-4.7286084355980895,1.8744194783199084) circle (2pt);
    			\draw[color=uuuuuu] (-4.483103474390676,2.6593325195196034) node {$Q$};
    			\draw [fill=uuuuuu] (-7.4030750508543814,4.320230802138943) circle (2pt);
    			\draw[color=uuuuuu] (-7.713242086520954,5.097172981504713) node {$P$};
    			\draw [fill=uuuuuu] (1.3446293619591614,-6.981519440458144) circle (2pt);
    			\draw[color=uuuuuu] (1.3067676228239737,-7.305340368844531) node {$Y$};
    			\draw [fill=uuuuuu] (1.4505245703631906,-3.7764244660961968) circle (2pt);
    			\draw[color=uuuuuu] (0.9410915535262063,-3.5876336643172397) node {$Z$};
    			\draw [fill=uuuuuu] (-1.707983777112049,-3.037642320591351) circle (2pt);
    			\draw[color=uuuuuu] (-1.4662759026840955,-2.4296594448743125) node {$M$};
    			\draw [fill=uuuuuu] (-3.344208119958603,-0.9236935643185484) circle (2pt);
    			\draw[color=uuuuuu] (-3.4470212780470018,-0.2051300233129001) node {$S$};
    		\end{scriptsize}
    	\end{tikzpicture}
    	
    	We introduce the additional points and lengths as follows: 
    	let $AO$ intersect the line through $K$ perpendicular to $BC$ at $X$, 
    	the circumcircle of $IBC$ intersect ray $KX$ at $Y$. 
    	Let $Z$ be the intersection of $PQ$ and $KX$; 
    	we're interested in the length $ZX$. 
    	
    	Set $AD = 1$, 
    	$BH\cdot HC = R^2$, 
    	and also $HI = x < 1$. 
    	Then we have the following relation: 
    	\begin{itemize}
    		\item $(Q, H; A, I)$ are harmonic, 
    		so $\frac{QI}{IH} = \frac{AQ}{AH}$. 
    		Consequently, 
    		we can compute 
    		\[
    		\frac{QI}{x} = \frac{1 - x - QI}{1}
    		\to QI = \frac{x(1 - x)}{1 + x}
    		\]
    		and also 
    		\[
    		QH = \frac{2x}{1 + x}
    		\qquad 
    		AQ = \frac{1 - x}{1 + x}
    		\]
    		
    		\item Notice that $A, B, C, X$ are concyclic, 
    		and since $H$ and $K$ are symmetric w.r.t. midpoint of $BC$, 
    		we may compute $BH\cdot HC = AH\cdot KX$, 
    		giving $KX = R^2$. 
    		The same goes to $I, B, C, Y$, 
    		yielding $KY = \frac{R^2}{x}$. 
    	\end{itemize}
    	
    	We now determine the length $YZ$. 
    	Observe that $\frac{YZ}{QI} = \frac{YP}{PI}$. 
    	Let $M$ be the circumcenter of $IBCY$, and $S$ be the intersection of $PY$ and line $BC$. 
    	We have $M$ on the line $YI$ itself 
    	(note that $\angle IBY = \angle ICY = 90^{\circ}$). 
    	This entails $IM=MY$ and therefore 
    	\[
    	\frac{YP}{YI}
    	=\frac{YM + MP}{2YM}
    	=\frac{YM + \frac{YM^2}{MS}}{2YM}
    	=\frac 12 \left(1 + \frac{YM}{MS}\right)
    	\]
    	Finally, notice that $M$ is equidistant from the two parallel lines $IH$ and $KY$ (the two lines are symmetric w.r.t. the perpendicular bisector of $BC$ which $M$ is on). 
    	This means we may now compute 
    	\[
    	\frac{YM}{MS}
    	=\frac{YS - MS}{MS}
    	=\frac{R^2/x}{(R^2/x - x) / 2} - 1
    	\]
    	which then means 
    	\[
    	\frac{YP}{YI} = \frac{R^2/x}{R^2/x - x}
    	=\frac{R^2}{R^2 - x^2}
    	\]
    	i.e. $\frac{YP}{PI} = \frac{YP}{YP - YI} = \frac{R^2}{x^2}$. 
    	This means 
    	\[
    	YZ = QI \cdot \frac{YP}{PI}
    	=\frac{x(1 - x)}{1 + x}\cdot  \frac{R^2}{x^2}
    	= \frac{R^2(1 - x)}{x(1+x)}
    	\]
    	and 
    	\[
    	XZ
    	= KY - KX - YZ
    	= \frac{R^2}{x} - R^2 - \frac{R^2(1 - x)}{x(1+x)}
    	=\frac{R^2(1 - x)}{1 + x}
    	\]
    	To finish off, we have $AJ = AH = 1$, so 
    	\[
    	\frac{AQ}{AJ} = \frac{1 - x}{ 1 + x}\div 1
    	=\frac{R^2(1 - x)}{1 + x}\div R^2
    	=\frac{XZ}{KX}
    	\]
    	and $J, A, Q$ in that order, 
    	and $K, X, Z$ in that order. 
    	It then follows that $JK, AX, QZ$ are concurrent, as advertised. 
    	
    	\item [G8.]
    	Let $AA'BCC'B'$ be a convex cyclic hexagon such that $AC$ is tangent to the incircle of triangle $A'B'C'$, 
    	and $A'C'$ is tangent to the incircle of triangle $ABC$. 
    	Let the lines $AB$ and $A'B'$ meet at $X$ and let the lines $BC$ and $B'C'$ meet at $Y$.
    	
    	Prove that if $XBYB'$ is a convex quadrilateral, then it has an incircle.
    	
    	\textbf{Solution.} 
    	Let $I$ and $I'$ be the incenters of $ABC$ and $A'B'C'$, respectively. 
    	Denote the intersection of $B'I'$ and $BI$ as $W$. 
    	Denote, also, $d(X, \ell)$ as the distance from $X$ to $\ell$. 
    	We now have $d(W, A'B')=d(W, B'C')$ and $d(W, AB)=d(W, BC)$, 
    	so it suffices to show that these four quantities are indeed equal. 
    	
    	Now, let $AC$ and $A'C'$ intersect at $Z$. 
    	By the problem condition, $Z$ lies inside the circumcircle, and $I, Z, I'$ are collinear and this line bisects the angle formed by lines $AC$ and $A'C'$. 
    	Let $BI$ and $B'I'$ intersect the circumcircle at $D$ and $D'$, respectively. 
    	The tangents at $D$ and $D'$ to the circumcircle are parallel to $AD$ and $A'D'$, respectively, 
    	so it $T$ is such that $TD$ and $TD'$ are tangent to the circumcircle, 
    	then $TD=TD'$ and therefore $DD'\parallel II'$. 
    	Thus we have 
    	\[
    	\frac{d(W, A'B')}{d(W, AB)}
    	=\frac{WB'\cdot\sin\angle A'B'D'}{WB\cdot\sin\angle ABD}
    	=\frac{WB'\cdot A'D'}{WB\cdot AD}
    	=\frac{WB'\cdot D'I'}{WB\cdot DI}
    	\]
    	where we used $AD=CD=DI$, and the chord length is proportional to the sine of angle subtended on circumference. 
    	
    	To continue, from $DD'\parallel II'$ we have $\frac{D'I'}{DI} = \frac{WD'}{WD}$ so 
    	$\frac{WB'\cdot D'I'}{WB\cdot DI}=\frac{WB'\cdot WD'}{WB\cdot WD}=1$, 
    	the last equality from the negative power of point of $W$ to the circumcircle, as claimed. 
    	
    \end{enumerate}
    
    \section*{Number Theory}
    \begin{enumerate}
    	\item [N1.]
    	Determine the smallest number that has three distinct positive divisors whose sum is 2022. (Note: The total number of positive divisors is allowed to be larger than 3.)
    	
    	\textbf{Answer.} $1344$, realized by its divisors $1344, 672, 6$. 
    	
    	\textbf{Solution}. Let our target number be $n$, then we have $n(\frac 1a, \frac 1b, \frac 1c)=2022$ for distinct positive integers $a < b < c$ dividing $n$. 
    	Observe that in our example above $a=1, b=2, c=224$, and we want to maximize $\frac 1a + \frac 1b + \frac 1c$. 
    	
    	In our example, $\frac 1a + \frac 1b + \frac 1c > 1 + \frac 12 = \frac 32$, which means we can consider only the case where $a = 1$. 
    	Since $\frac 1b + \frac 1c > \frac 12$, 
    	we may consider the case where $b\le 3$. 
    	If $b=3$, we are left with $c=4$ and $c=5$, 
    	giving us $\frac 1a + \frac 1b + \frac 1c = \frac{19}{12}, \frac{23}{15}$, repectively. 
    	Both are impossible given that $19\nmid 2022$ and $23\nmid 2022$. 
    	
    	Hence we restrict ourselves to $a=1, b=2$, meaning 
    	$\frac 1a + \frac 1b + \frac 1c = \frac{3c+2}{2c}$, and $c\ge 3$. 
    	Note that the prime factorization of $2022$ is $2\times 3\times 337$, 
    	and $\gcd(2c, 3c + 2)=\gcd(2c, c + 2) = \gcd(c + 2, 4)$, so $3c + 2\mid 2022 \times 4 = 8\times 3\times 337$. 
    	With $3\nmid 3c+2$, we have a stronger identity $3c+2\mid 8\times 337$. 
    	Since $c\ge 3$, $3c+2 > 8$ and so $3c+2\mid 337$ and by considering modulo 3, 
    	$3c+2\ge 2\times 337$ which means $c\ge 224$, as claimed. 
    	
    	\item [N2.]
    	Find all positive integers $n > 1$ such that
    	\[
    	n!\mid \prod_{p < q\le n: p, q\text{ primes}}
    	(p + q)
    	\]
    	
    	\textbf{Answer.} 
    	
    	\item [N3.]
    	Let $a > 1$ be a positive integer and $d > 1$ be a positive integer coprime to $a$. Let $x_1=1$, and for $k\geq 1$, define
    	$$x_{k+1} = \begin{cases}
    		x_k + d &\text{if } a \text{ does not divide } x_k \\
    		x_k/a & \text{if } a \text{ divides } x_k
    	\end{cases}$$Find, in terms of $a$ and $d$, the greatest positive integer $n$ for which there exists an index $k$ such that $x_k$ is divisible by $a^n$.
    
        \textbf{Answer.} Such $n$ is the minimum positive integer $n$ such that $a^n > d$. 
        
        \textbf{Solution.} 
        We construct a new sequence $y_k$ based on $x_k$ by the following rule: 
        $y_1=x_1=1$, and for each $k\ge 1$, if $y_k=x_j$ for some integer $j$, 
        then $y_{k+1}=x_{j'}$ where $j'$ is the smallest integer such that $j'>j$, 
        and $x_{j'} = x_{j'-1} / a$. 
        An example is where $a=2, d=5$ with 
        \[
        x_k = 1, 6, 3, 8, 4, 2, 1\cdots
        \quad 
        y_k = 1, 3, 4, 2, 1
        \]
        Since $\gcd(a, d)=1$, the multiplicative inverse $a^{-1}\pmod{d}$ is well-defined. 
        
        We show that $y_k$ has the form 
        \[
        1, a_1, a_2, \cdots, a_{\ell}=1
        \]
        and repeat, 
        where $\ell=\text{ord}_d(a)$ the order of $a$ mod $d$, 
        and $a_i$ is such that $a_i\equiv a^{-i}\pmod{d}$ and $1\le a_i < d$. 
        Indeed, given a fixed $a_i$, we wish to show $a_{i+1}$ is in the sequence. 
        Note that by our definition, in the sequence $\{x_k\}$, the number preceeding $a_{i + 1}$ is $aa_{i+1}$, 
        and also $aa_{i+1} \equiv a_i\pmod{d}$. 
        Since $a_i, a_{i+1} < d$, this means $a_i \le aa_{i+1} < ad$. 
        This means there is no number strictly in between $a_i$ and $aa_{i+1}$ in the form $a_i + kd$ that satisfies $a\mid a_i + kd$ 
        (exactly one $k\in \{0, 1, 2, \cdots, a - 1\}$ satisfies $a\mid a_i + kd$). 
        This shows $a_{i+1}$ is indeed in the sequence $\{y_k\}$. 
        
        Now, since all the numbers in $y_k$ are less than $d$, 
        $\{x_k\}$ cannot contain any number that is at least $ad$, 
        so the desired $n$ is at most $1 + \log_a d$. 
        On the other hand, the last few elements in $y_k$, in reversed order, 
        are $1, a, \cdots, a^{n-1}$ where $n$ is $1 + \lfloor \log_a d\rfloor$. 
        Thus in $\{x_k\}$, the term preceeding $a^{n-1}$ is $a^n$, 
        as we desire. 
    	
    	\item [N4.]
    	(IMO 5)
    	Find all triples $(a,b,p)$ of positive integers with $p$ prime and$$a^p=b!+p$$
    	
    	\textbf{Answer.} $(2, 2, 2)$ and $(3, 4, 3)$. 
    	
    	\textbf{Solution.} The pairs above work (giving the answers $2+2=2$ and $24+3=27$, respectively). 
    	It remains to show that these are the only answers. 
    	
    	We first show that $a=p$ is necessary, which we consider the following two cases. 
    	If $p\nmid a$, then $b < p$. This means $\gcd(b!, p)=1$ and so the smallest prime divisor of $a$ is greater than $b$, 
    	and in particular, $a > b$ 
    	(the fact that RHS $\ge p$ means $a > 1$). 
    	But since $b < p$, $b! + p < b^b + p < b^p + p< (b + 1)^p \le a^p$, contradiction 
    	(we use the identity that $(b+1)^p - b^p\ge 2^p - 1\ge p$ for all $p\ge 2$). 
    	
    	Now suppose that $p\mid a$, then RHS is divisible by $p^p$. 
    	This means $b!$ cannot be divisible by $p^2$, 
        and hence $p\le b < 2p$. 
        Again, $\gcd(p, b!) = p$, so the smallest prime divisor of $a$ is $p$, and therefore 
        either $a=p$ or $a\ge p^2$. 
        In the second case we have 
        \[
        p^{2p} \le  a^p = b! + p \le (2p - 1)! + p
        \]
        On the other hand, by AM-GM inequality we have 
        \[
        (2p - 1)! = (1\cdot (2p - 1))\cdot (2\cdots (2p-2))\cdots  p\le p^{2p-1}
        \]
        so $(2p - 1)! + p \le p^{2p-1} + p < p^{2p}$, contradiction. 
        
        Thus our task now reduces to finding $(b, p)$ such that $b! = p^p - p$. 
        The case $p=2, 3$ has been done, so we will focus on all primes $p\ge 5$. 
        Denote $\nu_2(n)$ as the highest power of 2 dividing a positive integer $n$. 
        On one hand, with $b\ge p$ we have 
        \[
        \nu_2(b!)\ge \nu_2(p!) = \sum_{i=k}^{\infty}\left\lfloor\frac{p}{2^k}\right\rfloor \ge \frac{p + 1}{2}
        \]
        where the last inequality holds for all $p\ge 5$. 
        On the other hand, let $p - 1 = c\cdot 2^k$ where $c$ is odd. Then given that $p$ is odd, 
        \[
        \nu_2(p^p-p)
        =\nu_2(p^{p-1} - 1)
        =\nu_2(p^{2^k} - 1)
        =\nu_2(p - 1) + \sum_{i=0}^{k - 1}\nu_2(p^{2^i} + 1)
        =k + \sum_{i=0}^{k - 1}\nu_2(p^{2^i} + 1)
        \]
        Where we note that $p^{p-1} - 1=(p^{2^k} - 1)(p^{(c-1)2^k} + \cdots + p^{2^k}+1)$ and the second factor of RHS is odd. 
        If $p \equiv 3\pmod{4}$, then the expression above reduces to $1 + \nu_2(p + 1)$; 
        if $p\equiv 1\pmod{4}$, then $p^{2^i} + 1\equiv 2\pmod{4}$ for all $i\ge 0$, 
        so $k + \sum_{i=0}^{k - 1}\nu_2(p^{2^i} + 1) = k + k = 2k = 2\nu_2(p - 1)$. 
        Using also that $\nu_2(n)\le \log_2(n)$, we have 
        \[
        \frac{p + 1}{2} \le \nu_2(b!) = 1 + \nu_2(p + 1) \le 1 + \log_2(p + 1)
        \]
        for the first case; 
        equality holds for $p=7$ but $\log_2(n)$ grows at a rate smaller than $\frac 12$ for all $n\ge \frac{2}{\log 2}$, 
        so we have $p\le 7$ in this case. 
        For the second case we similarly have 
        \[
        \frac{p+1}{2}\le 2\log_2(p - 1)
        \]
        We may check that the largest prime $p=4k+1$ that this identity holds is $p=13$ 
        (again, the RHS grows at rate smaller than $\frac 12$ whenever $p - 1\ge \frac{4}{\log 2}$). 
        
        Thus we only need to manually check $p=5, 7, 13$ (11 is eliminated under the $4k+3$ case). 
        We have $5^5-5=3120$ lies strictly between $6!$ and $7!$, 
        and $7^7-7=7(7-1)(7+1)(7^4+7^2+1)$ but with $b<2p=14$ here the largest prime divisor of $7^4+7^2+1$ will have to be $\le 13$, 
        which is impossible because it has a divisor 817 which has no prime divisor $\le 13$. 
        Finally, for $p=13$, a closer look on $\nu_2$ gives 
        $\nu_2(b!)\ge \nu_2(13!)\ge \lfloor 13/2\rfloor+\lfloor 13/4\rfloor+\lfloor 13/8\rfloor = 6 + 3 + 1 = 10$, 
        while $2\nu_2(13-1)=2\cdot 2 = 4$, again showing it's impossible here. 
        Thus our solution is complete. 
        
    \end{enumerate}
\end{document}