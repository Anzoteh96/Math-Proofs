\documentclass[11pt,a4paper]{article}
\usepackage{amsmath, amssymb, fullpage, mathrsfs, bm, pgf, tikz}
\usepackage{mathrsfs}
\usetikzlibrary{arrows}
\setlength{\textheight}{10in}
%\setlength{\topmargin}{0in}
\setlength{\topmargin}{-0.5in}
\setlength{\parskip}{0.1in}
\setlength{\parindent}{0in}

\begin{document}
\newcommand{\la}{\leftarrow}
\newcommand{\lra}{\leftrightarrow}
\newcommand{\bbN}{{\mathbb N}}
\newcommand{\bbZ}{{\mathbb Z}}
\newcommand{\bbQ}{{\mathbb Q}}
\newcommand{\bbR}{{\mathbb R}}
\newcommand{\bbC}{{\mathbb C}}
\newcommand{\bbH}{{\mathbb H}}
\newcommand{\dfeq}{\stackrel{\mathrm{def}}{=}}
\newcommand{\ra}{\rightarrow}
\newcommand{\Span}{\mathrm{span}}
\newcommand{\scrP}{\mathscr{P}}
\newcommand{\rank}{\mathrm{rank}}
\newcommand{\nullity}{\mathrm{nullity}}
\newcommand{\Col}{\mathrm{Col}}
\newcommand{\Row}{\mathrm{Row}}
\newcommand{\tr}{\mathrm{tr}}
\newcommand{\ol}{\overline}
\newcommand{\norm}[1]{||#1||}
\newcommand{\doubleline}[1]{\underline{\underline{#1}}}
\newcommand{\elemop}[1]{\stackrel{#1}{\longrightarrow}}
\newcommand{\Ind}{\mathrm{Ind}}
\newcommand{\Res}{\mathrm{Res}}
\newcommand{\End}{\mathrm{End}}
\newcommand{\cl}{\mathrm{cl}}
\newcommand{\code}[1]{\texttt{#1}}
\newcommand\tab[1][0.5cm]{\hspace*{#1}}
\newcommand{\<}{\langle}
\renewcommand{\>}{\rangle}
\newcommand{\qubits}[1]{|{#1}\rangle}
\newcommand{\powset}{\mathcal{P}}
\newcommand{\dsum}{\displaystyle\sum}
\newcommand{\dprod}{\displaystyle\prod}

\section*{Algebra}
\begin{enumerate}
	%%%A1%%%
	\item [\textbf{A1}]
	For every positive integer $N$, determine the smallest real number $b_{N}$ such that, for all real $x$,
	\[
	\sqrt[N]{\frac{x^{2 N}+1}{2}} \leqslant b_{N}(x-1)^{2}+x .
	\]
	
	\textbf{Answer.} $b_n = \frac{N}{2}$. 
	
	\textbf{Solution.} 
	First, we have LHS $> 0$, so RHS $> 0$ must hold, too. 
	This immediately gives $b_N(x^2-2x+1)+x> 0$ for all $x\ge 0$, 
	so $(2b_N-1)^2-4b_N^2<0$, or $(2-\frac{1}{b_N})$, so $b_N>\frac 14$. 
	Conversely when $b_N>\frac 14$ this constraint is guaranteed. 
	
	Now with both sides positives, we can freely raise power on both sides. 
	Raising both sides by power of $N$ and subtracting $x^N$, we get the following inequality: 
	\[
	\frac 12 (x-1)^2 (x^{N-1}+\cdots + x+1)^2
	=
	\frac{1}{2}(x^N-1)^2\le (b_N(x-1)^2+x)^N-x^N
	\]\[
	=b_N(x-1)^2\left(\sum_{k=0}^{N-1}(b_N(x-1)^2+x)^kx^{N-1-k}\right)
	\]
	So we need 
	\[
	(x^{N-1}+\cdots + x+1)^2
	\le 2b_N\left(\sum_{k=0}^{N-1}(b_N(x-1)^2+x)^kx^{N-1-k}\right)
	\]
	whenever $x\neq 1$. 
	
	We first see that by taking $x=1$, 
	\[
	(x^{N-1}+\cdots + x+1)^2=N^2
	\quad 
	\sum_{k=0}^{N-1}(b_N(x-1)^2+x)^kx^{N-1-k}
	=N
	\]
	so if $b_N<\frac{N}{2}$ there exists some $x$ in the neighbourhood of 1 such that 
	\[
	(x^{N-1}+\cdots + x+1)^2
	> 2b_N\left(\sum_{k=0}^{N-1}(b_N(x-1)^2+x)^kx^{N-1-k}\right)
	\]
	which isn't allowed. Therefore $b_n\ge \frac{N}{2}$ is necessary. 
	
	Now we show this $b_N$ works. 
	We'll proceed by induction: 
	base case $N=1$ just gives $(x^{N-1}+\cdots + x+1)^2=1$, 
	and $\sum_{k=0}^{N-1}(b_N(x-1)^2+x)^kx^{N-1-k}=1$, 
	so equality holds on both sides. 
	Now suppose that  for some $N\ge 2$, 
	$(b_k(x-1)^2+x)^k\ge \frac 12(x^{2k}+1)$ for all $k<N$. 
	We have $b_N>b_k$, so 
	\[
	\sum_{k=0}^{N-1}(b_N(x-1)^2+x)^kx^{N-1-k}
	\ge \sum_{k=0}^{N-1}x^{N-1-k}\left(\frac 12(x^{2k}+1)\right)
	=\frac 12 \sum_{k=0}^{N-1}\left(x^{N-k-1}+x^{N+k+1}\right)
	\]
	\[
	=\frac 12 (x^{N-1}+\cdots + x+1)(x^{N-1}+1)
	\]
	To proceed further, 
	we first consider the $x$ nonnegative case: 
	this way, $x^{N-1}+1$ majorizes $x^{k}+x^{N-1-k}$, 
	so Muirhead's inequality means that 
	\[
	\frac N2(x^{N-1}+1)\ge x^{N-1}+\cdots + x + 1
	\]
	which establishes our conclusion. 
	Now when $x<0$: 
	\begin{itemize}
		\item When $x\le -\frac{1}{2}$, $(x-1)^2\ge (-x-1)^2$ and therefore 
		\[
		\sqrt[N]{\frac{x^{2 N}+1}{2}} 
		=\sqrt[N]{\frac{(-x)^{2 N}+1}{2}} \leq b_{N}(-x-1)^{2}+x 
		\le b_{N}(x-1)^{2}+x 
		\]
		
		\item When $x<0$ and $|x|<\frac 12$, the LHS $\sqrt[N]{\frac{x^{2 N}+1}{2}} \le 1$. 
		On the other hand, $b_N\ge 1$ so $b_N(x-1)^2+x\ge (x-1)^2+x=(x-\frac 12)^2+\frac 34\ge 1$ whenever $x<0$. 
	\end{itemize}
	
	%%%A2%%%
	\item [\textbf{A2}] Let $\mathcal{A}$ denote the set of all polynomials in three variables $x, y, z$ with integer coefficients. Let $\mathcal{B}$ denote the subset of $\mathcal{A}$ formed by all polynomials which can be expressed as
	\begin{align*}
	(x + y + z)P(x, y, z) + (xy + yz + zx)Q(x, y, z) + xyzR(x, y, z)
	\end{align*}with $P, Q, R \in \mathcal{A}$. Find the smallest non-negative integer $n$ such that $x^i y^j z^k \in \mathcal{B}$ for all non-negative integers $i, j, k$ satisfying $i + j + k \geq n$.
	
	\textbf{Answer.} $n=4$. 
	
	\textbf{Solution.} We first identify the following members of $\mathcal{B}$. 
	Notice also that $\mathcal{B}$ is closed under scalar multiplication so it's legal to subtract and add (in a sense). 
	\begin{itemize}
		\item $x^iy^jz^k$: just take $R(x, y, z)=x^{i-1}y^{j-1}z^{k-1}$
		
		\item $x^ay^b$ with $a\ge 2, b\ge 2$ (with the symmetric sums): consider 
		$Q(x, y, z)=x^{a-1}y^{b-1}$, giving 
		\[
		(xy + yz + zx)Q(x, y, z) = x^ay^b+x^{a-1}y^{b}z+x^{a}y^{b-1}z
		\]
		But then we already had $x^{a-1}y^{b}z$, $x^{a}y^{b-1}z$ both in $\mathcal{B}$. 
		(Similarly so for each component in the symmetric sum of $x^ay^b$)
		
		\item $x^ay$ with $a\ge 3$. 
		Consider $P(x, y, z)=x^{a-1}y$. Then 
		\[
		(x + y + z)P(x, y, z)=x^ay+x^{a-1}y^2+x^{a-1}yz
		\]
		which, with $a\ge 3$, gives both $x^{a-1}y^2$ and $x^{a-1}yz$ in $\mathcal{B}$. 
		
		\item$x^a$, $a\ge 4$. 
		Now let $P=x^{a-1}$, giving 
		\[
		(x + y + z)P(x, y, z)=x^a+x^{a-1}y+x^{a-1}z
		\]
		Since $a\ge 4$, both $x^{a-1}y$ and $x^{a-1}z$ are in $\mathcal{B}$. 
	\end{itemize}
    Hence any $n\ge 4$ would work. 
    
    To show $n\ge 4$ is necessary, we show that we cannot have $x^2y$ in $\mathcal{B}$. 
    Observe that, when limiting in the case $x^iy^jz^k$ with $i+j+k=3$, it suffices to consider $P, Q, R$ homogeneous with degrees $2, 1, 0$, respectively. 
    We also notice that
    \[
    Q=ax+by+cz
    \Rightarrow 
    (xy+yz+zx)Q(x, y, z)=a(x^2y+x^2z)+b(y^2z+y^2x)+c(z^2x+z^2y)+(a+b+c)xyz
    \]
    Meanwhile, having $x^2$ as part of $P$ introduces $x^3$ term that cannot be found anywhere, so the coefficient of $x^2, y^2, z^2$ of $P$ must be 0. 
    Thus let $P=pxy+qyz+rzx$, which gives 
    \[
    p(x^2y+xy^2)+q(y^2z+yz^2)+r(zx^2+z^2x)+(p+q+r)xyz
    \]
    Since $R$ is constant (and cannot produce terms $x^2y$), 
    in order to get only $x^2y$ we need: $r=-a$ (eliminate $x^2z$), 
    $b=-p$ (eliminate $xy^2$), 
    $c=a$ (eliminate $xz^2$), 
    $q=p$ (eliminate $y^2z$),  
    giving us 
    \[
    (xy+yz+zx)Q(x, y, z)+(x+y+z)P(x, y, z)=(a+p)(x^2y+yz^2)
    \]
    but we need $a+p=1$, i.e. contradiction. 
	
	%%%A3%%%
	\item[\textbf{A3}]Suppose that $a,b,c,d$ are positive real numbers satisfying $(a+c)(b+d)=ac+bd$. 
	Find the smallest possible value of
	$$\frac{a}{b}+\frac{b}{c}+\frac{c}{d}+\frac{d}{a}.$$
	
	\textbf{Answer.} 8. 
	
	\textbf{Solution.} By a slew of AM-GM inequality (in the form of $x+y\ge 2\sqrt{xy}$)
	we have 
	\begin{flalign*}
	  \left(\frac{a}{b}+\frac{c}{d}\right)+\left(\frac{b}{c}+\frac{d}{a}\right)
	  &=\ge 2\sqrt{\frac{ac}{bd}}+2\sqrt{\frac{bd}{ac}}
	  \\&=\frac{2(ac+bd)}{\sqrt{abcd}}
	  \\&=\frac{(a+c)(b+d)}{\sqrt{abcd}}
	  \\&=2\left(\sqrt{\frac{ab}{cd}}+\sqrt{\frac{ad}{bc}}+\sqrt{\frac{bc}{ad}}+\sqrt{\frac{cd}{ab}}\right)
	  \\&\ge 2\left(2\sqrt[4]{\frac{abcd}{abcd}}+2\sqrt[4]{\frac{abcd}{abcd}}\right)
	  \\&=8
	\end{flalign*}
	Equality case when $a=c=2+\sqrt{3}$ and $b=d=1$, i.e. the solution to the equation $a^2+b^2=4ab$. 
	
	%%%A4%%%
	\item [\textbf{A4.}] (IMO 2) The real numbers $a, b, c, d$ are such that $a\geq b\geq c\geq d>0$ and $a+b+c+d=1$. Prove that
	\[(a+2b+3c+4d)a^ab^bc^cd^d<1\]
	
	\textbf{Solution.} The weighted AM-GM inequality says that $\sqrt[a+b+c+d]{(a^ab^bc^cd^d)}\le \frac{a^2+b^2+c^2+d^2}{a+b+c+d}$. Thus it suffices to prove that 
	\[(a+2b+3c+4d)(a^2+b^2+c^2+d^2)<1\]
	We have this identity: 
	\[
	(a+2b+3c+4d)(a^2+b^2+c^2+d^2)
	\]\[
	=
	\left(\frac{5}{2}-\frac{1}{2}(3(a-d)+(b-c)\right)\left(\frac{1}{4}+\frac{1}{4}((a-b)^2+(b-c)^2+(c-d)^2+(a-c)^2+(b-d)^2+(a-d)^2)\right)
	\]
	Thus it suffices to show that 
	$$
	\left(5-(3(a-d)+(b-c)\right)\left(1+((a-b)^2+(b-c)^2+(c-d)^2+(a-c)^2+(b-d)^2+(a-d)^2)\right)< 8
	$$
	Using the identity $\displaystyle\sum_{i=1}^n a_i^2\le (\displaystyle\sum_{i=1}^n a_i)^2$ for all $a_i\ge 0$, we have $(a-b)^2+(b-c)^2+(c-d)^2\le (a-d)^2$.
	In addition, with $a\ge b\ge c\ge d$ we have $(a-c)^2+(b-d)^2\le (a-d)^2+(b-c)^2$.
	Therefore substituting $a-d$ with $x$ and $b-c$ with $y$ (with $0\le y\le x$), we are left with proving
	$$
	\left(5-(3x+y)\right)\left(1+(2x^2+x^2+y^2)\right)< 8
	$$
	We now show that $3x^2+y^2\le \frac{(3x+y)^2}{3}$. Expanding this and subtracting like terms from both sides give this as equivalent to $\frac{2y^2}{3}\le 2xy$ but since $y\ge 0$, this is the as $y=0$ or $\frac{y}{3}\le x$. The conclusion immediately follows give $0\le y\le x$.
	Therefore all we need to show is
	$\left(5-(3x+y)\right)\left(1+\frac{(3x+y)^2}{3}\right)< 8$, or, after substituting $3x+y$ with $z$, we are left with
	$(5-z)(1+\frac{z^2}{3})<8$.
	
	Finally $(5-z)(1+\frac{z^2}{3})<8$ iff $z<3$ there's a root at $z=3$, the rest is just quadratic equation with no root. In addition, $x=a-d$ and $y=b-c$, so $3x+y\le 3(a-d)+3(b-c)\le 3(a+b-c-d)<3(a+b+c+d)=3$. Thus $z<1$ and the desired inequality follows.
	
	%%%A5%%%
	\item[\textbf{A5}]
	A magician intends to perform the following trick. She announces a positive integer $n$, along with $2n$ real numbers $x_1 < \dots < x_{2n}$, to the audience. A member of the audience then secretly chooses a polynomial $P(x)$ of degree $n$ with real coefficients, computes the $2n$ values $P(x_1), \dots , P(x_{2n})$, and writes down these $2n$ values on the blackboard in non-decreasing order. After that the magician announces the secret polynomial to the audience. Can the magician find a strategy to perform such a trick?
	
	\textbf{Answer.} No. 
	
	\textbf{Solution.} 
	We claim the following: 
	
	\emph{Lemma.} 
	There exist polynomials $P$ and $Q$ of degree $n$, and values $a_1, \cdots, a_{2n}$, not all equal, such that (indices taken modulo $2n$)
	\[
	P(x_i)=Q(x_{i+1})=a_i, \forall i=1,\cdots, 2n
	\]
	
	Proof: let's treat $a_i$'s as random variables. 
	Then Lagrange interpolation formula says that 
	\[
	P(x)\equiv \sum_{i=1}^{2n} a_i \prod_{j\neq i}\frac{x-x_j}{x_i-x_j}
	\quad 
	Q(x)\equiv \sum_{i=1}^{2n} a_i \prod_{j\neq i+1}\frac{x-x_j}{x_{i+1}-x_j}
	\]
	(again, indices taken modulo $2n$). 
	
	Now, this gives us a polynomial of degree at most $2n-1$, so we need the coefficients of $x^k$ to be 0 for $k>n$, for both $P$ and $Q$ (as well as having the coefficient at $x^n$ to be nonzero). 
	Notice each product is actually a polynomial: for each $i$, there exist $b_{i, 0}, \cdots, b_{i, 2n-1}$ such that 
	\[
	\prod_{j\neq i}\frac{x-x_j}{x_i-x_j}\equiv \sum_{j=0}^{2n-1}b_{ij}x^j
	\]
	so we essentially have the following: 
	\[
	\begin{pmatrix}
	  b_{1, 2n-1} & \cdots b_{2n, 2n-1}\\
	  b_{1, 2n-2} & \cdots b_{2n, 2n-2}\\
	  \vdots \\
	  b_{1, n+1} & \cdots b_{2n, n+1}\\
	  b_{2, 2n-1} & \cdots b_{1, 2n-1}\\
	  b_{2, 2n-2} & \cdots b_{1, 2n-2}\\
	  \vdots \\
	  b_{2, n+1} & \cdots b_{1, n+1}\\
	\end{pmatrix}
	\begin{pmatrix}
	  a_1\\a_2\\ \ldots\\ a_{2n}\\
	\end{pmatrix}
	=0
	\]
	This is an equation of $2n$ variables, with $2n-2$ equations to solve, so the null space has degree at least 2. 
	One guaranteed solution would be when $a_1=\cdots = a_{2n}$ (i.e. the span of $(1, 1, \cdots, 1)$). 
	Thus there's a solution that's not in the span of $(1, \cdots, 1)$, 
	i.e. $a_i$'s not all equal. 
	
	(TODO: ensure that the coeffs at $n$ is nonzero...check $b_{1, n}, ..., b_{2n, n}$ is not a linear combination of these??)
	
	Thus from our examples, $P$ and $Q$ are two possible polynomials to return, and $a_i$ not all the same means $P$ and $Q$ are not equal. 
	
	\item[\textbf{A6}] 
	Find all functions $f : \mathbb{Z}\rightarrow \mathbb{Z}$ satisfying
	\[f^{a^{2} + b^{2}}(a+b) = af(a) +bf(b)\]for all integers $a$ and $b$. 
	
	\textbf{Answer.} $f\equiv 0$ and $f(x) = x + 1$. 
	
	\textbf{Solution.} 
	We make a quick note that these two functions work, where we either have LHS = RHS = 0 or LHS = RHS = $a(a+1)+b(b+1)$. 
	
	We now show that these are the only functions that work. 
	Setting $a = -1, b = 0$ we have $f(-1)=-f(-1)$, so $f(-1)=0$. 
	This means setting $a=b=-1$ gives $f^2(-2)=0$. 
	
	Denote, now, the function $g(a) = af(a)$. 
	Note that setting $b=-a$ gives $f^{2a^2}(0) = g(a) + g(-a)$ and 
	plugging in $(a - 1, -a - 1)$ gives $f^{2a^2+2}(-2)=g(a-1)+g(-a-1)$. 
	But given also that $f^{2}(-2)=0$ we have $f^{2a^2}(0) = f^{2a^2+2}(-2)$ so 
	$g(a) + g(-a) = g(a-1)+g(-a-1)$. 
	This means if we have $g(-a)=g(a-1)$ to start with, then $g(-a-1)=g(a)$. 
	Given that this fact is true for $a=1$ since $g(0)=g(-1)=0$, 
	we have $g(-a)=g(a-1)$ for all $a$ (by proceeding with induction). 
	
	Now, if $b=-a-1$, then $f^{2a(a + 1)}(0) = f^{2a^2+2a+1}(-1)=g(a)+g(-a-1)=2g(a)$. 
	Given that $g(a)=af(a)$, we have $a\mid g(a)$. 
	Consider also the sequence $\{a_n\}_{\ge 0}$ defined by $a_n=f^n(0)$, 
	i.e. $a_m = a_n$ implies $a_{m+1}=a_{n+1}$ and $a_0=0$. 
	If there is $a_m=a_n$ for some $m\neq n$, then $\{a_n\}$ is eventually periodic (hence bounded from both sides), 
	and since $a_{2n(n+1)} = 2g(n)$ and $n\mid a_{2n(n+1)}$, 
	this means $g(n)=0$ for all $n$ sufficiently large. 
	If $p$ is the minimum period of $\{a_n\}$ (i.e. the minimum $p$ such that $f^p(0)=0$), 
	then $p\mid 2n(n + 1)$ for all $n$ sufficiently large. 
	This gives $4\mid p$ by considering $\gcd\{2n(n+1): n\ge n_0\}$ for some integer $n_0$, 
	i.e. $f^4(0)=0$. 
	Since $2n(n+1)$ is divisible by 4 for all $n$, we have $g(a)=0$ for all $a$, 
	so $f(a)=0$ for all $a\neq 0$. 
	Finally, setting $a=b=1$ gives $f^2(2)=0$, and since $f(2)=0$ we have $f(0)=0$ too. 
	This gives $f\equiv 0$. 
	
	Now suppose that $\{a_n\}$ is not periodic. 
	This means that $a_{2n(n+1)}$ is different for all $n$, and so $g(n): n\ge 0$ is also different for each nonnegative $n$. 
	Also, $f(0)\neq 0$. 
	
	We first note that $f(x)\neq f(y)$ if $x\equiv y\pmod{2}$ and $x\neq y$ 
	(we call this bipartite injectivity). 
	Indeed, let $a, b$ be such that $x = a + b$ and $y = a - b$ (i.e. $b\neq 0$), 
	Now, if $f(x)=f(y)$ then $f^{a^2+b^2}(a + b) = f^{a^2 + b^2})(a - b)$, 
	which then gives $g(a) + g(b) = g(a) + g(-b) = g(a) + g(b - 1)$. 
	But since $g(n): n\ ge 0$ are all different, this cannot happen. 
	
	Now going back to the identity $f^2(-2)=0$. 
	Since $f(0)\neq 0$ and $g(n)=0\to n\in \{0, -1\}$, we must have 
	$f(-2)=-1$, and so $g(-2) = g(1) = 2$, i.e. $f(1)=2$. 
	Given that $af(a) = (-1-a)f(1-a)$ and $\gcd(a, a + 1)=1$, we also have $a + 1\mid f(a)$. 
	Note that setting $a=2, b=0$ gives $f^4(-2)=f^2(0)=g(-2)=2$. 
	Given that $f(0) + 1\mid f(f(0)) = 2$ and $f(0)\neq 0$, 
	we have $f(0)\in \{-3, -2, 1\}$. The fact that $f(1)=2$ means $f(-3)\neq 2$ due to bipartite injectivity. 
	If $f(0)=-2$, then the sequence $a_n$ goes as $0\to -2 \to -1 \to 0\to -2\to\cdots$, 
	i.e. $f^n(0)$ is periodic. 
	This is also a contradiction. Hence $f(0)=1$. 
	Finally, setting $a=b=1$ gives $f^2(2)=2g(1)=4$. 
	We have $3\mid f(2), f(2) + 1\mid 4$, and $f(2)\neq 0$, 
	leaving with the possibilities $f(2)=-3, 3$. 
	However, if $f(2)=-3$ then $g(2)=g(-3)=-6$, i.e. $f(-3)=2$, 
	contradicting $f^2(2)=4$. 
	Therefore $f(2)=3$ and $f(3)=4$, 
	and via $g(a)=g(-1-a)$ we have $f(-4)=-3$ and $f(-3)=-2$. 
	Thus we have $f(n)=n + 1$ for $-4\le n\le 3$. 
	
	The next thing is to show that if, for some $a, b$ with $0 < a < b$ and $b< 2a$, 
	then $f^b(0)=b$ and $f(n)=n+1$ for all $-a\le n\le a - 1$ implies that $f(n)=n+1$ for all $-b\le b - 1$. 
	Indeed, let $c=f^{b-1}(0)$, i.e. $f(c)=b$. 
	Then $c + 1\mid b$. 
	By injectivity, either $c < -a$ or $c \ge a$. 
	Since $b< 2a$, the case $c\ge a$ and $c + 1\mid b$ means $c = b - 1$. 
	The case $c < -a$ and $c + 1\mid b$ means $c = -b-1$. 
	In this case, $g(b) = g(-b-1)=-b(b-1)$ so $f(-b-1)=b$. 
	In this case we have $f^n(0)$ alternating between $b$ and $-b-1$ for $n$ sufficiently large, 
	contradicting that $f^n(0)$ cannot be periodic. 
	This we have $c=b-1$ (i.e. $f(b-1)=b$). 
	We can then iterate backwards to get $f(n)=n+1$ for $n=b-2, b-3, \cdots, a$ 
	and the same deal for $-b, -b+1, \cdots, -a+1$. 
	
	We can now complete the solution. Pluging $a=-3, b=0$ gives $f^9(-3)=g(-3)=6$, 
	and since $f^{-3}(3)=0$, we have $f^6(0)=6$. 
	With $f(n)=n + 1$ for $-4\le n\le 3$, 
	this now means $f(n)=n+1$ for $-6\le n\le 5$. 
	Plugging $a=1, b=2$ gives $8=f^5(3) = f^8(0)$, so this equality can then be expanded into 
	$-8\le n\le 7$. 
	Next,  suppose for some $a\ge 3$ this  $f(n)=n+1$ holds for $-a(a-1)\le n\le a(a-1)+1$. 
	Plugging $(-(a+1), 0)$ gives $f^{(a + 1)^2}(-(a + 1))=g(-(a + 1))=-(a+1)f(-(a+1))$. 
	Note that since $a\ge 3$ we have $a(a-1)\ge a + 1$, 
	so $f(-(a+1))=-a$, 
	and therefore we have $f^{a(a + 1)}(0) = a(a + 1)$. 
	For $a = 3$ we may use $f(n)=n+1$ for $-8\le n\le 7$ to finish, since $a(a + 1) = 3$; 
	for $a\ge 4$, 
	note that $a(a+1) < 2a(a - 1)$, so the claim follows through by our lemma too. 
	Thus we have $f(n)=n+1$ for all integers $n$. 
		
	\item[\textbf{A8}] 
	Let $\bbR^+$ be the set of positive real numbers. Determine all functions $f:\bbR^+$ $\rightarrow$ $\bbR^+$ such that for all positive real numbers $x$ and $y:$
	$$f(x+f(xy))+y=f(x)f(y)+1$$
	
	\textbf{Answer.} $f(x) = x + 1$. 
	
	\textbf{Solution.} It's straightforward to verify that this function works (with both sides being $xy + x + y + 2$), 
	so we will show that this is the only such function. 
	
	Setting $x=1$, we get $f(1 + f(y)) + y = f(1)f(y) + 1$, so $f(y_1)=f(y_2)$ implies $y_1=y_2$, i.e. $f$ is injective. 
	We now show that $f$ is monotone increasing. 
	Indeed, for any $x_1, x_2, y  > 0$ we have the following relations: 
	\[
	x_1 + f(x_1y) = x_2 + f(x_2y)
	\to f(x_1 + f(x_1y)) + y = f(x_2 + f(x_2y)) + y
	\]\[
	\to f(x_1)f(y)=f(x_2)f(y)
	\to x_1=x_2
	\]
	the last equality follows from the injectivy of $f$ and that $f(y) > 0$. 
	Suppose there exist positive numbers $z_1 < z_2$ such that $f(z_1) > f(z_2)$. 
	Choose $y = \frac{z_2-z_1}{f(z_1)-f(z_2)}> 0$, 
	$x_1 = \frac{z_1}{y}, x_2 = \frac{z_2}{y}$, 
	then 
	\[
	x_1 + f(x_1y)
	=x_1 + f(z_1)
	=x_2 + f(z_2)
	=x_2 + f(x_2y)
	\]
	(the middle inequality is by our choice of $y$). 
	so $x_1=x_2$, and therefore $z_1=z_2$, contradiction. 
	This establishes the monotonicity of $f$. 
	
	Now we use the identity of monotone function that any discontinuity, if exists, must be jump continuity. 
	In particular, for each $x_0\ge 0$, the function $\lim_{x\to x_0^+}f(x)$ exists. 
	Denote $C = \lim_{x\to 0^+} f(x)$, 
	or equivalently, $\inf_{x > 0} \{f(x)\}$. 
	Denote, also, $D = \lim_{x\to C^+} f(C) =  \inf_{x > C} \{f(x)\}$. 
	We now show that for all $y$ we have 
	\[
	\lim_{x\to 0^+} f(x + f(xy)) = D
	\]
	To see why, for all $x, y > 0$, by the monotonicity of $f$ we have $x + f(xy) > x + C > C$, 
	and therefore $f(x + f(xy)) \ge D$. 
	For the other side, we may take $x > 0$ small enough such that $x, xy$ are both arbitrarily close to 0, 
	and so $x + f(xy)$ arbitrarily close to $C$. 
	This means $f(x + f(xy))$ can be arbitrarily close to $D$, as desired. 
	(A more rigorous argument will follow from the typical $\epsilon-\delta$ arguments, which we omit here). 
	
	Thus fixing $y$, taking limit of $x\to 0^+$ on both sides we have $D + y = Cf(y) + 1$, 
	so $f$ is a linear function. 
	Take $m, c$ such that $f(x) = mx + c$, 
	we get $m(x + mxy + c) + y + c = (mx+c)(my+c)+1$ holds identically for all $(x, y)$. 
	Since $f$ takes positive values only, we have both $m, c$ nonnegative, 
	and hence $m=1, c=1$ is the only solution. 
\end{enumerate}

\section*{Combinatorics}
\begin{enumerate}
	%%%C1%%%
	\item [\textbf{C1.}] Let $n$ be a positive integer. Find the number of permutations $a_1$, $a_2$, $\dots a_n$ of the
	sequence $1$, $2$, $\dots$ , $n$ satisfying
	$$a_1 \le 2a_2\le 3a_3 \le \dots \le na_n$$
	
	\textbf{Answer.} $F_n$, the $n$-th Fibonacci sequence defined as $F_1=1$, $F_2=2$ and $F_n=F_{n-1}+F_{n-2}$ for all $n\ge 3$. 
	
	\textbf{Solution.} We consider the set 
	\[
	S=\{k: 1\le k\le n, \{a_1, \cdots , a_k\}\neq \{1, 2, \cdots, k\}\}
	\]
	Suppose for some $k$, we have $k\in S$, and either $k=1$ or $k-1\not\in S$. This means $a_k\neq k$. Let $\ell>k$ such that $a_{\ell}=k$, which also means $a_{\ell-1}>k$. We have $(\ell-1)a_{\ell-1}\le \ell a_{\ell}=k\ell$, and therefore 
	\[
	a_{\ell-1}\le k(\frac{\ell}{ell-1})=k(1+\frac{1}{\ell-1})\le k(1+\frac{1}{k})=k+1
	\]
	and with $a_{\ell-1}>k$, all the inequality above are equality, so $a_{k+1}=k$ and $a_k=k+1$. This means $k+1\not\in S$. 
	
	Thus $S$ must contain numbers $\le n-1$ such that no two members are consecutive. In addition, for each $k\in S$ we have $a_k=k+1$ and $a_{k+1}=k$. This means the set $S$ uniquely determines such permutations, and it therefore reduces to finding the number of such set $S$, which is the same as the number of subsets of $\{1, 2, \cdots, n-1\}$ such that no two elements are consecutive. 
	
	We now proceed by induction, if $n=1$ then we only have a set $\emptyset$; if $n=2$ we have $\emptyset, \{1\}$. For $n\ge 3$, depending on whether $n-1\in S$, it reduces to the number of sets among $\{1, 2, \cdots , n-2\}$ or $\{1, 2, \cdots , n-3\}$, which by induction hypothesis gives us $F_{n-1}$ and $F_{n-2}$ such sets, respectively. 
	Therefore the number of such sets is $F_{n-2}+F_{n-1}=F_n$. 
	
	%%%C2%%%
	\item[\textbf{C2}] In a regular 100-gon, 41 vertices are colored black and the remaining 59 vertices are colored white. Prove that there exist 24 convex quadrilaterals $Q_{1}, \ldots, Q_{24}$ whose corners are vertices of the 100-gon, so that
	the quadrilaterals $Q_{1}, \ldots, Q_{24}$ are pairwise disjoint, and
	every quadrilateral $Q_{i}$ has three corners of one color and one corner of the other color.
	
	\textbf{Solution.} 
	We'll use the following claim: 
	
	\emph{Lemma.} 
	Consider a $4k$-gon with $b$ black and $w$ white vertices ($b+w=4k$), 
	such that $k<b<3k$ and $b\neq 2k$. 
	Then there exists four consecutive corners with odd number of  each colours. 
	
	Proof: Let $A_1, \cdots, A_{4k}$ be the vertices in that order, and $a_i$ be the number of black vertices among $A_i, A_{i+1}, A_{i+2}, A_{i+3}$. 
	Then $|a_{i+1}-a_i|\le 1$, 
	and the average among $a_i$ is $\frac{b}{k}$, which is a non-integer lying strictly between 1 and 3. 
	It therefore follows that we must be able to choose $i$ and $j$ with $a_i=\lfloor \frac{b}{k}\rfloor$, 
	and $a_j=\lceil \frac{b}{k}\rceil$. 
	This means either $a_i=1, a_j=2$, or $a_i=2, a_j=3$. 
	Thus we can always choose an $i$ with $a_i$ odd in both cases. 
	$\square$
	
	Now in addition, if $b>w$ we take the 4 vertices with 3 being black, and if $b<w$ we take 4 vertices with 1 being black, 
	we see that we have, in each cases: 
	\[
	2k-2 \le b-3 < 3k - 3
	\quad 
	k - 1< b - 1\le 2k - 2
	\]
	so with $b'$ black vertices left we still have $(k-1)< b' < 3(k-1)$. 
	
	Now let's do the algorithm as per the lemma. 
	That is, at each step, 
	if we have $b, w$ black, white vertices with $b+w=4k$ with $k<b<3k$, $b\neq 2k$, 
	then if $b>w$ choose 4 consecutive vertices with 3 blacks, and if $b<w$ choose 4 consecutive vertices with 1 black. 
	If we get to a point where the condition no longer holds then $b=w=2k$, which we do the following: 
	\begin{itemize}
		\item If $k=0$, stop. 
		
		\item If there exists 4 consecutive vertices with odd number of black, choose that and continue as usual. 
		
		\item Otherwise, each 4 consecutive vertices have 2 blacks and 2 whites. 
	\end{itemize}
    The final case is of our interest, so we'll tackle just on that. Notice that by the $|a_{i+1}-a_i|=1$ in the lemma we have: 
    either the whites and blacks alternate, or they come in pairs (i.e. 2 blacks, 2 whites, 2 blacks, 2 whites, etc in that order). 
    Moreover, up to this point, by choosing along the consecutive vertices, we ensure that the quadrilaterals formed are disjoint, and will be disjoint with the ones we form later on the remaining vertices. 
    Thus we only need to worry about those remaining ones. 
    
    Let our vertices be $A_1, \cdots, A_{4k}$. 
    In the first case (alternate) we choose $k-1$ quadrilaterals according to the following (indices modulo $4k$)
    \[
    (i-1, 3(i+1), 3(i+1)+1, 3(i+1)+2), i=1, \cdots, k-1
    \]
    where we can see that each selection is congruent to $(i-1, i+1, i, i+1)$ mod 2 so odd number of them is black. 
    In the second case, assume that $A_1, A_2$ black and $A_3, A_4$ white. Then we do the following: 
    \[(4n-1, 1, 2, 3), 
    (4n - 2i, 4n-1-2i, 2i+2, 2i+3), i=1, 2, \cdots, k-2
    \]
    which we can see that $2i+2$ and $2i+3$ are different colours, and also $4n - 2i, 4n-1-2i$ are of the same colour. 
    Also the construction above guarantees that the $k-1$ vertices are distjoint so we're done. 
    
    Finally, the conclusion follows since $k=25$ and $n=41$ has $25<41<75$. 
	
	%%%C3%%%
	\item [\textbf{C3.}] (IMO 4) There is an integer $n > 1$. There are $n^2$ stations on a slope of a mountain, all at different altitudes. Each of two cable car companies, $A$ and $B$, operates $k$ cable cars; each cable car provides a transfer from one of the stations to a higher one (with no intermediate stops). The $k$ cable cars of $A$ have $k$ different starting points and $k$ different finishing points, and a cable car which starts higher also finishes higher. The same conditions hold for $B$. We say that two stations are linked by a company if one can start from the lower station and reach the higher one by using one or more cars of that company (no other movements between stations are allowed). Determine the smallest positive integer $k$ for which one can guarantee that there are two stations that are linked by both companies.
	
	\textbf{Answer}. $k=n^2-n+1$. 
	
	\textbf{Solution.} We first show why $k=n^2-n$ won't work. Consider the following construction: 
	\begin{itemize}
		\item $A: i\to i+1, \forall i\in \{an+b: 0\le a\le n-1, 1\le b\le n-1\}$.  Pictorially, this gives a segmentation of $[1, n], [n+1, 2n], \cdots , [n^2n-n+1, n^2]$ where each segment is a chain of $1\to 2\to\cdots n$. 
		\item $B: i\to i+n, \forall i: 1\le i\le n^2-n$. 
	\end{itemize}
	Then in $A$, two stations $i$ and $j$ are linked if and only if they are in the same ``segment'' (i.e. $\lceil \frac{i}{n}\rceil=\lceil\frac{j}{n}\rceil$), while in $B$, two stations $i$ and $j$ are linked if and only if $i\equiv j\pmod{n}$. Thus no two stations are linked by both companies. 
	
	Notice that the proof above also means any $k<n^2-n$ won't work: we simply remove a subset of the cable car links from both companies. 
	
	Now we give an example why $n^2-n+1$ works. 
	We define a chain of cable cars $a_0\to a_1\to\cdots a_m$ as follows: 
	\begin{itemize}
		\item There's a cable car from $a_i$ to $a_{i+1}$. 
		\item There's no cable car ending at $a_0$, and no cable car starting at $a_m$ (i.e. this chain is ``maximal''). 
	\end{itemize}
	We see that two stations are linked if and only if they belong to the same chain (also, notice that these chains are disjoint union of the stations since all the starting points are different, and all the ending points are different). 
	
	The case where station $i$ is not part of any cable car (starting or ending) is considered a degenerate chain of length 1 on its own. Given that both $A$ and $B$ has $k=n^2-n+1$ cable car services, there are $n-1$ chains in total ($n-1$ is the number of stations that's not a starting point of any cable car, and also the number of stations that's not a ending point of any cable car). 
	
	The longest chain in $A$ now has length at least
	\[
	\frac{n^2}{n-1}>n+1>n-1
	\]
	and since there are $n-1$ chains in $B$, two of the stations in this longest chain must also in the same chain in $B$, thereby being linked by both companies. 
	
	\textbf{Comment}. To generalize to the case where there are $n$ stations (i.e. not square), the same idea applies: we need to find the largest integer $m$ such that $\frac{n}{m}>m$. In this case $m=\lfloor\sqrt{n-1}\rfloor$ which gives the bound $k=n-\lfloor\sqrt{n-1}\rfloor$. The construction of counterexample for $k-1$ can also be done like above by splitting into segments of either length $\lfloor\sqrt{n}\rfloor$ or $\lceil\sqrt{n}\rceil$. 
\end{enumerate}

\section*{Geometry}
\begin{enumerate}
	%%%G1%%%
	\item [\textbf{G1.}] Let $ABC$ be an isosceles triangle with $BC=CA$, and let $D$ be a point inside side $AB$ such that $AD< DB$. Let $P$ and $Q$ be two points inside sides $BC$ and $CA$, respectively, such that $\angle DPB = \angle DQA = 90^{\circ}$. Let the perpendicular bisector of $PQ$ meet line segment $CQ$ at $E$, and let the circumcircles of triangles $ABC$ and $CPQ$ meet again at point $F$, different from $C$.
	Suppose that $P$, $E$, $F$ are collinear. Prove that $\angle ACB = 90^{\circ}$.
	
	\textbf{Solution.} We now have $QPCF$ an isoceles trapezoid, and with angle chasing we can conclude that triangles $FQP$ and $FAB$ are similar. Triangles $FAQ$ and $FBP$ are also similar. Therefore, 
	\[
	\frac{CQ}{CP}=\frac{FP}{FQ}=\frac{FB}{FA}=\frac{PB}{QA}=\frac{DP}{DQ}
	\]
	Let $\angle QCD=a$ and $\angle PCD=b$. We have $\angle CQD=\angle CPD=90^{\circ}$, so $C, Q, P, D$ lie on a circle. If the circle has diameter $d$, we get 
	\[
	\frac{CQ}{CP}=\frac{d\cos a}{d\cos b}=\frac{\cos a}{\cos b}\qquad
	\frac{DP}{DQ}=\frac{d\sin b}{d\sin a}=\frac{\sin b}{\sin a}
	\]
	so we essentially have $\cos a\sin a=\cos b\sin b$, or $\sin 2a=\sin 2b$. This means, $a=b$ or $a+b=90^{\circ}$. The former is only possible if $CP=CQ$, which entails $AD=DB$ which is impossible here. Therefore $\angle ACB=a+b=90^{\circ}$. 
	%%%G2%%%
	\item [\textbf{G2.}] (IMO 1)
	Consider the convex quadrilateral $ABCD$. The point $P$ is in the interior of $ABCD$. The following ratio equalities hold:
	\[\angle PAD:\angle PBA:\angle DPA=1:2:3=\angle CBP:\angle BAP:\angle BPC\]Prove that the following three lines meet in a point: the internal bisectors of angles $\angle ADP$ and $\angle PCB$ and the perpendicular bisector of segment $AB$.
	
	\textbf{Solution.} The claim is that the lines will meet at the circumcenter $O$ of $ABP$. Here's how: 
	\begin{itemize}
		\item As $OA=OB=OP$, it's already on the perpendicular bisector of $AB$. 
		
		\item Now $\angle PAD+\angle DPA<180^{\circ}$ (they are part of interior angles of the triangle $PAD$), we have $\angle PBA=\frac{\angle PAD+\angle DPA}{2}<90^{\circ}$ based on the angle condition. This gives the following angle condition: 
		\[
		\angle POA=2\angle PBA=\angle PAD+\angle DPA=180^{\circ}-\angle PDA
		\]
		and therefore $DAOP$ is concyclic. As $OP=OA$, it then follows that $DO$ bisects $\angle ADP$. 
		
		\item Similarly, $CO$ bisects $\angle PCB$. 
	\end{itemize}

	\item [\textbf{G3}] Let $ABCD$ be a convex quadrilateral with $\angle ABC>90$, $CDA>90$ and $\angle DAB=\angle BCD$. Denote by $E$ and $F$ the reflections of $A$ in lines $BC$ and $CD$, respectively. Suppose that the segments $AE$ and $AF$ meet the line $BD$ at $K$ and $L$, respectively. Prove that the circumcircles of triangles $BEK$ and $DFL$ are tangent to each other.
	
	\textbf{Solution.} We first claim that the circumcircles of triangles $ALD$ and $AKB$ are tangent to each other. 
	This is the same as showing that $\angle AKB+\angle ALD=\angle BAD$. 
	Indeed, given that $E$ is the reflection of $A$ in $BC$, $AE\perp BC$ and therefore $\angle AKB=90^{\circ}-\angle CDB$ and similarly $\angle ALD=90^{\circ}-\angle CBD$. Therefore, 
	\[
	\angle AKB+\angle ALD=180^{\circ}-\angle CDB-\angle CBD=\angle BCD=\angle BAD
	\]
	as claimed. 
	
	Now, reflect $A$ in $BD$ to get $G$. To avoid cases on whether $K$ lies between $E$ and $A$ we use directed angles here. From $E$ being the reflection of $A$ in $BC$ we get $\angle (AE, AB)=\angle(EB, AE)$ and similarly by definition of $G$ we get $\angle (AE, AB)=\angle(AK, AB)=\angle(BG, GK)$. 
	Therefore, $\angle(BG, GK)=\angle(BE, EA)=\angle(BE, EK)$ so $B, G, K, E$ are concyclic. Similarly, $L, D, F, G$ are concyclic. This means that $G$ is an intersection of circumcircles of $BDK$ and $DFL$. Finally, since circles $BAK$ and $ALD$ are tangent to each other, same goes to circles $BDK$ and $DFL$ as they are simply the reflections of $BAK$ and $ALD$ in line $BD$. 
	
	\item [\textbf{G5}] 
	Let $ABCD$ be a cyclic quadrilateral. 
	Points $K, L, M, N$ are chosen on $AB, BC, CD, DA$ such that $KLMN$ is a rhombus with $KL \parallel AC$ and $LM \parallel BD$. 
	Let $\omega_A, \omega_B, \omega_C, \omega_D$ be the incircles of $\triangle ANK, \triangle BKL, \triangle CLM, \triangle DMN$.
	
	Prove that the common internal tangents to $\omega_A$, and $\omega_C$ and the common internal tangents to $\omega_B$ and $\omega_D$ are concurrent.
	
	\textbf{Solution.} 
	Let $O_A, O_B, O_C, O_D$ be the centers of $\omega_A, \omega_B, \omega_C, \omega_D$. 
	Denote the common internal tangency point of incircles $BAD$ and $BCD$ as $P$, 
	and the common internal tangency point of incircles $ABC$ and $ADC$ as $Q$. 
	Denote, also, the inradius of triangle $XYZ$ as $r(XYZ)$. 
	We see that $P$ lies on $BD$ and $Q$ lies on $AC$. 
	
	Let's first show the following: 
	\[
	\frac{BP}{PD} = \frac{r(BKL)}{r(DMN)}
	\]
	Indeed, we see that triangles $BKL$ and $BAC$ are similar, with similitude, say, $\beta$. 
	Similarly, $AKN$ and $ABD$ are similar, with similitude, say, $\alpha$. 
	Then $\alpha + \beta = \frac{AK+KB}{KB}=1$. 
	It then follows that the similitude of $DMN$ and $DCA$ is also $\beta$, 
	and the similitude of $CLM$ and $CBD$ is $\alpha$. 
	Thus $\frac{r(BKL)}{r(DMN)} = \frac{r(BAC)}{r(DCA)}$, so it suffices to show that 
	$\frac{BP}{PD} = \frac{r(BAC)}{r(DCA)}$. 
	
	Now $P$ is the intersection of $BD$ and $I_AI_C$, where $I_A$ is the incenter of $ABD$ and $I_C$
	the incenter of $CBD$. 
	Then we have (using $|\triangle|$ as the area of a triangle)
	\[
	\frac{BP}{PD}
	=\frac{|BI_AI_C|}{|DI_AI_C|}
	=\frac{BI_A\cdot BI_C\cdot \sin\angle I_ABI_C}
	{DI_A\cdot DI_C\cdot \sin\angle I_ADI_C}
	=\frac{\sin\angle I_ADB\cdot \sin\angle I_CDB\cdot \sin\angle I_ABI_C}
	{\sin\angle I_ABD\cdot \sin\angle I_CBD\cdot \sin\angle I_ADI_C}
	\]
	(the last equality by sine rule). 
	Meanwhile, denoting $I_B$ the incenter of $ABC$ and $I_D$ the incenter of $ADC$ we have 
	\[
	\frac{r(BAC)}{r(DCA)}
	=\frac{|AI_BC|}{|AI_DC|}
	=\frac{AI_B\cdot I_BC\cdot \sin\angle AI_BC}{AI_D\cdot I_DC\cdot \sin\angle AI_DC}
	=\frac{(AC\frac{\sin\angle I_BCA}{\sin\angle AI_BC})\cdot (AC\frac{\sin\angle I_BAC}{\sin\angle AI_BC})\cdot \sin\angle AI_BC}
	{(AC\frac{\sin\angle I_DCA}{\sin\angle AI_DC})\cdot (AC\frac{\sin\angle I_DAC}{\sin\angle AI_DC})\cdot \sin\angle AI_DC}
	\]
	\[
	=\frac{\sin\angle I_BCA\sin\angle I_BAC\sin\angle AI_DC}{\sin\angle I_DCA\sin\angle I_DAC\sin\angle AI_BC}
	\]
	We also have 
	\[
	\angle I_ADB
	=\frac 12\angle ADB=\frac 12\angle ACB=\frac 12\angle I_BCA
	\qquad 
	\angle I_CDB
	=\frac 12\angle CDB=\frac 12\angle CAB=\frac 12\angle I_BAC
	\]
	\[
	\angle I_ABD
	=\frac 12\angle ABD=\frac 12\angle ACD=\frac 12\angle I_DCA
	\qquad 
	\angle I_CBD
	=\frac 12\angle CBD=\frac 12\angle CAD=\frac 12\angle I_DAC
	\]
	and finally, 
	\[
	\angle AI_DC = 90^{\circ} + \frac{\angle ADC}{2}
	=180^{\circ} - \frac{\angle ABC}{2}
	=180^{\circ} - \angle I_ABI_C
	\]\[
	\angle AI_BC = 90^{\circ} + \frac{\angle ABC}{2}
	=180^{\circ} - \frac{\angle ADC}{2}
	=180^{\circ} - \angle I_ADI_C
	\]
	so the two ratios are indeed equal, as desired. 
	In a similar way we also have 
	\[
	\frac{AQ}{QC}
	=\frac{r(AKN)}{r(CLM)}
	\]
	Next, let $R$ be on $PQ$ such that $\frac{PR}{RQ} = \frac{\beta}{\alpha}$, 
	i.e. $R = \alpha P + \beta Q$ . 
	We show that $R$ is the desired point of concurrency. 
	To see why, let $\frac{AQ}{QC}=\frac{r(AKN)}{r(CLM)}=\frac{r(ABD)}{r(CBD)}=\frac{\gamma}{\mu}$ for some $\gamma, \mu>0$ and $\gamma + \mu = 1$. 
	Then we have $Q = \mu C + \gamma A$, 
	$P = \mu I_C + \gamma I_A$. 
	We also have $A, O_A, I_A$ collinear with $O_A = \beta A + \alpha I_A$ and 
	$C, O_C, I_C$ collinear with $O_C = \beta C + \alpha I_C$. 
	Thus the common internal tangents of $\omega_A, \omega_C$ intersect at the following point: 
	\[
	\mu O_C + \gamma O_A
	=\mu (\beta C + \alpha I_C) + \gamma (\beta A + \alpha I_A)
	=\beta(\mu C + \gamma A) + \alpha(\mu I_C + \gamma I_A)
	=\beta Q + \alpha P
	\]
	so this point is indeed $R$. 
	In a similar way we can also show that the common internal tangents of $\omega_B, \omega_D$ will intersect at $R$. 
	
	\item [\textbf{G6}] Let $ABC$ be a triangle with $AB < AC$, incenter $I$, and $A$ excenter $I_{A}$. The incircle meets $BC$ at $D$. Define $E = AD\cap BI_{A}$, $F = AD\cap CI_{A}$. Show that the circumcircle of $\triangle AID$ and $\triangle I_{A}EF$ are tangent to each other. 
	
	\textbf{Solution.} We first claim that the circles $AID$ and $I_AEF$ have an intersection on the circle $IBC$. 
	
	Now, $II_A$ is the diameter of circle $IBC$. Let $G$ be another intersection of circles $IBC$ and $I_AEF$. 
	Then triangles $GEF$ and $GBC$ are similar, also $GBE$ and $GCF$ are similar. 
	This means we have 
	\[
	\frac{GB}{GC}=\frac{BE}{CF}
	\]
	and moreover by Menelaus' theorem on line $DEF$ and triangle $BI_AC$, 
	\[
	\frac{EB}{EI_A}\cdot \frac{FI_A}{FC}\cdot \frac{DC}{DB}=-1
	\]
	(notice that $F$ lies outside segment $I_AC$ so $\frac{FI_A}{FC}$ is assumed negative here), 
	and also considering the triangle $BCG$ and line $GD$, 
	\[
	\frac{BD}{DC} = \frac{BG}{GC}\cdot \frac{\sin\angle BGD}{\sin\angle CGD}
	=\frac{BE}{CF}\frac{\sin\angle BGD}{\sin\angle CGD}
	\]
	so we ended up with 
	\[
	-1=\frac{EB}{EI_A}\cdot \frac{FI_A}{FC}\cdot \frac{DC}{DB}
	=\frac{EB}{EI_A}\cdot \frac{FI_A}{FC}\cdot \frac{CF}{BE}\frac{\sin\angle CGD}{\sin\angle BGD}
	=\frac{FI_A}{EI_A}\frac{\sin\angle CGD}{\sin\angle BGD}
	\]
	so ignoring sign and recognizing that $E$ is on segment $BI_A$ but $F$ outside segment $CI_A$, we have 
	\[
	\frac{\sin\angle I_AEF}{\sin\angle I_AFE}=\frac{FI_A}{EI_A} = \frac{\sin\angle BGD}{\sin\angle CGD}
	\]
	This means $\angle I_AEF = \angle BGD$ and $\angle I_AFE=\angle CGD$. 
	
	Now we claim that $G$ also lies on the circumcircle of triangle $AID$. By angle chasing, 
	\[
	\angle DAI
	=\angle FAI_A 
	=\angle II_AC - \angle AFI_A
	=\angle IGC - \angle DGC
	=\angle IGD
	\]
	as desired. 
	
	It remains to show that $G$ is in fact the tangency point of the circles $AID$ and $I_AEF$. Given $AIDG$ on one circle and $GEFI_A$ on one circle, what we need to show now is 
	\[
	\angle IGE = \angle GAI + \angle GI_AE
	\]
	The last quantity $\angle GI_AE$ is the same as $\angle GCB$; we also have 
	\[
	\angle GAI = \angle GAD+\angle IAD = \angle GID+\angle IGD 
	\]
	and finally 
	\[
	\angle IGE=\angle IGD+\angle DGE = \angle IGD+\angle BGE-\angle BGD 
	\]
	so now we need to prove that 
	\[
	\angle BGE=\angle GID + \angle GCB + \angle BGD
	\]
	We notice also that $\angle BGD=\angle I_AEF = \angle BED$ so $BEDG$ is concyclic. This means $\angle BGE=\angle CDF$. 
	Also, $\angle CDF-\angle BED=\angle DBE=\angle CBI_A$ so it suffices to show that $\angle CBI_A=\angle GID+\angle GCB$. 
	
	Let $ID$ intersect the circle $I_ABC$ again at $H$, then $\angle GID=\angle GIH=\angle GI_AH$ and $\angle GCB=\angle GI_AB$ so$\angle GID+\angle GCB=HI_AB=HCB$. We also have $BC_IAH$ an isoceles trapezoid so $\angle HCB=\angle I_ABC$, as desired. 
	
	%%%G7%%%
	\item [\textbf{G7}] Let $P$ be a point on the circumcircle of acute triangle $ABC$. Let $D,E,F$ be the reflections of $P$ in the $A$-midline, $B$-midline, and $C$-midline. Let $\omega$ be the circumcircle of the triangle formed by the perpendicular bisectors of $AD, BE, CF$.
	
	Show that the circumcircles of $\triangle ADP, \triangle BEP, \triangle CFP,$ and $\omega$ share a common point.
	
	\textbf{Solution.} We first show that the first three circles are coaxial. If $K$, $L$ and $M$ are the perpendiculars from $A$, $B$, $C$ to lines $BC$, $CA$, $AB$, respectively, then $K$ is the reflection of $A$ in the corresponding midline. Then $AKPD$ is an isoceles trapezoid, and therefore cyclic. 
	Similarly, $L$ lies on circumcicle of $BEP$ and $M$ lies on $CFP$. 
	
	(TODO...next time)
	
	%%%G8%%%
	\item [\textbf{G8}]
	Let $ABC$ be a triangle with incenter $I$ and circumcircle $\Gamma$. Circles $\omega_{B}$ passing through $B$ and $\omega_{C}$ passing through $C$ are tangent at $I$. Let $\omega_{B}$ meet minor arc $AB$ of $\Gamma$ at $P$ and $AB$ at $M\neq B$, and let $\omega_{C}$ meet minor arc $AC$ of $\Gamma$ at $Q$ and $AC$ at $N\neq C$. Rays $PM$ and $QN$ meet at $X$. Let $Y$ be a point such that $YB$ is tangent to $\omega_{B}$ and $YC$ is tangent to $\omega_{C}$.
	
	Show that $A,X,Y$ are collinear.
	
	\textbf{Solution.} 
	We break things down into several steps. 
	
	\definecolor{uuuuuu}{rgb}{0.26666666666666666,0.26666666666666666,0.26666666666666666}
	\definecolor{ududff}{rgb}{0.30196078431372547,0.30196078431372547,1}
	\begin{tikzpicture}[line cap=round,line join=round,>=triangle 45,x=1cm,y=1cm, scale = 1.2]
	\clip(-8.081105750848476,-4.620366998260985) rectangle (5.748374835436061,5.081302842687707);
	\draw [line width=2pt] (-1.9337545832442526,-0.021538726707125897) circle (3.575922427388784cm);
	\draw [line width=2pt] (-3.2613071793416193,0.8670881056349335) circle (1.431122720356785cm);
	\draw [line width=2pt] (-4.722615209762418,0.686896741137611) circle (1.472375661172013cm);
	\draw [line width=2pt] (-0.5339504250973116,1.203393747421322) circle (2.748013164018579cm);
	\draw [shift={(-4.197951117000675,4.539082876707524)},line width=2pt,dash pattern=on 1pt off 1pt]  plot[domain=4.47725100235184:5.869160831656574,variable=\t]({1*2.4478780793390276*cos(\t r)+0*2.4478780793390276*sin(\t r)},{0*2.4478780793390276*cos(\t r)+1*2.4478780793390276*sin(\t r)});
	\draw [line width=2pt] (-4.450297071196617,2.6863096461174916) circle (0.6161241708028135cm);
	\draw [line width=2pt] (-2.9807913308908303,3.430986288160808) circle (1.0312951510447146cm);
	\draw [line width=2pt] (-3.9007114291388914,2.9648138936460553)-- (-5.465072826256629,-0.5845775115790762);
	\draw [line width=2pt] (-5.465072826256629,-0.5845775115790762)-- (1.6074181960067557,-0.5188480411119442);
	\draw [line width=2pt] (1.6074181960067557,-0.5188480411119442)-- (-3.9007114291388914,2.9648138936460553);
	\draw [line width=2pt,dash pattern=on 1pt off 1pt] (-3.9007114291388914,2.9648138936460553)-- (-0.6886528380964085,-3.373693682990008);
	\draw [line width=2pt] (-3.397536542203804,1.9718750219500902)-- (-4.768250821765113,2.158565004887587);
	\draw [line width=2pt] (-3.397536542203804,1.9718750219500902)-- (-1.9568961986239144,3.5543088193076247);
	\draw [line width=2pt] (-5.465072826256629,-0.5845775115790762)-- (-0.6886528380964085,-3.373693682990008);
	\draw [line width=2pt] (-0.6886528380964085,-3.373693682990008)-- (1.6074181960067557,-0.5188480411119442);
	\draw [line width=2pt] (-4.28506317677037,2.0927553065830304)-- (-3.0077342819980286,2.4000431442610206);
	\begin{scriptsize}
	\draw [fill=ududff] (-3.9007114291388914,2.9648138936460553) circle (2.5pt);
	\draw[color=ududff] (-4.111045734633714,3.3263259812152812) node {$A$};
	\draw [fill=ududff] (-5.465072826256629,-0.5845775115790762) circle (2.5pt);
	\draw[color=ududff] (-5.359905673509218,-0.3019407885704084) node {$B$};
	\draw [fill=ududff] (1.6074181960067557,-0.5188480411119442) circle (2.5pt);
	\draw[color=ududff] (1.7125853487541667,-0.23621131810327634) node {$C$};
	\draw [fill=uuuuuu] (-3.2613071793416193,0.8670881056349335) circle (2pt);
	\draw[color=uuuuuu] (-3.296000300841279,0.6577094802497196) node {$I$};
	\draw [fill=uuuuuu] (-4.768250821765113,2.158565004887587) circle (2pt);
	\draw[color=uuuuuu] (-5.070696003453838,2.3403839242083) node {$P$};
	\draw [fill=uuuuuu] (-4.28506317677037,2.0927553065830304) circle (2pt);
	\draw[color=uuuuuu] (-4.176775205100845,2.353529818301727) node {$M$};
	\draw [fill=uuuuuu] (-3.0077342819980286,2.4000431442610206) circle (2pt);
	\draw[color=uuuuuu] (-2.9016234780384877,2.6558853824505344) node {$N$};
	\draw [fill=uuuuuu] (-1.9568961986239144,3.5543088193076247) circle (2pt);
	\draw[color=uuuuuu] (-1.7053471155366884,3.484076710336398) node {$Q$};
	\draw [fill=uuuuuu] (-0.6886528380964085,-3.373693682990008) circle (2pt);
	\draw[color=uuuuuu] (-0.5879461175954473,-3.115162124563661) node {$Y$};
	\draw [fill=uuuuuu] (-3.397536542203804,1.9718750219500902) circle (2pt);
	\draw[color=uuuuuu] (-3.4011674535886898,1.8671317368449498) node {$X$};
	\end{scriptsize}
	\end{tikzpicture}
	
	\emph{Step 1.} $Y$ lies on $\Gamma$. 
	Indeed, from the tangency of $\omega_B$ and $\omega_C$ at $I$ we got 
	$\angle BIC = \angle BMI + \angle CNI$, 
	and by definition of $Y$ (tangent to circles)
	we have $\angle IBY = \angle BMI, \angle ICY = \angle INC$. 
	Therefore, 
	\[
	\angle ABY + \angle ACY 
	= \angle ABI + \angle ACI + \angle BMI + \angle CNI 
	= \angle ABI + \angle ACI  + \angle BIC = 180^{\circ}
	\]
	
	\emph{Step 2.} 
	$I$ is the excenter opposite $A$ or triangle $AMN$. 
	
	Proof: by the tangency of $\omega$'s we have $\angle MCN = \angle ABI + \angle ACI = 90^{\circ}-\frac{BAC}{2}$. 
	Now let the tangent to incircle of $ABC$ from $M$ to be line $AB$ and $MN'$ with $N'$ on $AC$, 
	then by angle chasing we can get 
	\[\angle MCN'=180^{\circ}-\frac 12 (\angle BMN'+\angle CN'M)
	=180^{\circ}-\frac 12 (180^{\circ}+\angle BAC)
	=90^{\circ}-\frac{BAC}{2}=\angle MCN'
	\]
	so indeed, $N=N'$ and therefore the incircle is tangent to $MN$, establishing this step. 
	
	\emph{Step 3.} 
	$P, M, N, Q$ concyclic. 
	Here we'll just angle chase: 
	\[
	\angle PMN = \angle PMA + \angle AMN
	=180^{\circ} - \angle PMB + (180^{\circ} - 2\angle BMI)
	\]
	while 
	\[
	\angle PQN
	=\angle AQN - \angle AQP
	=\angle AQC - \angle NQC - \angle ABP 
	= (180^{\circ} - \angle ABC) - (180^{\circ} - \angle NIC)-\angle ABP
	\]
	Finally, $\angle MIN + \angle BIC = 90^{\circ}+\frac{\angle BAC}{2} + 90 ^{\circ}-\frac{\angle BAC}{2}= 180^{\circ}$ so $\angle BIM + \angle NIC = 180^{\circ}$. 
	Thus we're left with showing 
	\[
	\angle PMB + 2\angle BMI + \angle ABC + \angle MIB + \angle ABP = 360^{\circ}
	\]
	Indeed, 
	\begin{flalign*}
	  \angle PMB + 2\angle BMI + \angle ABC + \angle MIB + \angle ABP
	  \\= \angle PMB + \angle BPI + \angle ABC + (180^{\circ} - \frac{\angle ABC}{2})+ \angle ABP
	  \\= (180^{\circ} - \angle MPI)+ \angle ABC + (180^{\circ} - \frac{\angle ABC}{2})
	  \\=360^{\circ} + \angle ABC - \frac{\angle ABC}{2}-\frac{\angle ABC}{2}
	  \\=360^{\circ}
	\end{flalign*}
	as claimed. 
	
	\textbf{Step 4.} $APM$ and $AQN$ have circumcircles tangent at $A$, and common tangent line $AY$. 
	
	Proof: We have 
	\[\angle APM
	=\angle APB - \angle BPM 
	= (180^{\circ} - \angle ACB) - (180^{\circ} - \angle MIB)
	=\angle MIB - \angle ACB
	\]
	and also 
	\[
	\angle BAY = 180^{\circ}-\angle ABY - \angle AYB
	= \angle MIB - \angle ACB = \angle APM
	\]
	so $AY$ is tangent to the circle $APM$ at $A$. 
	Similarly $AY$ is tangent to circle $AQN$ at $A$, thus finishing the proof. 
	
	Now we can complete the solution. $AY$ is the radical axis of $APM$ and $AQN$, 
	and since $P, M, N, Q$ cyclic, their intersection $X$ has $XM\cdot XP=XN\cdot XQ$. 
	Therefore $X$ is on the radical axis of these two circles (i.e. done). 
	
	
	%%%G9%%%
	\item [\textbf{G9}] (IMO 6) Prove that there exists a positive constant $c$ such that the following statement is true:
	Consider an integer $n > 1$, and a set $\mathcal S$ of $n$ points in the plane such that the distance between any two different points in $\mathcal S$ is at least 1. It follows that there is a line $\ell$ separating $\mathcal S$ such that the distance from any point of $\mathcal S$ to $\ell$ is at least $cn^{-1/3}$.
	
	(A line $\ell$ separates a set of points $\mathcal{S}$ if some segment joining two points in $\mathcal S$ crosses $\ell$.)
	
	Note. Weaker results with $cn^{-1/3}$ replaced by $cn^{-\alpha}$ may be awarded points depending on the value of the constant $\alpha > 1/3$.
	
	\textbf{(Mini-)Solution.} I hereby attach my proof for $\alpha=\frac 12$ (which is worth 1 point). 
	
	Let $D$ be the diameter of $\mathcal{S}$. That is, the maximum distance between any two points in $\mathcal{S}$. W.l.o.g. let the diameter to be $(0, 0)$ and $(0, D)$. Then any point in $\mathcal{S}$ must have $x$-coordinate in $[-D, D]$ and $y$-coordinate in $[0, D]$. 
	Consider, now, drawing $\frac{1}{\sqrt{2}}\times \frac{1}{\sqrt{2}}$ boxes in the space $[-D, D]\times [0, D]$ space. Given that the distance between any two points in the box is at most 1, each point in $\mathcal{S}$ must be in different boxes. Thus there are at most $\lceil \sqrt{2}D\rceil\times \lceil 2\sqrt{2}D\rceil $ boxes. Omitting the ceiling functions for now (as they will be insignificant as $D$ and $n$ grow), we have at most $4D^2$ boxes. With $n$ points in $S$, we have at least $n$ boxes. Thus $D\ge\frac{\sqrt{n}}{2}$. 
	
	Now, as above we assumed that the two points have $y$-coordinates 0 and $D$, respectively. Putting a line parallel to the $x$-axis and with $y$-intercept between 0 and $D$ will separate the two points (hence separating $\mathcal{S}$). Sorting the points by $y$-axis, it then follows that some consecutive points have gap at least $\frac{D}{n}$. Thus placing the line passing through the midpoint of these two points (or rather, perpendicular bisector) will ensure that any point has distance at least $\frac{D}{2n}$ from this line. With $D\ge \frac{\sqrt{n}}{2}$, we have distance from any points of $\mathcal{S}$ to $\ell$ at least $\frac 14 n^{-\frac 12}$ (well maybe a bit lower to account for the ceiling function above). 
\end{enumerate}

\section*{Number Theory}
\begin{enumerate}
	%%%N1%%%
	\item[\textbf{N1}]
	Given a positive integer $k$ show that there exists a prime $p$ such that one can choose distinct integers $a_1,a_2\cdots, a_{k+3} \in \{1, 2, \cdots ,p-1\}$ such that $p$ divides $a_ia_{i+1}a_{i+2}a_{i+3}-i$ for all $i= 1, 2, \cdots, k$.
	
	\textbf{Solution.} 
	Let $p$ be a random variable (for now), and allowing multiplicative inverse, we get the following requirement: 
	\[
	\frac{a_{i+4}}{a_i}=\frac{a_{i+1}a_{i+2}a_{i+3}a_{i+4}}{a_ia_{i+1}a_{i+2}a_{i+3}} = \frac{i+1}{i}
	\]
	Now consider $q, r, s>k$ prime numbers, with $a_1=q, a_2=r, a_3=\frac{1}{q}, a_4 = \frac{1}{r}$. 
	Then $a_1a_2a_3a_4=1$ and by obeying $\frac{a_{i+4}}{a_i}=\frac{i+1}{i}$, we get $a_{i}a_{i+1}a_{i+2}a_{i+3}=i$. 
	
	Next we see that $a_i\neq a_j$ for all $i, j$: 
	if $i\not\equiv j\pmod{4}$, then $a_{4x+1}$ and $a_{4x+2}$ have factors $q$ and $r$ on their numerators respectively (but not both); while $a_{4x+3}$ and $a_{4x+4}$ have those on their denominators respectively (but not both). 
	This easily rules out the case where $i\not\equiv j\pmod{4}$. 
	For $i\equiv j\pmod{4}$ we have $i<j$ implying 
	\[
	\frac{a_j}{a_i}=\dprod_{x=0}^{(j-i)/4-1}\frac{i+4x+1}{i+4x}
	\]
	which is a rational number that's strictly $>1$. 
	So $a_i\neq a_j$ in this case. 
	
	Finally, having established above, all we need to do is to assemble all the rational numbers $\{\frac{a_i}{a_j}\}_{i\neq j}$, and choose $p$ such that: 
	\begin{itemize}
		\item $p$ is relatively prime to the numerators and denominators of all $\frac{a_i}{a_j}$. 
		
		\item $\frac{a_i}{a_j}\not\equiv 1\pmod{p}$. That is, choose $p$ such that $p$ does not divide the numerator of $a_i-a_j$ for all $(i, j)$. 
	\end{itemize}
	
	%%%N2%%%
	\item[\textbf{N2}] For each prime $p$, construct a graph $G_p$ on $\{1,2,\ldots p\}$, where $m\neq n$ are adjacent if and only if $p$ divides $(m^{2} + 1-n)(n^{2} + 1-m)$. Prove that $G_p$ is disconnected for infinitely many $p$. 
	
	\textbf{Solution.} We claim that $G_p$ is disconnected whenever $p\equiv 1\pmod{3}$. Notice that all the edges are given by $(m, m^2+1)$ for $m=0, 1, \cdots , p-1$ since $(m, n)$ are adjacent if and only if $p\mid m^2+1-n$ or $p\mid n^2+1-m$ (thus $(m, n)$ can be written in the form $(m, m^2+1)$ or $(n^2+1, n)$). 
	This means that there are at most $p$ edges in $G_p$. 
	
	Now, if $p\equiv 1\pmod{3}$, then $x^3\equiv 1\pmod{3}$ has 3 distinct solutions (one of them being 1). Therefore we have, for some $x\not\equiv 1$, 
	\[
	p\mid x^3-1=(x-1)(x^2+x+1)
	\]
	so $p\mid x^2+x+1$ here. 
	If $m=x+1$, then $m^2+1=(x+1)^2+2=(x^2+x+1)+x+1\equiv x+1\pmod{p}$ so $(m, m^2+1)$ is actually $(m, m)$ (i.e. a self-loop). 
	In addition, if $p\mid m^2+1-m$, then $(p-m+1)^2+1-(p-m+1)\equiv (m-1)^2+1+(m-1)=m^2+1-m\equiv 0\pmod{p}$. 
	So both $(m, m)$ and $(p-m+1, p-m+1)$ are self-loops. 
	
	Finally, we claim that $m\neq p-m+1$, otherwise $2m+1=p$ and we have 
	\[
	p\mid (\frac{p+1}{2})^2+1-(\frac{p+1}{2})\equiv \frac{3}{4}
	\]
	and therefore $p=3$ here, which is a contradiction. This means that among the $p$ edges just now, 2 of them are self-loops so there are at most $p-2$ actual edges. This means that our graph can no longer be connected for such $p$. Finally, by Dirichlet's theorem, there are infinitely many such $p$ with $p\equiv 1\pmod{3}$. 
	
	%%%N3%%%
	\item [\textbf{N3.}] (IMO 5) A deck of $n > 1$ cards is given. A positive integer is written on each card. The deck has the property that the arithmetic mean of the numbers on each pair of cards is also the geometric mean of the numbers on some collection of one or more cards.
	For which $n$ does it follow that the numbers on the cards are all equal?
	
	\textbf{Answer.} All $n>1$. 
	
	\textbf{Solution.} Call a set $S$ of positive integers ``good'' if the arithmetic mean of every pair of integers is also the geometric pairs of some subset of integers in $S$. Consider $aS=\{ak: k\in S\}$ for some rational $a$ such that $ak\in\bbN$ for all $k\in S$. The arithmetic and geometric mean of corresponding numbers are scaled by $a$, so $S$ is good if and only if $aS$ is good. 
	Therefore, given an original collection $S$, we can divide by its $\gcd$ and assume that $\gcd(S)=\gcd\{k: k\in S\}=1$. 
	
	In particular, some number $k\in S$ has to be odd. If $m\in S$ is even then $\frac{k+m}{2}$ is a half-integer. But $(\frac{k+m}{2})^d$ is never an integer for any integer $d>0$. Thus $\frac{k+m}{2}$ cannot be geometric mean of any subset of $S$. We thus have all numbers odd. 
	
	Suppose also that the numbers are $a_1\ge \cdots a_n$. If $a_1>1$, then there exists an odd prime $p$ dividing $a_1$ but since $\gcd(a_1, \cdots , a_n)=1$, there's some $i$ with $p\nmid a_i$. Thus we can choose $m$ that is the minimal index with $p\nmid a_m$ (i.e. $a_m$ is the biggest number among them not divisible by $p$. Then $\frac{a_1+a_m}{2}$ must be an integer not divisible by $p$, and since $a_1\neq a_m$, $a_1>a_m$ and so $\frac{a_1+a_m}{2}>a_m$. But since $a_1, \cdots , a_{m-1}$ are divisible by $p$, the numbers that form geometric mean of $\frac{a_1+a_m}{2}$ must be taken from $\{a_m, \cdots , a_n\}$ which are at most $a_m$. This gives a contradiction. Thus $a_1=1$ and all numbers are equal.
	
	%%%N4%%%
	\item[\textbf{N4}] For any odd prime $p$ and any integer $n,$ let $d_p (n) \in \{ 0,1, \dots, p-1 \}$ denote the remainder when $n$ is divided by $p.$ We say that $(a_0, a_1, a_2, \dots)$ is a p-sequence, if $a_0$ is a positive integer coprime to $p,$ and $a_{n+1} =a_n + d_p (a_n)$ for $n \geqslant 0.$
	\begin{itemize}
		\item[(a)] Do there exist infinitely many primes $p$ for which there exist $p$-sequences $(a_0, a_1, a_2, \dots)$ and $(b_0, b_1, b_2, \dots)$ such that $a_n >b_n$ for infinitely many $n,$ and $b_n > a_n$ for infinitely many $n?$
		
		\item[(b)] Do there exist infinitely many primes $p$ for which there exist $p$-sequences $(a_0, a_1, a_2, \dots)$ and $(b_0, b_1, b_2, \dots)$ such that $a_0 <b_0,$ but $a_n >b_n$ for all $n \geqslant 1?$
	\end{itemize}
	
	\textbf{Answer.} Yes to both. 
	
	\textbf{Solution.} We see that $a_{n+1}=a_n+d_p(a_n)\equiv 2a_n$ so similarly $d_p(a_{n+1})\equiv 2 d_p(a_n)$. 
	
	Let $c$ be the order of 2 modulo $p$. Then 
	\[(d_p(a_0), \cdots, d_p(a_{c-1}))=(d_p(a_0), 2d_p(a_0), \cdots, 2^{c-1}d_p(a_0))\pmod{p}
	\]
	and similarly, 
	\[(d_p(a_n), \cdots, d_p(a_{n+c-1}))=(d_p(a_0), 2d_p(a_n), \cdots, 2^{c-1}d_p(a_n))\pmod{p}
	\]
	for every $n$. 
	Moreover, $d_p(a_c)=d_p(a_0)$. Therefore for each $n$, 
	\[
	a_{n+c}-a_n = \dsum_{i=0}^{c-1} d_p(a_{n+i}) = \dsum_{i=0}^{c-1} d_p(a_i)
	\]
	is the same for all $n$. 
	
	Let's also investigate the properties of $\{d_p(a_0), \cdots, a_p(a_{c-1})\}$. Treating the nonzero residues modulo $p$ as a group $G$, the subgroup $H=\{2^k: k=0, \cdots, c-1\}$ paritions $H$ into cosets, where $\{d_p(a_0), \cdots, a_p(a_{c-1})\}$ is one of them. Thus $\dsum_{i=0}^{c-1} d_p(a_i)$ can also be viewed as a coset sum for the sequence $(a_n)$. Now we can proceed with the following. 
	
	\begin{enumerate}
		\item We just need to make sure that: 
		\begin{itemize}
			\item $(a_n)$ and $(b_n)$ have the same coset sum. 
			
			\item There exists $i$ and $j$ with $a_i<b_i$ and $a_j>b_j$, which, with the condition before, $a_{i+cn}<b_{i+cn}$ and $a_{j+cn}>b_{j+cn}$ for all $n$. 
		\end{itemize}
		We will in fact choose $a_0$ and $b_0$ such that they come from the same coset, and must have the same coset sum. 
		
		Consider any $p\equiv 1\pmod{4}$, and let $a_0=2^k$ for some $k$ such that $a_0<p<2a_0$. 
		With $p\ge 5$, $a_n\ge 4$ so $2a_0\ge p+3$ (since $4\mid p+3$ and $p\equiv 1\pmod{4}$). 
		Let $b_0=p+1$, and then $b_1=p+2$. We now have 
		\[
		a_0<p<p+1=b_0\qquad b_1=p+2<p+3\le 2a_0=a_0+d_p(a_0)=a_1
		\]
		and notice that both $a_0$ and $b_0$ belong to the coset defined as 
		\[
		\{2^k: 0\le k\le c-1\}
		\]
		completing the construction. Finally, we note that there are infinitely many such primes $p$. 
		
		\item We claim that all we need is having $a_0$ and $b_0$ such that the coset sum of $a_0$ is greater than that of $b_0$. Suppose that this condition is fulfilled. By adding $p$ to $b_0$ as many times as we want, we may assume $a_0<b_0$. 
		
		Consider the number 
		\[
		g = \min\{a_n-b_n: n=0, 1, \cdots, c-1\}
		\]
		and let coset sum of $(a_n)$ and $(b_n)$ to be $x$ and $y$ respectively with $x>y$. Then for all $k$ and $i$, 
		\[
		a_{i+ck}-b_{i+ck} = (a_i+kx)-(b_i+ky)=k(x-y)+(a_i-b_i)\ge g+k(x-y)
		\]
		so with $x-y>0$, for all $k$ sufficiently large and $i\ge 0$ we have $a_{i+ck}-b_{i+ck}$. 
		This also means that for all $n$ sufficiently large we have $a_n>b_n$. This means that there's a maximal index $m$ such that $a_m<b_m$ (but $a_n>b_n$ for all $n>m$). 
		All we need to do now is to shift both sequences by $m$ spots to the left and replace $a_0$ with $a_m$ and $b_0$ with $b_m$, which gives $a_0<b_0$ but $a_n>b_n$ for all $n>0$. 
		
		It remains to show that we can indeed find infinitely many $p$ with at least two cosets of different sum. We claim that all $p\equiv 7\pmod{8}$ fulfills this property. 
		First, we quote a well-known identity that if $1\equiv 1, 7\pmod{8}$, then 2 is a quadratic residue modulo 8. This means the number of cosets is even. 
		
		On the other hand, the sum of all cosets is 
		\[
		1=2+\cdots + (p-1)=\frac{p(p-1)}{2}
		\]
		and since $p\equiv 3\pmod{4}$ here, $\frac{p-1}{2}$ is odd and so is $\frac{p(p-1)}{2}$. Therefore with even number of cosets, the cosets cannot all have the same sum. 
		
		\item [\textbf{N5}] Determine all functions $f$ defined on the set of all positive integers and taking non-negative integer values, satisfying the three conditions:
		\begin{enumerate}
			\item $f(n)\neq 0$ for at least one $n$;
			\item $f(xy)=f(x)+f(y)$ for every positive integers $x$ and $y$;
			\item there are infinitely many positive integers $n$ such that $f(k)=f(n-k)$ for all $k<n$. 
		\end{enumerate}
		
		\textbf{Answer.} $f(n)=c\cdot v_p(n)$ where $p$ is a prime number, $c$ any nonnegative constant and $v_p(n)$ the greatest power of $p$ dividing $n$. 
		
		\textbf{Solution.} We first notice that if we prime factorize 
		\[
		n=p_1^{\alpha_1}\cdots p_k^{\alpha_k}
		\]
		then 
		\[
		f(n)=\dsum_{i=1}^k \alpha_k f(p_k)
		\]
		i.e. the function $f$ is determined solely by the values of the primes. 
		
		We need at least one prime $p$ to have $f(p)>0$. On the other hand, we shall also prove that such $p$ is unique. 
		
		Consider now, any such $n$ such that $f(k)=f(n-k)$ for all $k<n$. Then for each $k$ we have: 
		\[
		f(1)+f(2)+\cdots f(k) = f(k!)\qquad f(n-1)+\cdots + f(n-k) = f((n-1)\cdots (n-k))
		\]
		and with $k! \dbinom{n-1}{k} = (n-1)\cdots (n-k)$, we have $f(\dbinom{n-1}{k})=0$. 
		In other words, if $p$ is one such prime with $f(p)\neq 0$, then $p\not\mid \dbinom{n-1}{k}$ for all $k\le n-1$. 
		By Lucas' theorem, this happens only when for all $k\le n-1$, the $p$-ary representation of $k$ has all digits at most the corresponding digits of $n-1$ when written in base $p$. 
		On the other hand, if a non-leading digit (say $p^u$) of $n-1$ is not maximal (say, $v < p-1$) then pick $k=(v+1)\cdot p^u$ and the $p^u$ digit of $k$ is greater than that of $n-1$ but $k<n-1$ since $p^u$ is not the leading digit of $n-1$. 
		
		Therefore $n-1$ must have $p-1$ in all its digits base $p$ except for the leading digit, which means there exists $1\le m\le p-1$ and $a$ such that $n=m\cdot p^a$. 
		
		Now suppose that there are two primes $p$ and $q$ with $f(p)$ and $f(q)$ both nonzero. Then there exists $m$, $r$ with $1\le m\le p-1$ and $1\le r\le q-1$, and $a, b\ge 0$ with $n=m\cdot p^a=r\cdot q^b$. 
		Suppose also that $p\le q$. 
		Then $m<q$ and it would follow that $q\not\mid m\dot p^a$ so $b=0$, or $n=r < q$. This means that the condition (iii) would only hold for finitely many $n$ since they are all less than $q$, which is a contradiction. 
		
		Therefore only one prime $p$ can have $f(p)\neq 0$. For this $p$, by using prime factorization of an integer $n$ would yield $f(n)=f(p)v_p(n)$. To show that (iii) can be fulfilled, consider $n=p^{\alpha}$ for any $\alpha\ge 0$. Then for each $k<n$, if $b<\alpha$ is the highest power of $p$ dividing $k$ then $k=p^b\ell$ for some $\ell$ with $\ell$ not divisible by $p$. 
		Then $n-k=p^{b}(p^{\alpha-b}-\ell)$ so $v_p(n-k)=b$, too, as claimed. 
		
		%%%N6%%%
		\item[\textbf{N6}]
		For a positive integer $n$, let $d(n)$ be the number of positive divisors of $n$, and let $\varphi(n)$ be the number of positive integers not exceeding $n$ which are coprime to $n$. Does there exist a constant $C$ such that
		
		$$ \frac {\varphi ( d(n))}{d(\varphi(n))}\le C$$for all $n\ge 1$
		
		\textbf{Answer.} No. 
		
		\textbf{Solution.} 
		We first prove the following claim: 
		
		\emph{Lemma.} 
		Let $k$ be an arbitrary positive integer. Then there exists an $N$ such that if we take the first $N$ primes $p_1, \cdots, p_N$, 
		at least $k$ of those $p_i$'s is such that for all $j=1, \cdots, N$, $p_i\nmid p_j-1$. 
		
		Proof: In this collection, if $p_i$ is odd and $p_i\ge \frac{p_N}{2}$ then 
		$p_j-1$ is either 1 or has prime divisors that are 2 and others at most $\frac{p_j-1}{2}>p_i$. 
		Thus it suffices to show that $N$ can be chosen such that at least $k$ of those are at least $\frac{p_N}{2}$. 
		By prime number theorem there exists function $f(N)\in o(1)$ such that the number of primes up to $N$ is $(1+f(N))\frac{N}{\log N}$. 
		Thus, when considering $N$ and $2N$, big enough with $|f(N)|<0.01$ and $|f(2N)|<0.01$ for each of those we have the number of primes between $N$ and $2N$ as 
		\[
		(1+f(2N))\frac{2N}{\log (2N)} - (1+f(N))\frac{N}{\log N}
		\ge 0.99\frac{2N}{\log 2 + \log (N)} - 1.01\frac{N}{\log N}
		\]
		\[
		\ge\frac{0.95N}{\log N}
		\]
		for $N$ sufficiently big (to cover the addition $\log 2$ in denominator). 
		Thus we can choose $N$ that satisfies the aforementioned constraint, and such that $ \frac{0.95N}{\log N}\ge k$ (here $\log$ is natural log). 
		
		Turning back to the problem, 
		consider the collection $p_1, \cdots, p_{\ell}, q_1, \cdots, q_k$ as the first ${\ell}+k$ primes (in some order)
		such that $q_i-1$ and $p_j-1$ are not divisible by $q_{x}-1$ for all $i, j, x$. 
		Consider sequences $a_1, \cdots, a_{\ell}$ and the number 
		\[
		n = \prod_{i=1}^{\ell}p_i^{2^{a_i}-1}\prod_{j=1}^{k}q_j
		\]
		Then 
		\[
		d(n) = 2^k\prod_{i=1}^{\ell}2^{a_i}
		\quad 
		\varphi(n) = \prod_{i=1}^{\ell}(p_i-1)p_i^{2^{a_i}-2}\prod_{j=1}^{k}(q_j-1)
		\]
		By our construction, $p_i-1$ and $q_j-1$ are divisible by none of $q_x$'s, and since we're taking the first $n+k$ primes, $p_i-1$ and $q_j-1$ must have prime divisors that are only in the $p_x$'s. 
		Thus let's name 
		\[
		\prod_{i=1}^{\ell}(p_i-1)
		\prod_{j=1}^{k}(q_j-1)
		=\prod_{i=1}^{\ell}p_i^{b_i}
		\]
		and we have 
		\[\prod_{i=1}^{\ell}(p_i-1)p_i^{2^{a_i}-2}\prod_{j=1}^{k}(q_j-1)
		=\prod_{i=1}^{\ell}p_i^{2^{a_i}+b_i-2}
		\]
		
		Next, since $d(n)$ is power of 2, $\varphi(d(n))=\frac 12 d(n)$ and so 
		\[\varphi(d(n)) = \varphi(2^k\prod_{i=1}^{\ell}2^{a_i})=2^{k-1+\sum_{i=1}^{k}a_i}
		\]\[
		d(\varphi(n))
		=d\left(\prod_{i=1}^{\ell}(p_i-1)p_i^{2^{a_i}-2}\prod_{j=1}^{k}(q_j-1)\right)
		=\prod_{i=1}^{\ell} (2^{a_i}+b_i-1)
		\]
		so our ratio is now $2^{k-1}\prod_{i=1}^{\ell}\frac{2^{a_i}}{2^{a_i}+b_i-1}$. 
		By choosing $a_i$ large, each $\frac{2^{a_i}}{2^{a_i}+b_i-1}$ can stay close to 1 (while $\ell$ is finite), 
		so we can make the ratio to be arbitrarily close to $2^{k-1}$. 
	\end{enumerate}
\end{enumerate}

\end{document}