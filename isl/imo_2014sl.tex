\documentclass[11pt,a4paper]{article}
\usepackage{amsmath, amssymb, fullpage, mathrsfs, bm, pgf, tikz}
\usepackage{mathrsfs}
\usetikzlibrary{arrows}
\setlength{\textheight}{10in}
%\setlength{\topmargin}{0in}
\setlength{\topmargin}{-0.5in}
\setlength{\parskip}{0.1in}
\setlength{\parindent}{0in}

\begin{document}
\newcommand{\la}{\leftarrow}
\newcommand{\lra}{\leftrightarrow}
\newcommand{\bbN}{\mathbb{N}}
\newcommand{\bbZ}{\mathbb{Z}}
\newcommand{\dsum}{\displaystyle\sum}
\newcommand{\dprod}{\displaystyle\prod}

\section*{Algebra}
\begin{enumerate}
	\item [\textbf{A1}] (IMO 1) Let $a_0 < a_1 < a_2 \ldots$ be an infinite sequence of positive integers. Prove that there exists a unique integer $n\geq 1$ such that
	\[a_n < \frac{a_0+a_1+a_2+\cdots+a_n}{n} \leq a_{n+1}.\]
	
	\textbf{Solution.} The condition that $a_n<\dfrac{a_0+a_1+a_2+\cdots+a_n}{n}$ means that $a_0+a_1+\cdots + a_{n-1}-(n-1)a_n>0$ and similarly $\dfrac{a_0+a_1+a_2+\cdots+a_n}{n} \leq a_{n+1}$ means $a_0+\cdots + a_n-na_{n+1}\le 0$. 
	Let $f(n)=a_0+\cdots + a_{n-1}-(n-1)a_n$. We are to prove that there exists a unique $n$ such that $f(n)>0$ but $f(n+1)\le 0$. 
	
	We first claim that $f(n)$ is a decreasing function. To see this, we have 
	\[f(n)-f(n+1) = (a_0+a_1+\cdots + a_{n-1}-(n-1)a_n) - (a_0+\cdots + a_n-na_{n+1})
	=n(a_{n+1}-a_n)
	\]
	and since $\{a_n\}$ is a strictly increasing sequence, we have $n(a_{n+1}-a_n)>0$, and therefore $f(n)-f(n+1)>0$ for all $n$. 
	
	The next observation is that $f(1)=a_0-(1-1)a_1=a_0>0$. Since $f(n)$ decreases by at least 1 as $n$ increases(because $\{a_n\}$) are sequence of integers), we see that there exists an $n_0$ with $f(n_0)\le 0$ (in particular we can choose $f(a_0+1)$). 
	Since $f(n)\le 0$ for all $n\ge n_0$, the set $\{n: f(n)>0\}$ is finite but nonzero since $f(1)>0$. Let $n_1$ be the largest element in this set, and we see that $f(n_1)>0$ but $f(n_1+1)\le 0$, $f(n)>0$ for $n\le n_1$ and $f(n)\le 0$ for all $n>n_1$. Thus this $n_1$ is the only $n$ fitting the criteria. 
	
	\item[\textbf{A2}] Define the function $f:(0,1)\to (0,1)$ by \[\displaystyle f(x) = \left\{ \begin{array}{lr} x+\frac 12 & \text{if}\ \  x < \frac 12\\ x^2 & \text{if}\ \  x \ge \frac 12 \end{array} \right.\] Let $a$ and $b$ be two real numbers such that $0 < a < b < 1$. We define the sequences $a_n$ and $b_n$ by $a_0 = a, b_0 = b$, and $a_n = f( a_{n -1})$, $b_n = f (b_{n -1} )$ for $n > 0$. Show that there exists a positive integer $n$ such that \[(a_n - a_{n-1})(b_n-b_{n-1})<0.\]
	
	\textbf{Solution.} We notice that $f(x)>x$ if $x<\frac 12$, and if $x\ge 12$ then since $x<1$ as well, we have $f(x)<x$. Hence, it suffices to find $n$ such that either $a_n<\frac 12\le b_n$, or $b_n<\frac 12 \le a_n$. 
	
	Suppose otherwise, then for each $n$, we either have $a_n, b_n < 12$, or $a_n, b_n\ge\frac 12$. 
	Within each range $(0, \frac 12)$ and $[\frac 12, 1)$, we also have $f$ monotonous: $f(x)<f(y)$ iff $x<y$. 
	Hence from $a<b$ we also have $a_n<b_n$ for all $n$. We now consider the difference $b_n-a_n$ in terms of $b_{n-1}-a_{n-1}$: 
	\begin{itemize}
		\item If $a_{n-1}, b_{n-1}<\frac 12$ then $b_n-a_n=b_{n-1}-a_{n-1}$. 
		
		\item Otherwise we have $b_n-a_n = (b_{n-1}-a_{n-1})(b_{n-1}+a_{n-1})$, and by $\frac 12\le a_{n-1}<b_{n-1}$ we also have $b_{n-1}-a_{n-1}>1$ so $b_n-a_n>b_{n-1}-a_{n-1}$. 
	\end{itemize}
	Regardless, we always have $b_n-a_n \ge b_{n-1} - a_{n-1}$. 
	
	Now let $c=b-a=b_0-a_0$. We have $b_{n}-a_{n}\ge c$ as always. If $n_1< n_2, \cdots < n_k<n$ are indices such that $a_{n_i}\ge\frac 12$ then $b_{n_i+1}-a_{n_i+1} = (b_{n_i} - a_{n_i})(b_{n_i} + a_{n_i})\ge (1+c)(b_{n_i}-a_{n_i})$. Thus we now have $a_n-b_n\ge (1+c)^{k}c$. 
	Since $0< a_n<b_n<1$, $(1+c)^kc<1$, i.e. $k<-\log_{1+c}(c)$. 
	This also means that regardless of $n$, the number of $m<n$ with $a_m, b_m,\ge \frac 12$ is bounded by $-\log_{1+c}(c)$ which does not depend on $n$. Hence such $m$'s are also finite. 
	This means that there exists an $N$ such that for all $n\ge N$ we have $a_n<\frac 12$. This is impossible, since $a_n<\frac 12$ means that $a_{n+1}=a_n+\frac 12>\frac 12$. The desired contradiction proves our statement. 
	
	%%%A3%%%
	\item[\textbf{A3}] For a sequence $x_1,x_2,\ldots,x_n$ of real numbers, we define its $\textit{price}$ as \[\max_{1\le i\le n}|x_1+\cdots +x_i|.\] Given $n$ real numbers, Dave and George want to arrange them into a sequence with a low price. Diligent Dave checks all possible ways and finds the minimum possible price $D$. Greedy George, on the other hand, chooses $x_1$ such that $|x_1 |$ is as small as possible; among the remaining numbers, he chooses $x_2$ such that $|x_1 + x_2 |$ is as small as possible, and so on. Thus, in the $i$-th step he chooses $x_i$ among the remaining numbers so as to minimise the value of $|x_1 + x_2 + \cdots  x_i |$. In each step, if several numbers provide the same value, George chooses one at random. Finally he gets a sequence with price $G$.
	
	Find the least possible constant $c$ such that for every positive integer $n$, for every collection of $n$ real numbers, and for every possible sequence that George might obtain, the resulting values satisfy the inequality $G\le cD$.
	
	
	\textbf{Answer.} $c=2$. 
	
	\textbf{Solution.} First, we show that $c\ge 2$. For each $n$, consider the set $n$ copies of $1, -1$ each, along with $2n$ and $-(2n-1)$. George will do $1, -1, \cdots , 1, -1$ as part of the Greedy strategy before choosing $-(2n-1)$ and $2n$, making $G=-(2n-1)$. Dave will do $-1, -1, \cdots , -1, 2n, -(2n-1), 1, 1, \cdots , 1$ that will lead to the cumulative value as 
	\[-1, -2, \cdots , -n, n, -(n-1), -(n-2), \cdots , -1, 0, 1
	\]
	so $D=n$. Thus $c\ge \frac{2n-1}{n}=2-\frac{1}{n}$ for each $n$, and therefore $c=2$. 
	
	We now show that $G\le 2D$ always holds. Consider the arrangement by George $x_1, \cdots , x_n$ and let $k$ satisfies $|x_1+\cdots + x_k| = G$. W.l.o.g. we may assume that $x_1+\cdots + x_k = G>0$. 
	\begin{itemize}
		\item If there exists $x_i$ with $|x_i|\ge G$, then for any number $x$ we have $\max\{|x|, |x_i|\}\ge \frac{|x_i|}{2}$ (with equality when $x=-\frac{x_i}{2}$) so no matter how Dave rearranges, we have $D\ge \frac{G}{2}$. 
		
		\item Otherwise, $|x_i| < G$ for all $i$. We claim that $x_1+\cdots + x_n=G$ which then follows that $D=G$. We need to show that $x_{k+1}= \cdots = x_{n} = 0$. Consider, now, the sum $G'=x_1+\cdots +x+{k-1}$ and so $G=G'+x_k$. From $|x_k|<G$ we have $G'>0$. By the maximality of $G$ we also have $G'\le G$. 
		
		If $x_i<0$ for some $i>k+1$, from $|x_i|<G$ it means that replacing $x_k$ with $x_{i}$ gives 
		\[
		-G = 0 - G < G'-G < x_1+\cdots + x_{k-1}+ x_i < G' < G
		\]
		therefore $G'+x_i\in (-G, G)$ so replacing $x_k$ with $x_i$ yields a lower value at this point, which is a contradiction to George's algorithm. 
		
		This means $x_k, x_{k+1}, \cdots , x_n$ are all nonnegative. If $x_{k+1}>0$ then $x_{1}+\cdots + x_{k+1}>G$ which again contradicts this maximality, so $x_{k+1}=0$. Iteratively we can show that $x_i=0$ for all $i>k$. 
		
	\end{itemize}
	Summarizing the two cases above yields $D\ge \frac{G}{2}$ must always holds. 
	
	%%%A4%%%
	\item[\textbf{A4}] Determine all functions $f: \mathbb{Z}\to\mathbb{Z}$ satisfying \[f\big(f(m)+n\big)+f(m)=f(n)+f(3m)+2014\] for all integers $m$ and $n$.
	
	\textbf{Answer.} The only such function is $f(n)=2n+1007$. \\
	\textbf{Solution.} We first consider $m=0$, and let $f(0)=c$. So plugging $m=0$ yield $f(n+c)+c=f(n)+c+2014$, i.e. $f(n+c)-f(n)=2014$. By considering $n, n+c, n+2c, \cdots$ we can generalize this statement to $f(n+kc)-f(n)=2014k$ for all $k\ge 0$. 
	Next, consider $m_1=m+c$, and therefore $f(m_1)=f(m)+2014$ and $f(3m_1)=f(3m+3c)=f(3m)+3(2014)$. 
	Therefore we have the following: 
	\[f(f(m_1)+n)+f(m_1)=f(f(m)+n+2014)+f(m)+2014\]
	and
	\[
	f(n)+f(3m_1)+2014=f(n)+f(3m)+3(2014)+2014=f(n)+f(3m)+4(2014)
	\]
	and therefore $f(f(m)+n+2014)+f(m)=f(n)+f(3m)+3(2014)$. Comparing this with the original statement and subtracting both sides, we have $f(f(m)+n+2014)-f(f(m)+n)=2(2014)$. Since $f(m)+n$ attains all values in $\bbZ$ (by varying $n$), we have $f(n+2014)-f(n)=2(2014)$ for all $n$, or more generally (by following the logic above), $f(n+2014k)-f(n)=2k(2014)$. 
	
	We now have $f(n+kc)-f(n)=2014k$ for all $k\ge 0$ and $f(n+2014\ell)-f(n)=2\ell(2014)$ for all $\ell\ge 0$. Let $k=2014$ and $\ell=c$ we have $2014^2=2c(2014)$, forcing $c=1007$ (and therefore $f(0)=1007$). 
	
	We now consider the next set of formulation: $f(f(m)+n)-f(n)=f(3m)-f(m)+2014$. Denote $d=f(3m)-f(m)+2014$. 
	This means, for each $m$, considering the sequence $n, f(m)+n, 2f(m)+n$, etc, we get $f(kf(m)+n)-f(n)=kd$. Considering, that, when $k=1007$ we have $f(1007f(m)+n)-f(n)=1007d$, but by above $f(1007f(m)+n)-f(n)$ is also $2014f(m)$, yielding $2014f(m)=1007d$, or $d=2f(m)$. This means, $2f(m)=f(3m)-f(m)+2014$, i.e. $3f(m)=f(3m)+2014$, or simply $f(3m)=3f(m)-2014$. Inductively this also means that for all $m\neq 0$ we have $f(3^km)=3^kf(m)-(3^{k-1}+3^{k-2}+\cdots + 1)2014=3^kf(m)-\frac{3^k-1}{2}(2014)$. By Fermat's little theorem, we can choose $k_0$ such that $1007\mid 3^{k_0}-1$. 
	Thus for this $k_0$, we have $f(3^km)=f(m+(3^{k_0}-1)m)=f(m)+2(3^{k_0}-1)m$ (reason being that if $3^{k_0}-1=1007a$ then $f(m+(3^{k_0}-1)m)=f(m+1007am)=f(m)+2014am=f(m)+2(3^{k_0}-1)m$). 
	On the other hand the right hand suggests that this sum is also equal to $3^kf(m)-\frac{3^k-1}{2}(2014)$ and therefore
	\[(3^{k_0}-1)f(m)=2(3^{k_0}-1)m + \frac{3^{k_0}-1}{2}(2014) = (3^{k_0}-1)(2m + 1007)
	\]
	and therefore $f(m)=2m+1007$ (we need to choose $k_0>0$ so that $3^{k_0}-1\neq 0$, but such $k_0$ exists by FLT). 
	
	Finally, such function works: we have the left hand side as $4(2017)+6m+2n$, and same goes to right hand. Q.E.D. 
	
	\item[\textbf{A5}] 
	Consider all polynomials $P(x)$ with real coefficients that have the following property: for any two real numbers $x$ and $y$ one has\[|y^2-P(x)|\le 2|x|\quad\text{if and only if}\quad |x^2-P(y)|\le 2|y|.\]Determine all possible values of $P(0)$.
	
	\textbf{Answer.} $P(0)=1$ or $P(0) < 0$ are all the answers. 
	
	\textbf{Solution.} 
	Let $A(P)$ be the area in $\mathbb{R}^2$ covered by 
	$A(P) = \{(x, y): |y^2-P(x)|\le 2|x|\}$, and the condition suggests $A(P)$ is symmetric 
	w.r.t. the both the $x=y$ line, and also the $x$ axis (and therefore w.r.t. the $x=-y$ line and the $y$ axis). 
	
	We first give an example for $P(0) < 0$. 
	Take $a, c > 0$ such that $ac > 1$. 
	Consider, also, $P(x)=-(ax^2+c)$. 
	Then $P(x) \le -c < 0$ and therefore for any $y$, 
	\[
	y^2-P(x) \ge ax^2+c = a(|x| - \frac{1}{a})^2 + (c - \frac{1}{a}) + 2|x| > 2|x|
	\]
	since $c > \frac{1}{a}$. 
	This means that we in fact have $A(P)=\emptyset$ here, 
	which is valid. Since $P(0)=-c$ and $c$ can be anything $> 0$ (just make sure $a>\frac{1}{c}$), 
	any $P(0) < 0$ works. 
	
	Thus now we consider $P(0)\ge 0$. 
	Here, when $x=0$ the only possible points for $y$ are $\pm \sqrt{P(0)}$, 
	so analogously $(x, 0)\in A(P)$ iff $x=\pm \sqrt{P(0)}$. 
	To start off, we give a working example for $P(0)=1$. 
	This is taken with $P(x)=x^2+1$, and the condition now becomes 
	$y^2-(x^2+1)\le 2|x|$. Squaring both sides, we get 
	\[
	(x^2-1)^2+(y^2-1)^2-2x^2y^2\le 1
	\]
	which is symmetric with $x$ and $y$. 
	
	Let's first show we cannot have $P(x_0)-2|x_0| < 0 < P(x_0) + 2|x_0|$ for any $x_0$. 
	Otherwise, since $P$ and the absolute function are continuous, 
	there's an interval $[x_0-\epsilon, x_0+\epsilon]$ such that $P(x)-2|x| < 0 < P(x)+2|x|$ 
	for all $x$ in this interval (which is uncountable). 
	This contradicts $(x, 0)\in A(P)\Leftrightarrow x=\pm\sqrt{P(0)}$. 
	In particular, $P(x)\neq 0$ when $x\neq 0$. 
	
	Now, consider the case $P(0) > 0$. This would mean $P(x) > 0$ for all $x$, 
	and that $[P(x)-2|x|, P(x)+2|x|]$ can ever touch 0 only when $x^2=P(0)$. 
	In particular, $P(x)$ cannot be constant (otherwise for $x > P(0)$ we have $P(x) - 2x < 0 < P(x)+2x$). 
	
	Thus $P(x)$ has degree at least 2 (it cannot be linear since $P(x)$ has no real root). 
	Here, let's show that $P(x)$ must be quadratic, and must be monic. 
	Consider, now, $(x, y)\in A(P)$. 
	Let's consider what happens for $(x, y)$ sufficiently large. 
	Here, if the leading coefficient of $P(x)$ is $ax^b$ where $a>0$ and $b$ even, 
	then $y^2\in [P(x)-2|x|, P(x)+2|x|]$ for positive $y$ means for any $a'<a$ and sufficiently large $x$ we must have $y > a'x^{\frac{b}{2}}$. But the other side of the inequality would entail 
	$x > a'y^{\frac{b}{2}} > a'^2 x^{(\frac{b}{2})^2}$, 
	showing that we need $b=2$, and $a' < 1$. We therefore have $a\le 1$. 
	Likewise, for any $a''>a$ and sufficiently lagr $x$ we have $y < a''^2x$, which means $a''>1$ is necessary. 
	Repeating this argument gives us $1$, so $P$ is monic and quadratic. 
	
	Since $P(x)\ge 2x$ throughout, with the only positive root at $x=\sqrt{P(0)}$. 
	It follows that $P(x)-2x$ must have double root at $\sqrt{P(0)}$. 
	Similarly, $P(x)+2x$ must have double root at $-\sqrt{P(0)}$. 
	If $P(x)=x^2+ax+b$ then we have $(a-2)^2=(a+2)^2=4b$ (i.e. discriminant 0), 
	which then forces $a=0$, and $b=1$. 
	Thus $P(x)=x^2+1$ here and the only working $P(0)$ is 1. 
	
	Finally, we consider the case $P(0)=0$. Using the lemma that we cannot have $P(x)-2|x| < 0 < P(x)+2|x|$ for any $x$, 
	we see that the only root is 0. 
	Let's now distinguish into several cases: 
	\begin{itemize}
		\item $P(x) > 0$ for all $x\neq 0$. 
		By our lemma, we need $P(x)\ge 2|x|$ for all $x$, which is impossible since here, $P$ had double root at 0 and thus the derivative has $P'(0)=0$ (i.e. at 0 it will grow slower than $2|x|$). 
		
		\item $P(x) < 0$ for all $x\neq 0$. 
		Same thing, we need $P(x) \le -2|x|$ for all $x$, which is also impossible with $P'(0)=0$. 
		
		\item $P(x) > 0$ for $x>0$ and $P(x)<0$ for $x<0$. 
		Here, we need $P(x)\ge 2|x|$ for $x>0$, and $P(x)\le -2|x|$ for all $x<0$. 
		For $x>0$, the region $y\in [\sqrt{P(x)-2|x|}, \sqrt{P(x)+2|x|}]$ contains positive $y$ 
		(here $P(x)+2|x|\ge 2|x|+2|x|=4|x|>0$)
		while for $x<0$ this is either empty or can only be 0. 
		This contradicts the fact that $A(P)$ is symmetric w.r.t. the $y$-axis. 
		
		\item $P(x) < 0$ for $x>0$ and $P(x)>0$ for $x<0$, symmetric to above. 
	\end{itemize}
\end{enumerate}

\section*{Combinatorics}
\begin{enumerate}
	\item[\textbf{C1}] Let $n$ points be given inside a rectangle $R$ such that no two of them lie on a line parallel to one of the sides of $R$. The rectangle $R$ is to be dissected into smaller rectangles with sides parallel to the sides of $R$ in such a way that none of these rectangles contains any of the given points in its interior. Prove that we have to dissect $R$ into at least $n + 1$ smaller rectangles.
	
	\textbf{Solution.} We will prove this by induction: for $n=0$, we already have the rectangle $R$, hence one rectangle needed. 
	
	Now consider the set $S$ of $n$ points inside a given rectangle, and a configuration that splits it into $n+1$ rectangles.  Let a point $P$ be arbitrary. By induction hypothesis, the rectangle containing $S\backslash \{P\}$ must be dissected into at least $n$ rectangles. 
	
	By the problem requirement, $P$ must lie on some boundary, which is a segment that we have drawn to split it into the rectangles. We now recover the dissection of $S\backslash \{P\}$ by the following algorithm: 
	\begin{itemize}
		\item Remove the segment $\ell$ containing $P$ entirely. WLOG assume that it's horizontal. 
		
		\item The resulting configuration might not be rectangles, but this can fixed in the following way: let the interior of $\ell$ intersect with vertical lines $\ell_1, \cdots , \ell_k$, then we extend each $\ell_i$'s to meet the next posible horizontal line. 
	\end{itemize}
	Now from the second protocol, we know that $k+2$ rectangles correspond to our line $\ell$. After the removal of $\ell$ and extending the $\ell_i$'s, we have at most $k+1$ rectangles overlapping the space that would otherwise be covered by $\ell$. Thus the number of rectangles decrease by at least 1. By induction principle, we have at least $n$ rectangles after the removal of $\ell$, so we have at least $n+1$ of them in the beginning. 
	
	\item[\textbf{C2}] We have $2^m$ sheets of paper, with the number $1$ written on each of them. We perform the following operation. In every step we choose two distinct sheets; if the numbers on the two sheets are $a$ and $b$, then we erase these numbers and write the number $a + b$ on both sheets. Prove that after $m2^{m -1}$ steps, the sum of the numbers on all the sheets is at least $4^m$ .
	
	\textbf{Solution.} We consider the product on the $2^m$ paper, which is 1 in the beginning. Each time, the product increases by $\frac{(a+b)^2}{ab}$ times, and since $a, b > 0$ and $(a+b)^2-4ab=(a-b)^2\ge 0$, this ratio is at least 4. Thus the product after $m2^{m-1}$ steps is at least $4^{m2^{m-1}}=2^{m2^m}.$ The geometric mean is at least $2^m$ and by the AM-GM inequality, the arithmetic mean is at least $2^m$ too. Therefore the sum is at least $2^m\times 2^m=4^m$. 
	
	\item[\textbf{C3}] (IMO 2)
	Let $n \ge 2$ be an integer. Consider an $n \times n$ chessboard consisting of $n^2$ unit squares. A configuration of $n$ rooks on this board is peaceful if every row and every column contains exactly one rook. Find the greatest positive integer $k$ such that, for each peaceful configuration of $n$ rooks, there is a $k \times k$ square which does not contain a rook on any of its $k^2$ unit squares.
	
	\textbf{Answer.} $k=\lceil\sqrt{n}\rceil -1$\\
	\textbf{Solution.} We show that such $k$ fits if and only if $k^2<n$. 
	Consider, now, any peaceful configuration with $k^2<n$. Consider the first $k^2\times k^2$ squares: squares with coordinates $(i, j)$ with $1\le i, k\le k^2$, and suppose that the condition does not hold true. We now divide the $k^2\times k^2$ squares into $k\times k$ large grids, each large grid being a $k\times k$ square. By our assumption each large grid contains at least a rook, so there are at least $k^2$ rooks in the $k^2\times k^2$ squares. 
	But then these $k^2\times k^2$ squares span only $k^2$ columns (and $k^2$ rows) so there cannot be more than $k^2$ rooks. This means: 
	\begin{itemize}
		\item There are \emph{exactly} $k^2$ rooks in these $k^2\times k^2$ squares. 
		
		\item There is no rook in $(i, j)$ if exactly one of $i$ and $j$ lie in the interval $[1, k^2]$. 
		
		\item Consequently, the remaining $n-k^2$ rooks must be in the grid $(i, j)$ with $k^2+1\le i, j\le n$ (i.e. $m\times m$ square with $m=n-k^2$). 
	\end{itemize}
	In other words, there are $m$ rooks ($m=n-k^2$) in the bottom right corner of $m\times m$ squares. But then by rotational symmetry the same holds for the top left corner, top right corner and bottom left corner, which would contradict that each column / row contains exactly one rook. 
	
	Now we move on to prove that if $n$ and $k$ are such that $n\le k^2$, then there is a peaceful configuration such that any $k\times k$ square contains at least a rook. We first consider the case $n=k^2$, which allows us to consider (like above) $k\times k$ large grid, each of which is a $k\times k$ square. For the $(i, j)$-th large grid, we place a rook at coordinate $(j, i)$: in other words for each $1\le i, j\le k$ the rooks are at $((i-1)k+j, (j-1)k+i)$, and it's not hard to see that there are $k^2$ of them and each of them has different first- and second- coordinates. 
	
	To show that each $k\times k$ square must contain at least a rook, consider any $k\times k$ square. Now there are three cases: 
	\begin{itemize}
		\item This $k\times k$ square coincides with one of the big square before; by definition it contains at least a rook. 
		
		\item It overlaps with two of the big squares, and two opposite boundaries of this $k\times k$ square also coincide with the boundary of the two big squares. W.L.O.G. let the rows to be $(i-1)k+1, \cdots , ik$ (but the columns can be any $k$ consecutive columns). Recall that there are rooks at $((i-1)k+j, (j-1)k+i)$, $j=1, 2, \cdots , k$. Since the columns of the rooks are $(j-1)k+i$ for $j=1, 2, \cdots , k$, exactly one of them match the columns covered by this $k\times k$ square. Since the rows are $(i-1)k+j$, the rows of all rooks match the rows covered by this $k\times k$ square, too. Hence there's a rook in this $k\times k$ square. 
		
		\item Neither of the above holds, so it must overlap with four neighbouring $(2\times 2)$ big grids. Consider Let the four neighbouring big grids be $(i, j), (i, j+1), (i+1, j), (i+1, j+1)$. We also let the top left corner of our $k\times k$ square be $((i-1)k+a, (j-1)k+b)$ with $1\le a, b\le k$. 
		
		Considering $(i, j), (i, j+1)$, with rooks at $((i-1)k+j, (j-1)k+i)$ and $((i-1)k+(j+1), jk+i)$. Since our $k\times k$ square covers columns $(j-1)k+b, \cdots , jk+(b-1)$, either $(j-1)k+i$ or $jk+i$ is covered by these intervals. Hence, if both rooks are not in this grid then either $((i-1)k+j)$ or $((i-1)k+(j+1))$ is not covered as row. 
		Similarly, we have the following: 
		\begin{itemize}
			\item Considering $(i+1, j)$ and $(i+1, j+1)$ means that either $ik+j$ or $ik+(j+1)$ is not covered as row. 
			\item Considering $(i, j)$ and $(i+1, j)$ with rooks at $((i-1)k+j, (j-1)k+i)$ and $(ik+j, (j-1)k+(i+1))$ means either $(j-1)k+i$ or $(j-1)k+(i+1)$ is not covered as column. 
			
			\item Similarly considering $(i, j+1)$ and $(i+1, j+1)$ means either $jk+i$ or $jk+(i+1)$ is not covered as column. 
		\end{itemize}
		Thus considering our top left rows and columns are in the form $(i-1)k+a$ and $(j-1)k+b$ the only choice is $a=j+1$ and $b=i+1$, which covers rows $(i-1)k+(j+1), \cdots , ik+j$ and columns $(j-1)k+(i+1), \cdots jk+i$. 
		However, $(ik+j, (j-1)k+(i+1))$ itself contains a rook, contradiction. 
	\end{itemize}
	These effectively solve for $n=k^2$. 
	
	For $n<k^2$, all we need to do is simply extend the chess board from $n$ to $k^2$, put the rooks in the configuration above, and then remove those that do not lie in the first $n\times n$ grid. Now the condition still holds: each $k\times k$ square has at least a rook, and each of the remaining rooks are in different rows and columns. Then iteratively (one by one), we identify a column and a row without a rook, and place one at the intersection of the column and row until we have all $n$ of them. Each $k\times k$ square will still contain a rook since we only add the rooks, not remove them. 
	
	\item [\textbf{C6}] We are given an infinite deck of cards, each with a real number on it. For every real number $x$, there is exactly one card in the deck that has $x$ written on it. Now two players draw disjoint sets $A$ and $B$ of $100$ cards each from this deck. We would like to define a rule that declares one of them a winner. This rule should satisfy the following conditions:
	\begin{enumerate}
		\item The winner only depends on the relative order of the $200$ cards: if the cards are laid down in increasing order face down and we are told which card belongs to which player, but not what numbers are written on them, we can still decide the winner.
		
		\item If we write the elements of both sets in increasing order as $A =\{ a_1 , a_2 , \ldots, a_{100} \}$ and $B= \{ b_1 , b_2 , \ldots , b_{100} \}$, and $a_i > b_i$ for all $i$, then $A$ beats $B$.
		\item If three players draw three disjoint sets $A, B, C$ from the deck, $A$ beats $B$ and $B$ beats $C$ then $A$ also beats $C$.
		
	\end{enumerate}
	How many ways are there to define such a rule? Here, we consider two rules as different if there exist two sets $A$ and $B$ such that $A$ beats $B$ according to one rule, but $B$ beats $A$ according to the other.
	
	\textbf{Answer.} 100. 
	
	\textbf{Solution.} Denote $n=100$ here, and we want to define an order $>$ on the sets $A, B\in \mathbb{R}^{n}$ satisfying: 
	\begin{itemize}
		\item If $A\cap B=\emptyset$ then either $A>B$ or $B>A$ but not both. (The converse in this case is $<$ sign, provided the sets are disjoint). 
		
		\item The three conditions above. 
	\end{itemize}
	We claim that there are exactly $n$ ways to order them via the following: choose one $k$ such that $k\in \{1, 2, \cdots , n\}$, and let $A=\{a_1, \cdots , a_n\}$ and $B=\{b_1, \cdots, b_n\}$ be disjoint, ordered in increasing order. Then $A>B$ if and only if $a_k>b_k$. Clearly this ordering fulfills all three conditions above: if we order the $2n$ cards (jointly) we can identify whether $a_k>b_k$ based on the relative positions of $a_k$ and $b_k$ alone, and if $a_k>b_k$ and $b_k>c_k$ then $a_k>c_k$. 
	
	To show that these are the only ways, for each $A>B$ we denote $f(A, B)$ as the number of indices $k$ with $a_k>b_k$ (ordered in increasing order, again). That is, 
	\[
	f(A, B)|_{A, B} = |\{k: a_k>b_k\}|\qquad A=\{a_1<a_2<\cdots <a_n\}\quad B=\{b_1<b_n<\cdots < b_n\}
	\]
	Denote $m=\min\{f(A, B): A>B\}$. From the second condition, if $A>B$ then $a_i>b_i$ for some $i$ so $m\ge 1$. We now show that $m=1$. 
	
	Suppose, on the contrary, that $m>1$. Consider, now, $A, B$ with $A>B$ and $f(A, B)=m$. Let $s_0$ be the minimal index satisfying $a_{s_0} > b_{s_0}$ and $s_1>s_0$ be the maximal index satisfying $a_{s_1} > b_{s_1}$. Denote, also: 
	\[
	\epsilon = \min\{|a_i-a_j|: i\neq j\}\cup \{|b_i-b_j|: i\neq j\}\cup |\{|a_i-b_j|\}|
	\]
	which is positive since $A$ and $B$ are disjoint. 
	Now denote the sequence $C=\{c_1<\cdots < c_n\}$ as follows: 
	\begin{itemize}
		\item For each $i$ with $a_i<b_i$, we denote $c_i=a_i+\frac{\epsilon}{10}$
		\item $c_{s_0}=b_{s_0}-\frac{\epsilon}{10}$, $c_{s_1}=a_{s_1}+\frac{\epsilon}{10}$. 
		\item For each $i$ with $a_i>b_i$ and $i\neq s_0, s_1$, denote $c_i=a_i-\frac{\epsilon}{10}$. 
	\end{itemize}
	We need to show that the constraint $i<j\to c_i<c_j$ still holds. Indeed, if none of $i, j$ is the same as $s_0, s_1$ then for any $j>i$: 
	\[
	c_j-c_i \ge (a_j - \frac{\epsilon}{10})-(a_i + \frac{\epsilon}{10})= (a_j-a_i)-\frac{\epsilon}{5} > \epsilon - \frac{\epsilon}{5} > 0
	\]
	since the pairwise difference of the numbers is at least $\epsilon$. 
	For $s_0$ and $s_1$, we note that: 
	\[
	c_{s_1}-c_{s_0}= a_{s_1}-b_{s_0} + \frac{\epsilon}{5} > a_{s_1} - b_{s_0} > a_{s_1} - a_{s_0} = 0
	\]
	and for other cases (when $i\neq s_0, s_1$), 
	\[
	c_{s_1}-c_i = a_{s_1} - a_i\begin{cases}
	 > 0 & s_1 > i\\
	 < 0 & s_1 < i\\
	\end{cases}
	\]
	and finally for the less straightforward case $s_0$, if $i<s_0$ then 
	\[
	c_{s_0} - c_i = (b_{s_0} - \frac{\epsilon}{10}) - (a_i + \frac{\epsilon}{10}) < (b_{s_0} - \frac{\epsilon}{10}) - (b_i - \frac{\epsilon}{10}) = b_{s_0} - b_i - \frac{\epsilon}{5} > 0
	\]
	since $a_i<b_i$ for all $i<s_0$ due to the minimality of $s_0$, and if $i>s_0$ then 
	\[
	c_i-c_{s_0} \ge (a_i - \frac{\epsilon}{10}) - (b_{s_0} - \frac{\epsilon}{10}) > a_i - a_{s_0} > 0
	\]
	since $a_{s_0} > b_{s_0}$. This establishes the inequality. 
	
	Now we compare $A, C$ and $B$. Notice that if $a_i<b_i$, we have $a_i<c_i<b_i$ (the left inequality is immediate; the second follows from the assumption $b_i-a_i\ge \epsilon > \frac{\epsilon}{5} = c_i-a_i$). Similarly, if $a_i>b_i$ and $i\neq s_0, s_1$ then $a_i>c_i>b_i$ ($m-2$ of the indices here). For $s_0, s_1$ we have the following: 
	\[
	a_{s_0} > b_{s_0} > b_{s_0} - \frac{\epsilon}{10} = c_{s_0}\qquad 
	c_{s_1} = a_{s_1} + \frac{\epsilon}{10} > a_{s_1} > b_{s_1}
	\]
	So $a_i>c_i$ on exactly $m-1$ of the indices $i$ (including $s_0$, but not $s_1$), and $c_i>b_i$ on exactly $m-1$ of the indices $i$ (including $s_1$, but not $s_0$). By the minimality of $m$, we have $A<C$ and $C<B$, but then by the third problem condition, $A<B$ must then hold, which is a contradiction. 
	
	Thus we have proven that $m=1$, which means there exists $A>B$ and an index $k$ such that $a_i>b_i$ if and only if $i=k$. We show that this holds for all pairs $(A, B)$ satisfying this condition: whenever $A=\{a_1, \cdots , a_n\}$ and $B=\{b_1, \cdots , b_n\}$ with $a_1<\cdots < a_n$ and $b_1<\cdots < b_n$. 
	This requires the following lemma: 
	
	\emph{Lemma}. Provided there exists a $k$ such that $A>B$ with $a_i>b_i$ if and only if $i=k$, then there exists $B'$ satisfying $A>B'$ with each $b'_i$ be: 
	\begin{itemize}
		\item arbitrarily close to $b'_i$ for $i<k$
		\item arbitrarily large for for $i>k$. 
	\end{itemize}
	This is, we want to show that $B=\{b_1<\cdots < b_n\}$ (satisfying $a_k<b_k$) be our `target' set, we show there exists $B'=\{b'_1<\cdots < b'_n\}$ with $b_i'>b_i$ for all $i$, and $A>B'$. Specifically, we will define an iterative algorithm 
	\[
	A=B_0>B_1>\cdots > B_t
	\]
	where $t$ is the number of steps such that $B_t$ satisfies the condition needed for $B'$. For clarity, we also name $b_{u, i}$ as the $i$-th smallest element of $B_u$. 
	
	To see how, we recall that there exists $A_0<B_0$ such that $(A_0)_i>(B_0)_i$ only when $i=k$; consider the ordering $O$ of $A_0\cup B_0$ in ascending order. At each step $u$, we consider: 
	\begin{itemize}
		\item the greatest index $i_1 < k$ such that $b_{ui_1} < b_{i_1}$ (if exists)
		\item the greatest index $i_2 > k$ such that $b_{ui_2} < b_{i_2}$ (if exists)
	\end{itemize}
	 We show that we can position $b_{(u+1)i}$ such that $b_{(u+1)i}=b_i$ for both $i_1, i_2$, and $(B_u, B_{u+1})$ will satisfy the order $O$ (thereby giving $B_u>B_{u+1})$. 
	 
	 We first deal with the case $i_1=k-1$ on the very first step. Observe that we have $b_k<a_k=b_{0k}$ and according to the ordering $O$, $b_{1(k-1)}>b_{0(k-1)}$, so $b_{1(k-1)}$ must be between $b_{0(k-1)}$ and $b_{0(k)}$; we will make both $b_{1(k-1)}$ and $b_{1(k)}$ be arbitrarily close to (still smaller than) $b_{0(k)}$. 
	 
	 For $i_2$, if an ordering $O$ were to be fulfilled, then from $b_{ui_2} < b_{(u+1)i_2}$, if there were to be an index $j$ such that $b_{(u+1)i_2}<b_{uj}$ then $j>i_2$. By maximality of $i_2$ we have $b_{uj}>b_j$ so we can make $b_{(u+1)i_2}$ be arbitrarily close to $b_{uj}$ to fulfill $b_{(u+1)i_2}>b_j>b_{i_2}$. 
	 
	 For $i_1$, if an ordering $O$ were to be fullfilled, similarly we have $b_{ui_1}<b_{(u+1)i_1}$, if there were to be an index $j$ such that $b_{(u+1)i_1}<b_{uj}$ then $j>i_1$ (possibly, $j=k$). If $j\neq k$, by maximality of $i_1$ we have $b_{uj}>b_j$ so we can make $b_{(u+1)i_1}$ be arbitrarily close to $b_{uj}$ to fulfill $b_{(u+1)i_1}>b_j>b_{i_1}$. The case $j=k$ is dealt with before. 
	 
	 Notice that for other $i$, we can just position them arbitrarily as long as they fulfill the condition $O$; if $b_{ui}>b_i$ then $b_{(u+1)i}>b_{ui}>b_i$ for $i\neq k$ will hold and we don't have to worry about that. The only concern is when $i=k$ since $b_{(u+1)k}<b_{uk}$ but since $b_{0k}>b_k$, we just have to make sure that $b_{uk}>b_{k}$ all the time (which is always possible since we only need to be careful of the condition $b_{u(k-1)} < b_{(u+1)(k-1)} < b_{(u+1)k} < b_{uk}$ and so long as $b_{uk} > b_k$, we can make sure the same for $b_{(u+1)k}$). 
	 
	 Finally, now that we have the lemma, consider $A$ and $B$ such that, when arranged in ascending order, $a_i>b_i$ if and only if $i=k$. Choose $B'$ such that $b'_i>b_i$ for all $i$, and such that $A>B'$ (as per above). By the second condition, $B'>B$ and therefore $A>B$, as desired. 
\end{enumerate}

\section*{Geometry}
\begin{enumerate}
	%%%G1%%%
	\item [\textbf{G1}] (IMO 4) Let $P$ and $Q$ be on segment $BC$ of an acute triangle $ABC$ such that $\angle PAB=\angle BCA$ and $\angle CAQ=\angle ABC$. Let $M$ and $N$ be the points on $AP$ and $AQ$, respectively, such that $P$ is the midpoint of $AM$ and $Q$ is the midpoint of $AN$. Prove that the intersection of $BM$ and $CN$ is on the circumference of triangle $ABC$.
	
	\textbf{Solution.} We need to prove that $\angle CBM + \angle BCN = \angle BAC$. 
	Notice that the triangle $ABC$ is similar to both $QAC$ and $PBA$, so we have $\frac{QN}{QC}=\frac{QA}{QC}=\frac{PB}{PA}=\frac{PB}{PM}$. 
	Coupled with the fact that $\angle APB = \angle AQC = \angle BAC$ we have $\angle CQN = \angle BPM$, so the triangles $QNC$ and $PBM$ are similar, meaning that $\angle CBM + \angle BCN = \angle QNC + \angle QCN = 180^{\circ} - \angle CQN = \angle AQC = \angle BAC$, as desired. 
	
	%%%G2%%%
	\item [\textbf{G2}] Let $ABC$ be a triangle. The points $K, L,$ and $M$ lie on the segments $BC, CA,$ and $AB,$ respectively, such that the lines $AK, BL,$ and $CM$ intersect in a common point. Prove that it is possible to choose two of the triangles $ALM, BMK,$ and $CKL$ whose inradii sum up to at least the inradius of the triangle $ABC$.
	
	\textbf{Solution.} Let $A_B$ on line $AB$ and $A_C$ on line $AC$ such that $AA_BA_C$ and $ALM$ has the same incircle, and that $A_BA_C\parallel BC$. Define $B_A, B_C, C_A, C_B$ similarly. For each triangle $ABC$ denote $f(ABC)$ the inradius of triangle $ABC$. We have 
	\[
	\frac{AA_B}{AB}=\frac{AA_C}{AC}=\frac{f(AA_BA_C)}{f(ABC)}
	\]
	and if the conclusion is false, then we have $\frac{AA_B}{AB}+\frac{BB_A}{AB} < 1$, or $AA_B+BB_A<AB$. Similarly, $BB_C+CC_B<BC$ and $AA_C+CC_A<AC$. This means $B, B_C, C_B, C$ are on $BC$ in that order, and similarly for $A, A_B, B_A, B$ on $AB$, and $A, A_C, C_A, C$ on $AC$. 
	
	Now, if $K$ is on segment $B_CC_B$, then since $BK>BB_C$ and $BKM$ and $BB_AB_C$ share the same incircle, we have $BM < BB_A$ and similarly $CL < CC_B$. This means $AM > AB_A > AA_B$ and $AL > AC_A > AA_C$. This contradicts the fact that $ALM$ and $AA_BA_C$ have the same incircle. 
	
	Therefore $K$ lies either on segment $BB_C$ and $CC_B$. In the first case that $K$ on $BB_C$, we have $M$ on $AA_B$ and $L$ on $CC_A$. Since $AK, BL, CM$ intersect in a common point, we have 
	\[
	\frac{AM}{MB}\cdot \frac{BK}{KC}\cdot \frac{CL}{LA} = 1
	\]
	and since $AM < AA_B$ and $MB < BA_B$ (similarly for others), we have 
	\[
	\frac{AA_B}{BA_B}\cdot \frac{BB_C}{CB_C}\cdot \frac{CC_A}{AC_A} > 1
	\]
	which means the product of the two of above is $>1$. W.l.o.g. we assume $\frac{AA_B}{BA_B}\cdot \frac{BB_C}{CB_C} > 1$. This means: 
	\[
	\frac{AA_B}{AB-AA_B}\cdot \frac{BB_C}{CB - BB_C} > 1\;\to
	\;
	\frac{AA_B/AB}{1-AA_B/AB}\cdot \frac{BB_C/BC}{1 - BB_C/BC} > 1
	\]
	which then translates into $\frac{AA_B}{AB}+\frac{BB_C}{BC}>1$. But this means $f(AA_BA_C)+f(BB_AB_C) > 1$, contradiction. 
	
	%%%G3%%%
	\item [\textbf{G3}] Let $\Omega$ and $O$ be the circumcircle and the circumcentre of an acute-angled triangle $ABC$ with $AB > BC$. The angle bisector of $\angle ABC$ intersects $\Omega$ at $M \ne B$. Let $\Gamma$ be the circle with diameter $BM$. The angle bisectors of $\angle AOB$ and $\angle BOC$ intersect $\Gamma$ at points $P$ and $Q,$ respectively. The point $R$ is chosen on the line $P Q$ so that $BR = MR$. Prove that $BR\parallel AC$.
	(Here we always assume that an angle bisector is a ray.)
	
	\textbf{Solution.} We first notice that $OP$ is the perpendicular bisector of $AB$, and $OQ$ the perpendicular bisector of $BC$. Now extend $OP$ to meet $\Gamma$ again at $P_1$, and definte $Q_1$ similarly. 
	If $N$ is the midpoint of $BM$, then $ON$ is perpendicular to $BM$, and since $BM$ is the internal angle bisector of $\angle ABC$, line $OP$ and line $OQ$ are symmetric in the line passing through $O$ and parallel to $BM$. Thus $OP$ and $OQ$ are also symmetric in $ON$, meaning that $PQP_1Q_1$ is actually isoceles trapezoid with parallel sides $PQ_1$ and $QP_1$. 
	Moreover, if $PQ$ and $P_1Q_1$ were to intersect at $R_1$, the points $O, N, R_1$ are all collinear, and is the perpendicular bisector of $PQ_1$, $QP_1$. Considering that $BM$ is the diameter of the circle and $ON\perp BM$, $ON$ is also the perpendicular bisector of $BM$. Since $R_1$ is on this perpendicular bisector, $BR_1=MR_1$, and since $R_1$ is also on $PQ$, we have $R=R_1$. 
	
	By Brokard's theorem, taking the point of infinity $P_{\infty}$ determined by the parallel lines $PQ_1$ and $QP_1$, we have $R$ as the polar of the line $OP_{\infty}$ (i.e. the line through $O$ parallel to $BM$). Consider, now, the lines $BR$ and $MR$ and let them intersect $\Gamma$ again at $B_1$ and $M_1$. Since $BR=MR$, we also have $BM\parallel B_1M_1$. Also by above (Brokard's theorem again), $MB_1$ and $BM_1$ intersect at $O$. Now $\angle MB_1B=90^{\circ}$, so $BM_1$ (i.e. line $BR$) is perpendicular to $MB_1$, i.e. $MO$. But $AC$ is also perpendicular to $MO$ since $M$ is the midpoint of arc $AC$, so $AC\parallel BR$. Q.E.D. 
	
	%%%G4%%%
	\item [\textbf{G4}] Consider a fixed circle $\Gamma$ with three fixed points $A, B,$ and $C$ on it. Also, let us fix a real number $\lambda \in(0,1)$. For a variable point $P \not\in\{A, B, C\}$ on $\Gamma$, let $M$ be the point on the segment $CP$ such that $CM =\lambda\cdot  CP$ . Let $Q$ be the second point of intersection of the circumcircles of the triangles $AMP$ and $BMC$. Prove that as $P$ varies, the point $Q$ lies on a fixed circle.
	
	\textbf{Solution.} Invert w.r.t. $C$, then the problem formulation goes into the following: 
	
	Consider points $A, B, C$ that are not collinear, and $P$ variable line on line $AB$, and $M$ on line $CP$ with $CM=\frac{1}{\lambda}CP$. Let $Q$ be the second point of intersection of circle $AMP$ and line $BM$. Prove that as $P$ varies, $Q$ lies on a fixed circle. 
	
	Now, power of point theorem gives us $BP\cdot BA=BM\cdot BQ$ with $Q, M$ lie on the same side of $B$ iff $P, A$ lie on the same side of $B$. Thus, using signed length notation we have $BQ = BA\cdot \frac{BP}{BM}$ where we fix $BP>0$ iff $P, A$ lie on the same side of $B$. 
	
	Considering the triangle $BPC$ and the Cevian $BM$, we get (ignoring signed convention for now)
	\[
	1 - \lambda = \frac{1/\lambda - 1}{1/\lambda} = \frac{PM}{MC} = \frac{BP\sin\angle CBP}{BM\sin\angle CBM}
	\]
	and therefore 
	\[
	BQ = BA\cdot \frac{BP}{BM} = BQ = BA\cdot (1-\lambda) \cdot \frac{\sin\angle CBM}{\sin\angle CBP}
	= BA\cdot (1-\lambda) \cdot \frac{\sin\angle CBM}{\sin\angle CBA}
	\]
	and if we do care about sign notation, we can make $BP$ positive (that is, in the direction of ray $BA$) if and only if $\angle CBM$ is positive (that is, $A$ is in the angle domain of $\angle CBM$ or equivalently, $M$ and $A$ are on the same side w.r.t. $CB$). In either case (where $M$ and $A$ are on the same side w.r.t. $CB$), $Q$ will be on the same side as $A$ w.r.t. $CB$ (since, in case $BP<0$, $M$ will be on different side as $A$ w.r.t. $CB$ but then $Q$ and $M$ are on different side w.r.t. $B$, on the line $BM$). TThus we have $BQ=\sin\angle CBM\cdot [BA\cdot \frac{(1-\lambda)}{\sin\angle CBA}]$. Name $BA\cdot \frac{(1-\lambda)}{\sin\angle CBA}=\gamma$, then $BQ$ will be on a circle passing through $B$, tangent to $BC$ and has $A$ on the same side as the circle w.r.t. $BC$, and has diamater $\gamma$, as desired (notice that this $\gamma$ does not depend on $P$). 
	
	%%%G5%%%
	\item [\textbf{G5}] (IMO 3) Convex quadrilateral $ABCD$ has $\angle ABC = \angle CDA = 90^{\circ}$. Point $H$ is the foot of the perpendicular from $A$ to $BD$. Points $S$ and $T$ lie on sides $AB$ and $AD$, respectively, such that $H$ lies inside triangle $SCT$ and \[
	\angle CHS - \angle CSB = 90^{\circ}, \quad \angle THC - \angle DTC = 90^{\circ}. \] Prove that line $BD$ is tangent to the circumcircle of triangle $TSH$.
	
	\textbf{Solution.} We first show that $Q$, the circumcenter of $CHS$, lies on the line $AB$. To see this, by the identity of angle of chord subtending at the center vs angle subtending on the arc we have 
	\[2\angle CHS = 360^{\circ} - \angle CQS
	\]
	since $\angle CHS>90^{\circ}$ by the problem definition. 
	Since $CQ=CS$, we have $\angle CSQ = 90^{\circ} - \frac{\angle CQS}{2} = 90^{\circ} - \frac{360^{\circ} - 2\angle CHS}{2} = \angle CHS - 90^{\circ} = \angle CSB$, implying that $Q, S, B$ are collinear and therefore all lie on line $AB$. Similarly, naming $R$ as the circumcenter of $DHT$ we have $R$ on $AD$. 
	
	\definecolor{uuuuuu}{rgb}{0.26666666666666666,0.26666666666666666,0.26666666666666666}
	\definecolor{xdxdff}{rgb}{0.49019607843137253,0.49019607843137253,1}
	\definecolor{ududff}{rgb}{0.30196078431372547,0.30196078431372547,1}
	\begin{tikzpicture}[line cap=round,line join=round,>=triangle 45,x=1cm,y=1cm]
		\clip(-13.16618923340823,-11.014882328407293) rectangle (11.24783371628891,6.112065330220502);
		\draw [line width=2pt] (-3.58,0.04) circle (3.3458615956848012cm);
		\draw [line width=2pt,dash pattern=on 1pt off 1pt] (-5.095346372448951,-5.672474501237494) circle (4.257385531622247cm);
		\draw [line width=2pt,dash pattern=on 1pt off 1pt] (-1.4878035010175643,-1.1660966388607872) circle (1.7396394100390575cm);
		\draw [line width=2pt] (-5.256739923962409,2.935398598616256)-- (-8.287432668860308,-8.489550403858733);
		\draw [line width=2pt] (-5.256739923962409,2.935398598616256)-- (-0.23413840431519883,0.04);
		\draw [line width=2pt] (-0.23413840431519883,0.04)-- (-1.9032600760375908,-2.855398598616256);
		\draw [line width=2pt] (-6.471411222043291,-1.6436065343543205)-- (-1.9032600760375908,-2.855398598616256);
		\draw [line width=2pt] (-6.471411222043291,-1.6436065343543205)-- (-0.23413840431519883,0.04);
		\draw [line width=2pt] (-5.256739923962409,2.935398598616256)-- (-4.187174591870141,-1.027030054704387);
		\draw [line width=2pt] (-6.104229997941373,-0.25942571724807206)-- (-4.187174591870141,-1.027030054704387);
		\draw [line width=2pt] (-4.187174591870141,-1.027030054704387)-- (-4.086634001027033,2.260863094207449);
		\draw [line width=2pt] (-4.187174591870141,-1.027030054704387)-- (-1.9032600760375908,-2.855398598616256);
		\draw [line width=2pt] (-1.0723469259975378,0.523205320894682)-- (-8.287432668860308,-8.489550403858733);
		\draw [line width=2pt] (-3.045217333953865,-1.9412143266603212)-- (-6.471411222043291,-1.6436065343543205);
		\draw [line width=2pt] (-4.77036394304444,-1.9311315587874773) -- (-4.746264612952715,-1.6536893022271637);
		\draw [line width=2pt] (-3.045217333953865,-1.9412143266603212)-- (-0.23413840431519883,0.04);
		\draw [line width=2pt] (-1.719894078647529,-0.8367910590636544) -- (-1.5594616596215338,-1.064423267596667);
		\begin{scriptsize}
			\draw [fill=ududff] (-5.256739923962409,2.935398598616256) circle (2.5pt);
			\draw[color=ududff] (-5.066860327092354,3.4316283999068027) node {$A$};
			\draw [fill=ududff] (-1.9032600760375908,-2.855398598616256) circle (2.5pt);
			\draw[color=ududff] (-1.7250168815064333,-2.3469758914188352) node {$C$};
			\draw [fill=xdxdff] (-6.471411222043291,-1.6436065343543205) circle (2.5pt);
			\draw[color=xdxdff] (-6.296844373037172,-1.140199091623923) node {$B$};
			\draw [fill=xdxdff] (-0.23413840431519883,0.04) circle (2.5pt);
			\draw[color=xdxdff] (-0.05409515871347294,0.5307226311690324) node {$D$};
			\draw [fill=uuuuuu] (-4.187174591870141,-1.027030054704387) circle (2pt);
			\draw[color=uuuuuu] (-3.999327004196852,-0.5832251840262713) node {$H$};
			\draw [fill=uuuuuu] (-1.0723469259975378,0.523205320894682) circle (2pt);
			\draw[color=uuuuuu] (-0.8895560201099532,0.9716603080171735) node {$R$};
			\draw [fill=uuuuuu] (-8.287432668860308,-8.489550403858733) circle (2pt);
			\draw[color=uuuuuu] (-8.594361741877492,-8.636139598042321) node {$Q$};
			\draw [fill=uuuuuu] (-6.104229997941373,-0.25942571724807206) circle (2pt);
			\draw[color=uuuuuu] (-5.9255284346387365,0.18261393892049999) node {$S$};
			\draw [fill=uuuuuu] (-4.086634001027033,2.260863094207449) circle (2pt);
			\draw[color=uuuuuu] (-3.9064980195972425,2.7122037692598355) node {$T$};
			\draw [fill=uuuuuu] (-3.045217333953865,-1.9412143266603212) circle (2pt);
			\draw[color=uuuuuu] (-3.2799023735498825,-1.3954787992728468) node {$N$};
		\end{scriptsize}
	\end{tikzpicture}
	
	Now, with the fact that $AH$ is perpendicular to $BD$, the goal reduces to showing that the circumcenter $O$ of $TSH$ lies on $AH$. But then $O$ is on the perpendicular bisector of $SH$, which is in turn the internal angle bisector of $SQH$ (i.e. angle bisector of $AQH$) because $SQ=QH$. Similarly $O$ is on the angle bisector of $ARH$. So all we need to prove is that the angle bisectors of $\angle SQH$ and $\angle ARH$ intersect at $AH$. By the angle bisector theorem this is equivalent to proving the following ratio: 
	\[
	\frac{AQ}{QH} = \frac{AR}{RH}\to \frac{AQ}{AR} = \frac{QH}{RH}
	\]
	the second equivalence being equivalent to the first, hence we will prove the second one. 
	
	Let $N$ be the midpoint of $CH$, the $QR$ is the perpendicular of $CH$, hence perpendicular to $CH$ and passes through $N$. We also have $\angle CBQ=\angle CNQ = 90^{\circ}$ so $C, B, Q, N$ are concyclic. Similarly $C, B, R, N$ is concyclic. We now have the following ratio, thanks to sine rule: 
	\[
	\frac{AQ}{AR} = \frac{\sin\angle QRA}{\sin\angle RQA} = \frac{\sin\angle NCD}{\sin\angle NCB}
	\]
	and since $QH=QC$ and $RH=RC$, 
	\[
	\frac{QH}{RH} = \frac{QC}{RC} = \frac{\sin\angle CRQ}{\sin\angle CQR} = \frac{\sin\angle CDN}{\sin\angle CBN}
	\]
	but then $\frac{\sin\angle NCD}{\sin\angle NCB} = \frac{\sin\angle CDN}{\sin\angle CBN}$ if and only if $\frac{\sin\angle NCD}{\sin\angle CDN} = \frac{\sin\angle NCB}{\sin\angle CBN}$. The first ratio is (sine rule again) the same as $\frac{NB}{NC}$ and the second, $\frac{ND}{NC}$, so all we need now is $NB=ND$. 
	To see why this is true, let $L$ be the midpoint of $AC$. Then since $\angle B=\angle D=90^{\circ}$, $L$ is the circumcenter of the qudrilateral $ABCD$, and hence $LB=LD$. Since $N$ is the midpoint of $CH$ we also have $AH\parallel NL$, meaning that $NL\perp BD$. But since $L$ is on the perpendicular bisector of $BD$, so is $N$, and therefore $NB=ND$. Q.E.D. 
	
	%%%G6%%%
	\item[\textbf{G6}] Let $ABC$ be a fixed acute-angled triangle. Consider some points $E$ and $F$ lying on the sides $AC$ and $AB$, respectively, and let $M$ be the midpoint of $EF$ . Let the perpendicular bisector of $EF$ intersect the line $BC$ at $K$, and let the perpendicular bisector of $MK$ intersect the lines $AC$ and $AB$ at $S$ and $T$ , respectively. We call the pair $(E, F )$ $\textit{interesting}$, if the quadrilateral $KSAT$ is cyclic.
	Suppose that the pairs $(E_1 , F_1 )$ and $(E_2 , F_2 )$ are interesting. Prove that $\displaystyle\frac{E_1 E_2}{AB}=\frac{F_1 F_2}{AC}$
	
	\textbf{Solution.} 
	We'll show that $\frac{AE}{AB} + \frac{AF}{AC} = 2\cos\angle BAC$ whenever $(E, F)$ is interesting, which will immediately prove the problem. 
	
	Let's first look at the quadrilateral $KSAT$. 
	Since $EF$ and $ST$ are parallel, $AM$ would intersect $ST$ at midpoint of $ST$, say $N$. 
	This means we would also have $NM\cdot NA = NS^2=NT^2$ (reflect $M$ in $N$ to get $P$, $P$ is on circle $AST$ and $MTPS$ is parallelogram). 
	Therefore 
	\[
	\angle KMS = \angle KTS = \angle MTS = \angle MAT
	\]
	so $AK$ and $AM$ are reflection of each other w.r.t. bisector of $\angle BAC$. 
	It sollows then $AK$ is a symmedian of $\angle TAS$. 
	
	Next, let tangents to circle $AST$ at $S$ and $T$ to intersect at $L$. 
	From before, $A, K, L$ are collinear. 
	In addition, $NL$ and $MK$ are both perpendicular to $ST$, 
	so triangles $ALN$ and $AKM$ are similar. This means 
	\[
	\frac{AF}{AT}=\frac{AM}{AN} = \frac{AK}{AL}
	\]
	so $FK\parallel LT$ and similarly $EK\parallel LS$. 
	By tangency of the lines $LT$ and $LS$ we have 
	\[
	\angle KFT = 180^{\circ} - \angle ATL = \angle AST
	\qquad 
	\angle KES = \angle ATS
	\]
	Consider, now, circle with radius $KE=KF$ intersect $AT$ and $AS$ at 
	$F_1$ and $E_1$ again, respectively. 
	Then 
	\[
	\angle AF_1K = \angle AST\qquad \angle AE_1K = \angle ATS
	\]
	so in particular, $F_1, K, E_1$ are collinear and $K$ is the midpoint of $E_1F_1$. 
	This also gives us $AS\perp EF_1$ and $AT\perp FE_1$, giving 
	\[
	\frac{AE}{AF_1}=\frac{AF}{AE_1}=\cos \angle TAS
	\]
	Finally, consider $E_2, F_2$ as the midpoints of $AE_1$ and $AF_1$, respectively. 
	Then $F_2K\parallel AE_1\parallel AS\parallel AC$ ($AS_1, AS, AC$ are all the same line) 
	and similarly $E_2K\parallel AB$. 
	So looking (finally) at triangle $ABC$ we have 
	\[
	\frac{BC}{AB} = \frac{BK}{BF_2} = \frac{CK}{AF_2}
	\qquad 
	\frac{BC}{AC} = \frac{CK}{CE_2} = \frac{BK}{AE_2}
	\]
	
	\definecolor{qqwuqq}{rgb}{0,0.39215686274509803,0}
	\definecolor{uuuuuu}{rgb}{0.26666666666666666,0.26666666666666666,0.26666666666666666}
	\definecolor{ududff}{rgb}{0.30196078431372547,0.30196078431372547,1}
	\begin{tikzpicture}[line cap=round,line join=round,>=triangle 45,x=1cm,y=1cm]
		\clip(-9.311123409547324,-4.712123761117267) rectangle (6.2890994179244135,6.23175879275551);
		\draw[line width=2pt,color=qqwuqq,fill=qqwuqq,fill opacity=0.10000000149011612] (-3.418341224869141,1.122153962878833) -- (-3.431338770190094,0.8078496851175936) -- (-3.117034492428855,0.7948521397966402) -- (-3.1040369471079012,1.1091564175578796) -- cycle; 
		\draw[line width=2pt,color=qqwuqq,fill=qqwuqq,fill opacity=0.10000000149011612] (-2.8743042777612398,-0.19700245467904645) -- (-2.8873018230821934,-0.511306732440286) -- (-2.5729975453209537,-0.5243042777612396) -- (-2.56,-0.21) -- cycle; 
		\draw [line width=2pt] (-2.4794473077373302,1.7379105583518444) circle (3.299779428926932cm);
		\draw [line width=2pt] (-4.453755836691207,4.381894316257579)-- (-5.7843982618248075,-3.4012628126980693);
		\draw [line width=2pt] (-5.7843982618248075,-3.4012628126980693)-- (-0.6377943567825096,0.44179558525544976);
		\draw [line width=2pt] (-4.453755836691207,4.381894316257579)-- (0.1,-0.32);
		\draw [line width=2pt] (-2.09950330093086,1.951055462813537) -- (-2.2727479527239067,1.9380976172310977);
		\draw [line width=2pt] (-2.09950330093086,1.951055462813537) -- (-2.0810078839673007,2.123796699026482);
		\draw [line width=2pt] (-2.2542525357603473,2.110838853444042) -- (-2.427497187553394,2.0978810078616026);
		\draw [line width=2pt] (-2.2542525357603473,2.110838853444042) -- (-2.235757118796788,2.283580089656987);
		\draw [line width=2pt] (-1.9447540661013731,1.7912720721830313) -- (-2.1179987178944195,1.778314226600592);
		\draw [line width=2pt] (-1.9447540661013731,1.7912720721830313) -- (-1.9262586491378137,1.9640133083959763);
		\draw [line width=2pt] (-5.22,-0.1)-- (-2.7102122357443212,-3.842404973453586);
		\draw [line width=2pt] (-3.9031597585752285,-2.0635721948040557) -- (-4.075949767564874,-2.045538117883112);
		\draw [line width=2pt] (-3.9031597585752285,-2.0635721948040557) -- (-3.8542624681794466,-1.896866855570475);
		\draw [line width=2pt] (-2.7102122357443212,-3.842404973453586)-- (0.1,-0.32);
		\draw [line width=2pt] (-1.1663832988869638,-1.907323111599976) -- (-1.1314170833034722,-2.0774964905545033);
		\draw [line width=2pt] (-1.1663832988869638,-1.907323111599976) -- (-1.3400723334556526,-1.9110291077722668);
		\draw [line width=2pt] (-1.305106117872161,-2.081202486726794) -- (-1.2701399022886695,-2.2513758656813216);
		\draw [line width=2pt] (-1.305106117872161,-2.081202486726794) -- (-1.4787951524408498,-2.084908482899085);
		\draw [line width=2pt] (-4.999873874303608,1.187555688983266)-- (-3.2110963093036573,-1.4797336137213115);
		\draw [line width=2pt] (-4.0435387325067005,-0.23845867044628558) -- (-4.216328741496346,-0.22042459352534186);
		\draw [line width=2pt] (-4.0435387325067005,-0.23845867044628558) -- (-3.994641442110919,-0.07175333121270593);
		\draw [line width=2pt] (-3.2110963093036573,-1.4797336137213115)-- (-1.2082000199121943,1.0307571461324934);
		\draw [line width=2pt] (-2.0709253456227286,-0.05060885866759117) -- (-2.035959130039237,-0.22078223762211852);
		\draw [line width=2pt] (-2.0709253456227286,-0.05060885866759117) -- (-2.2446143801914173,-0.05431485483988193);
		\draw [line width=2pt] (-2.2096481646079256,-0.22448823379440927) -- (-2.174681949024434,-0.39466161274893663);
		\draw [line width=2pt] (-2.2096481646079256,-0.22448823379440927) -- (-2.3833371991766144,-0.22819422996670002);
		\draw [shift={(-3.2110963093036586,-1.4797336137213097)},line width=2pt]  plot[domain=0.6413968964197572:3.7829895500095505,variable=\t]({1*3.211566191341761*cos(\t r)+0*3.211566191341761*sin(\t r)},{0*3.211566191341761*cos(\t r)+1*3.211566191341761*sin(\t r)});
		\draw [line width=2pt] (-2.5457750967368584,2.4118449507565143)-- (-3.2110963093036573,-1.4797336137213115);
		\draw [line width=2pt] (-4.999873874303608,1.187555688983266)-- (-1.2082000199121943,1.0307571461324934);
		\draw [line width=2pt,dash pattern=on 1pt off 1pt] (-4.453755836691207,4.381894316257579)-- (-2.7102122357443212,-3.842404973453586);
		\draw [line width=2pt] (-5.22,-0.1)-- (0.1,-0.32);
		\draw [line width=2pt] (-4.453755836691207,4.381894316257579)-- (-2.56,-0.21);
		\draw [line width=2pt] (-3.1040369471079012,1.1091564175578796)-- (-3.2110963093036573,-1.4797336137213115);
		\draw [line width=2pt] (-2.56,-0.21)-- (-2.7102122357443212,-3.842404973453586);
		\draw [line width=2pt] (-5.119077049258007,0.49031575177975495)-- (-3.2110963093036573,-1.4797336137213115);
		\draw [line width=2pt] (-4.087712061866089,-0.574600626286031) -- (-4.260956713659135,-0.5875584718684704);
		\draw [line width=2pt] (-4.087712061866089,-0.574600626286031) -- (-4.0692166449025295,-0.4018593900730861);
		\draw [line width=2pt] (-4.242461296695576,-0.41481723565552625) -- (-4.415705948488622,-0.4277750812379656);
		\draw [line width=2pt] (-4.242461296695576,-0.41481723565552625) -- (-4.223965879732017,-0.24207599944258126);
		\draw [line width=2pt] (-3.9329628270366017,-0.7343840169165368) -- (-4.106207478829648,-0.747341862498976);
		\draw [line width=2pt] (-3.9329628270366017,-0.7343840169165368) -- (-3.9144674100730423,-0.5616427807035917);
		\begin{scriptsize}
			\draw [fill=ududff] (-4.453755836691207,4.381894316257579) circle (2.5pt);
			\draw[color=ududff] (-4.328542734613386,4.696945995566035) node {$A$};
			\draw [fill=ududff] (-5.22,-0.1) circle (2.5pt);
			\draw[color=ududff] (-5.099656410496019,0.2185550317861183) node {$T$};
			\draw [fill=ududff] (0.1,-0.32) circle (2.5pt);
			\draw[color=ududff] (0.2239937749244697,-0.0038816054877186257) node {$S$};
			\draw [fill=uuuuuu] (-2.7102122357443212,-3.842404973453586) circle (2pt);
			\draw[color=uuuuuu] (-2.6973407279385837,-3.9780828581136047) node {$L$};
			\draw [fill=uuuuuu] (-3.2110963093036573,-1.4797336137213115) circle (2pt);
			\draw[color=uuuuuu] (-3.2163595482442027,-1.7092291579204684) node {$K$};
			\draw [fill=uuuuuu] (-3.1040369471079012,1.1091564175578796) circle (2pt);
			\draw[color=uuuuuu] (-2.979093801818777,1.4048837639132485) node {$M$};
			\draw [fill=uuuuuu] (-4.999873874303608,1.187555688983266) circle (2pt);
			\draw[color=uuuuuu] (-4.8772197732221825,1.4790293096711942) node {$F$};
			\draw [fill=uuuuuu] (-1.2082000199121943,1.0307571461324934) circle (2pt);
			\draw[color=uuuuuu] (-1.0957969395669607,1.3159091090037138) node {$E$};
			\draw [fill=uuuuuu] (-5.7843982618248075,-3.4012628126980693) circle (2pt);
			\draw[color=uuuuuu] (-5.826282758923885,-3.5925260201722877) node {$F_1$};
			\draw [fill=uuuuuu] (-0.6377943567825096,0.44179558525544976) circle (2pt);
			\draw[color=uuuuuu] (-0.4729743552002182,0.7820611795465052) node {$E_1$};
			\draw [fill=uuuuuu] (-2.5457750967368584,2.4118449507565143) circle (2pt);
			\draw[color=uuuuuu] (-2.3859294357552128,2.7395035875562703) node {$E_2$};
			\draw [fill=uuuuuu] (-5.119077049258007,0.49031575177975495) circle (2pt);
			\draw[color=uuuuuu] (-4.951365318980128,0.8265485070012726) node {$F_2$};
			\draw [fill=uuuuuu] (-2.56,-0.21) circle (2pt);
			\draw[color=uuuuuu] (-2.445245872361569,0.08509304942181614) node {$N$};
		\end{scriptsize}
	\end{tikzpicture}
	
	We can now do our calculations. Observe that 
	\[
	\frac{AF_2}{AB} + \frac{AE_2}{AC} = \frac{CK}{BC} + \frac{BK}{BC} = 1
	\]
	and therefore, 
	\[
	\frac{AE}{AB} + \frac{AF}{AC}
	 = \frac{AF_1\cos\angle TAS}{AB} + \frac{AF_2\cos\angle TAS}{AC}
	 = 2\cos\angle(TAS)(\frac{AF_2}{AB} + \frac{AE_2}{AC})
	 =2\cos\angle(BAC)
	\]
	as desired. 
	%%%G7%%%
	\item[\textbf{G7}] Let $ABC$ be a triangle with circumcircle $\Omega$ and incentre $I$. Let the line passing through $I$ and perpendicular to $CI$ intersect the segment $BC$ and the arc $BC$ (not containing $A$) of $\Omega$ at points $U$ and $V$ , respectively. Let the line passing through $U$ and parallel to $AI$ intersect $AV$ at $X$, and let the line passing through $V$ and parallel to $AI$ intersect $AB$ at $Y$ . Let $W$ and $Z$ be the midpoints of $AX$ and $BC$, respectively. Prove that if the points $I, X,$ and $Y$ are collinear, then the points $I, W ,$ and $Z$ are also collinear.
	
	\textbf{Solution.} We assume that $I, X, Y$ is collinear. Then $X$ is the intersection of lines $IY$ and $AV$. We also have $VY\parallel AI$ and therefore if lines $AB$ and $IV$ intersect at $P$ we get 
	\[
	\frac{PV}{PI} = \frac{YV}{AI} = \frac{VX}{XA} = \frac{VU}{UI}
	\]
	where the first and second equality are due to $VY\parallel AI$, the third due to $UX \parallel AI$. Therefore, $(P, V; U, I)$ is a harmonic bundle, and so is the set of lines $(BA, BV; BC, BI)$ (we recognize that $BP$ is $BA$ and $BU$ is $BC$), which then gives 
	\[
	\frac{\sin\angle ABV}{\sin\angle ABI} = \frac{\sin\angle VBC}{\sin\angle CBI}
	\]
	Now, let $BI$ intersect $\Omega$ again at $M$, which is then the midpoint of arc $AC$ not containing $B$. The equality above translates into 
	\[
	\frac{AV}{AM}=\frac{VC}{CM}
	\]
	by considering the angle subtended by each chord of $\Omega$. Since $AM=CM$, we have $VA=VC$. 
	
	\definecolor{uuuuuu}{rgb}{0.26666666666666666,0.26666666666666666,0.26666666666666666}
	\definecolor{xdxdff}{rgb}{0.49019607843137253,0.49019607843137253,1}
	\definecolor{ududff}{rgb}{0.30196078431372547,0.30196078431372547,1}
	\begin{tikzpicture}[line cap=round,line join=round,>=triangle 45,x=1cm,y=1cm]
		\clip(-10.207022992333032,-8.241114601363567) rectangle (9.962403788475998,5.9081600794701385);
		\draw [line width=2pt] (-3.863628994462391,0.29940008165229465) circle (3.8283598084586115cm);
		\draw [line width=2pt] (-4.107257988924784,-3.521199836695412) circle (3.9614839759975045cm);
		\draw [shift={(-2.1068244469700144,1.2142643365728536)},line width=2pt]  plot[domain=2.0509095134795166:5.19250216706931,variable=\t]({1*3.2761257607118814*cos(\t r)+0*3.2761257607118814*sin(\t r)},{0*3.2761257607118814*cos(\t r)+1*3.2761257607118814*sin(\t r)});
		\draw [line width=2pt] (-3.62,4.12)-- (-7.36,-1.26);
		\draw [line width=2pt] (-3.62,4.12)-- (-0.5936488939400282,-1.691471326854293);
		\draw [line width=2pt] (-6.169078002858015,3.355744907209268)-- (-7.36,-1.26);
		\draw [line width=2pt] (-6.169078002858015,3.355744907209268)-- (-0.5936488939400282,-1.691471326854293);
		\draw [line width=2pt] (-6.169078002858015,3.355744907209268)-- (-5.2449424461670855,0.27340546133418475);
		\draw [line width=2pt] (-7.36,-1.26)-- (-0.5936488939400282,-1.691471326854293);
		\draw [line width=2pt] (-4.558687290865145,1.8979216641956065)-- (-5.78050142336836,1.0121129257428467);
		\draw [line width=2pt] (-5.2860113240136455,1.3706156140276593) -- (-5.270876374246633,1.5947176552454978);
		\draw [line width=2pt] (-5.2860113240136455,1.3706156140276593) -- (-5.068312339986873,1.315316934692955);
		\draw [line width=2pt] (-5.2449424461670855,0.27340546133418475)-- (-7.36,-1.26);
		\draw [line width=2pt] (-6.418888189980436,-0.5776989502744748) -- (-6.403753240213424,-0.3535969090566363);
		\draw [line width=2pt] (-6.418888189980436,-0.5776989502744748) -- (-6.201189205953662,-0.6329976296091793);
		\draw [line width=2pt] (-3.62,4.12)-- (-6.5130604453641565,2.0225465860955194);
		\draw [line width=2pt] (-5.182947189578972,2.986871612106193) -- (-5.167812239811959,3.2109736533240314);
		\draw [line width=2pt] (-5.182947189578972,2.986871612106193) -- (-4.965248205552197,2.9315729327714886);
		\draw [line width=2pt] (-5.2449424461670855,0.27340546133418475)-- (-0.5936488939400282,-1.691471326854293);
		\draw [line width=2pt,dash pattern=on 1pt off 1pt] (-6.57025071168418,-0.12394353712857664)-- (-2.1068244469700144,1.2142643365728536);
		\draw [line width=2pt,dash pattern=on 1pt off 1pt] (-4.063065377657387,0.6277514513934896) -- (-4.151246847489432,0.42117260455598854);
		\draw [line width=2pt,dash pattern=on 1pt off 1pt] (-4.063065377657387,0.6277514513934896) -- (-4.250356109495052,0.7517392465596398);
		\draw [line width=2pt,dash pattern=on 1pt off 1pt] (-4.338537579327097,0.5451603997221386) -- (-4.426719049159142,0.3385815528846375);
		\draw [line width=2pt,dash pattern=on 1pt off 1pt] (-4.338537579327097,0.5451603997221386) -- (-4.525828311164762,0.6691481948882888);
		\draw [line width=2pt] (-3.62,4.12)-- (-6.169078002858015,3.355744907209268);
		\draw [line width=2pt] (-5.170011203098718,3.6552814019332827) -- (-5.081829733266674,3.8618602487707845);
		\draw [line width=2pt] (-5.170011203098718,3.6552814019332827) -- (-4.982720471261051,3.5312936067671328);
		\draw [line width=2pt] (-4.894539001429007,3.737872453604634) -- (-4.806357531596964,3.944451300442136);
		\draw [line width=2pt] (-4.894539001429007,3.737872453604634) -- (-4.707248269591341,3.6138846584384843);
		\draw [line width=2pt] (-3.62,4.12)-- (-5.2449424461670855,0.27340546133418475);
		\begin{scriptsize}
			\draw [fill=ududff] (-3.62,4.12) circle (2.5pt);
			\draw[color=ududff] (-3.458317453507197,4.537329266896142) node {$V$};
			\draw [fill=ududff] (-7.36,-1.26) circle (2.5pt);
			\draw[color=ududff] (-7.752948250941819,-1.137718572571036) node {$A$};
			\draw [fill=xdxdff] (-0.5936488939400282,-1.691471326854293) circle (2.5pt);
			\draw[color=xdxdff] (-0.23734435543123064,-1.578685127664769) node {$C$};
			\draw [fill=uuuuuu] (-5.2449424461670855,0.27340546133418475) circle (2pt);
			\draw[color=uuuuuu] (-5.279701050633488,0.031801421373213484) node {$I$};
			\draw [fill=uuuuuu] (-6.169078002858015,3.355744907209268) circle (2pt);
			\draw[color=uuuuuu] (-6.276668914323668,3.7704309102113878) node {$B$};
			\draw [fill=uuuuuu] (-4.558687290865145,1.8979216641956065) circle (2pt);
			\draw[color=uuuuuu] (-4.24438826910907,2.064082066587811) node {$U$};
			\draw [fill=uuuuuu] (-5.78050142336836,1.0121129257428467) circle (2pt);
			\draw[color=uuuuuu] (-6.046599407318242,1.2013214153174627) node {$X$};
			\draw [fill=uuuuuu] (-6.5130604453641565,2.0225465860955194) circle (2pt);
			\draw[color=uuuuuu] (-6.832670222920115,2.1215994433391674) node {$Y$};
			\draw [fill=uuuuuu] (-3.3813634483990214,0.8321367901774877) circle (2pt);
			\draw[color=uuuuuu] (-3.2282479465017713,1.2013214153174627) node {$Z$};
			\draw [fill=uuuuuu] (-6.57025071168418,-0.12394353712857664) circle (2pt);
			\draw[color=uuuuuu] (-6.755980387251639,0.26187092837863957) node {$W$};
			\draw [fill=uuuuuu] (-2.1068244469700144,1.2142643365728536) circle (2pt);
			\draw[color=uuuuuu] (-1.962865657971927,1.5847705936598397) node {$N$};
		\end{scriptsize}
	\end{tikzpicture}
	
	Now that $BI$ intersects $\Omega$ again at $M$, we have $MA=MI=MC$. If $N$ were to be the midpoint of $VC$ then from $\angle VIC=90^{\circ}$, $I$ is simply the second intersection (other than $C$) of the circles with diameter $VC$ (and center $N$), and circle with center $M$ and radius $MA=MC$. 
	
	We first show thst $Z, I, N$ are collinear. Notice that $VM$ is the diameter of $\Omega$, so $\angle VBM=90^{\circ}$, so $VB$ and $NI$ are both perpendicular to $IM$ (and hence parallel to each other). 
	From $VN=NC$ we have $NI$ bisects the segment $BC$, hence passes throguh $Z$, the midpoint of $BC$, as desired. 
	
	Thus it suffices to show that $N, I, W$ are collinear. Indeed, let $W'$ be the intersection of lines $NI$ and $AV$. From $NI$ perpendicular to $IM$, and $\angle VAM=\angle VCM$ we have $VA, VC, NI$ tangent to the circle $IAC$, which means this circle $IAC$ is an excircle of $VW'N$. We also notice that (since $W'AMI$ is cyclic). 
	\[
	\angle W'AM = \angle W'IA=\angle W'AI=\angle VAI = \angle ICA
	\]
	the last equality is due to that $VA$ tangent to circle $IAC$. Therefore we have 
	\[
	\frac{W'A}{VA}=\frac{W'A/AM}{VA/AM}=\frac{\tan\angle W'MA}{\tan\angle VMA}=\frac{\tan\angle ICA}{\tan\angle VMA}=\frac{\tan\angle ICA}{\tan\angle VMC}
	\]
	meanwhile we also have $\angle UCI=\angle BCI=\angle ICA$ and $\angle VCI=\angle NCI=\angle NMI=\angle NMC$ (we have $NI=NC, IM=MC$ and $\angle NIM=\angle NCM=90^{\circ}$). 
	Combined with $UX\parallel IA$, we have 
	\[
	\frac{AX}{AV}=\frac{UI}{VI}=\frac{UI/IC}{VI/IC} = \frac{\tan\angle UCI}{\tan\angle VCI} = \frac{\tan\angle ICA}{\tan\angle NMC}
	\]
	Therefore, we have 
	\[
	\frac{AW'}{AX}=\frac{\tan\angle ICA}{\tan\angle VMA}\div \frac{\tan\angle ICA}{\tan\angle NMC} = \frac{\tan\angle NMC}{\tan\angle VMC}=\frac{NC/MC}{VC/MC}=\frac{NC}{VC}=\frac 12
	\]
	which then $W'$ is the midpoint of $AX$. This gives $W=W'$, as desired. 
	
\end{enumerate}

\section*{Number Theory}
\begin{enumerate}
	\item [\textbf{N1}] Let $n \ge 2$ be an integer, and let $A_n$ be the set \[A_n = \{2^n  - 2^k\mid k \in \mathbb{Z},\, 0 \le k < n\}.\] Determine the largest positive integer that cannot be written as the sum of one or more (not necessarily distinct) elements of $A_n$ .
	
	\textbf{Answer.} $(n-2)2^n+1$. 
	
	\textbf{Solution.} Let $f(n)$ be the answer w.r.t. $n$. We show that $f(n)=2f(n-1)+2^n-1$ for all $n$, and that $f(2)=1$. This way, the general formula can be obtained using induction since
	\[
	f(n)=2f(n-1)+2^n-1 + (n-3)2(2^{n-1})+2^n-1 = (n-2)2^n+1
	\]
	To show that $f(2)=1$, we have $A_2=\{2, 3\}$ so 1 cannot be represented while any even number $k$ can be represented as $\frac{k}{2}\times 2$ and odd number $k\ge 3$ can be represented as $\frac{k-3}{2}\times 2 + 3$. 
	
	Now, we first show that $(n-2)2^n+1$ cannot be represented in the form above. Consider $a_1, a_2, \cdots , a_k$ such that 
	\[
	(n-2)2^n+1 = \dsum_{i=1}^k (2^n-2^{a_i}) = k2^n-(\dsum_{i=1}^k 2^{a_i})
	\]
	We have $\dsum_{i=1}^k 2^{a_i}\equiv -1\pmod{2^n}$ and $i< n$. 
	Let's proceed to the following lemma: 
	
	\emph{Lemma}. Let $C_n(x)$ be the number of ones in the $n$ rightmost positions of the binary representation of $x$, and let $x=\dsum_{i=2}^k 2^{a_i}$, with $0\le a_i \le n$. Then the minimal such $k$ is $ \lfloor \frac{x}{2^{n}}\rfloor + C_{n}(x-2^{n}\lfloor \frac{x}{2^{n}}\rfloor)$ (notice $x-2^{n}\lfloor \frac{x}{2^{n}}\rfloor$ is the remainder of $x$ when divided by $2^{n}$). Denote $g_n(x)$ as the minimal such $k$ for each $n$ and $x$. 
	
	Proof: we do induction on both $n$ and $x$. For base case $n=0$, we have $C_0(x)=0$ for all $x$ and since adding 1 is the only choice, $g_0(x)=x$ for all $x$. Now consider the task of finding $g_n(x)$ and suppose $g_m(y)$ holds true for all $m\le n$, $y\le x$ with $(m, y)\neq (n, x)$. Notice the fact that $g_m(y+1)\le g_m(y)+1$ (given we can simply add 1 to go from $y$ to $y+1$). 
	\begin{itemize}
		\item If $x$ is even, then using exactly $2k$ ones means $g_n(x)=\min\{2k+g_{n-1}(x)(\frac{x-2k}{2})\}$. By above, $g_{n-1}(\frac{x}{2})\le g_{n-1}(x)(\frac{x-2k}{2})+k$ so $2k+g_{n-1}(\frac{x}{2})(\frac{x-2k}{2})\ge k+g_{n-1}(x)$. It therefore follows that the optimal answer is when $k$ is 0, so $g_{n}(x)=g_{n-1}(\frac{x}{2})$. We notice that: 
		\[
		\frac{x}{2^{n}} = \frac{x/2}{x^{n-1}}\qquad C_{n}(x-\lfloor\frac{x}{2^{n}}\rfloor)=C_{n-1}(x/2-\lfloor\frac{x/2}{2^{n-1}}\rfloor)
		\]
		(the equality by $C_n$ is simply because the rightmost $n$ digits of binary representation of $x$ is simply that of rightmost $n-1$ of $x/2$, and append a 0). 
		so the conclusion follows. 
		
		\item If $x$ is odd, then similarly we have $g_n(x)=1+g_{n-1}(\frac{x-1}{2})$, and we have 
		\[
		\lfloor\frac{x}{2^{n}}\rfloor = \lfloor\frac{(x-1)/2}{x^{n-1}}\qquad\rfloor C_{n}(x-\lfloor\frac{x}{2^{n}}\rfloor)=C_{n-1}((x-1)/2-\lfloor\frac{(x-1)/2}{2^{n-1}}\rfloor)+1
		\]
		(the equality by $C_n$ is simply because the rightmost $n$ digits of binary representation of $x$ is simply that of rightmost $n-1$ of $(x-1)/2$, and append a 1). 
	\end{itemize}
	
	The conclusion then follows. 
	
	Going back to the equation, we have $(n-2)2^n+1 = k2^n-(\dsum_{i=1}^k 2^{a_i})$ with $(\dsum_{i=1}^k 2^{a_i})=c\cdot 2^n-1$ for some $c\ge 1$. By the lemma above, we need $k\ge n + (c-1)$ so we have 
	\[
	(n-2)2^n+1=
	k2^n-(\dsum_{i=1}^k 2^{a_i})
	\ge (n+c-1)2^n - c\cdot 2^n + 1
	=(n-1)2^n+1
	\]
	which is a contradiction. 
	
	Now we show that any number bigger than $f(n)=(n-2)2^n+1$ can be represented in the required form. If $x$ is even, then it suffices to show that $\frac{x}{2}$ can be representation as sum of $A_{n-1}$ (and then double each summand to get the required form in $A_n$). We now have 
	\[
	\frac{x}{2}\ge (n-2)2^{n-1}+1>(n-1)2^{n-1}+1
	\]
	as desired. For $x$ odd, we show the same thing for $\frac{x-(2^n-1)}{2}$. Now 
	\[
	\frac{x}{2}> \frac{(n-2)2^{n}+1-(2^n-1)}{2}=(n-3)2^{n-1}+1=f_{n-1}(x)
	\]
	therefore proving out conclusion. 

	
	\item [\textbf{N2}] Determine all pairs $(x, y)$ of positive integers such that \[\sqrt[3]{7x^2-13xy+7y^2}=|x-y|+1.\]
	
	\textbf{Answer.} $\{x, y\}=\{1, 1\}$ and $\{c^3+c^2-2c-1, c^3+2c^2-c-1\}$. \\
	\textbf{Solution.} By symmetry we can assume that $y\ge x$, and let $k$ be such that $y=x+k$. Also, $7x^2-13xy+7y^2=7(x^2-2xy+y^2)+xy=7(x-y)^2+xy=7k^2+x(x+k)$. Therefore we now have 
	\[(k+1)^3 = 7k^2+x(x+k)
	\]
	and notice that this is actually a quadratic equation in terms of $x$, i.e. $x^2+kx+(7k^2-(k+1)^3)=0$. Since we are finding for integer solutions when $x$ and $k$ are both integers, we need the discriminant $k^2-4(7k^2-(k+1)^3)$ to be a perfect square. But notice the following: 
	\[
	k^2-4(7k^2-(k+1)^3)=4k^3-15k^2+12k+4=(k-2)(4k^2-7k-2)=(k-2)^2(4k+1)
	\]
	so either $k=2$ or $4k+1$ is a perfect square. Since $4k+1$ is odd, we have $4k+1=(2c+1)^2$ for some nonnegative integer $c$, leaving $k=c(c+1)$. Hence the following equation obtained by solving the squadratic equation: 
	\[
	x=\frac{-k\pm \sqrt{(k-2)^2(4k+1)}}{2}
	=\frac{-c(c+1)\pm |c(c+1)-2|\cdot (2c+1)}{2}
	\]
	The product of roots is given by $7k^2-(k+1)^3$, which is positive only when $k=2$ and negative for the rest of the cases. For $k=2$ the discriminant turned out to be 0, so we have $x^2+2x+1=0$, or $(x+1)^2=0$, so $x=-1$ which is impossible because $x$ must be positive. Otherwise, there would be one positive and one negative root in the quadratic equation, and therefore we choose the bigger root given by the plus sign, yielding: 
	\[
	x=\frac{-c(c+1)+ |c(c+1)-2|\cdot (2c+1)}{2}\]
	and
	\[
	y=x+k = \frac{-c(c+1)+ |c(c+1)-2|\cdot (2c+1)}{2} + c(c+1) 
	=\frac{c(c+1)+ |c(c+1)-2|\cdot (2c+1)}{2}
	\]
	for all $c=0, 2, 3, 4, 5, \cdots $. ($c=1$ corresponds to $k=2$). Notice for $c=0$ with $k=0$ we actually have $1=x^2$ so $x=y=1$. Otherwise we have $|c(c+1)-2|>0$ so 
	we can remove the modulus to get 
	\[
	x=\frac{-c(c+1)+ (c(c+1)-2)\cdot (2c+1)}{2}=c^3+c^2-2c-1
	\]
	and similarly $y=c^3+2c^2-c-1$. 
	
	\item [\textbf{N3}] (IMO 5) For each positive integer $n$, the Bank of Cape Town issues coins of denomination $\frac1n$. Given a finite collection of such coins (of not necessarily different denominations) with total value at most $99+\frac12$, prove that it is possible to split this collection into $100$ or fewer groups, such that each group has total value at most $1$.
	
	\textbf{Solution.} In fact, we shall prove that for any pile of coins worth at most $n-\frac 12$ we can split them into $n$ groups, each of total value at most 1. 
	
	The key to this problem is to `group' certain coins according to the following algorithm, in the hope to minimize the number of cape town coins and their denominators. To see this, we perform the following operations iteratively: for each $a>1$ and positive integer $b$, if there are $a$ coins of value $\frac {1}{ab}$, then we group them together into a single coin of value $\frac {1}{b}$ (notice that the total value never changes). Since the number of coins decrease each iterations and there are only a finite number fo coins, such operation can only be done finitely many times, hence it must terminate. This means, at the end of the process, for each positive integer $b$ and a prime number $p$ (or simply any $p>1$) dividing $b$, the number of coins of value $\frac{1}{b}$ must be less than $p$. In particular, if $b$ is even, choosing $p=2$ means there is at most one such coin. 
	
	Next, let $m$ be the number of coins with value 1. We place these coins into $m$ separate piles, leaving coins of total value at most $n-m-\frac 12$. Denote $n_1=n_1-\frac 12$ and consider the coins of values $\frac{1}{k}$ for $k=2, 3, \cdots , 2n_1$. 
	We have seen that if $k$ is even then there is at most 1 coin of value $\frac 1k$, and if $k$ is odd, there is at most $k-1$ of them. Consider, also, the fact that $\frac{2i-2}{2i-1}+\frac{1}{2i}<1$ for all $i\ge 2$, we can group all coins of value $\frac{1}{2i-1}$ and $\frac{1}{2i}$ into one pile, for all $i=2, 3, \cdots , n_1$ (i.e. $n_1-1$ of them), and place the coin (if exists; otherwise we just have an empty pile) with value $\frac 12$ into the last pile, making a total of $n_1$ piles right now. Now all piles have value at most 1. 
	
	We are not done yet: there might still be coins of value $\frac{1}{k}$ for $k>2n_1$. However, the coins in the piles have total value at most $n_1-\frac 12$, i.e. an average (w.r.t. pile) of at most $1-\frac{1}{2n_1}$. 
	By pigeonhole principle, there must be a pile of value at most $1-\frac{1}{2n_1}$. Since each leftover coin has value less than $\frac{1}{2n_1}$, we can choose any of the leftover coin and put into this pile of value at most $1-\frac{1}{2n_1}$, and after that this pile's value cannot exceed 1. This invariant holds as long as there is a coin not in any pile yet, so we can repeat this argument, find a pile of value at most $1-\frac{1}{2n_1}$ and put a coin into the pile. Eventually, all coins are in the piles. Q.E.D. 
	
	\item[\textbf{N4}] Let $n > 1$ be a given integer. Prove that infinitely many terms of the sequence $(a_k )_{k\ge 1}$, defined by \[a_k=\left\lfloor\frac{n^k}{k}\right\rfloor,\] are odd. (For a real number $x$, $\lfloor x\rfloor$ denotes the largest integer not exceeding $x$.)
	
	\textbf{Solution.} We settle the easier case first: $n$ is odd. Now for each $k=n^m$, we have $\frac{n^k}{k}=n^{k-m}=n^{n^m-m}$ and for sufficiently large $m$, $n^m\ge m$ ($n\ge 2$ and therefore it's well-known that $n^m\in \Omega(m)$ for each $n>1$ fixed). This means that $k\mid n^k$ and therefore in this case $a_k=\frac{n^k}{k}$. Since $n$ is odd, so is $n^k$ and therefore $\frac{n^k}{k}$ is also odd. 
	
	Now let's see the even case, and we show that there are infinitely many $m$'s such that $k=n^m(n+1)$ will work. To see how this works, consider the following: 
	\[
	\frac{n^k}{k} = \frac{n^{n^m(n+1)}}{n^m(n+1)}
	=\frac{n^{n^m(n+1)-m}}{n+1}
	\]
	we now proceed to a lemma: for $n\ge 2$ even, $\left\lfloor\frac{n^k}{n+1}\right\rfloor$ is even for all $k\ge 1$ odd, and even otherwise. To see why, notice that $\left\lfloor\frac{n^k}{n+1}\right\rfloor$ is even iff $n^k$ is congruent to $0, 1, \cdots , n$ modulo $2n+2$, and odd if congruent to $n+1, \cdots , 2n+1$ modulo $2n+2$. 
	When $k=1$ the congruence is $n$; when $k=2$ we have $n^2=(n+1)(n-1)+1=(n+1)(n-2)+n+2$ and since $n-2$ is even, we have $2n+2\mid (n+1)(n-2)$ and therefore $n^2\equiv n+2\pmod{2n+2}$. 
	Finally, when $k=3$ we have \[n^3\equiv (n^2)\cdot n\equiv (n+2)n\equiv n^2+2n\equiv n+2+2n\equiv 3n+2\equiv n\pmod{2n+2}\]
	so the congruence alternates between $n$ and $n+2$ when $k$ is even or odd, completing the proof for our lemma. 
	
	To finish the proof, we need to find those (infinitely many) $m$ such that $n^m(n+1)-m$ is even. Since $n$ is even, $n^m(n+1)$ is even for all $m>0$, which reduces to finding $m$ is even. Thus all $m\ge 2$ even works, Q.E.D. 
	
	\item[\textbf{N5}] Find all triples $(p, x, y)$ consisting of a prime number $p$ and two positive integers $x$ and $y$ such that $x^{p -1} + y$ and $x + y^ {p -1}$ are both powers of $p$.
	
	\textbf{Answer.} Any triples in the form $(2, x, 2^k-x)$ provided $2^k>x$, and $(3, 2, 5)$ and $(3, 5, 2)$. \\
	\textbf{Solution.} When $p=2$, we are only asked to find the pairs where $x+y$ is a power of 2 which is easily settled above. From now on we focus only on odd primes $p$. 
	
	First, we show that $x$ and $y$ cannot be simultanouesly divisible by $p$. Otherwise, let $v_p(x)\le v_p(y)$, then $v_p(y^{p-1})=(p-1)v_p(y)>v_p(y)\ge v_p(x)$ and therefore $v_p(y^{p-1}+x)=v_p(x)$. If $x=cp^k$ where $k=v_p(x)$ then $y^{p-1}+x=dp^k$ too, with $p\nmid d$. Since this $y^{p-1}+x$ must be a $p$-th power, we have $d=1$, but $dp^k>cp^k$ so $1=d>c\ge 1$, contradiction. 
	
	Therefore we have $p$ not dividing $x$ and $y$. Now let $x\le y$ and let $x^{p-1}+y=p^k$. THen $p^k\mid y^{p-1}+x$, too. We now have $y\equiv -x^{p-1}\pmod{p^k}$ and $0\equiv y^{p-1}+x\equiv (-x^{p-1})^{p-1}+x\equiv x^{(p-1)^2}+x=x(x^{(p-1)^2-1}+1)\pmod{p^k}$. 
	Since $p\nmid x$, we have $p^k\mid x^{(p-1)^2-1}+1$, i.e. $x^{(p-1)^2-1}\equiv -1\pmod{p^k}$. 
	
	Now by Fermat's Little theorem, $1\equiv x^{(p-1)^2}\equiv x\cdot(-1)=-x\pmod{p}$ so $p\mid x+1$. We now let $x=cp^{\ell}-1$ with $p\nmid c$ and $\ell\ge 1$. We now consider the expansion $(cp^{\ell}-1)^{(p-1)^2-1}+1$, with the following observation: 
	\begin{itemize}
		\item The expansion has the form $\dsum_{i=1}^{p(p-2)}\dbinom{p(p-2)}{i}(-1)^{p(p-2)-i}(cp^{\ell})^i$, with $i=0$ ommited since the $-1$ term is offset by the $+1$ term at $(cp^{\ell}-1)^{(p-1)^2-1}+1$. 
		
		\item Since $\dbinom{p(p-2)}{i}$ is divisible by $p$ for all $p\nmid i$ (well-known), this is also true for $i=2$, and since $\ell\ge 1$, $p^{2\ell+1}\mid \dsum_{i=2}^{p(p-2)}\dbinom{p(p-2)}{i}(-1)^{p(p-2)-i}(cp^{\ell})^i$. 
		
		\item If we consider $i=1$ we notice that the term is actually $p(p-2)cp^{\ell}$, with $v_p(p(p-2)cp^{\ell})=\ell+1$. 
	\end{itemize}
	so these points are enough to show that the highest power of $p$ dividing $x^{(p-1)^2-1}+1$ is $\ell+1$. But since this term is also divisible by $p^k$, we have $\ell\ge k-1$. 
	
	In other words, we have $x\ge p^{k-1}-1$ but then $x^{p-1}+y=p^k$ so $x^{p-1}<p^k$, i.e. $(p^{k-1}-1)^{p-1}<p^k$. Since $p\mid x+1$ anyways, we can safely assume that $(p^{k-1}-1)^{p-1}\ge (p-1)^{p-1}$. For $p\ge 5$, $(p-1)^{p-1}>p^2$ so we have $k\ge 3$, but then $(p^{k-2})^{p-1} < (p^{k-1}-1)^{p-1}<p^k$ so $(k-2)(p-1)<k$, which is impossible. Hence we only consider $p=3$, i.e. $(3^{k-1}-1)^2<3^k$. This holds true for $k=2$, but not any $k=3$, and for $k\ge 4$ we can reuse the result $(k-2)(p-1)<k$ i.e. $2(k-2)<k$ to produce a contradiction. 
	Therefore $k=2$, and $x^2+y=3^2=9$. Since $p\mid x+1$ with $x^2<9$, the only choice is $x=2$ and $y=5$, and it turned out that $5^2+2=27=3^3$ works too. 
	
	\item[\textbf{N6}] Let $a_1 < a_2 <  \cdots <a_n$ be pairwise coprime positive integers with $a_1$ being prime and $a_1 \ge n + 2$. On the segment $I = [0, a_1 a_2  \cdots a_n ]$ of the real line, mark all integers that are divisible by at least one of the numbers $a_1 ,   \ldots , a_n$ . These points split $I$ into a number of smaller segments. Prove that the sum of the squares of the lengths of these segments is divisible by $a_1$.
	
	\textbf{Solution.} Might as well give an alias for $a_1$, say $p$. 
	It's not hard to see that each segment must have length at most $p$. We show by fixing $p$ and inducting on $n$ that, if $x_1, \cdots , x_{p-1}$ represents the segments with lengths $1, 2, \cdots , p-1$, then there exists a polynomial $P(x)$ such that: 
	\begin{itemize}
		\item $P(i)=x_i$, $\forall i=1, \cdots , p-1$. 
		\item $P$ has coefficient that are rational numbers such that, when written in simplest form, the denominator is not divisible by $p$. 
		\item $P$ has degree at most $n-2$. 
	\end{itemize}
	We proceed by induction on $n$. When $n=1$ there's nothing to prove: we have a single segment of length $p$, so $x_1=\cdots =x_{p-1}=0$, which actually fits into the zero polynomial: the polynomial with degree $-1$, by our convention here. 
	
	Now suppose that for some $n$, the number of segments of length $p$ is $F(n)$ and let $P_n$ be a polynomial (satisfying our aforementioned conditions) such that for each $1\le i\le p-1$, the number of segments of length $i$ is $F(i)$. Now consider what happens when we add $a_{i+1}$. We have: 
	\begin{itemize}
		\item The segment extends from length $a_1\cdots a_n$ to $a_1\cdots a_{n+1}$. 
		
		\item If we do not mark the points divisible by $a_{n+1}$, we have $F(n)$ segments of length $p$ and $a_{n+1}P_n(i)$ segments of length $i$. 
	\end{itemize}
	Let's see what happens when we mark points divisible by $a_{n+1}$, which might or might not split a segment into two (it cannot split the a segment into three since $a_{n+1}>p$). 
	In the original configuration (where we have length of $I$ as $a_1\cdots a_n$), take $[x, x+i]$ as any segment of length $i$. 
	Consider, now, the $a_{n+1}$ copies of it: $[ja_{n+1}+x, ja_{n+1}+x+i]$ for $j=0, \cdots , a_{n+1}$. 
	For each $k=1, 2, \dots , i-1$, among the numbers $ja_{n+1}+x+k, j=0, \cdots , a_{n+1}$, from the fact that each $a_i$'s are pairwise coprime, exactly one of them is divisible by $a_{i+1}$. This means, $i-1$ of the segments are divided further into segments of length $(1, i-1), (2, i-2), \cdots , (i-1, 1)$, while the rest $a_{n+1}-i+1$ of them remain undivided. 
	
	Now for each $i$, we shall see how many segments are there have length $i$. Those must correspond to segments of length $j\ge i$ when considering only $a_1, \cdots, a_n$ and not $a_{n+1}$ since considering $a_{n+1}$ only splits them up (potentially). 
	By the fact above, the number of segments of length $i$ can be given by 
	\[P_n(i)(a_{n+1}-i+1) + \dsum_{j=i+1}^{p-1}2P_n(j) + 2F(n)
	\]
	We now consider 2 cases. If $n=1$ then $P_n\equiv 0$ so each term is $2F(n)$, i.e. a constant. 
	Otherwise, $P_n$ is assumed to have degree at most $n-2$. We now consider each term separately: 
	\begin{itemize}
		\item $2F(n)$ is constant throughout $i$, and is an integer (hence constant polynomial integer coefficient). 
		\item $P_n(i)(a_{n+1}-i+1) = (a_{n+1}+1)P_n(i) - iP_n(i)$. $(a_{n+1}+1)$ is constant while $\deg(iP_n(i)) = \deg(P_n)+1$ so by induction hypothesis, this $(a_{n+1}+1)P_n(i) - iP_n(i)$. $(a_{n+1}+1)$ has degree at most $n-1$. Moreover the polynomial of rational coefficient with denominator not divisible by $p$ is closed under multiplication and addition: the set $\{a/b: a\in\bbZ, \gcd(b, p)=1\}$ is a ring. 
	\end{itemize}
	We leave the middle term out deliberately: let $Q=\dsum_{j=1}^{p-1}2P_n(j)$, then $\dsum_{j=i+1}^{p-1}2P_n(j) = Q - \dsum_{j=1}^{i}2P_n(j)$. 
	Since $Q$ is a constant, we only need to consider the last term $\dsum_{j=1}^{i}P_n(j)$, which brings us to a lemma which directly addresses this subproblem: 
	
	\emph{Lemma}: let $P$ be a polynomial of degree $k\le p - 1$ with rational coefficients in terms of $\{a/b: a\in\bbZ, \gcd(b, p)=1\}$. Then if $Q(n)=\dsum_{i=1}^n P(i)$, $Q$ also has the same cclass of coefficients but with degree $k+1$. 
	
	Proof: Write $P(n) = \dsum_{i=0}^k a_ix^{k}$. Now, we need to solve for $Q=\dsum_{i=0}^{k+1}q_ix^i$ such that $Q(n)-Q(n-1) = P(n)$. Nevertheless, we also have the following: 
	\begin{flalign*}
		Q(n) - Q(n-1)
		&= \dsum_{i=0}^{k+1}q_i(n^i - (n-1)^i)
		\\&= \dsum_{i=0}^{k+1}q_i \dsum_{j=0^{i-1}}(-1)^{i-j-1}\dbinom{i}{j}n^j
	\end{flalign*}
	where each term $n^i-(n-1)^i$ is a polynomial of degree $i-1$ and leading coefficient $i$. 
	Putting this into the matrix form, this is what we need to solve: 
	\[
	\begin{pmatrix}
		k & -\dbinom{k}{2} & \cdots & (-1)^{k-1}\\
		0 & (k-1) & \cdots & (-1)^{k-2}\\
		\vdots\\
		0 & 0 & \cdots & 1\\
	\end{pmatrix}
	\begin{pmatrix}
		q_{k+1} \\ q_{k} \\ \vdots \\ q_1\\
	\end{pmatrix}
	=
	\begin{pmatrix}
	a_{k} \\ a_{k-1} \\ \vdots \\ a_0\\
	\end{pmatrix}
	\]
	and since this system of equations is upper triangular, it has determinant $k!$ which is relatively prime to $p$ since $k<p$. This means, this equation is solvable in $q_i$'s where our coefficients have denominator not divisible by $p$. Finally, we let $q_0=0$, and this proves the lemma. 
	
	Now we can finish the proof. Let $P_n(i)$ be the number of segments of length $i$, and we need to show that $p$ divides $\dsum_{i=1}^{p-1} i^2P_n(i)$. 
	Let $P_n(x) = \dsum_{i=0}^{n-2} a_ix^i$, then $\dsum_{i=1}^{p-1} i^2P_n(i) = \dsum_{i=0}^{n} a_n (\dsum_{j=1}^{p-1} j^{i+2})$. 
	Since $a_n$ has denominator not devisible by $p$, it suffices to show that $\dsum_{j=1}^{p-1} j^{i+2}$ is disivible by $p$ for all $0\le i\le n-2$. Let $g$ be the primitive root, then we can think $\dsum_{j=1}^{p-1} j^{i+2}$ as $\dsum_{j=0}^{p-2} (g^j)^{i+2}=\dfrac{g^{(p-1)(i+2)}-1}{g^{i+2}-1}$. 
	The numerator is disivible by $p$ by Fermat's Little theorem; the denominator is not because $g$ is a primitive root of $p$ but $i+2\le n<p-1$. The conclusion then follows. 
\end{enumerate}


\end{document}