\documentclass[11pt,a4paper]{article}
\usepackage{amsmath, amssymb, fullpage, mathrsfs, bm, pgf, tikz}
\usepackage{mathrsfs, amsthm}
\usetikzlibrary{arrows}
\setlength{\textheight}{10in}
%\setlength{\topmargin}{0in}
\setlength{\topmargin}{-0.5in}
\setlength{\parskip}{0.1in}
\setlength{\parindent}{0in}

\begin{document}
\newcommand{\la}{\leftarrow}
\newcommand{\lra}{\leftrightarrow}
\newcommand{\bbN}{\mathbb{N}}
\newcommand{\dsum}{\displaystyle\sum}
\newcommand{\dprod}{\displaystyle\prod}

\newtheorem{lemma}{Lemma}


\section*{Algebra}
\begin{enumerate}
	\item [\textbf{A1}] (IMO 1). Given any set $A = \{a_1, a_2, a_3, a_4\}$ of four distinct positive integers, we denote the sum $a_1 +a_2 +a_3 +a_4$ by $s_A$. Let $n_A$ denote the number of pairs $(i, j)$ with $1 \leq  i < j \leq 4$ for which $a_i +a_j$ divides $s_A$. Find all sets $A$ of four distinct positive integers which achieve the largest possible value of $n_A$.
	
	\textbf{Answer.} All sets in the form $\{d, 5d, 7d, 11d\}$ and $\{d, 11d, 19d, 29d\}$ with $d\in\bbN$. \\
	\textbf{Solution.} We first show that $n_A\le 4$. Since the numbers are distinct, we can order them as $a_1<a_2<a_3<a_4$ and we have $a_3+a_4<s_A=a_1+a_2+a_3+a_4<a_3+a_4+a_3+a_4=2(a_3+a_4)$ we have $a_3+a_4\nmid s_A$. Similarly $a_2+a_4<s_A=a_1+a_2+a_3+a_4<a_2+a_2+a_4+a_4=2(a_2+a_4)$ we also have $a_2+a_4\nmid s_A$. 
	Therefore to achieve the maximum we need $n_A=4$ (as we will show later that this is attainable), we need all $a_1+a_2, a_1+a_3, a_1+a_4, a_2+a_3$ to divide $s_A$. We can also see that this condition forces $a_1+a_4$ and $a_2+a_3$ to divide each other, and therefore these two sums must be equal, i.e. $a_1+a_4=a_2+a_3=s_A$. 
	
	Now there must exist integers $k$ and $\ell$ such that $s_A=k(a_1+a_2)=\ell(a_1+a_3)$, so $a_1+a_2=\frac 1k s_A$ and $a_1+a_3=\frac 1{\ell} s_A$. Adding these up gives $\frac {s_A}{2}=a_2+a_3<2a_1+a_2+a_3=s_A(\frac 1k + \frac 1{\ell})$ and we have both $a_1+a_2<a_2+a_3$ and $a_1+a_2<a_2+a_3$, so $k, \ell>2$. However $\frac 1k + \frac 1{\ell}>\frac 12$ we can easily see the the smaller of them, i.e. $\ell$, must be equal to 3, and $k$ must be equal to 4 or 5. 
	
	If $k=4$ then we can solve $a_1, a_2, a_3, a_4=(d, 5d, 7d, 11d)$ and if $k=5$ we can solve $a_1, a_2, a_3, a_4=(d, 11d, 19d, 29d)$. It's not hard to show that these answers work. 
	
	Remark 1: my solution in the real contest is way messier than this, where I wrote $a_4=a_2+a_3-a_1$ and solve the other variables in terms of $k$ and $\ell$ (though it worked in the end). 
	
	Remark 2: the relation $\frac 1k + \frac 1{\ell}>\frac 12$ also (kind of) resembles the reason why there are only 5 platonic solids. 
	
	\item[\textbf{A2}] Determine all sequences $(x_1,x_2,\ldots,x_{2011})$ of positive integers, such that for every positive integer $n$ there exists an integer $a$ with \[\sum^{2011}_{j=1} j  x^n_j = a^{n+1} + 1\]
	
	\textbf{Answer.} The only possible sequence is $x_1=1$ and $a=x_2=\cdots = x_{2011}=2+3+\cdots + 2011$. \\
	\textbf{Solution.} First, for each $j$ we have $\sum^{2011}_{j=1} j  x^n_j - 1\ge jx_j^n$ since each term is positive. Also using the notion of limits $\lim_{n\to\infty} (jx_j^n)^{\frac{1}{n+1}}=\lim_{n\to\infty}  j^{\frac{1}{n+1}}x_j^{\frac{n}{n+1}}=j^0x_j^1=x_j$, so we can conclude that $a\ge x_j$ for $n$ sufficiently large. In other words, for $n$ sufficiently large we have $a\ge X=\max\{x_1, x_2, \cdots, x_{2011}\}$. 
	On the other hand we have $\sum^{2011}_{j=1} j  x^n_j-1 \le X^n(1+\cdots + 2011)-1<CX^n$ where $C=1+\cdots + 2011$. 
	As usual $\lim_{n\to\infty}(CX^n)^{\frac{1}{n+1}}=C^{\frac{1}{n+1}}X^{\frac{n}{n+1}}=C^0X^1=X$ so we have $a\le X$ too for $n$ sufficiently large, too. Hence we have $a=X$ for all sufficiently large $n$. 
	
	Now we have $X^{n+1}=\sum^{2011}_{j=1} j  x^n_j - 1$, and dividing each side by $X^n$ we have 
	$X=\sum^{2011}_{j=1}j\left(\frac{x_j}{X}\right)^n-\left(\frac{1}{X}\right)^n$ for all sufficiently large $n$. Since each $x_j\le X$, $j\left(\frac{x_j}{X}\right)^n$ converges to $j$ if $x_j=X$ and 0 if $x_j<X$ (when $n\to\infty$). 
	Thus it's not hard to see that
	\[
		X=\lim_{n\to\infty}\sum^{2011}_{j=1}j\left(\frac{x_j}{X}\right)^n-\left(\frac{1}{X}\right)^n
		=\dsum_{1\le j\le 2011, x_j=X} j
	\]
	and therefore for $n$ sufficiently large we have 
	\begin{flalign*}
		\dsum_{1\le j\le 2011, x_j=X} j &= \sum^{2011}_{j=1}j\left(\frac{x_j}{X}\right)^n-\left(\frac{1}{X}\right)^n
		\\& =\dsum_{1\le j\le 2011, x_j=X} j \left(\frac{x_j}{X}\right)^n + \dsum_{1\le j\le 2011, x_j < X} j \left(\frac{x_j}{X}\right)^n - \left(\frac{1}{X}\right)^n
		\\& =\dsum_{1\le j\le 2011, x_j=X} j + \dsum_{1\le j\le 2011, x_j < X} j \left(\frac{x_j}{X}\right)^n - \left(\frac{1}{X}\right)^n
	\end{flalign*}
	which we can conclude that $\dsum_{1\le j\le 2011, x_j < X} j \left(\frac{x_j}{X}\right)^n - \left(\frac{1}{X}\right)^n=0$ for all sufficiently large $n$. Multiplying by $X^n$ again we get $1=\dsum_{1\le j\le 2011, x_j < X} jx_j^n$ for all sufficiently large $n$. Thus the only $j$ that can have $x_j<X$ is $j=1$, in which $x_1=1$. The other $j$ has $x_j=X=\dsum_{1\le j\le 2011, x_j=X} j = 2+\cdots + 2011$, as desired. 
	
	\item[\textbf{A3}]
	Determine all pairs $(f,g)$ of functions from the set of real numbers to itself that satisfy
	\begin{equation}\label{eqn:a3}
	  g(f(x+y)) = f(x) + (2x + y)g(y)
	\end{equation}
	for all real numbers $x$ and $y$.
	
	\textbf{Answer.} 
	Either $f\equiv g\equiv 0$, or $f(x)=x^2+C$ and $g(x)=x$ for some constant $C$. 
	The first one gives both side $=0$; 
	the second one gives both sides $=0$ $x^2+C$. 
	
	\textbf{Solution.} 
	We'll first show that $g$ is linear. 
	Fix constant $c$, and substitue $x+c$ into $x$ and $-x$ into $y$ into Eqn \ref{eqn:a3}, we get 
	\begin{equation}\label{eqn:a3a}
	  g(f(c)) = f(x+c) + (x+2c)g(-x)
	\end{equation}
	and comparing $x-c$ and $y-c$ in place of $x$ in \ref{eqn:a3a} we get 
	\begin{equation}\label{eqn:a3b}
	f(x) - f(y)= (y+c)g(-(y-c)) - (x+c)g(-(x-c)) 
	\end{equation}
	which holds for all $x, y$, and $c$. 
	Thus \ref{eqn:a3b} tells us that, compared to $c=0$, 
	\begin{equation}\label{eqn:a3c}
	f(x) - f(y)= (y+c)g(-(y-c)) - (x+c)g(-(x-c)) = yg(-y) - xg(-x) 
	\end{equation}
	and similarly, consider substituting $x-c, y-c$ into \ref{eqn:a3b}
	and $-c$ and 0, respectively into $c$ gives 
	\begin{equation}\label{eqn:a3d}
	f(x-c) - f(y-c) = (y-c)g(-(y-c)) - (x-c)g(-(x-c)) = (y-2c)g(-y) - (x-2c)g(-x) 
	\end{equation}
	Thus subtracting \ref{eqn:a3c} by \ref{eqn:a3d} (matching like terms) give 
	\begin{equation}\label{eqn:a3linear}
	  2cg(-(y-c)) - 2cg(-(x-c)) = 2cg(-y)-2cg(-x)
	\end{equation}
	i.e. for all $c\neq 0$, $g(-(y-c)) - g(-(x-c)) = g(-y)-g(-x)$, 
	meaning that $(-x, g(-x)), (-y, g(-y))$ and $(-(x-c), g(-(x-c))), (-(y-c), g(-(y-c)))$ 
	are two lines with the same slope. 
	This would mean there exists $m$ (the gradient), 
	$d_1, d_2$ such that 
	\[g(-x)=-mx+d_1, g(-y)=-my+d_1, 
	g(-(x-c))=-m(x-c)+d_2, g(-(y-c))=-m(y-c)+d_2
	\]
	which, substituting into \ref{eqn:a3c} gives 
	\[
	(y-x)d_1 = (y-x)d_2
	\]
	so whenever $x\neq y$, we have $d_1=d_2$. 
	This means when $x\neq y, c\neq 0$ we have the four points
	\[
	(-x, g(-x)), (-y, g(-y))-(x-c), g(-(x-c))), (-(y-c), g(-(y-c)))
	\]
	all lie on the same line. Thus fixing some $x$ and $y$ (say, 0 and 1) and consider all other $c\neq 0$ we have all $(x, g(x))$ on the same line, hence $g$ linear. 
	
	Now we can write $g(x)=mx+c$, 
	and substitute this into \ref{eqn:a3} once again. 
	If $m=1$ then consiering $y=0$ we have $f(x)+c=f(x)+2xc$ for all $x$, 
	so $c=0$ (i.e. $g(x)=x$). 
	Now putting $x=0$ into \ref{eqn:a3} and $g(x)=x$ gives $f(y)=f(0)+y^2$, 
	establishing $f(x)=x^2+C$. 
	
	If $m\neq 1$, then with $y=0$ into \ref{eqn:a3} we have $mf(x)+c=f(x)+2xc$, 
	and therefore $f(x)=\frac{(2x-1)c}{m-1}$. 
	This implies 
	\[
	m\left(\frac{(2x+2y-1)c}{m-1}\right)+c
	=\frac{(2x-1)c}{m-1} + (2x+y)(my+c)
	\]
	The only combination that works for all $x$ and $y$ is $m=c=0$. 
	
	\item[\textbf{A4}] 
	Determine all pairs $(f,g)$ of functions from the set of positive integers to itself that satisfy\[f^{g(n)+1}(n) + g^{f(n)}(n) = f(n+1) - g(n+1) + 1\]for every positive integer $n$. Here, $f^k(n)$ means $\underbrace{f(f(\ldots f)}_{k}(n) \ldots ))$.
	
	\textbf{Answer.} $f(n) = n$ and $g(n) = 1$ is the only answer. 
	
	\textbf{Solution.} We first verify the function above, where both sides are $n + 1$. 
	It now remains to show that this is the only pair. 
	
	Let $P_k$ be the assertion that $f(n) = n$ for $n = 1, 2, \cdots, n$, 
	and $f(m)\ge k + 1$ for all $m\ge k + 1$. 
	We show that $P_k$ holds for all $k$ by using induction, 
	where base case is given by $k = 0$ since $f(n)\ge 1$ for all $n\ge 1$. 
	
	Now assume $P_k$ for some $k\ge 0$. 
	Consider $m=\arg\min_{n\ge k + 1} f(n + 1)$, 
	i.e. the $m\ge k + 1$ that satisfies $f(m + 1)\le f(n + 1)$ for all $n\ge k + 1$. 
	Then we have 
	\[
	f^{g(m)+1}(m) + 1
	\le 
	f^{g(m)+1}(m) + g^{f(m)}(m) = f(m+1) - g(m+1) + 1
	\le f(m + 1)
	\]
	and therefore for this $m$, $f^{g(m)+1}(m) + 1 < f(m + 1)\le f(n + 1)$ for all $n\ge k + 1$. 
	We note that from $P_k$ and $m\ge k + 1$, $f^{a}(k + 1)\ge k + 1$ for any $a\ge 1$, 
	and so $f(n)\ge k + 2$ for all $k\ge n + 2$. 
	Since $f^{g(m)+1}(m) = f^{g(m)}(f(m)) < f(n)$ for all $n\ge k + 2$ and $f(n)\ge k + 2$ for all $n\ge k + 2$, 
	it follows that $m = k + 1$ and $f(m) = k + 1$, thus establishing $P_{k + 1}$. 
	
	Now with $f(n) = n$, substituting to the original equation gives 
	$g^{n}(n) + g(n + 1) = 2$ for each $n$. 
	As $g$ takes value $\ge 1$, we have $g(n + 1) = 1$ for all $n\ge 1$. 
	Finally, considering $n = 1$ gives $g(1) + g(2) = 2$, so $g(1) = 1$ too. 
	Thus $g$ must be identically 1. 
	
	\item[\textbf{A5}] Prove that for every positive integer $n,$ the set $\{2,3,4,\ldots,3n+1\}$ can be partitioned into $n$ triples in such a way that the numbers from each triple are the lengths of the sides of some obtuse triangle.
	
	\textbf{Solution.} 
	The idea is to form the triples $(a, b, c)$ such that, if $a<b<c$ then $c-b\in\Theta(a)$ as compared to other $a$ (and when $n$ grows big). We form the $n$ triples in the following way: all the $n$ smallest numbers $2, \cdots , n+1$ will be in different triples as the smallest element in the triple. For the remaining $2n$ numbers, we need the following: let $k$ be the greatest number such that $2^0+2^1+\cdots + 2^k\le n$, then $n=2^0+2^1+\cdots + 2^k+\ell$ with $\ell < 2^{k+1}$. 
	We now sort $2^0, \cdots , 2^k, \ell$ in nondecreasing order $a_1, a_2, \cdots , a_{k+1}$ ($\ell$ could be 0 for degenerate case, and we shall omit it for time being). 
	For each $i=1, 2, \cdots , n$, let $j$ be the minimal index with $a_1+a_2+\cdots + a_j\le i$. 
	We now name the $i$-th triple as follow: let $a_1+\cdots + a_{j-1}=p$, then the $i$-th triple is $(i+1, n+1+2p+(i-p), n+1+2p+a_j+(i-p))$. 
	
	To show that this triple works, we first need to see that it's disjoint and all within the range $\{2, 3, \cdots , 3n+1\}$. Observe that the first element $i+1$ corresponds to the $i$-th triple, so they are in the range $\{2, \cdots , n+1\}$. Then, the partition of $a_1, a_2, \cdots , a_{k+1}$ puts them into $k+1$ different ``buckets''. In the explanation above, $i$ is in the $j$-th bucket and the middle elements in this $j$-th bucket are in between $n+1+2p+1$ to $n+1+2p+a_j$ (inclusive) and the largest elements in this bucket are in between $n+1+2p+a_j+1$ and $n+1+2p+2a_j$, so recall that $p=a_1+\cdots + a_{j-1}$ we infer that the numbers are all distinct from $n+1+2(a_1+\cdots + a_{j-1})+1$ too $n+1+2(a_1+\cdots + a_j)$. The largest possible number (considering all $j$'s) is then $n+1+2(a_1+\cdots + a_{k+1})=n+1+2n=3n+1$, completing the claim. 
	
	We now show that the resulting triples are the side lengths of obtuse triangles. To this end, if $a<b<c$ are the elements of the triples then we need $c-b<a$ but then $(c-b)(c+b)=c^2-b^2>a^2$. 
	Let's use the example from above, i.e. our description. In the triple $(i+1, n+1+2p+(i-p), n+1+2p+a_j+(i-p))$ we have $a=i+1$, $b=n+1+2p+(i-p)$ and $c=n+1+2p+a_j+(i-p)$. 
	We have $c-b=a_j$. 
	Recall also that, $a_1+a_2+\cdots + a_{j-1}<i\le a_1+a_2+\cdots + a_j$. 
	For $a_j>1$, $m$ is the biggest number such that $2^m<a_j$ then the numbers $1, 2, \cdots , 2^{m}$ must all appear before $a_j$ so $a_1+\cdots + a_{j-1}\ge 1+\cdots + 2^{m}=2^{m+1}-1$. 
	By the maximality of $m$ we also have $a_j\le 2^{m+1}$, and since $i>a_1+\cdots + a_{j-1}\ge 2^{m+1}-1$, $i\ge 2^{m+1}$ and so $a=i+1>2^{m+1}\ge a_j=c-b$, so $(a, b, c)$ is indeed side of a triangle. 
	
	To see from above that $(a, b, c)$ is indeed obtuse, we have 
	$c^2-b^2=(c-b)(c+b)=a_j(a_j+2(n+1+2p+(i-p)))$ and we need to show that this is strictly greater than $a^2=(i+1)^2$. We first notice that $i$ is at most $a_1+a_2+\cdots + a_j$, therefore: 
	\begin{flalign*}
		\frac{i+1}{a_j}&\le \frac{a_1+a_2+\cdots + a_j+1}{a_j}
		\\&=\frac{a_1+a_2+\cdots + a_{j-1} + 1}{a_j}+1
	\end{flalign*}
	If $a_j$ is not in the form $2^m$ then $a_j=2^{j-2}+c$ with $a_{\ell}=2^{\ell-1}$ for $\ell<j$ so $a_1+a_2+\cdots + a_{j-1} + 1=1+2+\cdots + 2^{j-2}+1=2^{j-1}$ so $\frac{a_1+a_2+\cdots + a_{j-1} + 1}{a_j}+1=\frac{2^{j-1}}{\frac{2^{j-2}}+c}+1\le 2+1=3$;
	otherwise $a_j=2^m$ and $a_1+\cdots +a_j\le 1+2+\cdots +2^{m}=2^{m+1}-1$ so in this case we also have 
	$\frac{a_1+a_2+\cdots + a_{j-1} + 1}{a_j}+1\le 3$, too. 
	This means, $i+1\le 3a_j$. 
	Next, we investigate the number $a_j+2(n+1+2p+(i-p))=a_j+2(n+1)+4p+2(i-p)\ge a_j+2i+4p+2(i-p)$. 
	We have $i-p\ge 1$ since $p=a_1+\cdots +a_{j-1}<i$. 
	Also, recall that $i+1\le a_1+a_2+\cdots + a_j + 1$ so 
	$a_j+4p=a_j+4(a_1+\cdots + a_{j-1})\ge a_j+a_1+\cdots a_{j-1}$, with equality only if $a_1+\cdots + a_{j-1}=p=0$. This only happens when $i=1$, in which case we have $(2, n+2, n+3)$ which is obtuse since $(n+3)^2-(n+2)^2=2n+5>4$. 
	For $i>1$ we have $p\ge 1$ so $a_j+4(a_1+\cdots + a_{j-1}) \ge a_j+a_1+\cdots a_{j-1} + 3>i+1$, so 
	$a_j+4p>i+1$ and $2(n+1)+2(i-p)\ge 2i+2$, resulting in $a_j+2(n+1+2p+(i-p))>3(i+1)$. 
	Summarizing above, we have $c-b=a_j\ge \frac{i+1}{3}=\frac{a}{3}$ and $c+b>3(i+1)=3a$. 
	Thus $c^2-b^2=(c-b)(c+b)>\frac{a}{3}\times 3a=a^2$, as desired. 
	
	\item[\textbf{A6}] (IMO 3) Let $f : \mathbb R \to \mathbb R$ be a real-valued function defined on the set of real numbers that satisfies
	\[f(x + y) \leq yf(x) + f(f(x))\]
	for all real numbers $x$ and $y$. Prove that $f(x) = 0$ for all $x \leq 0$.
	
	\textbf{Solution.} As per the official solution, we need to first show that $f(x)\le 0$ for all $x$. 
	Let $x_0$ be a number such that $f(x_0)>0$. Then $f(x_0+y)\le yf(x_0)+f(f(x_0))$. 
	This means, for each threshold $M$, for all $y<\frac{M-f(f(x_0))}{f(x_0)}$ we have $f(x_0+y)\le yf(x_0)+f(f(x_0))<M$. This means, for all sufficiently small $x$ we have $f(x)<M$. 
	Now, if $x+y=x_0$ then $yf(x) + f(f(x))\ge f(x_0)>0$. 
	Choose $x$ such that $y=x_0-x>0$ (i.e. $x<x_0$) and $x$ small enough such that $f(x), f(f(x))<0$. 
	(That is, if $m_0$ is such that $f(x)<0$ for all $x<m_0$ and $m_1$ is such that $f(x)<m_0$ for all $x<m_1$ then we choose $x<\min\{m_0, m_1, x_0\}$ so $f(x)<0$ and $f(f(x))<0$, too. 
	Now, $y=x_0-x>0$, $f(x)<0$ and $f(f(x))<0$ so $yf(x)+f(f(x))<0$ but then $f(x+y)=f(x_0)>0$, contradiction. 
	
	Having established this, we need to show that 0 is a value of $f$. Suppose not, then $f(x)<0$ for all $x$. 
	We notice that $f(f(x))< 0$ so $f(x+y)\le yf(x)+f(f(x))< yf(x)$. 
	This means $f(x+y)<yf(x)<f(x)$ if $y\ge 1$, or equivalently $f(x-1)>f(x)$. Consider, for now, the number $f(-1)$. Then for all $x\le -2$ we have $f(x)>f(-1)$. So by choosing $x<\min\{f(-1), -2\}$ we have $f(x)>f(-1)>x$, i.e. $f(x)>x$. 
	Consider one such $x_2$ and by the lemma above, $f(x_2-1)>f(x_2)>x_2$. 
	Plugging $y=0$ gives $f(x_2-1)\le f(f(x_2-1))$. But since $f(x_2-1)>x_2$, $f(f(x_2-1))<f(x_2-1)$, which is a contradiction. Thus $f$ cannot be all negative. 
	
	Now that $f(x_1)=0$ for some $x_1$, by plugging $y=0$ we get $f(x_1)\le f(f(x_1))$, i.e. $f(0)\ge 0$ so $f(0)=0$ by the first lemma. Now for all $x>0$ we have $0=f(0)=f(x-x)\le xf(-x)+f(f(x))\le xf(-x)$ since $f(f(x))\le 0$. Since $x$ is positive, $f(-x)$ must be nonnegative. Since $f(-x)$ is also nonpositive, we have $f(-x)=0$ for all $x>0$, so $f(x)=0$ for all $x<0$. 
	
	\item [\textbf{A7}] Let $a,b$ and $c$ be positive real numbers satisfying $\min(a+b,b+c,c+a) > \sqrt{2}$ and $a^2+b^2+c^2=3.$ Prove that
	
	\[\frac{a}{(b+c-a)^2} + \frac{b}{(c+a-b)^2} + \frac{c}{(a+b-c)^2} \geq \frac{3}{(abc)^2}.\]
	
	\textbf{Solution.} We first show that $a+b-c, b+c-a, c+a-b$ are all positive. Suppose otherwise, then if $c=\max\{a, b, c\}$ then $c\ge a+b$. Now that $a+b>\sqrt{2}$, and $a^2+b^2\ge \frac{(a+b)^2}{2}>1$, so $a^2+b^2+c^2>1+(a+b)^2>1+2=3$, contradicting that $a^2+b^2+c^2=3$. Hence these three quantities mentioned must all be positive. 
	
	Now we can write $a=x+y, b=x+z, c=y+z$ for some $x, y, z > 0$, i.e. $a, b, c$ are sides of a triangle. 
	This transforms the inequality into $\frac{x+y}{(2z)^2} + \frac{x+z}{(2y)^2} + \frac{y+z}{(2x)^2} \geq \frac{3}{(x+y)^2(x+z)^2(y+z)^2}$. We also have $(x+y)^2+(y+z)^2+(z+x)^2=3$ so $x^2+y^2+z^2+xy+yz+zx=\frac 32$. 
	Since it's not hard to show that $xy+yz+zx\le x^2+y^2+z^2$ (i.e. $xy+yz+zx\le 34$), we have 
	\[(x+y+z)^2 = x^2+y^2+z^2+2(xy+yz+zx)=\frac 32 + xy+yz+yz \le \frac 32 + \frac 34 = \frac 94\]
	and consequently $x+y+z\le \frac 32$. 
	This means $\frac{x+y}{(2z)^2} + \frac{x+z}{(2y)^2} + \frac{y+z}{(2x)^2}\ge \frac 23 (x+y+z)(\frac{x+y}{(2z)^2} + \frac{x+z}{(2y)^2} + \frac{y+z}{(2x)^2})$ and thus it suffices to show that this is $\ge$ the right hand side. 
	Multiplying RHS by $\frac {1}{27} ((x+y)^{2}+(y+z)^{2}+(z+x)^{2})^{3}$ to homogenize it, and cancelling the denominators, we have the following to prove: 
	\begin{flalign*}
	3(x+y+z)(x^{2}y^{2}(x+y)+y^{2}z^{2}(y+z)+z^{2}x^{2}(z+x))(x+y)^2(x+z)^2(y+z)^2\\
	\ge 2x^2y^2z^2((x+y)^{2}+(y+z)^{2}+(z+x)^{2})^{3}
	\end{flalign*}
	Denoting $\sum$ by the symmetric sum, we have the left-hand-side as 
	\[3\sum x^{8}y^{4}z^{0}+6\sum x^{8}y^{3}z^{1}+3\sum x^{8}y^{2}z^{2}+12\sum x^{7}y^{5}z^{0}+33\sum x^{7}y^{4}z^{1}+39\sum x^{7}y^{3}z^{2}\]\[+9\sum x^{6}y^{6}z^{0}+69\sum x^{6}y^{5}z^{1}+102\sum x^{6}y^{4}z^{2}+51\sum x^{6}y^{3}z^{3}+69\sum x^{5}y^{5}z^{2}+153\sum x^{5}y^{4}z^{3}\]\[+27\sum x^{4}y^{4}z^{4}
	\]
	and the right hand side as 
	\[
	8\sum x^{8}y^{2}z^{2}+48\sum x^{7}y^{3}z^{2}+96\sum x^{6}y^{4}z^{2}+72\sum x^{6}y^{3}z^{3}\]\[+56\sum x^{5}y^{5}z^{2}+240\sum x^{5}y^{4}z^{3}+56\sum x^{4}y^{4}z^{4}
	\]
	Muirhead now settles everything. 
	
\end{enumerate}

\section*{Combinatorics}
\begin{enumerate}
	\item [\textbf{C1}] (IMO 4) Let $n > 0$ be an integer. We are given a balance and $n$ weights of weight $2^0, 2^1, \cdots, 2^{n-1}$. We are to place each of the $n$ weights on the balance, one after another, in such a way that the right pan is never heavier than the left pan. At each step we choose one of the weights that has not yet been placed on the balance, and place it on either the left pan or the right pan, until all of the weights have been placed.
	Determine the number of ways in which this can be done.
	
	\textbf{Answer.} $1\times 3 \times \cdots \times (2n-1)$. \\
	\textbf{Solution.} The key to the problem is to notice that if $a_1, a_2, \cdots , a_k$ are distinct nonnegative integers and $M=\max\{a_1, \cdots , a_k\}$ then $2^{M}\le 2^{a_1}+\cdots + 2^{a_k}\le 2^{M+1}-1<2^{M+1}$. 
	Therefore, a sequence is valid if and only if at each time, either the right pan is empty, or both pans are nonempty and the heaviest weight on the left pan is heavier than the heaviest weight on the right pan.
	We can also see that the pans can never be of the same weight for this reason, unless both pans are empty. 
	
	Having known these, we can use induction to show that if $f(n)$ is the number of ways when $n$ is the number weights, then $f(1)=1$ and $f(n)=(2n-1)f(n-1)$. 
	Base case $n=1$ is simple: the only weight can only be added on the left. 
	Now consider a sequence of $n$ weights $n\ge 2$ and consider the lightest of the weights (i.e. 1). By removing this weight from the sequence, we see that both pans contain weights in the set $\{2^1, \cdots , 2^{n-1}\}$, and moreover the pans have a weight difference of at least 2. Therefore, adding the weight 1 on either side will never change the validity of the pans (i.e. the left pan is heavier before the addition of 1 iff the left pan becomes heavier after that). 
	We then infer that a valid sequence of $n$ weights comes from the valid sequence of $n - 1$ weights, with 1 inserted. This 1 can be inserted anywhere in the sequence and on any pan, except that if it's the first to be placed it has to be on the left pan. This gives $2n-1$ ways, making $f(n)=(2n-1)f(n-1)$. 
	
	It's also technically possible to induct on the heaviest weight, i.e. the one with weight $2^{n-1}$. 
	For each $1\le k\le n$, suppose that this weight is the $k$-th weight to be placed onto the pans. We notice the following: 
	\begin{itemize}
		\item This heaviest weight must be on the left. 
		
		\item The $k - 1$ previous weights must follow the rule of "max of left $>$ max of right", so within the $k-1$ weights themselves, there are $f(k-1)$ ways to arrange them. Notice that we also have $\dbinom{n-1}{k-1}$ to choose those $k-1$ weights from the $n-1$ remaining weights. 
		
		\item Finally, the $n-k$ subsequent weights can be placed arbitrarily. This gives $(n-k)!$ permutations of the $n-k$ weights, each with 2 possible choices, left or right. 
	\end{itemize}
	Thus considering all these we have the summation $\dsum_{k=1}^{n} f(k-1)\dbinom{n-1}{k-1}(n-k)!2^{n-k}=\dsum_{k=1}^{n} f(k-1)2^{n-k}\dfrac{(n-1)!}{(k-1)!}$, with $f(0)=1$ by convention (we used the fact that $\dbinom{n-1}{k-1}=\dfrac{(n-1)!}{(k-1)!(n-k)!}$). 
	If we look at the term $f(k-1)2^{n-k}\dfrac{(n-1)!}{(k-1)!}$ more carefully, by the assumption $f(k-1)=1\times 3\times \cdot\times (2k-3)$ we have 
	\begin{flalign*}
		f(k-1)2^{n-k}\dfrac{(n-1)!}{(k-1)!}&=1\times 3\times \cdots\times (2k-3) 2^{n-k} k\times (k+1)\times\cdots \times (n-1)\\
		&=1\times 3\times \cdots \times (2k-3) \times 2k \times (2k+2)\times \cdots \times (2n-2)
	\end{flalign*}
	The final (and ultimate) computation requires us to do some telescoping sum. 
	Fixing $n$, let $g(k)=1\times 3\times \cdots (2k-3)\times 2k \times (2k+2)\times \cdots \times (2n-2)$, then we show that $g(1)+\cdots + g(k)=1\times 3\times \cdots \times (2k-1) \times 2k \times (2k+2)\times \cdots \times (2n-2)$. 
	Again we can do induction on $k$: $g(1)=2\times \cdots \times (2n-2)=1\times 2\times \cdots \times (2n-2)$ and 
	\begin{flalign*}
		g(0)+\cdots + g(k)+g(k+1)&=1\times 3\times \cdots \times (2k-1) \times 2k \times (2k+2)\times \cdots \times (2n-2) \\
		&+ 1\times 3\times \cdots \times (2k-1) \times (2k+2) \times (2k+2)\times \cdots \times (2n-2)
		\\&=1\times 3\times \cdots \times (2k-1) \times (2k+1) \times (2k+2)\times \cdots \times (2n-2)
	\end{flalign*}
	as desired. Thus letting $k=n$ we have $f(n)=g(1)+\cdots + g(n)=1\times 3\times \cdots \times (2n-1)$, as desired. 
	
	\item[\textbf{C2}] Suppose that $1000$ students are standing in a circle. Prove that there exists an integer $k$ with $100 \leq k \leq 300$ such that in this circle there exists a contiguous group of $2k$ students, for which the first half contains the same number of girls as the second half.
	
	\textbf{Solution.} Now consider the students as numbers $x_1, \cdots , x_{1000}$, indices taken modulo 1000, with boys representing 0 and girls representing 1.  We are to find an $\ell$ and $k$ (with $k$ within the constraint) such that $\dsum_{i=\ell}^{\ell+k-1}x_i=\dsum_{i=\ell+k}^{\ell+2k-1}x_i$. 
	Consider, now, the $\ell_0$ such that $\dsum_{i=\ell_0}^{\ell+100-1}x_i$ is the maximum, then 
	$\dsum_{i=\ell_0}^{\ell_0+100-1}x_i\ge\dsum_{i=\ell_0+k}^{\ell_0+200-1}x_i$ and 
	$\dsum_{i=\ell_0-100}^{\ell_0-1}x_i\le\dsum_{i=\ell_0}^{\ell_0+100-1}x_i$. 
	We shall consider the count $c(\ell)=\dsum_{i=\ell}^{\ell+100-1}x_i-\dsum_{i=\ell+k}^{\ell+200-1}x_i$ as $\ell$ ranges from $\ell_0-100$ to $\ell$. 
	Now, $c(\ell_0-100)\le 0$ and $c(\ell_0+100)\ge 0$. 
	If $c(\ell)=0$ for some $\ell$ then we are done, so assume they are all nonzero between $\ell_0-100$ and $\ell_0$. The inequality also suggests that $c(\ell)$ changes sign somewhere. 
	
	Consider, now, an $\ell$ such that $c(\ell)<0$ and $c(\ell+1)>0$, which also means $c(\ell+1)-c(\ell)\ge 1+1=2$. 
	We recognize that 
	\begin{flalign*}
		c(\ell+1)-c(\ell)
		&=(\dsum_{i=\ell+1}^{\ell+100}x_i-\dsum_{i=\ell+k+1}^{\ell+200}x_i) - (\dsum_{i=\ell}^{\ell+100-1}x_i-\dsum_{i=\ell+k}^{\ell+200-1}x_i)\\
		&=-x_{\ell}+2x_{\ell+100}-x_{\ell+200}\\
		\le -0+2(1)-(0)\\
		=2
	\end{flalign*}
	so all the inequalities above are inequality, and $x_{\ell}=x_{\ell+200}=0$, $x_{\ell+100}=1$. 
	In addition, we have $\dsum_{i=\ell+1}^{\ell+99}x_i=\dsum_{i=\ell+101}^{\ell+199}x_i$. 
	
	Let $p$ be the maximal integer such that $x_{\ell+200}=\cdots = x_{\ell+200+p-1}=0$, and also $x_{\ell}=\cdots = x_{\ell-p+1}=0$. By above, $p\ge 1$. 
	We now notice that, for $k=100+p$, $\dsum_{\ell+100}^{\ell+200+p-1}x_i=1+\dsum_{i=\ell+101}^{\ell+199}x_i + \dsum_{i=\ell+200}^{\ell+200+p-1}0=1+\dsum_{i=\ell+101}^{\ell+199}x_i$. 
	Similarly, we have $\dsum_{\ell-p}^{\ell+99}x_i=x_{\ell-p}+\dsum_{\ell-p+1}^{\ell} 0 + \dsum_{\ell+1}^{\ell+99}x_i=x_{\ell-p}+ \dsum_{\ell+1}^{\ell+99}x_i$. 
	Assume that $p\le 200$. If $x_{\ell-p}=1$ then by $\dsum_{i=\ell+1}^{\ell+99}x_i=\dsum_{i=\ell+101}^{\ell+199}x_i$ then $x_{\ell-p}+ \dsum_{\ell+1}^{\ell+99}x_i=1+\dsum_{i=\ell+101}^{\ell+199}x_i$ and we get a contagious pair of $100+p$ students each with same sum, solving the problem. 
	Otherwise, $x_{\ell-p}=0$. 
	By considering $\dsum_{\ell+101}^{\ell+200+p}x_i$ and $\dsum_{\ell-p+1}^{\ell+100}x_i$ we can either conclude that either these two are the contagious pair of $100+p$ elements, or $x_{\ell+200+p}=0$. Now if none of the $100+p$-pairs works, we have $x_{\ell-p}=x_{\ell+200+p}=0$, which contradicts the maximality of $p$. 
	
	Now suppose that $p>200$ instead, which means there are a consecutive row of $201$ 0's. Simply consider 200 of them, split them into contagious pairs of 100 0's each, and we are done. 
	
	\item[\textbf{C4}] 
	Determine the greatest positive integer $k$ that satisfies the following property: The set of positive integers can be partitioned into $k$ subsets $A_1, A_2, \ldots, A_k$ such that for all integers $n \geq 15$ and all $i \in \{1, 2, \ldots, k\}$ there exist two distinct elements of $A_i$ whose sum is $n.$
	
	\textbf{Answer.} $k = 3$. 
	
	\textbf{Solution.} The construction $k = 3$ is given as follows: (TODO)
	
	To show that $k = 4$ is impossible, suppose $A_1, A_2, A_3, A_4$ are such a partition. 
	Then one of them, say, $A_4$, must have at most three numbers in $\{1, 2, \cdots, 15\}$. 
	We show that $A_4$ must necessarily contain $1, 2, 14$ in this sequel. 
	
	Now given $A_4$ contains two distinct elements in $A_i$ summing to 15 as well as 16, 
	these elements must all be $\le 15$. 
	In particular, it cannot have only two elements in $[1, 15]$ range, hence by our assumption we can write the three integer in $[1, 15]$ as $a, b, c$, with one pair summing to 15 and the other summing to 16. 
	These two pairs must also share a common element, say, $a$, 
	so we may write these as $a, 15 - a, 16 - a$. 
	
	If $a = 7$, then these 3 numbers are 7, 8, 9 with pairwise sums 15, 16, 17. 
	Since the next smallest number in $A_4$ is $\ge 16$, it is impossible for $A_4$ to produce 18, 
	invalidating this. 
	Similarly, if $a\neq 7$, then $(15 - a), (16 - a)$ do not sum to 17, so one of the numbers summing to 17 must come outside the $[1, 15]$ range, 
	meaning that the two numbers must be 1 and 16. 
	We therefore infer that one of $a$ and $15 - a$ must be 1. 
	In the former case, $(a, b, c) = (1, 14, 15)$. 
	But without containing the elements $2, 3, \cdots, 13$, we see that in order to produce sum of 17, 18, $\cdots$, 28, 
	$A_4$ must contain $16, 17, \cdots, 27$. 
	Now one of $A_1, A_2, A_3$ contains no more than 4 elements in $[1, 15]$, 
	and cannot contain $16, \cdots, 27$. 
	This means it can produce only at most $\binom{4}{2}=6$ pairwise sums (using numbers in $[1, 15]$) among $15, 16, \cdots, 28$, thereby cannot contain all of those pairwise sums. 
	It therefore follows that the $a = 14$ and $\{a, b, c\} = \{1, 2, 14\}$. 
	
	The fact that $A_4$ must be $\{1, 2, 14\}$ means $A_1, A_2, A_3$ cannot have $\le 3$ elements in $[1, 15]$, 
	i.e. each having exactly 4 elements. 
	Having established this case, since $14 + 16 = 30$, it follows that for each $a = 17, \cdots, 29$, $A_4$ must contain either $a - 1$ or $a - 2$ (or possibly both). 
	In particular, it must contain 16, 
	and among 17, 18, 19, 20 it must contain two of these. 
	This means at most 2 of $\{16, 17, 18, 19, 20\}$ is available for $A_1, A_2, A_3$, 
	meaning one of them, say $A_3$, has none of the numbers in this set. 
	Since $1, 2\in A_4$, 
	$A_3$ contains 4 elements in range $[3, 15]$, and the remaining are $\ge 21$. 
	This means all the 9 pairwise sums adding to 15 to 23 (inclusive) must come from the 4 elements in $[3, 15]$ itself. 
	This is impossible as 4 elements can only produce $\binom{4}{2}=6$ pairwise sums. 
	
	\item[\textbf{C6}] Let $n$ be a positive integer, and let $W = \ldots x_{-1}x_0x_1x_2 \ldots$ be an infinite periodic word, consisting of just letters $a$ and/or $b$. Suppose that the minimal period $N$ of $W$ is greater than $2^n$.
	
	A finite nonempty word $U$ is said to appear in $W$ if there exist indices $k \leq \ell$ such that $U=x_k x_{k+1} \ldots x_{\ell}$. A finite word $U$ is called ubiquitous if the four words $Ua$, $Ub$, $aU$, and $bU$ all appear in $W$. Prove that there are at least $n$ ubiquitous finite nonempty words.
	
	\textbf{Solution.} For each positive integer $k$ we consider the words appearing in $W$ of length $k$; by the periodity of $W$ we also know that such set (of words of length $k$) is finite and won't exceed $N$. Of all such words, we also record the number of times each word appear as $x_{\ell}x_{\ell+1}\cdots x_{\ell+k-1}$ for $\ell=0, 1, \cdots , N-1$ (we only need to consider the first $N$ words and the others are just periodically repeating) and record $M(k)$, the highest of those frequencies with respect to $k$. The key to this problem is to exploit the properties of $M(k)$: 
	\begin{itemize}
		\item We first show that $M(N)=1$. Otherwise, there's a word appearing at least two times. Now, let $x_{k}x_{k+1}\cdots x_{k+N-1}=x_{\ell}x_{\ell+1}\cdots x_{\ell+N-1}$ with $0\le k<l\le N-1$. 
		This means, $x_{k+i}=x_{\ell+i}$ for all $i=0, 1, \cdots , N-1$. But since $W$ is periodic with period $N$, for each $i\in\mathbb{Z}$, choose $i'\equiv i\pmod{N}$ and $0\le i'<N$ and we have 
		$x_{k+i}=x_{k+i'}=x_{\ell+i'}=x_{\ell+i}$, so $W$ is also periodic with period $\ell-k$. But $0<\ell-k<N$, contradicting the minimality of $N$. Therefore $M(N)=1$. 
		
		\item Next, we see how $M(k)$ compares to $M(k+1)$. We show that $M(k)\le 2M(k+1)$. Indeed, if $U$ is a word of length $k$ with frequency $M(k)$, then one of $Ua$ and $Ub$ must have frequency at least $\frac{M(k)}{2}$, so $M(k+1)\ge \frac{M(k)}{2}$, as claimed. 
		
		\item We should also see what happens if $M(k)>M(k+1)$ for some $k$, and we claim that the word $U$ of length $k$ and frequency $M(k)$ is ubiquituous. Now, the sum of frequencies of $Ua$ and $Ub$ is $M(k)$ and since $M(k+1)<M(k)$, both the frequencies are strictly less than $M(k)$. 
		Hence both the frequencies must also be at least 1. Similarly, both the frequencies of $aU$ and $bU$ must also be at least 1, so $U$ is ubiquituous. In essence this means if $M(k+1)<M(k)$ then there exists a word of length $k$ that is ubiquituous. 
		
	\end{itemize}
	In view of the last bullet point it suffices to show that there are at leaast $n$ numbers $k$ such that $M(k+1)<M(k)$. That is, the function $M(k)$ decreases at least $k$ times. We first notice that $M(1)=\max \{freq(a), freq(b)\}\ge \frac{N}{2}>\frac{2^n}{2}=2^{n-1}$, and $M(N)=1$. Also, whenever $M$ decreases, it never decreases more than half based on the second bullet point. It follows that it must decrease at least $\log_2(M(1))>\log_2(2^{n-1})=n-1$ times, i.e. decrease at least $n$ times, which finishes the claim. 
	
	\item[\textbf{C7}] On a square table of $2011$ by $2011$ cells we place a finite number of napkins that each cover a square of $52$ by $52$ cells. In each cell we write the number of napkins covering it, and we record the maximal number $k$ of cells that all contain the same nonzero number. Considering all possible napkin configurations, what is the largest value of $k$?
	
	\textbf{Answer.} $2011^2-2\times 17\times 2011 + 38 \times 17^2 = 3986729$. \\
	\textbf{Solution.} An example has been shown in the official solution, whereby a total of $39^2$ napkins is used: label the napkins as $(i, j)$ with $1\le i, j\le 39$ (notice that $2011=38\times 52+35$). If $i+j\le 39$ then $(i, j)$ covers row from $52(i-1)+1$ to $52i$ and $52(j-1)+1$ to $52j$, inclusive; 
	if $i+j=40$ $(i, j)$ covers row from $52(i-1)+1$ to $52i$ and $52(j-1)-16$ to $52j-17$, inclusive (except $(39, 1)$ which covers the bottom left corner, as in the next case); finally if $i+j\ge 41$ then $(i, j)$ covers fow from $52(i-1)-16$ to $52i-17$ and $52(j-1)-16$ to $52j-17$. 
	This way, all squares have frequency 1 except the following cells: those in the form $52(i-1)+k, 52(39-i)+\ell$ and $52(i-1)+\ell$, $52(39-i)+k$ for $1\le i\le 39$, $1\le k\le 52$ (but $52(i-1)+k\le 2011$) and $36\le k\le 52$. 
	The first case corresponds to $2011\times 17$ of the cells; same goes to the second case. But then for the case $k\ge 36$ we double-counted so the number of cells not with frequency 1 is $2\times 17\times 2011-38\times 17^2$. 
	
	To show that this is indeed an upper bound, for each row, now, we investigate those cells with frequency the same frequency $k>0$. Also we classify the cells into four categories: for a cell $(i, j)$ with remainder $(i', j')$ when divided by 52 ($1\le i', j'\le 52$) we have category 1: $1\le i', j'\le 35$, category 2: $1\le i'\le 35$, $36\le j'\le 52$, category 3: $36\le i'\le 52$, $1\le j'\le 35$ and category 4: $36\le i', j'\le 52$. 
	Consider each row $r$ with cells $(r, j)$, $1\le j\le 2011$. 
	We notice that for each $1\le j'\le 52$, there are 38 cells congruent to $j'$ mod 52 if $36\le j'\le 52$ and 39 cells congruent to $j'$ otherwise. 
	We also notice that, consider each cell with coordinates modulo 52, for each $(i', j')\in \mathbb{Z}_{52}^2$, each napkin covers $(i', j')$ exactly once. 
	This means that if the number of napkins covering row $r$ is strictly less than $39k$, then for each $1\le j' \le 35$ there is at least one cell congruent to $j'$ mod 52 that has frquency strictly less than $k$; call those cells deficient. 
	Analogously, if the number of napkins covering row $r$ is strictly more than $38k$, then for each $36\le j' \le 52$ there is at least one cell congruent to $j'$ mod 52 that has frquency strictly more than $k$; call those cells excessive. 
	Since at least one of the scenarios above happen, we infer that for each row either there's at least 35 deficient cells or 17 excessive cells; moreover the 35 deficient cells are of category 1 if $1\le r'\pmod{52}\le 35$ and category 3 otherwise; the 17 excessive cells are of category 2 if $1\le r'\pmod{52}\le 35$ and category 4 otherwise. 
	In a similar way, for each column $c$ there's also either at least 35 deficient cells (category 1 or 2 depending on $c$), or at least 17 excessive cells (category 3 or 4 depending on $c$). 
	
	Now let $d$ and $e$ be rational numbers such that there are exactly $35d$ deficient type 1 cells and $17e$ excessive type 4 cells. Since there are $35\times 39$ rows (and respectively columns) that are of remainder $1\le r\le 35$, there must be at least $17(35\times 39-d)$ excessive type 2 cells (by considering the rows) and similarly $17(35\times 39-d)$ excessive type 3 cells. Likewise, since there are $17\times 38$ rows (and respectively columns), there must be at least $\max\{35(17\times 38-e), 0\}$ deficient type 2 cells and similarly $\max\{35(17\times 38-e), 0\}$ deficient type 3 cells. 
	The cells in the two scnearios are mutually disjoint: the deficient and excessive cells of same category can't overlap since we cannot have the frequency of some cell to be $>k$ an $<k$ simultaneously
	The cells with frquency not $k$ induced by $d$ in category 1 is at least 
	\begin{flalign*}
		35d+2\times 17(35\times 39-d)
		&=2\times 17\times 35\times 39 +d
		\\&\ge 2\times 17\times 35\times 39
	\end{flalign*}
	The cells with frquency not $k$ induced by $e$ in category 4 is lower-bounded by $17e+2\max\{35(17\times 38-e), 0\}$. When $e\le 17\times 38$ we have $17e+2\times 35\times 17\times 38 - 70e = 2\times 35\times 17\times 38 - 53e \ge 2\times 35\times 17\times 38 - 53(17\times 38) = 17^2\times 38$; when $e\ge 17\times 38$ this count is simply $17(17\times 38)$. Hence the lower bound here is $38\times 17^2$. 
	
	Summing up, a lower bound of the cells with frequency not $k$ is $2\times 17\times 35\times 39 + 38\times 17^2 = 17(2\times 35\times 39 + 38\times 17)=17(2011-38\times 17)$, as desired. 
	
\end{enumerate}

\section*{Geometry}
\begin{enumerate}
	\item[\textbf{G1}]
	Let $ABC$ be an acute triangle. Let $\omega$ be a circle whose centre $L$ lies on the side $BC$. Suppose that $\omega$ is tangent to $AB$ at $B'$ and $AC$ at $C'$. Suppose also that the circumcentre $O$ of triangle $ABC$ lies on the shorter arc $B'C'$ of $\omega$. Prove that the circumcircle of $ABC$ and $\omega$ meet at two points.
	
	\textbf{Solution.} 
	Let $R$ be circumradius of $ABC$ and $r$ the radius of $\omega$. It suffices to show that $R<2r$. 
	Henceforth we denote $\angle A = \angle BAC$. 
	
	\emph{Case 1.} 
	$\angle A < 60^{\circ}$. 
	Denote $h$ as the distance from $O$ to $BC$, then 
	\[
	r = OL \ge h = R\cos\frac{\angle BOC}{2} = R\cos\angle BAC > \frac{R}{2}
	\]
	
	\emph{Case 2.} 
	$\angle A \ge 60^{\circ}$. 
	Here we make use of $O$ on shorter arc $B'C'$, i.e. $O$ lies inside the quadrilateral $AB'LC'$. 
	Given also that the projection on $O$ to $AB$ and $AC$ are the midpoints of $AB$ and $AC$, respectively, 
	we have $AB'\ge B'B$ and $AC'\ge C'C$. 
	Consequently, we also have $AL\ge LB$ and $AL\ge LC$. 
	
	Let $AL$ intersect the circumcircle again at $M$, 
	then by power of point theorem we have $LM\le BL$ and $LM\le LC$. 
	W.l.o.g. let $AB\le AC$. 
	Then $\frac{\angle A}{2}=\angle CBM\le \angle BML$, 
	so $ML\le BM\cdot \frac{\sin \angle A/2}{\sin\angle A}$
	and given $MB^2 = ML\cdot MA=ML\cdot (ML+LA)$, 
	\[
	LA =\frac{BM^2}{ML}-ML \ge BM\cdot (\frac{\sin\angle A}{\sin \angle A/2} - \frac{\sin \angle A/2}{\sin\angle A})
	\]
	Finally, we have $r =LB'=LC'= AL\sin\frac{A}{2}$ and $BM=2R\sin \frac{\angle A}{2}$. 
	This gives 
	\[
	r = AL\sin\frac{A}{2}
	\ge 2R\sin^2\frac{A}{2}\cdot (\frac{\sin\angle A}{\sin \angle A/2} - \frac{\sin \angle A/2}{\sin\angle A})
	= 
	2R\sin^2\frac{A}{2}\cdot (\frac{\sin^2\angle A - \sin^2\angle A/2}{\sin \angle A/2\sin\angle A})
	\]
	and using the identity 
	$\sin^2 A - \sin^2 B = \sin(A-B)\sin(A+B)$
	 we get 
	\[
	2R\sin^2\frac{A}{2}\cdot (\frac{\sin^2\angle A - \sin^2\angle A/2}{\sin \angle A/2\sin\angle A})
	= 2R\sin^2\frac{A}{2}\cdot (\frac{\sin(A/2)\sin (3A/2)}{\sin \angle A/2\sin\angle A})
	\]
	\[
	=2R\sin^2 \frac{\angle A}{2}\sin\frac{3\angle A}{2}\div \sin\angle A
	\]
	
	which, for $\angle A\ge 60^{\circ}$, has value $\frac{1}{\sqrt{3}}R> \frac 12R$.
	
	\item[\textbf{G2}] Let $A_1A_2A_3A_4$ be a non-cyclic quadrilateral. Let $O_1$ and $r_1$ be the circumcentre and the circumradius of the triangle $A_2A_3A_4$. Define $O_2,O_3,O_4$ and $r_2,r_3,r_4$ in a similar way. Prove that
	\[\frac{1}{O_1A_1^2-r_1^2}+\frac{1}{O_2A_2^2-r_2^2}+\frac{1}{O_3A_3^2-r_3^2}+\frac{1}{O_4A_4^2-r_4^2}=0.\]
	
	\textbf{Solution.} Let diagonals $A_1A_3$ and $A_2A_4$ meet at point $P$. Notice that the quantity $O_iA_i-r_i^2$ is actually the power of point from $A_i$ to the circumcircle of triangle not containing $A_i$. To meaningfully compute these power of points, we need to consider the ``signed lengths'' of $PA_1, PA_2, PA_3, PA_4$: if for some angle, $\angle A_i>180^{\circ}$ (i.e. noncovex) then we let $PA_i<0$. 
	
	Now consider $PA_1$. We have $A_1A_3=PA_1+PA_3$, and if $Y$ is the second intersection of the circle $A_2A_3A_4$ and $A_1A_3$ then power of point theorm (on point $P$) says $PA_2PA_4=PA_3PY$. Consequently, the power of point of $A_1$ to this circle is $A_1A_3\cdot (PA_1-PY)=(PA_1+PA_3)\cdot (PA_1-\frac{PA_2PA_4}{PA_3})=(PA_1+PA_3)(\frac{PA_1PA_3-PA_2PA_4}{PA_3})$. This gives 
	$\frac{1}{O_1A_1^2-r_1^2}=\frac{PA_3}{(PA_1+PA_3)(PA_1PA_3-PA_2PA_4)}$. 
	Similarly, $\frac{1}{O_2A_2^2-r_2^2}=\frac{PA_4}{(PA_2+PA_4)(PA_2PA_4-PA_1PA_3)}$, 
	$\frac{1}{O_3A_3^2-r_3^2}=\frac{PA_1}{(PA_1+PA_3)(PA_1PA_3-PA_2PA_4)}$ and 
	$\frac{1}{O_4A_4^2-r_4^2}=\frac{PA_2}{(PA_2+PA_4)(PA_2PA_4-PA_1PA_3)}$. 
	Thus the desired sum becomes 
	\begin{flalign*}
		\dsum_{i=1}^4\frac{1}{O_iA_i^2-r_i^2}
		&=\frac{PA_1+PA_3}{(PA_1+PA_3)(PA_1PA_3-PA_2PA_4)}
		+ \frac{PA_2+PA_4}{(PA_2+PA_4)(PA_2PA_4-PA_1PA_3)}
		\\&=\frac{1}{PA_1PA_3-PA_2PA_4} + \frac{1}{PA_2PA_4-PA_1PA_3}
		\\&=\frac{1}{PA_1PA_3-PA_2PA_4} - \frac{1}{PA_1PA_3-PA_2PA_4} 
		\\&= 0
	\end{flalign*}
	as desired. 
	
	\item[\textbf{G3}] Let $ABCD$ be a convex quadrilateral whose sides $AD$ and $BC$ are not parallel. Suppose that the circles with diameters $AB$ and $CD$ meet at points $E$ and $F$ inside the quadrilateral. Let $\omega_E$ be the circle through the feet of the perpendiculars from $E$ to the lines $AB,BC$ and $CD$. Let $\omega_F$ be the circle through the feet of the perpendiculars from $F$ to the lines $CD,DA$ and $AB$. Prove that the midpoint of the segment $EF$ lies on the line through the two intersections of $\omega_E$ and $\omega_F$.
	
	\textbf{Solution.} Denote $M$ as the midpoint of $EF$. Let $P, Q, R, S$ be the perpendicular from $E$ to $AB, BC, CD, DA$ respectively. We claim that the power of point of $M$ to $\omega_E$, the circumcircle of $PQR$, is $-MP\cdot MR$. 
	
	\definecolor{uuuuuu}{rgb}{0.26666666666666666,0.26666666666666666,0.26666666666666666}
	\definecolor{ududff}{rgb}{0.30196078431372547,0.30196078431372547,1}
	\begin{tikzpicture}[line cap=round,line join=round,>=triangle 45,x=1cm,y=1cm]
	\clip(-5.464585606739619,-2.3761991570621834) rectangle (3.0700189598130034,3.757033782399586);
	\draw [shift={(-1.88,2.44)},line width=2pt]  plot[domain=-2.9546133836708823:0.18697926991891078,variable=\t]({1*2.2593804460515274*cos(\t r)+0*2.2593804460515274*sin(\t r)},{0*2.2593804460515274*cos(\t r)+1*2.2593804460515274*sin(\t r)});
	\draw [shift={(-1.54,-1.26)},line width=2pt]  plot[domain=0:3.141592653589793,variable=\t]({1*3.32*cos(\t r)+0*3.32*sin(\t r)},{0*3.32*cos(\t r)+1*3.32*sin(\t r)});
	\draw [line width=2pt,dash pattern=on 1pt off 1pt] (-1.8747179897087687,0.6528923233433481) circle (2.8243954385048053cm);
	\draw [line width=2pt] (-4.1,2.02)-- (0.34,2.86);
	\draw [line width=2pt] (0.34,2.86)-- (1.78,-1.26);
	\draw [line width=2pt] (1.78,-1.26)-- (-4.86,-1.26);
	\draw [line width=2pt] (-4.86,-1.26)-- (-4.1,2.02);
	\draw [line width=2pt] (-3.769011245925197,1.200469236860928)-- (0.203274443468224,1.5654900839943773);
	\draw [line width=2pt] (0.203274443468224,1.5654900839943773)-- (0.728311335345121,1.7489981238736816);
	\draw [line width=2pt] (0.203274443468224,1.5654900839943773)-- (-0.028445243397360778,2.7902941431410397);
	\draw [line width=2pt] (0.203274443468224,1.5654900839943773)-- (0.20327444346822401,-1.26);
	\draw [line width=2pt,dash pattern=on 1pt off 1pt] (0.15144094643721182,0.7688100561378238) circle (2.029472087834771cm);
	\draw [line width=2pt] (-3.9301374212444387,2.0521361635483495)-- (-3.769011245925197,1.200469236860928);
	\draw [line width=2pt] (-3.769011245925197,1.200469236860928)-- (-3.769011245925197,-1.26);
	\begin{scriptsize}
	\draw [fill=ududff] (-4.1,2.02) circle (2.5pt);
	\draw[color=ududff] (-4.036743017658572,2.1953309505921905) node {$A$};
	\draw [fill=ududff] (0.34,2.86) circle (2.5pt);
	\draw[color=ududff] (0.4009268472694554,3.0309433748839396) node {$B$};
	\draw [fill=ududff] (1.78,-1.26) circle (2.5pt);
	\draw[color=ududff] (1.8449949203173326,-1.0822168107074845) node {$C$};
	\draw [fill=ududff] (-4.86,-1.26) circle (2.5pt);
	\draw[color=ududff] (-4.7912280221161705,-1.0822168107074845) node {$D$};
	\draw[color=black] (-1.3838763890818533,0.36996400432380727) node {$c$};
	\draw [fill=uuuuuu] (0.203274443468224,1.5654900839943773) circle (2pt);
	\draw[color=uuuuuu] (0.27112297553481474,1.7247919155541187) node {$E$};
	\draw [fill=uuuuuu] (-3.769011245925197,1.200469236860928) circle (2pt);
	\draw[color=uuuuuu] (-3.704120596338555,1.3597185263004419) node {$F$};
	\draw [fill=uuuuuu] (-1.88,2.44) circle (2pt);
	\draw[color=uuuuuu] (-1.8138517142028505,2.6009680497629426) node {$X$};
	\draw [fill=uuuuuu] (-1.54,-1.26) circle (2pt);
	\draw[color=uuuuuu] (-1.4731165508994188,-1.0984422946743146) node {$Y$};
	\draw [fill=uuuuuu] (-1.7828684012284866,1.3829796604276527) circle (2pt);
	\draw[color=uuuuuu] (-1.7164988104018701,1.5381988499355728) node {$M$};
	\draw [fill=uuuuuu] (-0.028445243397360778,2.7902941431410397) circle (2pt);
	\draw[color=uuuuuu] (0.03585345801577858,2.949815955049789) node {$P$};
	\draw [fill=uuuuuu] (0.728311335345121,1.7489981238736816) circle (2pt);
	\draw[color=uuuuuu] (0.7903384624733774,1.9032722391892496) node {$Q$};
	\draw [fill=uuuuuu] (0.20327444346822401,-1.26) circle (2pt);
	\draw[color=uuuuuu] (0.27112297553481474,-1.0984422946743146) node {$R$};
	\draw[color=black] (-3.1037776895658418,3.01471789091711) node {$\omega_E$};
	\draw [fill=uuuuuu] (-1.8747179897087687,0.6528923233433481) circle (2pt);
	\draw[color=uuuuuu] (-1.7895134882526054,0.8323902973784644) node {$O_E$};
	\draw [fill=uuuuuu] (-3.9301374212444387,2.0521361635483495) circle (2pt);
	\draw[color=uuuuuu] (-3.8663754360068556,2.2115564345590206) node {$U$};
	\draw [fill=uuuuuu] (-3.769011245925197,-1.26) circle (2pt);
	\draw[color=uuuuuu] (-3.704120596338555,-1.0984422946743146) node {$V$};
	\end{scriptsize}
	\end{tikzpicture}
	
	We first notice that $\angle EPB=\angle BQE=90^{\circ}$ and $P, B, Q, E$ lie on a circle. Using directed angle to prevent case distinction we get 
	\[
	\angle(PQ, QE) = \angle(PB, BE)=\angle(AB, BE)
	\]
	and similarly 
	\[
	\angle(RQ, QE)=\angle(RC, CE)=\angle(DC, CE)
	\]
	and therefore 
	\[
	\angle(PQ, QR)=\angle(PQ, QE)+\angle(QE, QR)=\angle(AB, BE)+\angle(CE, DC)
	\]
	
	Denote, now, $X$ as midpoint of $AB$ and $Y$ as midpoint of $CD$, which means that $X$ is the center of circle containing $A, B, E, F$. This means 
	\[
	\angle(AX, XE)=2\angle(AB, BE)
	\]
	and similarly $\angle(DY, YE)=2\angle(DC, CE)$. In addition, line $XY$ is the perpendicular bisector of line $EF$ since $XE=XF$ and $YE=YF$, so $\angle XAE=\angle XME=90^{\circ}$, i.e. $X, A, M, E$ concyclic. 
	Therefore $\angle(AX, XE)=\angle(PX, XE)=\angle(PM, ME)$, and similarly $\angle(DY, YE)=\angle (RM, ME)$. 
	
	If $O_E$ is the center of $\omega_E$, then given that $P, Q, R$ all lie on $\omega_E$ we have 
	\[
	\angle(PO_E, O_ER)=2\angle(PQ, QR)=2(\angle(AB, BE)+\angle(CE, DC))
	=\angle(AX, XE)+\angle(EY, YD)
	\]\[
	=\angle(PM, ME)+\angle(EM, MR)=\angle(PM, MR)
	\]
	showing that $O_E, P, R, M$ are also concyclic. The fact that $O_EP=O_ER$ means line $MO_E$ is an angle bisector of $\angle PMR$. 
	
	Using angle chasing (that is, ditching directed angles and use conventional angles) we also get that each $M$ and $O_E$ are on the same side as $Q$ w.r.t. $PR$ if and only if $\angle PQR<90^{\circ}$, so regardless of $\angle PQR$, $M$ and $O_E$ are always on the same side w.r.t. $PR$. 
	This means, $MO_E$ will always be external angle bisector of $\angle PMR$. This means that the power of point of $M$ w.r.t. $\omega_E$, which is $MO_E^2-O_EP^2$, is negative. 
	
	Let $a=\angle MPR$ and $b=\angle MRP$, and $d$ the diameter of circle $P, Q, O_M, E$. Then $MR=d\sin a$ and $PM=d\sin b$. We would also have $MO_E=d\sin|\frac{a-b}{2}|$ and $OE_P=OE_Q=d\sin\frac{a+b}{2}$. 
	Therefore, 
	\[
	MO_E^2-O_EP^2=d^2(\sin^2|\frac{a-b}{2}|-\sin^2\frac{a+b}{2})
	=-4d^2\sin\frac{a}{2}\sin\frac{b}{2}\cos\frac{a}{2}\cos\frac{b}{2}
	\]\[
	=-d^2\sin a\sin b
	=-PM\cdot MR
	\]
	as desired. 
	
	Similarly, if the perpendicular from $F$ to $AB$ and $CD$ are $U$ and $V$ respectively then the power of point from $M$ to $\omega_F$ is $-MU\cdot MV$. Notice that both $EP$ and $FU$ are perpendicular to $AB$, hence parallel to each other. The 'midline' between the two line (that is, the line equidistant to the two aforementioned parallel lines) passes through the midpoint of $EF$ (i.e. $M$), and with $PU$ perpendicular to these parallel lines we have $MP=MU$ and similarly $MR=MV$. Therefore $MP\cdot MR=MU\cdot MV$, so $M$ lies on the radical axis of $\omega_E$ and $\omega_F$, as desired. 
	
	\item[\textbf{G4}]
	Let $ABC$ be an acute triangle with circumcircle $\Omega$. Let $B_0$ be the midpoint of $AC$ and let $C_0$ be the midpoint of $AB$. Let $D$ be the foot of the altitude from $A$ and let $G$ be the centroid of the triangle $ABC$. Let $\omega$ be a circle through $B_0$ and $C_0$ that is tangent to the circle $\Omega$ at a point $X\not= A$. Prove that the points $D,G$ and $X$ are collinear.
	
	\textbf{Solution.} 
	W.l.o.g. suppose that $AB<AC$. 
	We first determine the ratio $\frac{BX}{XC}$. 
	Observe, first, that $\frac{AB}{AC}=\frac{C_0X}{B_0X}$. 
	Indeed: the circle tangency condition implies that $\angle BXC_0=\angle CXB_0$, 
	and 
	\[
	\frac 12 = \frac{BC_0}{BA}=\frac{XC_0\sin\angle BXC_0}{AX\sin\angle BXA}=\frac{XC_0\sin\angle BXC_0}{AX\sin\angle BCA}
	\]
	and similarly 
	$\frac 12 = \frac{XB_0\sin\angle CXB_0}{AX\sin\angle CBA}$. 
	Comparing the two ratio, we have $\frac{XC_0}{XB_0}=\frac{\sin\angle BCA}{\sin\angle ABC}=\frac{AB}{AC}$. 
	
	Next, Stewart's theorem says that $AC_0^2 + C_0B^2 + 2XC_0^2 = BX^2 + AX^2$, 
	and similarly $AB_0^2 +B_0C^2+2XB_0^2 = CX^2 + AX^2$. 
	Since $\frac{AC_0}{AB_0}=\frac{C_0B}{B_0C}=\frac{C_0X}{B_0X}=\frac{AB}{AC}$, 
	we deduce that $\frac{BX^2+AX^2}{CX^2 + AX^2}=\frac{AB^2}{AC^2} = \frac{BX^2+AX^2 - 2BX\cdot AX \cdot \cos\angle BXA}{CX^2 + AX^2- 2CX\cdot AX \cdot \cos\angle CXA}$. 
	the latter by cosine rule. 
	Doing cross multiplication on the fraction terms, we get 
	\begin{equation}
		\frac{AB^2}{AC^2} = 
		\frac{BX\cdot AX \cdot \cos\angle BXA}{CX\cdot AX \cdot \cos\angle CXA}
		=\frac{BX \cdot \cos\angle BCA}{CX \cdot \cos\angle CBA}
	\end{equation}
    so it follows that 
    $\frac{BX}{CX} = \frac{AB^2 \cdot \cos\angle CBA}{AC^2\cos\angle BCA}$. 
    
    
    Now let $A_0$ be the midpoint of $BC$. 
    We see that $A_0DC_0B_0$ is an isoceles trapezoid. 
    Consider the homothety centered at $G$ with scale $-2$ (which brings $B_0$ to $B$, $C_0$ to $C$, and $A_0$ to $A$). 
    Suppose that it brings $D$ to $T$, 
    then $ATCB$ is also an isoceles trapezoid, so $A, T, C, B, X$ are concyclic. 
    In addition, we have 
    \begin{equation}
    	\frac{|\triangle TBX|}{|\triangle TCX|}
    	=\frac{TB\cdot BX\cdot \sin\angle TBX}{TC\cdot CX\cdot \sin\angle TCX}
    	=\frac{AC\cdot AB^2 \cdot \cos\angle CBA}{AB\cdot AC^2\cdot\cos\angle BCA}
    	=\frac{AB\cos\angle CBA}{AC\cos\angle BCA}
    \end{equation}
    where we used $\angle TBX+\angle TCX=180^{\circ}$ (and therefore the sine of the angles are equal). 
    Meanwhile, 
    $\frac{BD}{DC}=\frac{AB\sin\angle BAD}{AC\sin\angle CAD}
    =\frac{AB\cos\angle ABC}{AC\sin\angle ACB}$. 
    Thus we have $\frac{|\triangle TBX|}{|\triangle TCX|}=\frac{BH}{HC}$, 
    so $TX$ passes through $D$. 
    Since $D, G, T$ are collinear (by the definition of homothety), 
    $X$ also lies on this line. 
	
	\item[\textbf{G5}] Let $ABC$ be a triangle with incenter $I$ and circumcircle $\omega$. Let $D$ and $E$ be the second intersection points of $\omega$ with the lines $AI$ and $BI$, respectively. The chord $DE$ meets $AC$ at a point $F$, and $BC$ at a point $G$. Let $P$ be the intersection point of the line through $F$ parallel to $AD$ and the line through $G$ parallel to $BE$. Suppose that the tangents to $\omega$ at $A$ and at $B$ meet at a point $K$. Prove that the three lines $AE, BD,$ and $KP$ are either parallel or concurrent.
	
	\textbf {Solution}: 
	We first establish this. 
	
	\begin{lemma}
		$K, I, P$ are collinear. 
	\end{lemma}
	
	\begin{proof}
		First notice that $\angle DEB=\angle DAB=\frac{\angle A}{2}=\angle CAD=\angle CED$ and similarly $\angle CDE=\frac{\angle B}{2}=\angle EDA$. Therefore $\triangle CED\cong\triangle IED$ and $ED$ is the perpendicular bisector of segment $CI$. This entails that $CF=CG$ and $\angle CFG=\angle CGF=\angle IFG=\angle IGF=90^{\circ}-\frac{\angle C}{2}.$Knowing that $FP\parallel AI$ and $GP\parallel BI$, we want to prove that $\angle FPI=\angle AIK$, or equivalently $\angle GPI=\angle BIK$. It suffices to prove that $\frac{\sin\angle FPI}{\sin\angle GPI}=\frac{\sin\angle AIK}{\sin\angle BIK}$.
		
		\usetikzlibrary{arrows}
		\pagestyle{empty}
		\definecolor{uuuuuu}{rgb}{0.26666666666666666,0.26666666666666666,0.26666666666666666}
		\definecolor{qqqqff}{rgb}{0.3333333333333333,0.3333333333333333,0.3333333333333333}
		\begin{tikzpicture}[line cap=round,line join=round,>=triangle 45,x=1.0cm,y=1.0cm]
			\clip(-3.527272727272727,-3.39454545454545) rectangle (17.381818181818183,6.732727272727267);
			\draw(2.162839829942502,1.4242993272629518) circle (3.618338460700848cm);
			\draw (0.6181818181818187,4.696363636363633)-- (3.861297172049131,-1.7706366459970866);
			\draw (-0.7636363636363631,-0.7036363636363628)-- (3.5405214941984724,4.770097654741744);
			\draw (3.5405214941984724,4.770097654741744)-- (3.861297172049131,-1.7706366459970866);
			\draw (0.6181818181818187,4.696363636363633)-- (5.5636363636363635,2.66);
			\draw (-0.7636363636363631,-0.7036363636363628)-- (5.5636363636363635,2.66);
			\draw (0.6181818181818187,4.696363636363633)-- (-0.7636363636363631,-0.7036363636363628);
			\draw (3.604442394979802,3.466726928270348)-- (4.190404128514835,2.298277988671306);
			\draw (3.6927826311665535,1.6654369525663366)-- (4.190404128514835,2.298277988671306);
			\draw (3.604442394979802,3.466726928270348)-- (1.733588662509991,2.472163880836684);
			\draw (1.733588662509991,2.472163880836684)-- (3.6927826311665535,1.6654369525663366);
			\draw (0.6181818181818187,4.696363636363633)-- (-3.3336448230410487,2.830807184592077);
			\draw (-3.3336448230410487,2.830807184592077)-- (-0.7636363636363631,-0.7036363636363628);
			\draw [domain=0.6181818181818187:17.381818181818183] plot(\x,{(--13.67878875800446--0.07373401837811144*\x)/2.922339676016654});
			\draw [domain=-0.7636363636363631:17.381818181818183] plot(\x,{(-4.069071630730523-1.0670002823607239*\x)/4.624933535685494});
			\draw [dash pattern=on 5pt off 5pt,domain=-3.3336448230410487:17.381818181818183] plot(\x,{(--19.523868632187387-0.532529195920771*\x)/7.5240489515558835});
			\begin{scriptsize}
				\draw [fill=qqqqff] (0.6181818181818187,4.696363636363633) circle (2.5pt);
				\draw[color=qqqqff] (0.7454545454545459,5.02363636363636) node {$A$};
				\draw [fill=qqqqff] (-0.7636363636363631,-0.7036363636363628) circle (2.5pt);
				\draw[color=qqqqff] (-0.6363636363636358,-0.37636363636363585) node {$B$};
				\draw [fill=qqqqff] (5.5636363636363635,2.66) circle (2.5pt);
				\draw[color=qqqqff] (5.6909090909090905,2.987272727272725) node {$C$};
				\draw [fill=uuuuuu] (1.733588662509991,2.472163880836684) circle (1.5pt);
				\draw[color=uuuuuu] (1.854545454545455,2.7327272727272707) node {$I$};
				\draw [fill=uuuuuu] (3.861297172049131,-1.7706366459970866) circle (1.5pt);
				\draw[color=uuuuuu] (3.9818181818181824,-1.5218181818181804) node {$D$};
				\draw [fill=uuuuuu] (3.5405214941984724,4.770097654741744) circle (1.5pt);
				\draw[color=uuuuuu] (3.6727272727272733,5.02363636363636) node {$E$};
				\draw [fill=uuuuuu] (3.604442394979802,3.466726928270348) circle (1.5pt);
				\draw[color=uuuuuu] (3.727272727272728,3.7145454545454517) node {$F$};
				\draw [fill=uuuuuu] (3.6927826311665535,1.6654369525663366) circle (1.5pt);
				\draw[color=uuuuuu] (3.818181818181819,1.9145454545454532) node {$G$};
				\draw [fill=uuuuuu] (-3.3336448230410487,2.830807184592077) circle (1.5pt);
				\draw[color=uuuuuu] (-3.2,3.078181818181816) node {$K$};
				\draw [fill=uuuuuu] (4.190404128514835,2.298277988671306) circle (1.5pt);
				\draw[color=uuuuuu] (4.3090909090909095,2.550909090909089) node {$P$};
			\end{scriptsize}
		\end{tikzpicture}
	    
	    Recall that $FP\parallel AI\Rightarrow \angle CFP=\angle CAI$, $GP\parallel BI\Rightarrow\angle CPG=\angle CBI$. 
	    Thus $CFIG$ is a rhombus and $FI=GI$. 
	    This gives 
	    \[
	    \frac{\sin\angle FPI}{\sin\angle GPI}=\frac{FI}{GI}\cdot\frac{\sin\angle PFI}{\sin\angle PGI}=\frac{\sin (\angle CFI-\angle CFP)}{\sin(\angle CGI-\angle CGP)}=\dfrac{\sin (180^{\circ}-\angle C-\frac{\angle A}{2})}{\sin (180^{\circ}-\angle C-\frac{\angle B}{2})}=\dfrac{\sin (\angle C+\frac{\angle A}{2})}{\sin (\angle C+\frac{\angle B}{2})}
	    \]
	    Given also that $AK=BK$, we have 
	    \[\frac{\sin\angle AIK}{\sin\angle BIK}=\frac{AK}{BK}\cdot\frac{\sin\angle IAK}{\sin\angle IBK}=\dfrac{\sin (\angle C+\frac{\angle A}{2})}{\sin (\angle C+\frac{\angle B}{2})}=\frac{\sin\angle FPI}{\sin\angle GPI},\] as desired. 
	\end{proof}
	
	For convenience we assume that $AE$ and $BD$ are not parallel; the limit case (i.e. parallel) happens when $\angle C=60^{\circ}$ and quadrilateral $AIBK$ is cyclic, which is left as an exercise (no trigo needed because angle chasing method becomes straightforward). Now let $AE$ and $BD$ intersect at $Q$. We need $\frac{\sin\angle AKQ}{\sin\angle BKQ}=\frac{\sin\angle AKI}{\sin\angle BKI}$, or $\frac{AQ}{BQ}\cdot\frac{\sin\angle QAK}{\sin\angle QBK}=\frac{AI}{BI}\cdot\frac{\sin\angle IAK}{\sin\angle IBK}$. Now $\frac{AQ}{BQ}=\frac{\sin\angle QBA}{\sin\angle QAB}=\frac{\sin\angle DBA}{\sin\angle EAB}=\frac{DA}{EB}$ and $\angle QAK=\angle EAK=180^{\circ}-\angle EBA$ so $\sin\angle EAK=\sin\angle EBA$. Similarly $\sin\angle QBK=\sin\angle DAB$. Thus, $$\frac{\sin\angle QAK}{\sin\angle QBK}=\frac{\sin\angle EBA}{\sin\angle DAB}=\frac{AI}{BI}=\dfrac{\sin\frac{\angle B}{2}}{\sin\frac{\angle A}{2}}$$ and $\frac{\sin\angle IAK}{\sin\angle IBK}=\frac{\sin\angle DAK}{\sin\angle EBK}=\frac{AD}{EB}.$ Therefore $\frac{AQ}{BQ}\cdot\frac{\sin\angle QAK}{\sin\angle QBK}=\frac{\sin\frac{\angle B}{2}}{\sin\frac{\angle A}{2}}\cdot\frac{AD}{EB}=\frac{AI}{BI}\cdot\frac{\sin\angle IAK}{\sin\angle IBK}$. $\blacksquare$
	
	\item[\textbf{G6}] Let $ABC$ be a triangle with $AB=AC$ and let $D$ be the midpoint of $AC$. The angle bisector of $\angle BAC$ intersects the circle through $D,B$ and $C$ at the point $E$ inside the triangle $ABC$. The line $BD$ intersects the circle through $A,E$ and $B$ in two points $B$ and $F$. The lines $AF$ and $BE$ meet at a point $I$, and the lines $CI$ and $BD$ meet at a point $K$. Show that $I$ is the incentre of triangle $KAB$.
	
	\textbf{Solution.} To show that $BI$ bisects $\angle ABK$ is the same as showing that $BE$ bisects $\angle ABD$. If $D'$ is the second intersection of $AB$ and the circle $BEDC$ then $D$ and $D'$ are symmetric to each other in the line $AE$, hence $ED=ED'$ and so $\angle ABE=\angle D'BE=\angle EBD$, establishing the claim.
	
	To finish the rest, we consider a third circle centered at $D$ and passes through $C$ and $A$. We claim that $F$ is on this circle too by showing that $DA=DF$. Now by angle chasing we have $\angle ADB=180^{\circ}-\angle BDC=180^{\circ}-\angle BEC=refl\angle BEC - 180^{\circ}=2\angle AEB-180^{\circ}=2\angle AFB-180^{\circ}=2(180^{\circ}-\angle AFD)-180^{\circ}=180^{\circ}-2\angle AFD$ (the refl notation means reflex angle). 
	Thus, $\angle ADB=180^{\circ}-2\angle AFD$ and we have $\angle AFD+\angle FAD=2\angle AFD$, so $\angle AFD=\angle FAD$ so $DA=DF$, as claimed. 
	
	Now, we consider the three circles: $BEDC$, $AEFB$ and $AFC$, with the first two given in the problem and third defined on our own. The first two circles have $BE$ as the radical axis, and the last two has $AF$ as the radical axis. Hence, the intersection, $I$, is the radical center of the three circles. Since $C$ is on both circles $BEDC$ and $AFC$, $CI$ is the radical axis of these two circles. This means, if $O$ is the center of the circle $BEDC$ then $DO\perp CI$. Since $O$ is the center of $BDC$ we have $\angle CDO+\angle DBC=90^{\circ}$; since $DO\perp CI$ we have $\angle CDO+\angle DCI=90^{\circ}$. 
	Therefore $\angle DBC=\angle DCI=\angle DCK$, and equivalently, $\angle DKC=\angle DCB$. 
	Knowing this, the last part is to observe the following: 
	\begin{flalign*}
		\angle ABI+\angle FIK&=\frac{\angle ABF}{2}+(180^{\circ}-\angle IFK-\angle IKF)
		\\&=\frac{\angle ABC-\angle DBC}{2}+(180^{\circ}-\angle AFD-\angle DKC)
		\\&=\frac{\angle ABC-\angle DBC}{2}+(180^{\circ}-\frac{180^{\circ}-\angle ADF}{2}-\angle DCB)
		\\&=\frac{\angle ABC-(180^{\circ}-\angle BDC - \angle DCB)}{2}+(180^{\circ}-\frac{\angle BDC}{2}-\angle DCB)
		\\&=90^{\circ}-\frac{\angle DCB}{2}+\frac{\angle ABC}{2}
		\\&= 90^{\circ}
	\end{flalign*}
	which shows that the line $KI$ passes through the circumcenter $O'$ of the triangle $AIB$. 
	Thus considering the circumcircle of triangle $AO'B$, we see that, if $K'$ is the second intersection of the line $KI$ with this circle (other then $O'$) then $I$ is the incenter of triangle $K'AB$, and therefore $\angle ABI=\angle IBK'$ but as shown before, $\angle ABI=\angle IBK$, so $\angle IBK=\angle IBK'$ and since both $K$ and $K'$ are on $KI$, we have $K=K'$, showing that $I$ is indeed the incenter of $KAB$. 
	
	\item [\textbf{G7}] Let $ABCDEF$ be a convex hexagon all of whose sides are tangent to a circle $\omega$ with centre $O$. Suppose that the circumcircle of triangle $ACE$ is concentric with $\omega$. Let $J$ be the foot of the perpendicular from $B$ to $CD$. Suppose that the perpendicular from $B$ to $DF$ intersects the line $EO$ at a point $K$. Let $L$ be the foot of the perpendicular from $K$ to $DE$. Prove that $DJ=DL$.
	
	\textbf{Solution.} Let the circumradius of $ACE$ to be $R$, and the radius of $\omega$ to be $r$. Then the distance of sides $AB, BC, CD, DE, EF, FA$ to $O$ is $r$ and therefore we have $\cos\angle BAO=\cos \angle OAF=\cos \angle OEF=\cos \angle OED=\cos \angle DCO=\cos \angle OCB=\frac{r}{R}$, so $\angle BAO=\angle OAF=\angle OEF=\angle OED=\angle DCO=\angle OCB$; denote this common angle as $\theta$. 
	We also have $BA=BC$, $DC=DE$, $FE=FA$: we have $OC=OA$ and $\angle OCB=\angle OCA$ so it's not hard to see that $CBO$ and $ABO$ must be congruent. The other equivalences can be established similarly. 
	Having known these, consider $DJ$ and $DL$ in terms of signed lengths, and let $CJ$ be positive iff $J$ is on ray $DC$ beyond $C$; similarly let $FL$ be positive iff $L$ is on ray $DF$ beyond $F$. Then we have $DJ=DC+CJ$ and $DL=DE+EL$, so it suffices to prove that $CJ=EL$. 
	Moreover, in terms of unsigned length we have $|CJ|=|BC|\cdot |\cos BCJ|$ and we can infer that $CJ>0$ iff $\cos BCJ>0$ ($\cos BCD<90^{\circ}$ means that $J$ will be on the same side as $D$ w.r.t. $C$ so $CJ<0$ in this case). 
	Thus $CJ=BC\cdot \cos BCJ=-BC\cos BCD=-BC\cos 2\theta$. 
	Also since $\theta<90^{\circ}$, if $K$ extends beyond $OE$ then $\angle KEL=\angle OED=\theta$ and in this case $EL=EK\cos\theta > 0$, so $EL>0$ iff $K$ extends beyond $OE$, so we have to do our computation in that direction. 
	
	The immediate previous argument means that when considering the lengths $DK$ and $FK$ we have $\angle DEK=180^{\circ}-\theta$ so 
	\[DK^2=DE^2+EK^2-2DE\cdot EK\cdot \cos(180^{\circ}-\theta)=DE^2+EK^2+2DE\cdot EK\cdot \cos\theta\] and \[FK^2=FE^2+EK^2+2FE\cdot EK\cdot \cos\theta\] 
	We can then conclude that 
	\[
	DK^2-FK^2=DE^2-FE^2+2EK\cos\theta(DE-FE)
	\]
	The condition $BK\perp DF$ also implies $BD^2-BF^2=KD^2-KF^2$ but according to cosine rule, we have \[BD^2=BC^2+CD^2-2\cdot BC\cdot CD\cdot \cos 2\theta\qquad
	  BF^2=BA^2+AF^2-2\cdot BA\cdot AF\cdot \cos 2\theta
	\]
	and bearing in mind that $BC=BA$, the difference in square becomes 
	\[BD^2-BF^2 = CD^2-AF^2-2\cdot BC \cos 2\theta (CD-AF)
	\]
	and coupling with the fact that $CD=DE$ and $AF=FE$, and that the two sum of squares are the same we have 
	$DE^2-FE^2+2EK\cos\theta(DE-FE)=CD^2-AF^2-2\cdot BC \cos 2\theta (CD-AF)$ so $EK\cos\theta(DE-FE)=-BC \cos 2\theta (DE-FE)$. We may assume that $CD\neq AF$, otherwise $K$ won't be uniquely determined. This allows us to cancel $DE-FE$ from the equation and we have $EK\cos\theta = -BC\cos 2\theta$. 
	Finally, recall that $CJ=-BC\cos 2\theta$ and $EL=EK\cos\theta$ so we in turn have $EL=EK\cos\theta = -BC\cos 2\theta=CJ$, as desired. 
	
	
	\item[\textbf{G8}] (IMO 6)
	Let $ABC$ be an acute triangle with circumcircle $\Gamma$. Let $\ell$ be a tangent line to $\Gamma$, and let $\ell_a, \ell_b$ and $\ell_c$ be the lines obtained by reflecting $\ell$ in the lines $BC$, $CA$ and $AB$, respectively. Show that the circumcircle of the triangle determined by the lines $\ell_a, \ell_b$ and $\ell_c$ is tangent to the circle $\Gamma$.
	
	\textbf {Solution.} (cited directly from my notes for the IMO training camp). 
	
	\usetikzlibrary{arrows}
	
	\definecolor{uuuuuu}{rgb}{0.26666666666666666,0.26666666666666666,0.26666666666666666}
	\definecolor{xdxdff}{rgb}{0.6588235294117647,0.6588235294117647,0.6588235294117647}
	\definecolor{qqqqff}{rgb}{0.3333333333333333,0.3333333333333333,0.3333333333333333}
	\begin{tikzpicture}[line cap=round,line join=round,>=triangle 45,x=0.5cm,y=0.5cm]
	\clip(-2.787913411858677,-11.505541434031079) rectangle (34.61700525997879,6.611449609632774);
	\fill[color=uuuuuu,fill=uuuuuu,fill opacity=0.1] (8.989011148746778,4.700989741570521) -- (-0.5368788928682983,-1.1794237425742722) -- (11.11957705177755,-6.822854600128423) -- cycle;
	\draw(3.9505227834021883,1.0865696007367804) circle (1.6076404873839587cm);
	\draw [dotted] (6.386536470021772,-1.7390394147486157) circle (3.4729976240866263cm);
	\draw (-1.9026785109951698,-2.022543540932911)-- (10.273116764048051,-2.244504508564252);
	\draw (2.800647559664868,4.0892048960298295)-- (1.3261535628491259,-2.081404147180008);
	\draw (2.800647559664868,4.0892048960298295)-- (10.273116764048051,-2.244504508564252);
	\draw (-1.9026785109951698,-2.022543540932911)-- (7.100984717263056,0.4442135079049559);
	\draw (8.989011148746778,4.700989741570521)-- (-1.9026785109951698,-2.022543540932911);
	\draw (-0.5368788928682983,-1.1794237425742722)-- (11.11957705177755,-6.822854600128423);
	\draw (8.989011148746778,4.700989741570521)-- (11.11957705177755,-6.822854600128423);
	\draw [color=uuuuuu] (8.989011148746778,4.700989741570521)-- (-0.5368788928682983,-1.1794237425742722);
	\draw [color=uuuuuu] (-0.5368788928682983,-1.1794237425742722)-- (11.11957705177755,-6.822854600128423);
	\draw [color=uuuuuu] (11.11957705177755,-6.822854600128423)-- (8.989011148746778,4.700989741570521);
	\draw (2.800647559664868,4.0892048960298295)-- (11.11957705177755,-6.822854600128423);
	\draw (-0.5368788928682983,-1.1794237425742722)-- (6.541355622787623,-0.8175356865481822);
	\draw (6.541355622787623,-0.8175356865481822)-- (8.989011148746778,4.700989741570521);
	\begin{scriptsize}
	\draw [fill=qqqqff] (2.800647559664868,4.0892048960298295) circle (2.5pt);
	\draw[color=qqqqff] (3.099295526926176,4.692414651686334) node {$A$};
	\draw [fill=qqqqff] (1.5673936018816261,-1.071838852473516) circle (2.5pt);
	\draw[color=qqqqff] (1.79825487747096,-0.4792219298981409) node {$B$};
	\draw [fill=qqqqff] (7.100984717263056,0.4442135079049559) circle (2.5pt);
	\draw[color=qqqqff] (7.42525568636477,0.6917146546115516) node {$C$};
	\draw [fill=xdxdff] (3.8919189463350277,-2.1281772535249255) circle (2.5pt);
	\draw[color=xdxdff] (4.107602030253969,-1.5525804656986921) node {$T$};
	\draw [fill=uuuuuu] (-1.9026785109951698,-2.022543540932911) circle (1.5pt);
	\draw[color=uuuuuu] (-1.8771849572400259,-1.455002416989551) node {$A'$};
	\draw [fill=uuuuuu] (1.3261535628491259,-2.081404147180008) circle (1.5pt);
	\draw[color=uuuuuu] (1.635624796289058,-1.6176324981714527) node {$C'$};
	\draw [fill=uuuuuu] (10.273116764048051,-2.244504508564252) circle (1.5pt);
	\draw[color=uuuuuu] (10.61280527753005,-1.7802625793533546) node {$B'$};
	\draw [fill=uuuuuu] (-0.5368788928682983,-1.1794237425742722) circle (1.5pt);
	\draw[color=uuuuuu] (-0.8038264214394724,-0.5767999786072818) node {$B_1$};
	\draw [fill=uuuuuu] (8.989011148746778,4.700989741570521) circle (1.5pt);
	\draw[color=uuuuuu] (9.279238611838453,5.245356927704799) node {$C_1$};
	\draw [fill=uuuuuu] (11.11957705177755,-6.822854600128423) circle (1.5pt);
	\draw[color=uuuuuu] (11.425955683439561,-6.2688528199738425) node {$A_1$};
	\draw [fill=uuuuuu] (6.541355622787623,-0.8175356865481822) circle (1.5pt);
	\draw[color=uuuuuu] (6.6121052804552605,-1.1947942870985084) node {$I$};
	\end{scriptsize}
	\end{tikzpicture}
	
	Denote $A_1$, $B_1$, $C_1$ as $\ell_b\cap \ell_c$, $\ell_a\cap \ell_c$, $\ell_a\cap \ell_b,$ respectively. 
	
	\begin{lemma}
		$AA_1$, $BB_1$, $CC_1$ concur on $\Gamma$, 
		and the common point $I$ is the incenter of triangle $A_1B_1C_1$. 
	\end{lemma}
	
	\begin{proof}
		Now denote the intersection of $BC, CA, AB$ with $\ell$ as $A',B'$ and $C'$, respectively. let us consider triangle $B_1A'C'$. Since $A'B$ bisects $\angle B_1A'C'$ and $C'B$ bisects $\angle B_1C'A'$, we know that $B$ is either the incenter or excenter of $B_1A'C'$ and it follows that $B_1B$ bisects angle $A'B_1C'$, which then follows that $BB_1$ passes through the circumentre of $BA'C'$. Using th notation of directed angles it follows that $\angle (BB_1,AB)=\angle (BB_1,BC')=90^{\circ}-\angle (A'C',A'B)=90^{\circ}-\angle (\ell, BC)$ and $\angle (BB_1,BC)=\angle (BB_1,BA')=90^{\circ}-\angle (A'C',C'B)=90^{\circ}-\angle (\ell,AB)$. In a similar way, $\angle (CC_1,AC)=90^{\circ}-\angle (\ell, BC)$ and $\angle (CC_1,BC)=90^{\circ}-\angle (\ell, AC).$ Now $\angle (BB_1,CC_1)=\angle (BB_1, BC)-\angle (CC_1, BC)$=$(90^{\circ}-\angle (\ell,AB))-(90^{\circ}-\angle (\ell,AC))=\angle (AB, AC)$, yielding that $BB_1$ and $CC_1$ indeed intersect on $\Gamma$. The concurrence of $AA_1, BB_1$ and $CC_1$ follows from the fact that they intersect at either the incenter or excenter of triangle $A_1B_1C_1$, hence we are done.
		
		Now we need the fact that $I$ is the incenter (not excentre opposite to any of the sides), given that $ABC$ is an acute triangle. Now let's investigate the relationship between $\angle A$, $\angle B$, $\angle C$ (of $\triangle ABC$) and $\angle A_1$, $\angle B_1$, $\angle C_1$ (of $\triangle A_1B_1C_1$). Take $B$ again. If $\angle A'BC'<90^{\circ}$ then $\angle B=\angle A'BC'$, and it can be verified that $B$ is now the excentre of $\triangle B_1A'C'$ and $\angle B_1A'C'=180^{\circ}-2\angle A'BC=180^{\circ}-2\angle B$. On the other hand, if $\angle A'BC$ is obtuse then $\angle B=180^{\circ}-\angle A'BC$ and $B$ is the incenter of $\triangle B_1A'C'$ so $\angle B_1A'C'=2\angle A'BC-360^{\circ}=2(180^{\circ}-\angle B)-180^{\circ}=180^{\circ}-2\angle B.$ So $\angle B_1$ is either $2\angle B$ or $180^{\circ}-2\angle B$, depending whether $BB_1$ is the external or internal angle bisector of angle $B_1$, respectively. Same goes for line angles $A_1$ and $C_1$. If $I$ is the excentre, then exactly two of the lines $AA_1$, $BB_1$, $CC_1$ are the external angle bisectors of the respective angles. So let say $AA_1$ and $BB_1$ are the external angle bisectors of angles $A_1$ and $B_1$ then angles $A_1$, $B_1$, $C_1$ are $2\angle A$, $2\angle B$ and $180^{\circ}-2\angle C$ respectively. As the three angles sum up to $180^{\circ}$ and $\angle A+\angle B+\angle C=180^{\circ},$ comparing the two equations yields $\angle C=90^{\circ}$ (contradiction).
	\end{proof}
	
	
	Now let's proceed to the solving of the main problem. Denote $T$ as the tangent point of $\ell$ and $\Gamma$, and $a,b,c$ as $\angle (TA,\ell),\angle (TB,\ell),\angle (TC,\ell)$. Speaking in modulo $180^{\circ}$, $\angle A$ is congruent to $b-c$, so $\sin\angle A$=$|\sin (b-c)|$. Similarly $\sin\angle B$=$|\sin (c-a)|$ and $\sin\angle C$=$|\sin (a-b)|.$ (we can change the $\pm$ sign to modulus, though).
	Let's now show the following: 
	
	\begin{lemma}
		In unsigned length, 
		among the three quantities 
		$A_1B_1\cdot t(C_1)$, $C_1A_1\cdot t(B_1)$, $B_1C_1\cdot t(A_1)$, 
		one of them is the sum of the other two, 
		where $t(K)$ is the length of tangent from point $K$ to $\Gamma$.
	\end{lemma}
	
	\begin{proof}
		Notice that it is the ratio $A_1B_1\cdot t(C_1)$ : $C_1A_1\cdot t(B_1)$ : $B_1C_1\cdot t(A_1)$ that matters, so we can divide each of them by a constant whenever necessary.
		By power of point theorem we have 
		\[t(A_1)=\sqrt {A_1I\cdot A_1A}\] 
		Define $r$ as the inradius of triangle $A_1B_1C_1$ and $r_A$ the distance from $A$ to lines $A_1B'$, $A_1C'$ and $B'C'$ (or $\ell$) (the distance to the three lines are the same because $A$ is the incentre or excentre of $A_1B'C'$.)
		Now 
		\[A_1A=\dfrac{r_A}{\sin AA_1B'}=\dfrac{r_A}{\sin (90^{\circ}-\angle A)}=\dfrac{r_A}{\cos \angle A}\]
		as $\angle A_1=180^{\circ}-2\angle A.$ Similarly $\angle A_1I=\dfrac{r}{\cos \angle A}$ so $t(A_1)=\sqrt {A_1I\cdot A_1A}=\dfrac{\sqrt{r\cdot r_A}}{\cos \angle A}$. Now $B_1C_1=D\sin\angle A_1=D\sin\angle (180^{\circ}-2\angle A)=D\sin 2\angle A,$ where $D$ is the diameter of circumcircle $A_1B_1C_1$ so $B_1C_1\cdot t(A_1)=D\sin 2\angle A\cdot \dfrac{\sqrt{r\cdot r_A}}{\cos \angle A}=2D\sin \angle A\cdot\sqrt{r\cdot r_A}.$ 
		
		The original ratio now becomes $\sin \angle A\cdot\sqrt{r_A}$ : $\sin \angle B\cdot\sqrt{r_B}$ : $\sin \angle C\cdot\sqrt{r_C}$ by eliminating constants $D$ and $r$. Now if $d$ is the diameter of $\Gamma$ then it is not hard to notice that $r_A$=distance from $A$ to $\ell$=$TA\sin\angle (TA,\ell)$=$(d\sin a)\cdot\sin a=d\sin^{2} a.$ Meanwhile $\sin\angle A=|\sin (b-c)|.$ So our original ratio becomes $\sin a\cdot|\sin (b-c)|$ : $\sin a\cdot|\sin (c-a)|$ : $\sin c\cdot|\sin (a-b)|$. 
		Nevertheless, 
		\[
		\sin a\cdot\sin (b-c)+\sin a\cdot\sin (c-a)+\sin c\cdot\sin (a-b)=0
		\]
		which means one of them is the sum of the other two when we package them into absolute number. 
	\end{proof}
	
	Finally, notice that $A_1, B_1, C_1$ must all lie outside $\Gamma$. 
	W.l.o.g. let $T$ be opposite $A$ on $\Gamma$, 
	then given that $ABC$ is acute, 
	$OC'$ lies between $\ell$ and $AB$. 
	It then follows that $\ell_C$, the reflection of $\ell$ in $AB$, 
	has no intersection with $\Gamma$ (since $\ell$ is tangent to $\Gamma$). 
	Similarly, $\ell_B$ has no intersection with $\Gamma$, hence the two reflections 
	must lie entirely outside $\Gamma$, and therefore so are $A_1, B_1, C_1$. 
	Therefore, this and Lemma 2 means that the conclusion follows 
	 by applying Casey's theorem to degenerate circles $A_1, B_1, C_1$ and circle $\Gamma$. 

	
\end{enumerate}

\section*{Number Theory}
\begin{enumerate}
	\item [\textbf{N1}] For any integer $d > 0,$ let $f(d)$ be the smallest possible integer that has exactly $d$ positive divisors (so for example we have $f(1)=1, f(5)=16,$ and $f(6)=12$). Prove that for every integer $k \geq 0$ the number $f\left(2^k\right)$ divides $f\left(2^{k+1}\right).$
	
	\textbf{Solution.} Recall that each positive integer can be written as a nonnegative but finite length of product of primes, i.e. each $n\in\bbN$ can be written uniquely (up to reordering) as $\dprod_{i=1}^k p_i^{\alpha_i}$, and we also know that in this case the number of divisors of $n$ is $\dprod_{i=1}^k (\alpha_i+1)$. Thus, if $n$ has $2^k$ positive divisors then each $\alpha_i+1$ must be a power of 2. 
	
	In view of this, for each $k$, we can write $f(2^k)$ as 
	$\dprod_{i=1}^m p_i^{2^{d_i}-1}$ with $\dsum_{i=1}^m d_i=k$. 
	Now let $f(2^{k+1})=\dprod_{i=\ell}^m q_i^{2^{e_i}-1}$ with $\dsum_{i=1}^{\ell} e_i=k+1$. By inserting ``trivial factors'' into each $f(2^k)$ and $f(2^{k+1})$ (i.e. multiply by 1 = $p^0=p^{2^0-1}$ for arbitrary $p$) we may assume that they have the following form: 
	$f(2^k)=\dprod_{i=1}^m p_i^{2^{d_i}-1}$ and $f(2^{k+1})=\dprod_{i=1}^m p_i^{2^{e_i}-1}$, with $\dsum_{i=1}^m d_i=k$ and $\dsum_{i=1}^m e_i=k+1$. 
	
	Since $\dsum e_i>\dsum d_i$, there exists an index $j$ such that $e_j>d_j$. In particular, since $d_j\ge 0$, $e_j\ge 1$. Now let $d_i'=d_i$ if $i\neq j$, and $d_j'=d_j+1$. Then $\dsum_{i=1}^m d_i'=k+1$ and 
	$\dprod_{i=1}^m p_i^{2^{d_i'}-1}=p_j^{2^{d_j}}\dprod_{i=1}^m p_i^{2^{d_i}-1}=p_j^{2^{d_j}}f(2^k)$ has $2^{k+1}$ positive divisors. By the definition of $f$ we have $p_j^{2^{d_j}}f(2^k)\ge f(2^{k+1})$, 
	i.e. $p_j^{2^{d_j}}\ge \frac{f(2^{k+1})}{f(2^k)}$. 
	Similarly, define $e_i'$ as $e_i'=e_i$ if $i\neq j$, and $e_j'=e_j-1$. 
	Then $\dsum_{i=1}^m e_i'=k$ and $\dprod_{i=1}^m p_i^{2^{e_i'}-1}=p_j^{2^{-e_j'}}\dprod_{i=1}^m p_i^{2^{e_i}-1}=p_j^{2^{-e_j'}} f(2^{k+1})$ has $2^k$ positive divisors (recall that $e_j\ge 1$ so $e_j'\ge 0$). 
	Again by the definition of $f$ we have $p_j^{2^{-e_j'}} f(2^{k+1})\ge f(2^k)$, or 
	$\frac{f(2^{k+1})}{f(2^k)}\ge p_j^{2^{e_j'}}. $ 
	Combining these two inequalities we have $p_j^{2^{e_j'}}\le \frac{f(2^{k+1})}{f(2^k)}\le p_j^{2^{d_j}}$, so $e_j'\le d_j$. However, we have $e_j>d_j$, so $e_j'=e_j-1\ge d_j$, and therefore $p_j^{2^{e_j'}}=p_j^{2^{d_j}}$. 
	Thus all inequalities above become equality and thus $\frac{f(2^{k+1})}{f(2^k)}= p_j^{2^{d_j}}$, i.e. $f(2^k)\mid f(2^{k+1})$. 
	
	\item[\textbf{N2}] Consider a polynomial $P(x) =  \prod^9_{j=1}(x+d_j),$ where $d_1, d_2, \ldots d_9$ are nine distinct integers. Prove that there exists an integer $N,$ such that for all integers $x \geq N$ the number $P(x)$ is divisible by a prime number greater than 20.
	
	\textbf{Solution.} First, for each $j$, we consider the remainder of $\frac{P(x)}{x+d_j}$ when being divided by $x+d_j$. As each $d_j$ is distinct, the remainder is nonzero. Since $\deg(x+d_j)=1$, the remainder has degree 0, hence must in the form of $c_j$ for some nonzero integer constant $c_j$ (that is, $\frac{P(x)}{x+d_j}=Q_j(x)(x+d_j)+c_j$ for some polynomial $Q_j(x)$; such $Q_j(x)$ has integer coefficient since $x+d_j$ is monic). Now for each $x_0$, the number $\frac{P(x_0)}{x_0+d_j}$ is divisible by $x_0+d_j$ if and only if $c_j$ is divisible by $x_0+d_j$ (we have $Q_j(x_0)$ an integer). This is false for all $x_0>|c_j|-d_j$, so in terms of integers we have $x_0+d_j\nmid \frac{P(x)}{x+d_j}$ for $x_0$ sufficiently large. This also means that there exists a prime number $p(x_0, j)$ such that $v_{p(x_0, j)}(x_0+d_j)> v_{p(x_0, j)}\left(\frac{P(x_0)}{x_0+d_j}\right)=\dsum_{1\le i\le 9, i\neq j} v_{p(x_0, j)}(x_0+d_i)$ (here $v_p(q)$ is the highest exponent of the prime $p$ dividing the number $q$). 
	Since $v_p(q)\ge 0$ for all primes $p$ and all integers $q$, we have $v_{p(x_0, j)}(x_0+d_j)>v_{p(x_0, j)}(x_0+d_i)$ for all $i\neq j$. 
	Consider $N=\max\{|c_j|-d_j\}+1$, then for all $x_0\ge N$ and for each $j$ there exists a prime $p(x_0, j)$ such that $v_{p(x_0, j)}(x_0+d_j)>v_{p(x_0, j)}(x_0+d_i)$ for all $i\neq j$. 
	All such primes $p(x_0, j)$ must therefore be distinct (if $p(x_0, j)=p(x_0, i)$ for some $i\neq j$ then $v_{p(x_0, j)}(x_0+d_j)>v_{p(x_0, j)}(x_0+d_i)$ and $v_{p(x_0, j)}(x_0+d_i)>v_{p(x_0, j)}(x_0+d_j)$ hold simultaneously, which is absurd), and since $v_{p(x_0, j)}(x_0+d_j)\ge 1$ for each $j$, $P(x_0)$ is divisible by $p(x_0, j)$ for each $j$. In particular, $P(x_0)$ has at least 9 distinct prime divisors, so one of them must be greater than 20 (since there are exactly 8 prime numbers less than 20). 
	
	\item[\textbf{N3}] Let $n \geq 1$ be an odd integer. Determine all functions $f$ from the set of integers to itself, such that for all integers $x$ and $y$ the difference $f(x)-f(y)$ divides $x^n-y^n.$
	
	\textbf{Answer.} $f(x)=\pm x^d+c$ for any $d$ a positive divisor of $n$ and $c$ any constant. It's not hard to see that this works because $x^d-y^d\mid x^n-y^n$. \\
	\textbf{Solution.} It now remains to see that these are all the functions that work. First, notice that if $f$ is a function satisfying the condition, and $c$ is an arbitrary constant, then $g(x):=f(x)+c$ is also a working function. Hence we can assume that $f(0)=0$. 
	
	Consider any prime $p$. We have $f(p)\mid p^n$ and $f(-p)\mid (-p)^n$. Thus each $f(p)$ and $f(-p)$ must be in the form $\pm p^d$ for some nonnegative integer $d\le n$. 
	Considering that there are only finitely many nonnegative integers $d\le n$, and that there are infinitely many primes, there exists a $d_0$ such that, either $f(p)=p^{d_0}$ for infinitely many primes $p$, or $f(p)=-p^{d_0}$ for infinitely many primes $p$. 
	
	Let's assume the first case, and notice that for each nonzero integer $y$, we have $f(p)-f(y)\mid p^n-y^n$. 
	Since there's infinitely many $p$ with $f(p)=p^{d_0}$, we have $p^{d_0}-f(y)\mid p^n-y^n$ for infinitely many $p$. Consider the polynomial $q(x)=x^{d_0}$ we have $q(x)-f(y)\mid x^n-y^n$ for infinitely many $x$, too. If $r(x)$ is the remainder polynomial of $x^n-y^n$ when divided by $q(x)-f(y)$, then $q(x)-f(y)\mid r(x)$ for infinitely many positive integers $x$. But since $\deg(r)-\deg(q)$, $|r(x)|<|q(x)-f(y)|$ for all sufficiently large $x$, so we must have $r(x)=0$ for infinitely many $x$ (i.e. whenever $|r(x)|<|q(x)-f(y)|$ and $q(x)-f(y)\mid r(x)$). The only possibility is $r\equiv 0$, so $q(x)-f(y)\mid x^n-y^n$ as a polynomial. 
	Consider any complex solution $x_0$ to the solution $x^{d_0}=f(y)$, then $x_0^{d_0}-f(y)=0$ and therefore $x_0^n-y^n=0$, too, by the divisibility of polynomial (so $x_0^n=y^n$). We now have the following observations: 
	\begin{itemize}
		\item If $x^{d_0}=f(y)$ and $\epsilon$ is a root of unity of degree $d_0$, then $(\epsilon x_0)^{d_0}=f(y)$ and therefore $(\epsilon x_0)^{n}=y^n$. Thus $x_0^{n}=\epsilon^nx_0^n$. Since $x_0\neq 0$ as assumed (otherwise $f(y)=0$ and by the injectivity of $f$ we have $y=0$), we have $\epsilon^n=1$. Thus $\epsilon^{d_0}=1\to \epsilon^n=1$ for all complex number $\epsilon$, which happens only when $d_0\mid n$. 
		
		\item The above also implies that $d_0$ is odd, so the equation $x^{d_0}=f(y)$ actually has a unique real solution $x_0$ (recall that the mapping $x\to x^{d_0}$ for an odd $d_0$ is bijective if viewed from reals to reals). Consider one such $x_0$ and since $x_0^n=y^n$ too, we have $x_0=y$ since $n$ is also odd. Thus $f(y)=y^{d_0}$, as confirmed. 
		
	\end{itemize}
	Thus we have shown that $f(x)=x^{d_0}$ for some $d_0\mid n$ in the first case, i.e. infinitely many primes $p$ satisfying $f(p)=p^{d_0}$. The second case is similar, and we get $f(x)=-x^{d_0}$ for some $d_0\mid n$. 
	
	\item[\textbf{N4}] For each positive integer $k,$ let $t(k)$ be the largest odd divisor of $k.$ Determine all positive integers $a$ for which there exists a positive integer $n,$ such that all the differences
	
	\[t(n+a)-t(n); t(n+a+1)-t(n+1), \ldots, t(n+2a-1)-t(n+a-1)\] are divisible by 4.
	
	\textbf{Answer.} $a=1, 3, 5$. For each of those we can choose the pairs $(a, n)=(1, 1), (3, 1), (5,4)$. \\
	\textbf{Solution.} It now remains to show that these are all the possible $a$'s. If $a$ is even, then among the numbers $n, n+1, \cdots , n+a-1$, exactly one of them, say $ka+\frac{a}{2}$, must be congruent to $\frac{a}{2}$ modulo $a$. Now consider $t(ka+\frac{a}{2}+a)-t(ka+\frac{a}{2})$. 
	Now $ka+\frac{a}{2}=\frac{a}{2}(1+2k)$ and $ka+\frac{a}{2}+a=\frac{a}{2}(3+2k)$ so $t(ka+\frac{a}{2}+a)=t(\frac{a}{2})(3+2k)$ and $t(ka+\frac{a}{2})=t(\frac{a}{2})(1+2k)$ and therefore 
	$t(ka+\frac{a}{2}+a)-t(ka+\frac{a}{2})=2t(\frac{a}{2})$ and since $t(\frac{a}{2})$ is odd, $4\nmid 2t(\frac{a}{2})$. Hence $a$ cannot be even. 
	
	Next, let $a\ge 7$ be odd. Now one of $n$ and $n+a$ is odd, while the other is even. W.l.o.g. let $n$ be odd, then $n, n+2, n+4, n+6$ are all odd, and therefore $t(n+k)=n+k$ for all $k=0, 2, 4, 6$. 
	Now $n+a, n+2+a, n+4+a, n+6+a$ are all even, and since $4\mid t(n+a+k)-t(n+k)=t(n+a+k)-(n+k)$ for $k=0, 2, 4, 6$. Since $n\equiv n+4$ and $n+2\equiv n+6$, we have $t(n+a)\equiv t(n+a+4)\pmod{4}$ and $t(n+a+2)\equiv t(n+a+6)\pmod{4}$. If $n+a\equiv 2\pmod{4}$ then so is $n+a+4$, and therefore $t(n+a+4)-t(n+a)=(\frac{n+a+4}{2})-(\frac{n+a}{2})=2\not\equiv 0\pmod{4}$, contradiction. 
	If $n+a\equiv 0\pmod{4}$, then $n+a+2\equiv n+a+6\equiv 2\pmod{4}$, so $t(n+a+6)-t(n+a+2)=\frac{n+a+6}{2}-\frac{n+a+2}{2}=2\not\equiv 0\pmod{4}$, another contradiction. Therefore $a<7$ is necessary. 
	
	\item[\textbf{N5}] (IMO 5) Let $f$ be a function from the set of integers to the set of positive integers. Suppose that, for any two integers $m$ and $n$, the difference $f(m) - f(n)$ is divisible by $f(m- n)$. 
	Prove that, for all integers $m$ and $n$ with $f(m) \leq f(n)$, the number $f(n)$ is divisible by $f(m)$.
	
	\textbf{Solution.} We first notice that: 
	\begin{equation}
		f(m - 0)\mid f(m) - f(0)\qquad 
		\to 
		\qquad 
		\text{$f(0)$ is divisible by $f(m)$ for all integers $m$}
	\end{equation}
    And also 
    \begin{equation}
    	f(-n)=f(0-n)\mid f(0)-f(n)
    	\qquad 
    	\to 
    	\qquad f(-n)\mid f(n)
    \end{equation}
    the last one follows from $f(n)\mid f(0)$ for all integers $n$. 
    By symmetry, $f(n)\mid f(-n)$ and since both are positive, we have $f(n)=f(-n)$ for all $n$. 
    
    Next, we'll show that $f(n)\mid f(kn)$ for all integers $k, n$. 
    Notice that for all $k\ge 0$ we have 
    \begin{equation}
    	f((k+1)n - kn) = f(n)\mid f((k+1)n) - f(kn)
    \end{equation}
    and by considering $k=1$ as base case and perform induction from there, we have $f(n)\mid f(kn)$ for all $k\ge 1$. But since $f$ is an even function with $f(n)=f(-n)$ for all $n$, we have $f(n)\mid f(kn)$ for all $k$. 
	
	Now consider $m, n$ arbitrary, and let $\gcd(m, n)=d$. By Euclidean algorithm there exist integers $a$ and $b$ such that $am+bn=d$. We know by above, $f(d)\mid f(am)$ and $f(d)\mid f(bn)$. This gives us the following relation: 
	\[f(am)=f(d-bn)\mid f(d)-f(bn);\quad
	  f(bn)=f(d-am)\mid f(d)-f(am)
	\]
	W.L.O.G. let $f(bn)\le f(am)$, then since $f(d)$ and $f(am)$ are both positive, we have $|f(d)-f(am)|<\max\{f(d), f(am)\}\le f(bn)$ (we have $f(d)\mid f(bn)$ so $f(d)\le f(bn)$, too. This means that the only possibility is $f(d)-f(am)=0$, so $f(d)=f(am)$. But then $f(d)\mid f(m)\mid f(am)=f(d)$ so we also have $f(d)=f(m)$. Finally, $f(m)=f(d)\mid f(n)$ since $d\mid n$, completing the problem solution. 
	
	\item[\textbf{N6}]
	Let $P(x)$ and $Q(x)$ be two polynomials with integer coefficients, 
	such that no nonconstant polynomial with rational coefficients divides both $P(x)$ and $Q(x).$ 
	Suppose that for every positive integer $n$ the integers $P(n)$ and $Q(n)$ are positive, and $2^{Q(n)}-1$ divides $3^{P(n)}-1.$ 
	Prove that $Q(x)$ is a constant polynomial.
	
	\textbf{Solution.} 
	Suppose that $Q$ is nonconstant. 
	We first show that there exist infinitely many primes $p$ such that there exists $x$ (that depends on $x$) where $p$ divides $P(x)$ and $Q(x)$ simultanteously. 
	For each number $q$ with $\gcd(q, 3)=1$, denote $g(q)=\text{ord}_q(3)$, 
	i.e. the smallest $n > 0$ such that $q\mid 3^n-1$. 
	A quick check yields that $Q(x)$ even implies that $3\mid 2^{Q(x)} - 1$ but $3\nmid 3^{P(x)}-1$. 
	Thus $Q$ can only take odd values. 
	
	By Schur's theorem, there are infinitely many odd primes $p$ that divides one of $\{Q(1), Q(2), \cdots\}$ 
	(since $Q$ is nonconstant). 
	Among those primes, consider those such that $2^p > 3^{P(1)}$ (which there are also infinitely many of these since $3^{P(1)}$ is bounded), 
	which then follows that $2^p-1\nmid 3^{P(1)}-1$ and therefore 
	$g(2^p-1)\nmid P(1)$. 
	For each such prime $p$, let $n_p$ be such that $p\mid Q(n_p)$. 
	Using the fact that $x - y\mid Q(x)-Q(y)$ for each pairs $x, y$, 
	we have $p\mid Q(n_p+kp)$ for each integer $k\ge 0$. 
	Since $2^{p}-1\mid 2^{pq}-1$ for all nonnegative integers $q$, 
	we have $2^p-1 \mid 2^{Q(n_p+kp)}-1\mid 3^{P(n_p+kp)}-1$, 
	which also means $g(2^p-1)\mid P(n_p+kp)$ for all integers $k\ge 0$. 
	Suppose that $p\nmid g(2^p-1)$, 
	i.e. $\gcd(p, g(2^p-1))=1$. 
	Then $\{n_p+kp\pmod{g(2^p-1)}: k\ge 0\}$
	is complete residue mod $g(2^p-1)$. 
	Using $x - y\mid P(x)-P(y)$, 
	we have $g(2^p-1)\mid P(x)$ for all $x \ge 1$. 
	This contradicts $g(2^p-1)\nmid P(1)$. 
	It then follows that $p\mid g(2^p-1)\mid P(n_p)$, 
	and so $p$ divides $P(n_p)$ and $Q(n_p)$ simultaneously. 
	
	We now consider the following sequence following iteration Bezout's theorem: 
	set $P_0=P, Q_0=Q$, 
	and subsequently, 
	$P_{n+1}=Q_n, Q_{n+1}$ is the remainder of $P_n$ when divided by $Q_n$ 
	(i.e. there exists a polynomial $A_n$ such that $P_n=Q_{n+1}+Q_nA$, and $\deg(Q_{n+1}) < \deg Q_n$; 
	such $A_n$ must have rational coefficients). 
	This process persists until one of the terms hit constant or 0. 
	Since $P$ and $Q$ has no nonconstant polynomial divisors (with rational coefficients), 
	this also applies to each pair $(P_i, Q_i)$. 
	It follows that there exists an $n$ such that $Q_n\equiv d$ for some nonzero constant $d$. 
	Since all the terms $A_n$ involved are of rational coefficients, 
	it follows that $d$ is also rational. 
	This means that there exist $A, B$ or rational coefficients such that 
	$AP+BQ=d_0\neq 0$ for some rational constant $d_0$. 
	Multiplying by some suitable nonzero integer, we may assume that $A, B$ are of integer coefficient, 
	and that $d_0$ is an integer. 
	This means that for each $x$, $\gcd(A(x), B(x))\mid d_0$. 
	This contradicts the claim that there are infinitely many primes $p$ such that for some $n_p$, 
	$p$ divides $P(n_p), Q(n_p)$ simultanteously. 

	\item [\textbf{N7}]
	Let $p$ be an odd prime number. 
	For every integer $a,$ define the number $S_a = \sum^{p-1}_{j=1} \frac{a^j}{j}.$ Let $m,n \in \mathbb{Z},$ such that 
	\[S_3 + S_4 - 3S_2 = \frac{m}{n}\] 
	
	Prove that $p$ divides $m.$
	
	\textbf{Solution.} 
	We denote the following: 
	\begin{equation}
		O_a = \sum_{1\le j\le p-1, j\text{ odd}} \frac{a^j}{j}
		\qquad 
		E_a = \sum_{1\le j\le p-1, j\text{ even}} \frac{a^j}{j}
	\end{equation}
    Observe that $S_a = O_a + E_a$ for all integers $a$. 
    In addition, 
    we will consider the modulo of a rational number mod $p$ so long as the denominator is not divisible by $p$, 
    via multiplicative inverse. 
    Thus from now on, all the equivalences are in modulo $p$. 
    
    We first consider the following: 
    
    \begin{lemma}
    	$S_{a + 1}\equiv E_a - O_a$
    \end{lemma}
    
    \begin{proof}
    	Expanding $S_{a+1}$ gives us the following: 
    	\begin{flalign}
    		S_{a + 1} &= 
    		\sum^{p-1}_{j=1} \frac{1}{j}\sum_{\ell = 0}^j \binom{j}{\ell}a^{\ell}
    		\nonumber\\
    		&=\sum_{\ell = 0}^{p - 1} a^{\ell}\sum_{j = \max(1, \ell)}^{p - 1}\frac{1}{j}\binom{j}{\ell}
    		\nonumber\\
    		&=\sum_{\ell=1}^{p - 1}\frac{a^0}{\ell} 
    		+ \sum_{\ell = 1}^{p - 1} a^{\ell}\sum_{j = \ell}^{p - 1}\frac{1}{j}\binom{j}{\ell}
    		\nonumber\\
    		&=S_1 + \sum_{\ell = 1}^{p - 1} a^{\ell}\sum_{j = \ell}^{p - 1}\frac{(j - 1)!}{\ell!(j - \ell)!}
    		\nonumber\\
    		&=S_1 + \sum_{\ell = 1}^{p - 1} \frac{a^{\ell}}{\ell}\sum_{j = \ell}^{p - 1}\binom{j - 1}{\ell - 1}
    	\end{flalign}
        To proceed, notice the following: 
        \begin{equation}
        	S_1 = \sum_{j=1}^{p - 1}\frac{1}{j}
        	= \sum_{j=1}^{(p -1) / 2}\frac{1}{j} + \frac{1}{p -j}
        	=\sum_{j=1}^{(p -1) / 2}\frac{p}{j(p - j)}
        	\equiv 0
        \end{equation}
    and also by telescoping sum, we have 
    \begin{equation}
    	\sum_{j = \ell}^{p - 1}\binom{j - 1}{\ell - 1}
    	=\sum_{j = \ell}^{p - 1}\binom{j}{\ell} - \binom{j - 1}{\ell} 
    	=\binom{p - 1}{\ell} - \binom{\ell - 1}{\ell}
    	=\binom{p - 1}{\ell}
    \end{equation}
    Finally, we also have $\binom{p - 1}{\ell}=\frac{(p-1)\cdots (p-\ell)}{\ell!}
    \equiv \frac{(-1)\cdots (-\ell)}{\ell!}=(-1)^{\ell}$, therefore 
    \begin{equation}
    	S_{a+1}
    	=S_1 + \sum_{\ell = 1}^{p - 1} \frac{a^{\ell}}{\ell}\sum_{j = \ell}^{p - 1}\binom{j - 1}{\ell - 1}
    	\equiv 0 + \sum_{\ell = 1}^{p - 1} \frac{a^{\ell}}{\ell}(-1)^{\ell}
    	=E_a - O_a
    \end{equation}
    \end{proof}
    
    Now we have the next one: 
    \begin{lemma}
    	$S_{a^2} \equiv 2E_a + 2aO_a$
    \end{lemma}
    \begin{proof}
    	Again we have the following expansion: 
    	\begin{equation}
    		S_{a^2} = \sum^{p-1}_{j=1}\frac{a^{2j}}{j}
    		= \sum_{j=1}^{(p - 1)/ 2} \left(\frac{a^{2j}}{j} + \frac{a^{2j + (p - 1)}}{(p - 1)/ 2 + j}\right)
    		\equiv \sum_{j=1}^{(p - 1)/ 2} a^{2j}\left(\frac{1}{j} + \frac{1}{-\frac 12 + j}\right)
    	\end{equation}
        which can then be decomposed into the following: 
        \begin{equation}
        	\sum_{j=1}^{(p - 1)/ 2} \frac{a^{2j}}{j}
        	=2\sum_{j=1}^{(p - 1)/ 2} \frac{a^{2j}}{2j}
        	=2E_a
        \end{equation}
        and also 
        \begin{equation}
        	\sum_{j=1}^{(p - 1)/ 2} \frac{a^{2j}}{-\frac 12 + j}
        	\equiv \sum_{j=1}^{(p - 1)/ 2} \frac{2a^{2j}}{2j - 1}
        	=2a\sum_{j=1}^{(p - 1)/ 2} \frac{a^{2j - 1}}{2j - 1}
        	=2aO_a
        \end{equation}
    \end{proof}
    Therefore considering $a = 2$ we have 
    \begin{equation}
    	S_3 + S_4 - 3S_2
    	\equiv (E_2 - O_2) + (2E_2 + 2\cdot 2\cdot O_2) - 3(E_2 + O_2)
    	= 0
    \end{equation}
    as desired. 
	

	\item [\textbf{N8}] Let $k \in \mathbb{Z}^+$ and set $n=2^k+1.$ Prove that $n$ is a prime number if and only if the following holds: there is a permutation $a_{1},\ldots,a_{n-1}$ of the numbers $1,2, \ldots, n-1$ and a sequence of integers $g_{1},\ldots,g_{n-1},$ such that $n$ divides $g^{a_i}_i - a_{i+1}$ for every $i \in \{1,2,\ldots,n-1\},$ where we set $a_n = a_1.$
	
	\textbf{Solution.}
	We note that $n$ is odd. 
	Let's first show that if such sequence exists, then $n$ must be prime. 
	Suppose $n$ is composite, then there are two cases: 
	\begin{itemize}
		\item There's a odd prime number $p$ with $p^2\mid n$. 
		If $a_{\ell+1}=p$ and $a_{m+1}=2p$, which means $g_{\ell}^{a_{\ell}}$ and $g_m^{a_m}$ must both be divisible by $p$ but not $p^2$. This implies $p\mid g_{\ell}, g_m$ and $a_{\ell}=a_m=1$, which is impossible. 
		
		\item $n$ is squarefree with at least two distinct prime divisors. 
		We see that among $1, \cdots, n-1$, $\frac{n-1}{2}$ are even. 
		Also, if $a_i$ is even, then $a_{i+1}$ must be a quadratic residue mod $n$. 
		Now, for each prime $p$, the number of quadratic residues modulo $p$ is $\frac{p+1}{2}$. 
		This means that 
		if $p_1, \cdots, p_k$ are prime divisors of $n$ then the number of quadratic residue (excluding 0) is 
		\[
		\left(n\cdot \prod \frac{p_i+1}{2p_i}\right)- 1
		\]
		where the $-1$ is considering that we need to subtract 0 off. 
		This means $\left(n\cdot \prod \frac{p_i+1}{2p_i}\right)\ge \frac{n-1}{2}$, 
		or $\prod \frac{p_i+1}{2p_i}>\frac 12$. 
		On the other hand, for each prime $p_i$, $\frac{p_i+1}{2}=\frac 12 + \frac{1}{2p_i} \le \frac 12 + \frac 16=\frac 23$ so 
		with $k\ge 2$, 
		we have $\prod \frac{p_i+1}{2p_i}\le (\frac 23)^2=\frac 49 < \frac 12$. 
	\end{itemize}
    
    It now remains to provide a construction when such $n$ is prime. 
    Observe that $k$ must be a power of 2; 
    if $k=ab$ for some odd $a > 1$ then $2^b+1\mid 2^{ab}+1$ (thus not prime). 
    We write $k=2^m$ for some $m\ge 0$. 
    In fact, we may settle the case where $k=1$ immediately by just letting $a_1=g_1=1, a_2=g_2=1$, 
    so $k\ge 2$ and so $m\ge 1$. 
    This would also mean that $n\equiv 2\pmod{3}$. 
    
    Consider, now, a bipartite graph with $(1, 1), \cdots, (1, n-1)$ appearing on one side, 
    and $(2, 1), \cdots, (2, n-1)$ appearing on the other side. 
    Denote $\nu_2(x)$ as the highest power of 2 dividing $x$, 
    and consider any primitive root $g$ of $n$. 
    Then $g^1, \cdots, g^{n-2}, g^{n-1}=1$ are all distinct mod $n$, 
    so in our graph we can have edges $(1, a)$ and $(2, g^b)$ for all pairs $(a, b)$ satisfying $\nu_2(a)=\nu_2(b)$. 
    Note that a perfect matching exists: 
    just take $(1, a), (2, g^a)$. 
    
    Now we consider a directed graph $(V, E)$ where $V=\{1, \cdots, n-1\}$, 
    and $E=\{a\to b\}$ is determined by the perfect matching. 
    Observe that each node has an indegree and outdegree of 1 each, so the resulting graph is a set of disjoint cycles (including possible self-loop). 
    We now consider the following claim: 
    \begin{lemma}
    	By modifying the graph such that a new edge $a\to b$ can only be added if there is an edge between $(1, a), (2, b)$ (but there is no rule of edge deletion), 
    	we can produce $(V, E)$ with a single cycle that traverses all vertices (Hamiltonian). 
    \end{lemma}
    
    \begin{proof}
    	Consider, now, any two cycles $C_1$ and $C_2$. 
    	We claim that if $a\in C_1$ and $b\in C_2$ such that $\nu_2(a)=\nu_2(b)$, 
    	then it is possible to merge $C_1$ and $C_2$ into one. 
    	Indeed, 
    	let $C_1$ be $a_1\to \cdots \to a_u\to a_1$ and $b_1\to\cdots\to b_v\to b_1$, 
    	with $\nu_2(a_1)=\nu_2(b_1)$. 
    	Then by our construction, $a_2=g^{c_2}$ and $b_2=g^{d_2}$ for some $c_2, d_2$ 
    	such that $\nu_2(a_1)=\nu_2(b_1)=\nu_2(c_2)=\nu_2(d_2)$. 
    	We can therefore take the cycle 
    	$a_1\to b_2\to\cdots \to b_v\to b_1\to a_2\cdots \to a_u\to a_1$, 
    	justifying such a merge 
    	(in other words we delete $a_1\to a_2, b_1\to b_2$ and replace by $a_1\to b_2, b_1\to a_2$ to maintain the structure of disjoint cycles). 
    	
    	Now suppose after several such merging steps we can no longer proceed. 
    	In particular, all odd numbers are in the same cycle, say $C_1$. 
    	It then follows that all the quadratic nonresidues are also in $C_1$. 
    	We now show that $C_1$ contains all $1, \cdots, n-1$. 
    	
    	We have shown the case $n=3$ above. 
    	For $n=5$, using $g=3$ (i.e. $g, g^2, g^3, g^4=3,4,2,1$) we have the cycle 
    	$1\to 3\to 2\to 4\to 1$, 
    	hence we may assume $n\ge 17$.
    	Since $n\equiv 2\pmod{3}$ and $n\equiv 1\pmod{4}$, 
    	we use the famous property of Legendre's symbol 
    	$(\frac{3}{n})(\frac{n}{3})=(-1)^{\frac{(3-1)(n-1)}{2}}=1$, 
    	and with $(\frac{n}{3})=-1$, $(\frac{3}{n})=-1$. 
    	In other words, $\frac 3n$ is not a quadratic residue. 
    	On the other hand, 
    	since $2^k\equiv -1$ (i.e. $2^{2k}\equiv 1$) 
    	and $2k < 2^k$ for $k\ge 4$, 
    	2 is a quadratic residue of $n$. 
    	This means that $3\cdot 2^{\ell}$ is a quadratic nonresidue for 
    	$\ell=0, \cdots, k-2$, 
    	which means for each $\ell=0, \cdots, 2^{\ell}$, 
    	$C_1$ contains at least a member in the form $\nu_2(\cdot)=\ell$. 
    	Since we can no longer merge two different cycles, 
    	all numbers with $\nu_2(\cdot)\le k-2$ must be in this cycle $C_1$. 
    	
    	Finally, the only numbers not in the form $\nu_2(\cdot)\le k-2$ are $2^{k-1}$ and $2^k$. 
    	But in our matching, the only outdegree from $2^{k}$ is to 1, 
    	and the only outdegree from $2^{k-1}$ is to $-1=2^k$. 
    	Thus both numbers are in $C_1$ (since 1 is in this cycle), 
    	completing the proof. 
    \end{proof}
    
    Now we can finish the solution. In our single cycle above, we get $a_1\to\cdots \to a_{n-1}\to a_1$. 
    By the method we construct our edges, $a_i\to a_{i+1}$ implies that if $a_{i+1}=g^{b_{i+1}}$ then $\nu_2(a_i)=\nu_2(b_{i+1})$. 
    It then follows that a suitable $g_i$ can be found such that $g_i^{a_i}=a_{i+1}$ given that $a_{i+1}$ is the $2^{\nu_2(a_i)}$-th power residue of $n$ and $\gcd(n-1, a_i)=2^{\nu_2(a_i)}$. 
    
    
    
    
	
\end{enumerate}

\end{document}