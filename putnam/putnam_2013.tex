\documentclass[11pt,a4paper]{article}
\usepackage{amsmath, amssymb, fullpage, mathrsfs, bm, pgf, tikz}
\usepackage{mathrsfs}
\usetikzlibrary{arrows}
\setlength{\textheight}{10in}
%\setlength{\topmargin}{0in}
\setlength{\topmargin}{-0.5in}
\setlength{\parskip}{0.1in}
\setlength{\parindent}{0in}

\newcommand{\set}[2]{\{#1\,:\,\text{#2}\}}
\newcommand{\tup}[1]{\mathrm{#1}}
\newcommand{\sfP}{\mathsf{P}}
\newcommand{\M}{\mathsf{M}}
\newcommand{\bbR}{\mathbb R}
\newcommand{\bbC}{\mathbb C}
\newcommand{\bbZ}{\mathbb Z}
\newcommand{\bbN}{\mathbb N}
\newcommand{\bbQ}{\mathbb Q}
\newcommand{\bbF}{\mathbb F}
\newcommand{\bbP}{\mathbb P}
\newcommand{\bbE}{\mathbb E}
\newcommand{\dfeq}{\stackrel{\mathrm{def}}{=}}
\newcommand{\ra}{\rightarrow}
\newcommand{\la}{\leftarrow}
\newcommand{\lra}{\leftrightarrow}
\newcommand{\Span}{\mathrm{span}}
\newcommand{\scrP}{\mathscr{P}}
\newcommand{\rank}{\mathrm{rank}}
\newcommand{\nullity}{\mathrm{nullity}}
\newcommand{\Col}{\mathrm{Col}}
\newcommand{\Row}{\mathrm{Row}}
\newcommand{\tr}{\mathrm{tr}}
\newcommand{\ol}{\overline}
\newcommand{\norm}[1]{||#1||}
\newcommand{\doubleline}[1]{\underline{\underline{#1}}}
\newcommand{\elemop}[1]{\stackrel{#1}{\longrightarrow}}
\newcommand{\Ind}{\mathrm{Ind}}
\newcommand{\Res}{\mathrm{Res}}
\newcommand{\End}{\mathrm{End}}
\newcommand{\cl}{\mathrm{cl}}
\newcommand{\code}[1]{\texttt{#1}}
\newcommand\tab[1][0.5cm]{\hspace*{#1}}
\newcommand{\<}{\langle}
\renewcommand{\>}{\rangle}
\newcommand{\qubits}[1]{|{#1}\rangle}
\newcommand{\ord}{\mathrm{ord}}
\newcommand{\lcm}{\mathrm{lcm}}
\newcommand{\dsum}{\displaystyle\sum}
\newcommand{\dprod}{\displaystyle\prod}

\begin{document}

\newcommand{\sgn}{\text{sgn}}
\setcounter{secnumdepth}{0}

\section{Putnam 2013}
\begin{enumerate}
	\item [\textbf{A1}] Recall that a regular icosahedron is a convex polyhedron having 12 vertices and 20 faces; the faces are congruent equilateral triangles. On each face of a regular icosahedron is written a nonnegative integer such that the sum of all $20$ integers is $39.$ Show that there are two faces that share a vertex and have the same integer written on them.
	
	\textbf{Solution.} Each face corresponds to exactly 3 vertices, so on average each vertex corresponds to $20\times 3\div 12=5$ faces. Since this icosahedron is regular, each vertex corresponds to 5 faces. Suppose that for each vertex, the number written is different. Then the sum of the 5 faces joining a vertex is at least $0+1+2+3+4=10$. Since each vertex corresponds to 3 faces and there are 12 vertices, the total sum of 20 faces is at least $10\times 12\div 3=40$, contradiction. 
	
	\item[\textbf{A2}]Let $S$ be the set of all positive integers that are not perfect squares. For $n$ in $S,$ consider choices of integers $a_1,a_2,\dots, a_r$ such that $n<a_1<a_2<\cdots<a_r$ and $n\cdot a_1\cdot a_2\cdots a_r$ is a perfect square, and let $f(n)$ be the minimum of $a_r$ over all such choices. For example, $2\cdot 3\cdot 6$ is a perfect square, while $2\cdot 3,2\cdot 4, 2\cdot 5, 2\cdot 3\cdot 4,$ $2\cdot 3\cdot 5, 2\cdot 4\cdot 5,$ and $2\cdot 3\cdot 4\cdot 5$ are not, and so $f(2)=6.$ Show that the function $f$ from $S$ to the integers is one-to-one.
	
	\textbf{Solution.} Suppose that $f(k_1)=f(k_2)$ for some $k_1 < k_2$. Let $k_1<a_1<\cdots <a_r=f(k_1)$ and $k_2<b_1<\cdots<b_s=f(k_2)$ be such choices for $k_1$ and $k_2$. Given that $k_1\cdot a_1\cdots a_r$ and $k_2\cdot b_1\cdots b_s$ are both perfect square, their product $k_1\cdot a_1\cdots a_r\cdot k_2\cdot b_1\cdots b_s$ is also a perfect square. Suppose that some number $g$ appears in both sequence $\{a_i\}$ and $\{b_i\}$, then removing $g$ from the combined sequence $k_1\cdot a_1\cdots a_r$ and $k_2\cdot b_1\cdots b_s$ yields that $k_1\cdot a_1\cdots a_r\cdot k_2\cdot b_1\cdots b_s/g^2$ is still a perfect square. Now, we remove all such repeated elements and sort the numbers, we get $k_1<c_1<\cdots <c_t$, since $k_1$ only appears once ($k_1<k_2$) and since $a_r=b_s$, this number is removed from both sides and $c_t<a_r=f(k_1)$, contraidcting the minimality of $a_r$. 
	
	\item[\textbf{A3}]Suppose that the real numbers $a_0,a_1,\dots,a_n$ and $x,$ with $0<x<1,$ satisfy \[\frac{a_0}{1-x}+\frac{a_1}{1-x^2}+\cdots+\frac{a_n}{1-x^{n+1}}=0.\] Prove that there exists a real number $y$ with $0<y<1$ such that \[a_0+a_1y+\cdots+a_ny^n=0.\]
	
	\textbf{Solution.} First, since $0<x<1$, each sequence $1-x^n=1+x^n+x^{2n}+x^{3n}+\cdots$ converges absolutely. Hence we are free to permute the sequence and get the sum in the following sense: 
	\[0=\frac{a_0}{1-x}+\frac{a_1}{1-x^2}+\cdots+\frac{a_n}{1-x^{n+1}}
	=\sum_{i=1}^n a_i \left(\sum_{j=0}^{\infty} x^{ij}\right)
	=\sum_{j=0}^{\infty} \left(\sum_{i=0}^n a_i(x^{j})^i\right)
	\]
	Let $b_j=\sum_{i=0}^n a_i(x^{j})^i$ for all $j\ge 0$, then it follows that $\sum_{i=0}^\infty b_j$ also converges absolutely to 0. Now let $k$ to be the minimal index such that $b_k\neq 0$. If $b_j=0$ for some $j>k$ then we are done, since we can just pick $y=x^j$ and since $j>k\ge 0$, $y\in (0, 1)$. Otherwise, since $\sum_{i=k}^\infty b_j=\sum_{i=0}^\infty b_j=0$, there exists a $j\ge k$ such that $b_j<0$ and $b_{j+1}>0$, or vice versa. In either case, $a_0+a_1y+\cdots+a_ny^n=0$ for some $y\in (x^{j+1}, x^{j})$, which obviously lies in $(0, 1)$. 
	
	\item[\textbf{A4}] A finite collection of digits $0$ and $1$ is written around a circle. An arc of length $L\ge 0$ consists of $L$ consecutive digits around the circle. For each arc $w,$ let $Z(w)$ and $N(w)$ denote the number of $0$'s in $w$ and the number of $1$'s in $w,$ respectively. Assume that $|Z(w)-Z(w')|\le 1$ for any two arcs $w,w'$ of the same length. Suppose that some arcs $w_1,\dots,w_k$ have the property that \[Z=\frac1k\sum_{j=1}^kZ(w_j)\text{ and }N=\frac1k\sum_{j=1}^k N(w_j)\] are both integers. Prove that there exists an arc $w$ with $Z(w)=Z$ and $N(w)=N.$
	
	\textbf{Solution}. Let $n$ be the number of bits written around the circle, and $m$ the number of 1's written. We first prove the following lemma: for each $0\le k\le n$, consider $S_k=\{Z(w): |w|=k\}$ where $|w|$ is the length of $w$. Then $S_k=\{\lfloor \frac{km}{n}\rfloor, \lceil \frac{km}{n}\rceil\}$. To prove this, we notice that from the problem statement, $\max(S_k)-\min(S_k)\le 1$. 
	We consider two possible cases: 
	\begin{itemize}
		\item The quantity $\frac{km}{n}$ is not an integer. Then $\lceil \frac{km}{n}\rceil-\lfloor \frac{km}{n}\rfloor=1$. Now consider the $n$ arcs $W_1, W_2, \cdots , W_n$ with length $k$; each point belongs to exactly $k$ of the arcs. Thus $\sum_{i=1}^n Z(W_i)=km$ since there are $m$ one's written, and the average of $Z(w_i)$ is $\frac{km}{n}$. By piegonholw principle, there is at least one $W_i$ with $Z(W_i)\ge \lceil \frac{km}{n}\rceil$ and one $W_i$ with $Z(W_i)\le \lceil \frac{km}{n}\rceil$. Since $\max(W_i)-\min(W_i)\le 1$ and $\lceil \frac{km}{n}\rceil-\lfloor \frac{km}{n}\rfloor$, the conclusion follows. 
		
		\item Now that the quantity $\frac{km}{n}$ is an integer, meaning that $\lceil \frac{km}{n}\rceil=\frac{km}{n}=\lfloor \frac{km}{n}\rfloor$. Denote the $n$ arcs by $W_1, \cdots , W_n$, and by the logic above, the average of $Z(w_i)$ is $\frac{km}{n}$. 
		If there is $w_i$ with $Z(w_i)<\frac{km}{n}$, i.e. $Z(w_i)\le \frac{km}{n} -1$, then there must be $w_i$ with $Z(w_i)>\frac{km}{n}$, i.e. $Z(w_i)\ge \frac{km}{n}+1$. This is a contradiction that $\max(S_k)-\min(S_k)\le 1$, so we have $Z(W_i)=\frac{km}{n}$ for each $i$. 
	\end{itemize}
	Now going back to the problem. For each arc $w$, we have $|w|=Z(w)+N(w)$. Thus considering the $w_1, \cdots , w_k$ given in the problem we have \[Z+N=\frac1k\sum_{j=1}^kZ(w_j)+\frac1k\sum_{j=1}^kZ(w_j)=\frac1k\sum_{j=1}^k(Z(w_j)+N(w_j)=\frac1k\sum_{j=1}^k|w_j|\]
	And in addition, for each $w_j$ we have $Z(w_j)\in \{\lfloor \frac{|w_j|m}{n}\rfloor, \lceil \frac{|w_j|m}{n}\rceil\}$ so $\frac{|w_j|m}{n}-1<Z(w_j)<\frac{|w_j|m}{n}+1$. This means, 
	$\frac1k\sum_{j=1}^k(\frac{|w_j|m}{n}-1) < Z < \frac1k\sum_{j=1}^k(\frac{|w_j|m}{n}+1)$, i.e. 
	$(\frac{m}{kn}\sum_{j=1}^k |w_j|)-1 < Z < (\frac{m}{kn}\sum_{j=1}^k |w_j|)+1$, and 
	since $\frac 1k\sum_{j=1}^k |w_j|=Z+N$, we have 
	\[(\frac{m}{n}(Z+W))-1 < Z < (\frac{m}{n} (Z+N))+1
	\]
	Since $Z$ is an integer, this also implies that $Z\in \{\lfloor \frac{m}{n}(Z+N)\rfloor , \lceil \frac{m}{n}(Z+N)\rceil\}$. This means we can find an arc $w$ of length $Z+N$ that has $Z(w)=Z$, and therefore $N(W)=N$. 
	
	\item[\textbf{A5}]
	For $m\ge 3,$ a list of $\binom m3$ real numbers $a_{ijk}$ $(1\le i<j<k\le m)$ is said to be area definite for $\mathbb{R}^n$ if the inequality 
	\[\sum_{1\le i<j<k\le m}a_{ijk}\cdot\text{Area}(\triangle A_iA_jA_k)\ge0\]
	holds for every choice of $m$ points $A_1,\dots,A_m$ in $\mathbb{R}^n.$ 
	For example, the list of four numbers $a_{123}=a_{124}=a_{134}=1, a_{234}=-1$ is area definite for $\mathbb{R}^2.$ 
	Prove that if a list of $\binom m3$ numbers is area definite for $\mathbb{R}^2,$ then it is area definite for $\mathbb{R}^3.$
	
	\textbf{Solution.} 
	Let's first claim the following: 
	
	\emph{Lemma.} Consider a plane $p$, and let $A_1A_2A_3$ be on a plane $q$. 
	Let $B_i$ be the projection from $A_i$ to $p$. 
	Then 
	\[
	|\triangle B_1B_2B_3|
	=|\cos\angle(p, q)|\cdot |\triangle A_1A_2A_3|
	\]
	where $\angle(p, q)$ is the angle between the two planes. 
	
	Proof: if $A_1A_2A_3$ is degenerate (i.e. on the same line), 
	then so would $B_1B_2B_3$. 
	Now assume $A_1A_2A_3$ is not degenerate. 
	By doing parallel shift on $p$ (which wouldn't change the area of $B_1B_2B_3$), 
	we may shift $p$ such that $A_1$ on $p$, and therefore $B_1=A_1$. 
	
	Now, let the line $A_2A_3$ to intersect $p$ at $C$ 
	(the case when $A_2A_3$ is parallel to $C$ can be dealt separately). 
	Then $AC$ is the intersection line of $p$ and $q$ (either that, or $p$ and $q$ coincide). 
	In addition, we have $C, B_2, B_3$ collinear in this case. 
	If $h_A, h_B$ are the perpendiculars from $A_2, B_2$ to line $AC$, then 
	$h_A$ and $h_B$ intersect on $AC$ and $|h_B|=\cos\angle(p, q)|h_A|$. 
	Therefore, 
	\[
	|\triangle ACB_2|=\frac 12 AC|h_B|=\frac 12 AC\cos\angle(p, q)|h_A|=\cos\angle(p, q)|\triangle ACA_2|
	\]
	and similarly $|\triangle ACB_3|=\cos\angle(p, q)|\triangle ACA_3|$. 
	If $A_2, A_3$ are on the same side of $C$, then so will $B_2, B_3$ and in this case, 
	\[
	|\triangle AB_3B_2|
	=||\triangle ACB_2| - |\triangle ACB_3||
	\]\[
	=|\cos\angle(p, q)|\triangle ACA_2| - \cos\angle(p, q)|\triangle ACA_3||
	=|\cos\angle(p, q)||\triangle AA_3A_2|
	\]
	and if $A_2, A_3$ are on different side of $C$, then so will $B_2, B_3$ and in this case, 
	\[
	|\triangle AB_3B_2|
	=||\triangle ACB_2| + |\triangle ACB_3||
	\]\[
	=|\cos\angle(p, q)|\triangle ACA_2| + \cos\angle(p, q)|\triangle ACA_3||
	=|\cos\angle(p, q)||\triangle AA_3A_2|
	\]
	as desired. 
	
	Now we turn back to the problem. Consider any list of $\{a_{ijk}\}_{1\le i<j<k\le m}$ that's area definite w.r.t. $\bbR^2$. 
	Consider also, any $m$ points $A_1, \cdots, A_m$ in $\bbR^3$. 
	Define $\Pi$ as the set of all planes passing through $(0, 0, 0)$, 
	and let $B_{i, p}$ be the projection $A_i$ to plane $p\in \Pi$. 
	Let's consider the quantity 
	\[
	S = \int_{p\in \Pi} \sum_{1\le i<j<k\le m}a_{i, j, k}|\triangle B_{i, p}B_{j, p}B_{j, p}|dp
	\]
	Since all points $B_{i, p}$ are on plane $p$, 
	$\sum_{1\le i<j<k\le m}a_{i, j, k}|\triangle B_{i, p}B_{j, p}B_{j, p}|\ge 0$ since it's our numbers are
	area definite w.r.t. $\bbR^2$. 
	Hence $S\ge 0$. 
	
	Let's now compute $S$ in another way based on our lemma. Denote $p_{i, j, k}$ as the plane passing through $A_i, A_j, A_k$, then by our lemma we can rewrite $S$ as 
	\begin{flalign*}
	S &= \int_{p\in \Pi} \sum_{1\le i<j<k\le m}a_{i, j, k} |\cos\angle(p, p_{i, j, k})|\cdot |\triangle A_iA_jA_k|dp
	\\&=  \sum_{1\le i<j<k\le m}a_{i, j, k} |\triangle A_iA_jA_k|\int_{p\in \Pi}|\cos\angle(p, p_{i, j, k})|dp
	\end{flalign*}
	But then $\int_{p\in \Pi}|\cos\angle(p, q)|dp$ is the same for all planes $q$, 
	and it's positive. 
	To see why, we can consider the unit vectors $p_1$ and $q_1$ from $(0, 0, 0)$ perpendicular to $p, q$ respectively, so that $\cos\angle(p, q)=p_1\cdot q_1=\cos\angle(p_1, q_1)$. 
	For each $\theta\in [0, \pi]$, we have 
	\[
	\int_{p: \cos(p, q)=\theta}1 = 2\pi\sin\theta
	\]
	since such set $p$ is just the great circle on the unit sphere through the origin that's angle $\theta$ away from $q$, hence having radius $\sin\theta$. Therefore we have 
	\[
	\int_{p\in\Pi}|\cos\angle(p, q)|
	=\int_{0}^{\pi}2\pi|\sin\theta\cos\theta|
	=\pi\int_{0}^{\pi}|\sin2\theta|
	=2\pi
	\]
	Thus there's constant $C=2\pi$ where $\int_{p\in \Pi}|\cos\angle(p, q)|dp=C$ and 
	\[
	\sum_{1\le i<j<k\le m}a_{i, j, k} |\triangle A_iA_jA_k| = \frac{S}{C}
	\]
	and the conclusion thus follows when $S\ge 0$ and $C>0$. 
	
	\item[\textbf{B1}] For positive integers $n,$ let the numbers $c(n)$ be determined by the rules $c(1)=1,c(2n)=c(n),$ and $c(2n+1)=(-1)^nc(n).$ Find the value of \[\sum_{n=1}^{2013}c(n)c(n+2).\]
	
	\textbf{Answer.} $-1$. \\
	\textbf{Solution.}
	\begin{flalign*}
	\sum_{n=1}^{2013}c(n)c(n+2)
	&=c(1)c(3)+\sum_{n=1}^{1006}c(2n)c(2n+2)
	+\sum_{n=1}^{1006}c(2n+1)c(2n+3)\\
	&=c(1)c(3)+\sum_{n=1}^{1006}c(n)c(n+1)
	+\sum_{n=1}^{1006}(-1)^nc(n)(-1)^{n+1}c(n+1)\\
	&=c(1)c(3)+\sum_{n=1}^{1006}c(n)c(n+1)
	+\sum_{n=1}^{1006}(-1)^{2n+1}c(n)c(n+1)\\
	&=c(1)c(3)+\sum_{n=1}^{1006}c(n)c(n+1)
	-\sum_{n=1}^{1006}c(n)c(n+1)\\
	&=c(1)c(3)\\
	&=c(1)(-1)^1c(1)\\
	&=-1
	\end{flalign*}
	
	
	\item[\textbf{B2}] Let $C=\bigcup_{N=1}^{\infty}C_N,$ where $C_N$ denotes the set of 'cosine polynomials' of the form \[f(x)=1+\sum_{n=1}^Na_n\cos(2\pi nx)\] for which:
	
	(i) $f(x)\ge 0$ for all real $x,$ and\\
	(ii) $a_n=0$ whenever $n$ is a multiple of $3.$
	
	Determine the maximum value of  $f(0)$ as $f$ ranges through $C,$ and prove that this maximum is attained.
	
	\textbf{Answer.} 3.\\
	\textbf{Solution. }Consider the following:
	\begin{flalign*}
	f(x)&=1+\frac 43 \cos(2\pi x)+\frac 23\cos(4\pi x)\\
	&=1+\frac 43 \cos(2\pi x)+\frac 23 (2\cos^2(2\pi x)-1)\\
	&=\frac 13 (1+4\cos(2\pi x)+4\cos^2 (2\pi x))\\
	&=\frac 13 (1+2\cos(2\pi x))^2\\
	\end{flalign*}
	which is clearly nonnegative all the time. We also have $f(0)=1+\frac 43+\frac 23=3$, establishing the equality. To show that 3 is indeed the maximum, it suffices to show that $\sum_{n=1}^N a_n\le 2$ at all times. But plugging $x=\frac 13$ gives $\cos(\frac 23 n\pi)=-\frac 12$ if $n$ is not divisible by 3, and 1 otherwise. Considering that $a_n=0$ whenever $n$ is a multiple of 3, we have $f(\frac 13)=1-\frac 12 \sum_{n=1}^N a_n\ge 0.$ Thus $\sum_{n=1}^N a_n\le 2$ must hold. Finally note the motivation to get the example $f(x)$ as shown in the beginning: we simply find a suitable $a$ such that $2ax^2+(2-a)x+(1-a)$ is always nonnegative, which is essentially asking for the discriminant $(2-a)^2-4(2a)(1-a)\le 0$, and we get $a=\frac 23$ as the sole answer. 
	
	\item[\textbf{B3}] Let $P$ be a nonempty collection of subsets of $\{1,\dots,n\}$ such that:
	\begin{itemize}
		\item[(i)] if $S,S'\in P,$ then $S\cup S'\in P$ and $S\cap S'\in P,$ and
		\item [(ii)] if $S\in P$ and $S\ne\emptyset,$ then there is a subset $T\subset S$ such that $T\in P$ and $T$ contains exactly one fewer element than $S.$
	\end{itemize}
	
	Suppose that $f:P\to\mathbb{R}$ is a function such that $f(\emptyset)=0$ and \[f(S\cup S')= f(S)+f(S')-f(S\cap S')\text{ for all }S,S'\in P.\] Must there exist real numbers $f_1,\dots,f_n$ such that \[f(S)=\sum_{i\in S}f_i\] for every $S\in P?$
	
	\textbf{Answer.} Yes. \\
	\textbf{Solution.} Let $S_0$ be the subset such that $|S_0|\ge |S|$ for all $S\in P$. We first show that $S\subseteq S_0$ for all $S\in P$. Indeed, for an arbitrary set $S\in P$, we have $S\cup S_0\in P$ and $|S\cup S_0|\ge |S_0|$. By the maximality of $S_0$ we must have $|S\cup S_0|=|S_0|$, which can only happen when $S\subseteq S_0$, as desired. 
	
	Now, w.l.o.g. let $S_0=\{1, 2, \cdots , k\}$. By (ii) there exists $S_1\subseteq S_0$ with one element fewer than $S_0$; w.l.o.g. let it be $\{1, 2, \cdots , k-1\}$. Continuing this trend we can assume that for all $0\le i\le k$, $S_{k-i}=\{1, 2, \cdots , i\}\in P$. Consider, now, $f_1, f_2, \cdots , f_n$ such that $f_i = f(S_{k-i})-f(S_{k-i+1})$ for $i=1, 2, \cdots , k$, and arbitrary for $i=k+1, \cdots , n$. 
	The identity $f(S)=\sum_{i\in S}f_i$ holds when $S=S_0, S_1, \cdots, S_k$ with $S_k$ being the empty set (because $f(\emptyset)=0$). 
	
	To show that this identity holds for all $S\in P$, we first notice that $S\subseteq S_0=\{1, 2, \cdots , k\}$, so only $f_1, \cdots , f_k$ matter. We will proceed using the following premise with parameter $p=0, 1, \cdots , k$: the $f(S)=\sum_{i\in S} f_i$ identity holds for all $p$-element subsets $S$. We are to prove this statement for all $p=0, 1, \cdots , k$, and we will proceed by induction. 
	
	Base case: when $p=0$ we have emptyset (proven above), and when $p=1$ we have $S=\{j\}$ for some $1\le j\le k$. We have $S_{k-j}=\{1, 2, \cdots , j\}$ and $S_{k-j+1}=\{1, 2, \cdots , j-1\}$. Moreover, $f(S_{k-j+1})=\dsum_{i=1}^{j-1} f_i$ and $f(S_{k-j})=\dsum_{i=1}^j$ by how $f_i$'s are defined. Now by the definition of $f$, 
	\[
	f(S_{k-j}) = f(S\cup S_{k-j+1}) = f(S)+f(S_{k-j+1}) - f(S\cup S_{k-j+1})
	\]
	since $S=\{j\}$. Considering that $S\cup S_{k-j+1}=\emptyset$, we have $f(S\cup S_{k-j+1})=0$ and therefore 
	\[
	f(S)=f(S_{k-j}) - f(S_{k-j+1}) = \dsum_{i=1}^{j} f_i - \dsum_{i=1}^{j-1} f_i = f_j
	\]
	as desired. 
	
	Now let $2\le q\le p$ be such that the preimise is true for all $p=1, 2, \cdots , q-1$. Consider, now, any $q$-element subset $S=\{a_1, a_2, \cdots , a_q\}$. By condition (ii), there exists a subset of $S$ in $P$ with one fewer element; w.l.o.g. let it be $\{a_1, \cdots , a_{q-1}\}$. Consider, now, the set $S_{k-a_q}=\{1, 2, \cdots , a_q\}\in P.$ Consider now the two equations: 
	\[
	f(S_{k-a_q}\cup \{a_1, \cdots , a_{q-1}\}) = f(S_{k-a_q}) + f(\{a_1, \cdots , a_{q-1}\}) - f(S_{k-a_q}\cap \{a_1, \cdots , a_{q-1}\})
	\]
	\[
	f(S_{k-a_q}\cup \{a_1, \cdots , a_{q}\}) = f(S_{k-a_q}) + f(\{a_1, \cdots , a_{q}\}) - f(S_{k-a_q}\cap \{a_1, \cdots , a_{q}\})
	\]
	First, notice that $\{a_1, \cdots , a_{q-1}\}$ and $\{a_1, \cdots , a_{q}\}$ differ only by an element $a_q$, and since $a_q\in S_{k-a_q}$, we have $\{1, 2, \cdots , a_q\}\cup \{a_1, \cdots , a_{q-1}\} = \{1, 2, \cdots , a_q\}\cup \{a_1, \cdots , a_{q}\}$. 
	Comparing the two equations now give
	\[
	f(\{a_1, \cdots , a_{q}\}) - f(\{a_1, \cdots , a_{q-1}\}) = f(S_{k-a_q}\cap \{a_1, \cdots , a_{q}\}) - f(S_{k-a_q}\cap \{a_1, \cdots , a_{q-1}\})
	\]
	Since $a_q\in S_{k-a_q}$, we have $S_{k-a_q}\cap \{a_1, \cdots , a_{q-1}\}\subset S_{k-a_q}\cap \{a_1, \cdots , a_{q}\}$, differing only by an element $a_q$. 
	If $\{a_1, \cdots , a_q\}=S_{k-a_q}$ then the condition $f(S)=\dsum_{f_i\in S}f_i$ holds for this $S=\{a_1, \cdots, a_q\}$. 
	Otherwise, $S_{k-a_q}\cap \{a_1, \cdots , a_{q}\}$ will have less than $q$ elements. By the induction hypothesis, $f(S_{k-a_q}\cap \{a_1, \cdots , a_{q}\}) - f(S_{k-a_q}\cap \{a_1, \cdots , a_{q-1}\})=f_{a_q}$, and therefore $f(\{a_1, \cdots , a_{q}\}) = f(\{a_1, \cdots , a_{q-1}\}) + f_{a_q}$. But by induction hypothesis again $f(\{a_1, \cdots , a_{q-1}\}) = f_{a_1} + \cdots + f_{a_{q-1}}$, and from here the conclusion follows. 
	
	\item[\textbf{B5}] Let $X=\{1,2,\dots,n\},$ and let $k\in X.$ Show that there are exactly $k\cdot n^{n-1}$ functions $f:X\to X$ such that for every $x\in X$ there is a $j\ge 0$ such that $f^{(j)}(x)\le k.$
	
	[Here $f^{(j)}$ denotes the $j$th iterate of $f,$ so that $f^{(0)}(x)=x$ and $f^{(j+1)}(x)=f\left(f^{(j)}(x)\right).$]
	
	\textbf{Solution.} We perform induction on $n$ and $n-k$. When $k=n$ (when $n-k=0$) then any function $f:X\to X$ works, so there are $n^n$ such functions; when $k=n-1$, the only requirement is that $f(n)\neq n$ so there are $(n-1)n^{n-1}$ such functions. 
	
	Thus now consider any $k\le n-2$. Observe that since $f^{0}(x)=x$, there is no restriction on $f(1), f(2), \cdots , f(k)$, giving $n^k$ choices to each of them. We first make a following detour to a lemma: there exists $x>k$ with $f(x)\le k$. Suppose not, then we have $f:\{k+1, \cdots , n\}\to \{k+1, \cdots , n\}$ and for each $x>k$ we have $f^{(j)}(x)>k$ for any $j\ge 0$, contradiction. 
	
	Now fix $m\in [1, n-k]$ such that exactly $m$ of the numbers $k<x\le n$ have $f(x)\le k$. This gives rise of $\dbinom{n-k}{m}$ ways to choose those $x$, and each of them takes values $\{1, 2, \cdots , k\}$, giving rise to $k^m$ of them. Now w.l.o.g. assume that those $m$ elements are $k+1, \cdots , k+m$. 
	To see how would $f(x)$ looks like for the other $x>k+m$'s, consider the problem of $g:\{1, 2, \cdots , n-k\}\to \{1, 2, \cdots , n-k\}$ where for each $x$, $g^{(j)}(x)\le m$ for some $j\ge 0$. Here, $g(1), \cdots g(m)$ can be arbitrary, (i.e. $(n-k)^m$ choices), and by the induction hypothesis $n-k<n$ since $k\ge 1$, the number of such $g$'s is $m\cdot (n-k)^{n-k-1}$, meaning that there are $(n-k)^{n-k-m-1}$ choices for $g(m+1), \cdots , g(n-k)$. Thus going back to our original problem here, we consider $f|_{k+m+1, \cdots , n}$ such that $f^{(j)}(x)\le k+m$ for some $j$ (this is because, given that $f(x)>k$ for all $x>k+m$, if $j_0$ is the minimum $j$ with $f^{(j_0)}(x)\le k$ then $f^{(j_0-1)}(x)\in \{k+1, \cdots , k+m\}$). This gives $m\cdot (n-k)^{(n-k-m-1)}$ choices on $f|_{k+m+1, \cdots , n}$, giving rise of $\dbinom{n-k}{m}\cdot k^m\cdot m\cdot (n-k)^{(n-k-m-1)}$ of them in total. 
	
	Hence considering all such $m\in [1, n-k]$ gives 
	
	\begin{flalign*}
	\displaystyle\sum_{m=1}^{n-k}\dbinom{n-k}{m}\cdot k^m\cdot m\cdot (n-k)^{(n-k-m-1)}
	&=\displaystyle\sum_{m=1}^{n-k}\dfrac{(n-k)!}{m!(n-k-m)!}\cdot k^m\cdot m\cdot (n-k)^{(n-k-m-1)}
	\\&=\displaystyle\sum_{m=1}^{n-k}\dfrac{(n-k)(n-k-1)!}{(m-1)!(n-k-m)!}\cdot k^{m-1}\cdot k\cdot (n-k)^{(n-k-m-1)}
	\\&=k\displaystyle\sum_{m=1}^{n-k}\dbinom{n-k-1}{m-1}\cdot k^{m-1}(n-k)^{(n-k-m)}
	\\&=k(k+n-k)^{m-k-1}
	\\&=kn^{n-k-1}
	\end{flalign*}
	so combining with the arbitrary choice of $f|_{1, 2, \cdots , k}$ we have $kn^{n-k-1}n^k=kn^{n-1}$, as desired. 
	
\end{enumerate}

\end{document}