\documentclass[11pt,a4paper]{article}
\usepackage{amsmath, amssymb, fullpage, mathrsfs, bm, pgf, tikz, bbm}
\usepackage{mathrsfs,amsthm}
\usetikzlibrary{arrows}
\setlength{\textheight}{10in}
%\setlength{\topmargin}{0in}
\setlength{\topmargin}{-0.5in}
\setlength{\parskip}{0.1in}
\setlength{\parindent}{0in}

\begin{document}
	\newcommand{\la}{\leftarrow}
	\newcommand{\lra}{\leftrightarrow}
	\newcommand{\bbN}{{\mathbb N}}
	\newcommand{\bbZ}{{\mathbb Z}}
	\newcommand{\bbQ}{{\mathbb Q}}
	\newcommand{\bbR}{{\mathbb R}}
	\newcommand{\bbC}{{\mathbb C}}
	\newcommand{\bbH}{{\mathbb H}}
	\newcommand{\bbE}{{\mathbb E}}
	\newcommand{\bbP}{{\mathbb P}}
	\newcommand{\dfeq}{\stackrel{\mathrm{def}}{=}}
	\newcommand{\ra}{\rightarrow}
	\newcommand{\Span}{\mathrm{span}}
	\newcommand{\scrP}{\mathscr{P}}
	\newcommand{\rank}{\mathrm{rank}}
	\newcommand{\nullity}{\mathrm{nullity}}
	\newcommand{\Col}{\mathrm{Col}}
	\newcommand{\Row}{\mathrm{Row}}
	\newcommand{\tr}{\mathrm{tr}}
	\newcommand{\ol}{\overline}
	\newcommand{\norm}[1]{||#1||}
	\newcommand{\doubleline}[1]{\underline{\underline{#1}}}
	\newcommand{\elemop}[1]{\stackrel{#1}{\longrightarrow}}
	\newcommand{\Ind}{\mathrm{Ind}}
	\newcommand{\Res}{\mathrm{Res}}
	\newcommand{\End}{\mathrm{End}}
	\newcommand{\cl}{\mathrm{cl}}
	\newcommand{\code}[1]{\texttt{#1}}
	\newcommand\tab[1][0.5cm]{\hspace*{#1}}
	\newcommand{\<}{\langle}
	\renewcommand{\>}{\rangle}
	\newcommand{\qubits}[1]{|{#1}\rangle}
	\newcommand{\powset}{\mathcal{P}}
	\newcommand{\dsum}{\displaystyle\sum}
	\newcommand{\dprod}{\displaystyle\prod}
	
	\newtheorem{lemma}{Lemma}
	
	\section*{Putnam 2022 Solutions}
	
	\section*{Session A}
	\begin{enumerate}
		\item [\textbf{A1.}]
		Determine all ordered pairs of real numbers $(a,b)$ such that the line $y=ax+b$ intersects the curve $y=\ln(1+x^2)$ in exactly one point.
		
		\textbf{Answer.} 
		We have these three families of solutions: 
		\begin{equation}
			\begin{cases}
				b = 0 & a=0\\
				\forall b & |a|\ge 1\\
				b < \ln(\frac{2(1 + \sqrt{1-a^2})}{a^2}) - (1 + \sqrt{1-a^2})
				\text{ or }
				b > 
				\ln(\frac{2(1 - \sqrt{1-a^2})}{a^2}) - (1 - \sqrt{1-a^2})
				& 0 < |a| < 1\\
			\end{cases}
		\end{equation}
		
		\textbf{Solution.}
		For each $a\in\bbR$, 
		define the function $f_a(x) = \ln(1 + x^2) - ax$. 
		Then we have $(a, b)$ a suitable pair if and only if $f_a(x) = b$ has a unique solution. 
		Since $f_a(x)=f_{-a}(-x)$ for all $a, x\in\bbR$, it suffices to consider $a\ge 0$. 
		
		For $a=0$, we have $f_a(x)=f_a(-x)$, 
		and for all $x \neq 0$, $f_a(x) > 0 =  f_a(0)$. Thus $b=0$ is the only solution here. 
		For all $a > 0$, we see that 
		\begin{equation}
			\lim_{x\to-\infty} f_a(x)=+\infty\qquad 
			\qquad 
			\lim_{x\to+\infty} f_a(x)=-\infty
		\end{equation}
	    The first is because $f_a(x)\ge -ax$ for all $x$ and $-ax\to\infty$ 
	    as $-\infty$; 
	    the second is because asymtotically, 
	    as $x\to+\infty$ we have $\ln(1+x^2)=o(x)$ (i.e. grows slower than $x$). 
		It therefore follows that $f_a$ is surjective for all $a > 0$ (given that $f_a$ is also continuous). 
		
		Next, notice that (henceforth derivatives are w.r.t. $x$)
		\begin{equation}
			f_a'(x) = \frac{2x}{1 + x^2} - a
		\end{equation}
	    and we see that $1+x^2-|2x| = (|x|-1)^2\le 1$. Therefore $|\frac{2x}{1 + x^2}|\le 1$. 
	    This means that if $a\ge 1$, $f'_a(x)\le 0$ with the only equality at $a=1, x=1$. 
	    It then follows that $f_a$ is decreasing when $a\ge 1$, and therefore injective. 
	    Combined with the surjectivity of $f_a$ we have $f_a(x)=b$ has a unique solution for all real $b$. 
	    
	    Finally, 
	    consider $0 < a < 1$. 
	    Denote: 
	    \[
	    a_1 = \frac{1 - \sqrt{1-a^2}}{a}
	    \qquad 
	    a_2 = \frac{1 + \sqrt{1-a^2}}{a}
	    \]
	    These are the roots of $f_a'(x)=0$, 
	    and $f_a'(x)>0$ for $x\in (a_1, a_2)$ and $<0$ for $x<a_1$ or $x>a_2$. 
	    Therefore, $f_a$ is increasing in $x\in (a_1, a_2)$ and decreasing otherwise. 
	    Since $f_a$ is surjective and continuous, 
	    and $f_a(a_1) > f_a(a_2)$, 
	    it follows that for each $b$ there's a solution $f_a(x)=b$ with $x\not\in [a_1, a_2]$. 
	    On the other hand, if $b\not\in [f_a(a_2), f_a(a_1)]$, 
	    then it has solution in either $x < a_1$ or $x>a_2$ but not both. 
	    Therefore $b$ is suitable if and only if $b\not\in [f_a(a_2), f_a(a_1)]$. 
	    Now, 
	    \begin{equation}
	    	f_a(a_1) = 
	    	\ln(\frac{2(1 - \sqrt{1-a^2})}{a^2}) - (1 - \sqrt{1-a^2})
	    	\qquad
	    	f_a(a_2) = 
	    	\ln(\frac{2(1 + \sqrt{1-a^2})}{a^2}) - (1 + \sqrt{1-a^2})
	    \end{equation}
        The conclusion follows. 
        For $a < 0$ the answer is the same (by changing $a$ to $-a$). 
		
		\item [\textbf{A2.}]
		Let $n$ be an integer with $n\geq 2.$ Over all real polynomials $p(x)$ of degree $n,$ what is the largest possible number of negative coefficients of $p(x)^2?$
		
		\textbf{Answer.} $2n-2$. 
		
		\textbf{Solution.} 
		The construction is given as the following: 
		\begin{equation}
			p(x) = x^n + 1 - \epsilon(x + x^2+\cdots + x^{n-1})
		\end{equation}
	    where $0 < \epsilon < \frac{1}{2n}$. 
	    If $p(x)^2 = \sum a_ix^i$ for $i=0, \cdots, 2n$ then for $i=1, \cdots, n-1$: 
	    \begin{equation}
	    	a_k = 
	    	\begin{cases}
	    		(k-1)\epsilon^2 - 2\epsilon & 1\le k\le n-1\\
	    		(2n-1-k)\epsilon^2 - 2\epsilon & n + 1\le k\le 2n-1
	    	\end{cases}
	    \end{equation}
        With $0<\epsilon<\frac{1}{2n}$ we have $(k-1)\epsilon < 2$ for all $k\le n-1$ 
        and $(2n-1-k)\epsilon<2$ for all $k \ge n+1$. 
        Thus $a_k<0$ for all $k=1, \cdots, n-1, n+1, \cdots, 2n-1$. 
        
        To show that $2n-2$ cannot be improved, 
        let $p(x)$ be arbitrary and set $a_0, \cdots, a_{2n}$ to be the coefficient of $1, \cdots, x^{2n}$ in $p(x)^2$. 
        We have $a_0, a_{2n}\ge 0$ so it remains to show that $a_k\ge 0$ for one other index. 
        Let $b_0, \cdots, b_n$ to be the coefficient of $1, \cdots, x^n$ in $p(x)$. 
        Then either $b_0$ and $b_n$ have the same sign (assume that 0 is on the positive side here)
        or $b_1$ is of the same sign as exactly one of $b_0, b_n$. 
        In any case there exists $i$ such that $i\neq 0$ and $b_0b_i\ge 0$, 
        or $i\neq n$ and $b_nb_i\ge 0$. 
        
        Now suppose we have the former case, and let $k>0$ to be the minimal index such that $b_0b_k\ge 0$. 
        Then $a_k = \sum_{i+j=k} b_ib_j$. 
        Since $b_0b_k\ge 0$, and by the minimality of $k$, $b_ib_j\ge 0$ for all $0<i, j<k$, 
        $a_k\ge 0$, as claimed (notice that $1\le k\le n$). 
		
		\item [\textbf{A3.}]
		Let $p$ be a prime number greater than 5. Let $f(p)$ denote the number of infinite sequences $a_1, a_2, a_3,\ldots$ such that $a_n \in \{1, 2,\ldots, p-1\}$ and $a_na_{n+2}\equiv1+a_{n+1}$ (mod $p$) for all $n\geq 1.$ Prove that $f(p)$ is congruent to 0 or 2 (mod 5).
		
		\textbf{Solution.} 
		We'll use the fact that a valid sequence is uniquely determined by $(a_1, a_2)$, i.e. it suffices to count the number of suitable $(a_1, a_2)$.
		First, we see that the sequence is periodic. 
		In fact, we see that 5 is a period via the following construction on $a_1, a_2, \cdots$:
		\[
		a_1, a_2, \frac{a_2 + 1}{a_1}, \frac{a_1 + a_2 + 1}{a_1a_2}, \frac{a_1 + 1}{a_2}, a_1, a_2, \cdots
		\]
		Since 5 is a prime, the minimal period is either 1 or 5.
		
		Now to count the number of such sequences, we note that $a_i=a_j$ and $a_{i+1}=a_{j+1}$ implies that $a_{i+k}=a_{j+k}$ for all $k\ge -\min(i, j) + 1$, 
		so the 5 pairs $(a_1, a_2), (a_2, a_3), (a_3, a_4), (a_4, a_5), (a_5, a_1)$ are either all distinct or all equal. 
		Formally, we may define equivalence relation $\sim$ such that $(a, b)\sim (c, d)$ if and only if there exists a sequence 
		$(a_1, a_2, \cdots)$ such that $(a, b)=(a_i, a_{i+1})$ and $(c, d)=(a_j, a_{j+1})$. 
		(The identity and symmetry condition of $\sim$ is clear; for transitivity, if $a_1, \cdots$ and $b_1, \cdots$ are such that $a_i=b_j$ and $a_{i+1}=b_{j+1}$ for some $i\neq j$ then if $k=j-i > 0$ we have $a_{i'}=b_{i'+k}$, 
		i.e. a cyclic shift). 
		Each class has size either 5 or 1, so to verify the conclusion, it suffices to count the number of classes with size 1, 
		that is, the number of $x$'s such that there's a sequence with $x=a_1=a_2=a_3=\cdots$. 
		This is the same as saying that $x^2=x+1\pmod{p}$, 
		or $x^2-x=1\pmod{p}$. 
		Now for each $x, y$ we have 
		\[
		(x^2-x)-(y^2-y)=(x-y)(x+y-1)
		\]
		so $x^2-x\equiv y^2-y$ iff $x=y$ or $x+y\equiv 1$. Thus if there's one solution $x$ satisfying $x^2-x\equiv 1$, 
		we also have $(1-x)^2-(1-x)\equiv 1$. If $x\equiv 1-x$, then we have $x\equiv \frac{p+1}{2}$ 
		which means $x^2-x\equiv -\frac 14$ (multiplicate inverse allowed here since $p$ is odd). 
		But $-\frac 14\equiv 1$ happens only when $p=5$, so in fact $x\not\equiv 1-x\pmod{p}$. 
		This shows that the number of such $x$'s is either 0 or 2, 
		and so $f(p)\equiv 0\text{ or }2\pmod{5}$. 
		
		\textbf{Remark}. In fact, such a sequence is valid if and only if $a_1, a_2\in \{1, \cdots, p-1\}$ and $a_1+a_2\not\equiv -1\pmod{p}$. 
		This gives $(p-2)(p-3)\pmod{5}$, which is always congruent to 0 or 2 mod 5. 
		
		\item [\textbf{A4.}]
		Suppose that $X_1, X_2, \ldots$ are real numbers between 0 and 1 that are chosen independently and uniformly at random. Let \[S=\sum_{i=1}^k \frac{X_i}{2^i}\] 
		where $k$ is the least positive integer such that $X_k<X_{k+1},$ or $k=\infty$ if there is no such integer. Find the expected value of $S.$
		
		\textbf{Answer.} $2e^{1/2}-3$. 
		
		\textbf{Solution.} 
		By the definition of $k$, we have $X_1\ge X_2\ge \cdots X_k < X_{k + 1}$. 
		We may therefore write $S$ into the following form: 
		\begin{align}
			S&=\sum_{i=1}^k \frac{X_i}{2^i} 
			\nonumber\\
			&=\sum_{i=1}^{\infty} \frac{X_i}{2^i} \cdot \mathbbm{1}\{i\le k\}
			\nonumber\\
			&=\sum_{i=1}^{\infty} \frac{X_i}{2^i} \cdot \mathbbm{1}\{X_1\le X_2\le \cdots \le X_i\}
		\end{align}
	    Therefore, by the linearity of expectation, we have 
	    \begin{align}
	    	\bbE[S]
	    	= \bbE[\sum_{i=1}^{\infty} \frac{X_i}{2^i} \cdot \mathbbm{1}\{X_1\le X_2\le \cdots \le X_i\}]
	    	=\sum_{i=1}^{\infty} \frac{1}{2^i} \bbE[ X_i\cdot \mathbbm{1}\{X_1\le X_2\le \cdots \le X_i\}]
	    \end{align}
		Let's now establish the following: 
		\begin{lemma}
			For all $x\in [0, 1]$ we have 
			\begin{equation}
				\bbP[X_1\ge \cdots \ge X_k\ge x] = \frac{(1-x)^k}{k!}
			\end{equation}
		\end{lemma}
	    \begin{proof}
	    	We show two ways to establish this. 
	    	
	    	First principle solution: the LHS probability can be written as the following integration: 
	    	\begin{align}
	    		\int_{X_1=x}^1 \int_{X_2=x}^{X_1}\cdots \int_{X_k=x}^{X_{k-1}} 1 dX_k\cdots X_1
	    		\nonumber\\
	    		\int_{X_1=x}^1 \int_{X_2=x}^{X_1}\cdots \int_{X_{k-1}=x}^{X_{k-2}} (X_{k - 1} - x) dX_{k-1}\cdots X_1
	    		\nonumber\\
	    		\int_{X_1=x}^1 \int_{X_2=x}^{X_1}\cdots \int_{X_{k-2}=x}^{X_{k-3}} \frac{(X_{k - 2} - x)^2}{2} dX_{k-2}\cdots X_1
	    		\nonumber\\
	    		\vdots
	    		\nonumber\\
	    		\int_{X_1=x}^1 \int_{X_2=x}^{X_1}\cdots \int_{X_{k-\ell}=x}^{X_{k-\ell-1}} \frac{(X_{k - \ell} - x)^{\ell}}{\ell!} dX_{k-\ell}\cdots X_1
	    		\nonumber\\
	    		\vdots
	    		\nonumber\\
	    		\int_{X_1=x}^1 \frac{(X_1-x)^{k-1}}{(k-1)!} dX_1
	    		\nonumber\\
	    		=\frac{(1-x)^k}{k!}
	    	\end{align}
    	
    	    Solution via symmetry: we write the probability into the following: 
    	    \begin{equation}
    	    	\bbP[X_1\ge \cdots \ge X_k\ge x] = 
    	    	\bbP[X_i\ge x, \forall i=1, \cdots, k]
    	    	\bbP[X_1\ge\cdots \ge X_k \mid X_i\ge x, \forall i=1, \cdots, k]
    	    \end{equation}
            We have $\bbP[X_i\ge x, \forall i=1, \cdots, k]=(1-x)^k$. 
            For the second probability, we consider all the $k!$ permutations $\sigma$ of $1, \cdots, k$. 
            All the $k!$ combinations $(X_{\sigma(1)}, \cdots, X_{\sigma(k)})$ have the same probability density; 
            consider the case where $X_1, \cdots, X_k$ pairwise distinct, 
            we have exactly one $\sigma$ that has $X_{\sigma(1)}\ge\cdots\ge X_{\sigma(k)}$. 
            Since we have $X_1, \cdots, X_k$ pairwise distinct almost surely (i.e. probability 1), 
            we have $\bbP[X_1\ge\cdots \ge X_k \mid X_i\ge x, \forall i=1, \cdots, k]=\frac{1}{k!}$ 
            by the symmetry argument (i.e. the unique $\sigma$ among $k!$ permutations). 
            Thus multiplying the two gives $\frac{(1-x)^k}{k!}$. 
	    \end{proof}
        
        Thus now we have, for each $i\ge 1$, 
        \begin{align}
        	\bbE[ X_i\cdot \mathbbm{1}\{X_1\le X_2\le \cdots \le X_i\}]
        	&=\bbE[X_i\cdot \frac{(1-X_i)^{i-1}}{(i - 1)!}]
        	\nonumber\\
        	&=\int_0^1 \frac{x(1-x)^{i-1}}{(i-1)!}
        	\nonumber\\
        	&=\int_0^1 \frac{(1-x)^{i-1} - (1-x)^i}{(i-1)!}
        	\nonumber\\
        	&=\frac{1}{(i-1)!}(\frac{1}{i} - \frac{1}{i + 1})
        	\nonumber\\
        	&=\frac{1}{(i + 1)!}
        \end{align}
        And therefore, 
        \begin{align}
        	\bbE[S]&=\sum_{i=1}^{\infty}\frac{1}{2^i(i+1)!}
        	\nonumber\\
        	&=2\sum_{i=2}^{\infty}\frac{1}{2^i i!}
        	\nonumber\\
        	\\&=2(\sum_{i=0}^{\infty}\frac{1}{2^i i!} - \frac{1}{2^0 0!} - \frac{1}{2^1 1!})
        	\nonumber\\
        	\\&=2e^{1/2} - 3
        \end{align}
	\end{enumerate}
	
	\section*{Section B}
	
	\begin{enumerate}
		\item [\textbf{B1.}]
		Suppose that $P(x)=a_1x+a_2x^2+\ldots+a_nx^n$ is a polynomial with integer coefficients, with $a_1$ odd. Suppose that $e^{P(x)}=b_0+b_1x+b_2x^2+\ldots$ for all $x.$ Prove that $b_k$ is nonzero for all $k \geq 0.$
		
		\textbf{Solution.} 
		We first note that $b_0 = 1$ and $b_1 = a_1\neq 0$. For $k\ge 2$ we can show that:
		\begin{equation}
		b_k = \sum_{m=1}^k \sum_{(c_1, \cdots, c_m): c_1 + \cdots + c_m = k} \frac{a_{c_1}a_{c_2}\cdots a_{c_m}}{m!}
		=\frac{1}{k!}\sum_{m=1}^k \sum_{(c_1, \cdots, c_m): c_1 + \cdots + c_m = k}\frac{k!}{m!}a_{c_1}a_{c_2}\cdots a_{c_m}
		\end{equation}
		where $a_i = 0$ for all $i > n$. 
		Using the fact that $k(k - 1)$ is even for all $k$, we have $\frac{k!}{m!}$ an even integer for all $m\le k-2$. 
		Therefore, 
		\begin{align}
		k!b_k
		&=\sum_{m=1}^k \sum_{(c_1, \cdots, c_m): c_1 + \cdots + c_m = k}\frac{k!}{m!}a_{c_1}a_{c_2}\cdots a_{c_m}
		\nonumber\\
		&\equiv k(k - 1)a_1^{k-2}a_2 + a_1^k
		\nonumber\\
		&\equiv a_1^k
		\not\equiv 0\pmod{2}
		\end{align}
		
		\item [\textbf{B2.}]
		Let $\times$ represent the cross product in $\mathbb{R}^3.$ For what positive integers $n$ does there exist a set $S \subset \mathbb{R}^3$ with exactly $n$ elements such that$$S=\{v \times w: v, w \in S\}?$$
		
		\textbf{Answer.} $n=1, 7$. 
		
		\textbf{Solution.} 
		For $n=1$ we have $S=\{(0, 0, 0)\}$; 
		for $n=7$ we have $$S=\{(0, 0, 0), \pm(1, 0, 0), \pm(0, 1, 0), \pm(0, 0, 1)\}$$
		
		We now show that these are the only examples, via the following lemma: 
		\begin{lemma}
			  All nonzero vectors are unit vectors and are either perpendicular or parallel to each other. 
		\end{lemma}
	    
	    \begin{proof}
	    	  By the finiteness of $S$, we may consider $v$ to be the longest vector in $S$. 
	    	  Suppose $|v| > 1$. 
	    	  Let $v  = s\times t, s, t\in S$, then $1 < |v| = |s|\cdot |t|\cdot |\sin (s, t)|\le |s|\cdot |t|$ so either 
	    	  $|s| > 1$ or $|t| > 1$. 
	    	  In addition, both $s, t$ are perpendicular to $t$. 
	    	  Now, if $|s| > 1$ then $w = s\times v\in S$ and $|w| = |s| \cdot |v| > |v|$, 
	    	  contradicting the maximality of $|v|$. 
	    	  
	    	  Now consider $u$ to be the shortest nonzero vector in $S$. 
	    	  Suppose $|u| < 1$. 
	    	  Let $u  = s\times t, s, t\in S$ for some $s, t$. 
	    	  Let $r = u\times s$. Then $u, s, r$ are mutually perpendicular to each other and are nonzero. 
	    	  In addition, $|s|\le 1$ so $|r|\le |u|$. 
	    	  Since $|r\times u|=|r|\cdot |u|=|u|^2 < |u|$ and $r\times u\neq 0$, 
	    	  this contradicts the minimality of $|u|$ among the nonzero vectors. 
	    	  
	    	  Finally, now that all nonzero vectors are unit, 
	    	  let $s, t\in S$ with $|s| = |t| = 1$. 
	    	  Then $|s\times t|$ is either 0 or 1, 
	    	  showing that $s$ and $t$ are either parallel or perpendicular to each other. 
	    \end{proof}
        
        Now if $S$ is nonempty, then $v\in S$ means $0=v\times v\in S$.
        In addition, $v\times w=-w\times v$, 
        and therefore $s\in S\to -S\in S$. 
        If $n > 1$, 
        then we can pick $s\in S$ that's nonzero, 
        and therefore 
        $s=t\times u$ for some $t, u\in S$. 
        From our lemma, $s, t, u$ are all unit and are mutually perpendicular to each other. 
        We also have $-s, -t, -u\in S$ since $S$ is closed under negation. 
        Together with $0\in S$ we have $|S|\ge 7$. 
        Since a 3D space cannot admit more than 3 mutually perpendicular vectors (up to scalar constants), 
        these 7 vectors are all the elements in $S$. 
		
		\item [\textbf{B3.}]
		Assign to each positive real number a color, either red or blue. Let $D$ be the set of all distances $d>0$ such that there are two points of the same color at distance $d$ apart. Recolor the positive reals so that the numbers in $D$ are red and the numbers not in $D$ are blue. If we iterate the recoloring process, will we always end up with all the numbers red after a finite number of steps?
		
		\textbf{Answer}. 
		Yes. In fact, we show that all numbers will be red after 2 iterations. 
		
		\textbf{Solution.} 
		Consider all numbers not in $D$ (for the first iteration), and partition them into classes of rational ratio, 
		i.e. each class has the form $E_v \subseteq \bbQ_v\triangleq \{vq: q\in \bbQ^+\}$. 
		Consider one such class $E_v$ and let $Q_v = \{q\in\bbQ, qv\in E_v\}$. 
		For each rational number $q>0$, we define the function $\nu_2$ such that, 
		if $q\in\bbN$, then $\nu_2(q)=\max\{k: 2^k\mid q\}$ (i.e. the highest exponent of 2 dividing $q$), 
		and if $q=\frac{r}{s}$ with $r, s\in\bbN$ then $\nu_2(q)=\nu_2(r)=\nu_2(s)$. 
		Here comes our key claim: 
		
		\begin{lemma}
			  All the rationals in $Q_v$ has the same $\nu_2$. 
			  That is, there exists a constant $c\triangleq c(v)$ such that $\forall q\in Q_v: \nu_2(q)=c$. 
		\end{lemma}
	    
	    \begin{proof}
	    	  Consider $q_1$ and $q_2\in Q_v$, and choose any point $a$ on the first iteration. 
	    	  We see that by the definition of $D$ and that $q_1v, q_2v\not\in D$, 
	    	  $a, a+q_1v, a+2q_1v, a+3q_1v, \cdots$ must be in alternating colour, 
	    	  so $a$ and $a+nq_1v$ are of the same colour if and only if $n$ is even. 
	    	  Similarly, $a$ and $a+nq_2v$ are of the same colour if and only if $n$ is even. 
	    	  Now consider integers $r_1, t_2$ such that $q_1r_1=q_2r_2$, and such that $\gcd(r_1, r_2)=1$. 
	    	  Since $r_1q_1v=r_2q_2v$, 
	    	  $r_1, r_2$ must have the same parity, i.e. both odd. 
	    	  Thus $\nu_2(q_1)=\nu_2(q_2)$ must hold here. 
	    \end{proof}
        
        Now consider what happens after the first iteration. 
        On second iteration, we see that we have a constant (integer) $c(v)$ such that 
        if $\nu_2(q)\neq c(v)$ then $qv$ is painted red. 
        Consider, now, any rational number $r$. 
        Let $s=2^{\min(c(v), \nu_2(r)) - 1}$, 
        then with $\nu_2(s) < \nu_2(r)$ we have $\nu_2(s)=\nu_2(s+r) < c(v)$, 
        which means that both $sv$ and $rv$ are painted red here. 
        It then follows that $rv$ will be painted red in the next round. 
        Considering all such $r\in\bbQ^+$ and all such classes $Q_v$ we have all points painted red after this iteration. 
		
		\item [\textbf{B4.}]
		Find all integers $n$ with $n \geq 4$ for which there exists a sequence of distinct real numbers $x_1, \ldots, x_n$ such that each of the sets$$\{x_1, x_2, x_3\}, \{x_2, x_3, x_4\},\ldots,\{x_{n-2}, x_{n-1}, x_n\}, \{x_{n-1}, x_n, x_1\},\text{ and } \{x_n, x_1, x_2\}$$forms a 3-term arithmetic progression when arranged in increasing order.
		
		\textbf{Answer.} 
		All $n$ such that $n\ge 9$ and $3\mid n$.
		
		\textbf{Solution.} 
		We first determine one such construction. 
		Let $k\ge 2$ and $n=3(k+1)$, 
		and consider the following $n$ numbers: 
		\[
		0, 4, 8, \cdots, 4k, 4k-2, 4k-1, 4k-3, 4k-5, \cdots, 1, 2
		\]
		The odd numbers are $1, \cdots, 4k-1$, the even numbers are $0, \cdots, 4k$, and $2$ and $4k-2$ (which are distinct since $k\ge 2$). 
		Also the first $k+1$ numbers and the sequence of odd numbers define chains of arithmetic progression, so it remains to check the following: 
		\[
		(4k-4, 4k, 4k-2), (4k, 4k-2, 4k-1), (4k-2, 4k-1, 4k-3), 
		(3, 1, 2), (1, 2, 0), (2, 0, 4)
		\]
		which, when sorted within each triples, becomes 
		\[
		(4k-4, 4k-2, 4k), (4k-2, 4k-1, 4k), (4k-3, 4k-2, 4k-1), 
		(1, 2, 3), (0, 1, 2), (0, 2, 4)
		\]
		and therefore valid. 
		
		To show necessity, we first show that $3\mid n$. 
		Denote $d_i = x_{i+1}-x_i$ (indices taken modulo $n$), then we have the following properties on $d_i$: 
		\begin{itemize}
			\item $\sum_{i=1}^n = 0$ (consistency); 
			
			\item $d_{i + 1}\in \{d_i, -\frac{d_i}{2}, -2d_i\}$ (arithmetic progression); 
			
			\item For any $i < j$ and $(i, j)\neq (1, n)$ we have $d_i + \cdots + d_j\neq 0$ (all $x_i$'s distinct). 
		\end{itemize}
	    
	    By scaling, we may assume that $\min_{i=1, \cdots, n} |d_i| = 1$, 
	    and by flipping signs we may also assume that there exists an $i$ such that $d_i=1$. 
	    Since $1 = (-2)^0, -2=(-2)^1$ and $-\frac 12=(-2)^{-1}$, for each $i$ there exists a $k_i$ such that 
	    $d_i = (-2)^{k_i}$, 
	    and with $\min d_i=1$ we have $\min k_i=0$. 
	    In addition, $|k_{i+1} - k_i|\le 1$. 
	    Since $-2\equiv 1\pmod{3}$, $d_i\equiv 1\pmod{3}$ and therefore 
	    \begin{equation}
	    	0 = \sum_{i=1}^n d_i \equiv \sum_{i=1}^n 1 = n\pmod{3}
	    \end{equation}
        i.e. $3\mid n$. 
        
        It now remains to show that we do not have an example for $n=6$. 
        First, since $d_i$ are integers and $(-2)^k$ is even for $k\ge 1$, 
        there's at least two numbers $d_i$ that's equal to 1. 
        Next, we cannot have all $d_i=1$ since the sum has to be 0. 
        Considering a block $B$ of consecutive 1's among the $d_i$'s; 
        adjacent to the block has to be $-2$ (since $\min |d_i|=1$). 
        If $B$ has length $\ge 2$ then we have a consecutive chain of $-2, 1, 1$ which sums to 0, 
        so $x_i=x_{i+3}$ for some $i$, which is a contradiction. 
        It follows that we must have the two $d_i=1$'s having spaced either 2 or 3 apart (on the circle of 6 $d_i$'s). 
        Considering that the $d_i=1$ must be neighboured by $-2$'s, we have one of the following two scenarios: 
        \begin{align}
        	  (-2, 1, -2, -2, 1, -2)\\
        	  (1, -2, 1, -2, a, -2)\\
        \end{align}
        The first one is impossible since the sum is $-6$; 
        the second one has sum $-2+a$, which follows that $a=2$. 
        But this contradicts that $a=(-2)^k$ for some $k\in\bbZ_{\ge 0}$. 
		
	\end{enumerate}
\end{document}