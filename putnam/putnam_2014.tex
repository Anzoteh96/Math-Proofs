\documentclass[11pt,a4paper]{article}
\usepackage{amsmath, amssymb, fullpage, mathrsfs, bm, pgf, tikz}
\usepackage{mathrsfs}
\usetikzlibrary{arrows}
\setlength{\textheight}{10in}
%\setlength{\topmargin}{0in}
\setlength{\topmargin}{-0.5in}
\setlength{\parskip}{0.1in}
\setlength{\parindent}{0in}

\newcommand{\set}[2]{\{#1\,:\,\text{#2}\}}
\newcommand{\tup}[1]{\mathrm{#1}}
\newcommand{\sfP}{\mathsf{P}}
\newcommand{\M}{\mathsf{M}}
\newcommand{\bbR}{\mathbb R}
\newcommand{\bbC}{\mathbb C}
\newcommand{\bbZ}{\mathbb Z}
\newcommand{\bbN}{\mathbb N}
\newcommand{\bbQ}{\mathbb Q}
\newcommand{\bbF}{\mathbb F}
\newcommand{\bbP}{\mathbb P}
\newcommand{\bbE}{\mathbb E}
\newcommand{\dfeq}{\stackrel{\mathrm{def}}{=}}
\newcommand{\ra}{\rightarrow}
\newcommand{\la}{\leftarrow}
\newcommand{\lra}{\leftrightarrow}
\newcommand{\Span}{\mathrm{span}}
\newcommand{\scrP}{\mathscr{P}}
\newcommand{\rank}{\mathrm{rank}}
\newcommand{\nullity}{\mathrm{nullity}}
\newcommand{\Col}{\mathrm{Col}}
\newcommand{\Row}{\mathrm{Row}}
\newcommand{\tr}{\mathrm{tr}}
\newcommand{\ol}{\overline}
\newcommand{\norm}[1]{||#1||}
\newcommand{\doubleline}[1]{\underline{\underline{#1}}}
\newcommand{\elemop}[1]{\stackrel{#1}{\longrightarrow}}
\newcommand{\Ind}{\mathrm{Ind}}
\newcommand{\Res}{\mathrm{Res}}
\newcommand{\End}{\mathrm{End}}
\newcommand{\cl}{\mathrm{cl}}
\newcommand{\code}[1]{\texttt{#1}}
\newcommand\tab[1][0.5cm]{\hspace*{#1}}
\newcommand{\<}{\langle}
\renewcommand{\>}{\rangle}
\newcommand{\qubits}[1]{|{#1}\rangle}
\newcommand{\ord}{\mathrm{ord}}
\newcommand{\lcm}{\mathrm{lcm}}
\newcommand{\dsum}{\displaystyle\sum}
\newcommand{\dprod}{\displaystyle\prod}

\begin{document}

\newcommand{\sgn}{\text{sgn}}
\setcounter{secnumdepth}{0}

\section{Putnam 2014}
\begin{enumerate}
	\item [\textbf{A1}]Prove that every nonzero coefficient of the Taylor series of $(1-x+x^2)e^x$ about $x=0$ is a rational number whose numerator (in lowest terms) is either $1$ or a prime number.
	
	\textbf{Solution.} We consider the expansion $(1-x+x^2)\displaystyle\sum_{i=0}^{\infty} \frac{1}{i!}x^i$. The coefficient of the constant term is 1 and the coefficient of the $x$-term is $\frac{1}{2}-1=-\frac{1}{2}$. The conefficient of $x^{k+1}$ for all $k\ge 1$ is given by $\frac{1}{(k+1)!}-\frac{1}{k!}+\frac{1}{(k-1)!}
	=\frac{1-(k+1)+k(k+1)}{(k+1)!}
	=\frac{k^2}{(k+1)!}
	=\frac{k}{(k-1)!(k+1)}
	$
	If $k$ is prime we are done. 
	Now assume that it's not. If $k$ is not a prime power, write $k=ab$ with $1<a, b<k$ and $\gcd(a, b)=1$ (for example, let $p$ to be a prime divisor of $k$ and let $r$ to be the maximum power of $p$ dividing $k$. Then since $k$ is not a prime power, $p^r<k$ and $k/(p^r)$ is relatively prime to $p^r$ by the maximality of $r$). Since $a, b<k$, $a|(k-1)!$ and $b|(k-1)!$ and with $\gcd(a, b)=1$, this implies that $k=ab|(k-1)!$. Otherwise, $k=p^r$ for some prime $p$ and $r\ge 2$. Using the formula $v_p((r)!)=\sum_{i=1}^\infty\lfloor\frac{r}{p^i}\rfloor$, we have 
	$v_p((k-1)!)=\sum_{i=1}^\infty\lfloor\frac{k-1}{p^i}\rfloor
	=\sum_{i=1}^\infty\lfloor\frac{p^r-1}{p^i}\rfloor
	=p^{r-1}-1+p^{r-2}-1+\cdots + (p-1)
	\ge 1+1+\cdots + 1
	= r-1
	$
	since $p\ge 2$. Thus $p^{r-1}|(k-1)!$ and so when taking the lowest term the numerator can either be 1 or $p$. 
	
	\item[\textbf{A2}]Let $A$ be the $n\times n$ matrix whose entry in the $i$-th row and $j$-th column is \[\frac1{\min(i,j)}\] for $1\le i,j\le n.$ Compute $\det(A).$
	
	\textbf{Answer.} $(-1)^{n}\frac{1}{n[(n-1)!]^2}$\\
	\textbf{Solution.} We use the well-known matrix identity that row reduction preserves determinant, and we will do row reduce profusely. For brevity, we will denote $f(i)=\frac{1}{i}$ for all $i\ge 1$. Denoting $a_{ij}$ as the $i$-th row and the $j$-th column. Then we have $a_{ij}=f(\min(i, j))$. 
	
	Now, for each iteration stepped $i$, denoting row $i$ as $r_i$ and we will do $r_{j}:=r_{j}-r_i$ for all $j\ge i$. We show that after the $k$-th iteration below would be the value for $a_{ij}$: 
	\begin{itemize}
		\item For $i=1$, we have $a_{ij}=f(1)$ as always. 
		\item For $i\le k$, we have $a_{ij}=0$ for all $j<i$, and $a_{ij}=f(i)-f(i-1)$ for all $j\ge i$. 
		\item For $i>k$, $a_{ij}=0$ for $j\le k$, and $a_{ij}=f(\min(i, j))-f(k)$ otherwise. 
	\end{itemize}
	To prove this by inducting on $k$, the base case is given when all the numbers after the first row are subtracted by the corresponding number in the first row, so for $i>1$, $a_{ij}$ becomes $a_{ij}=f(\min(i, j))-f(1)$, and the condition above is satisfied. Suppose that the conjecture holds after $k$-th step for some $k$. At $k+1$-th step, all rows after the $k+1$-th row is subtracted against the corresponding index in $k+1$-th row. The $k+1$-th row is given by the following: 
	\[\begin{pmatrix}0 & \cdots 0 & f(k+1)-f(k) & \cdots f(k+1)-f(k)\end{pmatrix}\]
	where the first $k$ entries are 0. Now after the $k+1$-th iteration, for all $i>k+1$, if $j\le k$ then $a_{ij}$ becomes $0-0=0$ and if $j>k$ we have $a_{ij}$ becomes $(f(\min(i, j))-f(k))-f(k+1)-f(k)=f(\min(i, j))-f(k+1)$. This entry is 0 if $j=k+1$ since $i>k+1$. For all $i\le k$ the rows are unaffected by this row reduction, so we still have $a_{ij}=0$ for all $j<i$, and $a_{ij}=f(i)-f(i-1)$ for all $j\ge i$. Thus the claim is proven. 
	
	To finish the proof, after $n-1$ iterations, we have, for all $j<i$, $a_{ij}=0$. Thus $A$ is no upper triangular, and the determinant is simply the product of the diagonal entries. We also have $a_{ii}=1$ for $i=1$ and $f(i)-f(i-1)$ for $i\ge 2$. Now $f(i)-f(i-1)=\frac{1}{i}-\frac{1}{i-1}=-\frac{1}{i(i-1)}$. Hence we have 
	\[\det(A)=\prod_{i=2}^n -\frac{1}{i(i-1)}= - (-1)^{n-1}\frac{1}{n[(n-1)!]^2}\]
	
	\item[\textbf{A3}] 
	Let $a_0=5/2$ and $a_k=a_{k-1}^2-2$ for $k\ge 1.$ Compute\[\prod_{k=0}^{\infty}\left(1-\frac1{a_k}\right)\]in closed form.
	
	\textbf{Answer.} $\dfrac 37$. 
	
	\textbf{Solution.} Consider, in general, $a_0=a+\dfrac 1a$ for some $a>1$, 
	then we show that the answer is $\dfrac{a^2-1}{a^2+a+1}$. 
	Here the problem is a special case with $a=2$. 
	
	\textbf{Claim.} $a_k= a^{2^k}+\dfrac{1}{a^{2^k}}$. \\
	Indeed, this holds for $a_0$, and using induction we have 
	\[
	\left(a^{2^{k-1}}+\dfrac{1}{a^{2^{k-1}}}\right)^2-2
	=a^{2^k}+\dfrac{1}{a^{2^k}}+2-2
	=a^{2^k}+\dfrac{1}{a^{2^k}}
	\]
	Thus our term now becomes 
	\[
	\prod_{k=0}^{\infty}\frac{a^{2^{k+1}}-a^{2^k}+1}{a^{2^{k+1}}+1}
	=\prod_{k=0}^{\infty}\frac{1-a^{-2^k}+a^{-2^{k+1}}}{1+a^{-2^{k+1}}}
	\]
	Let's evaluate the limits of numerator and denominator separately. 
	For the denominator we simply have 
	\[
	\prod_{k=0}^{n}(1+a^{-2^{k+1}})
	=(1+a^{-2})(1+a^{-4})\cdots (1+a^{-2^{n+1}})
	=\sum_{i=0}^{2^n-1}a^{-2i}
	\]
	(well the right equality is easy to verify). 
	Taking $n\to\infty$ we have 
	\[
	\sum_{i=0}^{2^n-1}a^{-2i}=(1-a^{-2})^{-1}
	\]
	For the numerator, we claim that 
	\[
	\prod_{k=0}^{n}(1-a^{-2^k}+a^{-2^{k+1}})
	=\left(\sum_{i=0}^{2^{n+1}-2} r_i(a^{-i}+a^{-(2^{n+2}-2-i)})\right)
	+(-1)^{n-1}a^{-(2^{n+1}-1)}
	\qquad 
	r_i=
	\begin{cases}
	  1 & 3\mid i\\
	  -1 & 3\mid i - 1\\
	  0 & 3\mid i - 2
	\end{cases}
	\]
	Notice also that $r_{2^{n+1}-1}=(-1)^{n-1}$. 
	Again we use induction. 
	For base case we just have $1-a^{-1}+a^{-2}$, and the RHS is just 
	$1(1+a^{-2})-a^{-1}=1-a^{-1}+a^{-2}$ (and therefore it matches). 
	For induction step, let's consider 
	\[
	\left(\left(\sum_{i=0}^{2^{n+1}-2} r_i(a^{-i}+a^{-(2^{n+2}-2-i)})\right)
	+(-1)^{n-1}a^{-(2^{n+1}-1)}\right)\left(1-a^{-2^{n+1}}+a^{-2^{n+2}}\right)
	\]
	Now we need to consider the following: 
	\begin{itemize}
		\item $0\le i < 2^{-(n+1)}$, then naturally we have the coefficient as $r_i$. 
		
		\item $2^{-(n+1)}\le i \le 2^{n+2}-1$, 
		then we have the coefficient as $r_{2^{n+2}-2-i}-r_{i-2^{n+1}}$. 
		We see that $r$ is a mod 3 function, so we can consider $2^{n+1}\equiv 1$ or $2^{n+2}\equiv 2$. 
		For the first case we have $r_{-i}+r_{i-1}$, 
		and for $i=0, 1, 2$ this gives 
		\[r_0-r_{-1}=1=r_0; \quad r_{-1}-r_{0}=-1=r_1;  \quad r_{-2}-r_{1}=0=r_2
		\]
		and for the second casd we have $r_{2-i}-r_{i-2}$, which gives 
		\[
		r_2-r_{-2}=1=r_0; \quad r_{1}-r_{-1}=-1=r_1; \quad r_{0}-r_0 = 0 = r_2
		\]
		which means that the coefficient is just going to be $r_i$. 
		
		\item The case $i\ge 2^{n+2}$ follows from that the coefficient is symmetric w.r.t. $2^{n+2}-1$ at $n+1$. 
		
	\end{itemize}
    Thus as $n\to\infty$, this should behave like 
    \[
    1-a^{-1}+a^{-3}-a^{-4}+\cdots 
    =(1-a^{-1})(1+a^{-3}+a^{-6}+\cdots)
    =(1-a^{-1})(1-a^{-3})^{-1}
    =\frac{1}{1+a^{-1}+a^{-2}}
    \]
    (Well the latter half of the terms, i.e. $i\ge 2^{n+1}-2$, don't quite follow this rule, but the effect is $\to 0$ since $a^{-2^{n+1}}+\cdots + a^{-2^{n+2}}\le a^{-2^{n+1}}(1-a^{-1})^{-1}\to 0$). 
    Thus the limit we're looking for is now 
    \[
    \frac{1-a^{-2}}{a+a^{-1}+a^{-2}}=\frac{a^2-1}{a^2+a+1}
    \]
    as desired. 
	
	\item[\textbf{A4}] Suppose $X$ is a random variable that takes on only nonnegative integer values, with $E[X]=1,$ $E[X^2]=2,$ and $E[X^3]=5.$ (Here $E[Y]$ denotes the expectation of the random variable $Y.$) Determine the smallest possible value of the probability of the event $X=0.$
	
	\textbf{Answer.} $\dfrac 13$. \\
	\textbf{Solution.} (by Kiran Kedlaya, modified) We let $a_i=P(X=i)$ for each $i\ge 0$ (which means $a_i\ge 0$ for each $i$). Consider the power series $f(x)=\displaystyle\sum_{n=0}^{\infty}a_nx^n$, and we have $f(1)=1$. 
	By the problem condition, we also have 
	\begin{itemize}
		\item $f'(x)=\displaystyle\sum_{n=0}^{\infty}na_nx^{n-1}$, so $f'(1)=\displaystyle\sum_{n=0}^{\infty}na_n=E(X)=1$
		
		\item $f''(x)=\displaystyle\sum_{n=0}^{\infty}n(n-1)a_nx^{n-2}$, so
		$f''(1)=\displaystyle\sum_{n=0}^{\infty}n(n-1)a_n
		=\displaystyle\sum_{n=0}^{\infty}n^2a_n-\displaystyle\sum_{n=0}^{\infty}na_n
		=E(X^2)-E(X)
		=2-1=1
		$
		
		\item $f'''(x)=\displaystyle\sum_{n=0}^{\infty}n(n-1)(n-2)a_nx^{n-3}$ so 
		$f'''(1)=\displaystyle\sum_{n=0}^{\infty}(n^3-3n^2+2n)a_n
		=E(X^3)-3E(X^2)+2E(X)
		=5-3(2)+2(1)=1
		$
	\end{itemize}  
	Now we can rearrange $f(x)$ into $f(x)=\displaystyle\sum_{n=0}^{\infty} f^{(n)}(1)\dfrac{(x-1)^n}{n!}$, i.e. the Taylor's series. We also have, by Taylor's series, $f(x)=f(1)+f'(1)(x-1)+f''(1)\dfrac{(x-1)^2}{2!}+f'''(1)\dfrac{(x-1)^3}{3!}+f^{(4)}(c)\dfrac{(x-1)^4}{4!}$, with $c$ some value in $(1, x)$ or $(x, 1)$ depending whether $x<1$ or $1<x$. Thus in particular $a_0=f(0)=f(1)-f'(1)+\dfrac{f''(1)}{2}-\dfrac{f'''(1)}{6}+\dfrac{f^{(4)}(c)}{24}
	=1-1+\dfrac{1}{2}-\dfrac{1}{6}+\dfrac{f^{(4)}(c)}{24}
	=\dfrac{1}{3}+\dfrac{f^{(4)}(c)}{24}
	$
	for some $c\in (0, 1)$. 
	We also note that $f^{(4)}(x)=\displaystyle\sum_{n=0}^{\infty}n(n-1)(n-2)(n-3)a_nx^{n-4}$ and for $x\ge 0$ and $n\ge 0$, the quantities $n(n-1)(n-2)(n-3)$, $a_n$, and $x^{n-4}$ are all nonnegative. Thus $f^{(4)}(x)\ge 0$ for all $x\ge 0$, and in particular $\dfrac{f^{(4)}(c)}{24}\ge 0$. Thus we have $f(0)=\dfrac{1}{3}+\dfrac{f^{(4)}(c)}{24}\ge \dfrac 13$, with equality holding when $a_0=\frac 13, a_1=\frac 12$ and $a_3=\frac 16$ and $a_n=0$ for other $n$'s. 
	
	\item[\textbf{B1}] 
	A base 10 over-expansion of a positive integer $N$ is an expression of the form $N=d_k10^k+d_{k-1}10^{k-1}+\cdots+d_0 10^0$ with $d_k\ne 0$ and $d_i\in\{0,1,2,\dots,10\}$ for all $i.$ For instance, the integer $N=10$ has two base 10 over-expansions: $10=10\cdot 10^0$ and the usual base 10 expansion $10=1\cdot 10^1+0\cdot 10^0.$ Which positive integers have a unique base 10 over-expansion?
	
	\textbf{Answer.} All positive integers without any 0 in their decimal expansion. 
	
	\textbf{Solution.} 
	We use the fact that every positive integer has a unique base-10 expansion (that is, all digits 0, $\cdots, 9$). 
	Therefore, a non-unique over expansion is equivalent to the existence of an over expansion with the `digit' 10 being used. 
	
	Consider the expansion $n=\sum_{i=0}^k d_i10^i$ with $0\le d_i\le 9$ and $d_k\neq 0$. 
	If $n$ has 0 as one of the digits, then 
	there exists a position $j>0$ such that $d_j>0$ but $d_{j-1}=0$. 
	Then we can replace $d_j$ with $d_{j}-1$ and $d_{j-1}$ with 10, giving two over-expansions here. 
	
	Next we show that any number $n$ with $d_i=10$ for some $i$ in its over-expansion must contain a 0 somewhere in its decimal expansion. 
	Indeed, let $j$ be the minimal index with $d_j=10$. 
	Then $n\equiv \dsum_{i=0}^{j}d_i10^i \equiv\dsum_{i=0}^{j-1} d_i10^i\pmod{10^{j+1}}$. 
	We see that $0\le \dsum_{i=0}^{j-1} d_i10^i < 10^j$ by the minimality of $j$, 
	and with $d_j=10, n\ge 10^{j+1}$. 
	Thus this implies that the digit at position $j$ is indeed 0. 
	
	\item[\textbf{B2}]Suppose that $f$ is a function on the interval $[1,3]$ such that $-1\le f(x)\le 1$ for all $x$ and $\displaystyle \int_1^3f(x)\,dx=0.$ How large can $\displaystyle\int_1^3\frac{f(x)}x\,dx$ be?
	
	\textbf{Answer.} $\ln\frac 43$. \\
	\textbf{Solution.} Equality can be attained by taking $f(j)=1$ for all $1\le j<2$ and $f(j)=-1$ for all $2\le j\le 3$. We show that this is the maximum by the following: if $g(x)$ is defined as $\displaystyle \int_1^xf(y)\,dy$, we have $g(1)=g(3)=0$. Also since $f(x)\in [-1, 1]$ for all $x\in [1, 3]$, and by Mean value theorem, we have , for every $x$ in the said interval, $g'(c)=f(c)=\frac{g(x)-g(1)}{x-1}$ for some constant $c$ in the interval $(1, x)$, so $|\frac{g(x)}{x-1}|\le 1$. Similarly $|\frac{g(x)}{x-3}|\le 1$. This means that $g(x)\le x-1$ and $g(x)\le 3-x$ must hold simultaneously. Using this fact and integrating by parts give: 
	\begin{flalign*}
	\int_1^3\frac{f(x)}{x}\,dx&=\frac{g(x)}{x}|_1^3 +\int_1^3\frac{g(x)}{x^2}\,dx\\
	&=(0-0)+\int_1^3\frac{g(x)}{x^2}\,dx\\
	&\le \int_1^2\frac{x-1}{x^2}\,dx+\int_2^3\frac{3-x}{x^2}\,dx\\
	&=[\ln x + \frac 1x]_1^2+[-\frac 3x-\ln x]_2^3\\
	&=\ln 2 -\ln 1+\frac 12 - 1 +\frac 32-1-\ln 3+\ln 2\\
	&=\ln \frac 43
	\end{flalign*}
	as desired. 
	
	\item[\textbf{B3}] Let $A$ be an $m\times n$ matrix with rational entries. Suppose that there are at least $m+n$ distinct prime numbers among the absolute values of the entries of $A.$ Show that the rank of $A$ is at least $2.$
	
	\textbf{Solution.} 
	By the theorem of unique prime factorization, if $p, q, r, s$ are prime numbers with $pq=rs$ then $p=r$, $q=s$ or $p=s, q=r$ (so the four numbers cannot be pairwise distinct). 
	The fact that there's at least one prime (and hence nonzero) number in $A$ implies that the rank of $A$ cannot be zero, so we can now assume that the rank of $A$ is 1, which is equivalent to assuming that there exists rational numbers $a_1, a_2, \cdots , a_m, b_1, b_2, \cdots , b_n$ such that $A_{ij}=a_ib_j$. 
	
	Now consider a graph $(V, E)$ with $n$ vertices. and we consider adding coloured edge by the following mechanism: for a row $i$, if $x_1<x_2<\cdots < x_k$ are the all the indices such that $A_{ix_j}$ are among the $m+n$ distinct prime numbers, then we add an edge coloured $i$ between $x_j$ and $x_{j+1}$ for each $1\le j\le k-1$. This means if row $i$ has $i_k$ prime numbers the there will be $i_k-1$ edges coloured $i$. Our colouring also ensures that there will be no monochromatic cycle in our graph, and there are at least $\displaystyle\sum_{k=1}^m (i_k-1)=(\displaystyle\sum_{k=1}^m i_k)-m=(m+n-m)=n$. 
	
	We first see what happens if there are two vertices $c_1, c_2$ with two edges coloured $k_1$ and $k_2$. This means $A_{k_ic_j}=a_{k_i}b_{c_j}$ are all prime numbers for all combinations of $i\in \{1, 2\}$ and $j\in \{1, 2\}$. Notice also that $A_{k_1c_1}A_{k_2c_2}=a_{k_1}b_{c_1}a_{k_2}b_{c_2}=a_{k_1}b_{c_2}a_{k_1}b_{c_2}=A_{k_1c_2}A_{k_2c_1}$, contradicting that the four prime numbers must be pairwise distinct. 
	
	Hence we know that there is at most an edge between two vertices, and since there are exactly $n$ vertices and at least $n$ edges, there exists a cycle comprising at least two different colours (since we have proven that there cannot be a monochromatic cycle above). Let $x_1, x_2, \cdots , x_k$ to be the cycle, with $x_ix_{i+1}$ connected by colour $r_i$ for each $1\le i\le k$. For each $i$, $A_{r_ix_i}$ and $A_{r_ix_{i+1}}$ are both primes, and let $p_{r_ix_i}, p_{r_ix_{i+1}}$ be the primes. Now $\frac{p_{r_ix_i}}{p_{r_ix_{i+1}}}=\frac{A_{x_ii}}{A_{x_i(i+1)}}=\frac{a_{x_i}b_i}{a_{x_i}b_{i+1}}=\frac{b_i}{b_{i+1}}$ (the fact that both entries are prime, i.e. nonzero, means that we don't have to worry about the validity of division). 
	Thus we have 
	\[1=\displaystyle\prod_{i=1}^k\frac{b_i}{b_{i+1}}=\displaystyle\prod_{i=1}^k \frac{p_{r_ix_i}}{p_{r_ix_{i+1}}}
	\]
	and by the theorem of unique prime factorization, $\displaystyle\prod_{i=1}^k p_{ii}$ and $\displaystyle\prod_{i=1}^k p_{(i+1)i}$ also implies that $\{p_{r_ix_i}: 1\le i\le k\}=\{p_{r_ix_{i+1}}\}: 1\le i\le k\}$. Since $p_{r_ix_i}$ corresponds to the entry $(r_i, x_i)$ and $p_{r_ix_{i+1}}$ the entry $(r_i, x_{i+1})$, and each $x_1, x_2, \cdots , x_k$ assumed to be distinct and each of the $m+n$ primes are distinct, we have $p_{r_ix_{i+1}}=p_{r_{i-1}x_i}$, $r_i=r_{i-1}$, so $r_1= r_2 = \cdots = r_k$. This also means that the only possibility is all edges of the cycle coloured the same colour $r_1$, contradiction. 
	
	\item[\textbf{B4}] Show that for each positive integer $n,$ all the roots of the polynomial \[\sum_{k=0}^n 2^{k(n-k)}x^k\] are real numbers.
	
	\textbf{Solution.} Let $f(x)$ be the polynomial, which obviously takes positive values when $x\ge 0$. Consider, now, all $x$'s with $x=-(2^m)$. Then 
	\[
	f(x) = \sum_{k=0}^n 2^{k(n-k)}x^k = \sum_{k=0}^n 2^{k(n-k)}(-2^m)^k
	=\sum_{k=0}^n (-1)^k 2^{k(n+m-k)}
	\]
	We first notice that when $k$ varies, $k(n+m-k)$ takes maximum value when $k=\frac{n+m}{2}$. For this reason, we focus on $m=-n, -n + 2, \cdots, n - 2, n$, whereby $k(n+m-k)=(\frac{n+m}{2})^2 - (\frac{n+m}{2}-k)^2$, and therefore $f(x)=f(-2^m) = 2^{(\frac{n+m}{2})^2}\displaystyle\sum_{k=0}^{n} (-1)^k 2^{- (\frac{n+m}{2}-k)^2}$. Our only interest is the sign of this term, and since the sign of $ 2^{(\frac{n+m}{2})^2}\displaystyle\sum_{k=0}^{n} (-1)^k 2^{- (\frac{n+m}{2}-k)^2}$ is the same as the sign of $\displaystyle\sum_{k=0}^{n} (-1)^k 2^{- (\frac{n+m}{2}-k)^2}$, we will focus on the latter. 
	
	We isolate the cases $n\le 2$ first. For $n=1$ all we have is $x+1$ so $x=-1$ is a solution, obviously. When $n=2$ we have $x^2+2x+1=(x+1)^2$, so $-1$ is a double root. Thus we only deal with $n=3$ here. We recall that if $a_1, a_2, \cdots , a_k$ are distinct nonnegative numbers then $\dsum_{i=1}^k 2^{-a_i} < \dsum_{i=1}^\infty 2^{-i} = 1$. Now we have the following cases to consider: 
	\begin{itemize}
		\item Case 1: $m=\pm n$. In the $+n$ case we have \[\displaystyle\sum_{k=0}^{n} (-1)^k 2^{- (\frac{n+m}{2}-k)^2} = \displaystyle\sum_{k=0}^{n} (-1)^k 2^{- (n-k)^2}=(-1)^n + (-1)^{n-1}2^{-1}
		+(-1)^{n-2}2^{-4}+\cdots + (-1)^0 2^{-n^2}
		\]
		and by the lemma we had, 
		$|(-1)^{n-1}2^{-1}
		+(-1)^{n-2}2^{-4}+\cdots + (-1)^0 2^{-n^2}|\le 2^{-1}+2^{-4}+\cdots + 2^{-n^2} < 1$
		so $(-1)^{n-1}2^{-1}
		+(-1)^{n-2}2^{-4}+\cdots + (-1)^0 2^{-n^2}\in (-1, 1)$ which means 
		$\displaystyle\sum_{k=0}^{n} (-1)^k 2^{- (n-k)^2}$ has the same sign as $(-1)^n$. 
		Similarly, when $m=-n$ the expression \[
		\displaystyle\sum_{k=0}^{n} (-1)^k 2^{- (\frac{n+m}{2}-k)^2}
		=\displaystyle\sum_{k=0}^{n} (-1)^k 2^{-k^2}
		\]
		has the same sign as $(-1)^0=1$ (i.e. positive). 
		
		\item Case 2: now $-n<m<n$ and $n$ has the same parity as $n$. 
		Then $\displaystyle\sum_{k=0}^{n} (-1)^k 2^{- (\frac{n+m}{2}-k)^2}$ has the following form: 
		\[
		(-1)^0 2^{-(\frac{n+m}{2})^2} + (-1)^1 2^{-(\frac{n+m}{2} - 1)^2} +\cdots + (-1)^{(n+m)/2}2^0 + \cdots + (-1)^n 2^{-(\frac{n+m}{2} - n)^2} 
		\]
		W.L.O.G. assume $m\le 0$; the other case is symmetric to this. We notice that $(\frac{n+m}{2} - k)^2 = (\frac{n+m}{2} - (n+m -k))^2$, and moreover $n+m$ is even so $k$ and $n+m-k$ has the same parity. This means we can group these terms together for $k=0, 1, \cdots , \frac{n+m}{2}-1$ to get 
		\begin{flalign*}
		\dsum_{i=0}^{\frac{n+m}{2}-1}((-1)^i  + (-1)^{n+m-i}) 2^{-(\frac{n+m}{2}-i)^2}
		+(-1)^{\frac{n+m}{2}}
		+ \dsum_{i=n+m+1}^{n}(-1)^{i}2^{-(\frac{n+m}{2}-i)^2}
		\\=(-1)^{\frac{n+m}{2}} + 2\dsum_{i=0}^{\frac{n+m}{2}-1}(-1)^i2^{-(\frac{n+m}{2}-i)^2}
		+ \dsum_{i=n+m+1}^{n}(-1)^{i}2^{-(\frac{n+m}{2}-i)^2}
		\\=(-1)^{\frac{n+m}{2}}+2(-1)^{\frac{n+m}{2} - 1} 2^{-1} + 2\dsum_{i=0}^{\frac{n+m}{2}-2}(-1)^i2^{-(\frac{n+m}{2}-i)^2}
		+ \dsum_{i=n+m+1}^{n}(-1)^{i}2^{-(\frac{n+m}{2}-i)^2}
		\\2\dsum_{i=0}^{\frac{n+m}{2}-2}(-1)^i2^{-(\frac{n+m}{2}-i)^2}
		+ \dsum_{i=n+m+1}^{n}(-1)^{i}2^{-(\frac{n+m}{2}-i)^2}
		\\\dsum_{i=0}^{\frac{n+m}{2}-2}(-1)^i2^{-(\frac{n+m}{2}-i)^2+1}
		+ \dsum_{i=n+m+1}^{n}(-1)^{i}2^{-(\frac{n+m}{2}-i)^2}
		\end{flalign*}
		(basically, the two terms beside $(-1)^{(n+m)/2}$ are $(-1)^{(n+m)/2-1}2^{-1}+(-1)^{(n+m)/2+1}2^{-1}$ and therefore vanishes). 
		We recognize that the exponents $-(\frac{n+m}{2}-i)^2+1$ with $i=0, \cdots , \frac{n+m}{2}-2$ are different numbers in the range $[-(\frac{n+m}{2})^2+1, -3]$ and $-(\frac{n+m}{2}-i)^2$ with $i=n+m+1, n$ are different numbers in the range $[-(\frac{m-n}{2})^2, -(-\frac{m+n}{2}-1)^2]$ and $-(-\frac{m+n}{2}-1)^2 < -(\frac{n+m}{2})^2+1$ are disjoint, which means together all these exponents represent different negative numbers. Therefore by the lemma above, the sign with follow the dominating one, i.e. $(-1)^{(n+m+2)/2}$, i.e. $(-1)^{(n+m)/2}$. This conclusion will hold for $m>0$ too. 
	\end{itemize}
	Summarizing above, we know that when $x=2^{m}$ for $m=-n, -n+2, \cdots , n-2, n$, $f(x)$ follows the sign of $(-1)^{(n+m)/2}$. In particular, $(-1)^{(n+m)/2}$ and $(-1)^{(n+m+2)/2}$ have different signs, so there is a root between $(-2^{m+2}, -2^m)$. 
	Considering $m=-n, \cdots , -n+2$ we know that there are roots in the intervals 
	$(-2^{n}, -2^{n-2}), (-2^{n-2}, \cdots , -2^{n-4}), \cdots , (-2^{-n+2}, -2^{-n})$ which are $n$ disjoint intervals, hence at least $n$ real roots. On the other hand, $f$ is a polynomail with degree $n$, hence only at $n$ roots in total. Thus all roots are real. 
	
\end{enumerate}

\end{document}