\documentclass[11pt,a4paper]{article}
\usepackage{amsmath, amssymb, fullpage, mathrsfs, bm, pgf, tikz}
\usepackage{mathrsfs}
\usetikzlibrary{arrows}
\setlength{\textheight}{10in}
%\setlength{\topmargin}{0in}
\setlength{\topmargin}{-0.5in}
\setlength{\parskip}{0.1in}
\setlength{\parindent}{0in}

\newcommand{\set}[2]{\{#1\,:\,\text{#2}\}}
\newcommand{\tup}[1]{\mathrm{#1}}
\newcommand{\sfP}{\mathsf{P}}
\newcommand{\M}{\mathsf{M}}
\newcommand{\bbR}{\mathbb R}
\newcommand{\bbC}{\mathbb C}
\newcommand{\bbZ}{\mathbb Z}
\newcommand{\bbN}{\mathbb N}
\newcommand{\bbQ}{\mathbb Q}
\newcommand{\bbF}{\mathbb F}
\newcommand{\bbP}{\mathbb P}
\newcommand{\bbE}{\mathbb E}
\newcommand{\dfeq}{\stackrel{\mathrm{def}}{=}}
\newcommand{\ra}{\rightarrow}
\newcommand{\la}{\leftarrow}
\newcommand{\lra}{\leftrightarrow}
\newcommand{\Span}{\mathrm{span}}
\newcommand{\scrP}{\mathscr{P}}
\newcommand{\rank}{\mathrm{rank}}
\newcommand{\nullity}{\mathrm{nullity}}
\newcommand{\Col}{\mathrm{Col}}
\newcommand{\Row}{\mathrm{Row}}
\newcommand{\tr}{\mathrm{tr}}
\newcommand{\ol}{\overline}
\newcommand{\norm}[1]{||#1||}
\newcommand{\doubleline}[1]{\underline{\underline{#1}}}
\newcommand{\elemop}[1]{\stackrel{#1}{\longrightarrow}}
\newcommand{\Ind}{\mathrm{Ind}}
\newcommand{\Res}{\mathrm{Res}}
\newcommand{\End}{\mathrm{End}}
\newcommand{\cl}{\mathrm{cl}}
\newcommand{\code}[1]{\texttt{#1}}
\newcommand\tab[1][0.5cm]{\hspace*{#1}}
\newcommand{\<}{\langle}
\renewcommand{\>}{\rangle}
\newcommand{\qubits}[1]{|{#1}\rangle}
\newcommand{\ord}{\mathrm{ord}}
\newcommand{\lcm}{\mathrm{lcm}}
\newcommand{\dsum}{\displaystyle\sum}
\newcommand{\dprod}{\displaystyle\prod}

\begin{document}

\newcommand{\sgn}{\text{sgn}}
\setcounter{secnumdepth}{0}

\section{Putnam 2020 (No official contest)}
\begin{enumerate}
	\item [\textbf{A1}] 
	How many positive integers $N$ satisfy all of the following three conditions?
	\begin{itemize}
		\item $N$ is divisible by $2020$.
		\item $N$ has at most $2020$ decimal digits.
		\item The decimal digits of $N$ are a string of consecutive ones followed by a string of consecutive zeros.
	\end{itemize}

    \textbf{Answer.} 508536. 
    
    \textbf{Solution.} 
    Let $N$ to have $a$ 1's and $b$ 0's, so $N=10^b\times \underbrace{1\cdots 1}_{a}$. 
    Since $N$ is divisible by 20, it must have at least 2 ending zeros. 
    Moreover, $2020=101\times 20$ and $\gcd(10, 101)=1$, so $101\mid \underbrace{1\cdots 1}_{a}=\frac{10^a-1}{9}$. 
    Since $1111=101\times 11$ but $101\nmid 111, 11, 1$, we see that $\ord_{101}(10)=4$ so 
    $4\mid a$. 
    This gives the complete characterization: 
    \[
    a+b=2020, b\ge 2, 4\mid a, a\ge 1
    \]
    Now for each $a$ we can pick $b=2, 3, \cdots, 2020-a$ (thus giving $2019-a$ choices). The maximum $a$ is 2016. 
    This gives the following: 
    \[
    \sum_{k=1}^{504} (2019-4k)
    =2019\times 504 - 4\times 504\times 505\div 2
    =1009\times 504
    =508536
    \]
    
    \item [\textbf{A2}] 
    Let $k$ be a nonnegative integer. Evaluate
    \[ \sum_{j=0}^k 2^{k-j} \binom{k+j}{j}. \]
    \textbf{Answer.} $4^k$. 
    
    \textbf{Solution.} Let our sum be $S_k$ and do induction on it. Base case is given by $k=0$ when we just have 1. 
    Now suppose that our claim holds for some $k$, that is $S_k=4^k$ for some $k\ge 0$. We have 
    \[
    S_{k+1}-2S_k = \sum_{j=0}^{k+1} 2^{k+1-j} \binom{k+1+j}{j}-2\sum_{j=0}^k 2^{k-j} \binom{k+j}{j}
    =\binom{2(k+1)}{k+1}+2\sum_{j=1}^k 2^{k-j}\dbinom{k+j}{j-1}
    \]
    where we used the fact that $\binom{a}{b}+\binom{a}{b+1}=\binom{a+1}{b+1}$. 
    Notice again that the equivalence
    \[
    2\sum_{j=1}^k 2^{k-j}\dbinom{k+j}{j-1}
    =2\sum_{j=0}^{k-1} 2^{k-j-1}\dbinom{k+1+j}{j}
    =\frac12(S_{k+1} - 2\binom{2k+1}{k}-\binom{2(k+1)}{k+1})
    \]
    which gives 
    \[
    \frac 12 S_{k+1} = 2S_k - \binom{2k+1}{k} + \frac 12 \binom{2(k+1)}{k+1}
    \]
    Given that $\binom{2k}{k}=2\binom{2k-1}{k-1}$, we also have $-\binom{2k+1}{k} + \frac 12 \binom{2(k+1)}{k+1}=0$ and therefore $S_{k+1}=4S_k=4\cdot 4^k=4^{k+1}$, as desired. 
    
    \item [\textbf{A3}] 
    Let $a_0=\pi /2$, and let $a_n=\sin (a_{n-1})$ for $n\ge 1$. Determine whether
    \[ \sum_{n=1}^{\infty}a_n^2 \]converges.
    
    \textbf{Answer.} This sequence diverges. 
    
    \textbf{Solution.} We first see that $a_n>0$ all the while: if $0<a_n\le \frac{\pi}{2}$ for some $n$ then $0 < a_{n+1}\le 1$, 
    so we have $0<a_n\le \frac{\pi}{2}$ for all $n$. 
    
    If $a_n\not\to 0$ then the series would diverge, so we may assume $a_n\to 0$. 
    We first analyze the asymptotic behaviour of $a_n^2$ as $a_n$ small: 
    \[
    \frac{1}{a_{n+1}^2}-\frac{1}{a_n^2}
    =\frac{(a_n-a_{n+1})(a_n+a_{n+1})}{a_n^2a_{n+1}^2}
    =\frac{(a_n-\sin a_n)(a_n+\sin a_n)}{a_n^2\sin^2 a_n}
    \stackrel{a_n\to 0}{\sim} \frac{\frac{a_n^3}{6}\cdot 2a_n}{a_n^4}
    =\frac 13
    \]
    so as $n\to\infty$, we see that $a_n^2$ decays in the pace of $\frac{3}{n+c}$ for some constant $c$. 
    Since $\sum \frac{1}{n}$ diverges, it follows that $\sum a_n^2$ diverges too. 
	
	\item [\textbf{A5}] 
	Let $a_n$ be the number of sets $S$ of positive integers for which
	\begin{equation}\label{eqn:2020a5}
	  \sum_{k\in S}F_k=n,
	\end{equation}
	where the Fibonacci sequence $(F_k)_{k\ge 1}$ satisfies $F_{k+2}=F_{k+1}+F_k$ and begins $F_1=1$, $F_2=1$, $F_3=2$, $F_4=3$. Find the largest number $n$ such that $a_n=2020$.
	
	\textbf{Answer.} $F_{4040}-1$. 
	
	\textbf{Solution.} 
	We first note the following: 
	\begin{equation}\label{eqn:2020a5sum}
	\sum_{i=1}^k F_i = F_{k+2}-1
	\end{equation}
	for all $k\ge 1$ (which is clear from $k=1,2,3$ and the rest can be established via induction). 
	If we define $f(n)$ as the index satisfying $F_{f(n)}\le n < F_{f(n)+1}$, 
	then the set $S$ satisfying Equation \ref{eqn:2020a5} has $\max \{k: k\in S\}$ either $f(n)$ or $f(n)-1$. 
	This gives us two scenario whenever $n\ge 2$ (and so $f(n)\ge 3$): 
	\begin{itemize}
		\item If the max element in $S$ is $f(n)$, we have $0\le n - F_{f(n)} < F_{f(n)+1} - F_{f(n)} = F_{f(n)-1}$, 
		which gives us $a_{n-F_{f(n)}}$ choices (all chosen in $\{1, \cdots, f(n)-2\}$)
		
		\item If the max element in $S$ is $f(n)-1$, then it turned out it's more expedient to consider the elements \emph{not} chosen among $\{1, \cdots, f(n)-2\}$. 
		By Equation \ref{eqn:2020a5sum}, 
		summing over $\{1, \cdots, f(n)-1\}$ gives $F_{f(n)+1}-1\ge n$, 
		so the number of ways here is precisely $a_{F_{f(n)+1}-n-1}$. 
		
	\end{itemize}
    Therefore we have the iterative formula $a_{n-F_{f(n)}}+a_{F_{f(n)+1}-n-1}$. 
    
    Now with this, let's make the following claim: 
    \begin{itemize}
    	\item For all $k\ge 1$, $a_{F_{2k}-1}=k$. 
    	
    	\item For all $k\ge 1$ and $n\ge F_{2k}$, $a_n>k$. In other words, $a_{n}\ge \lfloor \frac{f(n)}{2}\rfloor+1$. 
    \end{itemize}
    These claims would suffice to show that our answer above. 
    
    For the first claim, we see that $a_{F_2-1}=a_0=1$, and $a_{F_4-1}=a_2=2$ (given that 2 can be written 2 and 1+1). 
    Also, $f(F_{2k}-1)=2k-1$, 
    so for $k\ge 3$, 
    \[a_{F_{2k}-1}=a_{F_{2k}-1-F_{2k-1}}+a_{F_{2k}-F_{2k}}=a_{F_{2k-2}-1}+1
    \]
    so the inductive hypothesis $a_{F_{2k-2}-1}=k-1$ would imply $a_{F_{2k}-1}=k$. 
    
    Now to prove the second claim, let us again use induction in the following sense: 
    for each $k$, we consider those $n$ with $F_{2k}\le n < F_{2k+2}-1$. 
    When $k=0$ this is just $n=0$ and $a_n=1$, and when $k=1$, $n=1, 2$ which gives $a_1=a_2=2$. 
    
    For $k\ge 2$, for $n$ in the said range we have $f(n)$ either $2k$ or $2k+1$. 
    Now consider the pair of numbers 
    \[
    n-F_{f(n)}, F_{f(n)+1}-n-1
    \]
    which sums up to $F_{f(n)-1}-1< F_{2k}$. Thus by induction hypothesis we can deduce that 
    \[a_n=a_{n-F_{f(n)}}+a_{F_{f(n)+1}-n-1}\ge 
    \lfloor\frac{f(n-F_{f(n)})}{2}\rfloor + \lfloor\frac{f(F_{f(n)+1}-n-1)}{2}\rfloor + 2
    \]
    Now, let $\lfloor\frac{f(n-F_{f(n)})}{2}\rfloor=x$ and $\lfloor\frac{f(F_{f(n)+1}-n-1)}{2}\rfloor =y$, 
    then 
    \[
    f(n-F_{f(n)})\le 2x+1 \Rightarrow n-F_{f(n)}\le F_{2x+2}-1
    \]
    and similarly $F_{f(n)+1}-n-1\le F_{2y+2}-1$. 
    Therefore we have the sum satisfying
    $F_{f(n)-1}-1\le F_{2x+2}+F_{2y+2}-2$, or $F_{f(n)-1}\le F_{2x+2}+F_{2y+2}-1$. 
    If $x+y<k-1$, then we have $F_{2k-1}\le F_{2x+2}+F_{2y+2}-1$ for some $x+y<k-1$, and for $x, y\ge 0$. 
    (can substitute $f(n)=2k$ here since if it holds for $2k+1$ it will hold for $2k$). 
    We however see that $F$ is convex hence 
    \[F_{2k-1}\le F_{2x+2}+F_{2y+2}-1\le F_2 + F_{2(x+y)+2}-1 = F_{2(x+y)+2}\le F_{2k-2}
    \]
    which is a contradiction. 
    Therefore $a_n\ge k-1+2=k+1$, as claimed. 
	
	\item [\textbf{B3}] Let $x_0=1$, and let $\delta$ be some constant satisfying $0<\delta<1$. Iteratively, for $n=0,1,2,\dots$, a point $x_{n+1}$ is chosen uniformly form the interval $[0,x_n]$. Let $Z$ be the smallest value of $n$ for which $x_n<\delta$. Find the expected value of $Z$, as a function of $\delta$.
	
	\textbf{Answer.} $1+\frac{1}{\delta}$. 
	
	\textbf{Solution.} 
	Let's consider the function $F_n(\delta)=\bbP[x_n<\delta]$, and $f_n(\delta)=\frac{d}{d\delta}F_n$. 
	This has the following recursive formula: 
	\[
	F_1(\delta)=\delta 
	\qquad 
	F_n(\delta) = F_{n-1}(\delta) + \int_{\delta}^1 \frac{\delta}{x} f_{n-1}(x)dx
	\]
	Let's now show that 
	\begin{equation}\label{eqn:2020b3}
	  F_n(\delta)=\delta\left(\sum_{k=0}^{n-1}\frac{\ln (\frac{1}{\delta})^k}{k!}\right)
	\end{equation}
	Using induction, we see that this holds for $n=1$, and if this holds for some $n$, i.e. 
	\[
	F_n(x)=x\left(\sum_{k=0}^{n-1}\frac{\ln (\frac{1}{x})^k}{k!}\right)
	\qquad 
	F_n(x) = \sum_{k=0}^{n-1}\frac{\ln (\frac{1}{x})^k}{k!} - \sum_{k=1}^{n-1}\frac{\ln (\frac{1}{x})^{k-1}}{(k-1)!}
	=\frac{\ln (\frac{1}{x})^{n-1}}{(n-1)!}
	\]
	and therefore 
	\[
	F_{n+1}(\delta)=
	\delta\left(\sum_{k=0}^{n-1}\frac{\ln (\frac{1}{\delta})^k}{k!}\right)
	+\int_{\delta}^1 \frac{\delta}{x} \frac{\ln (\frac{1}{x})^{n-1}}{(n-1)!}dx
	=\delta\left(\sum_{k=0}^{n}\frac{\ln (\frac{1}{\delta})^k}{k!}\right)
	\]
	establishing \ref{eqn:2020b3}. 
	
	To finish the solution, we have 
	$\bbP[Z=k]=\bbP[x_n<\delta \land x_{n-1}\ge \delta]=F_k(\delta)-F_{k-1}(\delta)=\delta\frac{\ln(\frac{1}{\delta})^{k-1}}{(k-1)!}$. 
	Therefore, 
	\begin{flalign*}
	  \bbE(Z)
	  &=\sum_{k=1}^{\infty}k\delta\frac{\ln(\frac{1}{\delta})^{k-1}}{(k-1)!}
	  \\&=\sum_{k=1}^{\infty}\delta\frac{\ln(\frac{1}{\delta})^{k-1}}{(k-1)!} + (k-1)\delta\frac{\ln(\frac{1}{\delta})^{k-1}}{(k-1)!}
	  \\&=\sum_{k=1}^{\infty}\delta\frac{\ln(\frac{1}{\delta})^{k-1}}{(k-1)!} + \sum_{k=2}^{\infty}\delta\frac{\ln(\frac{1}{\delta})^{k-2}}{(k-2)!}
	  \\&=\delta e^{\ln(\frac{1}{\delta})} + \delta\ln(\frac{1}{\delta})e^{\ln(\frac{1}{\delta})}
	  \\&=1+\ln\frac{1}{\delta}
	\end{flalign*}
	as claimed. 
	
	\item [\textbf{B4}] 
	Let $n$ be a positive integer, and let $V_n$ be the set of integer $(2n+1)$-tuples $\mathbf{v}=(s_0,s_1,\cdots,s_{2n-1},s_{2n})$ for which $s_0=s_{2n}=0$ and $|s_j-s_{j-1}|=1$ for $j=1,2,\cdots,2n$. Define
	\[
	q(\mathbf{v})=1+\sum_{j=1}^{2n-1}3^{s_j},
	\]and let $M(n)$ be the average of $\frac{1}{q(\mathbf{v})}$ over all $\mathbf{v}\in V_n$. Evaluate $M(2020)$.
	
	\textbf{Answer.} $\frac{1}{4040}$. 
	
	\textbf{Solution.} In general, we show that the average of $1+\sum_{j=1}^{2n-1}\alpha^{s_j}$ over $V_n$ is $\frac{1}{2n}$ for any $\alpha > 0$. 
	
	Consider $\mathbf{v}=(s_0,s_1,\cdots,s_{2n-1},s_{2n})$ and 
	$t(\mathbf{v}) := (t_0, \cdots, t_{2n-1})$ be such that $t_k = s_{k+1}-s_k$. 
	We say $\mathbf{v}\sim\mathbf{v'}$ if $t(\mathbf{v}')$ can be obtained from $t(\mathbf{v})$ via cyclic shift (hence we could also say $t(\mathbf{v})\sim t(\mathbf{v})'$). 
	Now $t$ maps $V_n$ to $T_n:=\{(t_0, \cdots, t_{n-1})\subseteq \{-1, 1\}^{2n}, \sum t_i=0\}$, 
	and is a bijection. 
	Moreover, relation defined via cyclic shift is both symmetric (just reverse cycle) and transitive, and also $\mathbf{v}\sim\mathbf{v}$ holds for all $\mathbf{v}$. 
	Thus $\sim$ is an equivalence relation. 
	
	Denote, now, the equivalence class of each $\mathbf{v}$: 
	\[
	E_{\mathbf{v}} = \{\mathbf{v}': \mathbf{v'}\sim \mathbf{v}\}
	\]
	We'll show that the average of $\frac{1}{q}$ in $E_{\mathbf{v}}$ is $\frac{1}{2n}$. 
	Let $t(\mathbf{v})=(t_0,t_1,\cdots,t_{2n-1})$ and for each $\mathbf{v}'\sim\mathbf{v}$ can be written as 
	$t(\mathbf{v})'=(t_j,t_{j+1},\cdots,t_{2n+j-1})$ for some $j\ge 0$ (indices taken modulo $2n$). 
	Thus if $\mathbf{v}=(0, s_1, \cdots, s_{n-1}, 0)$ we have 
	$\mathbf{v}'=(0, s_{j+1}-s_j, \cdots, s_{2n}-s_j, s_1-s_j, \cdots, s_{j-1}-s_j, 0)$, 
	and $\frac{1}{q(\mathbf{v}')}=\frac{\alpha^{s_j}}{q(\mathbf{v})}$. 
	Now considering $j=0, \cdots, 2n-1$ we see that the average of $\frac{1}{q}$ is now 
	\[
	\frac{1}{2n}\sum_{j=0}^{2n-1}\frac{\alpha^{s_j}}{q(\mathbf{v})} = \frac{1}{2n}
	\]
	since $\frac{\alpha^{s_j}}{q(\mathbf{v})}$ is just 1. 
	But we're not done yet -- we need to show that averaging over all $j=0, \cdots, 2n-1$ is the \emph{true} average of this equivalence class. 
	This is to show that as we loop over $j=0, \cdots, 2n-1$ each $\mathbf{v}'\in E_{\mathbf{v}}$ shows up equally many times. 
	If we extend $t_0, \cdots, t_{2n-1}$ infinitely (hence having $2n$ as period) and let $g$ as its \emph{minimal} period, we see that each $\mathbf{v}'\in E_{\mathbf{v}}$ shows up $\frac{2n}{g}$ times, 
	proving our claim. 
	
\end{enumerate}

\end{document}