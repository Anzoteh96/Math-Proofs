\documentclass[11pt,a4paper]{article}
\usepackage{amsmath, amssymb, fullpage, mathrsfs, bm, pgf, tikz}
\usepackage{mathrsfs}
\usetikzlibrary{arrows}
\setlength{\textheight}{10in}
%\setlength{\topmargin}{0in}
\setlength{\topmargin}{-0.5in}
\setlength{\parskip}{0.1in}
\setlength{\parindent}{0in}

\newcommand{\set}[2]{\{#1\,:\,\text{#2}\}}
\newcommand{\tup}[1]{\mathrm{#1}}
\newcommand{\sfP}{\mathsf{P}}
\newcommand{\M}{\mathsf{M}}
\newcommand{\bbR}{\mathbb R}
\newcommand{\bbC}{\mathbb C}
\newcommand{\bbZ}{\mathbb Z}
\newcommand{\bbN}{\mathbb N}
\newcommand{\bbQ}{\mathbb Q}
\newcommand{\bbF}{\mathbb F}
\newcommand{\bbP}{\mathbb P}
\newcommand{\bbE}{\mathbb E}
\newcommand{\dfeq}{\stackrel{\mathrm{def}}{=}}
\newcommand{\ra}{\rightarrow}
\newcommand{\la}{\leftarrow}
\newcommand{\lra}{\leftrightarrow}
\newcommand{\Span}{\mathrm{span}}
\newcommand{\scrP}{\mathscr{P}}
\newcommand{\rank}{\mathrm{rank}}
\newcommand{\nullity}{\mathrm{nullity}}
\newcommand{\Col}{\mathrm{Col}}
\newcommand{\Row}{\mathrm{Row}}
\newcommand{\tr}{\mathrm{tr}}
\newcommand{\ol}{\overline}
\newcommand{\norm}[1]{||#1||}
\newcommand{\doubleline}[1]{\underline{\underline{#1}}}
\newcommand{\elemop}[1]{\stackrel{#1}{\longrightarrow}}
\newcommand{\Ind}{\mathrm{Ind}}
\newcommand{\Res}{\mathrm{Res}}
\newcommand{\End}{\mathrm{End}}
\newcommand{\cl}{\mathrm{cl}}
\newcommand{\code}[1]{\texttt{#1}}
\newcommand\tab[1][0.5cm]{\hspace*{#1}}
\newcommand{\<}{\langle}
\renewcommand{\>}{\rangle}
\newcommand{\qubits}[1]{|{#1}\rangle}
\newcommand{\ord}{\mathrm{ord}}
\newcommand{\lcm}{\mathrm{lcm}}
\newcommand{\dsum}{\displaystyle\sum}
\newcommand{\dprod}{\displaystyle\prod}

\begin{document}

\newcommand{\sgn}{\text{sgn}}
\setcounter{secnumdepth}{0}

\section{Putnam 2019}
\begin{enumerate}
	\item [\textbf{A1}] Determine all possible values of $A^3+B^3+C^3-3ABC$ where $A$, $B$, and $C$ are nonnegative integers.
	
	\textbf{Answer.} Any integer that's nonnegative and have remainders 0, 1, 2, 4, 5, 7, 8 modulo 9. 
	
	\textbf{Solution.} We first have $A^3+B^3+C^3-3ABC=\frac 12 (A+B+C)((A-B)^2+(B-C)^2+(A-C)^2)$, so the fact that $A, B, C$ are all nonnegative means that $A^3+B^3+C^3-3ABC$ is also nonnegative. Next, consider the numbers $k$ with $A=B=k$ and $C=k+1$ for $k\ge 0$, we have: 
	\[
	A^3+B^3+C^3-3ABC=\frac 12 (A+B+C)((A-B)^2+(B-C)^2+(A-C)^2)
	=\frac 12(3k+1)(0+1+1)=3k+1
	\]
	so all numbers in terms of $3k+1$ can be expressed in the terms above (i.e. 1, 4, 7 modulo 9). Meanwhile, $A=k$ and $B=C=k+1$ gives us 
	\[
	\frac 12 (A+B+C)((A-B)^2+(B-C)^2+(A-C)^2)
	=\frac 12 (3k+2)(1+1+0)=3k+2
	\]
	which covers all nonnegative numbers congruent to 2, 5, 8 modulo 9. 
	and finally, setting $A=k, B=k+1, C=k+2$ gives 
	\[
	\frac 12 (3k+3)(1^2+1^2+4^2)=3(3k+3)=9(k+1)
	\]
	which gives us all multiples of 9 that's at least 9. The number 0 can be achieved by setting $A=B=C=0$, therefore giving us the representation of all nonnegative integers with remainders 0, 1, 2, 4, 5, 7, 8 modulo 9. 
	
	To show that numbers with remainders 3 and 6 modulo 9 cannot be represented in the form we desire, it suffices to show that if $A^3+B^3+C^3-3ABC$ is divisible by 3, then it's divisible by 9. We first notice that $x^3\equiv x\pmod{3}$ (since $x^3-x=x(x-1)(x+1)$ and one of $x, x-1, x+1$ is divisible by 3). Therefore we need $A+B+C$ to be divisible by 3. This gives us one of the following two scenarios: 
	\[
	A\equiv B\equiv C\qquad \{A, B, C\}=\{0, 1, 2\}\pmod{3}
	\]
	in the first case, we have $(A-B)^2+(B-C)^2+(A-C)^2\equiv 0\pmod{3}$; in the second case, we have $(A-B)^2+(B-C)^2+(A-C)^2\equiv 1+1+1\equiv 0\pmod{3}$. This means that $(A-B)^2+(B-C)^2+(A-C)^2$ is divisible by 3 and so is $A+B+C$. Therefore the product $\frac 12 (A+B+C)((A-B)^2+(B-C)^2+(A-C)^2)$ will be divisible by 9. 
	
	\item [\textbf{A2}] In the triangle $\triangle ABC$, let $G$ be the centroid, and let $I$ be the center of the inscribed circle.  Let $\alpha$ and $\beta$ be the angles at the vertices $A$ and $B$, respectively.  Suppose that the segment $IG$ is parallel to $AB$ and that $\beta = 2\tan^{-1}(1/3)$.  Find $\alpha$.
	
	\textbf{Answer.} $90^{\circ}=\frac{\pi}{2}$. 
	
	\textbf{Solution.} Let $CI$ intersect $AB$ at $M$, and the circumcircle of $ABC$ at $D\neq C$. 
	From $IG\parallel AB$ we have $CI:IM=2:1$, and using the well-known fact $DA=DI=DB$ and 
	$DM\cdot DC=DI^2$, we have $IM=MD$. 
	
	Now, $\angle ABC=\angle ADC=2\tan^{-1}(1/3)$ given that $A, B, C, D$ are concyclic. 
	Let $N$ be the midpoint of $AI$ and $P$ be the midpoint of $IM$, then $NP\parallel AM$. 
	Moreover, $IN/ND=\tan\angle IND=\frac{1}{3}$ since $DN$ bisects $\angle IDA$ (well, $N$ is midpoint of $IA$ and $DI=DA$). 
	But since $M$ is midpoint of $DI$ and $P$ is midpoint of $IM$, we also have $IP:PD=1:3$. 
	Therefore $\frac{IP}{PD}=\frac{IN}{ND}$ and we have by angle bisector theorem, 
	$NP$ bisects $|angle IND$ (which is in fact a $90^{\circ}$), 
	so $\angle INP=\angle IAM=45^{\circ}$ (the first equality is because $NP\parallel AM$). 
	Thus $\alpha=90^{\circ}$. 
	
	\item [\textbf{A3}] Given real numbers $b_0,b_1,\ldots, b_{2019}$ with $b_{2019}\neq 0$, let $z_1,z_2,\ldots, z_{2019}$ be the roots in the complex plane of the polynomial
	\[
	P(z) = \sum_{k=0}^{2019}b_kz^k.
	\]Let $\mu = (|z_1|+ \cdots + |z_{2019}|)/2019$ be the average of the distances from $z_1,z_2,\ldots, z_{2019}$ to the origin.  Determine the largest constant $M$ such that $\mu\geq M$ for all choices of $b_0,b_1,\ldots, b_{2019}$ that satisfy
	\[
	1\leq b_0 < b_1 < b_2 < \cdots < b_{2019} \leq 2019.
	\]
	
	\textbf{Answer.} $M=2019^{-\frac{1}{2019}}$. 
	
	\textbf{Solution.} We know that the product of roots is $-\frac{b_0}{b_{2019}}$ here. By AM-GM inequality, 
	\[
	\mu\ge \sqrt[2019]{(|z_1|\cdot\cdots\cdot |z_{2019}|)}=\sqrt[2019]{\frac{b_0}{b_{2019}}}\ge \sqrt[2019]{\frac{1}{2019}}
	\]
	To show that equality can hold, consider $b_k=2019^{\frac{k}{2019}}$, which satisfies $1\le b_0<b_1<\cdots<b_{2019}\le 2019$. 
	Consider $\epsilon$ as one of the 2020-th root of unity that is not 1 (that is, $\epsilon^{2020}=1$). Then we have 
	\[
	P(2019^{-\frac{1}{2019}}\epsilon)=\sum_{k=0}^{2019}2019^{\frac{k}{2019}}2019^{-\frac{k}{2019}}\epsilon^k = \sum_{k=0}^{2019}\epsilon^k = \frac{\epsilon^{2020}-1}{\epsilon-1}=0
	\]
	and therefore the 2019 roots are indeed all the 2020th root of unity that isn't 1 multiplied by $2019^{-\frac{1}{2019}}$, hence $\mu=2019^{-\frac{1}{2019}}$. 
	
	\item [\textbf{A5}] Let $p$ be an odd prime number, and let $\mathbb{F}_p$ denote the field of integers modulo $p$. Let $\mathbb{F}_p[x]$ be the ring of polynomials over $\mathbb{F}_p$, and let $q(x) \in \mathbb{F}_p[x]$ be given by $q(x) = \sum_{k=1}^{p-1} a_k x^k$ where $a_k = k^{(p-1)/2}$ mod $p$. Find the greatest nonnegative integer $n$ such that $(x-1)^n$ divides $q(x)$ in $\mathbb{F}_p[x]$.
	
	\textbf{Answer.} $\frac{p-1}{2}$. 
	
	\textbf{Solution.} We first claim that for each $0<n<p$, if $n$ is the highest $n$ such that $(x-1)^n$ divides $P(x)$, then $n-1$ is the highest power such that $(x-1)^{n-1}$ divides $P'(x)$ (we do differentiation the same manner as how we do it in $\bbR[x]$). To see this, let $P(x)=(x-1)^nQ(x)$ where $Q$ is not divisible by $x-1$. Then 
	\[
	P'(x)=(x-1)^nQ'(x)+n(x-1)^{n-1}Q(x)=(x-1)^{n-1}((x-1)Q'(x)+nQ(x))
	\]
	and since $0<n<p$, we have $P'(x)$ divisible by $(x-1)^{n-1}$ but $(x-1)Q'(x)+nQ(x)\equiv nQ(x)\pmod{(x-1)}$, and therefore $n-1$ is the highest power of $x-1$ dividing $P'(x)$, as claimed. 
	This means that, if $n$ is the highest power of $x-1$ dividing $P$ and $n<p$, then $P(x), P'(x), \cdots, P^{(n-1)}(x)$ are divisible by $x-1$ but not $P^{(n)}(x)$ (where $P^{(n)}(x)$ denotes the $n$-th derivative). 
	
	Now, consider our polynomial $q$ and the derivatives. For each $n<p$, the $n$-th derivative is 
	\[
	q^{(n)}(x)=\dsum_{k=1}^{p-1}a_kk(k-1)\cdots(k-n+1)x^k
	=\dsum_{k=1}^{p-1}k^{(p-1)/2}k(k-1)\cdots(k-n+1)x^k
	\]
	Notice that $x-1$ divides $q^{(n)}(x)$ iff $q^{(n)}(1)=0$. Therefore we're interested in the value of the sum 
	\[
	\dsum_{k=1}^{p-1}k^{(p-1)/2}k(k-1)\cdots(k-n+1)
	\]
	when evaluated in $\bbF_p$. 
	
	Denote, now, $f(x)=x^{(p-1)/2}x(x-1)\cdots(x-n+1)$, which is a degree $(p-1)/2+n$ polynomial. This means it can be written in the form 
	\[
	\dsum_{k=(p-1)/2}^{(p-1)/2+n} b_kx^k
	\]
	Then we're looking at the term 
	\[
	f(1)+f(2)+\cdots + f(p-1)= \dsum_{k=(p-1)/2}^{(p-1)/2+n} b_k(1^k+\cdots + (p-1)^k)
	\]
	If $g$ is a primitive root of $p$, then provided $p-1$ does not divide $k$, 
	\[
	1^k+\cdots + (p-1)^k
	=g^0+g^k+\cdots + g^{(p-2)k}
	=\frac{g^{(p-1)k-1}}{g^k-1}
	=0
	\]
	so if $(p-1)/2+n<p-1$, $\dsum_{k=(p-1)/2}^{(p-1)/2+n} b_k(1^k+\cdots + (p-1)^k)=0$. 
	This would mean that $q{(n)}(x)$ is divisible by $(x-1)$ for all $n=0, 1, \cdots, \frac{(p-3)}{2}$. 
	
	When $n=\frac{p-1}{2}$, we have the leading term, $b_{(p-1)/2}$ as 1, so in this case $q^{(n)}(1)\equiv 1\pmod{p}$. We thus conclude that the highest power of $n$ with $q^{(n)}(x)$ divisible by $(x-1)$ is $\frac{p-3}{2}$, and therefore the highest power we're looking for is $\frac{p-1}{2}$. 
	
	\item [\textbf{B1}] Denote by $\mathbb Z^2$ the set of all points $(x,y)$ in the plane with integer coordinates.  For each integer $n\geq 0$, let $P_n$ be the subset of $\mathbb Z^2$ consisting of the point $(0,0)$ together with all points $(x,y)$ such that $x^2+y^2=2^k$ for some integer $k\leq n$.  Determine, as a function of $n$, the number of four-point subsets of $P_n$ whose elements are the vertices of a square.
	
	\textbf{Answer.} 
	
	\textbf{Solution.} We first claim that all pairs $(x, y)$ with $x^2+y^2=2^k$ are of the following: 
	\[
	\begin{cases}
	(\pm 2^{\frac{k-1}{2}}, \pm 2^{\frac{k-1}{2}})\qquad k\text{ odd}\\
	(0, \pm 2^{\frac k2}), (\pm 2^{\frac k2}, 0)\qquad k\text{ even}\\
	\end{cases}
	\]
	To see why, let $\ell$ be the highest power of 2 dividing both $x$ and $y$ (which exists so long as $x$ and $y$ are not both zero). 
	This means, $x=2^{\ell}x_0$ and $y=2^{\ell}y_0$, where at least one of $x_0$ and $y_0$ is odd. 
	Given that $x^2+y^2=2^{2\ell}(x_0^2+y_0^2)$, we need $x_0^2+y_0^2$ to be power of 2. 
	If one of $x_0$ is odd and the other is even, then $x_0^2+y_0^2$ is odd and the only possibility here will be $x_0^2+y_0^2=1$, which menas $(x_0, y_0)=(\pm 1, 0)$ or $(0, \pm 1)$. 
	Otherwise, both are odd and both $x_0^2, y_0^2\equiv 1\pmod{4}$. 
	Therefore, $x_0^2+y_0^2\equiv 2\pmod{4}$. This means that $x_0^2+y_0^2=2$ is the only possibility, and therefore $x_0, y_0=(\pm 1, \pm 1)$. 
	
	Now to first show that we have $5n+1$ such constructions, we first notice that when $n=0$, the only possibility is $(1, 0), (0, 1), (-1, 0), (0, -1)$, and that the 5 new constructions consisting at least one point in $P_n\backslash P_{n-1}$ for all $n>0$ are the following: 
	\begin{itemize}
		\item For $n$ even, we can have $(2^{n/2}, 0), (2^{n/2-1}, 2^{n/2-1}), (0, 0), (2^{n/2-1}, -2^{n/2-1})$ as one square, and have this square rotate by $90^{\circ}, 180^{\circ}, 270^{\circ}$ around the origin. Finally, we have one big square at $(2^{n/2}, 0), (0, 2^{n/2}), (-2^{n/2}, 0), (0, -2^{n/2})$. 
		
		\item The case where $n$ odd is similar. We have $(2^{(n-1)/2}, 2^{(n-1)/2}), (2^{(n-1)/2}, 0), (0, 0), (0, 2^{(n-1)/2})$, and again have it rotate by $90^{\circ}, 180^{\circ}, 270^{\circ}$ around the origin. Finally, we have one big square at $(2^{(n-1)/2}, 2^{(n-1)/2}), (2^{(n-1)/2}, -2^{(n-1)/2}), (-2^{(n-1)/2}, 2^{(n-1)/2}), (-2^{(n-1)/2}, -2^{(n-1)/2})$. 
	\end{itemize}
	We're therefore left to show that these are the only solutions. We first notice that a square is a parallelogram, which means that the two opposite vertices have the same coordinate-wise sum as the other two opposite vertices. Now, let $k$ be the minimum power of 2 that divides all the $x$-coordinates of the square, i.e. one of them is $\pm 2^k$. By considering the sum of opposite vertices, we must have other one vertex that is $\pm 2^k$ (either the side, or diagonal). This means that this square must have 2 points from the following 6: $(\pm 2^k, \pm 2^k)$ (4 points here), and $(\pm 2^k, 0)$. 
	
	
	
	\item [\textbf{B2}] For all $n\ge 1$, let $a_n=\sum_{k=1}^{n-1}\frac{\sin(\frac{(2k-1)\pi}{2n})}{\cos^2(\frac{(k-1)\pi}{2n})\cos^2(\frac{k\pi}{2n})}$. Determine $\lim_{n\rightarrow \infty}\frac{a_n}{n^3}$.
	
	\textbf{Answer.} $\frac{8}{\pi^3}$
	
	\textbf{Solution.} We first notice the following: 
	\[
	\cos^2 A - \cos^2B = (\cos A-\cos B)(\cos A+\cos B)=(-2\sin\frac{A+B}{2}\sin\frac{A-B}{2})(2\cos\frac{A+B}{2}\cos\frac{A-B}{2})
	\]
	\[
	=\sin(A+B)\sin(B-A)
	\]
	and therefore 
	\[
	\frac{1}{\cos^2(\frac{k\pi}{2n})}-\frac{1}{\cos^2(\frac{(k-1)\pi}{2n})}
	=\frac{\cos^2(\frac{(k-1)\pi}{2n})-\cos^2(\frac{k\pi}{2n})}{\cos^2(\frac{(k-1)\pi}{2n})\cos^2(\frac{k\pi}{2n})}
	=\frac{\sin(\frac{(2k-1)\pi}{2n})\sin(\frac{\pi}{2n})}{\cos^2(\frac{(k-1)\pi}{2n})\cos^2(\frac{k\pi}{2n})}
	\]
	which means 
	\[
	\sum_{k=1}^{n-1}\frac{\sin(\frac{(2k-1)\pi}{2n})}{\cos^2(\frac{(k-1)\pi}{2n})\cos^2(\frac{k\pi}{2n})}
	=\sum_{k=1}^{n-1}\frac{1}{\sin\frac{\pi}{2n}}(\frac{1}{\cos^2(\frac{k\pi}{2n})}-\frac{1}{\cos^2(\frac{(k-1)\pi}{2n})})
	=\frac{1}{\sin\frac{\pi}{2n}}(\frac{1}{\cos^2(\frac{(2n-1)\pi}{2n})} - \frac{1}{\cos^2(\frac{\pi}{2n})})
	\]
	when $n\to\infty$, $\sin(\frac{\pi}{2n})\to \frac{\pi}{2n}$, $\frac{1}{\cos^2(\frac{\pi}{2n})}\to 1$ and $\frac{1}{\cos^2(\frac{(2n-1)\pi}{2n})}\to \frac{1}{\sin^2(\frac{\pi}{2n})}\to\frac{1}{(\frac{\pi}{2n})^2}$. Therefore, 
	\[
	\frac{1}{\sin\frac{\pi}{2n}}(\frac{1}{\cos^2(\frac{(2n-1)\pi}{2n})} - \frac{1}{\cos^2(\frac{\pi}{2n})})
	\to \frac{2n}{\pi}((\frac{2n}{\pi})^2-1)
	\]
	and so 
	\[
	\frac{a_n}{n^3}\to \frac{2}{\pi}((\frac{2}{\pi})^2-\frac{1}{n^2})=\frac{8}{\pi^3}
	\]
	
	\item [\textbf{B5}] Let $F_m$ be the $m$'th Fibonacci number, defined by $F_1=F_2=1$ and $F_m = F_{m-1}+F_{m-2}$ for all $m \geq 3$. Let $p(x)$ be the polynomial of degree 1008 such that $p(2n+1)=F_{2n+1}$ for $n=0,1,2,\ldots,1008$. Find integers $j$ and $k$ such that $p(2019) = F_j - F_k$.
	
	\textbf{Answer.} $j=2019, k=1010$. 
	
	\textbf{Solution.}
	Start with $p_0=p$, and for each $i\ge 1$, $p_{i+1}(x)=p_{i}(x+2)-p_{i}(x)$. 
	We notice the following: 
	\begin{itemize}
		\item By considering the expansion $(x+2)^k-x^k = \sum_{i=0}^{k-1} \binom{k}{i}2^{k-i}x_i$, we have $\deg(p_{i+1})=\deg(p_i)-1$ whenever $\deg(p_i)\ge 1$. 
		
		\item For each $k\le 1008$, we can inductively show that for all $n=0, 1, \cdots, 1008-k$, 
		$p_k(2n+1)=F_{2n+1+k}$. 
		Indeed, this is true for $k=0$. 
		If this is true for some $k\ge 0$ (but $k\le 1007$) 
		then for $n=0, \cdots, 1007-k$, we have 
		\[
		p_{k+1}(2n+1)=p_{k+1}(2n+3)-p_{k}(2n+1)
		=F_{2n+3+k}-F_{2n+1+k}
		=F_{2n+2+k}
		\]
		as desired. 
	\end{itemize}
	So from above, we have $p_{1008}(1)=F_{1009}$, but $\deg(p_{1008})=0$ so $p_{1008}$ is a constant (i.e. $F_{1009}$). 
	
	Next, we'll recover $p(2019)$ by the following: we see that $p_{1008}(3)=F_{1009}$, and moreover for all $k\le 1007$, 
	\[
	p_k(2(1008-k)+3) = p_{k+1}(2(1008-k)+1)+p_k(2(1008-k)+1)
	\]\[
	=p_{k+1}(2(1008-k)+1)+F_{k+2(1008-k)+1}
	=p_{k+1}(2(1008-k)+1)+F_{2016-k+1}
	\]
	So we can do telescoping sum to get 
	\[
	p(2019)-p_{1008}(3)
	=\sum_{k=0}^{1007} p_k(2(1008-k)+3)-p_{k+1}(2(1008-k)+1)
	=F_{2016-k+1}
	\]
	That is, 
	\[
	p(2019)=F_{1009}+F_{1010}+\cdots + F_{2017}
	=F_{2017}+\sum_{k=505}^{1008}F_{2k-1}+F_{2k}
	=F_{2017}+\sum_{k=505}^{1008}F_{2k+1}
	\]\[
	=F_{2017}+(F_{1011}+F_{1013}+\cdots + F_{2017})
	\]
	But then $F_{1011}+F_{1013}+\cdots + F_{2017}=F_{2018}-F_{1010}$, so 
	$F_{2017}+(F_{1011}+F_{1013}+\cdots + F_{2017})=F_{2017}+F_{2018}-F_{1010}=F_{2019}-F_{1010}$. 
	
\end{enumerate}

\end{document}