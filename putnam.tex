\documentclass[11pt,a4paper]{article}
\usepackage{amsmath, amssymb, fullpage, mathrsfs, bm, pgf, tikz}
\usepackage{mathrsfs}
\usetikzlibrary{arrows}
\setlength{\textheight}{10in}
%\setlength{\topmargin}{0in}
\setlength{\topmargin}{-0.5in}
\setlength{\parskip}{0.1in}
\setlength{\parindent}{0in}

\newcommand{\set}[2]{\{#1\,:\,\text{#2}\}}
\newcommand{\tup}[1]{\mathrm{#1}}
\newcommand{\sfP}{\mathsf{P}}
\newcommand{\M}{\mathsf{M}}
\newcommand{\bbR}{\mathbb R}
\newcommand{\bbC}{\mathbb C}
\newcommand{\bbZ}{\mathbb Z}
\newcommand{\bbN}{\mathbb N}
\newcommand{\bbQ}{\mathbb Q}
\newcommand{\bbF}{\mathbb F}
\newcommand{\dfeq}{\stackrel{\mathrm{def}}{=}}
\newcommand{\ra}{\rightarrow}
\newcommand{\la}{\leftarrow}
\newcommand{\lra}{\leftrightarrow}
\newcommand{\Span}{\mathrm{span}}
\newcommand{\scrP}{\mathscr{P}}
\newcommand{\rank}{\mathrm{rank}}
\newcommand{\nullity}{\mathrm{nullity}}
\newcommand{\Col}{\mathrm{Col}}
\newcommand{\Row}{\mathrm{Row}}
\newcommand{\tr}{\mathrm{tr}}
\newcommand{\ol}{\overline}
\newcommand{\norm}[1]{||#1||}
\newcommand{\doubleline}[1]{\underline{\underline{#1}}}
\newcommand{\elemop}[1]{\stackrel{#1}{\longrightarrow}}
\newcommand{\Ind}{\mathrm{Ind}}
\newcommand{\Res}{\mathrm{Res}}
\newcommand{\End}{\mathrm{End}}
\newcommand{\cl}{\mathrm{cl}}
\newcommand{\code}[1]{\texttt{#1}}
\newcommand\tab[1][0.5cm]{\hspace*{#1}}
\newcommand{\<}{\langle}
\renewcommand{\>}{\rangle}
\newcommand{\qubits}[1]{|{#1}\rangle}
\newcommand{\ord}{\mathrm{ord}}
\newcommand{\lcm}{\mathrm{lcm}}
\newcommand{\dsum}{\displaystyle\sum}
\newcommand{\dprod}{\displaystyle\prod}

\begin{document}

\newcommand{\sgn}{\text{sgn}}

\section{Putnam 2018}
\begin{enumerate}
	\item [\textbf{A1}]
	Find all ordered pairs $(a, b)$ of positive integers for which
	\[\frac{1}{a} + \frac{1}{b} = \frac{3}{2018}.\]
	
	\textbf{Answer.} There are 6 such pairs, obtained by $(1009, 2018), (673, 2018\times 673), (674, 337\times 1009)$ and their inverses. \\
	\textbf{Solution.} Since both $a$ and $b$ are positive, we can multiply both sides by $2018ab$ to get $2018a+2018b=3ab$. Even better, this is the same as $(3a-2018)(3b-2018)=2018^2$. Since both $3a-2018$ and $3b-2018$ are integers congruent to $1\pmod{3}$, it suffices to find the ways to decompose $2018^2=cd$ with each $c, d$ are $1\pmod{3}$. We also know that $2018^2=2^2\times 1009^2$ and both $2$ and $1009$ are primes, so each positive divisor of $2018^2$ must be in the form of $2^{k}\times 1009^{\ell}$ with $0\le k, \ell \le 2$. 
	The modulo 3 constraint also forces $k$ to be even, so $(c, d)=(1, 2018^2), (4, 1009^2), (1009, 4\times 1009)$ and their reverses, giving the answer as above (by letting $a=\frac{c+2018}{3}$ and $b=\frac{d+2018}{3}$). 
	
	\item[\textbf{A2}]
	
	Let $S_1, S_2, \dots, S_{2^n - 1}$ be the nonempty subsets of $\{1, 2, \dots, n\}$ in some order, and let $M$ be the $(2^n - 1) \times (2^n - 1)$ matrix whose $(i, j)$ entry is
	\[m_{ij}=\begin{cases}
		0 & \text{ if } S_i\cap S_j=\emptyset;\\
		1 & \text{ otherwise }\\
	\end{cases}\]
	Calculate the determinant of $M$.
	
	\textbf{Answer.} $1$ if $n=1$, and $-1$ if $n>1$. Alternatively, $(-1)^{(n>1)}$ in the computer science terms (where false is 0 and true is 1). \\
	\textbf{Solution.} If $n=1$ then the matrix is $(1)$, so the determinant is 1. Let $M_n$ be the matrix when the size of the matrix is $n\times n$, and we will show that $\det(M_{n+1})=-(\det(M_n))^2$. 
	
	First, notice that the determinant depends only on $n$ and not the order of the sets $S_1, \cdots , S_{2^n-1}$. (I wrote something way more rigorous during the contest to prevent risking points, but I think the following argument is sufficient). 
	Consider what happens when you swap two arbitrary sets $S_i$ and $S_j$, then we can show that this induces a swap of rows $i$ and $j$ followed by a swap of columns $i$ and $j$ in the matrix $M_n$ (though the two swaps could have been in any order). The two swaps of columns and rows negate the determinant twice, which in turn preserve the determinant. Since each permutation can be obtained from the another by iteratively swapping two items, we have our conclusion here. 
	
	Having established the above, we can rearrange (permute) the $2^n-1$ subsets in anyway we want. Now we need an induction argument to move from $n$ to $n+1$. Let $T_1, \cdots , T_{2^n-1}$ be an arbitrary arrangement of the non-empty subsets of $\{1, \cdots , n\}$, and $M_n$ be the corresponding $n\times n$ matrix. Now consider the arrangement of the subsets $S_i$ of $\{1, \cdots , n+1\}$ as the following: 
	$S_1=\{n+1\}$, $S_{k+1}=\{n+1\}\cup T_k$ for $1\le k\le 2^n-1$, and $S_{2^n+k}=T_k$. We notice the following: 
	\begin{itemize}
		\item $n+1\in S_i\cup S_j$ for $i, j\le 2^n$ so $m_{ij}=1$ for those $i, j$. In addition, if $i>2^k$ then $n+1\not\in S_i$ so $m_{1i}=m_{i1}=0$. 
		
		\item for all $1\le k\le 2^n-1$ we have $S_{2^n+i}\cap S_{j+1}=S_{i+1}\cap S_{2^n+j}=S_{2^n+i}\cap S_{2^n+j}=T_i\cap T_j$. This means that, if we view the $(2^n-1)\times (2^n-1)$ submatrices determined by rows $2$ to $2^n$ and columns $2^n+1$ to $2^{n-1}$, rows $2^n+1$ to $2^{n-1}$ and columns $2$ to $2^n$, and rows $2^n+1$ to $2^{n+1}-1$ and columns $2^n+1$ to $2^{n-1}$, they are all identical, i.e. equal to $M_n$. 
	\end{itemize}
	This gives the following configuration: 
	\[
	\left[\begin{array}{c|c|c}
	1 & [1] & [0]\\
	\hline
	(1) & \<1\> & M_n\\
	\hline
	(0) & M_n & M_n\\ 
	\end{array}\right]
	\]
	where $[x]$ means a $1\times (2^n-1)$ row matrix of $x$, $(y)$ means a $(2^n-1)\times 1$ column matrix of $y$ and $\<z\>$ means a $(2^n-1)\times (2^n-1)$ matrix, all of $z$. Keeping in mind that row reduction preserves the determinant, we subtract rows $i$ from row 1 for all $i=2, \cdots , 2^n$ to get the following: 
	\[
	\left[\begin{array}{c|c|c}
	1 & [1] & [0]\\
	\hline
	(0) & \<0\> & M_n\\
	\hline
	(0) & M_n & M_n\\ 
	\end{array}\right]
	\]
	and a further reduction of rows $2^n+k$ from $1+k$ ($k=1, \cdots , 2^n-1$) to get the following:
	\[
	\left[\begin{array}{c|c|c}
	1 & [1] & [0]\\
	\hline
	(0) & \<0\> & M_n\\
	\hline
	(0) & M_n & \<0\>\\ 
	\end{array}\right]
	\]
	Now this is the reverse block matrix (with the two $M_n$'s reversed), so the determinant is now $(-1)^{2^n-1}(\det(M_n))^2=-(\det(M_n))^2$, as desired (with the $2^n-1$ due to the size of $M_n$). 
	
	\item[\textbf{A4}]
	Let $m$ and $n$ be positive integers with $\gcd(m, n) = 1$, and let
	\[a_k = \left\lfloor \frac{mk}{n} \right\rfloor - \left\lfloor \frac{m(k-1)}{n} \right\rfloor\]for $k = 1, 2, \dots, n$. Suppose that $g$ and $h$ are elements in a group $G$ and that
	\[gh^{a_1} gh^{a_2} \cdots gh^{a_n} = e,\]where $e$ is the identity element. Show that $gh = hg$. (As usual, $\lfloor x \rfloor$ denotes the greatest integer less than or equal to $x$.)
	
	\textbf{Solution.} 
	(Modified from my post on AoPS). 
	We first notice the following relation: 
	\begin{flalign*}
		\frac mn - 1
		&=\frac{mk}{n} - \frac{m(k-1)}{n} - 1\\
		&\le\left\lfloor \frac{mk}{n} \right\rfloor -  \frac{m(k-1)}{n}\\
		&\le a_k = \left\lfloor \frac{mk}{n} \right\rfloor - \left\lfloor \frac{m(k-1)}{n} \right\rfloor\\
		&\le \frac{mk}{n} - \left\lfloor \frac{m(k-1)}{n} \right\rfloor\\
		&< \frac{mk}{n} - \frac{m(k-1)}{n} + 1\\
		&=\frac mn + 1
	\end{flalign*}
	Thus $\lfloor \frac mn\rfloor \le a_k\le \lceil \frac mn \rceil$ for each $a_k$. 
	
	Now we proceed by inducting on both $m$ and $n$. The case where $m=1$ gives rise to the situation $a_{k}=0$ for all $k<n$ and $a_n=1$ so we have $g^{n}h=e$. Since $h=g^{-n}$, $gh=hg=g^{-(n-1)}$, and thus commute. Similarly, the case where $m=2$ gives rise to the $a_1=m$ so $gh^n=e$, and thus $g=h^{-n}$ and by a similar logic, $g$ commutes with $h$.
	
	Now fix the pair $(m, n)$ and suppose that the desired statement holds true for all $(m', n')$ with $m'\le m$ and $n'\le n$ and $(m', n')=(m, n)$, and $\gcd(m, n)=1$. We split this into two cases, bearing in mind that $(m, n)\neq (1, 1)$ here (otherwise our base cases take care of it).
	
	The easier is $m>n$, in which case we have $a_k\ge 1$ for all $k$. Let $d=\lfloor \frac mn\rfloor$ and let $m'$ be $m'\equiv m\pmod{n}$ with $0\le m'<n$, so $m=dn+m'$. Now let $b_k=a_k-d$, and we can rewrite $a_k$ in the following form: 
	\begin{flalign*}
		a_k &= \left\lfloor \frac{mk}{n} \right\rfloor - \left\lfloor \frac{m(k-1)}{n} \right\rfloor\\
		&= \left\lfloor \frac{(m'+dn)k}{n} \right\rfloor - \left\lfloor \frac{(m'+dn)(k-1)}{n} \right\rfloor\\
		&=\left\lfloor \frac{m'k}{n} + dk \right\rfloor - \left\lfloor \frac{m'(k-1)}{n} + d(k-1)\right\rfloor\\
		&= \left\lfloor \frac{m'k}{n} \right\rfloor - \left\lfloor \frac{m'(k-1)}{n} \right\rfloor+d\\
		&=b_k+d
	\end{flalign*}
	so we have 
	\[
	e=gh^{a_1} gh^{a_2} \cdots gh^{a_n}
	=gh^{b_1+d} gh^{b_2+d} \cdots gh^{b_n+d}
	=(gh^d)h^{b_1}\cdots (gh^d)^{b_n}
	\]
	and by our induxtive hypothesis, since the pair $(gh^d, h)$ corresponds to $(m', n)$ and $\gcd(m', n)=\gcd(m, n)=1$, and $m'<n<m$, $gh^d$ and $h$ commutes. 
	Thus, $gh^{d+1}=gh^dh=hgh^d$, and by cancelling $h^d$ from the right we have $gh=hg$, as desired. 
	
	The harder part is $m<n$, i.e. $\frac mn<1$. By the lemma above, we have $0\le a_k\le 1$ for each $a_k$. 
	We also have $a_k=1$ if and only if there exists an integer $a$ such that 
	$\frac{m(k-1)}{n} < a \le \frac{mk}{n}$. 
	We can rearrange this to form 
	$a \le \frac{mk}{n} < a+\frac mn$, which by rearranging again becomes 
	$\frac{na}{m}\le k<\frac{na}{m}+1$. 
	This gives the condition $k=\lceil \frac{na}{m}\rceil$. 
	By omitting those $a_k$'s that are 0 we can rewrite the original equation as 
	\[e=g^{b_1}h\cdots g^{b_m}h\]
	with each $b_k=\lceil \frac{nk}{m}\rceil-\lceil \frac{n(k-1)}{m}\rceil$. Again let $d=\lfloor \frac nm\rfloor$ we can write $n=dm+n'$ with $n'\equiv n\pmod{m}$ and $n'<m$, and we have
	\begin{flalign*}
		b_k&=\left\lceil \frac{nk}{m}\right\rceil-\left\lceil \frac{n(k-1)}{m}\right\rceil
		\\&=\left\lceil \frac{(dm+n')k}{m}\right\rceil-\left\lceil \frac{(dm+n')(k-1)}{m}\right\rceil
		\\&=\left\lceil \frac{n'k}{m}+dk\right\rceil-\left\lceil \frac{n'(k-1)}{m}+d(k-1)\right\rceil
		\\&=\left\lceil \frac{n'k}{m}\right\rceil-\left\lceil \frac{n'(k-1)}{m}\right\rceil + d
	\end{flalign*}
	Thus letting $c_k=b_k-d=\left\lceil \frac{n'k}{m}\right\rceil-\left\lceil \frac{n'(k-1)}{m}\right\rceil$ we have 
	\[e=g^{b_1}h\cdots g^{b_m}h
	=g^{c_1}(g^dh)\cdots g^{c_m}(g^dh)
	\]
	and since $n'<m<n$ and the pair $(g, g^dh)$ corresponds to the pair $(m, n')$ with $\gcd(m, n')=1$, by inductive hypothesis, $g$ and $g^dh$ commutes. Thus $gg^{d}h=g^dhg$, and cancelling $g^d$ from the left we have $gh=hg$. 
	
	\item [\textbf{B2}]
	Let $n$ be a positive integer, and let $f_n(z) = n + (n-1)z + (n-2)z^2 + \dots + z^{n-1}$. Prove that $f_n$ has no roots in the closed unit disk $\{z \in \mathbb{C}: |z| \le 1\}$.
	
	\textbf{Solution.} 
	If $z$ were to be a root to $f_n$, then $z$ is also a root of 
	$f_n(z)(z-1)=z^n+z^{n-1}+\cdots + z - n$, i.e. 
	$n=z^n+z^{n-1}+\cdots + z$. 
	If $|z|\le 1$, then 
	\[n=|z^n+z^{n-1}+\cdots + z|
	\le |z^n|+\cdots |z|
	\le 1+\cdots + 1
	=n
	\]
	so all the equalities above must hold, i.e. we need both $|z^n+z^{n-1}+\cdots + z| = |z^n|+\cdots |z|$ and 
	$|z^n|+\cdots |z|=1+\cdots +1$. The second condition implies that $|z|=1$, so in particular $z$ cannot be 0. 
	The first condition implies that $\frac{z^k}{z^{k-1}}=z$ must be a positive real number, so these conditions together implies that $z=1$ is the only possibility. 
	But then $f_n(1)=\frac{n(n+1)}{2}>0$, contradiction. 
	
	\item[\textbf{B3}] Find all positive integers $n < 10^{100}$ for which simultaneously $n$ divides $2^n$, $n-1$ divides $2^n - 1$, and $n-2$ divides $2^n - 2$.
	
	\textbf{Answer.} $2^{2^{2^m}}$ for $m=0,1,2,3$. \\
	\textbf{Solution.} We first show that all numbers are in this form. The first condition $n\mid 2^n$ implies that $n$ cannot have prime divisors other than 2, so $n=2^k$ for some $k$. Notice, also that this is also a sufficient condition because $\frac{2^{2^k}}{2^k}=2^{2^k-k}$ with $2^k>k$ all the times, so the quotient is indeed an integer.  
	
	Next, $n-1\mid 2^n-1$ can be written as $2^k-1\mid 2^{2^k}-1$. Using the fact that $\gcd(2^a-1, 2^b-1)=2^{\gcd (a, b)}-1$, (or equivalently, $2^a-1\mid 2^b-1$ iff $a\mid b$), we have $k\mid 2^k$, and therefore $k=2^{\ell}$ for some integer $\ell$. 
	
	Finally, $n-2\mid 2^n-2$ also implies $2^{k}-2\mid 2^{2^k}-2$; 
	with $k=2^{\ell}$, $k>0$ so we can divide both sides by 2 and obtain 
	$2^{k-1}-1\mid 2^{2^k-1}-1$. 
	By the same logic as above, $k-1\mid 2^k-1$ and since $k=2^{\ell}$, we have $2^{\ell}-1\mid 2^{2^{\ell}}-1$, so $\ell\mid 2^{\ell}$ by the same logic. Thus $\ell=2^m$. We now have $n=2^{2^{2^m}}$ for $m\ge 0$. 
	
	Finally, we can verify that such $n$ works. To find those with $n<10^{100}$, notice that 
	\[2^{300}=(2^3)^{100}=8^{100}<10^{100}<16^{100}=(2^4)^{100}=2^{400}
	\]
	and therefore $2^{2^m}<400$. Since $400<2^9$, we also need $2^m<9$, so $m\le 3$. 
	On the other hand, when $n=2^{2^{2^3}}$ we have $n=2^{2^{2^3}}=2^{2^8}=2^{256}<2^{300}<10^{100}$ so $m=3$ works. 
	
	\item[\textbf{B4}] Given a real number $a$, we define a sequence by $x_0 = 1$, $x_1 = x_2 = a$, and $x_{n+1} = 2x_nx_{n-1} - x_{n-2}$ for $n \ge 2$. Prove that if $x_n = 0$ for some $n$, then the sequence is periodic.
	
	\textbf{Solution.} We first show that we can eliminate the case where $|a|>1$. Indeed, assume that $|a|>1$. Let's show that $|x_n|$ is nondecreasing. Indeed, if, for some $n$, we have $|x_n|\ge |x_{n-1}|\ge |x_{n-2}|$ then 
	\begin{flalign*}
		|x_{n+1}|&=|2x_nx_{n-1} - x_{n-2}|
		\\&\ge |2x_nx_{n-1}| - |x_{n-2}|
		\\&\ge |2x_nx_{n-1}| - |x_{n-1}|
		\\&=|x_{n-1}|(|2x_n|-1)
		\\&\ge |2x_n|-1
		\\&\ge 2|x_n|-|x_n|
		\\&\ge |x_n|
	\end{flalign*}
	where we used the fact that $|x_n|\ge |x_{n-1}|\ge 1$. Thus with the base case that $|a_0|\le |a_1|\le |a_2|$ (as $1\le a\le a$), we have $|x_n|$ nondecreasing, and thus cannot reach 0 at all. 
	
	Now that $|a|<1$, we can write $a=\cos\theta$ and show by induction that $x_{n}=\cos(F_n\theta)$ with $F_n$ being the $n$-th Fibonacci number with $F_0=0$, $F_1=F_2=1$. Again we can proceed by induction: base case is given by the initial problem condition. Now suppose that $x_k=\cos(F_k\theta)$ for all $1\le k\le n$ for some $n$. Then (using $F_{n-2}+F_{n-1}=F_n$)
	\begin{flalign*}
	x_{n+1}&=2x_nx_{n-1} - x_{n-2}
	\\&=2\cos(F_{n}\theta)\cos(F_{n-1}\theta)-\cos(F_{n-2}\theta)
	\\&=2\cos(F_{n}\theta)\cos(F_{n-1}\theta)-\cos((F_n-F_{n-1})\theta)
	\\&=2\cos(F_{n}\theta)\cos(F_{n-1}\theta)-(\cos (F_n\theta)\cos (F_{n-1}\theta) + \sin (F_n\theta)\sin (F_{n-1}\theta))
	\\&=\cos (F_n\theta)\cos (F_{n-1}\theta) - \sin (F_n\theta)\sin (F_{n-1}\theta)
	\\&=\cos (F_{n+1}\theta)
	\end{flalign*}
	as desired. 
	
	Having established this, let $n_0$ be such that $x_{n_0}=0$. Then $0=x_{n_0}=\cos(F_{n_0}\theta)$, meaning that $F_{n_0}\theta=(\frac{2k_0+1}{2})\pi$ for some integer $k_0$. 
	This also means that $\theta = (\frac{2k_0+1}{2F_{n_0}})\pi$. We first recall that for any integer $m$, the Fibonacci sequence is periodic modulo $m$: indeed, if we write the Fibonacci sequence modulo $m$ and consider all pairs $(F_n, F_{n+1})$, then there are at most $m^2$ distinct elements from the pairs, and thus there are some that are the same. If $(F_{k}, F_{k+1})=(F_{\ell}, F_{\ell+1})$ modulo $m$ for some $k<\ell$, then we can inductively show (both forward and backward) that $F_{k+x}=F_{\ell+x}$ for any integer (positive or negative) $x$. Now if $p_0$ is the period of the Fibonacci sequence when $m=4F_0$ then $F_{k+p_0}\equiv F_k\pmod{4F_{n_0}}$ for all $k$, and therefore $(F_{k+p_0}-F_k)\theta = (F_{k+p_0}-F_k)(\frac{2k_0+1}{2F_{n_0}})\pi$ is an integer multiple of $2\pi$. Since $\cos$ is periodic modulo $2\pi$, we have $\cos (F_{k+p_0}\theta) = \cos(F_k\theta)$, and thus $x_k=x_{k+p_0}$. 
	
\end{enumerate}

\section{Putnam 2017}
\begin{enumerate}
\item[\textbf{A1}]Let $S$ be the smallest set of positive integers such that

\begin{enumerate}
	\item $2$ is in $S,$
	\item $n$ is in $S$ whenever $n^2$ is in $S,$ and
	\item $(n+5)^2$ is in $S$ whenever $n$ is in $S.$
	\end{enumerate}

Which positive integers are not in $S?$

(The set $S$ is ``smallest" in the sense that $S$ is contained in any other such set.)

\textbf{Answer.} 1 and all integers divisible by 5. \\
\textbf{Solution.} To show that all numbers not in the above category must be in $S$, we note the following lemma: if $n$ is in $S$ for some $n$, then by (c), $(n+5)^2$ is in $S$ and by (b), $n+5$ is in $S$. Hence by repeated iteration of this process, we get 
\[n\in S\to n+5k\in S, \forall k\ge 0\cdots (d)\]
Thus starting from $2\in S$ as of (a), we get $2+5k\in S\forall k\ge 0$. Now (a) and (c) combined imply that $7^2=49\in S$, too. By (c) again, $(49+5)^2=54^2\in S$ too. Notice that $56^2-54^2=2\times 110$ is divisible by 5 and is nonnegative, so $56^2\in S$ by (d) again. By (b), $56\in S$ and by (d) again, $9^2=81=56+5(5)\in S$ and $11^2=121=56+5(13)\in S$, so by (b), $9, 11\in S$. By (b) again, $\sqrt{9}=3\in S$. Finally, since $11\in S$, by (d) again, $11+5=16\in S$, so by (b), $\sqrt{16}=4\in S$. Similarly, $11+5(5)=36\in S$, by (d) again. Thus $\sqrt{36}=6\in S$. Since $2, 3, 4, 6\in S$ so by (d), $2+5k, 3+5k, 4+5k, 6+5k\in S$. These are all the numbers that are not 1 and not divisible by 5. 

To show that $S_1\{a: a>1, 5\nmid a\}$ is valid, let $a$ be arbitrary integer in $S_1$. Clearly, $2\in S_1$, so (a) is satisfied. If $a=k^2$ for some $k$, then from $a>1$ then $k=\sqrt{a}>1$. Since $5\nmid a, 5\nmid\sqrt{a}=k$ too. So $5\nmid k$. Hence (b) is fulfilled. Finally, $(a+5)^2>a>1$, and from $5\nmid a$, we have $5\nmid a+5$. As 5 is a prime number, $5\nmid (a+5)^2$ too. Thus (c) is also fulfilled. 

\item[\textbf{A2}]
Let $Q_0(x)=1$, $Q_1(x)=x,$ and
\[Q_n(x)=\frac{(Q_{n-1}(x))^2-1}{Q_{n-2}(x)}\]for all $n\ge 2.$ Show that, whenever $n$ is a positive integer, $Q_n(x)$ is equal to a polynomial with integer coefficients.

\textbf{Solution.} We show that $Q_n(x)=xQ_{n-1}(x)-Q_{n-2}(x)$ for all $n\ge 2$ via induction. For $n=2$ (base case), we have $Q_2(x)=x^2-1=x(x)-1=xQ_1(x)-Q_0(x)$. Now suppose that $Q_{n-1}(x)=xQ_{n-2}(x)-Q_{n-3}(x)$ for some $n\ge 3$. We consider the following: 
\begin{flalign*}
	Q_{n-1}^2(x)-1&=(xQ_{n-2}(x)-Q_{n-3}(x))Q_{n-1}-1\\
	&=xQ_{n-2}(x)Q_{n-1}(x)-Q_{n-3}(x)Q_{n-1}(x)-1\\
	&=xQ_{n-2}(x)Q_{n-1}(x)-(Q_{n-3}(x)Q_{n-1}(x)+1)\\
	&=xQ_{n-2}(x)Q_{n-1}(x)-Q_{n-2}^2(x)\\
	&=Q_{n-2}(x)(xQ_{n-1}(x)-Q_{n-2}(x))
\end{flalign*}
notice the use of the fact $Q_{n-3}(x)Q_{n-1}(x)+1=Q_{n-2}^2(x)$ as followed form the definition $Q_{n-1}(x)=\frac{(Q_{n-2}(x))^2-1}{Q_{n-3}(x)}$> Therefore we have $Q_n(x)=\frac{(Q_{n-1}(x))^2-1}{Q_{n-2}(x)}=xQ_{n-1}(x)-Q_{n-2}(x)$. By inductive hypothesis, we get $Q_n(x)=xQ_{n-1}(x)-Q_{n-2}(x)$ for all $n\ge 2$. Since $Q_0$ and $Q_1$ are 

\item[\textbf{A3}]
Let $a$ and $b$ be real numbers with $a<b,$ and let $f$ and $g$ be continuous functions from $[a,b]$ to $(0,\infty)$ such that $\int_a^b f(x)\,dx=\int_a^b g(x)\,dx$ but $f\ne g.$ For every positive integer $n,$ define
\[I_n=\int_a^b\frac{(f(x))^{n+1}}{(g(x))^n}\,dx.\]Show that $I_1,I_2,I_3,\dots$ is an increasing sequence with $\displaystyle\lim_{n\to\infty}I_n=\infty.$

\textbf{Solution.} First, we notice the following use of the Cauchy-Schawz inequality in the form of integrals: 
\[I_{n-1}\cdot I_{n+1}=\int_a^b\frac{(f(x))^{n}}{(g(x))^{n-1}}\,dx\cdot \int_a^b\frac{(f(x))^{n+2}}{(g(x))^{n+1}}\,dx
\ge \left(\int_a^b\frac{(f(x))^{n+1}}{(g(x))^n}\,dx\right)^2
=I_n^2
\]
In particular, substituting $n=0$ we get $I_{-1}I_1\ge I_0$. Now $I_{0}=\int_a^b f(x)dx$ and $I_{-1}=\int_a^b g(x)dx$, so $I_0=I_{-1}$, and thus $I_1\ge I_0$. Since $f(x)$ and $g(x)$ are both continuous on $[a, b]$, so is the function $\frac{f(x)^2}{g(x)}$, so equality can only hold if and only if $\frac{f(x)^2}{g(x)}\div g(x)$ is constant on $[a, b]$. This requires $|f(x)|=|g(x)|$ on $[a, b]$, which becomes $f(x)=g(x)$ since both positive returns only positive values. However, this is not true since $f\neq g$. 

So $I_1>I_0$, and denote the ratio $\frac{I_1}{I_0}=c>1$. We will in fact claim that $\frac{I_{n+1}}{I_n}\ge c$ for all $n\ge 0$, which will finish the proof since $I_n\ge c^nI_0$ and $\lim_{n\to\infty}c^n=\infty$ as $c>1$. The base case is given as $\frac{I_1}{I_0}=c$. If $\frac{I_{n}}{I_{n-1}}\ge c$ for some $n\ge 1$, then from the Cauchy-schawrz inequality we had before, $I_{n-1}I_{n+1}\ge I_n^2$ means that $\frac{I_{n+1}}{I_n}\ge \frac{I_n}{I_{n-1}}=c$. Hence we completed our inductive hypothesis, and concludes the proof. 

\item[\textbf{A4}] A class with $2N$ students took a quiz, on which the possible scores were $0,1,\dots,10.$ Each of these scores occurred at least once, and the average score was exactly $7.4.$ Show that the class can be divided into two groups of $N$ students in such a way that the average score for each group was exactly $7.4.$

\textbf{Solution.} The total score of the group is $14.8N=\frac{74N}{5}$, which is an integer since all individual scores are integers. Since $\gcd(74, 5)=1$, we have $N=5k$ for some integer $k$. This means $14.8N=14.8(5k)=74k$, which is even. Thus the goal now becomes finding a group of $N$ students where the total score is $7.4N$, which is $37k$, an integer. 

Let $x_1\le\cdots\le x_{2N}$ be the scores of students. Let $m=x_1+\cdots + x_N$ and $M=x_{N+1}+\cdots x_{2N}$. Since $m+M=14.8N$ and from $x_i\le x_{N+i}$ we have $m\le M$, we have $m\le x\le M$. We will show that for any integer $x\in \{m, m+1, \cdots , M-1, M\}$ it is possible to choose a group of $N$ students such that the total score in this group is $x$, thereby showing that it is possible to choose a group of $N$ students with the total group score of $7.4N$. 

We first notice that for all $1\le i < N$, $0\le x_{i+1}-x_{i}\le 1$. The left inequality is obvious by our sorting algorithm. Suppose that $x_{i+1}-x_i\ge 2$ for some $i$. By our sorting algorithm, again, nobody scored $x_i + 1, \cdots , x_{i+1}-1$. We now define a sequence of $N^2+1$ numbers $y_0, y_1, \cdots , y^{N^2}$ as follows: 
\begin{itemize}
	\item $y_0=x_1+x_2+\cdots + x_N$
	\item For some $i<N^2$, denote $y_i=x_{a_1}+x_{a_2}+\cdots + x_{a_N}$ for some $1\le a_1<a_2<\cdots < a_N\le 2N$. If there exists $j<N$ such that $a_{j+1}-a_j>1$, then denote $y_{j+1} = x_{a_1}+x_{a_2}+\cdots + x_{a_j + 1} + x_{a_{j+1}}+\cdots x_{a_N}$ (basically, shift one of the indices to the right by 1). 
	Otherwise, denote $y_i = x_{a_1}+x_{a_2}+\cdots + x_{a_N+1}$. 
\end{itemize}
We first show that this construction sequence is legitimate: that is, when $i<2N$, either such $j$ can be found or $a_N < 2N$ (so $x_{a_N+1}$ exists). To see why, we consider the sum of indices $S(i)=a_1+a_2+\cdots +a_N$ when $y_i=x_{a_1}+\cdots + x_{a_N}$. 
When $i=0$ then sum is $S(0)=1+\cdots + N=\frac{N(N+1)}{2}$, and whenever the sequence $y_i$ and $y_{i+1}$ are both legitimate, $S(i+1)-S(i)=1$. Thus, the recursion from $y_i$ to $y_{i+1}$ is legitimate if and only if $y_i$ is not $x_{N+1}+\cdots + x_{2N}$. If such $i$ exists, then $S(i)=(N+1)+\cdots + (2N)=\frac{N(3N+1)}{2}=S(0)+N^2$. It then follows that such $i$ must be at least $N^2$ for this to happen. 
The converse is also true: we have $y_{N^2}=x_{N+1}+\cdots + x_{2N}$. 

In addition, for each $0\le i < 2N$, by the construction above there exists index $j$ such that $y_{i+1}-y_i = x_{j+1}-x_j$. By an earlier lemma, $0\le x_{j+1}-x_j\le 1$, so $0\le y_{i+1}-x_i\le 1$. We also have $y_0=x_1+\cdots +x_N=m$ and $y_{N^2}=X_{N+1}+\cdots + x_{2N}=M$, which means:
\[
m=y_0\le y_1\le \cdots \le y_{N^2}\le M
\]
which means the set $\{y_0, \cdots , y_{N^2}\}$ is precisely the set of integers in the interval $[m, M]$, inclusive. 

\item[\textbf{B2}]Suppose that a positive integer $N$ can be expressed as the sum of $k$ consecutive positive integers
\[N=a+(a+1)+(a+2)+\cdots+(a+k-1)\]for $k=2017$ but for no other values of $k>1.$ Considering all positive integers $N$ with this property, what is the smallest positive integer $a$ that occurs in any of these expressions?

\textbf{Answer.} $a=16$ \\
\textbf{Solution.} $N$ can be written as sum of $k$ consecutive positive integers if and only if $N=\frac{k(2a+k-1)}{2}$ for some positive integer $a$. This means $N$ need to satisfy the following properties: 
\begin{enumerate}
	\item $N\ge \frac{k(k+1)}{2}$
	\item $k|N$ for $k$ odd, and $k|N-\frac{k}{2}$ when $k$ is even. 
\end{enumerate}
The second condition is due to the fact that, when considering mod $k$, $a, a+1, \cdots , a+k-1$ is congruent to $1, 2, \cdots , k$ in some order, and thus $N\equiv \frac{k(k+1)}{2}\pmod{k}$. If $k$ is odd then this is divisible by 0; converse ly if $k$ is even, then $k+1$ is odd so it's congruent to $\frac{k}{2}$. 

Coming back to the problem, we need one such $N$ that can be written as sum of $k$ consecutive integers. Denote $N=2017\cdot m$ with $m\ge 1009$. Now consider the case when $m\le 1024$. If $m$ has an odd divisor that's greater than 1, say $q$, then $q|N$ too, and since $N\ge \frac{2017(2018)}{2}ge \frac{q\cdot (q+1)}{2}$ (since $q\le m< 2017$), it can be written as the sum of $q$ integers, too. This $m$ will then not be valid. This happens when $m\le 1024$ and has an odd divisor $>1$, which is equivalent to the fact that it is not a power of 2. Hence $m\ge 1024$. 

To show that $m=1024$ is good, observe that its only odd divisors are 1 and 2017, so if $q$ is odd and it can be written as sum of $q$ consecutive numbers, then $q=1$ or $q=2017$. Now suppose that $q$ is even, whereby we have $N\equiv \frac{q}{2}\pmod{q}$. This means that $2N\equiv 0\pmod{q}$, i.e. $q=2^k 2017^{\ell}$ with $1\le k\le 11$ and $0\le\ell\le 1$. With $q\nmid N$ we must have $k=11$, so the only choice is $q=2^{11}$ and $q=2^{11}\cdot 2017$. However, $q\ge 2048$ so $N\ge \frac{2048(2049)}{2}=1024\cdot 2049>1024\cdot 2017$, contradiction. Hence $k=2017$ is the only possibility here. Since $2a+k-1=2048$ in this case, $a=16$. 

\item[\textbf{B3}] Suppose that $$f(x) = \sum_{i=0}^\infty c_ix^i$$is a power series for which each coefficient $c_i$ is $0$ or $1$. Show that if $f(2/3) = 3/2$, then $f(1/2)$ must be irrational.

\textbf{Solution.} Consider $c=f(1/2)$, and consider the binary representation of $c$. We know that if $c$ is rational, then the binary digits (after decimal point) is eventually periodic. We show that this is the same for the sequence $\{c_i\}$ too. 

Clearly, if $c<2$ and $c=\overline{d_0.d_1d_2d_3\cdots}$, is the binary representation, then putting $c_i=d_i$ we have $\sum_{i=0}^\infty c_ix^i=c$, too (The case $c\ge 2$ happens only when $c\ge 2$, but then $c=f(1/2)\le 1+1/2+1/4+\cdots = 2$, so equality must hold and we have $c_i=1$ for all $i$, and therefore $\{c_i\}$ is periodic). 
If $\{c_i\}$ is indeed the binary representation we are done. 
Now, suppose that $\{c_i\}$ is not the binary representation: this means $c$ has more than 1 way to be represented as the power series. 
Let $\{c_i\}$ and $\{d_i\}$ to be two different representations and let $n_0$ be the minimum index such that $c_{n_0}\neq d_{n_0}$. WLOG let $c_{n_0}=0$ and $d_{n_0}=1$. Then 
\[
\sum_{i=n_0 + 1}^\infty c_i/2^i = c - \sum_{i=0}^{n_0-1} c_i/2^i = 1/2^{n_0}+\sum_{i=n_0+1}^\infty d_i/2^i 
\]
But then 
\[
\sum_{i=n_0 + 1}^\infty c_i/2^i\le \sum_{i=n_0 + 1}^\infty 1/2^i
=1/2^{n_0}
\le 1/2^{n_0}+\sum_{i=n_0+1}^\infty d_i/2^i
\]
therefore equality must hold: $c_i=0$ and $d_i=1$, both for all $i>n_0$. Thus both $\{c_i\}$ and $\{d_i\}$ is eventually periodic with period 1 (and we are done). 

Now, given that $\{c_i\}$ is eventually periodic: there is an $n_0\ge 0$ and $m\ge 1$ such that for all $n\ge n_0$ we have $c_n=c_{n+m}$. 
We now have 

\begin{flalign*}
	f(2/3)&=\dsum_{i=0}^{\infty}\frac{2^ic_i}{3^i}
	\\&=\dsum_{i=0}^{n_0-1}\frac{2^i3^{n_0-1-i}c_i}{3^{n_0-1}}
	+\dsum_{i=n_0}^{\infty}\frac{2^ic_i}{3^i}
	\\&=\dsum_{i=0}^{n_0-1}\frac{2^i3^{n_0-1-i}c_i}{3^{n_0-1}}
	+\dsum_{i=n_0}^{n_0+m-1}c_i\left(\frac{2^i}{3^i}+\frac{2^{i+m}}{3^{i+m}}+\frac{2^{i+2m}}{3^{i+2m}}+\cdots\right)
	\\&=\dsum_{i=0}^{n_0-1}\frac{2^i3^{n_0-1-i}c_i}{3^{n_0-1}}
	+\dsum_{i=n_0}^{n_0+m-1}c_i\cdot \frac{2^i}{3^i} \cdot \frac{3^m}{3^m-2^m}
\end{flalign*}
and since for each $i$ and $m$, all $3^{n_0-1}, 3^i$ and $3^m-2^m$ are odd, the quantity $f(2/3)$ can be written as $p/q$ with $q$ odd. However, given that $f(2/3)=1/2$ and $1/2$ doesn't have this property (2 is even and 1/2 is irreducible: if $q$ is odd then $\frac q2$ is not an integer), this is a contradiction. Thus $\{c_i\}$ cannot be eventually periodic and the conclusion follows. 


\item[\textbf{B4}]Evaluate the sum
\[\sum_{k=0}^{\infty}\left(3\cdot\frac{\ln(4k+2)}{4k+2}-\frac{\ln(4k+3)}{4k+3}-\frac{\ln(4k+4)}{4k+4}-\frac{\ln(4k+5)}{4k+5}\right)\]\[=3\cdot\frac{\ln 2}2-\frac{\ln 3}3-\frac{\ln 4}4-\frac{\ln 5}5+3\cdot\frac{\ln 6}6-\frac{\ln 7}7-\frac{\ln 8}8-\frac{\ln 9}9+3\cdot\frac{\ln 10}{10}-\cdots.\]

\textbf{Solution.} (Cited from my post on AoPS)
To avoid dealing with problems in absolute convergence, we deal with the $n$-th partial sum. That is,
\[\sum_{k=0}^{n}\left(3\cdot\frac{\ln(4k+2)}{4k+2}-\frac{\ln(4k+3)}{4k+3}-\frac{\ln(4k+4)}{4k+4}-\frac{\ln(4k+5)}{4k+5}\right)\]\[=3\cdot\frac{\ln 2}2-\frac{\ln 3}3-\frac{\ln 4}4-\frac{\ln 5}5+\cdots +3\cdot\frac{\ln(4n+2)}{4n+2}-\frac{\ln(4n+3)}{4n+3}-\frac{\ln(4n+4)}{4n+4}-\frac{\ln(4n+5)}{4n+5}.\]Because the sum is finite here, we have no issue of convergence and therefore can do the following conversion:
\[\sum_{k=0}^{n}\left(3\cdot\frac{\ln(4k+2)}{4k+2}-\frac{\ln(4k+3)}{4k+3}-\frac{\ln(4k+4)}{4k+4}-\frac{\ln(4k+5)}{4k+5}\right)\]\[=\sum_{k=0}^{n}\left(\frac{\ln(4k+2)}{4k+2}-\frac{\ln(4k+3)}{4k+3}+\frac{\ln(4k+4)}{4k+4}-\frac{\ln(4k+5)}{4k+5}\right)
+\sum_{k=0}^{n}2\left(\frac{\ln(4k+2)}{4k+2}-\frac{\ln(4k+4)}{4k+4}\right)
\]Also notice that
\[\sum_{k=0}^{n}2\left(\frac{\ln(4k+2)}{4k+2}-\frac{\ln(4k+4)}{4k+4}\right)
=\sum_{k=0}^{n}\left(\frac{\ln 2 + \ln(2k+1)}{2k+1}-\frac{\ln 2 + \ln(2k+2)}{2k+2}\right)\]So we have
\[\sum_{k=0}^{n}\left(\frac{\ln(4k+2)}{4k+2}-\frac{\ln(4k+3)}{4k+3}+\frac{\ln(4k+4)}{4k+4}-\frac{\ln(4k+5)}{4k+5}\right)
+\sum_{k=0}^{n}2\left(\frac{\ln(4k+2)}{4k+2}-\frac{\ln(4k+4)}{4k+4}\right)\]\[=\sum_{k=0}^{2n+1}\left(\frac{\ln(2k+2)}{2k+2}-\frac{\ln(2k+3)}{2k+3}\right)
+\sum_{k=0}^{n}\left(\frac{\ln 2 + \ln(2k+1)}{2k+1}-\frac{\ln 2 + \ln(2k+2)}{2k+2}\right)\]\[
=\ln 2\sum_{k=0}^{n}\left(\frac{1}{2k+1}-\frac{1}{2k+2}\right)-\frac {\ln (2n+2)}{2n+2}
+\sum_{k=n+1}^{2n+1}\left(\frac{\ln(2k+2)}{2k+2}-\frac{\ln(2k+3)}{2k+3}\right)
\]Now, notice that $\frac {\ln x}{x}$ is a decreasing sequence with limit $0$ as $x\to\infty$. Thus $\sum_{k=0}^{2n+1}\left(\frac{\ln(2k+2)}{2k+2}-\frac{\ln(2k+3)}{2k+3}\right)$ is an alternating sum hence converges), which means that $\sum_{k=n+1}^{2n+1}\left(\frac{\ln(2k+2)}{2k+2}-\frac{\ln(2k+3)}{2k+3}\right)\to 0$ as $n\to 0$. It is also well known that $\sum_{k=0}^{n}\left(\frac{1}{2k+1}-\frac{1}{2k+2}\right)\to\ln 2$ as $n\to\infty$. Therefore
$\lim_{n\to\infty}\ln 2\sum_{k=0}^{n}\left(\frac{1}{2k+1}-\frac{1}{2k+2}\right)-\frac {\ln (2n+2)}{2n+2}
+\sum_{k=n+1}^{2n+1}\left(\frac{\ln(2k+2)}{2k+2}-\frac{\ln(2k+3)}{2k+3}\right)
=(\ln 2)^2+0+0=(\ln 2)^2$

\item[\textbf{B5}]A line in the plane of a triangle $T$ is called an equalizer if it divides $T$ into two regions having equal area and equal perimeter. Find positive integers $a>b>c,$ with $a$ as small as possible, such that there exists a triangle with side lengths $a,b,c$ that has exactly two distinct equalizers.

\textbf{Answer.} $9, 8, 7$ \\
\textbf{Solution.} Throughout the solution we focus on lines that split $T$ into equal perimeter. This line is only meaningful if it either passes through two of the sides of the triangle, or it passes through a vertex and its opposite side. In the second case, the fact that this line is an equalizer means that it has to be a median of a side, say having length $c$. Let $m$ to be the length of median, then the perimeter of the first triangle is $a+\frac c2+m$ and the second, $b+\frac c2 + m$. But since $a\neq b$, this cannot be an equalizer. 

So now each equalizer must pass through exactly two of the sides (it has to be 2 or 0 by menelaus' theorem, and the case of 0 is impossible since it doesn't divide $T$ at all). 
From now on, denote $s=\frac{a+b+c}{2}$, the semiperimeter. 
We consider each of the three cases (following $a>b>c$):
\begin{enumerate}
	\item If the line passes through sides with length $b$ and $c$, let the line cut the first side into a smaller triangle of length $b_1$, $c_1, m$, with $b_1$ on the $b$-side and $c_1$ on the $c$-side. This splits $T$ into a triangle of perimeter $b_1+c_1+m$ and a quadrilateral of length $a+m+(b-b_1)+(c-c_1)$, which means $b_1+c_1=\frac{a+b+c}{2}=s$, and the ratio of area of smaller triangle to the bigger one is $\frac{b_1c_1}{bc}$ (for the case of equalizer, this ratio must be $\frac 12$). Given that $b-1+c_1=s$, we have $b_1c_1=\frac{s^2-(b_1-c_1)^2}{4}$. Now considering all such lines on the two sides satisfying the perimeter constraint, we have $b_1\le b$ and $c_1\le c$, which means we have $c_1\ge (s-b)$ and $b_1\le s-c$. Thus $b_1-c_1$ has to lie in the interval $[s-2c, 2b-s]$. Given that $b<a$ and $c<a$, when $b_1=b$ we have $c_1=s_b$ so the ratio of the triangle area is now $\frac{s-b}{c}=\frac{a+c-b}{2c}>\frac 12$ since $a>b$. Similarly when $c_1=c$ we have $b_1=s-c$ and the resulting ratio is $\frac{a+b-c}{2b}>\frac 12$ since $a>c$. Therefore we get $\frac{s^2-(b_1-c_1)^2}{4}>\frac 12 bc$ when $b_1-c_1\in\{s-2c, 2b-s\}$. For all $x\in [s-2c, 2b-s]$ we either have $|x|\le s-2c$ or $|x|\le 2b-s$, so we always have $\frac{s^2-(b_1-c_1)^2}{4}>\frac 12 bc$. Hence no equalizer in this case. 
	
	\item Similar to the case above we consider what happened when it passes through length $a$ and $c$. Now denote $a_1$ and $c_1$ like above; we get that $a_1-c_1$ is in the interval $[s-2c, 2a-s]$. Now when $a_1=a$ the resulting ratio is $\frac{(s-a)}{c}=\frac{b+c-a}{2c}<\frac 12$ while if $c_1=c$ the ratio is $\frac{s-c}{a}=\frac{a+b-c}{2a}>\frac 12$. Thus the value $a_1c_1=\frac{s^2-(a_1-c_1)^2}{4}>\frac 12 ac$ when $a_1-c_1=s-2c$ while is $<\frac 12 ac$ when $a_1-c_1=2a-s$. Therefore considering $x$ that satisfies $\frac{s^2-x^2}{4}=\frac 12 ac$, we get $|x|< 2a-s$ while $|x|>s-2c$. This implies that there's exactly one such $x$ in the interval $[s-2c, 2a-s]$, and has one equalizer. 
	
	\item Finally, let the line cuts the sides $a$ and $b$ which forms a smaller triangle with length $a_1$ on side $a$ and $b_1$ on side $b$, then $a_1-b_1\in [s-2b, 2a-s]$. When $a_1=a$ we have $b_1=s-a$ and the area ratio becomes $\frac{s-a}{b}=\frac{b+c-a}{2b}<\frac 12$, and similarly for $b_a=b$ we get $a_1=s-b$, so the ratio becomes $\frac{s-b}{a}=\frac{c+a-b}{2a}<\frac 12$. Thus $\frac{s^2-(a_1-c_1)^2}{4}<\frac 12 ac$ when $a_1-c_1$ is at these extreme points. If $s^2<2ac$ then there's no equalizer in this case; if $s^2=2ac$ then equalizer exists when $a_1=c_1$ (here, $0\in [s-2b, 2a-s]$ since $s-2b=\frac{a+c-3b}{2}<\frac{a-2b}{2}<0$) as $2b>b+c>a$ by triangle inequality, and $2a-s=\frac{3a-b-c}{2}>\frac{3a-a-a}{2}>0$); if $s^2>2ac$, denote $x$ as the two solutions to $s^2-x^2=2ac$. From our example we have $|x|<|s-2b|$, $|x|<|2a-s|$ and $s-2b<0<2a-s$ so both solutions lie in the interval $[s-2b, 2a-s]$. In this case we have two equalizers. 
\end{enumerate}
Now knowing all the cases above, there must be exactly 1 equalizer in the second case, and exactly 1 equalizer in the third case. The third case implies that $a_1=b_1=\frac{s}{2}$, which entails (by the equality of area) $\frac{s^2}{4}=\frac 12 ab$, or $(a+b+c)^2=8ab$. For $8ab$ to be a square, we need $ac=2\cdot k^2$ for some $k$, bearing in mind that $2a>2b>a$. Considering $k=1, 2, \cdots$, the smallest $k$ that has this property is when $a=9, b=8$, forcing $c=7$. For $k\ge 7$ we have $a>k\sqrt{2}=7\sqrt{2}>9$, so $a=9$ is the smallest possible answer. 
\end{enumerate}

\section{Putnam 2016}
\begin{enumerate}
	\item[\textbf{A1}]Find the smallest positive integer $j$ such that for every polynomial $p(x)$ with integer coefficients and for every integer $k,$ the integer
	\[p^{(j)}(k)=\left. \frac{d^j}{dx^j}p(x) \right|_{x=k}\](the $j$-th derivative of $p(x)$ at $k$) is divisible by $2016.$
	
	\textbf{Answer.} $j=8$. \\
	\textbf{Solution.} Consider $p(x)=x^j$, and we know that $p^{(j)}(x)=j(j-1)\cdots 1=j!$. The condition implies that $2016|j!$. Since $7!=5040$ is not divisible by 2016, and $j!|7!$ for $j\le 7$, we know that $2016\nmid j!$ for $j\le 7$. So $j\ge 8$. 
	
	Now suppose that $j\ge 8$. Let $p(x)=\displaystyle\sum_{i=0}^{n} a_ix^i$. Notice that differentiating the term $x^i$ $j$ times gives $i(i-1)\cdots (i-j+1)x^{i-j}$; in particular, this term is 0 if $i\le 7$. Hence multiplying each term by $a_i$ and summing them up we get 
	\[p^{(j)}(x)=\sum_{i=j}^{n} i(i-1)\cdots (i-j+1)a_ix^{i-j}\]
	Notice that we omit all terms with $i<j$ since they contribute 0 to the sum anyway, with reasons explained above. Observe also that the coefficient of $x^{i-j}$ in this derivative is $i(i-1)\cdots (i-j+1)=j!\dbinom{i}{j}$, hence is divisible by $j!$. For $j\ge 8$, $2016|40320=8!|j!$, so each term 
	$i(i-1)\cdots (i-j+1)a_ix^{i-j}$ is divisible by 2016 whenever $x$ is an integer (in particular this holds true for $x=k$). Hence any $j\ge 8$ works, and so the required $j$ is 8. 
	
	\item[\textbf{A2}]Given a positive integer $n,$ let $M(n)$ be the largest integer $m$ such that
	\[\binom{m}{n-1}>\binom{m-1}{n}.\]Evaluate
	\[\lim_{n\to\infty}\frac{M(n)}{n}.\]
	
	\textbf{Answer.} $\dfrac{3+\sqrt{5}}{2}$\\
	\textbf{Solution.} $m=n$ works since the left hand side is $n$ while the right hand side is $0$, so $m>n$ and we can then assume that $m$ is positive below. 
	We first try to consider the following inequality: 
	\[\frac{m!}{(n-1)!(m-n+1)!}>\frac{(m-1)!}{n!(m-n-1)!}\]
	Cancelling factors, we are left with $\dfrac{m}{(m-n)(m-n+1)}>\dfrac{1}{n}$, so $(m-n)(m-n+1)>mn$. Expanding this, we get 
	\[m^2-m(3n-1)+(n^2-n)<0\]
	Using the formula for quadratic inequality, we get 
	\[m\in\left(\frac{(3n-1)-\sqrt{(3n-1)^2-4(n^2-n)}}{2}, \frac{(3n-1)+\sqrt{(3n-1)^2+4(n^2-n)}}{2}\right)\]
	This means that $M(n)$ is the unique integer lying in the interval 
	$[\frac{(3n-1)+\sqrt{(3n-1)^2+4(n^2-n)}}{2} - 1, \frac{(3n-1)+\sqrt{(3n-1)^2+4(n^2-n)}}{2})$
	, which also means
	\[\frac{M(n)}{n}\in \left[\frac{(3-\frac 1n)+\sqrt{(3-\frac 1n)^2-4(1-\frac 1n)}}{2} - \frac 1n, \frac{(3-\frac 1n)+\sqrt{(3-\frac 1n)^2-4(1-\frac 1n)}}{2}\right)\]
	now $\lim_{n\to \infty} \frac{(3-\frac 1n)+\sqrt{(3-\frac 1n)^2-4(1-\frac 1n)}}{2} - \frac 1n
	=\frac{3+\sqrt{3^2-4}}{2}=\frac{3+\sqrt{5}}{2}$ and $\lim_{n\to \infty} \frac{(3-\frac 1n)+\sqrt{(3-\frac 1n)^2-4(1-\frac 1n)}}{2}
	=\frac{3+\sqrt{3^2-4}}{2}=\frac{3+\sqrt{5}}{2}$. By Squeeze's theorem, we get $\lim_{n\to\infty}\frac{M(n)}{n}=\frac{3+\sqrt{5}}{2}$, as desired. 
	
	\item[\textbf{A3}]
	Suppose that $f$ is a function from $\mathbb{R}$ to $\mathbb{R}$ such that
	\[f(x)+f\left(1-\frac1x\right)=\arctan x\]for all real $x\ne 0.$ (As usual, $y=\arctan x$ means $-\pi/2<y<\pi/2$ and $\tan y=x.$) Find
	\[\int_0^1f(x)\,dx.\]
	
	\textbf{Answer.} $\frac{3\pi}{8}$. \\
	\textbf{Solution.} We first focus on the case $x\neq 0, 1$. 
	Plugging $1-\frac1x$ into the equation above we get $f\left(1-\frac1x\right)+f\left(-\frac{1}{x-1}\right)=\arctan \left(1-\frac1x\right)$ and plugging $-\frac{1}{x-1}$ we get $f\left(-\frac{1}{x-1}\right)+f(x)=\arctan(-\frac{1}{x-1})$. Thus adding all these we get: 
	\[2\left(f(x)+f\left(1-\frac1x\right)+f\left(-\frac{1}{x-1}\right)\right)=\arctan x + \arctan \left(1-\frac1x\right) + \arctan\left(-\frac{1}{x-1}\right)\]
	Thus for all $x\neq 0, 1$ we have 
	\[f(x)=\frac{\arctan x - \arctan \left(1-\frac1x\right) + \arctan\left(-\frac{1}{x-1}\right)}{2}\]
	First, notice that $\arctan$ is an odd function (well-known), so $- \arctan \left(1-\frac1x\right)= \arctan \left(\frac1x -1\right)=\arctan \left(\frac{1-x}{x}\right)$, so we may rewrite $f(x)$ as 
	$\frac{\arctan x + \arctan \left(\frac{1-x}{x}\right) + \arctan\left(\frac{1}{1-x}\right)}{2}$
	 Second, we consider the following: 
	\[f(1-x)=\dfrac{\arctan (1-x) + \arctan \left(\frac{x}{1-x}\right) + \arctan\left(\frac{1}{x}\right)}{2}\]
	Then, we use the fact that if $a>0$, $\arctan a+\arctan \frac{1}{a}=\frac{\pi}{2}$. This gives 
	\[f(x)+f(1-x)=\frac{\arctan x + \arctan \left(\frac{1-x}{x}\right) + \arctan\left(\frac{1}{1-x}\right)+\arctan (1-x) + \arctan \left(\frac{x}{1-x}\right) + \arctan\left(\frac{1}{x}\right)}{2}\]
	$=\frac{3(\frac{\pi}{2})}{2}=\frac{3\pi}{4}$. 
	So $\int_0^1 f(x)+f(1-x)dx=\int_0^1 \frac{3\pi}{4}dx=\frac{3\pi}{4}$ (notice that all the computations are not valid when $x=0$ or 1, but the integral is still good even when we remove the two points 0 and 1 from our computation since a finite set of points do not influence the integral, or the integrability of the expression). We also have $\int_0^1 f(x)dx=\int_0^1 f(1-x)dx$, so the required answer is $f(x)=\frac{3\pi}{8}$. 
	
	\item[\textbf{A4}]Consider a $(2m-1)\times(2n-1)$ rectangular region, where $m$ and $n$ are integers such that $m,n\ge 4.$ The region is to be tiled using tiles of the two types shown:
	\[
	\begin{picture}(140,40)
	
	\put(0,0){\line(0,1){40}}
	\put(0,0){\line(1,0){20}}
	\put(0,40){\line(1,0){40}}
	\put(20,0){\line(0,1){20}}
	\put(20,20){\line(1,0){20}}
	\put(40,20){\line(0,1){20}}
	\multiput(0,20)(5,0){4}{\line(1,0){3}}
	\multiput(20,20)(0,5){4}{\line(0,1){3}}
	
	\put(80,0){\line(1,0){40}}
	\put(120,0){\line(0,1){20}}
	\put(120,20){\line(1,0){20}}
	\put(140,20){\line(0,1){20}}
	\put(80,0){\line(0,1){20}}
	\put(80,20){\line(1,0){20}}
	\put(100,20){\line(0,1){20}}
	\put(100,40){\line(1,0){40}}
	\multiput(100,0)(0,5){4}{\line(0,1){3}}
	\multiput(100,20)(5,0){4}{\line(1,0){3}}
	\multiput(120,20)(0,5){4}{\line(0,1){3}}
	
	\end{picture}
	\](The dotted lines divide the tiles into $1\times 1$ squares.) The tiles may be rotated and reflected, as long as their sides are parallel to the sides of the rectangular region. They must all fit within the region, and they must cover it completely without overlapping.
	
	What is the minimum number of tiles required to tile the region?
	
	\textbf{Answer.} $mn$. \\
	\textbf{Solution.} We first consider the region as $(i, j)$ with $1\le i\le 2m-1$ and $1\le j\le 2n-1$. We also label each square as types 1, 2, 3, 4 accoridng to the following rule: 
	\begin{enumerate}
		\item Type 1: Both $i, j$ odd.
		\item Type 2: $i$ even, $j$ odd. 
		\item Type 3: $i$ odd, $j$ even. 
		\item Type 4: Both $i, j$ even.
	\end{enumerate}
	Now there are $mn$ type 1 cells, $(m-1)n$ type 2 cells, $m(n-1)$ type 3 cells, and $(m-1)(n-1)$ type 4 cells. 
	
	Now name the first type of tile as 3-tile and the second type of tile as 4-tile. We first show that each tile covers cells of different type. Consider the 3-tile, and suppose the cell of the tile where its opposite is empty (in our example, it is the top left corner) covers the cell $(i, j)$. The other cells that are being covered are $(i\pm 1, j)$ and $(i, j\pm 1)$. These two cells do not have the same type as $(i, j)$ since for each of them, it cannot happen that both cooridnates have the same parity as that of $(i, j)$. $(i\pm 1, j)$ and $(i, j\pm 1)$ also have different types since $i$ and $i\pm 1$ must have different parity (same for $j$ and $j\pm 1$). For the 4-tile, we argue by considering the Manhattan distance of each cell of the tile. Two cells of the same type must have even distance in both coordinates, and hence an even Manhattan distance. Here in this 4-tile, the only two pairs of cells with even Manhattan distance are diasonally apart, with distance 1 in each of the two coordinates (for both pairs). Hence they cannot have the same type either. 
	
	Having established the above, we know that each 4-tile covers each type of cell exactly once, and each 3-tile covers 3 of the types of the cells exactly once. Let $a$ be the number of 4-tiles, and $b_1+b_2+b_3+b_4=b$ be the number of 3-tiles, with $b_i$ signifying the number of 3-tiles that do not cover any cell of type $i$. Therefore we have the following: $4a+3b=(2m-1)(2n-1)$, and 
	$a+b-b_1=mn, a+b-b_2=(m-1)n, a+b-b_3=m(n-1)$, and $a+b-b_4=(m-1)(n-1)$. Thus we know that $b_2-b_1=n, b_3-b_1=m$ and $b_4-b_1=m+n-1$. This forces $b=b_1+b_2+b_3+b_4\ge n+m+n+m-1=2(m+n)-1$, and so $4(a+b)=4a+3b+b\ge (2m-1)(2n-1)+2(m+n-1)=4mn$ and we have the number of tiles is $a+b$, which is at least $mn$. 
	
	\item[\textbf{A5}]Suppose that $G$ is a finite group generated by the two elements $g$ and $h,$ where the order of $g$ is odd. Show that every element of $G$ can be written in the form
	\[g^{m_1}h^{n_1}g^{m_2}h^{n_2}\cdots g^{m_r}h^{n_r}\]with $1\le r\le |G|$ and $m_n,n_1,m_2,n_2,\dots,m_r,n_r\in\{1,-1\}.$ (Here $|G|$ is the number of elements of $G.$)
	
	\textbf{Solution.} We first let $S$ to be the set of elements in $G$ that can be written in the desired form, but with the condition $1\le r\le G$ relaxed. To start with, $gh, g^{-1}h, gh^{-1}, g^{-1}h^{-1}\in S$. We also have $S$ closed in multiplication. 
	Now if $x=g^{m_1}h^{n_1}g^{m_2}h^{n_2}\cdots g^{m_r}h^{n_r}\in S$ then $x^{-1}=(g^{m_1}h^{n_1}g^{m_2}h^{n_2}\cdots g^{m_r}h^{n_r})^{|G|-1}\in S$ too, so we have $(g^{-1}h)^{-1}=h^{-1}g\in S$, and thus $ghh^{-1}g=g^2\in S$. Since the order of $G$ is odd, we also have $g=(g^2)^{\frac{ord(g)+1}{2}}\in S$, and $gg^{-1}h=h\in S$. Since $g$ and $h$ generate $G$, all elements in $G$ are in $S$. 
	
	It remains to show that the restriction $1\le r\le |G|$ can be imposed. For the identity element $e$ we observe that $(gh)^|G|=e$ (just to consider the possibility of $r=0$). Otherwise, if $x=g^{m_1}h^{n_1}g^{m_2}h^{n_2}\cdots g^{m_r}h^{n_r}$ with $r>|G|$, and suppose that this $r$ is the minimal number of index needed to represent $x$ in our desired form, then considering the element $x_k=g^{m_1}h^{n_1}g^{m_2}h^{n_2}\cdots g^{m_k}h^{n_k}$, and by the fact that $r>|G|$, we have $x_i=x_j$ for some $1\le i\neq j\le r$. This way, we also have 
	$x=g^{m_1}h^{n_1}g^{m_2}h^{n_2}\cdots g^{m_i}h^{n_i}g^{m_{j+1}}h^{n_{j+1}}\cdots g^{m_r}h^{n_r}$, contradicting the minimality of $r$. 
	
	\item[\textbf{B2}]Define a positive integer $n$ to be squarish if either $n$ is itself a perfect square or the distance from $n$ to the nearest perfect square is a perfect square. For example, $2016$ is squarish, because the nearest perfect square to $2016$ is $45^2=2025$ and $2025-2016=9$ is a perfect square. (Of the positive integers between $1$ and $10,$ only $6$ and $7$ are not squarish.)
	
	For a positive integer $N,$ let $S(N)$ be the number of squarish integers between $1$ and $N,$ inclusive. Find positive constants $\alpha$ and $\beta$ such that
	\[\lim_{N\to\infty}\frac{S(N)}{N^{\alpha}}=\beta,\]or show that no such constants exist.
	
	\textbf{Answer.}$\alpha=\frac 34, \beta = \frac 43$\\
	\textbf{Solution.} 
	We first consider the number of squarish numbers between 1 and $n^2$. Consider the number between $(k-1)^2+1$ to $k^2$ for some $k$, inclusive. For each of the numbers $m\in [(k-1)^2+1, k(k-1)]$, the closest square is $(k-1)^2$. For each of the numbers $m\in [k(k-1)+1, k^2]$, the closest square is $k^2$. In the first category, the distance from $(k-1)^2$ is $1, 2, \cdots , k-1$, so there are $\lfloor \sqrt{k - 1}\rfloor$ squarish numbers. In the second category, the distance to $k^2$ is $k-1, k-2, \cdots , 0$, so the number of squarish numbers is then $\lfloor \sqrt{k - 1}\rfloor + 1$. Therefore we have the sum as 
	\[S(n^2)=\sum_{k=1}^n 2\lfloor \sqrt{k - 1}\rfloor + 1\]
	The next thing is to evaluate the expression $\sum _{k=1}^n \lfloor{\sqrt{k-1}}\rfloor$. 
	Again we use the following inequality: 
	$\sum _{k=1}^n (\sqrt{k-1}-1)<\sum _{k=1}^n \lfloor{\sqrt{k-1}}\rfloor\le \sum _{k=1}^n \sqrt{k-1}$. Since $\sqrt{x}$ is also an increasing function, we have the following: 
	\[\int_0^{n-1}\sqrt{x}dx\le\sum _{k=1}^n \sqrt{k-1}=\sum _{k=0}^{n-1} \sqrt{k}\le\int_0^{n}\sqrt{x}dx\]
	(we are using the fact that $\int_{k-1}^{k}f(x)dx\le f(k)\le \int_{k}^{k+1} f(x)dx$ for $f$ increasing, so $\int_{0}^{n-1}f(x)dx\le \sum_{k=0}^{n-1} f(k)\le \int_{0}^{n} f(x)dx$, given that $f(0)=0$ for $f(x):=\sqrt{x}$). By evaluating the intergral $\int_0^{n}\sqrt{x}dx=\frac{2}{3}n^{3/2}$, we have
	\begin{flalign*}
	\frac{4}{3}(n-1)^{3/2} - n&\le (\sum _{k=1}^n 2\sqrt{k-1}) - n\\
	&= \sum _{k=1}^n (2(\sqrt{k-1}-1)+1)\\
	&<\sum _{k=1}^n (2\lfloor{\sqrt{k-1}}\rfloor+1)\\
	&\le \sum _{k=1}^n (2\sqrt{k-1}+1)\\
	&=n+2\sum _{k=1}^n \sqrt{k-1}\\
	&=\frac{4}{3}n^{3/2} + n\\
	\end{flalign*} 
	meaning that $S(n^2)\in[\frac{4}{3}(n-1)^{3/2} - n, \frac{4}{3}n^{3/2} + n]$. 
	Also notice that $S(n)$ is increasing with $n$ (we need it in the rest of the proof). 
	If we consider $N$ in general, then $\lfloor\sqrt{N}\rfloor^2\le N\le \lceil\sqrt{N}\rceil$. Thus 
	\[\frac{4}{3}(\lfloor\sqrt{N}\rfloor-1)^{3/2} - \lfloor\sqrt{N}\rfloor\le S(\lfloor\sqrt{N}\rfloor)\le S(n) \le S(\lceil\sqrt{N}\rceil)\le \frac{4}{3}(\lceil\sqrt{N}\rceil)^{3/2} + \lceil\sqrt{N}\rceil\]
	again we note that $\lfloor\sqrt{N}\rfloor>\sqrt{N}-1$ and $\lceil\sqrt{N}\rceil<\sqrt{N}+1$. 
	Thus we get 
	$\frac{4}{3}(\sqrt{N}-2)^{3/2} - \sqrt{N} \le S(N)\le \frac{4}{3}(\sqrt{N} + 1)^{3/2} + \sqrt{N} + 1$. Notice also that \[\lim_{N\to\infty}\frac{\frac{4}{3}(\sqrt{N}-2)^{3/2} - \sqrt{N}}{N^{3/4}}=\frac 43 = \lim_{N\to\infty}\frac{\frac{4}{3}(\sqrt{N} + 1)^{3/2} + \sqrt{N} + 1}{N^{3/4}}\], so the limit must exist by Squueze's theorem and equal to $\frac 43$, as desired. 
	
	\item[\textbf{B4}]Let $A$ be a $2n\times 2n$ matrix, with entries chosen independently at random. Every entry is chosen to be $0$ or $1,$ each with probability $1/2.$ Find the expected value of $\det(A-A^t)$ (as a function of $n$), where $A^t$ is the transpose of $A.$
	
	\textbf{Answer.}\\
	\textbf{Solution.} Denote $a_{ij}$ as the $(i, j)$-th entry of $A$, and $b_{ij}$ as the $(i, j)$-th entry of $A-A^t$. Notice that $b_{ij}=a_{ij}-a_{ji}$, which has $\frac 12$ chance of being zero, and $\frac 14$ chance of being -1, and $\frac 14$ chance of being 1, if $i\neq j$. If $i=j$ then $b_{ij}=0$ at all times. 
	
	Now consider $\det(A-A^t)=\sum_{\sigma\in S}\sgn(\sigma)\prod_{i=1}^{2n}b_{i\sigma(i)}$ where $S$ is the set of permutations of $\{1, 2, \cdots , n\}$, $\sigma$ is its permutation, and $\sgn=1$ if $\sigma$ is even, and $-1$ otherwise. The task is to find $E(\sum_{\sigma\in S}\sgn(\sigma)\prod_{i=1}^{2n}b_{i\sigma(i)})$, whixh is equal to $\sum_{\sigma\in S}\sgn(\sigma)E(\prod_{i=1}^{2n}b_{i\sigma(i)})$ by the linearity of expectation. Hence we can go ahead and investigate $E(\prod_{i=1}^{2n}b_{i\sigma(i)})$ individually for each of the $(2n)!$ permutations $\sigma$. We note the following: 
	\begin{enumerate}
		\item If $\sigma(i)=i$ for some $i$, then $b_{i\sigma(i)}=0$ at all times, so $\prod_{i=1}^{2n}b_{i\sigma(i)}=0$, and the expected value of this term is 0. 
		\item If $\sigma(\sigma(k))\neq k$ for some $k$, then the factor $b_{\sigma(k)k}$ is not present in the term $\prod_{i=1}^{2n}b_{i\sigma(i)}$. Recall that $b_{ij}$ is dependent with $b_{kl}$ if and only if $\{i, j\}\neq \{k, l\}$, so $b_{k\sigma(k)}$ is indepdendent from $b_{i\sigma(i)}$ for all $i\neq k$, and is thus independent to $\prod_{i=1, i\neq k}^{2n}b_{i\sigma(i)}$. 
		Now by the independence of $b_{k\sigma(k)}$ from the rest of the terms we get 
		\[E(\prod_{i=1}^{2n}b_{i\sigma(i)})=E(b_{k\sigma(k)})E(\prod_{i=1, i\neq k}^{2n}b_{i\sigma(i)})
		=0E(\prod_{i=1, i\neq k}^{2n}b_{i\sigma(i)})=0\]
		since $E(b_{ij})=0(\frac 12)+1(\frac 14)-1(\frac 14)=0$ for all $i, j$. Hence this gives a 0 expectation too. 
		\item Finally, assume that the two scenarios above do not happen, so for all $i$ we have $\sigma(i)\neq i$ but $\sigma(\sigma(i))=i$. This means that for each $i$, there are $\frac 12$ chance where $b_{i\sigma(i)}=b_{\sigma(i)i}=0$, $\frac 14$ chance when $b_{i\sigma(i)}=-b_{\sigma(i)i}=1$, and $\frac 14$ chance when $b_{i\sigma(i)}=-b_{\sigma(i){i}}=-1$. Now in the first case the product $b_{i\sigma(i)}b_{\sigma(i)i}=0$ while in the second and third case this product is $-1$, so the expected value of this product is $-\frac 12$. Since each such product is independent of all other $b_{j\sigma{j}}$ and $b_{\sigma{j}j}$, the expectation is then 
		\[E(\prod_{i=1}^{2n}b_{i\sigma(i)})=\prod E(b_{i\sigma(i)}b_{\sigma(i)i})=\left(-\frac 12\right)^n\]
	\end{enumerate}
    The task now is to consider all permutations falling into the third category. First, this gives rise to $n$ distinct orbits, so the parity of permutation is congruent to $2n-n=n\pmod{2}$, and thus $\sgn(\sigma)=(-1)^n$ for all such $\sigma$. Second, the number of such permutations depends on the pairing of the $2n$ numbers, so this is exactly the ways to split these numbers into pairs. In general we have the number of pairs as: 
    \[\frac{\displaystyle\prod_{i=1}^n\dbinom{2i}{2}}{n!}=\dfrac{(2n)!}{2^n n!}\]
    Hence the final expected value is $(\dfrac 12)^n \cdot \dfrac{(2n)!}{2^n n!}=\dfrac{(2n)!}{4^n n!}$. 
    
    \item[\textbf{B5}] Find all functions $f$ from the interval $(1,\infty)$ to $(1,\infty)$ with the following property: if $x,y\in(1,\infty)$ and $x^2\le y\le x^3,$ then $(f(x))^2\le f(y) \le (f(x))^3.$ 
    
    \textbf{Answer.} $f(x)=x^c$ for any constant $c>0$. \\
    \textbf{Solution.} We first show that the aforementioned functions satisfy the problem condition. Indeed, given that $x^2\le y\le x^3$ for any pairs of $x, y\in (1, \infty)$, from $x^2\le y$ and $c>0$ we have $(x^{2})^c\le y^c$, and from $y\le x^3$ we have $y^c\le (x^3)\le c$, as desired. 
    
    To show that there's no other suitable functions, denote $g:(0, \infty)\to\infty$ by $g(x)=\log f(e^x)$. Then the condition now becomes: if $e^{2x}\le 3^y\le e^{3x}$ then $2\log(f(e^x))\le \log(f(e^y))\le 3\log(f(e^x))$. In other words, if $2x\le y\le 3x$ then $2g(x)\le g(y)\le 3g(x)$. Notice that if a function $g$ is such a solution, then for any constant $c>0$, $cg$ is also a solution. Hence we can assume that $g(1)=1$. 
    
    We now show that $g(x)=x$ for all $x>0$. Before that, let's show a lemma: 
    
    \emph{Lemma (a)}. Let $4\le x\le 3^n$. Then $f(x)\le 3^nf(1)$. \\
    Proof: Let $k$ be the integer satisfying $3^{k-1}<x\le 3^k$ with $k\le n$. Then $3^{\frac{k-1}{k}}<\sqrt[k]{x}\le 3$. If $k\ge 3$ then $3^{\frac{k-1}{k}}>3^{\frac{2}{3}}=\sqrt[3]{9}>2$ and if $k=2$ then $4\le x\le 9$ and $2\le \sqrt{x}\le 3$. In either case, $2\le \sqrt[k]{x}\le 3$. This gives 
    \[
    f(x)=f(x^{\frac{1}{k}}x^{\frac{k-1}{k}})\le 3f(x^{\frac{k-1}{k}})
    =3f(x^{\frac{1}{k}}x^{\frac{k-2}{k}})\le 3^2f(x^{\frac{k-2}{k}})
    \le\cdots \le 3^kf(x)\le 3^nf(x)
    \]
    as desired (since $k\le n$). 
    
    \emph{Lemma (b)}. Let $x\ge 2^n$ with $n\ge 2$. Then $f(x)\ge 2^nf(1)$. \\
    Proof: the trick is essentially the same: let $k$ be the integer satisfying $2^k\le x<2^{k+1}$, then $2\le \sqrt[k]{x}<2^{1+1/k}\le 2^{\frac{3}{2}}=\sqrt{8}<3$. This means 
    \[
    f(x)=f(x^{\frac{1}{k}}x^{\frac{k-1}{k}})\ge 2f(x^{\frac{k-1}{k}})
    =2f(x^{\frac{1}{k}}x^{\frac{k-2}{k}})\ge 2^2f(x^{\frac{k-2}{k}})
    \le\cdots \le 2^kf(x)\ge 2^nf(x)
    \]
    since $k\ge n$ in this case. 
    
    To solve this problem, we also need another lemma: 
    
    \emph{Lemma 2}. Let $x$ be an irrational number, then $\{\{nx\}: n = 1, 2, \cdots \}$ is dense in $(0, 1)$ (i.e., the fractional part of the number). \\
    Proof: Let $(a, b)\subseteq (0, 1)$ be an interval of length $\varepsilon>0$, and let $N>\frac{1}{\varepsilon}$ be an integer. 
    Consider the $\{\{nx\}: n = 1, 2, \cdots , N+1\}$. Since $x$ is irrational, all the $\{nx\}$ are different. By pigeonhole principle, there exists $m, n\le N+1$ such that $0<|\{mx\}-\{nx\}|<\varepsilon$. From here, we can conclude that $0<\{(m-n)x\}<\varepsilon$. 
    Consider, now, any $k\ge 1$ and the quantity $\{kx\}$. If $\{kx\}\in (a, b)$ then we are done. Otherwise, we first consider the case where $m>n$. If $\{kx\}\le a$ let $\ell$ be the minimum positive integer such that $\{kx\}+\ell\{(m-n)x\}>a$. Then since $\{(m-n)x\}<\varepsilon$, the minimality of $\ell$ suggests that $\{kx\}+\ell\{(m-n)x\}< a + varepsilon=b$, so $\{(k + \ell(m-n))x\}\in (a, b)$. 
    Otherwise, $\{kx\}\ge b$, in which case we have to consider $\ell$ as the minimum positive integer such that $\{kx\}+\ell\{(m-n)x\}> 1 + a$. Since a similar analysis shows that $1+a< \{kx\}+\ell\{(m-n)x\} < 1+b$, we have $\{(k + \ell(m-n))x\}\in (a, b)$. 
    This is valid since $m>n$ and therefore $k + \ell(m-n)>0$. 
    In the case where $m<n$, if $\{kx\}\le b$ then let $\ell$ be the minimum positive integer such that $\{kx\}-\ell\{(m-n)x\}<b$. By the similar proof before we would have $\{kx\}-\ell\{(m-n)x\} > a$ too, so $\{(k + \ell(n-m))x\}\in (a, b)$. 
    Otherwise, $\{kx\}\le a$ and let $\ell$ be the minimum positive integer such that $\{kx\}-\ell\{(m-n)x\} < -(1-b)$. By the similar proof before we would have $\{kx\}-\ell\{(m-n)x\} > -(1-a)$ so $-(1-a)<\{kx\}-\ell\{(m-n)x\} < -(1-b)$ and hence $\{(k + \ell(n-m))x\}\in (a, b)$, as desired. 
    Notice we actually showed something slightly stronger (well not quite) -- for each open interval $(a, b)\in (0, 1)$ and $k\ge 1$ there exists $\ell\in k$ such that $\{\ell x\}\in (a, b)$. 
    
    We are ready to go back to the proof now, which we assumed $g(1)=1$. Suppose there exists $x_0$ with $g(x_0)>x_0$. Let $g(x_0)=cx_0$ with $c>1$. If $n, m\ge 2$ are such that $2^nx_0\le 3^m$, then choosing $y\in [2^nx_0, 3^m]$ and by lemma 1 we have $g(y)\ge 2^ng(x_0)=2^ncx_0$ and $g(y)\le 3^mg(1)=3^m$. Therefore, $2^ncx_0\le g(y)\le 3^m$, or $2^ncx_0\le 3^m$. This means that whenever $n, m\ge 2$ are integers such that $2^nx_0\le 3^m$ (or $n\log 2 + \log x_0\le m\log 3$) we have $2^nx_0c\le 3^m$ (or $n\log 2 + \log x_0 + \log c \le m\log 3$). This is to mean that there cannot exist $m, n\ge 2$ such that $m\log 3 - n\log 2\in [\log x_0, \log x_0 + \log c]$. 
    Let $q$ be an integer that is greater than 2, $2\log 3/log 2$ and $2\log 2/log 3$. Given that $\log 2/\log 3$ is irrational, 
		
\end{enumerate}

\section{Putnam 2015}
\begin{enumerate}
	\item [\textbf{A1}] Let $A$ and $B$ be points on the same branch of the hyperbola $xy=1.$ Suppose that $P$ is a point lying between $A$ and $B$ on this hyperbola, such that the area of the triangle $APB$ is as large as possible. Show that the region bounded by the hyperbola and the chord $AP$ has the same area as the region bounded by the hyperbola and the chord $PB.$
	
	\textbf{Solution.} Let the coordinates of $A$ to be $(x_A, \frac 1{x_A})$ and the coordinates of $B$ to be $(x_B, \frac 1{x_B})$. Same goes for $(x_P, \frac 1{x_P})$. Thus the area is given by: 
	\[\frac 12 |\frac{x_A}{x_B} +\frac{x_B}{x_P} + \frac{x_P}{x_A} - \frac{x_B}{x_A} -\frac {x_P}{x_B} -\frac {x_A}{x_P}|\]
	Ignoring absolute value, differentiating with respect to $x_P$ we get that any stationary point happens when $(x_B-x_A)\left(\frac{1}{x_P^2}-\frac{1}{x_Ax_B}\right)=0$. This happens when $x_P=\pm\sqrt{x_Ax_B}$. W.l.o.g. we assume that both $x_A$ and $x_B$ are both positive, and thus $x_P$ must also be positive. Thus $x_P=\sqrt{x_Ax_B}$. Also w.l.o.g. we assume that $x_A<x_P<x_B$. Since the area is the lowest possible (i.e. 0) when $x_P=x_A$ or $x_P=x_B$, and positive at other times, and also since $x_P=\sqrt{x_Ax_B}$ is the only stationary point, this area must be nondecreasing in the interval $x_P\in (x_A, \sqrt{x_Ax_B})$  and nonincreasing in the interval $x_P\in (\sqrt{x_Ax_B}, x_B)$, we know that $x_P=\sqrt{x_Ax_B}$ is indeed the point where the area attains the maximum. Now the area of bounded by the hyperbola and the chord $AP$ is given by the following: 
	\[\frac 12 (x_P-x_A)\left(\frac{1}{x_P}+\frac{1}{x_A}\right)-\int_{x_A}^{x_P}\frac{1}{x}dx
	=\frac 12 \left(\frac{x_P}{x_A}-\frac{x_A}{x_P}\right)-(\ln x_P -\ln x_A)
	\]
	substituting $x_P=\sqrt{x_Ax_B}$ we get 
	\[\frac 12 \left(\frac{\sqrt{x_Ax_B}}{x_A}-\frac{x_A}{\sqrt{x_Ax_B}}\right)-(\ln \sqrt{x_Ax_B} -\ln x_A)
	=\frac 12 \left(\sqrt{\frac{x_B}{x_A}}-\sqrt{\frac{x_A}{x_B}}\right)-\frac 12(\ln x_B-\ln x_A)
	\]
	Similarly the area bounded by $PB$ and the hyperbola is given by 
	\[\frac 12 (x_B-x_P)\left(\frac{1}{x_P}+\frac{1}{x_B}\right)-\int_{x_P}^{x_B}\frac{1}{x}dx
	=\frac 12 \left(\frac{x_B}{x_P}-\frac{x_P}{x_B}\right)-(\ln x_B -\ln x_P)
	\]
	and since $x_P=\sqrt{x_Ax_B}$ we get 
	\[\frac 12 \left(\frac{x_B}{\sqrt{x_Ax_B}}-\frac{\sqrt{x_Ax_B}}{x_B}\right)-(\ln x_B -\ln \sqrt{x_Ax_B})
	=\frac 12 \left(\sqrt{\frac{x_B}{x_A}}-\sqrt{\frac{x_A}{x_B}}\right)-\frac 12(\ln x_B-\ln x_A)
	\]
	hence showing that they have the same area. 
	
	\item[\textbf{A2}]Let $a_0=1,a_1=2,$ and $a_n=4a_{n-1}-a_{n-2}$ for $n\ge 2.$
	
	Find an odd prime factor of $a_{2015}.$
	
	\textbf{Answer.} 181. \\
	\textbf{Solution.} The characteristic polynomial of this recurrence equation is $x^2-4x+1=0$, which has roots $\frac{4\pm\sqrt{4^2-4}}{2}=2\pm\sqrt{3}$. Thus $a_n=a(2+\sqrt{3})^n+b(2-\sqrt{3})^n$, and since $a+b=1$ and $a(2+\sqrt{3})+b(2-\sqrt{3})=2$, we get $a=b=\frac 12$. Thus we have $a_n=\frac 12 ((2-\sqrt{3})^n + (2+\sqrt{3})^n)$. 
	Now $\frac{a_{2015}}{a_5}=\sum_{i=0}^{402}(-1)^{i}(2-\sqrt{3})^{5i}(2-\sqrt{3})^{2010+5i}$, and since both $a_{2015}$ and $a_5$ are both integers, this expression must also be a rational number. From the right hand side we can also deduce that this ratio is in the form of $x+y\sqrt{3}$ with $x, y$ both integers and since $x+y\sqrt{3}\in \mathbb{Q}$, $y=0$ hence $\frac{a_{2015}}{a_5}$ is actually an integer. So it suffices to find a prime factor of $a_5$. Finally, since $a_5=362=2\times 181$ and 181 is a prime, this is a possible answer. 
	
	\item[\textbf{A3}]Compute \[\log_2\left(\prod_{a=1}^{2015}\prod_{b=1}^{2015}\left(1+e^{2\pi iab/2015}\right)\right)\]Here $i$ is the imaginary unit (that is, $i^2=-1$).
	
	\textbf{Answer.} \\
	\textbf{Solution.} Since $e^x = e^{2\pi i + x}$, we will consider everything in the cycle of $2\pi i$. In this context, if $ab\equiv k\pmod{2015}$, and let $ab-k=2015c$ with $c$ an integer, then $e^{2\pi iab/2015}=e^{2\pi i(k+2015c)/2015}=e^{2\pi ik/2015}e^{2\pi ic}=e^{2\pi ik/2015}$. Thus we can consider everything modulo 2015. 
	
	Let $d=\gcd(a, 2015)$. Then $2015|ab$ if and only if $c=\frac{2015}{d}|b$. In addition, $\{a, 2a, \cdots ca\} = \{d, 2d, \cdots , cd=2015\}$ in modulo 2015. Thus we have 
	\[\prod_{b=1}^{2015}\left(1+e^{2\pi iab/2015}\right)=\left(\prod_{b=1}^{c}\left(1+e^{2\pi ibd/2015}\right)\right)^d=\left(\prod_{b=1}^{e}\left(1+e^{2\pi ib/c}\right)\right)^d\]
	Bearing in mind that $c$ is odd, we now investigate this sum. Now, it is given that $\prod_{b=1}^{c}\left(x-e^{2\pi ib/c}\right)=x^c-1$, since $e^{2\pi b/c}$ are all the roots of unity for $b=1, 2, \cdots , c$. Substituting $c=-1$ we get $\prod_{b=1}^{e}\left(-1-e^{2\pi ib/c}\right)=x^c-1=-1-1=-2$ since $c$ is odd. 
	Reversing the sign we get $\prod_{b=1}^{e}\left(1+e^{2\pi ib/c}\right)=(-2)(-1)^{c}=2$. Therefore we have $\prod_{b=1}^{2015}\left(1+e^{2\pi iab/2015}\right)=2^d$. Summing up we get 
	\[\log_2\left(\prod_{a=1}^{2015}\prod_{b=1}^{2015}\left(1+e^{2\pi iab/2015}\right)\right) =\log_2\left(\prod_{a=1}^{2015}2^{\gcd(2015, a)}\right)
	=\sum_{a=1}^{2015} \gcd(2015, a)\]
	By the Euler's totient function, there are $\phi(2015)=\phi(5\cdot 13\cdot 31)=4\cdot 12\cdot 30=1440$ such $a$'s with $\gcd(a, 2015)=1$. The number of $a$'s with $\gcd(a, 2015)=d$ is $\phi(\frac{2015}{d})$, so this gives the total as 
	\begin{flalign*}
	\sum_{a=1}^{2015} \gcd(2015, a)&=\sum_{d|2015} \phi(d)\frac{2015}{d}\\
	&=1440+4(12\cdot 30)+13(4)(30)+5(13)(30)+4(13)(31)+5(12)(31)+5(13)(30)+5(13)(31)\\
	&=13725\\
	\end{flalign*}

	\item[\textbf{A5}] Let $q$ be an odd positive integer, and let $N_q$ denote the number of integers $a$ such that $0<a<q/4$ and $\gcd(a,q)=1.$ Show that $N_q$ is odd if and only if $q$ is of the form $p^k$ with $k$ a positive integer and $p$ a prime congruent to $5$ or $7$ modulo $8.$
	
	\textbf{Solution.} We first eliminate the case where $q=pr$ with $p>1, r>1$ and $\gcd(p, r)=1$. First w.l.o.g. (to make our computations easier) that $r$ is a prime power, say $s^k$. We first calculate the number $M_p$ of integers $a$ with $0<a<q/4=pr/4$ and $\gcd(a, p)=1$. Notice that $a<pr/4$ if and only if $a/p<r/4$. Consider the numbers in the intervals $[1, p], [p+1, 2p], \cdots , [(d-1)p+1, dp]$ where $d=\lfloor r/4 \rfloor$. Each number mentioned is less than $q/4$, and in each category, there are $\phi(p)$ numbers relatively prime to $p$. So these sets contributed $\phi(p)\cdot d$ to $M_p$, which is even since $\phi(p)$ is always even for $p>2$. It remains to investigate contribution of the interval $[dp, (d+1)p]$ to $M_q$. Now $dp+k<q/4$ if and only if $k<(r/4 - \lfloor r/4 \rfloor)p$. If $r\equiv 1\pmod{4}$ then the bound is $p/4$, in which case the contribution is precisely $N_p$. Otherwise, the bound is $3p/4$, and the contribution is precisely $\phi(p)-N_p$. Thus $M_q\equiv N_p\pmod{2}$, as always. 
	
	To investigate the relation between $M_q$ and $N_q$, we note that if a number counts into $N_q$, then it counts into $M_q$. Conversely, an integer $a$ counts into $M_q$ but not $N_q$ if and only if $a=st$ with $\gcd(t, p)=1$ and $t<q/4s=ps^{k-1}/4$. To count the number of such $t$, we notice that the number of such $t$ with $t\le p\lfloor s^{k-1}/4\rfloor$ is $\lfloor s^{k-1}/4\rfloor\phi(p)$, which is again even. As of the case above, it remains to consider the contribution of such $t$ in the next set of $p$ numbers. Similar to above, if $s^{k-1}\equiv 1\pmod{4}$ then this contribution is $N_p$, and if $s^{k-1}\equiv 3\pmod{4}$ then this contribution is $\phi(p)-N_p$. In either case it's congruent to $N_p\pmod{2}$. Thus $N_q\equiv M_q-N_p\equiv N_p-N_p=0\pmod{2}$. 
	
	Thus the case of $q$ having more than two primes have been eliminated. If $q=1$ then $N_1=0$, which serves as an edge case. If $q=p^k$ with $k\ge 1$, then $N_q$ is the number of the integers between 1 and $\lfloor q/4\rfloor$ minus the number of integers in this range and divisible by $p$. This gives the bound $\lfloor (p^k)/4\rfloor-\lfloor (p^{k-1})/4\rfloor$. Letting $p^{k-1}=4\ell+a$ with $a\in\{1, 3\}$ we get $\lfloor (p^{k-1})/4\rfloor=\ell$ and $\lfloor (p^k)/4\rfloor=\lfloor p(4\ell+a)/4\rfloor=\ell p+\lfloor ap/4\rfloor$. Since $p$ is odd we have $\lfloor (p^k)/4\rfloor-\lfloor (p^{k-1})/4\rfloor=\ell(p-1)+\lfloor ap/4\rfloor\equiv \lfloor ap/4\rfloor\pmod{2}$. If $a=1$, then $\lfloor p/4\rfloor$ is odd if and only if $p\equiv 5, 7\pmod{8}$. If $a=3$, then again writing $p=8c+d$ we get $\lfloor 3(8c+d)/4\rfloor=6c+\lfloor 3d/4\rfloor\equiv \lfloor 3d/4\rfloor\pmod{2}$ we only need to consider the cases where $d\in \{1, 3, 5, 7\}$, which gives the values $\lfloor 3/4\rfloor, \lfloor 9/4\rfloor, \lfloor 15/4\rfloor, \lfloor 21/4\rfloor=0, 2, 3, 5$. Hence only $p\equiv 5, 7\pmod{8}$ satisfies this condition. 
	
	\item[\textbf{B3}] Let $S$ be the set of all $2\times 2$ real matrices \[M=\begin{pmatrix}a&b\\c&d\end{pmatrix}\]whose entries $a,b,c,d$ (in that order) form an arithmetic progression. Find all matrices $M$ in $S$ for which there is some integer $k>1$ such that $M^k$ is also in $S.$
	
	\textbf{Answer.} $M=\begin{pmatrix}a&a\\a&a\end{pmatrix}$ and $M=\begin{pmatrix}-3a&-a\\a&3a\end{pmatrix}$ for any real number $a$. \\
	\textbf{Solution.} If $M=\begin{pmatrix}a&a\\a&a\end{pmatrix}$ then $M^2=\begin{pmatrix}a&a\\a&a\end{pmatrix}
	\begin{pmatrix}a&a\\a&a\end{pmatrix}
	=\begin{pmatrix}2a^2&2a^2\\2a^2&2a^2\end{pmatrix}
	$
	which is also in $S$. Hence now we only consider those $M$ with nonzero common difference. 
	
	First, consider $M=\begin{pmatrix}a&a+d\\a+2d&a+3d\end{pmatrix}$ with $d$ as the common difference, then the characteristic polynomial is $x^2-(2a+3d)x-2d^2$, which has discriminant $(2a+3d)^2+8d^2>0$. Hence $M$ has two real and distinct eigenvalues, which implies that $M$ is diagonalizable. Write $M=PDP^{-1}$ where $P$ is the matrix determined by $M$'s eigenvectors, and $D$ is the diagonal matrix symbolizing the eigenvalues. We proceed with the following claim: 
	
	\emph{Lemma}: If $M^k\in S$ with $k\ge 1$, then $M^k=cM$ for some constant $c$. \\
	\emph{Proof}: First, notice that $S$ is closed under matrix addition (that is, if $M_1$ and $M_2$ are both in $S$ then $aM_1+bM_2\in S$ for all constants $a$ and $b$). Next, we also have $M^k=(PDP^{-1})^k=PD^kP^{-1}$ with $D^k$ remains diagonal. Suppose that there exist real constants $a$ and $b$ such that $aD+bD^k=I$ with $I$ being the identity matrix. Then $aM+bM^k=P(aD)P^{-1}+P(bD^k)P^{-1}=PIP^{-1}=I$, which is not in $S$. So in this case, either $M$ or $M^k$ cannot be in $S$. This happens if the eigenvalues of $M$ and $M^k$, when each treated as a 2-dimensional vector, is linearly independent. That is, if $a, b$ are the eigenvalues of $M$, then $a^k$ and $b^k$ are the eigenvalues of $M^k$ and thus 
	$\begin{pmatrix}
	a & a^k \\ b & b^k
	\end{pmatrix}$
	is linearly independent. To have $M$ and $M^k$ both in $S$, this matrix
	$\begin{pmatrix}
	a & a^k \\ b & b^k
	\end{pmatrix}$
	must be linearly dependent, i.e. $ab^k-a^kb=0$, or $ab(a^{k-1}-b^{k-1})=0$. If $a=0$ or $b=0$, then $M$ has determinant 0, which implies that $-2d^2=\det(M)=0$, so the common difference is 0, contradiction (this case has been handled in the beginning of the proof). Hence we have $a^{k-1}=b^{k-1}$, which means $|a|=|b|$. The case where $a=b$ means $D=aI$ and so $M=aI\not\in S$, so $a=-b$. 
	
	Going back to the proof, we now know that the eigenvalues of $M$ are in the form of $e, -e$ for some $e$. This forces the characteristic polynomial of $M$ to be $x^2+e^2$, which also implies that $2a+3d=0$ in the beginning. Therefore we have $M=\begin{pmatrix}-3a&-a\\a&3a\end{pmatrix}$ for some real number $a$. To show that this is a valid example, $M^3=8a^3\begin{pmatrix}-3&-1\\1&3\end{pmatrix}$ which is indeed in $S$. 
	
	
	\item[\textbf{B5}]Let $P_n$ be the number of permutations $\pi$ of $\{1,2,\dots,n\}$ such that \[|i-j|=1\text{ implies }|\pi(i)-\pi(j)|\le 2\]for all $i,j$ in $\{1,2,\dots,n\}.$ Show that for $n\ge 2,$ the quantity \[P_{n+5}-P_{n+4}-P_{n+3}+P_n\]does not depend on $n,$ and find its value.
	
	\textbf{Answer.} This value is always 4. \\
	\textbf{Solution.} For each $n$ we denote $Q_n$ as the number of permutations satisfying the conditions $i-j|=1\text{ implies }|\pi(i)-\pi(j)|\le 2$ and $\pi(n)=n$. Fix $n$, and we consider the number of such permutations when $\pi(n)=k$ for each $k=1, 2, \cdots , n$. When $k=n$ this number is $Q_n$ as defined, and by symmetry this holds true when $k=1$. Hence we proceed to consider the cases when $\pi(n)=2, \cdots , n-1$. Now, denote $\pi(n)=k$ and we have we consider any $j$ satisfying $\pi(j)<k$ and $\pi(j+1)>k$. Since $\pi(j+1)-\pi(j)\le 2$, we must have $\pi(j)=k-1$ and $\pi(j+1)=k+1$. Similarly, if $\pi(j)>k$ and $\pi(j+1)<k$ then we must have $\pi(j)=k+1$ and $\pi(j+1)=k-1$. Since permutation is a bijection, exactly one of the above happens and exactly one $j$ satisfies this condition. Thus the numbers $\pi(1), \cdots , \pi(n-1)$ are partitioned into two consecutive regions, one with values $<k$ and the other $>k$. In other words exactly one of the following holds: $\pi(j)<k$ for all $1\le j\le k-1$ and $\pi(j)>k$ for all $k\le j\le n$, or $\pi(j)>k$ for all $1\le j\le n-k$, and $\pi(j)<k$ for $n-k+1\le j\le n-1$. In the first case, $\pi(k-1)$ must be equal to $k-1$ and $\pi(k)=k+1$, so this gives $Q_{k-1}$ ways to arrange $\pi(1), \cdots \pi(k-1)$. We now claim that there's only one way to arrange $\pi(k), \cdots , \pi(n-1)$ given the constraint. To begin with, since $\pi(n)=k$ and $\pi(k)=k+1$ and everything in between has $\pi(j)>k$, we have 
	$\pi(n-1)=k+2, \pi(k+1)=k+3$. Repeating the process gives a unique arrangement given by 
	$\pi(k), \cdots , \pi(n-1)=k+1, k+3, \cdots n-1, n, n-2, \cdots k+2$ for $k\equiv n\pmod{2}$, and $\pi(k), \cdots , \pi(n-1)=k+1, k+3, \cdots , n-2, n, n-1, \cdots , k+2$ otherwise. This gives a total of $Q_{k-1}$. For the second case, similarly, we have $Q_{n-k}$ ways of arranging the first $n-k$ permutation numbers, and exactly 1 way for the next $k-1$ numbers. Thus summing above and considering all $k$ we get, for all $n\ge 2$,  
	\[P_n=2Q_n+\sum_{i=2}^{n-1}Q_{k-1}+Q_{n-k}=2(Q_n+\sum_{i=1}^{n-2} Q_i)\]
	Now the desired value becomes 
	$2(Q_{n+5}+\sum_{i=1}^{n+3} Q_i )- 2(Q_{n+4}+\sum_{i=1}^{n+2} Q_i)-2(Q_{n+3}+\sum_{i=1}^{n+1} Q_i)+2(Q_n+\sum_{i=1}^{n-2} Q_i)
	=2(Q_{n+5}-Q_{n+4}-Q_{n+1}-Q_{n-1})
	$
	To calculate the above, we find an iterative formula for $Q_n$ for all $n\ge 4$. Since $\pi(n)=n$, we have $\pi(n-1)=n-1$ or $n-2$. For $\pi(n-1)=n-1$, $Q_{n-1}$ permutation arises as claimed above. For $\pi(n-1)=n-2$, we can use the argument above to establish that $Q_{n-3}+Q_1$ permutations arise. Thus $Q_n=Q_{n-1}+Q_{n-3}+Q_1$. This means that for all $n\ge 2$ we have 
	$2(Q_{n+5}-Q_{n+4}-Q_{n+1}-Q_{n-1})
	=2(Q_{n+4}+Q_{n+2}+Q_1-Q_{n+4}-Q_{n+1}-Q_{n-1})
	=2(Q_{n+1}+Q_{n-1}+Q_1+Q_1-Q_{n+1}-Q_{n-1})
	=4Q_1.
	$
	It's not hard to see that $Q_1=1$ so our desired answer must be 4. 
	
\end{enumerate}

\section{Putnam 2014}
\begin{enumerate}
	\item [\textbf{A1}]Prove that every nonzero coefficient of the Taylor series of $(1-x+x^2)e^x$ about $x=0$ is a rational number whose numerator (in lowest terms) is either $1$ or a prime number.
	
	\textbf{Solution.} We consider the expansion $(1-x+x^2)\displaystyle\sum_{i=0}^{\infty} \frac{1}{i!}x^i$. The coefficient of the constant term is 1 and the coefficient of the $x$-term is $\frac{1}{2}-1=-\frac{1}{2}$. The conefficient of $x^{k+1}$ for all $k\ge 1$ is given by $\frac{1}{(k+1)!}-\frac{1}{k!}+\frac{1}{(k-1)!}
	=\frac{1-(k+1)+k(k+1)}{(k+1)!}
	=\frac{k^2}{(k+1)!}
	=\frac{k}{(k-1)!(k+1)}
	$
	If $k$ is prime we are done. 
	Now assume that it's not. If $k$ is not a prime power, write $k=ab$ with $1<a, b<k$ and $\gcd(a, b)=1$ (for example, let $p$ to be a prime divisor of $k$ and let $r$ to be the maximum power of $p$ dividing $k$. Then since $k$ is not a prime power, $p^r<k$ and $k/(p^r)$ is relatively prime to $p^r$ by the maximality of $r$). Since $a, b<k$, $a|(k-1)!$ and $b|(k-1)!$ and with $\gcd(a, b)=1$, this implies that $k=ab|(k-1)!$. Otherwise, $k=p^r$ for some prime $p$ and $r\ge 2$. Using the formula $v_p((r)!)=\sum_{i=1}^\infty\lfloor\frac{r}{p^i}\rfloor$, we have 
	$v_p((k-1)!)=\sum_{i=1}^\infty\lfloor\frac{k-1}{p^i}\rfloor
	=\sum_{i=1}^\infty\lfloor\frac{p^r-1}{p^i}\rfloor
	=p^{r-1}-1+p^{r-2}-1+\cdots + (p-1)
	\ge 1+1+\cdots + 1
	= r-1
	$
	since $p\ge 2$. Thus $p^{r-1}|(k-1)!$ and so when taking the lowest term the numerator can either be 1 or $p$. 
	
	\item[\textbf{A2}]Let $A$ be the $n\times n$ matrix whose entry in the $i$-th row and $j$-th column is \[\frac1{\min(i,j)}\] for $1\le i,j\le n.$ Compute $\det(A).$
	
	\textbf{Answer.} $\frac{1}{n[(n-1)!]^2}$\\
	\textbf{Solution.} We use the well-known matrix identity that row reduction preserves determinant, and we will do row reduce profusely. For brevity, we will denote $f(i)=\frac{1}{i}$ for all $i\ge 1$. Denoting $a_{ij}$ as the $i$-th row and the $j$-th column. Then we have $a_{ij}=f(\min(i, j))$. 
	
	Now, for each iteration stepped $i$, denoting row $i$ as $r_i$ and we will do $r_{j}:=r_{j}-r_i$ for all $j\ge i$. We show that after the $k$-th iteration below would be the value for $a_{ij}$: 
	\begin{itemize}
		\item For $i=1$, we have $a_{ij}=f(1)$ as always. 
		\item For $i\le k$, we have $a_{ij}=0$ for all $j<i$, and $a_{ij}=f(i)-f(i-1)$ for all $j\ge i$. 
		\item For $i>k$, $a_{ij}=0$ for $j\le k$, and $a_{ij}=f(\min(i, j))-f(k)$ otherwise. 
	\end{itemize}
    To prove this by inducting on $k$, the base case is given when all the numbers after the first row are subtracted by the corresponding number in the first row, so for $i>1$, $a_{ij}$ becomes $a_{ij}=f(\min(i, j))-f(1)$, and the condition above is satisfied. Suppose that the conjecture holds after $k$-th step for some $k$. At $k+1$-th step, all rows after the $k+1$-th row is subtracted against the corresponding index in $k+1$-th row. The $k+1$-th row is given by the following: 
    \[\begin{pmatrix}0 & \cdots 0 & f(k+1)-f(k) & \cdots f(k+1)-f(k)\end{pmatrix}\]
    where the first $k$ entries are 0. Now after the $k+1$-th iteration, for all $i>k+1$, if $j\le k$ then $a_{ij}$ becomes $0-0=0$ and if $j>k$ we have $a_{ij}$ becomes $(f(\min(i, j))-f(k))-f(k+1)-f(k)=f(\min(i, j))-f(k+1)$. This entry is 0 if $j=k+1$ since $i>k+1$. For all $i\le k$ the rows are unaffected by this row reduction, so we still have $a_{ij}=0$ for all $j<i$, and $a_{ij}=f(i)-f(i-1)$ for all $j\ge i$. Thus the claim is proven. 
    
    To finish the proof, after $n-1$ iterations, we have, for all $j<i$, $a_{ij}=0$. Thus $A$ is no upper triangular, and the determinant is simply the product of the diagonal entries. We also have $a_{ii}=1$ for $i=1$ and $f(i)-f(i-1)$ for $i\ge 2$. Now $f(i)-f(i-1)=\frac{1}{i}-\frac{1}{i-1}=-\frac{1}{i(i-1)}$. Hence we have 
    \[\det(A)=\prod_{i=2}^n -\frac{1}{i(i-1)}= - (-1)^{n-1}=\frac{1}{n[(n-1)!]^2}\]
    
    \item[\textbf{A4}] Suppose $X$ is a random variable that takes on only nonnegative integer values, with $E[X]=1,$ $E[X^2]=2,$ and $E[X^3]=5.$ (Here $E[Y]$ denotes the expectation of the random variable $Y.$) Determine the smallest possible value of the probability of the event $X=0.$
    
    \textbf{Answer.} $\dfrac 13$. \\
    \textbf{Solution.} (by Kiran Kedlaya, modified) We let $a_i=P(X=i)$ for each $i\ge 0$ (which means $a_i\ge 0$ for each $i$). Consider the power series $f(x)=\displaystyle\sum_{n=0}^{\infty}a_nx^n$, and we have $f(1)=1$. 
    By the problem condition, we also have 
    \begin{itemize}
    	\item $f'(x)=\displaystyle\sum_{n=0}^{\infty}na_nx^{n-1}$, so $f'(1)=\displaystyle\sum_{n=0}^{\infty}na_n=E(X)=1$
    	
    	\item $f''(x)=\displaystyle\sum_{n=0}^{\infty}n(n-1)a_nx^{n-2}$, so
    	$f''(1)=\displaystyle\sum_{n=0}^{\infty}n(n-1)a_n
    	=\displaystyle\sum_{n=0}^{\infty}n^2a_n-\displaystyle\sum_{n=0}^{\infty}na_n
    	=E(X^2)-E(X)
    	=2-1=1
    	$
    	
    	\item $f'''(x)=\displaystyle\sum_{n=0}^{\infty}n(n-1)(n-2)a_nx^{n-3}$ so 
    	$f'''(1)=\displaystyle\sum_{n=0}^{\infty}(n^3-3n^2+2n)a_n
    	=E(X^3)-3E(X^2)+2E(X)
    	=5-3(2)+2(1)=1
    	$
    \end{itemize}  
	Now we can rearrange $f(x)$ into $f(x)=\displaystyle\sum_{n=0}^{\infty} f^{(n)}(1)\dfrac{(x-1)^n}{n!}$, i.e. the Taylor's series. We also have, by Taylor's series, $f(x)=f(1)+f'(1)(x-1)+f''(1)\dfrac{(x-1)^2}{2!}+f'''(1)\dfrac{(x-1)^3}{3!}+f^{(4)}(c)\dfrac{(x-1)^4}{4!}$, with $c$ some value in $(1, x)$ or $(x, 1)$ depending whether $x<1$ or $1<x$. Thus in particular $a_0=f(0)=f(1)-f'(1)+\dfrac{f''(1)}{2}-\dfrac{f'''(1)}{6}+\dfrac{f^{(4)}(c)}{24}
	=1-1+\dfrac{1}{2}-\dfrac{1}{6}+\dfrac{f^{(4)}(c)}{24}
	=\dfrac{1}{3}+\dfrac{f^{(4)}(c)}{24}
	$
	for some $c\in (0, 1)$. 
	We also note that $f^{(4)}(x)=\displaystyle\sum_{n=0}^{\infty}n(n-1)(n-2)(n-3)a_nx^{n-4}$ and for $x\ge 0$ and $n\ge 0$, the quantities $n(n-1)(n-2)(n-3)$, $a_n$, and $x^{n-4}$ are all nonnegative. Thus $f^{(4)}(x)\ge 0$ for all $x\ge 0$, and in particular $\dfrac{f^{(4)}(c)}{24}\ge 0$. Thus we have $f(0)=\dfrac{1}{3}+\dfrac{f^{(4)}(c)}{24}\ge \dfrac 13$, with equality holding when $a_0=\frac 13, a_1=\frac 12$ and $a_3=\frac 16$ and $a_n=0$ for other $n$'s. 
    
    \item[\textbf{B2}]Suppose that $f$ is a function on the interval $[1,3]$ such that $-1\le f(x)\le 1$ for all $x$ and $\displaystyle \int_1^3f(x)\,dx=0.$ How large can $\displaystyle\int_1^3\frac{f(x)}x\,dx$ be?
    
    \textbf{Answer.} $\ln\frac 43$. \\
    \textbf{Solution.} Equality can be attained by taking $f(j)=1$ for all $1\le j<2$ and $f(j)=-1$ for all $2\le j\le 3$. We show that this is the maximum by the following: if $g(x)$ is defined as $\displaystyle \int_1^xf(y)\,dy$, we have $g(1)=g(3)=0$. Also since $f(x)\in [-1, 1]$ for all $x\in [1, 3]$, and by Mean value theorem, we have , for every $x$ in the said interval, $g'(c)=f(c)=\frac{g(x)-g(1)}{x-1}$ for some constant $c$ in the interval $(1, x)$, so $|\frac{g(x)}{x-1}|\le 1$. Similarly $|\frac{g(x)}{x-3}|\le 1$. This means that $g(x)\le x-1$ and $g(x)\le 3-x$ must hold simultaneously. Using this fact and integrating by parts give: 
    \begin{flalign*}
    \int_1^3\frac{f(x)}{x}\,dx&=\frac{g(x)}{x}|_1^3 +\int_1^3\frac{g(x)}{x^2}\,dx\\
    &=(0-0)+\int_1^3\frac{g(x)}{x^2}\,dx\\
    &\le \int_1^2\frac{x-1}{x^2}\,dx+\int_2^3\frac{3-x}{x^2}\,dx\\
    &=[\ln x + \frac 1x]_1^2+[-\frac 3x-\ln x]_2^3\\
    &=\ln 2 -\ln 1+\frac 12 - 1 +\frac 32-1-\ln 3+\ln 2\\
    &=\ln \frac 43
    \end{flalign*}
    as desired. 
    
    \item[\textbf{B3}] Let $A$ be an $m\times n$ matrix with rational entries. Suppose that there are at least $m+n$ distinct prime numbers among the absolute values of the entries of $A.$ Show that the rank of $A$ is at least $2.$
    
    \textbf{Solution.} 
    By the theorem of unique prime factorization, if $p, q, r, s$ are prime numbers with $pq=rs$ then $p=r$, $q=s$ or $p=s, q=r$ (so the four numbers cannot be pairwise distinct). 
    The fact that there's at least one prime (and hence nonzero) number in $A$ implies that the rank of $A$ cannot be zero, so we can now assume that the rank of $A$ is 1, which is equivalent to assuming that there exists rational numbers $a_1, a_2, \cdots , a_m, b_1, b_2, \cdots , b_n$ such that $A_{ij}=a_ib_j$. 
    
    Now consider a graph $(V, E)$ with $n$ vertices. and we consider adding coloured edge by the following mechanism: for a row $i$, if $x_1<x_2<\cdots < x_k$ are the all the indices such that $A_{ix_j}$ are among the $m+n$ distinct prime numbers, then we add an edge coloured $i$ between $x_j$ and $x_{j+1}$ for each $1\le j\le k-1$. This means if row $i$ has $i_k$ prime numbers the there will be $i_k-1$ edges coloured $i$. Our colouring also ensures that there will be no monochromatic cycle in our graph, and there are at least $\displaystyle\sum_{k=1}^m (i_k-1)=(\displaystyle\sum_{k=1}^m i_k)-m=(m+n-m)=n$. 
    
    We first see what happens if there are two vertices $c_1, c_2$ with two edges coloured $k_1$ and $k_2$. This means $A_{k_ic_j}=a_{k_i}b_{c_j}$ are all prime numbers for all combinations of $i\in \{1, 2\}$ and $j\in \{1, 2\}$. Notice also that $A_{k_1c_1}A_{k_2c_2}=a_{k_1}b_{c_1}a_{k_2}b_{c_2}=a_{k_1}b_{c_2}a_{k_1}b_{c_2}=A_{k_1c_2}A_{k_2c_1}$, contradicting that the four prime numbers must be pairwise distinct. 
    
    Hence we know that there is at most an edge between two vertices, and since there are exactly $n$ vertices and at least $n$ edges, there exists a cycle comprising at least two different colours (since we have proven that there cannot be a monochromatic cycle above). Let $x_1, x_2, \cdots , x_k$ to be the cycle, with $x_ix_{i+1}$ connected by colour $r_i$ for each $1\le i\le k$. For each $i$, $A_{r_ix_i}$ and $A_{r_ix_{i+1}}$ are both primes, and let $p_{r_ix_i}, p_{r_ix_{i+1}}$ be the primes. Now $\frac{p_{r_ix_i}}{p_{r_ix_{i+1}}}=\frac{A_{x_ii}}{A_{x_i(i+1)}}=\frac{a_{x_i}b_i}{a_{x_i}b_{i+1}}=\frac{b_i}{b_{i+1}}$ (the fact that both entries are prime, i.e. nonzero, means that we don't have to worry about the validity of division). 
	Thus we have 
	\[1=\displaystyle\prod_{i=1}^k\frac{b_i}{b_{i+1}}=\displaystyle\prod_{i=1}^k \frac{p_{r_ix_i}}{p_{r_ix_{i+1}}}
	\]
	and by the theorem of unique prime factorization, $\displaystyle\prod_{i=1}^k p_{ii}$ and $\displaystyle\prod_{i=1}^k p_{(i+1)i}$ also implies that $\{p_{r_ix_i}: 1\le i\le k\}=\{p_{r_ix_{i+1}}\}: 1\le i\le k\}$. Since $p_{r_ix_i}$ corresponds to the entry $(r_i, x_i)$ and $p_{r_ix_{i+1}}$ the entry $(r_i, x_{i+1})$, and each $x_1, x_2, \cdots , x_k$ assumed to be distinct and each of the $m+n$ primes are distinct, we have $p_{r_ix_{i+1}}=p_{r_{i-1}x_i}$, $r_i=r_{i-1}$, so $r_1= r_2 = \cdots = r_k$. This also means that the only possibility is all edges of the cycle coloured the same colour $r_1$, contradiction. 
	
	\item[\textbf{B4}] Show that for each positive integer $n,$ all the roots of the polynomial \[\sum_{k=0}^n 2^{k(n-k)}x^k\] are real numbers.
	
	\textbf{Solution.} Let $f(x)$ be the polynomial, which obviously takes positive values when $x\ge 0$. Consider, now, all $x$'s with $x=-(2^m)$. Then 
	\[
	f(x) = \sum_{k=0}^n 2^{k(n-k)}x^k = \sum_{k=0}^n 2^{k(n-k)}(-2^m)^k
	=\sum_{k=0}^n (-1)^k 2^{k(n+m-k)}
	\]
	We first notice that when $k$ varies, $k(n+m-k)$ takes maximum value when $k=\frac{n+m}{2}$. For this reason, we focus on $m=-n, -n + 2, \cdots, n - 2, n$, whereby $k(n+m-k)=(\frac{n+m}{2})^2 - (\frac{n+m}{2}-k)^2$, and therefore $f(x)=f(-2^m) = 2^{(\frac{n+m}{2})^2}\displaystyle\sum_{k=0}^{n} (-1)^k 2^{- (\frac{n+m}{2}-k)^2}$. Our only interest is the sign of this term, and since the sign of $ 2^{(\frac{n+m}{2})^2}\displaystyle\sum_{k=0}^{n} (-1)^k 2^{- (\frac{n+m}{2}-k)^2}$ is the same as the sign of $\displaystyle\sum_{k=0}^{n} (-1)^k 2^{- (\frac{n+m}{2}-k)^2}$, we will focus on the latter. 
	
	We isolate the cases $n\le 2$ first. For $n=1$ all we have is $x+1$ so $x=-1$ is a solution, obviously. When $n=2$ we have $x^2+2x+1=(x+1)^2$, so $-1$ is a double root. Thus we only deal with $n=3$ here. We recall that if $a_1, a_2, \cdots , a_k$ are distinct nonnegative numbers then $\dsum_{i=1}^k 2^{-a_i} < \dsum_{i=1}^\infty 2^{-i} = 1$. Now we have the following cases to consider: 
	\begin{itemize}
		\item Case 1: $m=\pm n$. In the $+n$ case we have \[\displaystyle\sum_{k=0}^{n} (-1)^k 2^{- (\frac{n+m}{2}-k)^2} = \displaystyle\sum_{k=0}^{n} (-1)^k 2^{- (n-k)^2}=(-1)^n + (-1)^{n-1}2^{-1}
		+(-1)^{n-2}2^{-4}+\cdots + (-1)^0 2^{-n^2}
		\]
		and by the lemma we had, 
		$|(-1)^{n-1}2^{-1}
		+(-1)^{n-2}2^{-4}+\cdots + (-1)^0 2^{-n^2}|\le 2^{-1}+2^{-4}+\cdots + 2^{-n^2} < 1$
		so $(-1)^{n-1}2^{-1}
		+(-1)^{n-2}2^{-4}+\cdots + (-1)^0 2^{-n^2}\in (-1, 1)$ which means 
		$\displaystyle\sum_{k=0}^{n} (-1)^k 2^{- (n-k)^2}$ has the same sign as $(-1)^n$. 
		Similarly, when $m=-n$ the expression \[
		\displaystyle\sum_{k=0}^{n} (-1)^k 2^{- (\frac{n+m}{2}-k)^2}
		=\displaystyle\sum_{k=0}^{n} (-1)^k 2^{-k^2}
		\]
		has the same sign as $(-1)^0=1$ (i.e. positive). 
		
		\item Case 2: now $-n<m<n$ and $n$ has the same parity as $n$. 
		Then $\displaystyle\sum_{k=0}^{n} (-1)^k 2^{- (\frac{n+m}{2}-k)^2}$ has the following form: 
		\[
		(-1)^0 2^{-(\frac{n+m}{2})^2} + (-1)^1 2^{-(\frac{n+m}{2} - 1)^2} +\cdots + (-1)^{(n+m)/2}2^0 + \cdots + (-1)^n 2^{-(\frac{n+m}{2} - n)^2} 
		\]
		W.L.O.G. assume $m\le 0$; the other case is symmetric to this. We notice that $(\frac{n+m}{2} - k)^2 = (\frac{n+m}{2} - (n+m -k))^2$, and moreover $n+m$ is even so $k$ and $n+m-k$ has the same parity. This means we can group these terms together for $k=0, 1, \cdots , \frac{n+m}{2}-1$ to get 
		\begin{flalign*}
			\dsum_{i=0}^{\frac{n+m}{2}-1}((-1)^i  + (-1)^{n+m-i}) 2^{-(\frac{n+m}{2}-i)^2}
			+(-1)^{\frac{n+m}{2}}
			+ \dsum_{i=n+m+1}^{n}(-1)^{i}2^{-(\frac{n+m}{2}-i)^2}
			\\=(-1)^{\frac{n+m}{2}} + 2\dsum_{i=0}^{\frac{n+m}{2}-1}(-1)^i2^{-(\frac{n+m}{2}-i)^2}
			+ \dsum_{i=n+m+1}^{n}(-1)^{i}2^{-(\frac{n+m}{2}-i)^2}
			\\=(-1)^{\frac{n+m}{2}}+2(-1)^{\frac{n+m}{2} - 1} 2^{-1} + 2\dsum_{i=0}^{\frac{n+m}{2}-2}(-1)^i2^{-(\frac{n+m}{2}-i)^2}
			+ \dsum_{i=n+m+1}^{n}(-1)^{i}2^{-(\frac{n+m}{2}-i)^2}
			\\2\dsum_{i=0}^{\frac{n+m}{2}-2}(-1)^i2^{-(\frac{n+m}{2}-i)^2}
			+ \dsum_{i=n+m+1}^{n}(-1)^{i}2^{-(\frac{n+m}{2}-i)^2}
			\\\dsum_{i=0}^{\frac{n+m}{2}-2}(-1)^i2^{-(\frac{n+m}{2}-i)^2+1}
			+ \dsum_{i=n+m+1}^{n}(-1)^{i}2^{-(\frac{n+m}{2}-i)^2}
		\end{flalign*}
		(basically, the two terms beside $(-1)^{(n+m)/2}$ are $(-1)^{(n+m)/2-1}2^{-1}+(-1)^{(n+m)/2+1}2^{-1}$ and therefore vanishes). 
		We recognize that the exponents $-(\frac{n+m}{2}-i)^2+1$ with $i=0, \cdots , \frac{n+m}{2}-2$ are different numbers in the range $[-(\frac{n+m}{2})^2+1, -3]$ and $-(\frac{n+m}{2}-i)^2$ with $i=n+m+1, n$ are different numbers in the range $[-(\frac{m-n}{2})^2, -(-\frac{m+n}{2}-1)^2]$ and $-(-\frac{m+n}{2}-1)^2 < -(\frac{n+m}{2})^2+1$ are disjoint, which means together all these exponents represent different negative numbers. Therefore by the lemma above, the sign with follow the dominating one, i.e. $(-1)^{(n+m+2)/2}$, i.e. $(-1)^{(n+m)/2}$. This conclusion will hold for $m>0$ too. 
	\end{itemize}
	Summarizing above, we know that when $x=2^{m}$ for $m=-n, -n+2, \cdots , n-2, n$, $f(x)$ follows the sign of $(-1)^{(n+m)/2}$. In particular, $(-1)^{(n+m)/2}$ and $(-1)^{(n+m+2)/2}$ have different signs, so there is a root between $(-2^{m+2}, -2^m)$. 
	Considering $m=-n, \cdots , -n+2$ we know that there are roots in the intervals 
	$(-2^{n}, -2^{n-2}), (-2^{n-2}, \cdots , -2^{n-4}), \cdots , (-2^{-n+2}, -2^{-n})$ which are $n$ disjoint intervals, hence at least $n$ real roots. On the other hand, $f$ is a polynomail with degree $n$, hence only at $n$ roots in total. Thus all roots are real. 
    
\end{enumerate}

\section{Putnam 2013}
\begin{enumerate}
	\item [\textbf{A1}] Recall that a regular icosahedron is a convex polyhedron having 12 vertices and 20 faces; the faces are congruent equilateral triangles. On each face of a regular icosahedron is written a nonnegative integer such that the sum of all $20$ integers is $39.$ Show that there are two faces that share a vertex and have the same integer written on them.
	
	\textbf{Solution.} Each face corresponds to exactly 3 vertices, so on average each vertex corresponds to $20\times 3\div 12=5$ faces. Since this icosahedron is regular, each vertex corresponds to 5 faces. Suppose that for each vertex, the number written is different. Then the sum of the 5 faces joining a vertex is at least $0+1+2+3+4=10$. Since each vertex corresponds to 3 faces and there are 12 vertices, the total sum of 20 faces is at least $10\times 12\div 3=40$, contradiction. 
	
	\item[\textbf{A2}]Let $S$ be the set of all positive integers that are not perfect squares. For $n$ in $S,$ consider choices of integers $a_1,a_2,\dots, a_r$ such that $n<a_1<a_2<\cdots<a_r$ and $n\cdot a_1\cdot a_2\cdots a_r$ is a perfect square, and let $f(n)$ be the minimum of $a_r$ over all such choices. For example, $2\cdot 3\cdot 6$ is a perfect square, while $2\cdot 3,2\cdot 4, 2\cdot 5, 2\cdot 3\cdot 4,$ $2\cdot 3\cdot 5, 2\cdot 4\cdot 5,$ and $2\cdot 3\cdot 4\cdot 5$ are not, and so $f(2)=6.$ Show that the function $f$ from $S$ to the integers is one-to-one.
	
	\textbf{Solution.} Suppose that $f(k_1)=f(k_2)$ for some $k_1 < k_2$. Let $k_1<a_1<\cdots <a_r=f(k_1)$ and $k_2<b_1<\cdots<b_s=f(k_2)$ be such choices for $k_1$ and $k_2$. Given that $k_1\cdot a_1\cdots a_r$ and $k_2\cdot b_1\cdots b_s$ are both perfect square, their product $k_1\cdot a_1\cdots a_r\cdot k_2\cdot b_1\cdots b_s$ is also a perfect square. Suppose that some number $g$ appears in both sequence $\{a_i\}$ and $\{b_i\}$, then removing $g$ from the combined sequence $k_1\cdot a_1\cdots a_r$ and $k_2\cdot b_1\cdots b_s$ yields that $k_1\cdot a_1\cdots a_r\cdot k_2\cdot b_1\cdots b_s/g^2$ is still a perfect square. Now, we remove all such repeated elements and sort the numbers, we get $k_1<c_1<\cdots <c_t$, since $k_1$ only appears once ($k_1<k_2$) and since $a_r=b_s$, this number is removed from both sides and $c_t<a_r=f(k_1)$, contraidcting the minimality of $a_r$. 
	
	\item[\textbf{A3}]Suppose that the real numbers $a_0,a_1,\dots,a_n$ and $x,$ with $0<x<1,$ satisfy \[\frac{a_0}{1-x}+\frac{a_1}{1-x^2}+\cdots+\frac{a_n}{1-x^{n+1}}=0.\] Prove that there exists a real number $y$ with $0<y<1$ such that \[a_0+a_1y+\cdots+a_ny^n=0.\]
	
	\textbf{Solution.} First, since $0<x<1$, each sequence $1-x^n=1+x^n+x^{2n}+x^{3n}+\cdots$ converges absolutely. Hence we are free to permute the sequence and get the sum in the following sense: 
	\[0=\frac{a_0}{1-x}+\frac{a_1}{1-x^2}+\cdots+\frac{a_n}{1-x^{n+1}}
	=\sum_{i=1}^n a_i \left(\sum_{j=0}^{\infty} x^{ij}\right)
	=\sum_{j=0}^{\infty} \left(\sum_{i=0}^n a_i(x^{j})^i\right)
	\]
	Let $b_j=\sum_{i=0}^n a_i(x^{j})^i$ for all $j\ge 0$, then it follows that $\sum_{i=0}^\infty b_j$ also converges absolutely to 0. Now let $k$ to be the minimal index such that $b_k\neq 0$. If $b_j=0$ for some $j>k$ then we are done, since we can just pick $y=x^j$ and since $j>k\ge 0$, $y\in (0, 1)$. Otherwise, since $\sum_{i=k}^\infty b_j=\sum_{i=0}^\infty b_j=0$, there exists a $j\ge k$ such that $b_j<0$ and $b_{j+1}>0$, or vice versa. In either case, $a_0+a_1y+\cdots+a_ny^n=0$ for some $y\in (x^{j+1}, x^{j})$, which obviously lies in $(0, 1)$. 
	
	\item[\textbf{A4}] A finite collection of digits $0$ and $1$ is written around a circle. An arc of length $L\ge 0$ consists of $L$ consecutive digits around the circle. For each arc $w,$ let $Z(w)$ and $N(w)$ denote the number of $0$'s in $w$ and the number of $1$'s in $w,$ respectively. Assume that $|Z(w)-Z(w')|\le 1$ for any two arcs $w,w'$ of the same length. Suppose that some arcs $w_1,\dots,w_k$ have the property that \[Z=\frac1k\sum_{j=1}^kZ(w_j)\text{ and }N=\frac1k\sum_{j=1}^k N(w_j)\] are both integers. Prove that there exists an arc $w$ with $Z(w)=Z$ and $N(w)=N.$
	
	\textbf{Solution}. Let $n$ be the number of bits written around the circle, and $m$ the number of 1's written. We first prove the following lemma: for each $0\le k\le n$, consider $S_k=\{Z(w): |w|=k\}$ where $|w|$ is the length of $w$. Then $S_k=\{\lfloor \frac{km}{n}\rfloor, \lceil \frac{km}{n}\rceil\}$. To prove this, we notice that from the problem statement, $\max(S_k)-\min(S_k)\le 1$. 
	We consider two possible cases: 
	\begin{itemize}
		\item The quantity $\frac{km}{n}$ is not an integer. Then $\lceil \frac{km}{n}\rceil-\lfloor \frac{km}{n}\rfloor=1$. Now consider the $n$ arcs $W_1, W_2, \cdots , W_n$ with length $k$; each point belongs to exactly $k$ of the arcs. Thus $\sum_{i=1}^n Z(W_i)=km$ since there are $m$ one's written, and the average of $Z(w_i)$ is $\frac{km}{n}$. By piegonholw principle, there is at least one $W_i$ with $Z(W_i)\ge \lceil \frac{km}{n}\rceil$ and one $W_i$ with $Z(W_i)\le \lceil \frac{km}{n}\rceil$. Since $\max(W_i)-\min(W_i)\le 1$ and $\lceil \frac{km}{n}\rceil-\lfloor \frac{km}{n}\rfloor$, the conclusion follows. 
		
		\item Now that the quantity $\frac{km}{n}$ is an integer, meaning that $\lceil \frac{km}{n}\rceil=\frac{km}{n}=\lfloor \frac{km}{n}\rfloor$. Denote the $n$ arcs by $W_1, \cdots , W_n$, and by the logic above, the average of $Z(w_i)$ is $\frac{km}{n}$. 
		If there is $w_i$ with $Z(w_i)<\frac{km}{n}$, i.e. $Z(w_i)\le \frac{km}{n} -1$, then there must be $w_i$ with $Z(w_i)>\frac{km}{n}$, i.e. $Z(w_i)\ge \frac{km}{n}+1$. This is a contradiction that $\max(S_k)-\min(S_k)\le 1$, so we have $Z(W_i)=\frac{km}{n}$ for each $i$. 
	\end{itemize}
	Now going back to the problem. For each arc $w$, we have $|w|=Z(w)+N(w)$. Thus considering the $w_1, \cdots , w_k$ given in the problem we have \[Z+N=\frac1k\sum_{j=1}^kZ(w_j)+\frac1k\sum_{j=1}^kZ(w_j)=\frac1k\sum_{j=1}^k(Z(w_j)+N(w_j)=\frac1k\sum_{j=1}^k|w_j|\]
	And in addition, for each $w_j$ we have $Z(w_j)\in \{\lfloor \frac{|w_j|m}{n}\rfloor, \lceil \frac{|w_j|m}{n}\rceil\}$ so $\frac{|w_j|m}{n}-1<Z(w_j)<\frac{|w_j|m}{n}+1$. This means, 
	$\frac1k\sum_{j=1}^k(\frac{|w_j|m}{n}-1) < Z < \frac1k\sum_{j=1}^k(\frac{|w_j|m}{n}+1)$, i.e. 
	$(\frac{m}{kn}\sum_{j=1}^k |w_j|)-1 < Z < (\frac{m}{kn}\sum_{j=1}^k |w_j|)+1$, and 
	since $\frac 1k\sum_{j=1}^k |w_j|=Z+N$, we have 
	\[(\frac{m}{n}(Z+W))-1 < Z < (\frac{m}{n} (Z+N))+1
	\]
	Since $Z$ is an integer, this also implies that $Z\in \{\lfloor \frac{m}{n}(Z+N)\rfloor , \lceil \frac{m}{n}(Z+N)\rceil\}$. This means we can find an arc $w$ of length $Z+N$ that has $Z(w)=Z$, and therefore $N(W)=N$. 
	
	\item[\textbf{B1}] For positive integers $n,$ let the numbers $c(n)$ be determined by the rules $c(1)=1,c(2n)=c(n),$ and $c(2n+1)=(-1)^nc(n).$ Find the value of \[\sum_{n=1}^{2013}c(n)c(n+2).\]
	
	\textbf{Answer.} $-1$. \\
	\textbf{Solution.}
	\begin{flalign*}
	\sum_{n=1}^{2013}c(n)c(n+2)
	&=c(1)c(3)+\sum_{n=1}^{1006}c(2n)c(2n+2)
	+\sum_{n=1}^{1006}c(2n+1)c(2n+3)\\
	&=c(1)c(3)+\sum_{n=1}^{1006}c(n)c(n+1)
	+\sum_{n=1}^{1006}(-1)^nc(n)(-1)^{n+1}c(n+1)\\
	&=c(1)c(3)+\sum_{n=1}^{1006}c(n)c(n+1)
	+\sum_{n=1}^{1006}(-1)^{2n+1}c(n)c(n+1)\\
	&=c(1)c(3)+\sum_{n=1}^{1006}c(n)c(n+1)
	-\sum_{n=1}^{1006}c(n)c(n+1)\\
	&=c(1)c(3)\\
	&=c(1)(-1)^1c(1)\\
	&=-1
	\end{flalign*}
	
	
	\item[\textbf{B2}] Let $C=\bigcup_{N=1}^{\infty}C_N,$ where $C_N$ denotes the set of 'cosine polynomials' of the form \[f(x)=1+\sum_{n=1}^Na_n\cos(2\pi nx)\] for which:
	
	(i) $f(x)\ge 0$ for all real $x,$ and\\
	(ii) $a_n=0$ whenever $n$ is a multiple of $3.$
	
	Determine the maximum value of  $f(0)$ as $f$ ranges through $C,$ and prove that this maximum is attained.
	
	\textbf{Answer.} 3.\\
	\textbf{Solution. }Consider the following:
	\begin{flalign*}
	f(x)&=1+\frac 43 \cos(2\pi x)+\frac 23\cos(4\pi x)\\
	&=1+\frac 43 \cos(2\pi x)+\frac 23 (2\cos^2(2\pi x)-1)\\
	&=\frac 13 (1+4\cos(2\pi x)+4\cos^2 (2\pi x))\\
	&=\frac 13 (1+2\cos(2\pi x))^2\\
	\end{flalign*}
	which is clearly nonnegative all the time. We also have $f(0)=1+\frac 43+\frac 23=3$, establishing the equality. To show that 3 is indeed the maximum, it suffices to show that $\sum_{n=1}^N a_n\le 2$ at all times. But plugging $x=\frac 13$ gives $\cos(\frac 23 n\pi)=-\frac 12$ if $n$ is not divisible by 3, and 1 otherwise. Considering that $a_n=0$ whenever $n$ is a multiple of 3, we have $f(\frac 13)=1-\frac 12 \sum_{n=1}^N a_n\ge 0.$ Thus $\sum_{n=1}^N a_n\le 2$ must hold. Finally note the motivation to get the example $f(x)$ as shown in the beginning: we simply find a suitable $a$ such that $2ax^2+(2-a)x+(1-a)$ is always nonnegative, which is essentially asking for the discriminant $(2-a)^2-4(2a)(1-a)\le 0$, and we get $a=\frac 23$ as the sole answer. 
	
	\item[\textbf{B3}] Let $P$ be a nonempty collection of subsets of $\{1,\dots,n\}$ such that:
	\begin{itemize}
		\item[(i)] if $S,S'\in P,$ then $S\cup S'\in P$ and $S\cap S'\in P,$ and
		\item [(ii)] if $S\in P$ and $S\ne\emptyset,$ then there is a subset $T\subset S$ such that $T\in P$ and $T$ contains exactly one fewer element than $S.$
	\end{itemize}
	
	Suppose that $f:P\to\mathbb{R}$ is a function such that $f(\emptyset)=0$ and \[f(S\cup S')= f(S)+f(S')-f(S\cap S')\text{ for all }S,S'\in P.\] Must there exist real numbers $f_1,\dots,f_n$ such that \[f(S)=\sum_{i\in S}f_i\] for every $S\in P?$
	
	\textbf{Answer.} Yes. \\
	\textbf{Solution.} Let $S_0$ be the subset such that $|S_0|\ge |S|$ for all $S\in P$. We first show that $S\subseteq S_0$ for all $S\in P$. Indeed, for an arbitrary set $S\in P$, we have $S\cup S_0\in P$ and $|S\cup S_0|\ge |S_0|$. By the maximality of $S_0$ we must have $|S\cup S_0|=|S_0|$, which can only happen when $S\subseteq S_0$, as desired. 
	
	Now, w.l.o.g. let $S_0=\{1, 2, \cdots , k\}$. By (ii) there exists $S_1\subseteq S_0$ with one element fewer than $S_0$; w.l.o.g. let it be $\{1, 2, \cdots , k-1\}$. Continuing this trend we can assume that for all $0\le i\le k$, $S_{k-i}=\{1, 2, \cdots , i\}\in P$. Consider, now, $f_1, f_2, \cdots , f_n$ such that $f_i = f(S_{k-i})-f(S_{k-i+1})$ for $i=1, 2, \cdots , k$, and arbitrary for $i=k+1, \cdots , n$. 
	The identity $f(S)=\sum_{i\in S}f_i$ holds when $S=S_0, S_1, \cdots, S_k$ with $S_k$ being the empty set (because $f(\emptyset)=0$). 
	
	To show that this identity holds for all $S\in P$, we first notice that $S\subseteq S_0=\{1, 2, \cdots , k\}$, so only $f_1, \cdots , f_k$ matter. We will proceed using the following premise with parameter $p=0, 1, \cdots , k$: the $f(S)=\sum_{i\in S} f_i$ identity holds for all $p$-element subsets $S$. We are to prove this statement for all $p=0, 1, \cdots , k$, and we will proceed by induction. 
	
	Base case: when $p=0$ we have emptyset (proven above), and when $p=1$ we have $S=\{j\}$ for some $1\le j\le k$. We have $S_{k-j}=\{1, 2, \cdots , j\}$ and $S_{k-j+1}=\{1, 2, \cdots , j-1\}$. Moreover, $f(S_{k-j+1})=\dsum_{i=1}^{j-1} f_i$ and $f(S_{k-j})=\dsum_{i=1}^j$ by how $f_i$'s are defined. Now by the definition of $f$, 
	\[
	f(S_{k-j}) = f(S\cup S_{k-j+1}) = f(S)+f(S_{k-j+1}) - f(S\cup S_{k-j+1})
	\]
	since $S=\{j\}$. Considering that $S\cup S_{k-j+1}=\emptyset$, we have $f(S\cup S_{k-j+1})=0$ and therefore 
	\[
	f(S)=f(S_{k-j}) - f(S_{k-j+1}) = \dsum_{i=1}^{j} f_i - \dsum_{i=1}^{j-1} f_i = f_j
	\]
	as desired. 
	
	Now let $2\le q\le p$ be such that the preimise is true for all $p=1, 2, \cdots , q-1$. Consider, now, any $q$-element subset $S=\{a_1, a_2, \cdots , a_q\}$. By condition (ii), there exists a subset of $S$ in $P$ with one fewer element; w.l.o.g. let it be $\{a_1, \cdots , a_{q-1}\}$. Consider, now, the set $S_{k-a_q}=\{1, 2, \cdots , a_q\}\in P.$ Consider now the two equations: 
	\[
	f(S_{k-a_q}\cup \{a_1, \cdots , a_{q-1}\}) = f(S_{k-a_q}) + f(\{a_1, \cdots , a_{q-1}\}) - f(S_{k-a_q}\cap \{a_1, \cdots , a_{q-1}\})
	\]
	\[
	f(S_{k-a_q}\cup \{a_1, \cdots , a_{q}\}) = f(S_{k-a_q}) + f(\{a_1, \cdots , a_{q}\}) - f(S_{k-a_q}\cap \{a_1, \cdots , a_{q}\})
	\]
	First, notice that $\{a_1, \cdots , a_{q-1}\}$ and $\{a_1, \cdots , a_{q}\}$ differ only by an element $a_q$, and since $a_q\in S_{k-a_q}$, we have $\{1, 2, \cdots , a_q\}\cup \{a_1, \cdots , a_{q-1}\} = \{1, 2, \cdots , a_q\}\cup \{a_1, \cdots , a_{q}\}$. 
	Comparing the two equations now give
	\[
	f(\{a_1, \cdots , a_{q}\}) - f(\{a_1, \cdots , a_{q-1}\}) = f(S_{k-a_q}\cap \{a_1, \cdots , a_{q}\}) - f(S_{k-a_q}\cap \{a_1, \cdots , a_{q-1}\})
	\]
	Since $a_q\in S_{k-a_q}$, we have $S_{k-a_q}\cap \{a_1, \cdots , a_{q-1}\}\subset S_{k-a_q}\cap \{a_1, \cdots , a_{q}\}$, differing only by an element $a_q$. 
	If $\{a_1, \cdots , a_q\}=S_{k-a_q}$ then the condition $f(S)=\dsum_{f_i\in S}f_i$ holds for this $S=\{a_1, \cdots, a_q\}$. 
	Otherwise, $S_{k-a_q}\cap \{a_1, \cdots , a_{q}\}$ will have less than $q$ elements. By the induction hypothesis, $f(S_{k-a_q}\cap \{a_1, \cdots , a_{q}\}) - f(S_{k-a_q}\cap \{a_1, \cdots , a_{q-1}\})=f_{a_q}$, and therefore $f(\{a_1, \cdots , a_{q}\}) = f(\{a_1, \cdots , a_{q-1}\}) + f_{a_q}$. But by induction hypothesis again $f(\{a_1, \cdots , a_{q-1}\}) = f_{a_1} + \cdots + f_{a_{q-1}}$, and from here the conclusion follows. 
	
	\item[\textbf{B5}] Let $X=\{1,2,\dots,n\},$ and let $k\in X.$ Show that there are exactly $k\cdot n^{n-1}$ functions $f:X\to X$ such that for every $x\in X$ there is a $j\ge 0$ such that $f^{(j)}(x)\le k.$
	
	[Here $f^{(j)}$ denotes the $j$th iterate of $f,$ so that $f^{(0)}(x)=x$ and $f^{(j+1)}(x)=f\left(f^{(j)}(x)\right).$]
	
	\textbf{Solution.} We perform induction on $n$ and $n-k$. When $k=n$ (when $n-k=0$) then any function $f:X\to X$ works, so there are $n^n$ such functions; when $k=n-1$, the only requirement is that $f(n)\neq n$ so there are $(n-1)n^{n-1}$ such functions. 
	
	Thus now consider any $k\le n-2$. Observe that since $f^{0}(x)=x$, there is no restriction on $f(1), f(2), \cdots , f(k)$, giving $n^k$ choices to each of them. We first make a following detour to a lemma: there exists $x>k$ with $f(x)\le k$. Suppose not, then we have $f:\{k+1, \cdots , n\}\to \{k+1, \cdots , n\}$ and for each $x>k$ we have $f^{(j)}(x)>k$ for any $j\ge 0$, contradiction. 
	
	Now fix $m\in [1, n-k]$ such that exactly $m$ of the numbers $k<x\le n$ have $f(x)\le k$. This gives rise of $\dbinom{n-k}{m}$ ways to choose those $x$, and each of them takes values $\{1, 2, \cdots , k\}$, giving rise to $k^m$ of them. Now w.l.o.g. assume that those $m$ elements are $k+1, \cdots , k+m$. 
	 To see how would $f(x)$ looks like for the other $x>k+m$'s, consider the problem of $g:\{1, 2, \cdots , n-k\}\to \{1, 2, \cdots , n-k\}$ where for each $x$, $g^{(j)}(x)\le m$ for some $j\ge 0$. Here, $g(1), \cdots g(m)$ can be arbitrary, (i.e. $(n-k)^m$ choices), and by the induction hypothesis $n-k<n$ since $k\ge 1$, the number of such $g$'s is $m\cdot (n-k)^{n-k-1}$, meaning that there are $(n-k)^{n-k-m-1}$ choices for $g(m+1), \cdots , g(n-k)$. Thus going back to our original problem here, we consider $f|_{k+m+1, \cdots , n}$ such that $f^{(j)}(x)\le k+m$ for some $j$ (this is because, given that $f(x)>k$ for all $x>k+m$, if $j_0$ is the minimum $j$ with $f^{(j_0)}(x)\le k$ then $f^{(j_0-1)}(x)\in \{k+1, \cdots , k+m\}$). This gives $m\cdot (n-k)^{(n-k-m-1)}$ choices on $f|_{k+m+1, \cdots , n}$, giving rise of $\dbinom{n-k}{m}\cdot k^m\cdot m\cdot (n-k)^{(n-k-m-1)}$ of them in total. 
	 
	 Hence considering all such $m\in [1, n-k]$ gives 
	 
	 \begin{flalign*}
	 	\displaystyle\sum_{m=1}^{n-k}\dbinom{n-k}{m}\cdot k^m\cdot m\cdot (n-k)^{(n-k-m-1)}
	 	&=\displaystyle\sum_{m=1}^{n-k}\dfrac{(n-k)!}{m!(n-k-m)!}\cdot k^m\cdot m\cdot (n-k)^{(n-k-m-1)}
	 	\\&=\displaystyle\sum_{m=1}^{n-k}\dfrac{(n-k)(n-k-1)!}{(m-1)!(n-k-m)!}\cdot k^{m-1}\cdot k\cdot (n-k)^{(n-k-m-1)}
	 	\\&=k\displaystyle\sum_{m=1}^{n-k}\dbinom{n-k-1}{m-1}\cdot k^{m-1}(n-k)^{(n-k-m)}
	 	\\&=k(k+n-k)^{m-k-1}
	 	\\&=kn^{n-k-1}
	 \end{flalign*}
	 so combining with the arbitrary choice of $f|_{1, 2, \cdots , k}$ we have $kn^{n-k-1}n^k=kn^{n-1}$, as desired. 
	
\end{enumerate}

\section{Putnam 2012}
\begin{enumerate}
	\item [\textbf{A1}]Let $d_1,d_2,\dots,d_{12}$ be real numbers in the open interval $(1,12).$ Show that there exist distinct indices $i,j,k$ such that $d_i,d_j,d_k$ are the side lengths of an acute triangle.
	
	\textbf{Solution.} We sort the numbers into $d_1 \le d_2 \le \cdots \le d_{12}$. Suppose otherwise, then $d_i^2+d_{i+1}^2\le d_{i+2}^2$. From $d_1, d_2 > 1$ we have $d_3>\sqrt{2}$, $d_4 > \sqrt{3}$, $d_5>\sqrt{5}$, $d_6 > \sqrt{8}$, $d_7 > \sqrt{13}$, $d_8 > \sqrt{21}$, $d_9 > \sqrt{34}$, $d_{10} > \sqrt{55}$, $d_{11} > \sqrt{89}$, $d_{12}>\sqrt{144}=12$, contradiction. 
	
	\item[\textbf{A2}] Let $*$ be a commutative and associative binary operation on a set $S.$ Assume that for every $x$ and $y$ in $S,$ there exists $z$ in $S$ such that $x*z=y.$ (This $z$ may depend on $x$ and $y.$) Show that if $a,b,c$ are in $S$ and $a*c=b*c,$ then $a=b.$
	
	\textbf{Solution.} Now we suppose that $a*c=b*c$. Let $x$ be the element satisfying $a*x=b$, $y$ be the element satisfying $a*y=a$, then we have
	\[(x * a) * c = (a * x) * c = b * c = a * c = (a * y) * c= (y * a) * c \]
	and by the commutativity and associativity of $*$ we can arrange this into $x * (a * c) = y * (a * c)$. Now let $d$ be the element such that $(a * c) * d = a$ we have $b = x * a = x * (a * c) * d = y * (a * c) * d = y * a = a$, as desired. 
	
	\item[\textbf{A3}]Let $f:[-1,1]\to\mathbb{R}$ be a continuous function such that
	
	(i) $f(x)=\frac{2-x^2}{2}f\left(\frac{x^2}{2-x^2}\right)$ for every $x$ in $[-1,1],$
	
	(ii) $ f(0)=1,$ and
	
	(iii) $\lim_{x\to 1^-}\frac{f(x)}{\sqrt{1-x}}$ exists and is finite.
	
	Prove that $f$ is unique, and express $f(x)$ in closed form.
	
	\textbf{Answer.} The function $f(x)=\sqrt{1-x^2}$ is the unique solution. \\
	\textbf{Solution.} We will skip the verification on the fact that this $f$ works (with the third limit equals to $\sqrt{1+x}$), and jump straight into the fact that there's no other $f$ that fits this criterion. 
	
	First, notice from the first condition that 
	$f(x)=\frac{2-x^2}{2}f\left(\frac{x^2}{2-x^2}\right)
	=f(x)=\frac{2-(-x)^2}{2}f\left(\frac{(-x)^2}{2-(-x)^2}\right)
	=f(-x)$
	so it suffices to consider those functions in the range $[0, 1]$. We first define a polynomial $P_n(x)$ in the following fashion: $P_0(x)=x$ and for each $n$, $P_{n+1}(x)=2P_n^2(x)-1$ (basically $P_n(\cos x)=\cos (2^n x)$). We have the following observations: 
	
	\begin{itemize}
		\item If $x>1$, then $\lim_{n\to\infty}P_{n}(x)=\infty$. To see why, for each $x>1$ we have $2x^2-1=x^2+(x^2-1)>x^2$ so $P_1(x)>P_0(x)^2=x^2$ and we can inductively show that $P_n(x)>x^{2^n}$. Since $\lim_{n\to\infty} x^{2^n}=\infty$ for each $x>1$, the conclucion follows. 
		
		\item Now for each $x>1$ we have $f(\frac 1{P_n(x)})=\frac{2P_n(x)^2-1}{2P_n(x)^2}f(\frac{1}{2P_n(x)^2-1})=\frac{P_{n+1}(x)}{2P_n(x)^2}f(\frac{1}{P_{n+1}(x)})$. Inductively we get the following telescoping sum: 
		\[
		f\left(\frac 1x\right)=f\left(\frac 1{P_0(x)}\right)
		=\frac{P_{1}(x)}{2P_0(x)^2}f\left(\frac{1}{P_1(x)}\right)
		=\frac{P_{2}(x)}{2^2P_0(x)^2P_1(x)}f\left(\frac{1}{P_2(x)}\right)
		\]
		which leads to the ultimate $\dfrac{P_n(x)}{2^n P_0(x)\prod_{k=1}^{n-1}P_k(x)}f\left(\frac{1}{P_n(x)}\right)$. 
		We also recall that as $x>1$, $\lim_{n\to\infty}P_n(x)=\infty$ so $f\left(\frac{1}{P_n(x)}\right)\to f(0)=1$ since $f$ is continuous. 
		Thus, $f\left(\frac 1x\right)=\lim_{n\to\infty} \dfrac{P_n(x)}{2^n P_0(x)\prod_{k=1}^{n-1}P_k(x)}$. 
		
		\item The identity that $2^n\cos(x)\cos(2x)\cos(4x)\cdots \cos(2^{n-1}(x))$ can be telescoped in the following manner: multiplying by $\sin(x)$ we have $2^n\sin(x)\cos(x)\cos(2x)\cos(4x)\cdots \cos(2^{n-1}(x))=2^{n-1}\sin(2x)\cos(2x)\cos(4x)\cdots \cos(2^{n-1}(x))=\cdots 
		=\sin(2^n x)
		$
		since $2\sin x\cos x=\sin(2x)$ so $2^n\cos(x)\cos(2x)\cos(4x)\cdots \cos(2^{n-1}(x))=\frac{\sin 2^n x}{\sin x}$. Unfortunately, we are dealing with $P_n(x)$ where each value is greater than 1, and is therefore less helpful. We nevertheless notice that the realization $\sin 2x = 2\cos x\sin x$ comes from the fact that $\sqrt{1-\cos^2 2x}=2\cos x\sqrt{1-\cos ^2 x}$, i.e. for each $0\le x<1$ we have $\sqrt{1-P_1(x)^2}=2 x\sqrt{1-x^2}$, or more generally, $\sqrt{1-P_{n+1}(x)^2}=2P_n(x)\sqrt{1-P_n(x)^2}$. Multiplying $\sqrt{-1}$ into both sides (bad writing, only for iidea illustration purpose) we get $\sqrt{P_{n+1}(x)^2-1}=2P_n(x)\sqrt{P_n(x)^2-1}$, which will help in the case $x>1$.Thus we have the following telescoping sum for the following: 
		
		\begin{flalign*}
		2^n\sqrt{x^2-1}\prod_{k=1}^{n-1}P_k(x)
		&=2^n\sqrt{P_0(x)^2-1}P_0(x)P_1(x)\cdots P_{n-1}(x)
		\\&=2^{n-1}\sqrt{P_1(x)^2-1}P_1(x)\cdots P_{n-1}(x)
		\\&=2^{n-2}\sqrt{P_2(x)^2-1}P_2(x)\cdots P_{n-1}(x)
		\\&=\cdots
		\\&=2\sqrt{P_{n-1}(x)^2-1}P_{n-1}(x)
		\\&=\sqrt{P_n(x)^2-1}
		\end{flalign*}
		This means we actually have 
		\[
		f\left(\frac 1x\right)=\lim_{n\to\infty} \dfrac{P_n(x)}{2^n x\prod_{k=1}^{n-1}P_k(x)}
		=\lim_{n\to\infty} \dfrac{P_n(x)\sqrt{x^2-1}}{2^n x\sqrt{x^2-1} \prod_{k=1}^{n-1}P_k(x)}
		=\lim_{n\to\infty} \dfrac{P_n(x)\sqrt{x^2-1}}{x\sqrt{P_n(x)^2-1}}
		\]
		and since $P_n(x)\to \infty$, we have $\frac{P_n(x)}{\sqrt{P_n(x)^2-1}}\to 1$, so $f\left(\frac 1x\right)=\dfrac{\sqrt{x^2-1}}{x}=\sqrt{1-\frac 1{x^2}}$. Thus $f(x)=\sqrt{1-x^2}$, as desired. 
		\end{itemize}
	
	\item[\textbf{A5}]Let $\mathbb{F}_p$ denote the field of integers modulo a prime $p,$ and let $n$ be a positive integer. Let $v$ be a fixed vector in $\mathbb{F}_p^n,$ let $M$ be an $n\times n$ matrix with entries in $\mathbb{F}_p,$ and define $G:\mathbb{F}_p^n\to \mathbb{F}_p^n$ by $G(x)=v+Mx.$ Let $G^{(k)}$ denote the $k$-fold composition of $G$ with itself, that is, $G^{(1)}(x)=G(x)$ and $G^{(k+1)}(x)=G(G^{(k)}(x)).$ Determine all pairs $p,n$ for which there exist $v$ and $M$ such that the $p^n$ vectors $G^{(k)}(0),$ $k=1,2,\dots,p^n$ are distinct.
	
	\textbf{Answer.} All $n=1$ with any prime $p$, together with $n=2, p=2$. 
	
	\textbf{Solution.} For $n=1$, taking $M=v=
	\begin{pmatrix}
	1
	\end{pmatrix}
	$ gives us the desired solution since $G^{(k)}(0)=
	\begin{pmatrix}
	k
	\end{pmatrix}
	$. For $n=2$ and $p=2$, take $M=
	\begin{pmatrix}
	1 & 1\\
	0 & 1\\
	\end{pmatrix}
	$ and $v=
	\begin{pmatrix}
	0\\
	1\\
	\end{pmatrix}
	$ gives us
	$G^{(1)}(0)=
	\begin{pmatrix}
	0\\
	1\\
	\end{pmatrix}
	$, 
	$G^{(2)}(0)=
	\begin{pmatrix}
	1\\
	0\\
	\end{pmatrix}
	$,
	$G^{(3)}(0)=
	\begin{pmatrix}
	1\\
	1\\
	\end{pmatrix}
	$,
	$G^{(4)}(0)=
	\begin{pmatrix}
	0\\
	0\\
	\end{pmatrix}
	$, establishing the claim. 
	
	To show that these are the only pairs, we first expand $G^{(k)}(0)$ in the `brute-force' manner, meaning that $G^{(k)}(0)=(1+M+M^2+\cdots + M^{k-1})v$. By Cayley-Hamilton theorem, if $P(x)$ is the characteristic polynomial of $M$, then $P$ has degree $n$ and $P(M)=0$. Hence it suffices to consider the remainder of $1+M+M^2+\cdots + M^{k-1}$ when divided by $P(x)$. Now, denote the remainder of $1+M+M^2+\cdots + M^{k-1}$ as $a_0+a_1M+\cdots + a_{n-1}M^{n-1}$. By our assumption, each of the $p^n$ vectors are distinct, so this polynomial remainder must also be distinct as $k$ varies across $1, 2, \cdots , p^n$. Denoting $Q_k(x)$ as the remainder of $1+x+\cdots + x^{k-1}(x)$. Then we have $Q_k(x)\equiv 1+xQ_{k-1}(x)\pmod{P(x)}$ for all $k\ge 1$, with $Q_0(x)=0$. It then follows that $Q_k(x)$ are all distinct for $k=1, 2, \cdots , p^n$ and since there are exactly $p^n$ elements in the set $\{\sum_{j=0}^{n-1}a_jx^j, a_j\in\bbF_p\}$, each element in the set occurs exactly once in $Q_1, Q_2, \cdots , Q_{p^n}$ and since $Q_k=1+xQ_{k-1}$, $Q_1, Q_2, \cdots , Q_{p^n}$ form a cycle, and considering $Q_0=0$ we have $Q_{p^n}=0$ too. 
	
	The relation $Q_{p^n}=0$ implies that $P(x)|1+x+\cdots x^{k-1}$ if and only if $p^n|k$. We first claim that $1+x+\cdots x^{p^n-1}=(x-1)^{p^n-1}$. Indeed, the coefficient of $x^k$ in this polynomial is \[(-1)^k\dbinom{p^n-1}{k}
	=(-1)^k \prod_{j=1}^k\left(\frac{p^n-j}{j}\right)
	=(-1)^k \prod_{j=1}^k\left(\frac{p^n}{j}-1\right)
	\]
	since $j$ is not divisible by $p^n$, $\frac{p^n-j}{j}$ can be treated as an integer modulo $p$. In fact, $\frac{p^n}{j}$ is 0 modulo $p$ since $j$ is not divisible by $n$, so $\left(\frac{p^n}{j}-1\right)\equiv -1\pmod{p}$. This means that 
	\[(-1)^k \prod_{j=1}^k\left(\frac{p^n}{j}-1\right)\equiv (-1)^k \prod_{j=1}^k(-1)=(-1)^{2k}=1\]
	and therefore $(x-1)^{p^n-1}= 1+x+\cdots + x^{p^n-1}$ when considering elements in $\bbF_p[x]$, as desired. This means that $Q_{p^n-1}$ has 1 as root, repeated $p^n-1$ times, and so the only possible $P(x)$ (which must have degree $n$) is $(x-1)^n$. Nevertheless, our computation above also suggests that $1+x+\cdots + x^{p^k-1}=(x-1)^{p^k-1}$ for all $k\ge 1$, too, so $P(x)|Q_{p^k-1}$ whenever $p^k-1\ge n$. This should not happen for $k<n$; otherwise $Q_{p^k-1}=0$ for all such $k$ and we have $G^{(p^k)}(0)=0$. Therefore we need to have $p^{n-1}-1<n$, or $p^{n-1}\le n$. This holds whenever $n=1$, and when $n=2$ we need $p\le 2$, i.e. $p=2$. For $n\ge 3$, we have $2^{n-1}>n$ (can be proven by induction) so no prime can satisfy this. This completes the proof, QED. 
	
	\item [\textbf{B1}]Let $S$ be a class of functions from $[0,\infty)$ to $[0,\infty)$ that satisfies:
	
	(i) The functions $f_1(x)=e^x-1$ and $f_2(x)=\ln(x+1)$ are in $S;$
	
	(ii) If $f(x)$ and $g(x)$ are in $S,$ the functions $f(x)+g(x)$ and $f(g(x))$ are in $S;$
	
	(iii) If $f(x)$ and $g(x)$ are in $S$ and $f(x)\ge g(x)$ for all $x\ge 0,$ then the function $f(x)-g(x)$ is in $S.$
	
	Prove that if $f(x)$ and $g(x)$ are in $S,$ then the function $f(x)g(x)$ is also in $S.$
	
	\textbf{Solution.} Now  suppose that $f(x)\in S$ and $g(x)\in S$, we have $f_2(f(x))=\ln(f(x)+1)\in S$ and $f_2(g(x))=\ln(g(x)+1)\in S$. By (ii), we can add them up and obtain $\ln(f(x)+1)\ln(g(x)+1)=\ln((f(x)+1)(g(x)+1))\in S$ and by (i) and (ii) again, $f_1(\ln((f(x)+1)(g(x)+1)))
	=e^{\ln((f(x)+1)(g(x)+1))}-1=(f(x)+1)(g(x)+1)-1=f(x)g(x)+f(x)+g(x)\in S$. By (ii) we have $f(x)+g(x)\in S$ and since $f(x), g(x)$ are both nonnegatively valued, $f(x)g(x)\ge 0$ and so $f(x)g(x)=(f(x)g(x)+f(x)+g(x))-(f(x)+g(x))\in S$. 
	
	\item[\textbf{B2}] Let $P$ be a given (non-degenerate) polyhedron. Prove that there is a constant $c(P)>0$ with the following property: If a collection of $n$ balls whose volumes sum to $V$ contains the entire surface of $P,$ then $n>c(P)/V^2.$
	
	\textbf{Solution.} The power-mean inequality says that, if $r_1, r_2, \cdots , r_n$ are the radii of the given circles, then considering the quadratic mean and cubic mean gives
	\[\sqrt[3]{\frac{\sum_{i=1}^n r_i^3}{n}}\ge \sqrt{\frac{\sum_{i=1}^n r_i^2}{n}}\]
	Raising each part to the sizth power gives $\left(\dfrac{\sum_{i=1}^n r_i^3}{n}\right)^2\ge \left(\dfrac{\sum_{i=1}^n r_i^2}{n}\right)^3$, which means $n(\sum_{i=1}^n r_i^3)^2\ge (\sum_{i=1}^n r_i^2)^3\cdots (*)$. 
	
	Having established (*), denote $A$ as the surface area of our polyhedron, $k$ as the number of faces of the polyhedron, and $A_i$ as the total surface area covered by the spheres. Since the spheres jointly cover the polyhedron, we must have $\sum_{i=1}^n A_i\ge A$. On the other hand, since the $i$-th sphere has radius $r_i$, for each face of the polyhedron, the coverage of that sphere with this surface is at most $\pi r_i^2$ (since each face is part of some plane in the space, and the interesection of a plane and a sphere of radius $r$, if they do intersect, is a circle, and has area at most $\pi r^2$.) This gives $A_i\le kr_i$, considering all faces. Thus we have the following: 
	\[A\le \sum_{i=1}^n A_i\le k\pi \sum_{i=1}^n r_i^2, \qquad \sum_{i=1}^n r_i^2\ge \left(\frac{A}{k\pi}\right)\]
	The total volume $V$ is given by $\frac 43 \pi \sum_{i=1}^n r_i^3$, so we have the following: 
	\[nV^2 = n\left(\frac 43 \pi \sum_{i=1}^n r_i^3\right)^2\ge \left(\frac 43 \pi \right)^2 \left(\sum_{i=1}^n r_i^2\right)^3 \ge \frac {16}{9} \pi ^2 \left(\frac{A}{k\pi}\right)^3=\frac{16}{9}\cdot \frac{A^3}{k^3\pi}\]
	The rightmost quantity above is positive since $P$ is not degenerate, and so we can take $c(P)$ to be any constant in the interval $\left(\frac{16}{9}\cdot \frac{A^3}{k^3\pi}\right)$. 
	
	\item[\textbf{B3}]A round-robin tournament among $2n$ teams lasted for $2n-1$ days, as follows. On each day, every team played one game against another team, with one team winning and one team losing in each of the $n$ games. Over the course of the tournament, each team played every other team exactly once. Can one necessarily choose one winning team from each day without choosing any team more than once?
	
	\textbf{Answer.} Yes. \\
	\textbf{Solution.} We will proceeed by inducting on $n$. The case where $n=1$ is trivial since we simply choose the winning team on the only match. Now suppose that for some $n\ge 2$, for any $k<n$ and any round-robin tournament of $2k$ teams of $2k-1$ days, we can always choose a unique winning team each day. We proceed with the following cases: 
	\begin{enumerate}
		\item Suppose that there is a $k\in [1, 2n-1]$ and a subset of $k$ teams such that, the number of days in which at least one of the $k$ teams win the game in that day is $\ell$ and $\ell < k$. If $k=1$ then this team (namely $T$) loses in all the $2n-1$ days, so we can choose the team that beats $T$ in each day, and these $2n-1$ teams selected are different. Otherwise, each of the matches between the $k$ teams have exactly 1 winner, so they must happen within the $\ell$ days. Considering the matches between Team $i$ and team $j$ for $j\in [1, k]\backslash \{i\}$, which must happen on $k-1$ different days, we know that $\ell = k-1$. By considering the matches between the $k$ teams, there are $\dbinom{k}{2}\div (k-1)=\frac{k}{2}$ matches in each of the $k-1$ days, hence $k$ must be even. By induction hypothesis, since $k<2n-1$, we can choose a winner in each of the $k-1$ days such that the $k-1$ winners chosen are all distinct, and are part of the $k$ teams we mentioned before. In the rest of the $2n-k$ teams, fix one of the teams from the $k$ teams (say, Team 1) which loses in all $2n-k$ matches. We then pick the player who beats Team 1 every day in the $2n-k$ days. Each of the $2n-k$ teams are different, and are also different from the $k-1$ teams we picked in the initial $k-1$ days. This concludes the inductive proof for this part. 
		
		\item Now suppose that for each $k<2n$ and for each subset of $k$ teams, the number of days in which at east one of the $k$ teams win the game in that day is at least $k$. Pick $2n-1$ teams from the $2n$ teams arbitrarily. Now consider the bipartite graph consisting the said $2n-1$ teams and $2n-1$ days, with two vertices being connected by an edge if and only if the team wins on the day. By assumption, any $k$ teams are paired jointly to at least $k$ days. By Hall's lemma, there exists a matching between the players and the days. This also finishes the inductive proof. 
	\end{enumerate}

	\item[\textbf{B4}] Suppose that $a_0=1$ and that $a_{n+1}=a_n+e^{-a_n}$ for $n=0,1,2,\dots.$ Does $a_n-\log n$ have a finite limit as $n\to\infty?$ (Here $\log n=\log_en=\ln n.$)
	
	\textbf{Answer.} Yes, and the limit is in fact 0. \\
	\textbf{Solution.} First we notice it's obvious that the sequence $\{a_n\}$ is increasing since $a_{n+1}>a_n$ for each $n$. 
	We also note that, the function $x+e^{-x}$ has derivative $1-e^{-x}$ and therefore has staionary point at $x=0$. For $x>0$ this derivative is positive, hence $x+e^{-x}$ is increasing when $x>0$. 
	Suppose that the limit doesn't exist. That is, either there exists $m>0$ such that $a_n-\log n\ge m$ for infinitely many $n$ or there exists $m<0$ such that $a_n-\log n\le m$ for infinitely many $n$. We will deal with each case separately. 
	
	In the first case, we show that if $n$ is any number such that $a_{n_0}-\log {n_0}\ge m$, then there must exists $n_1>n$ with $a_{n_1}-\log {n_1}< m$. Suppose otherwise, then $a_n-\log n\ge m$ for all $n\ge n_0$. For all such $n$ we have 
	\begin{flalign*}
		a_{n+1}-\log (n+1)
		&=(a_n-\log n) +e^{-a_n} - \log\left(\frac{n+1}{n}\right)
		\\&\le (a_n-\log n) +e^{-(m+\log n)} - \log\left(\frac{n+1}{n}\right)
		\\&= (a_n-\log n) + e^{-m}\cdot\frac 1n - \log\left(1+\frac{1}{n}\right)
	\end{flalign*}
	and on the other hand we have, for all $x\in (-1, 1)$, the Taylor expansion of $\log (1+x)$ as $\displaystyle\sum_{j=1}^{\infty} (-1)^{j-1}\dfrac{x^j}{j}$. In terms of Taylor approximation (into finite terms) this translates into $(1+h(x))x$ with $h(x)\to 0$ as $x\to 0$. Thus, 
	$a_{n+1}-\log (n+1)= (a_n-\log n)+\frac1n (e^{-m}-1 - h(\frac 1n))$. 
	Given that $e^{-m}<1$ and that $h(\frac 1n)\to 0$ as $n\to \infty$, for sufficiently large $n$ we have 
	$e^{-m}-1 - h(\frac 1n) < \frac{e^{-m}-1}{2}$. 
	If $n_0$ is such that all $n\ge n_0$ has such property, we have $(a_{n+1}-\log (n+1)) - (a_n - \log n) \le \frac 1n \cdot \frac{e^{-m}-1}{2}$ and therefore 
	\[
	(a_{n+k}-\log (n+k)) - (a_n - \log n) \le \frac{e^{-m}-1}{2}\sum_{i=0}^{k-1}\frac{1}{n+i}
	\]
	but then the series $\{\frac 1n\}$ diverges so as $k\to\infty$, $(a_{n+k}-\log (n+k)) - (a_n - \log n)\to -\infty$, which is a contradiction. Thus for all $n$ with $a_n-\log n\ge m$ there is an $n_1>n$ with $a_{n_1}-\log n_1 < m$. 
	
	Now with the claim above established, suppose, still, there are infinitely many $n$ such that $a_n-\log n\ge m$. For each such $n$, let $b(n)$ be any integer greater than $n$ with $a_{(b(n))} - \log b(n) < m$. 
	Notice also from earlier that $x+e^{-x}$ is increasing with $x>0$, and therefore 
	\begin{flalign*}
	a_{b(n) + 1} - \log (b(n) + 1) =
	&=a_{(b(n))} - \log b(n) +e^{a_{b(n)}} - \log (1+\frac{1}{b(n)})
	\\&\le m + e^{-m-\log b(n)}- \log (1+\frac{1}{b(n)})
	\\&=m + e^{-m}\cdot \frac{1}{b(n)}-\log (1+\frac{1}{b(n)})
	\\&=m + \frac{1}{b(n)}(e^{-m}-1-h(1/b(n)))
	\end{flalign*}
	with $h$ being the function introduced above. Using the assumption that there are infinitely many $n$ with such property, we can find $n$ such that $e^{-m}-1-h(1/n)<0$ as $e^{-m}<1$ and $h(1/n)\to 0$. 
	Therefore, $b(n)$ must also follow such property (as $b(n)>n$), and therefore $m + \frac{1}{b(n)}(e^{-m}-1-h(1/b(n))) < m$. 
	This means for $n$ sufficiently large, once $a_{b(n)} - \log (b(n)) < m$ we have $a_{b(n) + 1} - \log (b(n) + 1) < m$, and this trend stays forever, contradicting our assumption. 
	
	The opposite case is the similar except flipping sign. Now that $m<0$, we have $e^{-m}>1$ and the assumption that $a_n-\log n\le m$ for all $n\ge n_0$ leads to the case $a_{n+1}-\log (n+1) \ge (a_n-\log n) + e^{-m}\cdot\frac 1n - \log\left(1+\frac{1}{n}\right) = (a_n-\log n) + \frac 1n (e^{-m}-1-h(1/n))$ but then $h(1/n)\to 0$, and $\{\frac 1n\}$ diverges which will lead $a_{n}-\log n\to +\infty$, a contradiction. 
	Thus for each $n$ with $a_n-\log n\le m$ there's $b(n)$ with $a_{b(n)}-\log b(n)>m$, which also means (by the increasing property of $x+e^{-x}$) $a_{b(n) + 1} - \log (b(n) + 1) \ge m + \frac{1}{b(n)}(e^{-m}-1-h(1/b(n)))$ and for $n$ sufficiently large, $e^{-m}-1-h(1/b(n))>0$. Q.E.D. 
	
	\item[\textbf{B6}]Let $p$ be an odd prime number such that $p\equiv 2\pmod{3}.$ Define a permutation $\pi$ of the residue classes modulo $p$ by $\pi(x)\equiv x^3\pmod{p}.$ Show that $\pi$ is an even permutation if and only if $p\equiv 3\pmod{4}.$
	
	\textbf{Solution.} Since $\pi(0)=0$ (identity), removal of 0 from our consideration does not affect the parity of $\pi$. Hence we consider the numbers $1, 2, \cdots , p-1$, which are all invertible elements in $\mathbb{Z}_p$. We shall first say that $\pi$ is indeed a permutation since 3 is relatively prime to $p-1=\Phi(p)$. Now for each number $x$ we consider the permutation orbit formed by $x$, and the length of the orbit $\ell$ is the smallest positive integer such that $\pi^\ell{x}=x$. On the other hand $\pi^{k}(x)=x^{3^k}$ as defined (taking modulo $p$), so $\ell$ is the minimal positive $k$ satisfying $x^{3^{k}}=x$. Dividing by $x$ (allowed since $x$ is invertible in $\bbZ_p$) yields $x^{3^k - 1}\equiv 1\pmod{p}$ so we have $\ell$ as the smallest $k$ with $\ord_p(x)|3^k-1$, which means $\ell$ is the order of 3 modulo $\ord_p(x)$. 
	
	Now that $\Phi(p)=p-1$, consider the primitive root $g$ and $\{1, 2, \cdots , p-1\}=\{1, g, g^2, \cdots , g^{p-2}\}$. The order of $g^k$ is the minimal $\ell$ such that $p-1|k\ell$, which is $\frac{p-1}{\gcd(p-1, k)}$. Therefore for each divisor $x$ of $p-1$ the number of elements with order $x$ is $\phi(x)$. As discussed above, for each number $y$ with order $x$ modulo $p$, the length of orbit of $y$ is $\ord_x(3)$, which means that the number of different orbits containing all numbers of order $x$ is $\frac{\phi(x)}{\ord_x(3)}$, and here comes the total number of orbits arising fro $\pi$: 
	\[\sum_{x|p-1} \frac{\phi(x)}{\ord_x(3)}\]
	The next task is to determine the parity of this number, which we will do this by grouping all the divisors of $p-1$ based on the largest odd factor of $x$. To be precise, if $q$ is the largest exponent of 2 dividing $p-1$ then for each off factor $y$ of $p-1$ we consider the sum $\sum_{j=0}^q \frac{\phi(2^j y)}{\ord_{2^j y}(3)}$. Assuming $y\neq 1$, since $y$ is odd (relatively prime to $2^j$), $\phi(2^j y)=\phi(2^j)\phi(y)=2^{j-1}\phi(y)$ and $\ord_{2^j y}(3)=\lcm(\ord_{2^j}(3), \ord_{y}(3))$ (all assuming $j\ge 1$). Knowing that $q\ge 1$, $\lcm(\ord_{2^j}(3), \ord_{y}(3))|\ord_{2^j}(3)\ord_{y}(3)$, and $\ord_{y}(3)|\phi(y)$, $\ord_{2^j}(3)|\phi(2^j)$, we have the following observation: if $\frac{\phi(y)}{\ord_{y}(3)}$ is even, then for all $j\ge 1$ we have  
	\[\frac{\phi(2^j)\phi(y)}{\ord_{2^j}(3)\ord_{y}(3)}|\frac{\phi(2^j)\phi(y)}{\lcm(\ord_{2^j}(3), \ord_{y}(3))}=\frac{\phi(2^j y)}{\ord_{2^j y}(3)}\]
	and since $\frac{\phi(2^j)}{\ord_{2^j}(3)}$ is an integer, the expression above is even. Otherwise, $\frac{\phi(y)}{\ord_{y}(3)}$ is odd. We first notice that $2^1||3-1$, $2^3||3^2-1$ and if $2^k|3^{\ell}-1$ then $2^{k+1}|2(3^{2\ell}-1)|(3^{\ell}-1)(3^{\ell}+1)$ so we have, for $j\ge 2$, $\ord_{2^j} (3)\le 2^{j-2}$. This means that $2|\frac{\phi(2^j)}{\ord_{2^j} (3)}$. This leaves us with $j=0$ and $j=1$, and we assumed that for $j=0$ the fraction is odd. For $j=1$ we have $\phi(2y)=\phi(y)$ and $\ord_{2y}(3)=\ord_{y}(3)$ since for $k\ge 1$, $y|3^k-1$ implies $2y|3^k-1$. Hence the fraction is also odd when $j=1$. So this gives an even sum in total, considering $y$ fixed and $j=0, 1, \cdots , q$. 
	
	It remains to consider the case $y=1$, which gives us the following sum: 
	\[\sum_{j=0}^{q}\frac{\phi(2^q)}{\ord_{2^j}(3)}\]
	Now this sum is 1 when $j=0, 1, 2$ and 2 otherwise, so if $q\ge 2$ this sum is odd and if $q=1$ this sum is even. The first case corresponds to the case where $4|p-1$, and since we have an odd number of orbits, $\pi$ is odd. The second case corresponds to the case where $4\nmid p-1$ and since we have a n even number of orbits, $\pi$ is even. 
	
	\section{Putnam 2011}
	\item[\textbf{A1}] Define a growing spiral in the plane to be a sequence of points with integer coordinates $P_0=(0,0),P_1,\dots,P_n$ such that $n\ge 2$ and:
	
	• The directed line segments $P_0P_1,P_1P_2,\dots,P_{n-1}P_n$ are in successive coordinate directions east (for $P_0P_1$), north, west, south, east, etc.
	
	• The lengths of these line segments are positive and strictly increasing.
	
	\[\begin{picture}(200,180)
	
	\put(20,100){\line(1,0){160}}
	\put(100,10){\line(0,1){170}}
	
	\put(0,97){West}
	\put(180,97){East}
	\put(90,0){South}
	\put(90,180){North}
	
	\put(100,100){\circle{1}}\put(100,100){\circle{2}}\put(100,100){\circle{3}}
	\put(115,100){\circle{1}}\put(115,100){\circle{2}}\put(115,100){\circle{3}}
	\put(115,130){\circle{1}}\put(115,130){\circle{2}}\put(115,130){\circle{3}}
	\put(40,130){\circle{1}}\put(40,130){\circle{2}}\put(40,130){\circle{3}}
	\put(40,20){\circle{1}}\put(40,20){\circle{2}}\put(40,20){\circle{3}}
	\put(170,20){\circle{1}}\put(170,20){\circle{2}}\put(170,20){\circle{3}}
	
	\multiput(100,99.5)(0,.5){3}{\line(1,0){15}}
	\multiput(114.5,100)(.5,0){3}{\line(0,1){30}}
	\multiput(40,129.5)(0,.5){3}{\line(1,0){75}}
	\multiput(39.5,20)(.5,0){3}{\line(0,1){110}}
	\multiput(40,19.5)(0,.5){3}{\line(1,0){130}}
	
	\put(102,90){P0}
	\put(117,90){P1}
	\put(117,132){P2}
	\put(28,132){P3}
	\put(30,10){P4}
	\put(172,10){P5}
	
	\end{picture}\]
	
	
	How many of the points $(x,y)$ with integer coordinates $0\le x\le 2011,0\le y\le 2011$ cannot be the last point, $P_n,$ of any growing spiral?
	
	\textbf{Answer.} 10053. 
	
	\textbf{Solution.} For $1\le x< y$, we can use $|P_0P_1|=x$ and $|P_1P_2|=y$. For $(x, y)$ with $x\ge 3$ and $y\ge 4$ we can also use $|P_iP_{i+1}|=1, 2, 3, x+1, x+2, x+y-1$ for $1\le i\le 6$, then $x=1-3+x+2=x$ and $y=2-(x+1)+(x+y-1)=y$ (we need $x+1>3$ and $y-1>2$, so $x\ge 3$ and $y\ge 4$ at least 4 will work. )
	
	To show that these are the all the possible values, we first show that if $a_1<a_2<\cdots a_k$ are increasing sequences of positive numbers, then $\displaystyle\sum_{i=1}^k (-1)^{i-1}a_i$ is positive if $k$ is odd, and negative otherwise. If $k$ is odd, then we have $a_k>0$ and therefore $\displaystyle\sum_{i=1}^k (-1)^{i-1}a_i=(a_k-a_{k-1})+(a_{k-2}-a_{k-3})+\cdots + (a_3-a_2)+a_1$ with each of $a_i-a_{i+1}>0$. Similarly for for $k$ even we have $\displaystyle\sum_{i=1}^k (-1)^{i-1}a_i=-(a_k-a_{k-1})-\cdots -(a_2-a_1)$ and each term negative. 
	Now going back to the core lemma, each change in the coordinates (for each $x$- and $y$-coordinates) are in alternate directions, with magnitude increasing by at least 2 each time. Both start with a positive change, so there must be an odd number of changes for both $x$ and $y$ coordinates. This implies $n$ is congruent to 2 mod 4. 
	
	If $n=2$, then we have the $x$-coordinate as the length $P_0P_1$ and the $y$-coordinate as $P_1P_2$. In this case we need $x<y$, with $x\ge 1$. If $n\ge 6$, let $x_1, x_2, \cdots , x_{n/2}$ be the lengths of the $x$-segments, and we have the $x$-coordinate as $x_1-x_2+x_3-\cdots +x_{n/2}=(x_{n/2}-x_{n/2 - 1})+\cdots + x_1$. Since each term in the form $x_i-x_{i-1}$ must be at least 2, so is $(x_{n/2}-x_{n/2 - 1})$, with $x_1\ge 1$. This gives $x\ge 3$. Similarly, if $y_1, y_2, \cdots , y_{n/2}$ are the $y$-segments then each $(y_{n/2}-y_{n/2 - 1})\ge 2$ with $y_1\ge 2$, giving the lower bound for $y$-coordinate as 4. 
	
	Hence for each $0\le x\le 2011$, if $x=0$ then all $y\in [0, 2011]$ cannot be one such point (2012 values); if $1\le x\le 2$ then we need $y\ge x+1$ so $y=0, 1, \cdots , x$ are impossible, hence $x+1$ values. When $x\ge 3$, each $y\ge 4$ fits. However, those with $y>x$ also have $y\ge 4$, so $y=0, 1, 2, 3$ are the ones that cannot fit (4 values each). Hence the answer is $2012+2+3+\displaystyle\sum_{k=3}^{2011}4=2017+4(2009)=10053$. 
	
	\item[\textbf{A2}]Let $a_1,a_2,\dots$ and $b_1,b_2,\dots$ be sequences of positive real numbers such that $a_1=b_1=1$ and $b_n=b_{n-1}a_n-2$ for $n=2,3,\dots.$ Assume that the sequence $(b_j)$ is bounded. Prove that \[S=\sum_{n=1}^{\infty}\frac1{a_1\cdots a_n}\] converges, and evaluate $S.$
	
	\textbf{Answer.} We necessarily have $S=\frac 32$. \\
	\textbf{Solution.} Now we write $a_n=\dfrac{b_n+2}{b_{n-1}}$ for $n\ge 2$ so 
	$\dfrac{1}{a_1\cdots a_n}=\dfrac{1}{\prod_{k=1}^n a_k}=\dfrac{1}{\prod_{k=2}^n \dfrac{b_k+2}{b_{k-1}}}
	=\dfrac{1}{(b_n+2)\prod_{k=2}^{n-1}\left(1+\frac{2}{b_k}\right)}
	$
	This motivates us to do the following telescoping sum: we consider the difference $\dfrac 32 - \sum_{k=1}^{n}\dfrac1{a_1\cdots a_k}$ for each $n$. When $n=1$ we have $\dfrac 32 - \dfrac{1}{a_1}=\dfrac 32 - 1= \dfrac 12$ and when $n=2$ we have $\dfrac 12 - \dfrac{1}{b_2+2}=\dfrac{b_2}{2(b_2+2)}=\dfrac{1}{2(1+\frac{2}{b_2})}. $ We claim from here that $\dfrac 32 - \sum_{k=1}^{n}\dfrac1{a_1\cdots a_k} = \dfrac{1}{2\prod_{k=2}^n (1+\frac{2}{b_k})}$. Suppose that this is true for some $n$ (we have done base case $n=2$ above), then for $n+1$ we have 
	\begin{flalign*}
		\dfrac 32 - \sum_{k=1}^{n}\dfrac1{a_1\cdots a_k} 
		&=\dfrac{1}{2\prod_{k=2}^n (1+\frac{2}{b_k})} - \dfrac{1}{(b_{n+1}+2)\prod_{k=2}^{n}\left(1+\frac{2}{b_k}\right)}\\
		&=\dfrac{1}{\prod_{k=2}^n (1+\frac{2}{b_k})} \left(\dfrac 12 - \dfrac 1 {b_{n+1}+2}\right)\\
		&=\dfrac{1}{\prod_{k=2}^n (1+\frac{2}{b_k})} \left(\dfrac {b_{n+1}} {2(b_{n+1}+2)}\right)\\
		&=\dfrac{1}{\prod_{k=2}^{n+1} (1+\frac{2}{b_k})}
	\end{flalign*}
	and therefore we have $S=\dfrac 32 - \lim_{n\to\infty} \dfrac{1}{\prod_{k=2}^{n} (1+\frac{2}{b_k})}$. Since $(b_k)$ is bounded, there is $M$ positive such that $b_k\le M$ for each $k$. This means $\dfrac{1}{\prod_{k=2}^{n} (1+\frac{2}{b_k})}\le \dfrac{1}{\prod_{k=2}^{n} (1+\frac{2}{M})}=\dfrac 1{(1+\frac{2}{M})^{n-1}}$ and so 
	$\lim_{n\to\infty} \dfrac{1}{\prod_{k=2}^{n} (1+\frac{2}{b_k})}\le \lim_{n\to\infty} \dfrac 1{(1+\frac{2}{M})^{n-1}}\to 0$. So $S=\dfrac 32$. 
	
	\item[\textbf{A3}] Find a real number $c$ and a positive number $L$ for which
	\[\lim_{r\to\infty}\frac{r^c\int_0^{\pi/2}x^r\sin x\,dx}{\int_0^{\pi/2}x^r\cos x\,dx}=L.\]
	
	\textbf{Answer.} $c=-1$ and $L=\frac{2}{\pi}$.\\
	\textbf{Solution.} Denote $S_r=\int_0^{\pi/2}x^r\sin x\,dx$ and $C_r=\int_0^{\pi/2}x^r\cos x\,dx$. We first find the relation between $S_{r+1}$ and $S_r$ for each $r$. In fact, we will prove that $\lim_{r\to\infty} \frac{S_{r+1}}{S_r}=\frac{\pi}{2}$. First, for each $r$ we have 
	$S_{r+1}=\int_0^{\pi/2}x^{r+1}\sin x\,dx=\int_0^{\pi/2}x\cdot x^r\sin x\,dx\le \int_0^{\pi/2}\frac{\pi}{2}x^r\sin x\,dx
	=\frac{\pi}{2} S_r$. 
	On the other hand, we show that for each $\epsilon>0$, there exists $r_0$ such that $\frac{S_{r+1}}{s_r}>\frac{\pi}{2}-\epsilon$ for all $r\ge r_0$. 
	Now let $0<\delta<\epsilon$. We split $S_r$ into two parts: 
	$\int_0^{\pi/2-\delta}x^r\sin x\,dx$ and $\int_{\pi/2-\delta}^{\pi/2}x^r\sin x\,dx$. 
	Since $\sin x\le 1$ for all $x$, we have 
	\[\int_0^{\pi/2-\delta}x^r\sin x\,dx\le \int_0^{\pi/2-\delta}x^r\,dx
	=\frac{(\pi/2-\delta)^{r+1}}{r+1}
	\]
	and
	\[\int_{\pi/2-\delta}^{\pi/2}x^r\sin x\,dx
	\ge \int_{\pi/2-\delta}^{\pi/2}(\pi/2-\delta)^r\sin (\pi/2-\delta)\,dx
	=\delta(\pi/2-\delta)^r\sin (\pi/2-\delta)
	\]
	which means
	\[
	\dfrac{\int_{\pi/2-\delta}^{\pi/2}x^r\sin x\,dx}{S_r}
	\ge \frac{\delta(\pi/2-\delta)^r\sin (\pi/2-\delta)}{\delta(\pi/2-\delta)^r\sin (\pi/2-\delta)+\frac{(\pi/2-\delta)^{r+1}}{r+1}}
	=\frac{\delta\sin (\pi/2-\delta)}{\delta\sin (\pi/2-\delta)+\frac{\pi/2-\delta}{r+1}}
	\]
	We see that this ratio converges to 1 as $r\to\infty$, and since $\delta<\epsilon$, the ratio  $\dfrac{\int_{\pi/2-\delta}^{\pi/2}x^r\sin x\,dx}{S_r}>\frac{\pi/2-\epsilon}{\pi/2-\delta}$ for sufficiently large $r$. 
	Now we also have 
	\begin{flalign*}
		S_{r+1}&=\int_0^{\pi/2}x^{r+1}\sin x\,dx
		\\&>\int_{\pi/2-\delta}^{\pi/2}x^{r+1}\sin x\,dx
		\\&\ge \int_{\pi/2-\delta}^{\pi/2}(\pi/2-\delta)x^{r}\sin x\,dx
		\\&>(\pi/2-\delta)(\frac{\pi/2-\epsilon}{\pi/2-\delta})S_r
		\\&=(\pi/2-\epsilon)S_r
	\end{flalign*}
	with the last inequality holds true for sufficiently large $r$. This concludes the claim that $\frac{S_{r+1}}{S_r}>\frac{\pi}{2}-\epsilon$ for all sufficiently large $r$. Considering the fact that this holds for each $\epsilon>0$, we have $\lim_{r\to\infty}\frac{S_{r+1}}{S_r}=\frac{\pi}{2}$. 
	
	Now going back to the problem, by virtue of integration by parts we get $C_r=\int_0^{\pi/2}x^r\cos x\,dx
	=[\frac{x^{r+1}}{r+1}\cos x]_0^{\pi/2}+\int_0^{\pi/2}\frac{x^{r+1}}{r+1}\sin x\,dx
	=0+\frac{1}{r+1}S_{r+1}
	=\frac{S_{r+1}}{r+1}
	$
	and by the claim above we have $\frac{2}{\pi}=\lim_{r\to\infty}\frac{S_r}{S_{r+1}}
	=\lim_{r\to\infty}\frac{S_r}{(r+1)C_r}
	=\lim_{r\to\infty}\frac{S_r}{rC_r}
	\lim_{r\to\infty}\frac{r^{-1}S_r}{C_r}
	$
	since $\frac{r+1}{r}$ as $r\to\infty$. Thus $c=-1$ and $L=\frac{2}{\pi}$. 
	
	\item[\textbf{A4}] For which positive integers $n$ is there an $n\times n$ matrix with integer entries such that every dot product of a row with itself is even, while every dot product of two different rows is odd?
	
	\textbf{Answer.} When $n$ is odd. For this example we can use $A$ as the matrix full with ones, and return the answer $A-I$. (Basically, the $ij$-entry is 1 iff $i\neq j$). 
	
	\textbf{Solution.} It suffices to produce a contradiction when $n$ is even. Now, consider the matrix $A$ of $n\times n$ with the desired property, and it will be more useful to consider it in the $\bbZ_2$ space. Let $v$ be the $n\times 1$ matrix with all entries 1 (i.e. $\begin{pmatrix}1 & 1 & \cdots & 1\\ \end{pmatrix}^T$). Then $Av$ is contains the sum of entries of each row, which is essentially also the dot product of each row with itself in $\bbZ_2$. Hence, $Av=0$, and thus $v$ is in the null space of $A$ (also $v$ is nonempty). On the other hand, the $ij$-th entry of $AA^t$ is the dot product of the $i$-th and $j$-th row of $A$, and is therefore odd if $i\neq j$, and even otherwise. This gives $AA^t=B-I$ where $B$ is the $n\times n$ matrix with all ones. 
	
	Now $\det(AA^t)=\det(A)\det(A^t)=\det(A^t)\det(A)=\det(A^tA)$ and since $Av=0$, we have $A^tAv=0$ too, so $A^tA$ and $AA^t$ cannot be invertible in $\bbZ_2$. On the other hand, consider the matrix $B-I=AA^t$, and we claim that the determinant is odd by induction on $n$. Base case when $n=2$ and we have $B_2=\begin{pmatrix}
	0 & 1\\1 & 0\\
	\end{pmatrix}$ with determinant $-1$ (and hence odd). Now suppose that for some even $n$, $B_{n-2}$ has odd determinant. We consider $B_n$: 
	$\begin{pmatrix}
		0 & 1 & \cdots & 1\\
		1 & 0 & \cdots & 1\\
		\ddots \\
		1 & 1 & \cdots & 0\\
	\end{pmatrix}$. 
	Consider, now, $B_{1k}$ for $k>1$ where $B_{ij}$ is the matrix obtained by deleting $i$-th row and $j$-th column from $B$, and we have $\det(B)=\displaystyle\sum_{k=1}^n(-1)^{k-1}b_{1k}\det(B_{1k})=\displaystyle\sum_{k=2}^n\det (B_{1k})$ since $b_{1k}$ is 1 except for $b_{11}=0$, and also removing all the $(-1)^k$'s since we are doing $\bbZ_2$. Now each $C=B_{1k}$ for $k\ge 2$, the matrix has the following form: $c_{1j}=1$ for all $j$'s, and $c_{j\ell}=1$ with the exception when $j\ge 2$ and $\ell = j-1$ for $j<k$, and $\ell=j$, otherwise. Since row reduction preserves the determinant, we subtract every row by the first row. Since the first row is all ones, we essentially flipped all rows $2$ to $n-1$. Thus we now have $c_{j\ell}=0$ unless $j\ge 2$, and $\ell=j-1$ for $j<k$ and $\ell=j$, otherwise. 
	This means, there's exactly 1 nontrivial entries in each row $c_{j(j-1)}$ ($j<k$) or $c_{jj}$ ($j\ge k$), and each of them are in different rows and columns. Multiplying them with $c_{1(k-1)}=1$ gives the only possible contribution to the determinant of $C$, i.e. $\pm 1=1$ in $\bbZ_2$. Thus $\det(B)=\displaystyle\sum_{k=2}^n\det (B_{1k})=
	\displaystyle\sum_{k=2}^n 1 = n-1=1$ since $n$ is even. Thus now $B$ is invertible, which is a contradiction. 
	
	\item[\textbf{B1}] Let $h$ and $k$ be positive integers. Prove that for every $\varepsilon >0,$ there are positive integers $m$ and $n$ such that \[\varepsilon < \left|h\sqrt{m}-k\sqrt{n}\right|<2\varepsilon.\]
	
	\textbf{Solution.} We first show that for all $\varepsilon>0$, there exists $m$ and $n$ such that $0<\left|h\sqrt{m}-k\sqrt{n}\right|<2\epsilon$. 
	
	Let $d > 0$ be the greatest common divisor of $h^2$ and $k^2$. By Euclid's algorithm, there exists $m_0$ and $n_0$ such that $h^2m_0-k^2n_0=d$. And if $m_0$ and $n_0$ are such solutions, other solutions can be obtained by changing $(m_0, n_0)$ with $(m_0+xk^2/d, n_0+xh^2/d)$ for all $x\ge 0$. 
	
	We now proceed to another crucial observation: $\lim_{N\to \infty}\sqrt{N+d}-\sqrt{N}=0$. To this end, notice that for each $\varepsilon>0$, we have $(\sqrt{N}+\varepsilon)^2=N+\varepsilon^2 + 2\varepsilon\sqrt{N}>N+2\varepsilon\sqrt{N}$, so choosing $N$ such that $d<2\varepsilon\sqrt{N}$ (i.e. $N>(\frac{d}{4\varepsilon^2})$) we get $(\sqrt{N}+\varepsilon)^2 > N+d$ and therefore $\sqrt{N+d}-\sqrt{N}<\varepsilon$ for all such $N$. 
	This means, fixing $N_0$ such that $0<\sqrt{N+d}-\sqrt{N}<\varepsilon$ for all $N>N_0$ and choosing $x$ such that $n_0+xh^2/d>N$ we have $0<h\sqrt{m_0+xk^2/d} - k\sqrt{n_0+xh^2/d} < \varepsilon$. In other words, there exists $m_1$ and $n_1$ such that $0<h\sqrt{m_1} - k\sqrt{n_1}<\varepsilon$ (by assigning $m_1 = m_0+xk^2/d$ and $n_1 = n_0+xh^2/d$). 
	
	Finally, since $0 < h\sqrt{m_1} - k\sqrt{n_1}$, let $c = \varepsilon / h\sqrt{m_1} - k\sqrt{n_1}$. Consider the number $g = \lfloor c\rfloor + 1$. From the choices of $m_1$ and $n_1$, we also have $c>1$, and from $ c < g = \lfloor c\rfloor + 1\le c+1$ we have $1 < g/c < 2$. 
	Thus, making $m=g^2m_1$ and $n=g^2n_1$ we get 
	\[
	h\sqrt{m} - k\sqrt{n} = g(h\sqrt{m_1} - k\sqrt{n_1}) = \varepsilon\cdot (g/c)
	\]
	and with $g/c\in (1, 2)$ we gave $h\sqrt{m} - k\sqrt{n}\in (\varepsilon, 2\varepsilon)$. 
	
	\item[\textbf{B2}] Let $S$ be the set of all ordered triples $(p,q,r)$ of prime numbers for which at least one rational number $x$ satisfies $px^2+qx+r=0.$ Which primes appear in seven or more elements of $S?$
	
	\textbf{Answer.} 2 and 5\\
	\textbf{Solution.} We will use without proof that a rational solution exists to $px^2+qx+r=0$ if and only if the discriminant $q^2-4pr$ is a perfect square. In other words, we want to solve for $q^2-4pr=s^2$ with $s$ being an integer. Rearranging gives $(q-s)(q+s)=4pr$, with the prime factorization of $4pr$ being $2\time 2\times p\times r$. 
	
	If both $p$ and $r$ are 2, we have $(q-s)(q+s)$ is 16, so $(q-s, q+s)$ is either $(1, 16), (2, 8)$ or $(4, 4)$. The first one will force $q$ and $s$ to be non-integer; the second one gives $(q, s)$ as $(5, 3)$. The third example gives $(4, 0)$, neither of which is a prime. Thus the only possibility is $(p, q, r)=(2, 5, 2)$. 
	
	If one of them, say $p$ is 2 while $r$ prime, then $(q-s)(q+s)=8r$. Bearing in mind that $q-s\equiv q+s\mod 2$, both factors have to be even and therefore in the category of $(2, 4r), (4, 2r)$. Since $r>2$, we have $2r>4$. This forces $q, s$ to be $(2r+1, 2r-1)$ in the first case, and $(r+2, r-2)$ in the second case. Thus we have $(p, q, r)=(2, 2r+1, r)$, $(r, 2r+1, 2)$, $(2, r+2, r)$ or $(r, r+2, 2)$, condition on that $2r+1$ or $r+2$ actually being a prime. 
	
	If both $p$ and $r$ are odd primes, we have $(q-s)(q+s)=4pr=2p\times 2r$. Again both $q-s$ and $q+s$ are even, so $(q-s, q+s)$ are $(2, 2pr)$ or $(2p, 2r)$, assuming $p\le r$. The first case gives $(q, s)=(pr+1, pr-1)$ and the second case gives $(p+r, p-r)$. Notice, however, that this is hardly possible: since $p$ and $r$ are odd, $q=pr+1$ and $q=p+r$ are both odd, and greater than 2, hence cannot be even. 
	
	Thus a prime $r\not\in \{2, 5\}$ will appear two times when $2r+1$ is prime, when $r+2$ being a prime, when $\frac{r-1}{2}$ is a prime, when $r-2$ is a prime. If $r$ were to appear at least 7 times, then all conditions must hold. If $r\ge 7$, then one of $r-2, r, r+2$ must be divisible by 3, contradiction. Hence $r\ge 7$ is impossible. When $r=3$, $r-2=1$ is not prime. Now we claim that the primes 2 and 5 are possible: we have an example $(2, 5, 2)$ as above and since $2r+1=11, 5+2=7, 5-2=3$ are primes, we can do $(2, 11, 5), (5, 11, 2), (2, 7, 5), (5, 7, 2), (2, 5, 3), (3, 5, 2)$. These give the 7 occurences of 2 and 5. 
	
	\item[\textbf{B3}] Let $f$ and $g$ be (real-valued) functions defined on an open interval containing $0,$ with $g$ nonzero and continuous at $0.$ If $fg$ and $f/g$ are differentiable at $0,$ must $f$ be differentiable at $0?$
	
	\textbf{Answer.} Yes. \\
	\textbf{Solution.} We need to see if $\lim_{x\to 0}\frac{f(x)-f(0)}{x}$ is defined. By the rules of limits we have 
	\[
	\lim_{x\to 0}\frac{f(x)g(x)-f(0)g(0)}{x}=(fg)'(0)\]
	\[
	\lim_{x\to 0}\frac{f(x)g(0)-f(0)g(x)}{x}=\lim_{x\to 0}\frac{f(x)/g(x)-f(0)/g(0)}{x}\cdot \lim_{x\to 0} g(0)g(x)
	=(f/g)'(0) \cdot g(0)^2
	\]
	Adding the two limits up give
	\begin{flalign*}
		(fg)'(0) + (f/g)'(0) \cdot g(0)^2
		&=\lim_{x\to 0}\frac{f(x)g(x)-f(0)g(0)}{x} + \lim_{x\to 0}\frac{f(x)g(0)-f(0)g(x)}{x}
		\\&=\lim_{x\to 0}\frac{(f(x)-f(0))(g(x)+g(0))}{x} 
	\end{flalign*}
	and since $\lim_{x\to 0}g(x)+g(0)=2g(0)\neq 0$ (before $f$ is continuous at 0), we have 
	\begin{flalign*}
		f'(0)
		&=\lim_{x\to 0}\frac{f(x)-f(0)}{x}
		\\&=\lim_{x\to 0}\frac{(f(x)-f(0))(g(x)+g(0))}{x}\div \lim_{x\to 0}(g(x)+g(0))
		\\&=(fg)'(0) + (f/g)'(0) \cdot g(0)^2 \div 2g(0)
	\end{flalign*}
	as desired. 
		
\end{enumerate}

\section*{Putnam 2010}
\begin{enumerate}
	\item [\textbf{A1}] Given a positive integer $n,$ what is the largest $k$ such that the numbers $1,2,\dots,n$ can be put into $k$ boxes so that the sum of the numbers in each box is the same?
	
	[When $n=8,$ the example $\{1,2,3,6\},\{4,8\},\{5,7\}$ shows that the largest $k$ is at least 3.]
	
	\textbf{Answer.} $\lfloor\frac{n+1}{2}\rfloor$.\\
	\textbf{Solution.} The sum in each box is at least $n$ since this must be the case for the box containing $n$. Since the total sum of the $n$ numbers is $\frac{n(n+1)}{2}$, the number of boxes cannot exceed $\frac{n+1}{2}$, so the answer is at most $\lfloor\frac{n+1}{2}\rfloor$. 
	
	To see this is achievable, we split into cases where $n$ is even and $n$ is odd. When $n$ is odd, we can do $\{n\}, \{1, n-1\}, \{2, n-2\}, \cdots, \{\frac{n-1}{2}, \frac{n+1}{2}\}$. 
	When $n$ is even, the example $\{1, n\}, \{2, n-1\}, \cdots , \{\frac{n}{2}-1, \frac{n}{2}+1\}$ gives the example of $\frac{n}{2}=\lfloor\frac{n+1}{2}\rfloor$ boxes. 
	
	\item [\textbf{A2}]
	Find all differentiable functions $f:\mathbb{R}\to\mathbb{R}$ such that
	\[f'(x)=\frac{f(x+n)-f(x)}n\]
	for all real numbers $x$ and all positive integers $n.$
	
	\textbf{Answer.} All linear functions $f(x)=mx+c$ in which case $f'(x)=\frac{f(x+n)-f(x)}n=m$\\
	\textbf{Solution.} To show that $f$ must be linear, we notice that for each $x$, $f'(x)=f(x+1)-f(x)=\frac{f(x+2)-f(x)}2$ so this gives the relation $f(x+2)-f(x)=2(f(x+1)-f(x))$, and therefore $f'(x+1)=f(x+2)-f(x+1)=f(x+1)-f(x)=f'(x)$. 
	
	We now introduce the function $g(x)=f(x+1)-f(x)$. Notice that: 
	
	\begin{flalign*}
		g'(x)&=\lim_{\epsilon\to 0}\frac{g(x+\epsilon)-g(x)}{\epsilon}
		\\&=\lim_{\epsilon\to 0}\frac{f(x+1+\epsilon)-f(x+\epsilon)-f(x+1)+f(x)}{\epsilon}
		\\&=\lim_{\epsilon\to 0}\frac{f(x+1+\epsilon)-f(x+1)}{\epsilon}-\frac{f(x+\epsilon)-f(x)}{\epsilon}
		\\&=f'(x+1)-f'(x)
		\\&=0
	\end{flalign*}
	since $f'(x+1)=f'(x)$ for all $x$. Therefore $g$ is a constant function, meaning that, $f(x+1)-f(x)$ is constant. But since $f(x+1)-f(x)=f'(x)$, $f'(x)$ is also a constant, which means that $f$ has to be linear. 
	
	\item [\textbf{A3}]
	Suppose that the function $h:\mathbb{R}^2\to\mathbb{R}$ has continuous partial derivatives and satisfies the equation
	\[h(x,y)=a\frac{\partial h}{\partial x}(x,y)+b\frac{\partial h}{\partial y}(x,y)\]
	for some constants $a,b.$ Prove that if there is a constant $M$ such that $|h(x,y)|\le M$ for all $(x,y)$ in $\mathbb{R}^2,$ then $h$ is identically zero.
	
	\textbf{Solution.} Fix a point $(x_0, y_0)\in\bbR^2$. Denote $g(z)=h(x_0+az, y_0+bz)$. We have the following: 
	\begin{flalign*}
		g'(z)&=\lim_{\epsilon\to 0}\frac{g(z+\epsilon)-g(z)}{\epsilon}
		\\&=\lim_{\epsilon\to 0}\frac{h(x_0+a(z+\epsilon), y_0+b(z+\epsilon))-h(x_0+az, y_0+bz)}{\epsilon}
		\\&=a\frac{\partial h}{\partial x}(x_0+az,y_0+bz)+b\frac{\partial h}{\partial y}(x_0+az,y_0+bz)
		\\&=h(x_0+az,y_0+bz)
		\\&=g(z)
	\end{flalign*}
	So the condition $g'(z)=g(z)$ holds for all $z\in\bbR$. It follows from the identity of differential equation that the solution to $g$ is $g(z)=Ae^z$ for all $z\in\bbR$. Since $h$ us bounded in $\bbR^2$, so is $g$ and the only possibility is $A=0$. Thus $g\equiv 0$ and in particular, $g(0)=h(x_0, y_0)=0$. 

	
	\item [\textbf{A4}]
	Prove that for each positive integer $n,$ the number $10^{10^{10^n}}+10^{10^n}+10^n-1$ is not prime.
	
	\textbf{Solution.} Let $k$ be the maximum positive integer such that $2^k\mid n$. We claim that the number $10^{10^{10^n}}+10^{10^n}+10^n-1$ is divisible by $10^{2^k}+1$. It suffices to show that $10^{10^{10^n}}$ and $10^{10^n}$ are congruent to 1 modulo $p$ whereas $10^n$ congruent to 1. 
	
	Now, $n=d\cdot 2^k$ for some odd number $d$. Therefore $10^{n}=10^{d\cdot 2^k}\equiv (-1)^d=-1\pmod{10^{2^k}+1}$ since $d$ is odd. On the other hand, $10^{10^n}=10^{2^n\cdot 5^n}$ and from $n=d\cdot 2^k\ge 2^k>k$ we have $n\ge k+1$, so $2^{k+1}\mid 2^n\cdot 5^n$. Since $10^{2^{k+1}}\equiv(-1)^2=1\pmod{10^{2^k}+1}$ we have $10^{10^n}\equiv 1\pmod{10^{2^k}+1}$, and similarly for $10^{10^{10^n}}$. 
	
	Finally, it's not hard to see that $10^{10^{10^n}}+10^{10^n}+10^n-1 > 10^{2^k}+1$ because $n\ge 2^k$ and $10^{10^{10^n}}$ and $10^{10^n}$ are both strictly greater than $1$ (because $n$ is positive so $10^n>1$ and so is $10^{10^n}$ and $10^{10^{10^n}}$). Therefore
	$10^{10^{10^n}}+10^{10^n}+10^n-1 > 1+1+\cdot 10^n-1=10^n+1=10^{2^k}+1$. 
	
	\item [\textbf{A5}]
	Let $G$ be a group, with operation $*$. Suppose that
	
	(i) $G$ is a subset of $\mathbb{R}^3$ (but $*$ need not be related to addition of vectors);
	
	(ii) For each $\mathbf{a},\mathbf{b}\in G,$ either $\mathbf{a}\times\mathbf{b}=\mathbf{a}*\mathbf{b}$ or $\mathbf{a}\times\mathbf{b}=\mathbf{0}$ (or both), where $\times$ is the usual cross product in $\mathbb{R}^3.$
	
	Prove that $\mathbf{a}\times\mathbf{b}=\mathbf{0}$ for all $\mathbf{a},\mathbf{b}\in G.$
	
	\textbf{Solution.} Let $\mathbf{e}\in\bbR^3$ be the identity element in the group. If $\mathbf{e}\times\mathbf{a}\neq\mathbf{0}$ for some $\mathbf{a}\in G$, then $\mathbf{e}\times\mathbf{a}=\mathbf{e} * \mathbf{a}=\mathbf{a}$. Since both $\mathbf{e}$ and $\mathbf{a}$ are not parallel and nonzero, $\mathbf{a}=\mathbf{e}\times\mathbf{a}$ is perpendicular to $\mathbf{a}$, which is impossible for $\mathbf{a}$ nonzero. 
	Hence, for all $\mathbf{a}\in G$ we have $\mathbf{e}\times\mathbf{a}=\mathbf{0}$. 
	
	Now if $\mathbf{e}$ is not the zero vector, then from $\mathbf{e}\times\mathbf{a}=\mathbf{0}$ for all $\mathbf{a}\in G$ all such $\mathbf{a}$'s are either 0 or parallel to $\mathbf{e}$. In this case, the condition $\mathbf{a}\times\mathbf{b}=\mathbf{0}$ for all $\mathbf{a},\mathbf{b}\in G.$ We can now assume that $\mathbf{e}=\mathbf{0}$. 
	
	Suppose there are $\mathbf{a}$ and $\mathbf{b}\in G$ such that $\mathbf{a}\times\mathbf{b}\neq \mathbf{0}$. From (ii), $\mathbf{a}*\mathbf{b}=\mathbf{a}\times\mathbf{b}\in G$ and is perpendicular to $\mathbf{a}$ and $\mathbf{b}$, i.e. there exists a vector in $G$ that is perpendicular to $\mathbf{a}$. Thus we can choose $\mathbf{a}$ and $\mathbf{b}$ (in the beginning) such that they are perpendicular to each other, and we will now assume that $\mathbf{a}$ and $\mathbf{b}$ are perpendicular to each other. 
	
	Let $\mathbf{a}\times \mathbf{b}=\mathbf{c}=-\mathbf{b}\times\mathbf{a}$. Since $\mathbf{e}=0$, we have $(\mathbf{a}\times \mathbf{b}) * (\mathbf{b}\times\mathbf{a}) =\mathbf{0}$ and with $\mathbf{a}\times\mathbf{b}=\mathbf{a}*\mathbf{b}$ as of above (and similarly $\mathbf{b}\times\mathbf{a}=\mathbf{b}*\mathbf{a}$) we get 
	$\mathbf{0}=\mathbf{a}*\mathbf{b}*\mathbf{b}*\mathbf{a}$ (notice the omission of brackets because $*$ is associative). 
	``Multiplying'' both sides by $-\mathbf{a}$ from the left and $\mathbf{a}$ from the right gives $\mathbf{b}^2 * \mathbf{a}^2=\mathbf{0}$ (second power means multiplication by itself), i.e. $\mathbf{b}^2$ and $\mathbf{a}^2$ are inverses of each other. But since $\mathbf{c}$ is perpendicular to $\mathbf{a}$ and $\mathbf{b}$, a similar conclusion yields $\mathbf{c}^2$ and $\mathbf{a}^2$ are inverses of each other, and so are $\mathbf{c}^2$ and $\mathbf{b}^2$. With $\mathbf{a}^2$, $\mathbf{b}^2$ and $\mathbf{c}^2$ being inverses of one another, the only possibility is all of them being $\mathbf{0}$. 
	
	Since $\mathbf{a}$ and $\mathbf{c}$ are also perpendicular to each other, we get $\mathbf{a}\times \mathbf{c}=\mathbf{a} * \mathbf{c}=\mathbf{a} * \mathbf{a} * \mathbf{b}=\mathbf{0} * \mathbf{b}=\mathbf{b}$. Similarly, $\mathbf{c}\times \mathbf{b}=\mathbf{a}$. Denote $\mathbf{i}$, $\mathbf{j}$ and $\mathbf{k}$ as the vectors parallel to $x, y, z$ axes, respectively. By rotating $\mathbf{a}$ and $\mathbf{b}$, we can assume that $\mathbf{a}=x\mathbf{i}$ and $\mathbf{b}=y\mathbf{j}$, making $\mathbf{c}=xy(\mathbf{i}\times \mathbf{j})=(xy)\mathbf{k}$. 
	But then $y\mathbf{j}=\mathbf{b}=\mathbf{a}\times\mathbf{c}=x^2y\mathbf{i}\times\mathbf{k}=- x^2y\mathbf{j}$, so $y=-x^2y$, or $x^2=-1$ because $y\neq 0$. This is impossible since this forces $x=\sqrt{-1}$, which is imaginary. This contradiction shows that there cannot be $\mathbf{a}$ and $\mathbf{b}$ with nonzero cross products. 
	
	\item [\textbf{B1}]
	Is there an infinite sequence of real numbers $a_1,a_2,a_3,\dots$ such that
	\[a_1^m+a_2^m+a_3^m+\cdots=m\]
	for every positive integer $m?$
	
	\item [\textbf{B2}]
	Given that $A,B,$ and $C$ are noncollinear points in the plane with integer coordinates such that the distances $AB,AC,$ and $BC$ are integers, what is the smallest possible value of $AB?$
	
	\item [\textbf{B3}]
	There are 2010 boxes labeled $B_1,B_2,\dots,B_{2010},$ and $2010n$ balls have been distributed among them, for some positive integer $n.$ You may redistribute the balls by a sequence of moves, each of which consists of choosing an $i$ and moving exactly $i$ balls from box $B_i$ into any one other box. For which values of $n$ is it possible to reach the distribution with exactly $n$ balls in each box, regardless of the initial distribution of balls?
\end{enumerate}
\end{document}