\documentclass[11pt,a4paper]{article}
\usepackage{amsmath, amssymb, fullpage, mathrsfs, bm, pgf, tikz}
\usepackage{mathrsfs, float}
\usetikzlibrary{arrows}
\setlength{\textheight}{10in}
%\setlength{\topmargin}{0in}
\setlength{\topmargin}{-0.5in}
\setlength{\parskip}{0.1in}
\setlength{\parindent}{0in}

\begin{document}
	\newcommand{\la}{\leftarrow}
	\newcommand{\lra}{\leftrightarrow}
	\newcommand{\bbN}{\mathbb{N}}
	\newcommand{\bbZ}{\mathbb{Z}}
	\newcommand{\dsum}{\displaystyle\sum}
	\newcommand{\dprod}{\displaystyle\prod}
	
	
	\title{Solution to APMO 2022 Problems}
	\author{Anzo Teh}
	\date{}
	\maketitle
	
	\textbf{Problem 1.} 
		Find all pairs $(a,b)$ of positive integers such that $a^3$ is multiple of $b^2$ and $b-1$ is multiple of $a-1$.
		
		\textbf{Answer.} 
		All pairs in the form $a=b$, and also $b=1$. 
		
		\textbf{Solution.} We can easily show that the combinations above would work, 
		so it remains to show that these are the only combinations. 
		
		Now, let $a=p_1^{\alpha_1}\cdots p_k^{\alpha_k}$ (i.e. prime factorize), 
		then $b=p_1^{\beta_1}\cdots p_k^{\beta_k}$ with $0\le \beta_i\le \frac 32\alpha_i$ to satisfy the condition $b^2|a^3$. 
		In addition, $\gcd(p_i, a-1)=1$ for all the primes $p_i$, so we have 
		\[
		p_1^{\beta_1}\cdots p_k^{\beta_k}\equiv 1\pmod{a-1}
		\]
		and since we already have 
		\[
		p_1^{\alpha_1}\cdots p_k^{\alpha_k}\equiv 1\pmod{a-1}
		\]
		we can take multiplicative inverse to get 
		\[
		p_1^{\beta_1-\alpha_1}\cdots p_k^{\beta_k-\alpha_k}\equiv 1\pmod{a-1}
		\]
		which then becomes 
		\[
		\prod_{\beta_i\ge\alpha_i}p_i^{\beta_i-\alpha_i}
		\equiv \prod_{\beta_i<\alpha_i}p_i^{\alpha_i-\beta_i}\pmod{a-1}
		\]
		Now the LHS is at most $\prod_{i}p_i^{\alpha_i}=\sqrt{a}$, 
		and RHS is at most $a$ (both sides are at least 1). Also the gcd of the two sides is 1 so the only possibilities are: 
		\begin{itemize}
			\item LHS=RHS=1, which implies $\alpha_i=\beta_i$ for all $i$ and so $a=b$
			
			\item $|LHS-RHS|\ge a-1$, the only possibility is when $b=1$. 
		\end{itemize}
	    
	    \textbf{Problem 2.}
	    Let $ABC$ be a right triangle with $\angle B=90^{\circ}$. Point $D$ lies on the line $CB$ such that $B$ is between $D$ and $C$. Let $E$ be the midpoint of $AD$ and let $F$ be the seconf intersection point of the circumcircle of $\triangle ACD$ and the circumcircle of $\triangle BDE$. Prove that as $D$ varies, the line $EF$ passes through a fixed point.
	    
	    \textbf{Solution.} 
	    Let $EF$ intersect line$ BC$ at $G$, and we claim that $BC=BG$, 
	    i.e. $EF$ will always pass through the reflection of $C$ in $B$ regardless of $D$. 
	    
	    We see that by focusing on the triangles $ACD$ and $BDE$, 
	    the point $F$ is the center of spiral similarity that brings $FEB$ to $FAC$, so 
	    \[
	    \frac{FE}{FB}=\frac{EA}{BC}=\frac{EB}{BC}
	    \]
	    where the equality $EA=EB=ED$ follows from that $\angle B=90^{\circ}$. 
	    Now, the inversion with center $E$ and radius $ED=BE$ maps line $CG$ to circle $EDB$ 
	    (and vice versa), so $F$ and $G$ are mapped to each other in this inversion, resulting in 
	    \[
	    \frac{EF}{EB} = \frac{EB}{EG}
	    \]
	    i.e. we have triangles $EFB$ and $EBG$ similar. Therefore 
	    \[
	    \frac{FE}{FB}=\frac{BE}{BG}=\frac{BE}{BC}
	    \]
	    i.e. $BG=BC$ as desired. 
	    
	    \textbf{Problem 3.} 
	    Find all positive integers $k<202$ for which there exist a positive integers $n$ such that
	    $$\bigg {\{}\frac{n}{202}\bigg {\}}+\bigg {\{}\frac{2n}{202}\bigg {\}}+\cdots +\bigg {\{}\frac{kn}{202}\bigg {\}}=\frac{k}{2}$$
	    
	    \textbf{Answer.} 
	    1, 100, 101, 201. 
	    
	    \textbf{Solution.} 
	    Let's first give the constructions for those: we have these working pairs $(k, n)$: 
	    \[
	    (1, 101), (100, 2), (101, 99), (201, 1)
	    \]
	    The cases for $k=1, 100, 201$ are pretty clear, let's show for $k=101$, where, the numerators (in that order) are 
	    \[
	    99,198,95,194,\cdots, 3, 102, 201, 98, \cdots, 105, 2, 101
	    \]
	    which can be rearranged into 
	    \[
	    2+\cdots + 6 + \cdots +198+(3+\cdots + 99 + (101+105+\cdots + 201)
	    =100\cdot 50+51\cdot 25+\cdot 151\cdot 26
	    =101^2
	    \]
	    as desired. 
	    
	    To show that the others won't work, we first notice that the condition implies 
	    \[
	    n+2n+\cdots + kn\equiv 101k
	    \]
	    so we need $n\frac{k(k+1)}{2}$ to be divisible by 101. 
	    The only $k$ with $101\mid k(k+1)$ are 100, 101, 201, so 
	    we may now assume that $101\mid n$, 
	    which then leaves us with $n=101$. 
	    From here, we see that only $k=1$ works since the sequence alternates in $\frac 12, 0, \frac 12, 0, \cdots$. 
	    
	    \textbf{Problem 4.}
	    Let $n$ and $k$ be positive integers. Cathy is playing the following game. There are $n$ marbles and $k$ boxes, with the marbles labelled $1$ to $n$. Initially, all marbles are placed inside one box. Each turn, Cathy chooses a box and then moves the marbles with the smallest label, say $i$, to either any empty box or the box containing marble $i+1$. Cathy wins if at any point there is a box containing only marble $n$.
	    Determine all pairs of integers $(n,k)$ such that Cathy can win this game.
	    
	    \textbf{Answer.} 
	    All $n$ with $n\le 2^{k-1}$. 
	    
	    \textbf{Solution.} 
	    We consider the following observations: 
	    
	    \begin{itemize}
	    	\item \textbf{Consecutive}. 
	    	For each box, there exists $i\le j$ such that $i, i+1, \cdots, j$ are all the balls in the box. 
	    	This is an invariant that will be maintained via the valid moves. 
	    	Thus the assignment of balls into boxes can now be seen as partition of $n$ into at most $k$ parts where the goal is to make the last part having size 1.
	    	
	    	\item \textbf{Reversibility}. 
	    	Fix $i$, and the two types of moves (moving $i$ to an empty box vs to the box containing $i+1$) are reverse of each other. 
	    	
	    	\item \textbf{Dominance}. 
	    	If $(n, k)$ is a winning pair for Cathy then so is $(n-1, k)$. 
	    	To see why, the size of the last part cannot decrease by more than 1 for each iteration. 
	    	Thus if we have a winning algorithm for $(n, k)$, we stop when the last part has size 2 and that's exactly the winning algorithm for $(n-1, k)$ (i.e. ignoring the ball of label $n$). 
	    \end{itemize}
        
        Before we proceed, one other quantity is that, given $\ell<k$, 
        what are the maximum number of marbles we can have in the first $\ell$ (or fewer) parts that exclude the last part. 
        Let's also claim that this number satisfies the dominance property. 
        Observe that, if $(n_0, \ell, k)$ is a winning triple 
        (i.e. there's an algorithm to dump $n_0$ marbles to the first $\ell$ parts excluding the last part), 
        then a winning strategy for $(n_0-1, \ell, k)$ will proceed like $(n_0, \ell, k)$ except ignoring the first marble (and hence will always use up no more boxes at any time). 
        This means it suffices to consider the maximum $n$ in the following context: 
        \begin{itemize}
        	\item $f(k)$ the maximum $n$ where Cathy can win with $k$ boxes. 
        	
        	\item For $\ell<k$, $g_{\ell}(k)$ the maximum number of balls in the first $\ell$ parts (and excluding the last part). 
        \end{itemize}
	    We note that $f(k)=g_{k-1}(k)+1$, and proceed to the following. 
	    
	    \emph{Lemma.} For all $\ell\ge 1$, $g_{\ell}(k+\ell) = \sum_{j=0}^{\ell-1}f(k+j)$. 
	    
	    Proof: we first show $f(k)=g_{1}(k+1)$. 
	    Consider, first, any algorithm that puts $g_1(k+1)$ balls into the first block (when we have $k+1$ boxes in total), 
	    let this final configuration be $B$. 
	    Let the initial configuration ($n$ balls all in one box) as $A$. 
	    Then by the reversibility of the iterations, it's possible to go from $B$ to $A$. 
	    Now, in the $B\to A$ algorithm, consider $C$, the iteration immediately before the first time where all parts except the last has fewer than $g_1(k+1)$ balls 
	    (i.e. the last part now has $n-g_1(k+1)+1$ after $C$). 
	    Then the second last part of $C$ has 1 ball, 
	    and so $B\to C$ is a winning algorithm for $(g_1(k+1), k)$, and so $g_1(k+1)\ge f(k)$. 
	    
	    Conversely, consider the setting of $k+1$ boxes, and let's suppose we have $f(k)$ balls all in one box, and $n-f(k)$ balls in the rest (for arbitrary $n$). 
	    Let's now show that we can now combine all $n$ balls into one box. 
	    Iteratively, we do the following: 
	    \begin{itemize}
	    	\item With $x\le f(k)$ balls in all except last part, iterate until the second last part has $\le 1$ ball; 
	    	
	    	\item now combine the second last and the last part; 
	    	
	    	\item By dominance and reversibility, we can now make the first $x-1$ balls into one part again. Now repeat. 
	    \end{itemize}
        
        To extend this for different $\ell$, 
        we first show one side of the bound $g_{\ell}(k+\ell)\ge \sum_{j=0}^{\ell-1} f(k+j)$. 
        Indeed, with $k$ balls, 
        first place $f(k-1)$ balls into the first box, 
        and then with $k-2$ empty ones in the middle, 
        place $f(k-2)$ balls from the last part into the second part (leaving the first one untouched), 
        etc. 
        
        For the converse bound, suppose otherwise. 
        Then consider the minimum $k$ such that there's an $\ell$ such that 
        $g_{k}(\ell) > f(k-1)+\cdots + f(k-\ell)$. 
        We may also consider the minimum $\ell$ for this $k$ that it's possible for this to happen; 
        this way, the middle part (i.e. excluding the first $\ell-1$ and the last) would have $>$ 
        $f(k-\ell)$ balls (with all in one box at some point). 
        This means, we must be able to find an algorithm to make the second last part as 1 with only the middle $k-\ell$ boxes, contradicting the definition of $f(k-\ell)$. $\square$
        
        Now with the lemma shown, 
        we can now use the observation that $f(1)=1$ to conclude rather immediately that $f(k)=2^{k-1}$ and $g_\ell(k+1)=2^k - 2^{k-\ell}$. 
	    
	    \textbf{Problem 5.}
	    Let $a,b,c,d$ be real numbers such that $a^2+b^2+c^2+d^2=1$. Determine the minimum value of $(a-b)(b-c)(c-d)(d-a)$ and determine all values of $(a,b,c,d)$ such that the minimum value is achived.
	    
	    \textbf{Answer.} $\dfrac{-1}{8}$, realized by the following: 
	    \[
	   \left(-\frac{\sqrt{3}}{4} - \frac 14, -\frac{\sqrt{3}}{4} + \frac 14, 
	    +\frac{\sqrt{3}}{4} - \frac 14, +\frac{\sqrt{3}}{4} + \frac 14\right)
	    \]
	    assigned to $(a, b, c, d), (d, c, b, a)$, and all their cyclic shifts. 
	    
	    \textbf{Solution.} 
	    It now remains to show that $-\frac 18$ is the optimal value. 
	    Since it's negative, among $a-b, b-c, c-d, d-a$, either exactly 1 or 3 are positive. 
	    This means that there's a cyclic shift among $(a, b, c, d)$ that's either monotonically increasing, 
	    or monotonically decreasing. 
	    These cases are symmetric, so we may assume $a<b<c<d$. 
	    
	    Next, let's show that we can consider only the case $a+b+c+d=0$. 
	    Indeed, let $m$ be such that $a+b+c+d=4m$, 
	    and consider the mapping $f:x\to \frac{x-m}{\sqrt{1-4m^2}}$. 
	    Then 
	    \[
	    f(a)^2+f(b)^2+f(c)^2+f(d)^2=1
	    \qquad 
	    \]\[
	    (f(a)-f(b))(f(b)-f(c))(f(c)-f(d))(f(d)-f(a)) 
	    = \frac{(a-b)(b-c)(c-d)(d-a)}{(1-4m^2)^2}
	    \le (a-b)(b-c)(c-d)(d-a)
	    \]
	    since $(a-b)(b-c)(c-d)(d-a)<0$ and $1-4m^2\le 1$. 
	    
	    We now show that we can consider only the case where $|d-c|=|b-a|$, 
	    which now becomes $d-c=b-a$ since $a<b<c<d$. Indeed, 
	    consider, now, the tuples $(a', b', c', d')$ with sum 0,  $c-b=c'-b'$ and $d'-c'=b'-a'=\frac{b-a+d-c}{2}$
	    Then $d'-a'=d-a$. 
	    In addition we have $a=-d$ and $c=-b$, so 
	    \[
	    a^2+b^2+c^2+d^2
	    \ge \frac{(d-a)^2}{2}+\frac{(c-b)^2}{2}
	    =a'^2+b'^2+c'^2+d'^2
	    \]
	    while 
	    \[
	    (a-b)(b-c)(c-d)(d-a)
	    =(a-b)(b'-c')(c-d)(d'-a')
	    \ge (a'-b')(b'-c')(c'-d')(d'-a')
	    \]
	    where we used $(a-b)(c-d)\le (\frac{d-c+b-a}{2})^2=(a'-b')(c'-d')$ 
	    (since $a-b$ and $c-d$ have the same sign). 
	    Thus by rescaling we can actually attain 
	    \[
	    \frac{1}{(a'^2+b'^2+c'^2+d'^2)^2}(a'-b')(b'-c')(c'-d')(d'-a')
	    \le (a'-b')(b'-c')(c'-d')(d'-a')\le (a-b)(b-c)(c-d)(d-a)
	    \]
	    proving the claim. 
	    
	    Now that $a+b+c+d=0$ and $d-c=b-a$, we have $a=-d$ and $b=-c$, 
	    so $(a, b, c, d)=(-y, -x, x, y)$ for some $x, y>0, y>x$ with $x^2+y^2=\frac 12$. 
	    This gives 
	    \[
	    (a-b)(b-c)(c-d)(d-a)
	    =-4(x-y)^2xy
	    =-4\left(\frac 12 - 2xy\right)xy
	    =8\left((xy-\frac 18)^2 - \frac{1}{64}\right)
	    \ge -\frac 18
	    \]
	    which shows $-\frac 18$ is indeed optimal. 
	    Equality holds when we have $x^2+y^2=\frac 12$ and $xy=\frac 18$, 
	    i.e. $(x+y)^2=\frac 34$ and $(x-y)^2=\frac 14$. 
	    Using $y>x>0$ we can solve these to get $y=\frac{\sqrt{3}}{4}+\frac 14$ and 
	    $x=\frac{\sqrt{3}}{4}-\frac 14$. 
	    
\end{document}