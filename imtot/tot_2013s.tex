\documentclass[11pt,a4paper]{article}
\usepackage{amsmath, amssymb, fullpage, mathrsfs, bm, pgf, tikz}
\usepackage{mathrsfs, algorithm, algpseudocode}
\usetikzlibrary{arrows}
\setlength{\textheight}{10in}
%\setlength{\topmargin}{0in}
\setlength{\topmargin}{-0.5in}
\setlength{\parskip}{0.1in}
\setlength{\parindent}{0in}

\begin{document}
\newcommand{\la}{\leftarrow}
\newcommand{\lra}{\leftrightarrow}
\newcommand{\bbN}{\mathbb{N}}
\newcommand{\bbZ}{\mathbb{Z}}
\newcommand{\dsum}{\displaystyle\sum}
\newcommand{\dprod}{\displaystyle\prod}


\title{Solutions to Tournament of Towns, Spring 2013, Senior}
\author{Anzo Teh}
\date{}
\maketitle

\section*{O-Level}
\begin{enumerate}
	\item[1.] 
	There is a positive integer $A$. Two operations are allowed: increasing this number by
	9 and deleting a digit equal to 1 from any position. Is it always possible to obtain $A + 1$ by
	applying these operations several times?
	
	\textbf{Answer.} Yes. 
	
	\textbf{Solution.} Starting from $A$, we can apply operation 1 repeatedly to get $A+9k$. 
	Now consider the number
	\[
	B=(A+1)\underbrace{11\cdots 1}_{8\text{ one's}}
	\]
    By considering the sum of digits and congruence mod 9, we have $B\equiv A+1+9=A\pmod{9}$. 
    So we can attain $B$. 
    Now  apply operation 2 for 8 times from the ones digits to finish. 
    
    \item [2.] 
    Let $C$ be a right angle in triangle $ABC$. On legs $AC$ and $BC$ the squares $ACKL$,
    $BCMN$ are constructed outside of triangle. If $CE$ is an altitude of the triangle prove that
    $LEM$ is a right angle.
    
    \textbf{Solution.} TODO
    
    \item [3.] 
    Eight rooks are placed on a $8 \times 8$ chessboard, so no two rooks attack one another.
    All squares of the board are divided between the rooks as follows. 
    A square where a rook is placed belongs to it. 
    If a square is attacked by two rooks then it belongs to the nearest rook; 
    in case these two rooks are equidistant from this square then each of them possesses a half of the square. Prove that every rook possesses the equal area.
    
    \textbf{Solution.} 
    Consider a rook $R_1$, and let the other rooks be $R_2, R_3, \cdots, R_8$. 
    For each $R_i$ with $i\ge 2$, consider the two squares $A_i$ and $B_i$ attacked by both $R_1$ and $R_2$. 
    Thus $A_i, R_1, B_i, R_i$ form a rectangle with $A_i, B_i$ being the opposite corner. 
    If  this rectangle is not a square then exactly one of $A_i$ and $B_i$  is closer to $R_1$ than $R_i$, 
    while the other further. 
    If it's a square, both $A_i, B_i$ are equidistant to $R_1, R_i$. 
    Thus $R_1$ owns exactly area 1 from $A_i$ and $B_i$. 
    
    Since no rooks attack each other, $A_2, \cdots, A_8$ and $B_2, \cdots, B_8$ 
    are all distinct, and represent the 14 squares attacked by $R_1$. 
    It then follows that $R_1$ owns area 7 from these squares, so adding the grid it occupies, $R_1$ owns area 8 in total. 
    
    \item [4.] 
    Each of 100 stones has a sticker showing its true weight. No two stones weigh the
    same. Mischievous Greg wants to rearrange stickers so that the sum of the numbers on the
    stickers for any group containing from 1 to 99 stones is different from the true weight of this
    group. Is it always possible?
    
    \textbf{Answer.} Yes. 
    
    \textbf{Solution.} 
    Let $a_1<\cdots < a_{100}$ be the original weights of the stones. 
    Rearrange such that $a_1$ gets $a_{100}$ and $a_{i+1}$ gets $a_i$ for all $i\ge 1$. 
    Then for any subset, if $a_1$ is not chosen, 
    the stickers are $\sum a_i$ with true weights $\sum a_{i+1}$ means the stickers are less than true weight. 
    Similarly by symmetry, if $a_1$ is chosen, 
    then the stickers are necessarily more than the true weight. 
    
    \item[5.] 
    A quadratic trinomial with integer coefficients is called admissible if its leading coefficient is 1, its roots are integers and the absolute values of coefficients do not exceed 2013.
    Basil has summed up all admissible quadratic trinomials. Prove that the resulting trinomial
    has no real roots. 
    
\end{enumerate}

\section*{A-Level}
\begin{enumerate}
	\item[1.] 
	Several positive integers are written on a blackboard. The sum of any two of them is a
	positive integer power of two (for example, 2, 4, 8, . . .). What is the maximal possible number of
	different integers on the blackboard?
	
	\textbf{Answer.} 2. 
	
	\textbf{Solution.} The example can be achieved by taking $1, 3$. 
	To show we cannot do better, consider any 3 different integers on the blackboard $a<b<c$. 
	Then $b+c=2^x$ and $a+c=2^y$. 
	With $x>y$ we have $b+c=2^x\ge 2^{y+1}=2\cdot 2^y=2(a+c)$
	so $b=2a+c$. This violates $a<b<c$. 
	
	\item[2.] 
	A boy and a girl were sitting on a long bench. 
	Then twenty more children one after another came to sit on the bench, 
	each taking a place between already sitting children. Let us call a girl brave if she sat down between two boys, and let us call a boy brave if he sat down between two girls. 
	It happened, that in the end all girls and boys were sitting in the alternating order. 
	Is it possible to uniquely determine the number of brave children?
	
	\textbf{Answer.} Yes, and there's 10. 
	
	\textbf{Solution.} 
	Transform the boys into 0 and girls into 1, 
	and treat this as a string insertion game. 
	At each stage, consider the parts of the string as the minimum number we need to partition the string into such that each partition has the same character. 
	E.g. in 001011 there are four parts: 00, 1, 0, 11. 
	
	We see that if we insert 0, we have the following configuration: for brave: 
	\[
	11\to 101
	\]
	I.e. it splits the part of 1's into two and introduce a new single part 0 (hence the number of parts increase by 2). 
	For non-brave:
	\[
	00\to 000, 10\to 100, 01\to 001
	\]
	so in each case the number never changes. 
	Therefore the number of brave character is just half the number of increase of the parts. 
	
	Finally, there are two parts in the beginning (i.e. 01 or 10), 
	while in the end we have $1010\cdots 10$ or $0101\cdots 01$, i.e. 22 parts. 
	Thus the number of brave children is $(22-2)/2=10$. 
	
\end{enumerate}
\end{document}