\documentclass[11pt,a4paper]{article}
\usepackage{amsmath, amssymb, fullpage, mathrsfs, bm, pgf, tikz}
\usepackage{mathrsfs, algorithm, algpseudocode}
\usetikzlibrary{arrows}
\setlength{\textheight}{10in}
%\setlength{\topmargin}{0in}
\setlength{\topmargin}{-0.5in}
\setlength{\parskip}{0.1in}
\setlength{\parindent}{0in}

\begin{document}
\newcommand{\la}{\leftarrow}
\newcommand{\lra}{\leftrightarrow}
\newcommand{\bbN}{\mathbb{N}}
\newcommand{\bbZ}{\mathbb{Z}}
\newcommand{\dsum}{\displaystyle\sum}
\newcommand{\dprod}{\displaystyle\prod}


\title{Solutions to Tournament of Towns, Fall 2013, Senior}
\author{Anzo Teh}
\date{}
\maketitle

\section*{O-Level}
\begin{enumerate}
	\item[1.] 
	Does there exist a ten-digit number such that all its digits are different and after removing any six digits we get a composite four-digit
	number?
	
	\textbf{Answer.} Yes. 
	
	\textbf{Solution.} Pick 1397246850. 
	We first note that removing 246850 gives 1397=127$\times$11. 
	In other cases, at least one of 2, 4, 6, 8, 5, 0 will remain so the last digit would be 2, 4, 6, 8, 5 or 0, 
	resulting in a number divisible by 2 or 5. 
	Finally, 0 can never be a leading digit, so the resulting number will always have 4 digits. 
	
	\item[2.]
	On the sides of triangle $ABC$, three similar triangles are constructed with triangle $Y BA$ and triangle $ZAC$ in the exterior and triangle $XBC$ in the interior. 
	(Above, the vertices of the triangles are ordered so that the similarities take vertices to corresponding vertices, for example, the similarity between triangle $Y BA$ and triangle $ZAC$ takes
	$Y$ to $Z$, $B$ to $A$ and $A$ to $C$). Prove that $AY XZ$ is a parallelogram.
	
	\item[3.]
	Denote by $[a, b]$ the least common multiple of $a$ and $b$. Let $n$ be a
	positive integer such that
	\[
	[n, n+1] > [n, n+2] > \cdots [n, n+35]
	\]
	Prove that $[n, n+35]>[n, n+36]$. 
	
	\item[4.]
	Eight rooks are placed on a chessboard so that no two rooks attack
	each other. Prove that one can always move all rooks, each by a move
	of a knight so that in the final position no two rooks attack each other
	as well. (In intermediate positions several rooks can share the same
	square).
	
	\item[5.]
	A spacecraft landed on an asteroid. It is known that the asteroid is
	either a ball or a cube. The rover started its route at the landing site
	and finished it at the point symmetric to the landing site with respect
	to the center of the asteroid. On its way, the rover transmitted its
	spatial coordinates to the spacecraft on the landing site so that the
	trajectory of the rover movement was known. Can it happen that
	this information is not sufficient to determine whether the asteroid is
	a ball or a cube?
\end{enumerate}

\section*{A-Level}
\begin{enumerate}
  \item[2.]
  Find all positive integers $n$ for which the following statement holds:
  For any two polynomials $P(x)$ and $Q(x)$ of degree $n$ there exist monomials $ax^k$ and 
  $bx^\ell$, $0 \le  k  \le n$, such that the graphs of $P(x) + ax^k$ and $Q(x) + bx^{\ell}$ have no common points.
  
  \textbf{Answer.} $n=1$, and also $n$ even. 
  
  \textbf{Solution.} 
  Denote $P(x)-Q(x)$ as $R(x)$, which has degree $\le n$ 
  and $R(x)+ax^k-bx^{\ell}$ is $R(x)$ with at most two coefficients being altered, and must have no real roots. 
  
  Let's first get rid of the case where $n\ge$ 3 is odd. 
  Choose $(P(x), Q(x))$ such that $R(x)=x^n+x^{n-2}$. 
  Since we need $R(x)+ax^k-bx^{\ell}$ to have even degree, 
  one of $k, \ell$ must be $n$. 
  Since we cannot have $x=0$ as root, 
  one of $k, \ell$ must be 0. 
  Thus $ax^k-bx^{\ell}=-x^n+c$ for some constant $c$. 
  But now $x^n+x^{n-2}-bx^{\ell}+ax^{k}=x^{n-2}+c$ has degree $n-2$, contradiction. 
  
  Now if $n=1$, consider $R(x)=rx+s$, then take $k=1, a=-r$ and $\ell=0, b=-s+1$, 
  which gives $R(x)+ax^k-bx^{\ell}=1$. 
  If $n$ is even, choose $k=n, \ell=0$ and $a, -b$ big enough such that, 
  if $R(x)=\sum_{i=0}^n r_ix^i$, then 
  \[
  r_0-b = \sum_{i=1}^{n-1}|r_i|+1\quad 
  r_n+a=\sum_{i=1}^{n-1} |r_i|+1
  \]
  i.e. $R(x)+ax^k-bx^{\ell}$ has coefficients  at $x^n$ and $x^0$ that's greater than 
  the absolute values of $x^i$ combined for $i=1, \cdots, n-1$. 
  This means: 
  \[
  |x|\le 1: 
  R(x)+ax^k-bx^{\ell}
  =(\sum_{i=1}^{n-1}|r_i|+1)(x^0+x^{2n}) + \sum_{i=1}^{n-1} r_ix^i
  \ge (\sum_{i=1}^{n-1}|r_i|+1)(1) - \sum_{i=1}^{n-1} |r_i|
  \ge 1
  \]
  \[
  |x|>1: 
  R(x)+ax^k-bx^{\ell}
  =(\sum_{i=1}^{n-1}|r_i|+1)(x^0+x^{2n}) + \sum_{i=1}^{n-1} r_ix^i
  \ge x^{2n}\left((\sum_{i=1}^{n-1}|r_i|+1)(1) - \sum_{i=1}^{n-1} |r_i|\right)
  \ge x^{2n}
  \]
  and therefore $R(x)+ax^k-bx^{\ell}$ has no real root. 
  
  \item[4.]
  Is it true that every integer is a sum of finite number of cubes of distinct integers?
  
  \textbf{Answer.} Yes. 
  
  \textbf{Solution.} TODO
  
  \item[6.]
  On a table, there are 11 piles of ten stones each. Pete and Basil play the
  following game. In turns they take 1, 2 or 3 stones at a time: Pete takes stones
  from any single pile while Basil takes stones from different piles but no more
  than one from each. Pete moves first. The player who cannot move, loses.
  Which of the players, Pete or Basil, has a winning strategy?
  
  \textbf{Answer.} Basil has a winning strategy. 
  
  \textbf{Solution.} Let $a_{i, j}$ be the $j$-th stone in $i$-th pile, $1\le i\le 11$ and $1\le j\le 10$. 
  Consider the following pairing: 
  for each $1\le i\le j\le 10$, 
  pair $(i, j)$ with $(j+1, i)$. 
  This gives 55 disjoint pairs and cover all 110 stones 
  (to see why, $j+1>i$ and for each $x>y$ we have $(x, y)$ paired to $(y, x-1)$ so they are disjoint). 
  
  The next observation is that Pete can only take 1 stone from each pair for each move: 
  each move must be in the form of $(i, j)$ for a fixed $i$ for that move, which is paired either with $(j+1, i)$ for $i\le j$, 
  or $(j, i-1)$ if $i>j$. 
  In both cases, we have $i\le j\to j+1>j$ and $i>j\to j\neq i$ 
  so the other element from the pair must be from other piles. 
  
  We also note that if $(i, j_1), \cdots, (i, j_c)$ are picked by Pete, 
  then the mirror images, $(j_k+1, i)$ or $(j_k, i-1)$ must be from different piles. 
  To see why, $(i, 1), \cdots, (i, i-1)$ are mapped to $(1, i-1), \cdots, (i-1, i-1)$, 
  and $(i, i), \cdots, (i, 10)$ are mapped to $(i+1, i), \cdots, (11, i)$. 
  The numbers $1, \cdots, i-1, i+1, \cdots, 11$ are all different. 
  In other words, whenever Pete takes his stones, 
  Basil can always pick the mirror image of those stones so long as they are not taken. 
  
  Having this, we can design the strategy for Basil as follows: 
  take exactly the mirror image of the stones taken by Pete. 
  This ensures that after each move of Basil, for each pair, each both or none of the stones are taken (as the invariant). 
  This would allow Basil to make the last move. 
  
\end{enumerate}
\end{document}