\documentclass[11pt,a4paper]{article}
\usepackage{amsmath, amssymb, fullpage, mathrsfs, bm, pgf, tikz}
\usepackage{mathrsfs}
\usetikzlibrary{arrows}
\setlength{\textheight}{10in}
%\setlength{\topmargin}{0in}
\setlength{\topmargin}{-0.5in}
\setlength{\parskip}{0.1in}
\setlength{\parindent}{0in}

\begin{document}
\newcommand{\la}{\leftarrow}
\newcommand{\lra}{\leftrightarrow}
\newcommand{\bbN}{\mathbb{N}}
\newcommand{\bbZ}{\mathbb{Z}}
\newcommand{\dsum}{\displaystyle\sum}
\newcommand{\dprod}{\displaystyle\prod}


\title{Solutions to Tournament of Towns, Fall 2018, Senior}
\author{Anzo Teh}
\date{}
\maketitle

\section*{O-Level}
\begin{enumerate}
	\item
\end{enumerate}

\section*{A-Level}
\begin{enumerate}
	\item[1.]
	On an island with 2018 inhabitants each person is either a knight, or a liar, or a conformist.
	Everyone knows about everyone who is who. One day all inhabitants of the island were arranged
	in a line and every person answered in turn the same yes-or-no question “Are there more knights
	than liars on the island?” Everybody heard all the previous answers. A knight always says the
	truth, the liar always lies, and a conformist answers as the majority of people before him. In
	case of the same number of “Yes” and “No” he chooses one of these answers at random. It
	occurred that exactly 1009 inhabitants answered “Yes”. Determine the greatest possible number
	of conformists among the inhabitants of the island.
	
	\textbf{Answer.} 1009. 
	
	\textbf{Solution.} 
	The equality case is taken where we have 1009 conformist standing in front of 1009 knights (and 0 liar). 
	Then it's possible that the first conformist says ``No'' (since it's random), 
	and the rest would follow. 
	And then the 1009 knights would say ``Yes''. 
	
	To establish the upper bound, let $a_k$ be the number of ``Yes'' minus the number of ``No'' from among the answers given by first $k$ people. 
	Let $b_k$ be the number of conformists minus the number of nonconformists (knights and liars) among the first $k$ people. 
	We claim that for all $k\ge 0$, $b_k\le |a_k|$. 
	This claim is clear when $k=0$ since $b_0=a_0=0$, which serves as our base case. 
	
	Now for inductive step, assume that $b_k\le |a_k|$. 
	Note that $|b_{k+1}-b_k|=|a_{k+1}-a_k|=1$. 
	If $k+1$-th person is not a conformist we have $ b_{k+1}-b_k=1$, 
	and $|a_{k+1}|-|a_k|\in \{-1, 1\}$ 
	in which case our conclusion will still hold. 
	Otherwise, $b_{k+1}-b_k=1$. 
	If $a_k>0$ then the conformist will answer yes (so $a_{k+1}=a_k+1$), 
	if $a_k<0$ the conformist will answer no (so $a_{k+1}-a_k=-1$), 
	if $a_k=0$ then $a_k=\pm 1$. 
	In all cases we have $|a_{k+1}|-|a_k|=1$, 
	and therefore the induction step is complete. 
	
	\item[2.]
	 In the acute non-isosceles triangle $ABC$, the point $O$ is the center of the circumcircle and $AH_a$
	and $BH_b$ are altitudes. The points $X$ and $Y$ are symmetric to the points $H_a$ and $H_b$ with respect
	to the midpoints of the sides $BC$ and $CA$, respectively. Prove that the line $CO$ divides the
	segment $XY$ in half.
	
	\textbf{Solution.}
	We first show that triangles $OBH_A$ and $OAH_B$ have the same area. 
	Indeed, let $M_A$ and $M_B$ be midpoints of $BC$ and $CA$, respectively, 
	and $H$ is the orthocenter of $ABC$, 
	and $R$ the circumradius of $ABC$. 
	We have $HAH_B$ and $HBH_A$ similar so $\frac{BH_A}{AH_B}=\frac{BH}{AH}$. 
	Thus
	\[
	\frac{\text{Area}(OBH_A)}{\text{Area}(OAH_B)}
	=\frac{BH_A\cdot OM_A/2}{AH_B\cdot OM_B/2}
	=\frac{BH\cdot OM_A}{AH\cdot OM_B}
	=\frac{\sin\angle HAB\cdot R\cos \angle BOM_A}{\sin\angle HBA\cdot R\cos \angle AOM_B}
	\]
	\[
	\frac{\cos\angle ABC\cdot R\cos \angle BAC}{\cos\angle BAC\cdot R\cos \angle ABC}
	=1
	\]
	And since $BH_A=CX$, $OBH_A$ and $OXC$ have the same area, 
	and so do $OAH_B$ and $OYC$. 
	So $OXC$ and $OYC$ have the same area, which means $CO$ bisects $XY$. 
	
	\item[3.]Prove that 
	\begin{enumerate}
		\item any integer of the form $3k - 2$, where $k$ is an integer, is the sum of a square and two cubes of
		some integers
		\item any integer is the sum of a square and three cubes of some integers.
	\end{enumerate}
	
	\textbf{Solution.} (a) is taken care by the following:
	\[
	(3k+5)^2 + k^3 + (-(k+3))^3 = 9k^2+30k+25-9k^2-27k-27=3k-2
	\]
	(b) uses (a) in that for any integer $n$, either $n$, $n-1$ or $n-8$ is in the form of $3k-2$ (corresponding to the case where $n$ has remainder $1, 2, 0$ modulo 3). If $w\in \{n, n-1, n-8\}$ is in the form of $3k-2$ then one of $w+0^3, w+1^3, w+2^3$ is equal to $n$, and each of these expression is then sum of a square and 3 cubes. 
	
	\item[4.]
	A finite number of cells of an infinite grid are painted black, all other cells are white. Consider
	a paper polygon lying on the plane with sides along the grid lines containing at least two cells.
	This polygon may be translated (but not rotated) at any direction and distance, so that after a
	translation its sides are along the grid lines. If after a translation exactly one cell covered by the
	translated polygon is white, then this cell is painted black. Prove that there exists a white cell
	which will never be painted black, no matter of the number of translations.
	
	\textbf{Solution.} Take, say, the bottom left corner of all the cells in the polygon. 
	They form a set $S$ of points in plane, 
	and since $|S|\ge 2$, 
	we can consider the ``convex hull'' of $S$ 
	(which could be just a line if all points in $S$ are collinear). 
	Now choose any side on this convex hull 
	(which could have more than three points if they are collinear), 
	and consider the line represented by the side as $ax+by=c'$ with $(a, b)\neq (0, 0)$. 
	It then follows that for all points $(x, y)$ in $S$ we have $ax+by\le c'$
	(technically, we may have $ax+by\ge c'$ for all $(x, y)$, 
	but we could turn that by replacing $(a, b, c')$ by $(-a, -b, -c')$). 
	In addition, equality would hold for at least 2 points (or more, if there are collinear ones). 
	As the paper polygon is translated with coordinates $(x_0, y_0)$, 
	it would mean that for all points in the polygon we have $ax+by\le c'+(ax_0+by_0)$, 
	with equality holding for at least 2 points on the polygon. 
	
	Now let the black cells have coordinates $(x_1, y_1), \cdots, (x_n, y_n)$, 
	and let $M=\max\{ax_i+by_i: 1\le i\le n\}$. 
	We claim that all $(x, y)$ with $ax+by>M$ will never be painted black. 
	Suppose otherwise, then consider the moment where the paper polygon supposedly paints a point $(x_1, y_1)$ with $ax+by>M$ black. 
	Then by before, there exists a constant $M'$ such that $ax+by\le M'$ for all $(x, y)\in$ the polygon with equality for at least 2 points (grids). 
	Since it passes through $(x_1, y_1)$, $M' > M$. 
	But since $ax+by=M'$ for at least two $(x, y)$'s in the polygon, 
	the polygon would pass through two white squares, hence contradiction. 
	
	\item[5.]
	The three medians of a triangle divide its angles into six angles. What is the greatest possible
	number $k$ of angles greater than $30^{\circ}$ among these six angles?
	
	\textbf{Answer.} $k=3$. 
	
	\textbf{Solution.} 
	Let the triangle be $ABC$, and medians be $AD, BE, CF$. 
	The key to the solution is the following: 
	
	\emph{Lemma.} angles $CAD$ and $EBC$ cannot both exceed $30^{\circ}$. 
	Similarly, each of angles $BAD$ and $BCF$, and $ABE$ and $ACF$, cannot both exceed $30^{\circ}$. 
	
	Proof: notice that $ED\parallel AB$ and $ED=\frac 12 AB$. 
	Consider the point $O$ such that $EDO$ is an equilateral triangle, and $O, A, B$ are on the same side as $DE$, 
	and the circle $\omega$ with center $O$ and passes through $DE$. 
	Then the angle subtended by $DE$ on to circumference of $\omega$ (on the $AB$ side) is $30^{\circ}$. 
	If $\angle CAD$ and $\angle EBC$ are both $>30^{\circ}$, 
	then $A$ and $B$ are both inside $\omega$. 
	It follows that $AB$ must be on some chord of length $>AB=2DE=\text{diameter}(\omega)$, 
	which is impossible since the diameter is the longest chord possible on the circle. 
	The proof for the other two are similar. 
	$\square$
	
	Now the lemma effectively put an upper bound of 3. 
	To show that this is feasible, 
	consider the construction of points as follows: 
	$B, G, E$ on the line in that order, with $\frac{BG}{GE}=2$. 
	Consider the point $A'$ such that $\angle A'EB=90^{\circ}$, 
	and $\angle EBA'=30^{\circ}$. 
	Then triangle $ABE'$ is a $90-60-30$ triangle with $\frac{A'B}{A'E}=2$, so 
	by angle bisector theorem, 
	$\angle BA'G=\angle EA'G=30^{\circ}$.
	
	\definecolor{zzttqq}{rgb}{0.6,0.2,0}
	\definecolor{uuuuuu}{rgb}{0.26666666666666666,0.26666666666666666,0.26666666666666666}
	\definecolor{ududff}{rgb}{0.30196078431372547,0.30196078431372547,1}
	\begin{tikzpicture}[line cap=round,line join=round,>=triangle 45,x=1cm,y=1cm, scale=1.5]
	\clip(-11.406320119340561,-3.2102877246846733) rectangle (-1.9525549567736689,2.7028087363125204);
	\fill[line width=2pt,color=zzttqq,fill=zzttqq,fill opacity=0.10000000149011612] (-10.22,-1.2) -- (-7.588283042439342,0.9587912235848227) -- (-4.763973956175252,-3.0682518225712023) -- cycle;
	\draw [line width=2pt] (-10.22,-1.2)-- (-6.176128499307297,-1.0547302994931897);
	\draw [line width=2pt] (-6.8920428331024315,0.08842323350227034) circle (1.3488266505868471cm);
	\draw [line width=2pt] (-10.22,-1.2)-- (-6.26,1.28);
	\draw [line width=2pt] (-6.26,1.28)-- (-6.176128499307297,-1.0547302994931897);
	\draw [line width=2pt,color=zzttqq] (-10.22,-1.2)-- (-7.588283042439342,0.9587912235848227);
	\draw [line width=2pt,color=zzttqq] (-7.588283042439342,0.9587912235848227)-- (-4.763973956175252,-3.0682518225712023);
	\draw [line width=2pt,color=zzttqq] (-4.763973956175252,-3.0682518225712023)-- (-10.22,-1.2);
	\draw [shift={(-8.955914333795134,1.1831535329954586)},line width=2pt]  plot[domain=-2.0584871218806757:0.035907980512520636,variable=\t]({1*2.6976533011736956*cos(\t r)+0*2.6976533011736956*sin(\t r)},{0*2.6976533011736956*cos(\t r)+1*2.6976533011736956*sin(\t r)});
	\begin{scriptsize}
	\draw [fill=ududff] (-6.26,1.28) circle (2.5pt);
	\draw[color=ududff] (-6.1492358797002264,1.4761557470631026) node {$A'$};
	\draw [fill=uuuuuu] (-6.176128499307297,-1.0547302994931897) circle (2.5pt);
	\draw[color=uuuuuu] (-6.104303535771677,-0.8603261372215028) node {$E$};
	\draw [fill=ududff] (-10.22,-1.2) circle (2.5pt);
	\draw[color=ududff] (-10.148214489341164,-1.004109637792863) node {$B$};
	\draw [fill=uuuuuu] (-7.524085666204864,-1.1031535329954598) circle (2pt);
	\draw[color=uuuuuu] (-7.4522738536281725,-0.9322178875071829) node {$G$};
	\draw [fill=ududff] (-7.588283042439342,0.9587912235848227) circle (2.5pt);
	\draw[color=ududff] (-7.515179135128142,1.1526428707775418) node {$A$};
	\draw [fill=ududff] (-4.763973956175252,-3.0682518225712023) circle (2.5pt);
	\draw[color=ududff] (-4.693427936415211,-2.8732951452205477) node {$C$};
	\end{scriptsize}
	\end{tikzpicture}
	
	Consider, now, the circumcircles of triangle $A'BG$ and $A'GE$, where $A'B$ is part of in the first one. 
	For the second one, 
	we see that $A'B>BG$ and $A'B>GE$ so $A'B$ is not tangent to the circle $A'GE$. 
	It follows that there's a non-degenerate region lying in the interior of the intersection of the two circles, 
	and also outside angle $\angle A'BE$ (beyond $A'B$). 
	Thus for any $A$ in this region we have $\angle ABE, \angle BAG, \angle EAG$ all $>30^{\circ}$. 
	Since $BG:GE=2:1$, reflect $A$ in $E$ to get $C$ and $G$ is the centroid of $ABC$. 
	This gives us the construction of $k=30^{\circ}$ as desired. 
	
	\textbf{Remark}: 
	We can, in fact, choose $A$ such that $\angle BAE=90^{\circ}$, $BG:GE=2:1$ and $AG$ is perpendicular to $BE$. 
	This gives $\frac{BG}{AG}=\frac{AG}{AE}=\sqrt{2}$ and $\angle ABE=\angle GAE=\arctan\sqrt{1/2}$, $\angle BAG=\arctan\sqrt{2}$, 
	all of which are greater than $30^{\circ}=\arctan\sqrt{1/3}$. 
\end{enumerate}
\end{document}