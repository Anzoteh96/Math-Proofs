\documentclass[11pt,a4paper]{article}
\usepackage{amsmath, amssymb, fullpage, mathrsfs, bm, pgf, tikz}
\usepackage{mathrsfs, algorithm, algpseudocode}
\usetikzlibrary{arrows}
\setlength{\textheight}{10in}
%\setlength{\topmargin}{0in}
\setlength{\topmargin}{-0.5in}
\setlength{\parskip}{0.1in}
\setlength{\parindent}{0in}

\begin{document}
\newcommand{\la}{\leftarrow}
\newcommand{\lra}{\leftrightarrow}
\newcommand{\bbN}{\mathbb{N}}
\newcommand{\bbZ}{\mathbb{Z}}
\newcommand{\dsum}{\displaystyle\sum}
\newcommand{\dprod}{\displaystyle\prod}


\title{Solutions to Tournament of Towns, Spring 2015, Senior}
\author{Anzo Teh}
\date{}
\maketitle

\section*{O-Level}
\begin{enumerate}
	\item[1.]
	Pete summed up 100 consecutive powers of two, while Basil summed up several
	first consecutive positive integers. Can they get the same result?
	
	\textbf{Answer.} Yes, given by the following example: 
	\[
	2^{99}+2^{100}+\cdots + 2^{198}
	=2^{99}(2^{100}-1)
	\qquad 
	1+2+\cdots + (2^{100}-1)
	=\frac{2^{100}(2^{100-1})}{2}
	=2^{99}(2^{100}-1)
	\]
	
	\item[2.]
	A moth made four small holes in a square carpet with a 275 cm side. Can one
	always cut out a square piece with a 1 m side without holes? (Consider holes
	as points).
	
	\textbf{Answer.} Yes. 
	
	\textbf{Solution.} Consider the four squares of side length $1$ contained in the square carpet and passing through the 4 corners of the carpet. 
	The uncovered area then contains two strips of $0.75\times 2.75$ that intersect perpendicularly, 
	the intersection being a square of side length $0.75$. 
	
	Now, consider the biggest square with center the center of the intersection of the two strips, 
	oriented at $45^{\circ}$. 
	This square has side length $0.75\sqrt{2} > 1$. Thus, we have 5 squares of side $\ge 1$ that's mutually disjoint. 
	Since one of them cannot have holes, the conclusion follows. 
	
	\item[3.]
	Among $2n+1$ positive integers there is exactly one 0, while each of the numbers
	$1, \cdots, n$ is presented exactly twice. 
	For which $n$ can one line up these numbers
	so that for any $m = 1, \cdots, n$ there are exactly $m$ numbers between two $m$’s?
	
	\textbf{Answer.} All $n\ge 0$. 
	
	\textbf{Solution.} $n=0$ is vacuously true, and for $n=1, 2$ we have $(1, 0, 1)$ and $(2, 0, 1, 2, 1)$, 
	respectively. 
	Now, call an $n$-sandwich as a sequence of numbers 
	\[
	\begin{cases}
	  n, n-2, \cdots,  3,1, a, 1, 3, \cdots, n-2, n  & n\text{ odd}\\
	  n, n-2, \cdots, 4, 2, a, b, 2, 4, \cdots, n-2, n & n\text{even}\\
	\end{cases}
	\]
	where $a, b$ could be anything. 
	We denote the sandwiches above as $[n:(a):n]$ for $n$ odd, and $[n:(a,b):n]$ for $n$ even. 
	We see that in $[n:(a):n]$, we have $m$ numbers between two $m$'s for $m=1, 3, \cdots, n$, 
	and in $[n:(a,b):n]$ we have $m$ numbers between two $m$'s for $m=2, 4, \cdots, 2n$. 
	
	Then when $n\ge 4$ even we do 
	\[
	[n:(0, n-1):n][n-3:(n-1):n-3]
	\]
	and for $n\ge 3$ odd we do 
	\[
	[n:(n-1):n][n-3:(n-1, 0):n-3]
	\]
	By the properties of the sandwich, the only thing we need to verify is the number of numbers between the 
	two $n-1$'s. 
	The first one ($n$ even) gives $\frac{n}{2}+\frac{n-2}{2}=n-1$, 
	the second one ($n$ odd) gives $\frac{n+1}{2}+\frac{n-3}{2}=n-1$, as desired. 
	
	\item[4.]Points $K$ and $L$ are marked on the median $AM$ of triangle ABC, so that
	$AK = KL = LM$. Point $P$ is chosen so that triangles $KPL$ and $ABC$ are
	similar (the corresponding vertices are listed in the same order). Given that
	points $P$ and $C$ are on the same side of line $AM$, prove that point $P$ lies on
	line $AC$.
	
	\item[5.]
	2015 positive integers are arranged in a circular order. The difference between
	any two adjacent numbers coincides with their greatest common divisor. 
	Determine the maximal value of $N$ which divides the product of the numbers,
	regardless of their choice.
	
	\textbf{Answer.} $3\times 2^{1009}$. 
	
	\textbf{Answer.} Replace 2015 with any odd number $k\ge 3$, and we show the answer is $3\cdot 2^{\frac{k+3}{2}}$. 
	This example can be taken with 
	\[2,4, 3, \underbrace{1,2,\cdots, 1,2}_{\frac{n-3}{2}\text{ pairs of 1, 2}}
	\]
	which has product precisely $3\cdot 2^{\frac{k+3}{2}}$. 
	
	It remains to show that all such numbers must be divisible by both 3 and $2^{\frac{k+3}{2}}$. 
	First, we note that no two adjacent numbers can both be odd (its $\gcd$ is odd while the difference is even). 
	Thus there must be at least $\frac{k+1}{2}$ even numbers. 
	Since $k$ is odd, two of those even numbers are consecutive. 
	We cannot have both these numbers in the form $4k+2$: the difference is divisible by 4 but the $\gcd$ is not. 
	Hence, we must have one number in the sequence divisible by 4. 
	This means $N$ is divisible by $4\cdot 2^{\frac{k-1}{2}}=2^{\frac{k+1}{2}}$. 
	
	It remains to show that one of the numbers are divisible by 3. 
	Suppose otherwise, then since no $\gcd$ (and hence no difference) can be multiple of 3, 
	the remainder of the numbers, when divided by 3, must alternate between $1, 2, 1, 2\cdots$. 
	This is impossible since $k$ is odd. 
\end{enumerate}

\section*{A-Level}
\begin{enumerate}
	\item[1.]
	\begin{enumerate}
		\item The integers $x, x^2$ and $x^3$ begin with the same digit. 
		Does it imply that this digit is 1?
		\item The same problem for $x, x^2, \cdots, x^{2015}$. 
	\end{enumerate}
	
	\textbf{Answer.} No. 
	
	\textbf{Solution.} In fact, for each $n\ge 1$ we can choose $x$ such that $x, x^2, \cdots, x^{n}$ begin with 9. 
	Now, consider $k$ such that the number $1-\frac{1}{10^k}\le \sqrt[n]{0.9}$. 
	Then for all $m\le n$ we have $0.9\cdot 10^{km}\le (10^k-1)^m < 10^{km}$, 
	meaning that $(10^k-1)^m$ has $km$ digits starting with 9. 
	
	\item[2.]
	A point $X$ is marked on the base $BC$ of an isosceles triangle $ABC$, and points $P$ and $Q$ are
	marked on the sides $AB$ and $AC$ so that $AP XQ$ is a parallelogram. Prove that the point $Y$
	symmetrical to $X$ with respect to line $PQ$ lies on the circumcircle of the triangle $ABC$.
	
	\textbf{Solution.} 
	Let $Y'$ be the intersection of circles $ABC$ and $APQ$, 
	with $Y'\neq A$ unless the two circles are tangent to each other. 
	We show that $Y'=Y$. 
	In fact, $Y'$ is the center of spiral similarity that maps $Y'BC$ to $Y'PQ$, and consequently, $Y'PB$ to $Y'QC$. 
	So in particular we have $\frac{Y'P}{Y'Q}=\frac{PB}{QC}=\frac{PX}{QX}$, 
	the last equality following from $APXQ$ parallelogram and $AB=AC$ so $PB=PX$ and $QC=QX$. 
	But since we also have $\angle PY'Q=PAQ=\angle PXQ$ by the definition of parallelogram, 
	$PY'Q$ and $PXQ$ are congruent, so $Y'$ is the reflection of $X$ in $PQ$ (i.e. $Y=Y'$). 
	
	\item[4.]
	A convex $N$-gon with equal sides is located inside a circle. Each side is extended in both directions
	up to the intersection with the circle so that it contains two new segments outside the polygon.
	Prove that one can paint some of these new $2N$ segments in red and the rest in blue so that the
	sum of lengths of all the red segments would be the same as for the blue ones.
	
	\textbf{Solution.} 
	Let the vertices as $A_1, \cdots, A_n$, and for each $i$ (taken modulo $i$), 
	let $A_{i\to i+1}$ be the intersection of extension of $A_iA_{i+1}$ beyond $A_{i+1}$ and the circle, 
	and define $A_{i+1\to i}$ similarly. 
	Then by taking power of point theorem on $A_i$ and the coords $A_{i\to i+1}A_{i+1\to i}$ and $A_{i\to i-1}A_{i-1\to i}$, we have 
	\[
	A_{i\to i+1}A_i\cdot A_iA_{i+1\to i} = A_{i\to i-1}A_i\cdot A_iA_{i-1\to i}
	\]
	Now, let the extended segment $A_{i+1}A_{i\to i+1}$ have length $a_{i\to i+1}$, 
	and similarly, the extended segment $A_iA_{i+1\to i}$ have length $a_{i+1\to i}$. 
	Also let the common length $A_iA_{i+1}$ to be $b$. Then the equation above can be rewritten as 
	\[
	(b+a_{i\to i+1})a_{i+1\to i}= (b+a_{i\to i-1})a_{i-1\to i}
	\]
	Summing over all $i=1,\cdots, n$ gives 
	\[
	b\sum_{i=1}^na_{i+1\to i} + \sum_{i=1}^n a_{i\to i+1}a_{i+1\to i}
	=b\sum_{i=1}^na_{i-1\to i} + \sum_{i=1}^n a_{i\to i-1}a_{i-1\to i}
	\]
	but since we're taking mod $n$, $\sum_{i=1}^n a_{i\to i+1}a_{i+1\to i}=\sum_{i=1}^n a_{i\to i-1}a_{i-1\to i}$. 
	Therefore $\sum_{i=1}^na_{i+1\to i}=\sum_{i=1}^na_{i-1\to i}$. 
	This means we can colour the $n$ segments $A_{i+1}A_{i\to i+1}$ red and $A_iA_{i+1\to i}$ blue. 
	
	\item[5.]
	Do there exist two polynomials with integer coefficients such that each polynomial has a coefficient
	with an absolute value exceeding 2015 but all coefficients of their product have absolute values
	not exceeding 1?
	
	\textbf{Answer.} Yes. 
	
	\textbf{Solution.} Let $n=2017$. 
	Consider the following polynomials: 
	\[
	f(x)=(x-1)^n
	\qquad 
	g(x) = \prod_{k=1}^{n-1} (1+x+\cdots + x^{2^k-1})
	\]
	Then given that $(x-1)(1+x+\cdots + x^{2^k-1})=x^{2^k}-1$, we have 
	\[
	f(x)g(x) = \prod_{k=0}^{n-1}(x^{2^k}-1)=\sum_{\ell=0}^{2^{n-1}-1}(-1)^{d(\ell)+n}x^{\ell}
	\]
	where $d(\ell)$ is the number of digits in the binary representation of $\ell$. 
	Thus $fg$ does have coefficients that are in $\{-1, 0, 1\}$. 
	Meanwhile, $f(x)$ has coefficient $-n$ at $x^{n-1}$, 
	and $g(x)$ has coefficient $n-1$ at $x$, 
	so each of them has coefficient with magnitude $\ge n-1=2016$. 
	
\end{enumerate}
\end{document}