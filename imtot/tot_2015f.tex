\documentclass[11pt,a4paper]{article}
\usepackage{amsmath, amssymb, fullpage, mathrsfs, bm, pgf, tikz}
\usepackage{mathrsfs, algorithm, algpseudocode}
\usetikzlibrary{arrows}
\setlength{\textheight}{10in}
%\setlength{\topmargin}{0in}
\setlength{\topmargin}{-0.5in}
\setlength{\parskip}{0.1in}
\setlength{\parindent}{0in}

\begin{document}
\newcommand{\la}{\leftarrow}
\newcommand{\lra}{\leftrightarrow}
\newcommand{\bbN}{\mathbb{N}}
\newcommand{\bbZ}{\mathbb{Z}}
\newcommand{\dsum}{\displaystyle\sum}
\newcommand{\dprod}{\displaystyle\prod}


\title{Solutions to Tournament of Towns, Fall 2015, Senior}
\author{Anzo Teh}
\date{}
\maketitle

\section*{O-Level}
\begin{enumerate}
	\item[1.]
	Let $p$ be a prime number. Determine the number of positive integers $n$ such
	that $pn$ is a multiple of $p + n$.
	
	\textbf{Answer.} 1. 
	
	\textbf{Solution.} 
	We have $p(p+n)-pn = p^2$ so this is equivalent to saying $p+n\mid p^2$. 
	The only positive divisors of $p^2$ are $1, p, p^2$, 
	and only $p^2>p+n$. 
	Thus $n=p(p-1)$ is the only answer (and it works). 
	
	\item[2.]
	Suppose that ABC and ABD are right-angled triangles with common hypotenuse AB (D and C are on the same side of line AB). If AC = BC and
	DK is a bisector of angle ADB, prove that the circumcenter of triangle ACK
	belongs to line AD. 
	
	\item[3.]
	Three players play the game “rock-paper-scissors”. In every round, each player
	simultaneously shows one of these shapes. Rock beats scissors, scissors beat
	paper, while paper beats rock. If in a round exactly two distinct shapes are
	shown (and thus one of them is shown twice) then 1 point is added to the score
	of the player(s) who showed the winning shape, otherwise no point is added.
	After several rounds it occurred that each shape had been shown the same
	number of times. Prove that the total sum of points at this moment was a
	multiple of 3.
	
	\textbf{Solution.} Keep another table that adds 0 for any occurence of scissor, 
	1 for any occurence of rock and 2 for any occurence of paper. 
	Then after each round, the number added to the table is congruent to the score added modulo 3. 
	Since $3\mid 0+1+2$, the conclusion follows. 
	
	\item[4.]In a country there are 100 cities. Every two cities are connected by direct
	flight (in both directions). Each flight costs a positive (not necessarily integer)
	number of doubloons. The flights in both directions between two given cities
	are of the same cost. The average cost of a flight is 1 doubloon. A traveller
	plans to visit any $m$ cities for $m$ flights, starting and ending at his native city
	(which is one of these $m$ cities). Can the traveller always fulfil his plans given
	that he can spend at most $m$ doubloons if 
	\begin{enumerate}
		\item $m=99$;
		\item $m=100$?
	\end{enumerate}
    
    
    \item[5.] 
    An infinite increasing arithmetical progression is given. A new sequence is
    constructed in the following way: its first term is the sum of several first terms
    of the original sequence, its second term is the sum of several next terms of the
    original sequence and so on. Is it possible that the new sequence is a geometrical
    progression?
\end{enumerate}

\section*{A-Level}
\begin{enumerate}
	\item[1.]
	A geometrical progression consists of $37$ positive integers. 
	The first and the last terms are relatively prime numbers. 
	Prove that the $19^{th}$ term of the progression is the $18^{th}$ power of some positive integer.
	
	\textbf{Solution.} 
	Consider a prime $p$ that divides some term in the sequence, 
	and let $a_i$ be the highest power of $p$ dividing the $i$-th term. 
	Then either $\{a_i\}$ is an arithmetic sequence. 
	Since the first and last term given are relatiely prime, either $a_1=0$ or $a_{37}=0$. 
	It then follows that $a_{19} = \frac{\max(a_1, a_{37})}{2}$ and since all $a_i$ are integers, 
	$a_{19}$ is divisible by 18. 
		
	\item[2.]
	A $10 \times 10$ square on a grid is split by $80$ unit grid segments into $20$ polygons of equal area (no one of these segments belongs to the boundary of the square). 
	Prove that all polygons are congruent.
	
	\textbf{Solution.} 
	There are $10\times 10 \times 2 - 10-10=180$ unit grid segments that are not boundary of the square. 
	This means 100 of those segments are in the interior of the polygons. 
	
	Now, each of the polygons has area 5 (hence comprises 5 square grids). 
	Consider each square grid as a graph with edge iff they are adjacent on the $10\times 10$ square. 
	Consider, also, coloring the square in checkerboard fashion. Then here's some properties of this graph: 
	\begin{itemize}
		\item it's connected;
		\item separating into white and black squares would show that this graph is bipartite;
		\item two squares (of the same colour) that's distance 2 apart and on the same row/column, 
		has exactly one common neighbour: the square in between
		\item the only two squares with $\ge 2$ common neighbours are those share a common vertex: 
		the two common neighbours also have common vertex. 
	\end{itemize}
	
	We now show that the number of edges in each polygon cannot exceed 5. 
	Given our black vs white partition, the only possible way when we have this edge count $>5$ is when we have $K_{2, 3}$: the sides have 2, 3 squares, and every pair of  squares (of different color) are neighbours. 
	W.l.o.g. let our polygon to have 2 whites, $W_1, W_2$, each with the same 3 black neighbours, $B_1, B_2, B_3$. 
	Then two of the black squares, say, $B_1, B_2$, must lie on opposite sides of $W_1$. 
	But this would mean that the only common neighbour of $B_1$ and $B_2$ can be $W_1$, 
	contradiction. 
	
	On the other hand, the fact that the 20 polygon graphs have total edge count 100 means each graph has average edge count 5. With the upper bound shown before, 
	each graph must have edge count 5. 
	Thus each of them must have a cycle (having $>4$ edges), and with the bipartiteness, 
	this cycle must have length 4. 
	If $A_1, A_2, A_3, A_4$ are the cycle in that order then $A_i$ and $A_{i+2}$ have the other two as common neighbours for each $i$ (indices taken modulo 2). 
	Then $A_1$ and $A_3$ share a common vertex, and so do $A_2$ and $A_4$. 
	This means, $A_1, A_2, A_3, A_4$ must be a $2\times 2$ square. 
	
	We conclude that each polygon is $2\times 2$ square, plus another square attached to this $2\times 2$ square. 
	It therefore follows that all those polygons are congruent. 
	
	\item[3.]
	Each coefficient of a polynomial is an integer with absolute value not exceeding $2015$.
	Prove that every positive root of this polynomial exceeds $\frac{1}{2016}$.
	
	\textbf{Solution.} 
	By dividing by $x^k$ for some $k$, i.e. removing all the root at $x=0$, we may assume that the constant term is nonzero. 
	Let $y$ be a positive root. 
	Then 
	\[
	0 = P(y) := \sum_{k=0}^n a_ky^k
	\]
	Thus we have 
	\[
	1\le |a_0| = |-\sum_{k=1}^n a_ky^k|\le \sum_{k=1}^n |a_ky^k|
	\]
	Suppose, now, $|y|\le \frac{1}{2016}$. 
	Then with $|a_k|\le 2015$ for each $k$ we have 
	\[
	\sum_{k=1}^n |a_ky^k|
	\le \sum_{k=1}^n 2015\cdot \frac{1}{2016^k}
	=2015\left(\frac{1}{2016}\frac{1-(1/2016)^n}{1-1/2016}\right)
	< 1
	\]
	i.e. contradiction. 
	
	\item[4.]
	Let $ABCD$ be a cyclic quadrilateral, $K$ and $N$ be the midpoints of the diagonals and $P$ and $Q$ be points of intersection of the extensions of the opposite sides. 
	Prove that $\angle PKQ + \angle PNQ = 180$.
	
	\textbf{Solution.} 
	W.l.o.g. let rays $BA$ and $CD$ intersect at $P$ and rays $DA$ and $CB$ intersect at $Q$. 
	Also consider $K$ as midpoint of $BD$ and $N$ the midpoine of $AC$. 
	Then triangles $PAC$ and $PDB$ are similar, 
	so $\angle APK=\angle DPN$, 
	i.e. $PK$ and $PN$ are reflections of each other w.r.t. the internal angle bisector of 
	$\angle BPC$. 
	Similarly, $QK$ and $QN$ are reflections of each other w.r.t. the internal angle bisector of $\angle CQD$. 
	Therefore, we have $\angle PKQ+\angle PNQ = \angle PAQ+\angle PCQ=\angle PBQ+\angle PDQ=180^{\circ}$, 
	the last two equality is because $ABCD$ cyclic. 
	
\end{enumerate}
\end{document}