\documentclass[11pt,a4paper]{article}
\usepackage{amsmath, amssymb, fullpage, mathrsfs, bm, pgf, tikz}
\usepackage{mathrsfs, algorithm, algpseudocode}
\usetikzlibrary{arrows}
\setlength{\textheight}{10in}
%\setlength{\topmargin}{0in}
\setlength{\topmargin}{-0.5in}
\setlength{\parskip}{0.1in}
\setlength{\parindent}{0in}

\begin{document}
\newcommand{\la}{\leftarrow}
\newcommand{\lra}{\leftrightarrow}
\newcommand{\bbN}{\mathbb{N}}
\newcommand{\bbZ}{\mathbb{Z}}
\newcommand{\dsum}{\displaystyle\sum}
\newcommand{\dprod}{\displaystyle\prod}


\title{Solutions to Tournament of Towns, Fall 2012, Senior}
\author{Anzo Teh}
\date{}
\maketitle

\section*{O-Level}
\begin{enumerate}
	\item[1.] A table $10 \times 10$ was filled according to the rules of the game ``Bomb Squad'': several
	cells contain bombs (one bomb per cell) while each of the remaining cells contains a number,
	equal to the number of bombs in all cells adjacent to it by side or by vertex.
	Then the table is rearranged in the “reverse” order: bombs are placed in all cells previously
	occupied with numbers and the remaining cells are filled with numbers according to the same
	rule. Can it happen that the total sum of the numbers in the table will increase in a result?
	
	\textbf{Answer.} No. 
	
	\textbf{Solution.} 
	We see that the total sum of numbers is really just the number of pairs of  adjacent cells where 
	one has bomb and one is empty. Flipping this would just make the set of pairs the same. 
	
	\item[2.]Given a convex polyhedron and a sphere intersecting each its edge at two points so that
	each edge is trisected (divided into three equal parts). Is it necessarily true that all faces of
	the polyhedron are
	\begin{enumerate}
		\item congruent polygons?
		\item regular polygons?
	\end{enumerate}
    
    \textbf{Solution.} The answer is no for (a) and yes for (b). 
    
    We first show that the polyhedron must be circumscribed in a sphere that's concentric with the said sphere. 
    Let our sphere to have center $O$, 
    and take any edge with endpoints $AB$. 
    Let the sphere intersects $AB$ at $X$ and $Y$, 
    and let the midpoint of $XY$ be $M$. 
    We see that $M$ is also the midpoint of $AB$. 
    Moreover, $OX=OY$ implies that $OM\perp AB$. 
    It then follows that $OA=OB$ too. 
    Doing this for all edges shows that $O$ must be equidistant from all points on the polyhdron. 
    In particular, each face is a cyclic polygon. 
    
    Now let $r$ be the radius of the sphere, and $R$ be the distance from $O$ to each vertex. 
    Consider the edge $AB$ and points $X, Y, M$ in previous paragraph again. 
    We have 
    \[
    R^2=OA^2=OM^2+MA^2\quad 
    r^2=OX^2=OM^2+MX^2
    \]
    Since $X, Y$ trisect $AB$, we have $MX=\frac 13 MA$, so 
    \[
    R^2=OM^2+MA^2\quad r^2=OM^2+(MX/3)^2=OM^2+MX^2/9
    \]
    Solving simultaneous equations give $OM^2=(9r^2-R^2)/8$, the perpendicular distance of $O$ to the side. 
    Since this $r$ and $R$ do not depend on the choice of side $AB$, 
    it follows that $O$ is also equidistant to each side. 
    Finally, take any face of polyhedron. 
    Given it's cyclic, we can let $O_1$ be its circumcenter, 
    which is just the projection of $O$ to the face. 
    It then follows that $O_1$ is equidistant to all points, and all edges. 
    Thus this face has to be regular. 
    
    To construct a counterexample for (a), 
    consider a pyramid with square base and 4 equilateral triangle faces, 
    then we can have a sphere with center coincide with the ``circumcenter'' of this pyramid, and adjust its radius such that it will trisect one edge. 
    It then follows that it will trisect the other edges, too. 
    
    \item[3.] For a class of 20 students several field trips were arranged. In each trip at least four students participated. Prove that there was a field trip such that each student who participated
    in it took part in at least 1/17-th of all field trips
    
    \textbf{Solution.} 
    Let $S$ be the set of students who went for fewer than $\frac{1}{17}$ of the trips. 
    Let $S=\{1, 2, \cdots, k\}$ and $i$ went to $b_i$ of the $N$ trips, then 
    $17b_i<N$ for each $i$. 
    
    Denote $c$ as the maximum possible integer such that each trip is attended by at least $c$ of the members in $S$. If $c\ge 1$, then $k\ge 1$ and 
    Then 
    Thus 
    \[
    cN\le \sum b_i < \sum_{i=1}^k \frac{N}{17} = N(\frac{k}{17})
    \]
    i.e. $k > 17c$. Thus $k\ge 17c+1$. 
    Given that we have 20 students, we can only have $c\le 1$ here, so only $c=1$ is allowed and $|S|\ge 18$. 
    Choose a trip attended by exactly one student in $S$, 
    which means at least $|S|-1\ge 18-1=17$ students did not attended the trip. 
    This contradicts that each trip is attended by at least 4 students. 
    Hence $c=0$ and some trip is attended by nobody in $S$. 
    
    \item[4.] 
    Let $C(n)$ be the number of prime divisors of a positive integer $n$.
    \begin{enumerate}
    	\item Consider set S of all pairs of positive integers $(a, b)$ such that $a\neq b$ and
    	\[
    	C(a + b) = C(a) + C(b)
    	\]
    	Is $S$ finite or infinite?
    	\item Define $S'$ as a subset of $S$ consisting of the pairs $(a, b)$ such that $C(a+b) > 1000$. 
    	Is $S$ finite or infinite?
    	
    \end{enumerate}

    \textbf{Answer.} Infinite for both cases. 
    
    \textbf{Solution.} 
    Let's first find an example $c$ where there exists $a, b$ with $a+b=c$, $C(c)\ge C(a)+C(b)$, and $C(c)>1000$. 
    Indeed, take $a=1$ and consider $C(2), C(3), \cdots$. 
    Such sequence is unbounded, and therefore for each threshold $r$ we can let $c$ be the minimum number with $C(c)\ge r$. 
    Then $C(c-1)<C(c)$, and we have $C(c)\ge r > C(c-1)=C(1)+C(c-1)$; 
    the $>$ inequality follows from the minimality of $c$. 
    
    Now, 
    consider any number $k$ with $\gcd(k, c)=\gcd(k, c-1)=1$. 
    Then 
    \[
    C(kc)=C(k)+C(c)\quad 
    C(k(c-1))=C(k)+C(c-1)
    \]
    so we just need to find such $k$ such that $C(k)+C(c)=C(k)+C(k)+C(c-1)$. 
    This means $k$ needs to have $C(c)-C(c-1)$ prime factors, and mutually relatively prime to both $c$ and $c-1$. 
    This can be chosen by taking $k$ as product of $g:=C(c)-C(c-1)$ prime factors that are not part of $c$ or $c-1$'s prime factors; 
    there are infinitely many such choices. 
    Moreover $C(kc)=C(k)+C(c)\ge C(c)>r$, so taking $r=1000$ and we're done by taking $a=k, b=k(c-1)$. 
    
    \item[5.] 
    Among 239 coins identical in appearance there are two counterfeit coins. Both counterfeit
    coins have the same weight different from the weight of a genuine coin.
    Using a simple balance, determine in three weighings whether the counterfeit coin is heavier
    or lighter than the genuine coin. A simple balance shows if both sides are in equilibrium or
    left side is heavier or lighter. It is not required to find the counterfeit coins. 
    
    \textbf{Solution.} 
    Throughout the solution we use the following facts: 
    
    \emph{Lemma 1.} if we know a subset $P$ of the coins with even size having either 0 or 1 counterfeits, 
    then we can split it evenly into $(P/2, P/2)$ coins. 
    
    Proof: An equilibrium means 0 counterfeit in $P$; a disequilibrium implies 1 counterfeit in $P$ (since the counterfeit has to be on one side of the balance).  
    
    \emph{Lemma 2.} If two disjoint subsets $P_1$ and $P_2$ of equal size, are not in equilibrium, 
    and we know which one is heavier, then we can determine whether the counterfeit coins are heavier or lighter by determining which one among the two have counterfeit coins. 
    
    Proof: now $P_1$ and $P_2$ have different number of counterfeit coins, i.e. at least one of them has one. 
    They cannot both have counterfeit coins since this means at least $1+2=3$ counterfeit coins in total. 
    So exactly one of them has counterfeit coin and since we also know which one is heavier, then claim follows. 
    
    Now let's first single out a coin $C$, and for the rest we split into piles $P_1, P_2, P_3$ of weights 80, 79, 79. 
    We do two weighings: $(P_1, P_2+C)$ (i.e. 80 on each side), and $(P_2, P_3)$ (i.e. 79 on each side). 
    The third weighing will be determined according to the cases below. 
    (Notice the use of Lemma 1 counts as one weighing). 
    
    \emph{Case 1.} Equilibrium for both weighings. 
    Since $P_2$ and $P_3$ have the same number of counterfeits, and so does $P_1$ and $P_2+C$, the only possibility is 
    where $P_1$ has one counterfeit, $C$ is counterfeit, and $P_2$ and $P_3$ are both genuine. 
    Now weigh $C$ against any coin in $P_2$ to see if it's heavier. 
    
    \emph{Case 2.} Equilibrium in first but not second. 
    Here, $P_1$ and $C+P_2$ either have both 0 or both 1 counterfeit coins. 
    The former case gives $P_3$ both counterfeit coins; 
    the latter case gives $P_3$ no counterfeit coin, so the counterfeits must be in $P_1$ and $P_2$
     (in both cases, $C$ is real). 
    Now use Lemma 1 on $C+P_2$ (size 80) to distinguish between the two cases, 
    and then use Lemma 2 to conclude on ($P_2, P_3$). 
    
    \emph{Case 3.} Equilibrium in second but not first. 
    Now $P_2$ and $P_3$ have the same number of counterfeits, 
    so each has 0 or 1 counterfeit coins. 
    In the former case, the only possibility is $P_1$ have both the counterfeits, i.e. $C+P_2$ has zero counterfeit. 
    In the second case $P_2$ and $P_3$ have one counterfeit each, $P_1$ has none, $C$ is genuine. 
    To distinguish between the two cases, we use Lemma 1 on $C+P_2$ ($C+P_2$ has size 80), 
    and finish with Lemma 2 on $(P_1, C+P_2)$. 
    
    \emph{Case 4.} Both have disequilibrium. 
    This has the most cases, but they can be characterized into the following: 
    \begin{itemize}
    	\item 
    	case where $P_1$ has more counterfeit than $C+P_2$: 
    	the only possibility is one counterfeit in each of $P_1$ and $P_3$
    	($P_1$ cannot have two counterfeits since this means $(P_2, P_3)$ have all genuine coins, hence equilibrium). 
    	\item 
    	case where $P_1$ has fewer counterfeit than $C+P_2$: 
    	This means $P_1$ cannot have counterfeit (otherwise $P_1$ has at least one, $C+P_2$ has at least two, contradiction). 
    	
    \end{itemize}
    Therefore all we need to do is to apply Lemma 1 on $P_1$ and finish with Lemma 2 on $(P_1, C+P_2)$. 
    
\end{enumerate}

\section*{A-Level}
\begin{enumerate}
	\item[1.] 
	Given an infinite sequence of numbers $a_1, a_2, a_3, .\cdots$.
	For each positive integer $k$ there exists a positive integer $t = t(k)$ such that 
	$a_k = a_{k+t} = a_{k+2t} = \cdots$. 
	Is this sequence necessarily periodic? 
	That is, does a positive integer $T$ exist such that $a_k = a_{k+T}$ for each positive
	integer $k$?
	
	\textbf{Answer.} No. 
	One example will be $a_k=v_2(k)$, 
	the highest power of 2 dividing $k$. 
	It's unbounded (so cannot be periodic), but we can pick $t(k)=2^{a_k+1}$. 
	
	\item [2.]
	Chip and Dale play the following game. Chip starts by splitting 1001 nuts between three
	piles, so Dale can see it. In response, Dale chooses some number $N$ from 1 to 1001. Then
	Chip moves nuts from the piles he prepared to a new (fourth) pile until there will be exactly
	$N$ nuts in any one or more piles. When Chip accomplishes his task, Dale gets an exact
	amount of nuts that Chip moved. What is the maximal number of nuts that Dale can get
	for sure, no matter how Chip acts? (Naturally, Dale wants to get as many nuts as possible,
	while Chip wants to lose as little as possible).
	
	\textbf{Answer.} 71. 
	
	\textbf{Solution.} 
	Consider any initial configuration, 
	and let $S$ be the power set of $\{1, 2, 3\}$, 
	(i.e. the set of  all subsets of $\{1, 2, 3\}$). 
	Now, for each $T\subseteq \{1, 2, 3\}$, 
	let $a_T$ be the number of  of nuts in $T$ 
	(e.g. if $T=\{1, 3\}$ then we're looking for the total number of nuts in boxes 1 and 3). 
	We claim the following: 
	
	\emph{Lemma 1.} For each $N$, the number of nuts Dale can get is $\min_{T\in S}|a_T-N|$. 
	
	Proof: Consider, for each $T$, the minimum number of nuts Chip needs to move such that, either $T$ has $N$ nuts, or $T\cup \{4\}$ has $N$ nuts. 
	The first case is possible only when $a_T\ge N$ to start with (since nuts move out of $T$ to 4). 
	In this case, we need to move $a_T-N$ nuts from $T$ to 4. 
	The second case is possible only when $a_{\{1, 2, 3\}\backslash T}\ge 1001 - N$ to start with, 
	which is the same as saying $a_T\le N$. 
	Here, we need to move $N-a_T$ nuts from $\{1, 2, 3\}\backslash T$ to 4. 
	Therefore, considering both cases (where only one scenario can happen unless $N=a_T$), 
	we need to move $|a_T-N|$ nuts. 
	Since Chip can choose which $T$ he wants to achieve this goal, he can choose such $T$ that minimizes $|a_T-N|$, 
	therefore giving the answer $\min_{T\in S}|a_T-N|$. 
	
	\emph{Lemma 2.} Label the elements in $S$ as 
	$T_0, T_1, \cdots, T_7$ such that $0=a_{T_0}\le \cdots \le a_{T_7}=1001$. 
	Then by optimizing across $N$, Dale can get $\max_{0\le i\le 6} \lfloor \frac{a_{T_{i+1}} - a_{T_i}}{2}\rfloor$
	nuts. 
	
	Proof: in raw form, we're simply looking for $\max_N(\min_{T\in S}|a_T-N|)$. 
	For any $N\le 1001$ there's a unique $i$ with $0\le i\le 6$ with $a_{T_i}\le N\le a_{T_{i+1}}$, 
	so $\min_{T\in S}|a_T-N|=\min (N-a_{T_i}, a_{T_{i+1}}-N)$. 
	Within each $N$ with $a_{T_i}\le N\le a_{T_{i+1}}$, 
	$\min (N-a_{T_i}, a_{T_{i+1}}-N)$ is maximized when we take 
	$N = \lfloor \frac{a_{T_{i+1}} + a_{T_i}}{2}\rfloor$, 
	where $\min (N-a_{T_i}, a_{T_{i+1}}-N)=\lfloor \frac{a_{T_{i+1}} - a_{T_i}}{2}\rfloor$. 
	Thus we just have to choose such $T_i$ that maximizes $\lfloor \frac{a_{T_{i+1}} - a_{T_i}}{2}\rfloor$. 
	
	Now we turn back to the problem. 
	With $a_{T_0}=0$ and $a_{T_7}=1001$ (here $T_0=\emptyset, T_7=\{1, 2, 3\}$), 
	we have $\max_{0\le i\le 6}a_{T_{i+1}} - a_{T_i}\ge \frac{1001}{7}=143$ by pigeonhole principle. 
	Thus, $\max_{0\le i\le 6} \lfloor \frac{a_{T_{i+1}} - a_{T_i}}{2}\rfloor\ge 71$, 
	and Dale is guaranteed to get 71 nuts. 
	To achieve this bound, 
	in the beginning Chip can distribute the nuts in the form $143, 286, 572$ to boxes 1, 2, 3, 
	thereby giving $a_{T_i}=i\times 143$. 
	
	\item [3.] 
	A car rides along a circular track in the clockwise direction. At noon Peter and Paul took
	their positions at two different points of the track. Some moment later they simultaneously
	ended their duties and compared their notes. The car passed each of them at least 30 times.
	Peter noticed that each circle was passed by the car 1 second faster than the preceding
	one while Paul’s observation was opposite: each circle was passed 1 second slower than the
	preceding one.
	Prove that their duty was at least an hour and a half long.
	
	\item [4.] 
	In a triangle $ABC$ two points, $C_1$ and $A_1$ are marked on the sides $AB$ and $BC$ respectively
	(the points do not coincide with the vertices). 
	Let $K$ be the midpoint of $A_1C_1$ and $I$ be the incentre of the triangle $ABC$. 
	Given that the quadrilateral $A_1BC_1I$ is cyclic, prove that the angle $AKC$ is obtuse.
	
	\textbf{Solution.} 
	Let the incircle to intersect $AB$ and $BC$ at $F$ and $D$, respectively. 
	Then $F, K, D$ are feet of perpendicular from $I$ to $AC_1, C_1A_1, A_1A$, respectively. 
	Thus by Simpson's theorem, $F, K, D$ are collinear since $I$ lies on the circumcircle of $A_1BC_1$. 
	Therefore $K$ lies on line $DF$ even as $A_1$ and $C_1$ vary. 
	
	Now, let $K_A$ be what $K$ would have been, if $C_1=A$. 
	Then $\angle IAA_1=\angle IBC$. 
	We also have, by previous points, $\angle IK_AA=90^{\circ}$. 
	This gives 
	\[
	\angle IAK_A+\angle IAC+\angle ICA = 
	\angle IBC + \angle IAC+\angle ICA 
	=\frac{\angle ABC+\angle BAC+\angle BCA}{2}
	=90^{\circ}
	\]
	so coupled with $\angle IK_AA=90^{\circ}$, 
	we havd $AK_AC=90^{\circ}$. 
	Similarly, let $K_C$ be what $K$ would have been if $A_1=C$, then $AK_CC=90^{\circ}$. 
	
	Finally, as we vary $A_1$ and $C_1$ such that $A_1$ is on side $BC$ and $C_1$ on side $AB$, 
	$K$ must lie strictly between $K_A$ and $K_C$. 
	With $\angle AK_AC=\angle AK_CC=90^{\circ}$, 
	$K_AK_C$ is a chord of circle with diameter $AC$ and therefore 
	$K$ lies on the chord (segment), hence inside the circle. 
	It then follows that $\angle AKC>90^{\circ}$. 
	
	\item[5.] 
	Peter and Paul play the following game. First, Peter chooses some positive integer $a$ with
	the sum of its digits equal to 2012. Paul wants to determine this number; he knows only
	that the sum of the digits of Peter’s number is 2012. On each of his moves Paul chooses a
	positive integer $x$ and Peter tells him the sum of the digits of $|x - a|$. What is the minimal
	number of moves in which Paul can determine Peter’s number for sure?
	
	\textbf{Answer.} 2012. 
	
	\textbf{Solution.} Throughout we denote $S(g)$ as the sum of digits of $|g|$. 
	
	Replace 2012 with any integer $k$. 
	We first show Paul can succeed in $k$ steps by inducting on $k$. 
	Base case: when $k=1$, $a=10^g$ for some $g$. 
	Then Paul can pick $x=1$ and $a-x=10^g-1$ is a string of $g$ 9's (hence $S=9g$ here). 
	so $g$ can be determined uniquely. 
	
	Inductive step: we see that Paul can guess the last digit of $a$ by asking $1, 2, \cdots, 9$, until for the first time that the sum of digit goes beyond $k$. 
	To see why, if $d$ is the last digit, and if $a=c\cdot 10^g+d$ with $10\nmid c$
	\[
	S(a-x) = k - x\forall d=1, \cdots, \min(9, d)
	\quad 
	S(a-(d+1)) = S(c\cdot 10^g - 1) = S(c) + 9g = S(a) - d + 9g
	\]
	Hence both $d$ and $g$ can determined uniquely after $d+1$ steps. 
	
	(TODO)
\end{enumerate}
\end{document}