\documentclass[11pt,a4paper]{article}
\usepackage{amsmath, amssymb, fullpage, mathrsfs, bm, pgf, tikz}
\usepackage{mathrsfs}
\usetikzlibrary{arrows}
\setlength{\textheight}{10in}
%\setlength{\topmargin}{0in}
\setlength{\topmargin}{-0.5in}
\setlength{\parskip}{0.1in}
\setlength{\parindent}{0in}

\begin{document}
\newcommand{\la}{\leftarrow}
\newcommand{\lra}{\leftrightarrow}
\newcommand{\bbN}{\mathbb{N}}
\newcommand{\bbZ}{\mathbb{Z}}
\newcommand{\dsum}{\displaystyle\sum}
\newcommand{\dprod}{\displaystyle\prod}


\title{Solutions to Tournament of Towns, Spring 2021, Senior}
\author{Anzo Teh}
\date{}
\maketitle

\section*{O-Level}
\begin{enumerate}
	\item
\end{enumerate}

\section*{A-Level}
\begin{enumerate}
	\item[1.] In a room there are several children and a pile of 1000 sweets. The children come to the pile one
	after another in some order. Upon reaching the pile each of them divides the current number of
	sweets in the pile by the number of children in the room, rounds the result if it is not integer, takes
	the resulting number of sweets from the pile and leaves the room. All the boys round upwards and
	all the girls round downwards. The process continues until everyone leaves the room. Prove that the
	total number of sweets received by the boys does not depend on the order in which the children
	reach the pile.
	
	\textbf{Solution.}
	Consider the situation with $n$ sweets, $b$ boys and $g$ girls. Then we show that the number of sweets boys get will always be
	\[
	f(n, b, g) := b\lfloor \frac{n}{b+g}\rfloor + \min(n\% (b+g), b)
	\]
	where $0\le n\% (b+g) < b+g$ is the remainder after $n$ is divided by $b+g$. 
	We proceed by induction on the total number of people, $b+g$. 
	
	Base case: $b+g=1$, in which case $f(n, 1, 0)=n$ and $f(n, 0, 1)=0$. 
	
	Base case: either $b=0$ or $g=0$. Then $f(n, b, 0)=b\lfloor \frac{n}{b}\rfloor +\min(n\%b, b)=n$
	and $f(n, 0, g)=0$. 
	
	Inductive step: now consider the case where $b, g\ge 1$. 
	Also let $n = k(b+g)+\ell$ with $0\le \ell < b+g$. 
	We'll need to show that the boys will get $kb+\min(b, \ell)$ sweets. 
	
	If $\ell=0$, the first boy will take $k$ sweets; else he will take $k+1$ (hence he will always take $k+\min(1, \ell)$ sweets). 
	Thus, $n-k=k(b+g-1)+\max(0, \ell-1)$ and the boys will get
	\[
	k + \min(1, \ell)
	+f(n-k, b-1, g)
	\]\[
	=k + \min(1, \ell)+(b-1)\lfloor\frac{k(b+g-1)+\max(0, \ell-1)}{b+g-1}\rfloor
	+\min\{\max(0, \ell-1), b\}
	\]
	\[
	=kb + \min(1, \ell)+\min \{\max \{0, \ell-1\}, b\}
	\]
	Depending on the cases $\ell=0, 1\le\ell\le b, b\le \ell < b+g$, we will always have 
	$\min(1, \ell)+\min \{\max \{0, \ell-1\}, b\}=\min(\ell, b)$. 
	
	If girl is the first, then she will take $k$ sweets regardless. Now our tally becomes 
	\[
	f(n-k, b, g-1)
	=b\lfloor\frac{k(b+g-1)+\ell}{b+g-1}\rfloor + \min((n-k)\%(b+g-1), b)
	\]
	\[
	=kb+b\lfloor\frac{\ell}{b+g-1}\rfloor + \min(\ell\%(b+g-1), b)
	\]
	So it suffices to show that $b\lfloor\frac{\ell}{b+g-1}\rfloor + \min((\ell\%(b+g-1), b)=\min(\ell, b)$. 
	This is clear when $\ell<b+g-1$. 
	If $\ell=b+g-1$ we get $b+0=b=\min(\ell, b)$ instead. This solves the claim. 
	
	\item[2.]
	Does there exist a positive integer $n$ such that for any real $x$ and $y$ there exist real numbers $a_1, \cdots , a_n$
	satisfying
	\[
	x=a_1+\cdots a_n
	\quad 
	y=\frac{1}{a_1}+\cdots+\frac{1}{a_n}
	\]
	
	\textbf{Answer.} Yes, pick $n=4$. 
	
	\textbf{Solution.} 
	Consider $x_1, x_2>0$ arbitrary such that $x_1-x_2=x$ (e.g. $x_1=2x, x_2=x$). 
	Cosider, also, $a_1, a_2>0$ with $a_1+a_2=x_1$, and $a_3, a_4<0$ such that $a_3+a_4=-x_2$. 
	Then as we vary $a_1, a_2, a_3, a_4$ in this range with $0<a_1, a_2<x_1$ and $-x_2<a_3, a_4<0$, we have 
	\[
	\text{Range}(\frac{1}{a_1}+\frac{1}{a_2}) = [\frac{x_1^2}{2}, \infty)
	\quad 
	\text{Range}(\frac{1}{a_3}+\frac{1}{a_4}) = (-\infty, -\frac{x_2^2}{2}]
	\]
	Thus $\frac{1}{a_1}+\frac{1}{a_2}+\frac{1}{a_3}+\frac{1}{a_4}$ has range $(-\infty, \infty)$, 
	and therefore can satisfy the sum as $y$. 
	
	\item[3.]
	Let $M$ be the midpoint of the side $BC$ of the triangle $ABC$. 
	The circle $\omega$ passes through $A$, touches the line $BC$ at $M$, 
	intersects the side $AB$ at the point $D$ and the side $AC$ at the point $E$. 
	Let $X$ and $Y$ be the midpoints of $BE$ and $CD$ respectively. 
	Prove that the circumcircle of the triangle $MXY$ touches $\omega$.
	
	\textbf{Solution.} 
	Using power of point theorem, 
	\[
	BD\cdot BA = BM^2 = CM^2 = CE\cdot CA
	\]
	and since $\frac{BM}{BC}=\frac 12 = \frac{BX}{BE}$ and $\frac{CM}{CB}=\frac 12 =\frac{CY}{CD}$, 
	we have triangles $BXM$ and $BEC$ similar, and $CMY$ and $CBD$ similar. 
	Therefore, 
	\[
	MX =  \frac 12 CE\quad MY = \frac 12 BD
	\]
	and therefore $\frac{MX}{MY}=\frac{CE}{BD}=\frac{BA}{CA}$. 
	Moreover, $\angle BMX = \angle BCE$ and $\angle CMY = \angle CBD$, 
	and therefore $\angle XMY = \angle ABC$. 
	It therefore follows that $ABC$ and $MXY$ are similar. 
	This means, 
	\[
	\angle YMC = \angle ABC = \angle MXY
	\]
	and so $MXY$ touches $\omega$. 
	
	\item[4.]
	There is a row of $100N$ sandwiches with ham. A boy and his cat play a game. In one action the
	boy eats the first sandwich from any end of the row. In one action the cat either eats the ham from
	one sandwich or does nothing. The boy performs 100 actions in each of his turns, and the cat makes
	only 1 action each turn; the boy starts first. The boy wins if the last sandwich he eats contains ham.
	Is it true that he can win for any positive integer $N$ no matter how the cat plays?	
	
	\textbf{Answer.} No. 
	
	\textbf{Solution.} Consider the general setting where we have $MN$ sandwiches and the boy making $M$ actions at one turn (thus we have $M=100$ here). 
	We show that the cat can prevent the boy from winning when $N=2^M$. 
	
	Denote $a_i$ as sandwich with position $i$, such that $a_i=1$ if with ham, and $a_i=0$ if without ham. 
	Denote also $S_i$ as the subsequence $a_{Mk+i}$ for $k=0, \cdots, N-1$. 
	We claim the following claims: 
	
	\emph{Lemma 1.} For each $i$ with $1\le i\le M$, the boy will eat exactly one sandwich in $S_i$ in each of his turn. 
	
	Proof: since the boy takes $M$ sandwiches in each turn, the number of sandwiches after each turn is multiple of $M$. 
	Let $N'\le N$ be such that there are $MN'$ sandwiches left, 
	and that the index of the first sandwich at left end as $\ell$. 
	Then the remaining sandwiches are 
	\[
	a_{\ell}, \cdots, a_{\ell+MN'-1}
	\]
	Now suppose the boy takes $k$ sandwiches from left end and $M-k$ sandwiches from right end. 
	The sandwiches taken have indices
	\[
	\ell, \cdots, \ell+k-1, \ell+MN'-(M-k), \cdots, \ell+MN'-1
	\]
	which, considering modulo $M$, have indicies 
	\[
	\ell, \cdots, \ell+k-1, \ell+k, \cdots, \ell+M-1
	\]
	those containing all the remainders mod $M$ exactly once. In particular, this holds for those with indices $\equiv i\pmod{M}$. 
	
	\emph{Lemma 2.} Fix $i$ with $1\le i\le M$. 
	Then starting with $MN'$ sandwiches with $N'$ even, 
	the cat can perform actions such that after $\frac{N'}{2}$ turns from each player, 
	all the remaining sandwiches in $S_i$ have no ham (i.e. $a_{Mk+i}=0, \forall k$ remaining). 
	
	Proof: 
	By lemma 1, it suffices to focus on those $N'$ sandwiches in the form $S_i$, in which the boy will take just one sandwich from one end. 
	Now at first turn, after the boy makes his move, the cat considers the middle sandwich in $S_i$ (i.e. $\frac{N'+1}{2}$-th from the left), 
	and eats the ham from that sandwich (or do nothing, if the sandwich already didn't have ham). 
	Now relabel the remaining $N'-2$ sandwiches into the following: 
	\[
	b_1, b_2, \cdots, b_{\frac{N'}{2}-1}, b_{\frac{N'}{2}+1}, \cdots, b_{N'-1}
	\]
	and consider the pairing $(b_g, b_{g+\frac{N'}{2}})$ for $g=1, \cdots, \frac{N'}{2}-1$. 
	Then at each step, for each sandwich $A$ the boy takes, the cat removes ham from the sandwich paired with $A$. 
	We see the boy will never take the image later on within the $\frac{N'}{2}$ steps: 
	if the boy takes $b_g$ for some $g\le \frac{N'}{2}-1$, 
	this means it already takes $g$ sandwiches from $S_i$ from the left, 
	while $b_{\frac{N'}{2}+g}$ is the $\frac{N'}{2}-g$-th sandwich from the right. 
	Thus for the boy to take that there needs to be at least $g+(\frac{N'}{2}-g)+1=\frac{N'}{2}+1$ steps 
	(the +1 is for the very first step before the cat acts from the center). 
	Consequently, after $\frac{N'}{2}$ steps, for each of the $\frac{N'}{2}-1$ pairs there's one sandwich being eaten by the boy and the other has ham taken out, establishing the lemma. 
	
	Now by Lemma 2, we can do the following: starting from $N=2^M$, 
	the cat removes ham from $S_1$, then $M2^{M-1}$ sandwiches left, 
	and then removes ham from $S_2$, then $M2^{M-2}$ sandwiches left. 
	Continuing this until the cat reaches $S_M$, and exactly $M$ sandwiches left. 
	But since $S_1, \cdots, S_M$ represents all sandwiches, 
	we conclude that the all of the last $M$ sandwiches do not have ham, so the boy loses. 
\end{enumerate}
\end{document}