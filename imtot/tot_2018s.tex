\documentclass[11pt,a4paper]{article}
\usepackage{amsmath, amssymb, fullpage, mathrsfs, bm, pgf, tikz}
\usepackage{mathrsfs}
\usetikzlibrary{arrows}
\setlength{\textheight}{10in}
%\setlength{\topmargin}{0in}
\setlength{\topmargin}{-0.5in}
\setlength{\parskip}{0.1in}
\setlength{\parindent}{0in}

\begin{document}
\newcommand{\la}{\leftarrow}
\newcommand{\lra}{\leftrightarrow}
\newcommand{\bbN}{\mathbb{N}}
\newcommand{\bbZ}{\mathbb{Z}}
\newcommand{\dsum}{\displaystyle\sum}
\newcommand{\dprod}{\displaystyle\prod}


\title{Solutions to Tournament of Towns, Spring 2018, Senior}
\author{Anzo Teh}
\date{}
\maketitle

\section*{O-Level}
\begin{enumerate}
	\item
\end{enumerate}

\section*{A-Level}
\begin{enumerate}
	\item[1.]
	Aladdin has several gold ingots, and sometimes he asks the Genie to give him more. The Genie
	first adds a thousand ingots, but then takes half of the total number. Could it be possible
	that after asking the Genie for gold ten times, the number of Aladdin’s gold ingots increased,
	assuming that each time the Genie took half, he took an integer number of ingots?
	
	\textbf{Answer.} No. 
	
	\textbf{Solution.} Let $a$ be the number of ingots Aladdin has the beginning. 
	Iteratively, we can show that after asking Genie for $n$ times, 
	the number of ingots Aladdin has becomes $\frac{a}{2^n}+1000(1-\frac{1}{2^n})$. 
	Since the Genie takes an integer number every time, 
	it means that $2^n\mid a-1000$. 
	Moreover, this number increases (i.e. $>a$) if and only if $a < 1000$. 
	But with $n=10$, the smallest positive $a$ satisfying $2^n\mid a-1000$ is $a=1000$, hence Aladdin cannot have more ingots later on. 
	
	\item[2.] Do there exist 2018 positive irreducible fractions, each with a different denominator, so that that the denominator of the difference of any two (after reducing the fraction) is less than the denominator of any of the initial 2018 fractions?
	
	\textbf{Solution.} Let $n=2018$. Define: 
	\[
	a_k = \frac{1}{2^k} + \frac{1}{2^{n}+1} = \frac{2^n+2^k+1}{2^k(2^n+1)}\quad \forall k = 1, 2, \cdots , n
	\]
	Then $a_k-a_{\ell}$ has denominator $2^{\max(k, \ell)}\le 2^n$. We now show that $\gcd(2^n+2^k+1, 2^k(2^n+1))=1$. Otherwise, there exists a prime $p$ dividing both numerator and denominator, and therefore $p$ divides either $2^k$ or $2^n+1$. In the former case, $p=2$ but $2^n+2^k+1$ is odd; in the latter case, $p\mid 2^n+1$ so $p$ is odd but then $2^n+2^k+1\equiv 2^k\pmod{2^n+1}$ so $p\mid 2^k$, contradicting that $p$ is odd. So 
	\[
	a_k = \frac{2^n+2^k+1}{2^k(2^n+1)}
	\]
	is already in its lowest term, with denominator $2^k(2^n+1)\ge 2(2^n+1)>2^n$ (hence different for each of them), and certainly greater than $2^n$. 
	
	\item[3.] One hundred different numbers are written in the cells of a 10 $\times$ 10 square. Each step you can
	select a rectangle formed by the cells, and for every cell inside swap its number with the number
	in the cell opposite to it with respect to the center of the rectangle (``turn the rectangle by $180^{\circ}$'')
	Is it always possible to arrange the numbers in the square so that they increase in every row
	from left to right, and in every column from bottom to top, in no more than 99 turns?
	
	\textbf{Answer.} Yes. 
	
	\textbf{Solution.} 
	Let $a_{i, j}$ with $1\le i, j\le 10$ be the cell, with $a_{1, 1}$ being the bottom-left cell. 
	Denote $R_{i, j}$ as the rectangle with coordinates $(i', j')$ with $i'\ge i, j'\ge j$. 
	Consider the following 99 operations: 
	at step $10(i-1)+j$ (which we would also say step $(i, j)$) where $1\le i\le 10, 1\le j\le 10$ and $(i, j)\neq (10, 10)$, 
	take 
	\[a_{i_0, j_0} := \min\{a_{i', j'}\in R_{i, j}\}
	\]
	Then we choose a rectangle with opposite vertices $a_{i, j}$ and $a_{i_0, j_0}$ and perform the turn. 
	(Basically, we're re-filling the first row from left to right, second row from left to right, etc). 
	Notice that we could have $i_0=i$ and $j_0=j$, in which case we save one step. 
	
	To verify this works, we first see that after each step $(i, j)$ where we place $a_{i_0, j_0}$ to $a_{i, j}$, 
	the new $a_{i, j}$ will not be changed anymore: in subsequent steps $(i', j')$, 
	either $i'>i$ or $j'>j$, and therefore $a_{i, j}\not\in R_{i', j'}$. 
	Thus it suffices to show that after each step $(i, j)$: 
	\[
	\forall j\ge 2: a_{i, j} > a_{i, j - 1}
	\quad 
	\forall i\ge 2: a_{i, j} > a_{i-1, j}
	\]
	The first one is relatively more straightforward: step $(i, j)$ follows right after step $(i, j-1)$. 
	According to our algorithm, after step $(i-1, j)$ we have 
	\[
	a_{i, j-1} = \min\{a_{i', j'}\}_{i\le i', j-1\le j'}
	\]
	and therefore after this step, $\min\{a_{i', j'}\}_{i\le i', j\le j'} > a_{i, j-1}$
	and $\min\{a_{i', j'}\}_{i\le i', j\le j'}$ would be the new $a_{i, j}$ after step $(i, j)$. 
	
	For the second claim, consider the set $S_{i, j}$ as the cells with values $<a_{i-1, j}$ after step $(i-1, j)$
	and with coordinates $(i', j')$ with $10i'+j'>10(i-1)+j$. 
	By how $a_{i-1, j}$ is constructed, $S_{i, j}\cap R_{i-1, j}=\emptyset$. 
	It follows that all the cells in $S_{i, j}$ must have coordinates $(i', j')$ with $i'\ge i, j'<j$, 
	after step $(i-1, j)$. 
	We show that they will stay in that region until step $(i, j)$ by showing that this invariant stays from one step to the another. 
	
	Indeed, 
	\begin{itemize}
		\item At step $(i-1, j_1)$ where $j_1>j$, $S_{i, j}\cap R_{i-1, j_1}=\emptyset$, so cells in $S_{i, j}$ won't be moved. 
		
		\item At step $(i, j_1)$ where $j_1<j$, if $S_{i, j}\cap R_{i, j_1}=\emptyset$ then $S_{i, j}$ won't be moved; 
		otherwise, given that $S_{i, j}$ have cell values $<a_{i-1, j}$, 
		we have $a_{i_0, j_0}:=\min\{a_{i', j'}\in R_{i, j_1}\}$ coming from $S_{i, j}$. 
		Since $j_0<j$, the rectangle where we swap at this step $(i, j_1)$ has $y$-coordinates in $j_1\le y\le j_0<j$. 
		It follows that $S_{i, j}$ cannot be moved into $R_{i, j}$. 
	\end{itemize}
	Consequently, $a_{i, j}$ won't consider any cell in $S_{i, j}$ and therefore $a_{i, j}>a_{i-1, j}$. 
	
	Finally, we see why the 100-th step is not needed. At steps $(i, j)$ for $(i, j)\neq (10, 10)$, $R_{i, j}$ has area at least 2, 
	so the biggest cell can never be placed at $a_{i, j}$. 
	It follows that after 99 steps, $a_{10, 10}$ must have the biggest value. 
	
	\item[4.]
	An equilateral triangle lying in the plane $\alpha$ is orthogonally projected onto a plane $\beta$, which is
	not parallel to $\alpha$. The resulting triangle is again orthogonally projected onto a plane $\gamma$, and it
	turned out that the image is again an equilateral triangle. Prove that
	\begin{enumerate}
		\item the angle between the planes $\alpha$ and $\beta$ is equal to the angle between the planes $\beta$ and $\gamma$;
		\item the plane $\beta$ intersects the plane $\alpha$ and $\gamma$ along perpendicular lines.
	\end{enumerate}

    \textbf{Solution.}
    Let the triangle on $\beta$ be $A_2B_2C_2$. 
    Denote $d(X, y)$ as the unsigned distance of point $X$ to line $y$. 
    Define a $(\ell, a)$transformation of a point $A$ on $\beta$ as point $A_1$ such that $A$ and $A_1$ has the same foot of perpendicular to $\ell$, 
    and $d(A_1, \ell)=a\cdot d(A, \ell)$. 
    
    Now let $\alpha$ and $\beta$ intersect at $\ell_{1}$, with angle between them $\theta_{1}$. 
    and $\beta$ and $\gamma$ intersect at $\ell_3$, with angle between them $\theta_3$. 
    Let's consider the following lemma: 
    
    \emph{Lemma 1.} Suppose that $A_1, B_1, C_1$ are obtained by doing 
    $(\ell_1, \frac{1}{\cos\theta_1})$-transformation on $A_2, B_2, C_2$, 
    and $A_3, B_3, C_3$ are obtained by doing $(\ell_3, \cos\theta_3)$-transformation on $A_3, B_3, C_3$. 
    Then $A_1B_1C_1$ and $A_3B_3C_3$ are both equilateral. 
    
    Proof: let $A_{\alpha}, B_{\alpha}, C_{\alpha}$ be the original equilateral triangle on 
    plane $\alpha$. 
    Recall that $\alpha$ can be obtained by rotating $\beta$ about line $\ell_{1}$ at angle $\theta_{\alpha}$. 
    Since $A_2$ is projection of $A_{1}$, 
    it follows that $A_{\alpha}$ can be obtained by rotating $A_1$ in that manner (and similarly for $B_1, C_1)$. 
    Thus $A_1B_1C_1$ is congruent to $A_{\alpha}B_{\alpha}C_{\alpha}$. 
    In a similar way, if $A_{\gamma}B_{\gamma}C_{\gamma}$ is the equilateral triangle on 
    $\gamma$, 
    then $A_3, B_3, C_3$ can be obtained by rotating $A_{\gamma}B_{\gamma}C_{\gamma}$ 
    about $\ell_{3}$ at angle $\theta_3$, 
    hence the two are congruent. 
    
    Before proceeding, let's dig further into the properties of  the transformations we talked about above: 
    \begin{itemize}
    	\item it maps lines to lines. I.e. $\ell_0$ is mapped under $(\ell, a)$-transformation 
    	to a line $\ell_1$ such that $\ell_0, \ell, \ell_1$ are concurrent (or mutually parallel), 
    	and $\tan(\angle(\ell_1, \ell))=a\tan\angle(\ell_0, \ell)$ (assuming $\neq 0$: if $\ell_0\perp\ell$ then $\ell_1\perp \ell$, for this limiting case). 
    	
    	\item By the first point, the same transformation will map parallel lines to parallel lines. 
    	
    	\item If $A_1, B_1, C_1$ on the same line with $\overrightarrow{A_1B_1}=\lambda\overrightarrow{A_1C_1}$ 
    	are mapped under $(\ell, a)$-transformation to $A_2, B_2, C_2$, then we also have 
    	$\overrightarrow{A_2B_2}=\lambda\overrightarrow{A_2C_2}$. 
    \end{itemize}
    Thus from now on we only consider lines mod $180^{\circ}$ after the transformations. 
    Let's now proceed to the following: 
    
    \emph{Lemma 2.} There's a constant angle $\theta_0$ such that if any line $m$ is mapped under $(\ell_1, \frac{1}{\cos\theta_1})$-transformation to $m_1$ and under $(\ell_3, \cos\theta_3)$-transformation to $m_3$, then $\angle(m_1, m_3)=\theta_0$ (i.e. $\theta_0$ doesn't depend on $m$). 
    
    Proof: we may assume that $\theta_1$ and $\theta_3$ are acute (if any of them is $90^{\circ}$ we end up with degenerate triangles, if it's obtuse we may replace $\theta$ with $180^{\circ}-\theta$). 
    This means that $A_1, B_1, C_1$ and $A_3, B_3, C_3$ have the same orientation, 
    i.e. in directed angle, $\angle(A_1B_1,  A_3B_3)=\angle(A_1C_1, A_3C_3)=\angle(B_1C_1, B_3 C_3)$. 
    Consider, now, any point $D_2$ on line $A_2B_2$ with $\overrightarrow{A_2D_2}=\lambda\overrightarrow{B_2D_2}$, 
    and map $D_2$ under $(\ell_1, \frac{1}{\cos\theta_1})$-transformation
    to $D_1$, 
    and $(\ell_3, \cos\theta_3)$ to $D_3$, 
    then we also have 
    \[
    \overrightarrow{A_1D_1}=\lambda\overrightarrow{B_1D_1}
    \qquad 
    \overrightarrow{A_3D_3}=\lambda\overrightarrow{B_3D_3}
    \]
    But given that $A_1B_1C_1$ and $A-3B_3C_3$ are similar, 
    we also have 
    \[
    \angle(C_1D_1,  C_3D_3)=\angle(A_1B_1,  A_3B_3)
    \]
    and by varying $\lambda$ (i.e. varying $D$), lines $C_1D_1$ and $C_3D_3$ would run through all the lines in the plane (up to parallels) (technically, extra care need to be taken when $D_i$ are points of infinity, but that's fine since the transformations we talked about map parallel lines to parallel lines), 
    which proves the lemma with $\theta_0=\angle(A_1B_1,  A_3B_3)$. 
    
    Now we can finish the solution. 
    Consider, first, mapping $\ell_1$ under both transformations, 
    which maps to $\ell_1$ and $\ell_2$ such that $\tan\angle(\ell_2, \ell_3)=\cos\theta_3\tan\angle(\ell_1, \ell_3)$. 
    Then we map $\ell_3$ under both transformations, 
    which maps to $\ell_4, \ell_3$ such that $\tan\angle(\ell_4, \ell_1)=\frac{1}{\cos\theta_1}\tan\angle(\ell_3, \ell_1)$. 
    Since $\angle(\ell_1, \ell_2)=\angle(\ell_1, \ell_3)+\angle(\ell_3, \ell_2)$ 
    and $\angle(\ell_4, \ell_3)=\angle(\ell_4, \ell_1)+\angle(\ell_1, \ell_3)$, 
    from $\angle(\ell_1, \ell_2)=\angle(\ell_4, \ell_3)$
    we have $\angle(\ell_3, \ell_2)=\angle(\ell_4, \ell_1)$, 
    i.e. 
    \[
    \arctan(\cos\theta_3\tan\angle(\ell_3, \ell_1)) = \arctan(\frac{1}{\cos\theta_1}\tan\angle(\ell_3, \ell_1))
    \]
    Given that $0<b\le 1$ and $0<a<1$, by injectivity of $\arctan$ (even in modulo $180^{\circ}$)
    we have either $\angle(\ell_3, \ell_1)\in \{0, 90^{\circ}\}$. 
    
    In the first case where $\ell_1\parallel \ell_3$, the expression $\angle(\ell_3, \ell_2)=\angle(\ell_4, \ell_1)$ gives 0. 
    It then follows the outputs of $(\ell_1, \frac{1}{\cos\theta_1})$-transformation 
    and $(\ell_3, \cos\theta_3)$-transformation must be parallel for every input line. 
    However, this would only work if $\frac{1}{\cos\theta_1}=\cos\theta_3$, 
    which is impossible with $|\cos\theta_1|<1$. 
    
    Thus we need $\ell_1\perp \ell_3$ (solving (b)). 
    When the input angle is $\ell_1$ (resp $\ell_3$) then both transformations return $\ell_1$ (resp $\ell_3$), 
    so we see that the two transformations must produce lines that are parallel. 
    Now take any line $\ell$ that's the bisector of $\ell_1, \ell_3$ (i.e. $45^{\circ}$), and let $\ell_2$, $\ell_4$ be its image under the two transformations. 
    Then  $\tan$ of $(\ell_2, \ell_3)$ and $(\ell_4, \ell_3)$ will be $\cos\theta_1$ and $\cos\theta_3$, 
    respectively. 
    Thus $\theta_1=\theta_3$ (solving (a)). 
    
	
	\item[6.]
	The quadrilateral $ABCD$ is inscribed into a circle $S$.
	 Let $P$ be the intersection point of the rays $BA$ and $CD$. 
	 Let $U$ and $V$ be the intersection points of the line, 
	 which passes through $P$ and is parallel to the tangent to $S$ at the point $D$, 
	 with the tangents to $S$ at points $A$ and $B$
	respectively. Prove that the circumcircle of the triangle $CUV$ is tangent to the circle $S$.
	
	\textbf{Solution.}
	    Let's define the following new points: 
	    \begin{itemize}
	    	\item $E$ as the point diametrically opposite $D$ on $S$
	    	
	    	\item $W$ as the intersection of the tangent to $S$ at $C$, and line $UV$  (could be point of infinity if $CD$ is the diemeter of $S$, i.e. $C$ coincides with $E$)
	    	
	    	\item $T_A$ be such that $T_AA$ and $T_AD$ tangent to $S$
	    	
	    	\item $T_B$ be such that $T_BB$ and $T_BD$ tangent to $S$
	    	
	    	\item $X$ the intersection of $AU$ and $CD$
	    	
	    	\item $Y$ the intersection of $BV$ and $CD$
	    \end{itemize}
        Also denote the diameter and radius of $S$ as $d$ and $r$, respectively, 
        and $\angle(AB)$ as the non-obtuse angle subtended by chord $AB$ on $S$
        (define this for other chords similarly). 
        Then $AB=d\sin\angle (AB)$. 
        Let's now set up some identity on the length of tangents
        \[
        T_AA=T_AD = r|\tan \angle(AD)|
        =r|\sin\angle (AD) / \cos\angle(AD)|
        =r|\frac{AD/d}{\sin\angle (AE)}|
        =r\frac{AD}{AE}
        \]
        and similarly, $T_BB=T_BD=r\frac{BD}{BE}$. 
        In a similar way we can show that $CW=WP=r\frac{CD}{CE}\cdot \frac{CP}{CD}=r\frac{CP}{CE}$. 
        
	    
		\definecolor{uuuuuu}{rgb}{0.26666666666666666,0.26666666666666666,0.26666666666666666}
		\definecolor{xdxdff}{rgb}{0.49019607843137253,0.49019607843137253,1}
		\definecolor{ududff}{rgb}{0.30196078431372547,0.30196078431372547,1}
		\begin{tikzpicture}[line cap=round,line join=round,>=triangle 45,x=1cm,y=1cm, scale=0.5]
		\begin{scope}[shift={(-100,-100)}]
		\clip(-13.891835513705532,-9.774652621583082) rectangle (21.896914183311154,15.365923448849884);
		\draw [line width=2pt] (-3.5727197604952736,2.0334410291816316) circle (2.370067656956057cm);
		\draw [line width=2pt] (-6.960243444604625,-1.4828124441950605)-- (7.400749642851678,-3.73963805339991);
		\draw [line width=2pt] (0.4723076389937778,-2.6508355415558857) -- (0.17272074781344549,-2.9136906966417873);
		\draw [line width=2pt] (0.4723076389937778,-2.6508355415558857) -- (0.26778545043360963,-2.308759800953183);
		\draw [line width=2pt] (-5.863180638709392,2.6425466554237285)-- (-6.960243444604625,-1.4828124441950605);
		\draw [line width=2pt] (-2.7432580368129584,-2.14551039300614)-- (-2.0886541179199245,3.8813506230407544);
		\draw [line width=2pt] (-3.9406595420199375,-0.3078921414117284)-- (4.847249945936461,-1.6889093916127134);
		\draw [line width=2pt] (0.7053497418285125,-1.038011059270622) -- (0.4057628506481821,-1.3008662143565215);
		\draw [line width=2pt] (0.7053497418285125,-1.038011059270622) -- (0.5008275532683444,-0.6959353186679194);
		\draw [line width=2pt] (-5.721960978953942,-1.6774081495819488)-- (-2.52,-0.09);
		\draw [line width=2pt] (-11.700168486756919,11.600406953129657)-- (-2.7432580368129584,-2.14551039300614);
		\draw [line width=2pt] (-11.700168486756919,11.600406953129657)-- (7.400749642851678,-3.73963805339991);
		\draw [line width=2pt,dash pattern=on 1pt off 1pt] (1.3777089221149161,0.7169662569780524) circle (7.492552402414858cm);
		\begin{scriptsize}
		\draw [fill=ududff] (-2.52,-0.09) circle (2.5pt);
		\draw[color=ududff] (-2.25708988977045,0.6523927986979794) node {$A$};
		\draw [fill=ududff] (-2.0886541179199245,3.8813506230407544) circle (2.5pt);
		\draw[color=ududff] (-1.8148334771647305,4.598680788102884) node {$B$};
		\draw [fill=ududff] (-5.863180638709392,2.6425466554237285) circle (2.5pt);
		\draw[color=ududff] (-5.591022846336643,3.3739707224254993) node {$C$};
		\draw [fill=xdxdff] (-3.9406595420199375,-0.3078921414117284) circle (2.5pt);
		\draw[color=xdxdff] (-4.468371952799047,0.38023500632522744) node {$D$};
		\draw [fill=xdxdff] (-3.2047799789706097,4.374774199774992) circle (2.5pt);
		\draw[color=xdxdff] (-3.5838591275876084,5.108976648801794) node {$E$};
		\draw [fill=uuuuuu] (-2.7432580368129584,-2.14551039300614) circle (2pt);
		\draw[color=uuuuuu] (-2.4612082340500128,-1.4908498162374426) node {$P$};
		\draw [fill=uuuuuu] (-6.960243444604625,-1.4828124441950605) circle (2pt);
		\draw[color=uuuuuu] (-6.679654015827645,-0.8104553353055626) node {$W$};
		\draw [fill=uuuuuu] (-5.721960978953942,-1.6774081495819488) circle (2pt);
		\draw[color=uuuuuu] (-5.454943950150268,-1.0145736795851266) node {$U$};
		\draw [fill=uuuuuu] (7.400749642851678,-3.73963805339991) circle (2pt);
		\draw[color=uuuuuu] (7.676669531834942,-3.0897768464273607) node {$V$};
		\draw [fill=uuuuuu] (-3.7010969457247964,-0.6755420879308777) circle (2pt);
		\draw[color=uuuuuu] (-3.821997195913765,-1.0485934036317206) node {$X$};
		\draw [fill=uuuuuu] (-11.700168486756919,11.600406953129657) circle (2pt);
		\draw[color=uuuuuu] (-11.442415382350779,12.253118698586533) node {$Y$};
		\draw [fill=uuuuuu] (-3.1956639025945712,-0.42496797506488015) circle (2pt);
		\draw[color=uuuuuu] (-3.2096421630750767,0.48229417846500944) node {$T_A$};
		\draw [fill=uuuuuu] (4.847249945936461,-1.6889093916127134) circle (2pt);
		\draw[color=uuuuuu] (5.227249400480187,-0.9125145074453445) node {$T_B$};
		\end{scriptsize}
		\end{scope}
		\end{tikzpicture}
		
		Next, take triangle $ADP$ and the cevian $AX$. 
		Then given $AX$ is tangent to $S$ we have 
		\[
		\frac{DX}{XP}
		=\frac{\text{Area}(ADX)}{\text{Area}(AXP)}
		=\frac{1/2 \cdot AD\cdot AX\cdot \sin\angle DAX}{1/2 \cdot AP\cdot AX\cdot\sin\angle PAX}
		=\frac{AD\cdot \sin\angle DAX}{AP\cdot\sin\angle PAX}
		\]\[
		=\frac{AD\cdot \sin\angle (AD)}{AP\cdot\sin\angle (AB)}
		=\frac{AD^2/d}{AB/d\cdot AP}
		=\frac{AD^2}{AB\cdot AP}
		\]
		and since $DT_A$ and $UP$ are parallel with $T_AU$ and $PD$ meeting at $X$, 
		\[
		PU = T_AD\frac{PX}{XD}
		=r\frac{AD}{AE}\cdot \frac{AB\cdot AP}{AD^2}
		=r\frac{AB\cdot AP}{AE\cdot AD}
		\]
		and similarly, $PV = r\frac{AB\cdot BP}{BE\cdot BD}$. 
		Finally, we also have 
		\[
		\frac{AP}{BP}
		=\frac{\text{Area}(APD)}{\text{Area}(BPD)}
		=\frac{1/2 \cdot PD\cdot DA\cdot \sin\angle PDA}{1/2\cdot PD\cdot DB\cdot \sin\angle PDB}
		=\frac{DA\cdot \sin\angle (AC)}{\cdot DB\cdot \sin\angle (BC)}
		=\frac{DA\cdot AC}{DB\cdot BC}
		\]
		This gives 
		\[
		\frac{PU}{PV}
		=\frac{AP \cdot BE\cdot BD}{AE\cdot AD\cdot BP}
		=\frac{DA\cdot AC\cdot BE\cdot BD}{DB\cdot BC\cdot AD\cdot AE}
		=\frac{AC\cdot BE}{BC\cdot AE}
		\]
		Meanwhile, 
		we have by sine rule 
		\[
		\frac{AP}{CP}=\frac{\sin\angle ACP}{\sin\angle CAP}
		=\frac{\sin\angle (AD)}{\sin\angle (BC)}
		=\frac{AD}{BC}
		\quad 
		\frac{BP}{CP}=\frac{\sin\angle BCP}{\sin\angle CBP}
		=\frac{\sin\angle (BD)}{\sin\angle (AC)}
		\]
		so 
		\[
		\frac{PU}{PW}
		=r\frac{AB\cdot AP}{AE\cdot AD}/r\frac{CP}{CE}
		=\frac{AB\cdot AD \cdot CE}{AE\cdot AD\cdot BC}
		=\frac{AB\cdot CE}{AE\cdot BC}
		\]
		and similarly 
		\[
		\frac{PV}{PW}
		=r\frac{AB\cdot BP}{BE\cdot BD}/r\frac{CP}{CE}
		=\frac{AB\cdot BD \cdot CE}{BE\cdot BD\cdot AC}
		=\frac{AB\cdot CE}{BE\cdot AC}
		\]
		
		Now that we have 
		\[PU:PV:PW = (AC\cdot BE) : (BC\cdot AE) : (\frac{1}{AB\cdot CE})
		\]
		Ptolemy's theorem says that among $AC\cdot BE, BC\cdot AE$, and $AB\cdot CE$, 
		one will be the sum of the other two. 
		We'll distinguish into three cases as below: 
		
		\emph{Case 1.} $E, B, A, C$ are in that order (like provided in the figure). 
		Then on line $UV$, $W, U, P, V$ will be in that order. In addition, we have 
		by Ptolemy's theorem we have 
		\[
		EA\cdot BC = EC\cdot AB + EB\cdot AC
		\]
		Therefore 
		\begin{flalign*}
		  WU \cdot WV
		  &= (WP-UP) \cdot (WP+PV)
		  \\&=WP^2 + WP(PV-UP) - UP\cdot PV
		  \\&=WP^2 + WP^2(\frac{AB\cdot CE}{BE\cdot AC}-\frac{AB\cdot CE}{AE\cdot BC})
		  -WP^2 (\frac{AB^2\cdot CE^2}{BE\cdot AC\cdot AE\cdot BC})
		  \\&=WP^2 + WP^2(AB\cdot CE)(\frac{AE\cdot BC - BE\cdot AC}{BE\cdot AC\cdot AE\cdot BC})
		  -WP^2 (\frac{AB^2\cdot CE^2}{BE\cdot AC\cdot AE\cdot BC})
		  \\&=WP^2 + WP^2(AB\cdot CE)(\frac{AB\cdot CE}{BE\cdot AC\cdot AE\cdot BC})
		  -WP^2 (\frac{AB^2\cdot CE^2}{BE\cdot AC\cdot AE\cdot BC})
		  \\&=WP^2
		\end{flalign*}
		
		\emph{Case 2.} 
		$E, C, B, A$ in that order. 
		Then on line $UV$, $U, P, V, W$ will be in that order and Ptolemy's theorem would give 
		\[
		EB\cdot AC = EA\cdot BC + EC\cdot AB
		\]
		Therefore 
		\begin{flalign*}
		WU \cdot WV
		&= (WP+UP) \cdot (WP-PV)
		\\&=WP^2 + WP(UP-PV) - UP\cdot PV
		\\&=WP^2 + WP^2(\frac{AB\cdot CE}{AE\cdot BC}-\frac{AB\cdot CE}{BE\cdot AC})
		-WP^2 (\frac{AB^2\cdot CE^2}{BE\cdot AC\cdot AE\cdot BC})
		\\&=WP^2 + WP^2(AB\cdot CE)(\frac{ BE\cdot AC-AE\cdot BC }{BE\cdot AC\cdot AE\cdot BC})
		-WP^2 (\frac{AB^2\cdot CE^2}{BE\cdot AC\cdot AE\cdot BC})
		\\&=WP^2 + WP^2(AB\cdot CE)(\frac{AB\cdot CE}{BE\cdot AC\cdot AE\cdot BC})
		-WP^2 (\frac{AB^2\cdot CE^2}{BE\cdot AC\cdot AE\cdot BC})
		\\&=WP^2
		\end{flalign*}
		
		\emph{Case 3.} $E, B, C, A$ in that order. Then on line $UV$, we either have $P, W, U, V$ or $P, W, V, U$ in that order and Ptolemy's theorem gives 
		\[
		EC\cdot AB = EA\cdot BC + EB\cdot AC
		\]
		Therefore 
		\begin{flalign*}
		WU \cdot WV
		&= (UP-WP) \cdot (VP-WP)
		\\&=WP^2 - WP(UP+PV) + UP\cdot PV
		\\&=WP^2 - WP^2(\frac{AB\cdot CE}{AE\cdot BC}+\frac{AB\cdot CE}{BE\cdot AC})
		+WP^2 (\frac{AB^2\cdot CE^2}{BE\cdot AC\cdot AE\cdot BC})
		\\&=WP^2 - WP^2(AB\cdot CE)(\frac{ BE\cdot AC+AE\cdot BC }{BE\cdot AC\cdot AE\cdot BC})
		+WP^2 (\frac{AB^2\cdot CE^2}{BE\cdot AC\cdot AE\cdot BC})
		\\&=WP^2 + WP^2(AB\cdot CE)(\frac{AB\cdot CE}{BE\cdot AC\cdot AE\cdot BC})
		+WP^2 (\frac{AB^2\cdot CE^2}{BE\cdot AC\cdot AE\cdot BC})
		\\&=WP^2
		\end{flalign*}
		
		Thus all three cases point to $WP^2=WU\cdot WV$, and that $W$ does not lie between $U$ and $V$. 
		Finally, $WC=WP$. This gives $WC$ tangent to both circles $S$ and $CUV$ so $CUV$ is indeed tangent to $S$. 
		
		\textbf{Remark}: we could have made things shorter by considering directed length for the three cases, 
		but it's also harder to parse that way. 
\end{enumerate}
\end{document}