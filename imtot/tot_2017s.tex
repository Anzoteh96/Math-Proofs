\documentclass[11pt,a4paper]{article}
\usepackage{amsmath, amssymb, fullpage, mathrsfs, bm, pgf, tikz}
\usepackage{mathrsfs}
\usetikzlibrary{arrows}
\setlength{\textheight}{10in}
%\setlength{\topmargin}{0in}
\setlength{\topmargin}{-0.5in}
\setlength{\parskip}{0.1in}
\setlength{\parindent}{0in}

\begin{document}
\newcommand{\la}{\leftarrow}
\newcommand{\lra}{\leftrightarrow}
\newcommand{\bbN}{\mathbb{N}}
\newcommand{\bbZ}{\mathbb{Z}}
\newcommand{\dsum}{\displaystyle\sum}
\newcommand{\dprod}{\displaystyle\prod}


\title{Solutions to Tournament of Towns, Spring 2017, Senior}
\author{Anzo Teh}
\date{}
\maketitle

\section*{O-Level}
\begin{enumerate}
	\item[3.]
	Each cell of a square $1000 \times 1000$ table contains a number. It is known
	that the sum of the numbers in each rectangle of area $S$ with sides along
	the borders of cells, contained in the table, is the same. Find all values
	of S which guarantee that all the numbers in the table are equal. 
	
	\textbf{Answer.} $S=1$ is the only possibility. 
	
	\textbf{Solution.} The case $S=1$ is clear. We now show that for any $S>1$ we can construct a counterexample. 
	
	Let $p$ be any prime divisor of $S$. Denote $a_{i, j}$ the squares of table, $1\le i, j\le 1000$. 
	Let $a_{i, j}$ be filled with $i+j\pmod{p}$, taking values $0, 1, \cdots, p-1$. 
	Then any rectangle with area $S$ will have either its row or column size divisible by $p$. 
	In the first case (row size divisible by $p$), for each column there's equal representation of $0, 1, \cdots, p-1$; 
	for the second case (column size divisible by $p$), 
	for each row there's equal representation of $0, 1, \cdots, p-1$. 
	It therefore follows that the average of the squares in $S$ is $\frac{p-1}{2}$, independent of $S$. 
\end{enumerate}

\section*{A-Level}
\begin{enumerate}
	\item[1.] On the plane, there is a triangle and ten lines. Every line is equidistant from two of the
	triangle’s vertices. Prove that either at least two of these lines are parallel or at least three
	of them pass through a common point.
	
	\textbf{Solution.}
	Let $A, B, C$ be the vertices of a triangle. We show that if no two lines are parallel and no three lines are concurrent then there can be at most 9 lines. 
	
	Consider any line $\ell$ equidistant to $A$ and $B$. 
	If $A$ and $B$ are on the same side of $AB$, then $AB\parallel \ell$. 
	Otherwise, $AB$ passes through midpoint of $AB$. 
	Thus we see that there's at most one line parallel to $AB$ and two lines through the midpoint of $AB$. 
	Doing the same analysis for $BC$ and $AC$, we get at most $3\times 3=9$ lines. 
	
	\item[2.] From given positive numbers, the following infinite sequence is defined: $a_1$ is the sum of all
	original numbers, $a_2$ is the sum of the squares of all original numbers, $a_3$ is the sum of the
	cubes of all original numbers, and so on ($a_k$ is the sum of the $k$-th powers of all original
	numbers).
	\begin{enumerate}
		\item Can it happen that $a_1 > a_2 > a_3 > a_4 > a_5$ and $a_5 < a_6 < a_7 < \ldots$?
		
		\item Can it happen that $a_1 < a_2 < a_3 < a_4 < a_5$ and $a_5 > a_6 > a_7 > \ldots$? 
	\end{enumerate}
    
    \textbf{Answer.} Yes for (a), no for (b). 
    
    \textbf{Solution.}
    For (a), 
    let the numbers given be $k$ copies of $2$ and $\ell$ copies of $0.5$. Then $a_n = k2^n+\ell\cdot \frac{1}{2^n}$ and therefore 
    \[
    a_{n+1}-a_n = k(2^{n+1}-2^n) + \ell(\frac{1}{2^{n+1}} - \frac{1}{2^{n}})
    =k\cdot 2^n - \ell\cdot \frac{1}{2^{n+1}}
    \]
    which is strictly increasing. 
    Now choose $k, \ell$ such that 
    \[
    2^4k-\frac{1}{2^5}\ell < 0 < 2^5k - \frac{1}{2^6}\ell
    \]
    or, in other words, $2^9<\frac{\ell}{k}<2^{11}$. We can, for example, take $k=1$ and $\ell=2^{10}$. 
    
    For (b), we see that the sequence $\{a_n\}_{n\ge 1}$ is bounded. 
    Given that $\{a^n\}_{n\ge 1}$ is unbounded for any given $a>1$, we have all numbers given as $\le 1$. 
    But then $\{a^n\}_{n\ge 1}$ is nonincreasing for any $a\le 1$, 
    so we cannot have $a_1>a_2$. 
    
    \item[5.]
    In a triangle $ABC$ with $\angle A = 45^{\circ}$, 
    $AM$ is a median. 
    Line $b$ is symmetrical to line $AM$
    with respect to altitude $BB_1$, and line $c$ is symmetrical to line $AM$ with respect to altitude
    $CC_1$. Lines $b$ and $c$ meet at point $X$. Prove that $AX = BC$.
    
    \textbf{Solution.}
    Let $H, O$ be the orthocenter and circumcenter of triangle $ABC$, respectively. 
    Then $HM$ and $AO$ intersect at a point $Z$ diametrically opposite $A$ on circumcircle of $ABC$, 
    with $M$ being midpoint of $ZH$ and $O$ midpoint of $AZ$, so $AH\parallel OM$, 
    with $AH=2OM$. The $45^{\circ}$ condition also implies 
    $OM=MA=\frac 12 AB$ so $AH=AB$. 
    We henceforth turn our attention to proving $AX=AH$. 
    If $AB=AC$ we're done since $X=H$ here so we may assume, say, $AB>AC$. 
    
    Next, $ABC$ and $AC_1B_1$ are similar. Thus $AM$ being the median w.r.t. triangle $ABC$ means $AM$ is a symmedian w.r.t. triangle $AB_1C_1$. 
    Additionally, $MB_1$ and $MC_1$ are the points of tangency to circle $AB_1C_1$. 
    
    Let $BB_1, CC_1$ intersect $AM$ at $E, F$, respectively. 
    Then by the reflection condition and that $\angle EHF=45^{\circ}$ (i.e. acute), 
    $H$ is the excenter of triangle $XEF$ and $\angle EXF=90^{\circ}$, 
    with $\angle EXH=\angle FXH=45^{\circ}$. 
    Let $Y$ be the intersection of $EF$ and $HX$, then we have the following equalities: 
    \[
    \frac{HE}{HF}\frac{\sin\angle EHX}{\sin\angle FHX}
    =\frac{EY}{FY}
    =\frac{XE}{XF}\frac{\sin\angle EXH}{\sin\angle FXH}
    \]
    \[
    \frac{HE}{HF}\frac{\sin\angle EHA}{\sin\angle FHA}
    =\frac{EA}{FA}
    =\frac{XE}{XF}\frac{\sin\angle EXA}{\sin\angle FXA}
    \]
    Thus combining these two gives 
    \[
    \frac{\sin\angle EXA}{\sin\angle FXA}
    =\frac{\sin\angle EHA}{\sin\angle FHA}\cdot\frac{HE}{HF}\cdot\frac{XF}{XE}
    =\frac{\sin\angle EHA}{\sin\angle FHA}\cdot 
    \frac{\sin\angle EXH}{\sin\angle FXH}
    \frac{\sin\angle FHX}{\sin\angle EHX}
    \]
    Notice that $AB_1HC_1$ is cyclic so $\angle EHA = \angle B_1HA = \angle B_1C_1A$ and similarly 
    $\angle FHA = \angle C_1B_1A$. 
    $\angle EXH=\angle FXH=45^{\circ}$, and finally 
    since $H$ is an excenter of $XEF$, $HX$ passes through the circumcenter of $HEF$ so 
    $\angle EHX = 90^{\circ}-\angle HFE=90^{\circ}-\angle C_1FA = \angle FAC_1=\angle MAC_1$, 
    since $\angle AC_1F=90^{\circ}$. 
    Similarly, $\angle FHX = \angle MAB_1$. 
    Thus putting this together, we get 
    \[
    \frac{\sin\angle EXA}{\sin\angle FXA}
    =\frac{\sin\angle B_1C_1A}{\sin\angle C_1B_1A}
    \frac{\sin\angle MAB_1}{\sin\angle MAC_1}
    =\frac{AB_1}{AC_1}\cdot \frac{AB_1}{AC_1}
    =\left(\frac{AB_1}{AC_1}\right)^2
    \]
    the first factor $\frac{AB_1}{AC_1}$ comes from sine rule on triangle $AB_1C_1$, 
    the second comes from that $AM$ is a symmedian of $AB_1C_1$. 
    
    Finally, $\angle FXA=\sin\angle EXA+90^{\circ}$ so $\frac{\sin\angle EXA}{\sin\angle FXA}=
    \frac{\sin\angle EXA}{\cos\angle EXA}=\tan\angle EXA$. 
    On the other hand, $AM$ being a symmedian also means it cuts $B_1C_1$ into the ratio $\left(\frac{AB_1}{AC_1}\right)^2$, 
    and with $MB_1=MC_1$, 
    $\left(\frac{AB_1}{AC_1}\right)^2=\frac{\sin\angle B_1MA}{\sin\angle C_1MA}=
    \frac{\sin\angle B_1MA}{\cos\angle B_1MA}=\tan\angle B_1MA$. 
    This gives 
    \[
    \angle EXA=\angle B_1MA
    \]
    To finish off, recalling the identity $\angle MAC_1=\angle EHX$ we have 
    \[
    \angle AXH = \angle AXE+\angle EXH
    =\angle B_1MA+\angle 45^{\circ}
    =\angle B_1MA+\angle MB_1C
    \]\[
    =\angle M_AC_1+\angle ACB_1
    =\angle EHX+\angle AHB_1
    =\angle AHX
    \]
    so $AH=AX$, as desired. 
    
    
    \item[6.]
    Find all positive integers $n$ such that for any integer $k \ge n$ there is a number divisible by
    $n$ and with the sum of digits equal to $k$.
    
    \textbf{Answer.} All integers not divisible by 3. 
    
    \textbf{Solution.} 
    We see that if $3\mid n$, then any number of sum of digits equal to $k=n+1$ will be congruent to 1 mod 3, and therefore impossible to be divisible by 3. 
    
    Now consider $n$ not divisible by 3. 
    Since adding trailing zeros to an integer $k$ will take care of the powers of 2 and 5 dividing $n$ without changing the sum of digits, 
    we may assume that $\gcd(n, 10)=1$. 
    Let's now claim the following: 
    
    \emph{Lemma.} For any $k\ge n$, there exist nonnegative integers $a$ and $b$ with $a+b=k$, and $n\mid a+10b$. 
    
    Proof: 
    consider the sequence $x_a = a + 10(k-a)$ for $a=0, 1, \cdots, n-1$. 
    Then for any $a\neq b$, we have $x_a-x_b = 9(b-a)$. Since $n$ is not divisible by 3, $\gcd(9, n)=1$ and since $0\le a, b\le n-1$, $n\nmid x_a-x_b$. 
    It follows that $x_0, \cdots, x_{n-1}$ are different modulo $n$ and since there are $n$ of them, 
    exactly one of them must be 0 mod $n$. $\square$
    
    Thus now we let $(a, b)$ satisfying the lemma condition. Let $\phi(n)$ be the number of integers in $[0, n-1]$ and relatively prime to $n$. 
    Then the number 
    \[
    \sum_{i=0}^{a-1} 10^{i\phi(n)}
    +\sum_{j=0}^{b-1} 10^{j\phi(n)+1}
    =
    1+10^{\phi(n)}+\cdots + 10^{(a-1)\phi(n)}
    +10+10^{\phi(n)+1}+\cdots + 10^{(b-1)\phi(n)+1}
    \]
    has sum of digits $k$ and remainder $a+10b\equiv 0$ modulo $n$. 

\end{enumerate}
\end{document}