\documentclass[11pt,a4paper]{article}
\usepackage{amsmath, amssymb, fullpage, mathrsfs, bm, pgf, tikz}
\usepackage{mathrsfs}
\usetikzlibrary{arrows}
\setlength{\textheight}{10in}
%\setlength{\topmargin}{0in}
\setlength{\topmargin}{-0.5in}
\setlength{\parskip}{0.1in}
\setlength{\parindent}{0in}

\begin{document}
	\newcommand{\la}{\leftarrow}
	\newcommand{\lra}{\leftrightarrow}
	\newcommand{\bbN}{\mathbb{N}}
	\newcommand{\bbZ}{\mathbb{Z}}
	\newcommand{\dsum}{\displaystyle\sum}
	\newcommand{\dprod}{\displaystyle\prod}
	
	
	\title{Solutions to Tournament of Towns, Spring 2022, Senior}
	\author{Anzo Teh}
	\date{}
	\maketitle
	
	\section*{A-Level}
	\begin{enumerate}
		\item[1.] 
		For each of the nine positive integers $n, 2n, 3n, \cdots, 9n$ Alice takes the first decimal digit
		(from the left) and writes it onto a blackboard. She selected $n$ so that among the nine
		digits on the blackboard there is the least possible number of different digits. What is this
		number of different digits equal to?
		
		\textbf{Answer.} 4. This is achieved by taking $n=29$, 
		and the different digits are $1, 2, 5, 8$. 
		
		\textbf{Solution} We now show that this cannot be improved. 
		Let's change the setting and consider $n$ not an integer, 
		but a rational number with $1\le n < 10$, instead (this can be done by dividing $n$ by $10^k$ for some integer $k$), and let $d=\lfloor n\rfloor$, the leading digit of $n$. 
		
		Let $m$ be the minimum integer such that $mn\ge 10$ (i.e. $2\le m\le 10$). Then: 
		\begin{itemize}
			\item $n, 2n, \cdots, (m-1)n$ all have different leading digits, all at least $d$. 
			
			\item $mn, \cdots, 9n, 10n$ are all $\ge 10$ and $<100$, so the leading digits differ by at most 1 at each step. Considering $(m-1)n$, too, means leading digit of $mn$ is 1. 
			Since the leading digit of $10n$ is $d$, it follows that we must have the leading digits of $mn, \cdots, 9n$ as $1, \cdots, d-1$, and maybe $d$ too. 
		\end{itemize}
	    Thus the number of digits is $d-1 + (m-1)=\lfloor n\rfloor + \lceil \frac{10}{n} \rceil - 2$, 
	    and in all cases, $\lceil \frac{10}{n} \rceil\ge 2$ since $n<10$. 
	    So if $\lfloor n\rfloor \ge 4$ the number of digits is $\ge 4$, 
	    and if $\lfloor n\rfloor =1$ then $\lceil \frac{10}{n} \rceil=6$, giving a total of $\ge 1+6-2=5$. 
	    Thus we only need to consider $d=2, 3$. 
	    
	    When $d=2$, $\lfloor n\rfloor + \lceil \frac{10}{n} \rceil \ge \lceil \frac{10}{3} \rceil =4$, 
	    when $d=3$, $\lfloor n\rfloor + \lceil \frac{10}{n} \rceil \ge \lceil \frac{10}{4} \rceil =3$, 
	    so in both cases the number of digits is $\ge 4$. 
		
		\item [2.]
		On a blank paper there were drawn two perpendicular axes $x$ and $y$ with the same scale.
		The graph of a function $y = f(x)$ was drawn in this coordinate system. Then the $y$ axis
		and all the scale marks on the $x$ axis were erased. Provide a way how to draw again the
		$y$ axis using pencil, ruler and compass if:
		\begin{enumerate}
			\item $f(x)=3^x$; 
			\item $f(x)=\log_a x$, where $a>1$ is an unknown number. 
		\end{enumerate}
	    
	    \textbf{Solution.} 
	    The main goal of both part is to draw a distance of exactly 1. 
	    For $f(x)=3^x$, this can be achieved via the following: 
	    \begin{itemize}
	    	\item choose any point on the graph, say, $A=(x_0, 3^{x_0})$, with $x_0$ unknown.  
	    	
	    	\item Draw the line perpendicular from $A$ to the $x$-axis, which intersects the $x$-axis at $B=(x_0, 0)$. 
	    	
	    	\item Let $C$ be on line $AB$ such that $\vec{BC}=3\vec{BA}$, we know $C=(x_0, 3^{x_0+1})$. 
	    	
	    	\item Let $D$ be the foot of perpendicular to $AB$ to $f(x)$, we have $D=(x_0+1, 3^{x_0+1})$. Hence we have $CD=1$. 
	    \end{itemize}
        Now that we know how to draw distance 1, we can draw the line $y=1$ by referencing to the $x$-axis. 
        Then the intersection of $y=1$ and $f(x)$ is $(0, 1)$, 
        and the $y$-axis can be taken as the perpendicular from $(0, 1)$ to the $x$-axis. 
        
        For $f(x)=\log_a x$, we may draw a line $\ell_1$ parallel to the $x$-axis and above the $x$-axis, and $\ell_2$ that's the $x$-axis reflected in $\ell_1$. 
        Now consider $A_1, A_2$ the intersections of $\ell_1$ and $\ell_2$ (respectively) with the $f(x)$, 
        and $B_1, B_2$ their projection to the $x$-axis. 
        We have $B_1=(x_0, 0)$ and $B_2=(x_0^2, 0)$ for some unknown $x_0 \ge 1$. 
        We also have the point $(1, 0)$, which is just the intersection of $x$-axis and $f(x)$. 
        
        \item [5.] 
        What is the maximal possible number of roots on the interval (0, 1) for a polynomial of
        degree 2022 with integer coefficients and with the leading coefficient equal to 1?
        
        \textbf{Answer.} 2021. 
        
        \item [6.]
        The king assembled 300 wizards and gave them the following challenge. For this challenge,
        25 colors can be used, and they are known to the wizards. Each of the wizards receives a
        hat of one of those 25 colors. If for each color the number of used hats would be written
        down then all these numbers would be different, and the wizards know this. Each wizard
        sees what hat was given to each other wizard but does not see his own hat. Simultaneously
        each wizard reports the color of his own hat. Is it possible for the wizards to coordinate
        their actions beforehand so that at least 150 of them would report correctly?
        
        \textbf{Answer.} Yes. 
        
        \textbf{Solution.} 
        If the frequencies of the colours used are $c_0, \cdots, c_{24}$ in some order, 
        then $\sum c_i = 300$. On the other hand, all $c_i$ are different, which gives another bound: 
        \[
        300= \sum c_i \ge 0 + 1 + \cdots + 24 = 300
        \]
        so $\{c_i\}=\{0, \cdots, 24\}$ in some order. 
        This means, when wearing the hats later, if the color worn by a wizard has frequency $i\ge 1$, 
        then the wizard will see the colour frequency as $0, 1, \cdots, i-1, i-1, i+1, \cdots, 24$, 
        i.e. they will only have to decide between the two. 
        
        Now, the wizards can do the following coordination: 
        label the colours as $0, \cdots, 24$, and the wizards themselves as $1, \cdots, 300$. 
        Let $c_i$ be the frequency of colour $i$, which is a permutation $\sigma$ of $0, \cdots, 24$, used later on. 
        Then for each wizard: 
        \begin{itemize}
        	\item The wizard knows all $c_i$ except for two values, $x$ and $y$. 
        	
        	\item The wizard also knows $\{c_x, c_y\}=\{j-1, j\}$, the hat they have is either $x$ or $y$, 
        	with frequency $j$. 
        	
        	\item One of the selection ($c_x=j$, or $c_y=j$) will yield $\sigma$ (and therefore correct hat). Since swapping any two elements in a permutation changes its parity, 
        	the other permutation has parity different from $\sigma$. 
        \end{itemize}
        Thus the wizard will choose the permutation with the same parity as their own label. 
        Consequently, regardless of $\sigma$, the wizards can always achieve exactly accuracy count of 150. 
        
	\end{enumerate}
\end{document}