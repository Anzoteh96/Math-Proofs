\documentclass[11pt,a4paper]{article}
\usepackage{amsmath, amssymb, fullpage, mathrsfs, bm, pgf, tikz}
\usepackage{mathrsfs, algorithm, algpseudocode}
\usetikzlibrary{arrows}
\setlength{\textheight}{10in}
%\setlength{\topmargin}{0in}
\setlength{\topmargin}{-0.5in}
\setlength{\parskip}{0.1in}
\setlength{\parindent}{0in}

\begin{document}
\newcommand{\la}{\leftarrow}
\newcommand{\lra}{\leftrightarrow}
\newcommand{\bbN}{\mathbb{N}}
\newcommand{\bbZ}{\mathbb{Z}}
\newcommand{\dsum}{\displaystyle\sum}
\newcommand{\dprod}{\displaystyle\prod}


\title{Solutions to Tournament of Towns, Spring 2016, Senior}
\author{Anzo Teh}
\date{}
\maketitle

\section*{O-Level}
\begin{enumerate}
	\item[1.]
	A point inside a convex quadrilateral is connected with all the vertices
	and with four more points such that each side contains one of them. The
	quadrilateral dissects into eight triangles with equal radii of circumcircles.
	Prove that the original quadrilateral is cyclic.
	
	\textbf{Solution}. 
	Let the point be $O$, the quadrilateral be $ABCD$, and the point on $AB$ be $E$. 
	We have, the cirumradii $R_1, R_2$ of $AEO$ and $BEO$ equal. 
	Now, $AO=2R_1\sin \angle OEA$ and $BO=2R_2\sin\angle OEB$. 
	But since $\angle OEA+\angle OEB=180^{\circ}$, 
	$\sin\angle OEA=\sin\angle OEB$. 
	Hence $R_1=R_2$ means $AO=BO$. 
	In a similar way we can show that $BO=CO$ and $CO=DO$ so $O$ is equidistant from the four vertices. 
	Therefore $ABCD$ is circumscribed in a circle with center $O$. 
	
	\item[2.] 
	 Is it possible that the sum and the product of 2016 integers are both equal
	to 2016?
	
	\textbf{Answer.} Yes: take 504 $\times -1$'s, 1510 $\times 1$'s, and 2 and 1008. 
	
	\item[5.]At each of the 12 edges of a cube, the midpoint is marked. Does a sphere
	necessarily contain all these points if it contains
	
	\begin{enumerate}
		\item at least 6 of marked points;
		\item at least 7 of marked points?
	\end{enumerate}
    
    \textbf{Answer.} No for (a) and yes for (b). 
    
    \textbf{Solution.} 
    W.l.o.g. let our cube to have coordinates $\{0, 1\}^3$. 
    Let the 6 points be 
    \[
    (0.5, 0, 0), (0, 0, 0.5), (0, 0.5, 1), (1, 0.5, 0), (1, 1, 0.5), (0.5, 1, 1)
    \]
    which are all on the plane $p$ given by the equation $x-y+z=0.5$. 
    We know that the center of the cube $O=(0.5, 0.5, 0.5)$ is equidistant from the all the 12 points, 
    so any point $O'$ lies on the line through $O$ perpendicular to plane $p$ is equidistant from the 6 said points. 
    However, any such $O'\neq O$ cannot be equidistant to all 12 points since $O$ is the unique point in the space satisfying this property. 
    
    Now let any sphere pass through any 7 of the points. 
    Then there's a face, say, $z=0$, that the sphere passes through at least three of the points. 
    Let $O$ be the projection of the center of the sphere to this face $z=0$, 
    then there's a circle passing through 3 of the points on this face, hence must pass through the 4th too. 
    We therefore know that our sphere passes through all 4 points on $z=0$. 
    
    Now this means the center of the sphere must be $(0.5, 0.5, z_0)$ for some $z_0$. 
    We see that $(0.5, 0.5, z_0)$ is equidistant from all 4 midpoints on the face $z=0$, 
    all 4 midpoints on the face $z=1$, and the 4 remaining points $\{0, 1\}^2\times \{0.5\}$. 
    This means if the sphere passes through any one of the 4 points in each category, 
    it passes through all 4 such points. 
    Since we also know our sphere passes through 7 such points, 
    two of such things must hold (i.e. the sphere passes through midpoints on $z=0$, by our assumption, and also midpoints on one of $z=1$ and $\{0, 1\}^2\times \{0.5\}$). 
    In any case, solving for them simultaneously would give $z_0=0.5$, 
    showing that our sphere does pass through all 12 points. 
\end{enumerate}
    

\section*{A-Level}
\begin{enumerate}
	\item[1.] 
	All integers from 1 to one million are written on a tape in some arbitrary order. Then the
	tape is cut into pieces containing two consecutive digits each. Prove that these pieces contain
	all two-digit integers for sure, regardless of the initial order of integers.
	
	\textbf{Solution.} Consider an arbitrary two-integer $10a+b$, 
	i.e. with decimal representation $\overline{ab}$. 
	Now consider the six-digit number $\overline{abaaba}$. 
	Depending on its position on the tape, it's being cut into two consecutive digits into one of the two following fashions: 
	\[
	(\overline{ab}, \overline{aa}, \overline{ba})
	\quad 
	(\overline{?a}, \overline{ba}, \overline{ab},  \overline{a?}, 
	)
	\]
	where ? denotes a digit before or after the digit from another number. 
	Thus both contain the string $\overline{ab}$. 
	
	\item[3.]
	 Let $M$ be the midpoint of the base $AC$ of an isosceles triangle $ABC$. Points $E$ and $F$ on
	the sides $AB$ and $BC$ respectively are chosen so that $AE \neq CF$ and $\angle FMC = \angle MEF = \alpha$.
	Determine $\angle AEM$.
	
	\textbf{Answer.} $\alpha$
	
	\textbf{Solution.} 
	The angle condition implies that the circle passing through $M, E, F$ is tangent to $AC$. 
	Thus, this circle is symmetric w.r.t. $BM$. 
	Let this circle intersect $AB$ at $E\neq E_1$, 
	so $\angle AEM=\angle E_1FM$. 
	
	Given the symmetry of the circle w.r.t. $BM$, the reflection of $F$ in $BM$ is also on this circle, and must be on $AB$. 
	By the condition $AE\neq CF$, this reflection is indeed $E_1$, so $CF=AE_1$. 
	Thus $ME_1=MF$ and so 
	\[
	\angle AEM=\angle E_1FM=\angle FE_1M=\angle FEM=\alpha
	\]
	
	\item[4.]
	There are 64 towns in a country, and some pairs of towns are connected by roads but we
	don’t know these pairs. We may choose any pair of towns and find out whether they are
	connected by a road. Our aim is to determine whether it is possible to travel between any
	two towns using roads. Prove that there is no algorithm which would enable us to do this
	in less than 2016 questions.
	
	\textbf{Solution.} Notice that $\binom{64}{2}=2016$, so we can transform the problem above into a more general one: 
	given a graph $G=(V, E)$, 
	and at each move we can ask if $(i, j)\in E$. 
	Our goal is to find out if $G$ is connected. 
	Prove that we need at least $\binom{|V|}{2}$ moves. 
	
	Suppose, on the contrary, that there exists such an algorithm. 
	Let the queries $q_k$ to query about $(i_k, j_k)$ for $k=1, \cdots, Q$ with $Q<\binom{|V|}{2}$. 
	(We may assume that $\{i_k, j_k\}\neq \{i_{\ell}, j_{\ell}\}$ 
	since repeated queries will only waste things). 
	Consider the following scenario: 
	call a partition $V$ into $V_1, V_2$ of nonnegative subsets $(i_0, j_0)$-\emph{risky} for some
	$i_0\in V_1$, $j_0\in V_2$
	 if  for any $i\in V_1, j\in V_2$ where $\{i, j\}\neq \{i_0, j_0\}$, 
	 $(i, j)$ is part of the previous queries, and answered ``no''. 
	 Then we at each step with query $(i_k, j_k)$, 
	 the answer is ``no'' is and only if there's no $(i_k, j_k)$-risky partition. 
	 In other words, the answer is ``no'' if and only if the knowledge of 
	$(i_1, j_1), \cdots, (i_k, j_k)$ (with the last one, $(i_k, j_k)\not\in E$) 
	is insufficient to determine that $G$ is disconnected. 
	
	We first show that this algorithm cannot tell for sure that $G$ is disconnected. 
	Otherwise, there's a partition of $V$ into two subsets $V_1$ and $V_2$ such that 
	$(i, j)\not\in E$ for each $i\in V_1, j\in V_2$. 
	This $(i, j)\not\in E$ must appear in the queries of the algorithm. 
	Let $(i_{\ell}, j_{\ell})$ be the last query among all $(i, j)$ with $i\in V_1, j\in V_2$. 
	Then as we query $(i_{\ell}, j_{\ell})$, the partition $V_1, V_2$ is $(i_{\ell}, j_{\ell})$-risky. 
	By the construction of our answer, $(i_{\ell}, j_{\ell})$ must return ``yes'', contradiction. 
	
	Now suppose that the algorithm can tell for sure that $G$ is connected. 
	Suppose, also, there's an edge $(i_0, j_0)$ that's never part of the query when our algorithm makes this verdict. 
	Consider the first moment when we can guarantee that $(i_0, j_0)$ must be from the same connected component: 
	this must be after receiving a ``yes'' from a query $(i_k, j_k)$ for some $k$, 
	so there's a partition $V_1, V_2$ that is $(i_k, j_k)$-risky. 
	Since $(i_0, j_0)$ is never asked, they must be in the same partition, $V_1$ or $V_2$. 
	W.l.o.g. let $i_0, j_0$ both in $V_1$. 
	Now consider any path connecting $i_0$ and $j_0$. 
	Since there's only one (new) edge between $V_1$ and $V_2$ and $i_0, j_0\in V_1$, 
	this path must use only vertices in $V_1$. 
	This means $i_0$ and $j_0$ are already connected before $(i_k, j_k)$ is asked, 
	again a contradiction. 
	
	\item[5.]
	On a blackboard, several polynomials of degree 37 are written, each of them has the leading
	coefficient equal to 1. Initially all coefficients of each polynomial are non-negative. By one
	move it is allowed to erase any pair of polynomials $f, g$ and replace it by another pair
	of polynomials $f_1, g_1$ of degree 37 with the leading coefficients equal to 1 such that either
	$f_1+g_1$ = $f+g$ or $f_1g_1 = fg$. Prove that it is impossible that after some move each polynomial
	on the blackboard has 37 distinct positive roots.
	
	\textbf{Solution.} 
	We claim the following invariant: 
	after the moves, the sum of  coefficients at $x^{36}$ of the polynomials remain the same. 
	Indeed, if 
	$f, g, f_1, g_1$ have $a_f, a_g, a_{f'}, a_{g'}$ as coefficient at $x^{36}$ then 
	$f+g$ and $f_1+g_1$ have $a_f+a_g$ and $a_{f'}, a_{g'}$ as coefficient at $x^{36}$, respectively. 
	So $f+g=f_1+g_1$ implies $a_f+a_g=a_{f'}+a_{g'}$. 
	In addition, $fg$ and $f_1g_1$ have $a_f+a_g$ and $a_{f'}, a_{g'}$ as coefficient at $x^{73}$, respectively. 
	So $fg=f_1g_1$ implies $a_f+a_g=a_{f'}+a_{g'}$. 
	Thus for both of the operations, 
	changing $(f, g)$ to $(f_1, g_1)$ does not change the sum of coefficient at $x^{36}$. 
	
	Now, the sum of coefficients at $x^{36}$ is nonnegative in the beginning. 
	If the polynomials each have 37 distinct positive roots, then each of them has negative coefficient at $x^{36}$: this coefficient is the negative of sum of all roots. 
	Thus in this case we end up with negative sum of coefficients at $x^{36}$, contradicting the invariant. 
\end{enumerate}
\end{document}