\documentclass[11pt,a4paper]{article}
\usepackage{amsmath, amssymb, fullpage, mathrsfs, bm, pgf, tikz}
\usepackage{mathrsfs}
\usetikzlibrary{arrows}
\setlength{\textheight}{10in}
%\setlength{\topmargin}{0in}
\setlength{\topmargin}{-0.5in}
\setlength{\parskip}{0.1in}
\setlength{\parindent}{0in}

\begin{document}
	\newcommand{\la}{\leftarrow}
	\newcommand{\lra}{\leftrightarrow}
	\newcommand{\bbN}{\mathbb{N}}
	\newcommand{\bbZ}{\mathbb{Z}}
	\newcommand{\dsum}{\displaystyle\sum}
	\newcommand{\dprod}{\displaystyle\prod}
	
	
	\title{Solutions to Tournament of Towns, Fall 2011, Senior}
	\author{Anzo Teh}
	\date{}
	\maketitle
	
	\section*{O-Level}
	\begin{enumerate}
		\item[1.] Several guests at a round table are eating from a basket containing $2011$ berries. Going in clockwise direction, each guest has eaten either twice as many berries as or six fewer berries than the next guest. Prove that not all the berries have been eaten.
		
		\textbf{Solution.} Suppose there's a guest who've eaten an odd number of berries. Then it cannot eat twice as many berries as the guest before it, so it must 6 fewer berries than the guest before, who therefore also eats and odd number of berries. Thus by iterating this around the table, each guest eats 6 fewer berries than the guest before. But this is impossible (by considering the guest who eats the most berries). Therefore each guest eats an even number of berries, and therefore the total number of berries eaten is also even, and hence cannot be 2011. 
		
		\item[3.] In a convex quadrilateral $ABCD, AB = 10, BC = 14, CD = 11$ and $DA = 5$. Determine the angle between its diagonals.
		
		\textbf{Answer.} $90^{\circ}$. 
		
		\textbf{Solution.} Let the diagonals meet at $P$ and denote $\theta = \angle PAB$. This gives $\angle PCD=\theta$ and $\angle PAD=\angle PBC=180^{\circ}-\theta$. Then: 
		\[
		AB^2 + CD^2 = (PA^2+PB^2-2PA\cdot PB\cos\theta) + (PC^2+PD^2-2PC\cdot PD \cos\theta) 
		\]\[
		= PA^2+PB^2+PC^2+PD^2 - 2\cos\theta (PA\cdot PB + PC\cdot PD)
		\]
		and similarly $AD^2+BC^2 = PA^2+PB^2+PC^2+PD^2 + 2\cos\theta (PA\cdot PD + PC\cdot PB)$. Here, $AD^2+BC^2=5^2+14^2=221=AB^2+CD^2$ so subtracting the two equations give 
		\[
		0=2\cos\theta (PA\cdot PD + PC\cdot PB+PA\cdot PB + PC\cdot PD)
		\]
		but since $PA, PB, PC, PD>0$ by the convexity of $ABCD$, $\cos\theta=0$ must hold. Therefore $\theta=90^{\circ}$. 
		
		\item[4.] Positive integers $a < b < c$ are such that $b + a$ is a multiple of $b - a$ and $c + b$ is a multiple of $c-b$. If $a$ is a $2011$-digit number and $b$ is a $2012$-digit number, exactly how many digits does $c$ have?
		
		\textbf{Answer.} 2012. 
		
		\textbf{Solution.} $c$ must have at least 2012 digits as per the inequality condition. Now if $(b+a)=k(b-a)$ for some integer $k$ then $\frac{b}{a}=\frac{k+1}{k-1}=1+\frac{2}{k-1}\le 1+2=3$ since $k-1\ge 1$. Therefore $b\le 3a$ and similarly $c\le 3b$. This gives $c\le 9a<10a$ and since $a$ has 2011 digits, $c$ cannot have more than 2012 digits. 
		
		\item[5.] In the plane are $10$ lines in general position, which means that no $2$ are parallel and no $3$ are concurrent. Where $2$ lines intersect, we measure the smaller of the two angles formed between them. What is the maximum value of the sum of the measures of these $45$ angles?
		
		\textbf{Answer.} $2250^{\circ}$. 
		
		\textbf{Solution.} Label the 10 angles in clockwise rotational order as $\ell_1, \cdots , \ell_{10}$. The angles $(\ell_i, \ell_{i+1})$ (indices taken modulo 10) form a rotational order of $180^{\circ}$. and therefore the sum of smaller angles cannot be more than $180^{\circ}$. In a similar way, using clockwise order, the sum of angles travelled between $(\ell_i, \ell_{i+k})$ for each $k\le 5$ and summing across all $i=1, 2, \cdots , 10$ is at most $k\cdot 180^{\circ}$. Therefore the for $k=1, 2, 3, 4$ the total angle is at most $(1+2+3+4)\cdot 180^{\circ}=1800^{\circ}$. Finally, the last five angles $(\ell_i, \ell_{i+5})$ for $i=1,2,3,4,5$ gives at most $90^{\circ}$ each, giving the bound of $2250^{\circ}$ in total. 
		
		This can be achived by, for example, making $\ell_i$ to form an angle of $i\cdot 18^{\circ}$ clockwise with the $x$-axis. 
	\end{enumerate}
	
	\section*{A-Level}
	\begin{enumerate}
		\item[2.] Given that $0 < a, b, c, d < 1$ and $abcd = (1 - a)(1 - b)(1 - c)(1 - d)$, prove that $(a + b + c + d) -(a + c)(b + d) \ge 1$. 
		
		\textbf{Solution.} Denote $u=a+c$ and $v=b+d$. Then we need to show that 
		\[
		(u-1)(v-1)=uv-(u+v)+1=(a+c)(b+d)-(a+b+c+d)+1\le 0
		\]
		Thus it suffices to show that we cannot have both $u, v>1$ and similarly we cannot have both $u, v<1$. If $u, v>1$, then $(1-a)(1-c)=ac+1-u<ac$ and similarly $(1-b)(1-d)<bd$. Therefore $(1-a)(1-b)(1-c)(1-d)<abcd$ now. Similarly $u, v<1$ gives $(1-a)(1-c)>ac$ and $(1-b)(1-d)<bd$ and therefore $(1-a)(1-b)(1-c)(1-d)>abcd$. These are contradictions. 
		
		\item[3.]  In triangle $ABC$, points $A_1,B_1,C_1$ are bases of altitudes from vertices $A,B,C$, and points $C_A,C_B$ are the projections of $C_1$ to $AC$ and $BC$ respectively. Prove that line $C_AC_B$ bisects the segments $C_1A_1$ and $C_1B_1$.
		
		\textbf{Solution.} Let $H$ be the orthocenter, $C_0$ the reflection of $H$ in $C_1$, and the projection of $C_0$ to $AC$ and $BC$ as $C^{0}_A, C^{0}_B$. It's well-known that $C_0$ is on the circumcircle of $ABC$, so line $C^{0}_AC^{0}_B$ is a Simpson line that passes through the projection of $C_0$ to $AB$, which is actually $C_1$. 
		
		Now, the projection of $H$ to $AC$ and $BC$ are $B_1$ and $A_1$, respectively. Given that $C_1$ is the midpoint of $C_0H$, $C_AC_B$ lies exactly halfway between the lines $C^{0}_AC^{0}_B$ and $C_1B_1$. (That is, all three lines are parallel and $C_AC_B$ is equidistant from $C^{0}_AC^{0}_B$ and $C_1B_1$). But given that $C^{0}_AC^{0}_B$ passes through $C_1$, we conclude that $C_AC_B$ bisects the segments $C_1A_1$ and $C_1B_1$.
		
		\item[5.] 
		We will call a positive integer good if all its digits are nonzero. A good integer will be
		called special if it has at least $k$ digits and their values strictly increase from left to
		right. Let a good integer be given. At each move, one may either add some special
		integer to its digital expression from the left or from the right, or insert a special integer
		between any two its digits, or remove a special number from its digital expression.What
		is the largest $k$ such that any good integer can be turned into any other good integer
		by such moves?
		
		\textbf{Answer.} 8. 
		
		\textbf{Solution.} 
		We cannot do 9 since the only special number is 123456789, 
		and since the sum of digits of this number is 45, any operation will not change the congruence modulo 9. 
		Therefore we cannot go from, say, 1 to 2. 
		
		Now we show that $k=8$ is doable by iteratively showing that for any digit $d$ among $1, \cdots, 9$, 
		we can add a single digit $d$, and delete a single digit $d$. 
		We'll proceed by induction: 
		base case where $d=1$, adding $123456789$ and deleting $23456789$ adds a 1. 
	    Adding 23456789 after a 1 and deleting this 123456789 string deletes a 1. 
	    
	    Now suppose that for some $d$ we can freely add and delete any single digit among 1, 2, $\cdots, d-1$ for some $d\ge 2$. 
	    To add $d$, add 123456789, delete $1, 2, \cdots, d-1$ to left with $d\cdots9$, 
	    insert $1, 2, \cdots, d-1$ after $d$ to get $d12\cdots(d-1)(d+1)\cdots 9$, 
	    and finally delete $12\cdots(d-1)(d+1)\cdots 9$. 
	    To delete $d$, add $12\cdots(d-1)(d+1)\cdots 9$ after $d$ to get $d12\cdots(d-1)(d+1)\cdots 9$, 
	    then delete $1, 2, \cdots, d-1$, and then add this to the front of $d$
	    and we get 1234$\cdots 9$, and delete this string. 
		
		\item[6.] Prove that the integer $1^1 + 3^3 + 5^5 + .. + (2^n - 1)^{2^n-1}$ is a multiple of $2^n$ but not a multiple of $2^{n+1}$.
		
		\textbf{Solution.} We'll go by induction on $n$. Base case $n=2$ yields the sum 28 which is a multiple of 4 but not 8. 
		
		For inductive step, we need the following intermediate steps: 
		
		Lemma 1: for all $k\ge 1$, and $a$ odd, we have $a^{2^k}\equiv 1\pmod{2^{k+2}}$. 
		
		Proof: base case we have $k=1$ and $1^2, 3^2, 5^2, 7^2\equiv 1\pmod{8}$. Inductive step: suppose $a^{2^k}\equiv 1\pmod{2^{k+2}}$, write $a^{2^k}=2^{k+2}c+1$. Then 
		\[
		a^{2^{k+1}}=(a^{2^k})^2\equiv (2^{k+2}c+1)^2 = 2^{2k+4}c^2+2^{k+3}c+1\equiv 1\pmod{2^{k+3}}
		\]
		as desired. 
		
		Lemma 2: for all $n\ge 2$, the following identities hold: 
		\[
		(2^{n+1} - 1)^{2^{n+1}-1} + 2^{n+1} - 3)^{2^{n+1}-3} + \cdots + (2^n + 1)^{2^n+1} - (1^1 + 3^3 + 5^5 + .. + (2^n - 1)^{2^n-1})\equiv 0\pmod{2^{n+2}}
		\]
		Proof: we play around with $(a+2^n)^{a+2^n}-a^a$ for each $1\le a\le 2^n-1$ and $a$ odd. We know from lemma 1 that $a^{2^n}\equiv 1\pmod{2^{n+2}}$ and therefore $(a+2^n)^{a+2^n}-a^a\equiv (a+2^n)^{a}-a^a$. Now 
		\[
		(a+2^n)^a-a^a=\dsum_{i=1}^a \dbinom{a}{i}(2^n)^ia^{a-i}\equiv a(2^n)a^{a-1}=a^a(2^n)\pmod{2^{n+2}}
		\]
		using the fact that $2n\ge n+2$ for $n\ge 2$ (and therefore all term with $i\ge 2$ above are divisible by $2^{n+2}$). Thus we have 
		\[
		(2^{n+1} - 1)^{2^{n+1}-1} + 2^{n+1} - 3)^{2^{n+1}-3} + \cdots + (2^n + 1)^{2^n+1} - (1^1 + 3^3 + 5^5 + .. + (2^n - 1)^{2^n-1})
		\]\[
		\equiv 2^{n}(1^1+3^3+\cdots + (2^n-1)^{(2^n-1)})\pmod{2^{n+2}}
		\]
		and by induction hypothesis, $1^1+3^3+\cdots + (2^n-1)^{(2^n-1)}$ is divisible by $2^n$ and therefore $2^{n}(1^1+3^3+\cdots + (2^n-1)^{(2^n-1)})$ is divisible by $2^{2n}$, hence divisible by $2^{n+2}$, as claimed. 
		
		Now we can complete the proof. Let the sum $1^1 + 3^3 + 5^5 + .. + (2^n - 1)^{2^n-1}=S_n$; we've shown that $2^{n+2}\mid S_{n+1}-2S_n$ so $S_{n+1}\equiv 2S_n\pmod{2^{n+2}}$. By induction hypothesis there's a number $a$ odd such that $S_n=a\cdot 2^{n}$ so $S_{n+1}\equiv a\cdot 2^{n+1}\pmod{2^{n+2}}$ is divisible by $2^{n+1}$ but not $2^{n+2}$, as desired. 
	\end{enumerate}
\end{document}