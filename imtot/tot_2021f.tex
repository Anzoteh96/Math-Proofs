\documentclass[11pt,a4paper]{article}
\usepackage{amsmath, amssymb, fullpage, mathrsfs, bm, pgf, tikz}
\usepackage{mathrsfs}
\usetikzlibrary{arrows}
\setlength{\textheight}{10in}
%\setlength{\topmargin}{0in}
\setlength{\topmargin}{-0.5in}
\setlength{\parskip}{0.1in}
\setlength{\parindent}{0in}

\begin{document}
\newcommand{\la}{\leftarrow}
\newcommand{\lra}{\leftrightarrow}
\newcommand{\bbN}{\mathbb{N}}
\newcommand{\bbZ}{\mathbb{Z}}
\newcommand{\dsum}{\displaystyle\sum}
\newcommand{\dprod}{\displaystyle\prod}


\title{Solutions to Tournament of Towns, Fall 2021, Senior}
\author{Anzo Teh}
\date{}
\maketitle

\section*{O-Level}
\begin{enumerate}
	\item
\end{enumerate}

\section*{A-Level}
\begin{enumerate}
	\item[1.]
	The wizards $A, B, C, D$ know that the integers 1, 2, . . . , 12 are written on 12 cards, one integer
	on each card, and that each wizard will get three cards and will see only his own cards. Having
	received the cards, the wizards made several statements in the following order.
	
	$A$: “One of my cards contains the number 8”.
	
	$B$: “All my numbers are prime”.
	
	$C$: “All my numbers are composite and they all have a common prime divisor”.
	
	$D$: “Now I know all the cards of each wizard”.
	
	What were the cards of $A$ if everyone was right?
	
	\textbf{Answer.} 1, 8, 9. 
	
	\textbf{Solution.}
	We first claim that $A$ must have all composite numbers or 1. 
	Notice that there are 5 primes 2, 3, 5, 7, 11. 
	If $A$ has at least one prime numbers, then $D$ will not be able to tell which of the prime numbers belong to $A$, and which one belong to $D$ based on information from $A, B, C$. 
	
	Thus $D$ will have two of the remaining prime numbers (after $B$). We show that the other cards by $D$ must either be 6 or 12. 
	Now, we see that the cards of $C$ could be one of the following: 
	\[
	(6, 9, 12), 
	(4, 6, 10), (4, 6, 12), 
	(6, 10, 12), 
	(4, 10, 12)
	\]
	Thus for $D$ to uniquely determine the cards of $C$, this non-prime card of $D$ must fall into all but one of the triples above, and hence can either be 6 or 12. 
	If it's 6, then $C$ has $4, 10, 12$; if it's 12, then $C$ has $4, 6, 10$. 
	Thus, the cards of $B, C, D$ are the 5 primes and $4, 6, 10, 12$, leaving $A$ with $1, 8, 9$. 
	
	\item[2.]
	 There was a rook at some square of a 10 $\times 10$ chessboard. At each turn it moved to a square
	adjacent by side. It visited each square exactly once. Prove that for each main diagonal (the
	diagonal between the corners of the board) the following statement is true: in the rook’s path
	there were two consecutive steps at which the rook first stepped away from the diagonal and
	then returned back to the diagonal.
	
	\item[3.]
	Grasshopper Gerald and his 2020 friends play leapfrog on a plane as follows. At each turn
	Gerald jumps over a friend so that his original point and his resulting point are symmetric
	with respect to this friend. Gerald wants to perform a series of jumps such that he jumps
	over each friend exactly once. Let us say that a point is achievable if Gerald can finish the
	2020th jump in it. What is the maximum number $N$ such that for some initial placement of
	the grasshoppers there are just $N$ achievable points?
	
	\textbf{Answer.} $\dbinom{2020}{1010}$. 
	
	\textbf{Solution.} We'll use the following identity: the reflection of $a$ in $b$ is $2b-a$. 
	Thus if $a_0$ is the original location of Gerald and 
	$a_1, \cdots, a_n$ are the locations of other $n=2020$ grasshoppers (which Gerald jumps over in that order) 
	then Gerald's final position is 
	\[
	a_0 + 2\left(\sum_{i=1}^{n/2}a_{2i} - \sum_{i=1}^{n/2}a_{2i-1}\right)
	\]
	(note the use that $n$ is even here). 
	Note that Gerald is allowed to permute $(a_1, \cdots, a_n)$, 
	and also this sum depends only on the $n/2$ points Gerald chooses to jump on odd (respectively even) time. 
	This means $N\le \dbinom{n}{n/2}$. 
	
	To show equality, we just need to show that it's possible to have $\sum_{i=1}^{n/2}a_{2i}\neq \sum_{i=1}^{n/2}b_{2i}$ whenever $\{a_{2i}\}$ and $\{b_{2i}\}$ are two different subsets of 
	$\{a_j\}_{j\le n}$. 
	This can be accomplished by, say, having the $n$ points to be $a_i = 2^i$. 
	
	
	\item[4.]
	What is the minimum $k$ for which among any three nonzero real numbers there are two
	numbers $a$ and $b$ such that either $|a-b|\le k$ or $|\frac{1}{a}-\frac{1}{b}|\le k$? 
	
	\textbf{Answer.} $\frac 32$. 
	
	\textbf{Solution.} 
	For any triple $(a_1, a_2, a_3)$ nonzero real numbers, denote 
	\[
	f(a_1, a_2, a_3) = \min_{i\neq j}\left\{|a_i-a_j|, |\frac{1}{a_i}-\frac{1}{a_j}|\right\}
	\]
	Then $f(-1, \frac 12, 2)=\frac 32$. 
	
	To show that $f\le \frac 32$ in all cases, 
	consider $a_1<a_2<a_3$. 
	Consider the following cases: 
	
	\emph{Case 0.} $\min \{a_2-a_1, a_3-a_2\}\le \frac 32$. This is obvious. 
	So for each of the following cases, assume $\min \{a_2-a_1, a_3-a_2\}\ge \frac 32$. 
	
	\emph{Case 1a}. $a_1> 0$. 
	Then $a_2\ge \frac 32$ and 
	\[
	\frac{1}{a_2}-\frac{1}{a_3}< \frac{1}{a_2} \le \frac{2}{3}<\frac 32
	\]
	
	\emph{Case 1b}. $a_3< 0$. 
	This case is symmetric to 1a. 
	
	\emph{case 2a}. $a_1<0<a_2$. 
	Let $x=-a_1$. 
	Then we have $a_2\ge a_1+\frac 32 = \frac 32 - x$, 
	and $\frac{1}{a_3}\ge \frac 32 +\frac{1}{a_1}=\frac 32-\frac 1x$, i.e. 
	$a_3\le \frac{1}{\frac 32-\frac 1x}$. 
	Now consider the function 
	\[
	\frac{1}{\frac 32-\frac 1x} - (\frac 32 - x)
	\]
	which has desirvative $1-\frac{1}{(\frac 32x-1)^2}$, hence decreasing for $x\ge 1$. 
	We thus see that if $a_1\le -1$, 
	\[
	a_3-a_2\le \frac{1}{\frac 32-\frac 1x} - (\frac 32 - x)\le 
	\frac{1}{\frac 32-1} - (\frac 32 - 1) = \frac 32
	\] 
	Similarly, consider the function 
	\[
	\frac{1}{\frac 32 - x} - \frac{1}{\frac 32- \frac{1}{x}}
	\]
	similar to before cases, when $-1\le a_1<0$, we have $0\le x\le 1$
	\[
	\frac{1}{a_2}-\frac{1}{a_3}
	\le \frac{1}{\frac 32 - x} - \frac{1}{\frac 32- \frac{1}{x}}
	\le \frac{1}{\frac 32-1} - (\frac 32 - 1) = \frac 32
	\]
	Thus $f(a_1, a_2, a_3)\le \frac 32$. 
	
	\emph{case 2b}. $a_2<0<a_3$. It's symmetric to 2a. 
	
	\item[5.]
	Let $ABCD$ be a parallelogram and let $P$ be a point inside it such that $\angle P DA = \angle P BA$.
	Let $\omega_1$ be the excircle of $P AB$ opposite to the vertex $A$. Let $\omega_2$ be the incircle of the
	triangle $PCD$. Prove that one of the common tangents of $\omega_1$ and $\omega_2$ is parallel to $AD$.
	
	\textbf{Solution.} 
	Let $E$ be such that $AEPD$ is a parallelogram. 
	Then $\angle AEP=\angle ADP=\angle ABP$, and $B$ and $E$ are on the same side of $AP$. 
	Therefore $A, E, B, P$ are concyclic. 
	In addition, $PE\parallel AD\parallel BC$ and $PE=AD=BC$ (length) so $PEBC$ is also a parallelogram. 
	Thus $AEB$ and $DPC$ are congruent triangles, 
	and are in fact translation with axis parallel to $AD$. 
	Therefore, if we denote $\omega_3$ as the incircle of $AEB$ then it suffices to show that $\omega_1$ and $\omega_3$ have common tangent parallel to $PE$ (given also that $PE\parallel AD$). 
	
	Now let the centers of $\omega_1$ and $\omega_3$ be $I_1$ and $I_3$, respectively. 
	We first have $\angle I_1AI_3=\frac{\angle EAP}{2}$ 
	since $I_1$ is the incenter of $AEB$ and $I_3$ the $A$-excenter of $ABP$. 
	Also $\angle I_1BI_3=90^{\circ}+\frac{\angle EBP}{2}$. 
	Thus $\angle I_1AI_3+\angle I_1BI_3=180^{\circ}$, 
	since $\angle EAP+\angle EBP=180^{\circ}$, 
	($A, E, B, P$ concyclic). 
	So $I_1BPI_3$ also cyclic. This means: 
	\[
	\angle(AB, EP)
	=\angle(AB, AE)+\angle(AE, EP)
	=\angle(AB, AE)+\angle(AB, BP)
	=2\angle(AB, AI_3)+2\angle(AB, BI_1)
	\]\[
	=2\angle(AB, AI_3)+2\angle(AI_3, I_1I_3)
	=2\angle(AB, I_1I_3)
	\]
	and since $AB$ is a common tangent to both $\omega_1, \omega_3$, it follows that so is $AB$'s reflection in $I_1I_3$, namely $\ell$. 
	With $2\angle(AB, I_1I_3)=\angle(AB, EP)=\angle(AB, \ell)$ we have $\ell\parallel EP$, as desired. 
	
\end{enumerate}
\end{document}