\documentclass[11pt,a4paper]{article}
\usepackage{amsmath, amssymb, fullpage, mathrsfs, bm, pgf, tikz}
\usepackage{mathrsfs}
\usetikzlibrary{arrows}
\setlength{\textheight}{10in}
%\setlength{\topmargin}{0in}
\setlength{\topmargin}{-0.5in}
\setlength{\parskip}{0.1in}
\setlength{\parindent}{0in}

\begin{document}
\newcommand{\la}{\leftarrow}
\newcommand{\lra}{\leftrightarrow}
\newcommand{\bbN}{\mathbb{N}}
\newcommand{\bbZ}{\mathbb{Z}}
\newcommand{\dsum}{\displaystyle\sum}
\newcommand{\dprod}{\displaystyle\prod}


\title{Solutions to Tournament of Towns, Spring 2014, Senior}
\author{Anzo Teh}
\date{}
\maketitle

\section*{O-Level}
\begin{enumerate}
	\item[1.] 
	Inspector Gadget has 36 stones with masses 1 gram, 2 grams, . . . , 36 grams.
	Doctor Claw has a superglue such that one drop of it glues two stones together
	(thus two drops glue 3 stones together and so on). Doctor Claw wants to glue
	some stones so that in obtained set Inspector Gadget cannot choose one or more
	stones with the total mass 37 grams. Find the least number of drops needed
	for Doctor Claw to fulfil his task.
	
	\textbf{Answer.} 9. 
	
	\textbf{Solution.} 
	The example of 9 is obtained simplyby gluing $(1, 18), (2, 17), \cdots, (9, 10)$. 
	This way, we have the stones having weights 19, $\cdots, 19$, 20, $\cdots, 36$. 
	Any one stone has weight at most 36, but any two or more stones have weight at least 38 in total. 
	
	To show 9 is necessary, consider the 18 pairs $(1, 36), (2, 35), \cdots, (18, 19)$. 
	Since one drop of glue can take one or more stones from at most two of the pairs, 
	at least one pair $(k, 37-k)$ is untouched after 8 (or fewer) drops of glue. 
	
	\item [2.] In a convex quadrilateral $ABCD$ the diagonals are perpendicular. Points $M$
	and $N$ are marked on sides $AD$ and $CD$ respectively. Prove that lines $AC$ and
	$MN$ are parallel given that angles $ABN$ and $CBM$ are right angles.
	
	\textbf{Solution.} 
	Let $BM$ and $BN$ intersect $AC$ at $M_1$ and $N_1$, respectively, 
	and $O$ be the intersection of diagonals $BD$ and $AC$. 
	Then $\angle CBM_1=\angle BOC=90^{\circ}$ means $BO^2=M_1O\cdot OC$ and similarly 
	$CO^2=AO\cdot N_1O$. Thus $\frac{AO}{OC}=\frac{M_1O}{N_1O}$. 
	
	Now consider the line from $M$ parallel to $AC$. 
	Let this line intersect $CD$, $BN_1$ and $BD$ at $N_2$, $N_3$ and $O_1$, respectively.  
	Then we have $\frac{MO_1}{O_1N_2}=\frac{AO}{OD}$, 
	and also 
	$\frac{MO_1}{O_1N_3}=\frac{M_1O}{N_1O}$. 
	But then we have $\frac{AO}{OC}=\frac{M_1O}{N_1O}$ 
	so we have $O_1N_2=O_1N_3$, and thus $N_2=N_3$. 
	It follows that $N_2$ and $N_3$ are both equal to $N$, so $MN\parallel AC$. 
	
	\item [3.] 
	Ali Baba and the 40 thieves want to cross Bosporus strait. They made a line
	so that any two people standing next to each other are friends. Ali Baba is the
	first; he is also a friend with the thief next to his neighbour. There is a single
	boat that can carry 2 or 3 people and these people must be friends. Can Ali
	Baba and the 40 thieves always cross the strait if a single person cannot sail?
	
	\textbf{Answer.} Yes. 
	
	\textbf{Solution.} Denote Ali Baba as index 0 and the thieves (in that order) as $1, 2, \cdots, n$ (here $n=40$). 
	Let's claim the following: 
	
	\emph{Lemma}. Suppose that at given $k\ge 2$, we have $0, \cdots, k$ successfully crossed the strait (with boat on their side) and $k+1, \cdots, n$ still haven't. 
	Then there exists an algorithm such that in the end, the only thing that changed is that $k+1$ is on the other side of the river. 
	
	Proof: ignore $k+2, \cdots, n$ for now. 
	Now consider the following: 
	\begin{itemize}
		\item Among $0, 1, \cdots, k+1$, let $x\ge 3$ be the only guy who haven't crossed the river. 
		Bring $x-1, x-2$ back home, and $x-1, x$ over. Now $x-2$ is the only one who haven't crossed the river. 
		
		\item Repeat until $x\le 2$. 
	\end{itemize}
    Since we started with $x=k+1$, we eventually reach $x=1$ or $x=2$. 
    In either case, we can either bring $0, 2$ back or $0, 1$ back, and then all three will sail to the other side. 
    This way we have 0, 1, $\cdots, k+1$ on the other side. $\square$. 
    
    Now with this lemma, observe that we can have $0, 1, 2$ to the other side of the river on the first step (hence $k=2$). Applying the lemma iteratively, we see that it will also work for $k=3, 4, \cdots, n$. 
    
    \item [4.] 
    Positive integers $a, b, c, d$ are pairwise coprime and satisfy the equation
    $ab + cd = ac - 10bd$.
    Prove that one can always choose three numbers among them such that one
    number equals the sum of two others.
    
    \textbf{Solution.} Rearranging, we have the following: 
    \[
    d(c+10b) = a(c-b)
    \qquad 
    b(a+10d) = c(a-d)
    \]
    Given $\gcd(a, d)=1$ and $\gcd(b, c)=1$, we have 
    \[
    a\mid c+10b\quad d\mid c-b\quad  b\mid a-d\quad c\mid a+10d
    \]
    Also $c-b>0$ and $a-d>0$ since $a, b, c, d$ are positive. 
    Therefore, in particular, there exists a positive integer $k$ with $c+10b=ka$ and $c-b=kd$, 
    which means $k$ divides both $c+10b$ and $c-b$. 
    This gives $k\mid c+10b-(c-b)=11b$, and $k\mid c+10b+10(c-b)=11c$. 
    But given $\gcd(b, c)=1$, we have either $k=1$ or $k=11$ (since 11 is prime). 
    If $k=1$ then $c=b+d$. 
    If $k=11$, then 
    $11a - 11d = (c+10b)-(c-b)=11b$, so $a-d=b$. 
    This gives $a=b+d$. 
	
	\item[5.] Park’s paths go along sides and diagonals of the convex quadrilateral $ABCD$.
	Alex starts at A and hikes along $AB-BC -CD$. 
	Ben hikes along $AC$; 
	he leaves $A$ simultaneously with Alex and arrives to $C$ simultaneously with Alex. 
	Chris hikes along $BD$; he leaves B at the same time as Alex passes $B$ and arrives to
	$D$ simultaneously with Alex. 
	Can it happen that Ben and Chris arrive at point $O$ of intersection of $AC$ and $BD$ at the same time? The speeds of the hikers are constant.
	
	\textbf{Answer.} No. 
	
	\textbf{Solution.} Let $B_1$ be on $AC$ such that Alex is at $B$ when Ben at $B_1$. 
	Then $\frac{AB}{BC} = \frac{AB_1}{B_1C}$. 
	Thus $BB_1$ bisects $\angle ABC$ by angle bisector theorem. 
	Similarly, let $C_1$ be on $BD$ such that Alex is at $C$ when Chris at $C_1$. 
	Then we also have $CC_1$ bisects angle $\angle BCD$. 
	
	Now as Alex moves from $B$ to $C$, 
	Ben moves from $B_1$ to $C$ and Chris from $B$ to $C_1$. 
	Thus $\frac{B_1O}{OC} = \frac{BO}{OC_1}$. 
	It follows that $BB_1\parallel CC_1$, 
	and therefore $\angle B_1BC+\angle BCC_1=180^{\circ}$ and 
	since $BB_1$ and $CC_1$ bisect $\angle ABC$ and $\angle BCD$, 
	$\angle ABC+\angle BCD=360^{\circ}$, 
	which is impossible since this implies $\angle BAD=\angle BCD=0$. 
\end{enumerate}

\section*{A-Level}
\begin{enumerate}
	\item[7.] 
	Consider a polynomial $P(x)$ such that
	$P(0) = 1; (P(x))^2 = 1 + x + x^{100}Q(x)$, 
	where $Q(x)$ is also a polynomial.
	Prove that in the polynomial $(P(x) + 1)^{100}$ the coefficient at $x^{99}$ is zero.
	
	\textbf{Solution.} 
	Throughout the solution we use the generalized binomial: 
	\[
	\dbinom{x}{k} = \frac{1}{k!} x(x-1)\cdots (x-k+1)
	\]
	for real $x$ and negative integer $k$. 
	We follow the following steps: 
	
	\emph{Lemma 1.} 
	Let $N\ge 2$. 
	In the expression $(1+\sqrt{1+x})^{2N}$, treated as power series, 
	the coefficient is 0. 
	
	Proof: Expanding the outer binomial gives 
	\[
	\sum_{k=0}^{2N}\binom{2N}{k}\sqrt{1+x}^k
	\]
	and therefore we're interested in the sum 
	\[
	\sum_{k=0}^{2N}\binom{2N}{k}\binom{\frac k2}{2N-1}
	\]
	Let's claim that 
	\[
	\sum_{k=0}^{N-1}\binom{2N}{2k+1}\binom{x}{2N-1}
	\equiv \binom{2x}{2N-1}
	\]
	(TODO)
	
	\emph{Lemma 2.} 
	Let $P(x)$ be a polynomial be such that $P(0)=1$ and $P(x)^2 \equiv 1+x\pmod{x^{2N}}$. 
	Then the coefficient of $x^k$ for $P(x)^{n}$ is $\dbinom{\frac n2}{k}$ for $k=0, 1, \cdots, 2N-1$. 
	
	Proof: Let the coefficient of $x^k$ of $P(x)^{n}$ be $a_k$. We have $a_0=P(0)^n=1$. 
	Matching coefficients for $P(x)^{2n}$, which is $\equiv (1+x)^n\equiv{2N}$, we have 
	\[
	\sum_{i=0}^k a_{i}a_{k-i} = 
	\dbinom{n}{k}
	\]
	Recursively, we have 
	\[
	a_k = \left(\frac 12\binom{n}{k}-\left(\sum_{i=1}^{k-1}a_ia_{k-i}\right)\right)
	\]
	If we consider $\sqrt{1+x}^{k}=b_0+b_1x+b_2x^2+\cdots$ then we also have $b_0=1$ and 
	\[
	b_k = \left(\frac 12\binom{n}{k}-\left(\sum_{i=1}^{k-1}b_ib_{k-i}\right)\right)
	\]
	so we need $a_k=b_k$ for all $k$ for $k=0, \cdots, 2N-1$. 
	Now the conclusion since follows from 
	\[
	b_k=\dbinom{\frac n2}{k}
	\]
	
	Having this, we have 
	\[
	(P(x)+1)^{2N}
	=\sum_{n=0}^{2N}\binom{2N}{n}P(x)^n
	\equiv \sum_{n=0}^{2N}\binom{2N}{n}\sqrt{1+x}^n
	\equiv(1+\sqrt{1+x})^{2N}
	\pmod{x^{2N}}
	\]
	and the conclusion then follows from Lemma 1 where we take $N=50$. 
	
\end{enumerate}
\end{document}