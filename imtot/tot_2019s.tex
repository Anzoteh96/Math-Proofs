\documentclass[11pt,a4paper]{article}
\usepackage{amsmath, amssymb, fullpage, mathrsfs, bm, pgf, tikz, float}
\usepackage{mathrsfs}
\usetikzlibrary{arrows}
\setlength{\textheight}{10in}
%\setlength{\topmargin}{0in}
\setlength{\topmargin}{-0.5in}
\setlength{\parskip}{0.1in}
\setlength{\parindent}{0in}

\begin{document}
\newcommand{\la}{\leftarrow}
\newcommand{\lra}{\leftrightarrow}
\newcommand{\bbN}{\mathbb{N}}
\newcommand{\bbZ}{\mathbb{Z}}
\newcommand{\dsum}{\displaystyle\sum}
\newcommand{\dprod}{\displaystyle\prod}


\title{Solutions to Tournament of Towns, Spring 2019, Senior}
\author{Anzo Teh}
\date{}
\maketitle

\section*{O-Level}
\begin{enumerate}
	\item
\end{enumerate}

\section*{A-Level}
\begin{enumerate}
	\item[1.]
	 Some positive integer, divisible by 7, is shown on a computer screen. The cursor marks
	a gap between some two of its consecutive digits. Prove that there is a digit that can
	be inserted into the marked gap any number of times so that the resulting number is
	always divisible by 7.
	
	\textbf{Solution.} 
	Let $k$ be the position of split (0-indexed, counting from right), 
	and $(a, b)$ be the digits of split. 
	Then $7\mid 10^ka+b$. 
	If we insert digit $c$ for $\ell$ times then we have $10^{k+\ell}a+10^k\cdot c\cdot \frac{10^{\ell}-1}{9}+b$ as the final number. 
	The difference of the two numbers are now 
	\[
	10^ka(10^{\ell}-1)+10^k\cdot c\cdot \frac{10^{\ell}-1}{9}
	=\frac{10^{\ell}-1}{9}\cdot 10^k\cdot (9a+c)
	\]
	Thus, we just need to pick the unique $c\in \{0, 1, 2, 3, 4, 5, 6\}$ such that $7\mid 9a+c$. 
	
	\item[2.]
	$2019$ point grasshoppers sit on a line. 
	At each move one of the grasshoppers jumps over another one and lands at the point the same distance away from it. 
	Jumping only to the right, the grasshoppers are able to position themselves so that some two of them are exactly $1$ mm apart. 
	Prove that the grasshoppers can achieve the same, jumping only to the left and starting from the initial position.
	
	\textbf{Solution.} 
	Here, we see that only the relative (not absolute) positions of the grasshoppers matter. 
	Let $a_0$ be the leftmost grasshopper in the original position. 
	Thus from the initial position, let's have all the grasshoppers except $a_0$ jump to the left with respect to the grasshopper $a_0$, 
	this position is now the mirror image of the initial position. 
	
	Thus if from the original position there's an algorithm of ``jumping only to the right'', 
	we can mimic the algorithm in a reverse manner (from the mirrored image, and then to the left). 
	
	\item[3.]
	Two not necessarily equal non-intersecting wooden disks, one gray and one black, are glued to a plane. 
	An infinite angle with one gray side and one black side can be moved along the plane so that the disks remain outside the angle, 
	while the colored sides of the angle are tangent to the disks of the same color (the tangency points are not the vertices). 
	Prove that it is possible to draw a ray in the angle, 
	starting from the vertex of the angle and such that no matter how the angle is positioned, 
	the ray passes through some fixed point of the plane.
	
	\item[4.]
	One needs to fill the cells of an $n\times n$ table ($n > 1$) with distinct integers from $1$ to $n^2$ so that every two consecutive integers are placed in cells that share a side, 
	while every two integers with the same remainder if divided by $n$ are placed in distinct rows and distinct columns. 
	For which $n$ is this possible?
	
	\textbf{Answer.} All $n$ even. 
	
	\textbf{Solution.} 
	Colour the cells in checkerboard fashion with top left corner black. 
	Since consecutive integers must be on adjacent cells (which have different colours), 
	we see that all odd integers must be of the same colour, and even integers of same (and opposite) colour. 
	Consider, now, $n$ odd. 
	There is one more black square than white, and one more odd integers than even, 
	so all odd integers are in black square and even in white. 
	
	Break the integers into classes $S_1, \cdots, S_n$ where $S_i$ has $i$ integers congruent to $i$ mod $n$. 
	If we label the positions of grids as $a_{ij}$ with $1\le i, j\le n$ then for each $k$ there's a permutation $\sigma_k$ of $1, \cdots, n$ such that numbers in $S_k$ have coordinates $a_{i, \sigma_k(i)}, i=1, \cdots, n$. 
	We see that $a_{i, j}$ is black if $i+j$ is even and white if $i+j$ is odd (since $a_{1, 1}$, the top left corner, is black). 
	Then when we sum up all the $(x, y)$, coordinates  we get $\sum i = \sum_i \sigma_k(i) = \frac{n(n+1)}{2}$, which add up to $n(n+1)$ which is even. 
	This means, there are evenly many numbers in $S_k$ in white squares (and oddly many in black squares). 
	On the other hand, consider: 
	\[
	S_1 = \{1, n+1, \cdots, n^2-n+1\}
	\equiv \{1, 0, 1, 0, \cdots, 1\}\pmod{2}
	\]\[
	S_2 = \{2, n+2, \cdots, n^2-n+2\}
	\equiv \{0, 1, 0, 1, \cdots, 0\}
	\]
	so $S_1$ has $\frac{n+1}{2}$ has black squares while $S_2$, $\frac{n-1}{2}$. 
	But $\frac{n+1}{2}$ and $\frac{n-1}{2}$ cannot be simultaneously odd. 
	
	Now consider $n$ even. Let's claim that the following construction works: 
	\[1, 2, \cdots, n
	\]
	\[n^2, n^2-1, \cdots, n^2-n+2, n+1
	\]
	\[n^2-2n+3 ,n^2-n+1,n+2
	\]
	\[
	\vdots
	\]
	\[
	3n-1, 3n, \cdots,  4n-3, 2n-2
	\]
	\[
	3n-2, 3n-3, \cdots, 2n+1, 2n, 2n-1
	\]
	Pictorially, it means go from top left corner (1), go to the top right corner ($n$), then to bottom right ($2n-1$), 
	and then from there traverse the remaining $(n-1)^2$ squares in a horizontal zigzag manner. 
	Here's an example for $n=4$: 
	
	\begin{table}[H]
		\centering 
		\begin{tabular}{|c|c|c|c|}
			\hline
			1 & 2 & 3 & 4\\
			\hline
			16 & 15 & 14 & 5\\
			\hline
			11 & 12 & 13 & 6\\
			\hline
			10 & 9 & 8 & 7\\
			\hline
		\end{tabular}
		\caption{Example for $n=4$}
		\label{tab:q4}
	\end{table}
    To see why the modulo condition is fulfilled, in terms of mod $n$ the first row is 1 to $n$, 
    and the last $k$ rows comprise $kn$ consecutive integers for each $k=1, \cdots, n-1$ so each of those rows must also be $1, 2, \cdots, n$ modulo $n$ in some order. 
    For column check, consider $a_{i, k}$ and $a_{j, k}$ on the same column, with the following cases: 
    \begin{itemize}
    	\item if $i\neq j\pmod{2}$ then $a_{i, k}$ and $a_{j, k}$ are of different colour. But then with $n$ even all numbers in the same class $S_{\ell}$ must be of same parity. 
    	
    	\item If $2\le i\neq j\le n$ then $a_{i, k}$ and $a_{j, k}$ are part of the zigzag pattern on same direction for $k\le n-1$, or equal to $i+n-1$ vs $j+n-1$ when $k=n$. 
    	In either case,  $a_{i, k}-a_{j, k}\equiv\pm (i-j)\not\equiv 0\pmod{n}$. 
    	
    	\item If one of $i, j$, say $i$, is 1, then $a_{1, k}=k$. But from the construction we can show that when $j$ odd, 
    	$a_{j, k}\equiv j+k\pmod{n}$. 
    \end{itemize}
	
	\item[5.]
	The orthogonal projection of a tetrahedron onto a plane containing one of its faces is a trapezoid of area $1$, which has only one pair of parallel sides.
	\begin{enumerate}
		\item Is it possible that the orthogonal projection of this tetrahedron onto a plane containing another its face is a square of area $1$?
		\item The same question for a square of area $1/2019$.
	\end{enumerate}
    
    \textbf{Answer.} No for (a) and yes for (b). 
    
    \textbf{Solution.} 
    We consider the problem in reverse: 
    consider a tetrahedron where the orthogonal projection onto a plane containing a face is 
    square of side length $a$. 
    Is it possible that the orthogonal projection onto a plane containing another face is a non-paralellogram trapezoid of area $c$? 
    
    Now consider the face $ABC$ which gives the square orthogonal projection, 
    this face must be an isoceles right triangle with side lengths $a, a, \sqrt{2}a$, all positive. 
    W.l.o.g. the face (and plane) be the $xy$-plane, 
    and $A=(a, 0, 0), B=(0, 0, 0), C=(0, a, 0)$. 
    Then there's a nonzero real number $b$ such that the last vertex is $D=(a, a, b)$.
    
    We first show that $ACD$ cannot be the face we're looking for. 
    We have $AB=BC$ so the projection $B'$ from $B$ to the plane containing $ACD$ also has $AB'=B'C$. 
    But we also have $AD=CD$, 
    so line $B'D$ is the perpendicular bisector of $AC$ (i.e. $AB'CD$ is a kite). 
    Thus, if $AB'CD$ is a trapezoid then it has to be a rhombus, which isn't allowed. 
    
    Now consider face $BCD$. 
    Observe that $\angle BCD=90^{\circ}$. 
    Denote $\theta\neq 0$ as the angle between the planes $ABD$ and $ABC$. 
    Then $\tan\theta = \frac{b}{a}$. 
    Let $A'$ be the projection from $A$ to this plane. 
    Then since the plane $BCD$ contains the line $BC$ (i.e. the $y$-axis), 
    $A$ and $A'$ must have the same $y$-coordinate, i.e. 0. 
    Thus $\angle A'BC=90^{\circ}$ too and so $A'B\parallel CD$. 
    We also have $\angle A'BA=\theta$ so $A'B = |\cos\theta| = \frac{a^2}{\sqrt{a^2+b^2}}$, 
    and recall that $CD=\sqrt{a^2+b^2}$. 
    Thus the area of the trapezoid is now 
    \[
    \frac 12 \cdot a(\frac{a^2}{\sqrt{a^2+b^2}} + \sqrt{a^2+b^2})
    \]
    and since $\sqrt{a^2+b^2}>\frac{a^2}{\sqrt{a^2+b^2}}$ whenever $b\neq 0$, 
    this trapezoid only has one pair of parallel sides (hence fulfiling the condition). 
    Finally, by AM-GM we have $\frac 12 (\frac{a^2}{\sqrt{a^2+b^2}} + \sqrt{a^2+b^2})\ge a$, 
    but equality cannot hold given that $\sqrt{a^2+b^2}>\frac{a^2}{\sqrt{a^2+b^2}}$. 
    But as we vary $b$ continuously in $b>0$: 
    \begin{itemize}
    	\item as $b\to 0^+$ we have $\frac{a^2}{\sqrt{a^2+b^2}} + \sqrt{a^2+b^2}\to 2a$
    	\item as $b\to\infty$ we have $\frac{a^2}{\sqrt{a^2+b^2}} + \sqrt{a^2+b^2}\to \infty$
    \end{itemize}
    Therefore, all the allowable $c$ is $(a^2, \infty)$. 
    In particular, when $a=1$ we need $c>1$, but when $a=\sqrt{1/2019}$ we can have $c=1$. 

\end{enumerate}
\end{document}