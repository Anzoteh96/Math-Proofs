\documentclass[11pt,a4paper]{article}
\usepackage{amsmath, amssymb, fullpage, mathrsfs, bm, pgf, tikz}
\usepackage{mathrsfs}
\usetikzlibrary{arrows}
\setlength{\textheight}{10in}
%\setlength{\topmargin}{0in}
\setlength{\topmargin}{-0.5in}
\setlength{\parskip}{0.1in}
\setlength{\parindent}{0in}

\begin{document}
\newcommand{\la}{\leftarrow}
\newcommand{\lra}{\leftrightarrow}
\newcommand{\bbN}{\mathbb{N}}
\newcommand{\bbZ}{\mathbb{Z}}
\newcommand{\dsum}{\displaystyle\sum}
\newcommand{\dprod}{\displaystyle\prod}


\title{Solutions to Tournament of Towns, Fall 2016, Senior}
\author{Anzo Teh}
\date{}
\maketitle

\section*{O-Level}
\begin{enumerate}
	\item
\end{enumerate}

\section*{A-Level}
\begin{enumerate}
	\item [1.]
	Each of 100 boys has 100 biscuits in his plate. Instead of eating, they play a
	game. At each step, one of the boys chooses a group of boys (including at least
	one boy) and gives one biscuit to each of them. What is the minimal number of
	steps needed to make the number of biscuits different for all the boys?
	
	\textbf{Answer.} 50. 
	
	\textbf{Solution.} 
	We first show that 50 is necessary. 
	Suppose $k<50$ steps are taken. 
	This means at most $k$ kids have given their biscuits to other people, which means there are at most $k$ kids with $<100$ biscuits. 
	Meanwhile, each kid has received at most $k$ biscuits, so they have at most $100+k$ biscuits. 
	Consider, now, the following set: 
	\[
	S=\{a_1, \cdots, a_{100}\}
	\]
	where $a_i$ is the number of biscuit that kid $i$ has. 
	We see that $|S\cap \{1, \cdots, 99\}|\le k$, and $\max S\le 100+k$, 
	so $|S|\le k+(100+k-99)=2k+1\le 99$, i.e. not all of them can be distinct. 
	
	To construct the equality case, for each $i=1, \cdots, 50$, let kid $i$ give sweets to $50+j$ for $j=1, \cdots, i$. 
	Then $a_i=100-i$ for $i\le 50$, and $a_i=i+50$ for $i>50$, 
	so in total they have $50, \cdots, 99, 101, \cdots, 150$ sweets, i.e. all different. 
	
	\item [3.]
	A quadrilateral $ABCD$ is inscribed into a circle centered at point $O$ not belonging
	to the diagonals of the quadrilateral. The circle passing through points $A, O$ and
	$C$ contains the midpoint of $BD$. Prove that the circle passing through points $B,
	O$ and $D$ contains the midpoint of $AC$.
	
	\textbf{Solution.}
	Let $T$ be such that $TA$ and $TC$ are tangent to the said circle centered at $O$, 
	and let $M$ be midpoint of $BD$. 
	This gives $OM\perp BD$
	Since $M$ is on circle $AOC$ then line $BD$ will intersect the circle $AOC$ again at $T'$ such that $T'O$ is the diameter of the circle. 
	But since $OT$ is also a diameter, we have $T=T'$. 
	Hence $T, B, D$ are collinear so $ABCD$ is harmonic. 
	This means $AB\cdot CD= AD\cdot BC$. 
	
	Now let $U$ be such that $UB$, $UD$ tangent to circle $ABCD$. 
	Because $ABCD$ is harmonic, 
	there $U, A, C$ are collinear. 
	If $UA$ intersects circle $BOD$ again at $N$ then $\angle UNO=\angle UBO=90^{\circ}$ so 
	$AN\perp NO$. 
	Thus $N$ is the midpoint of $AC$ and is on circle $BOD$, done. 
	
	\item[6.]
	Alice and Bob play the following game: Alice chooses a polynomial $P(x)$ with
	integer coefficients. At each his move, Bob pays 1 dollar to Alice and reports her
	some integer $a$. He cannot choose the same integer $a$ twice. Alice responds by
	returning the number of integer solutions of the equation $P(x) = a$. Bob wins
	when Alice repeats the number already reported by her (not necessarily at the
	preceding move). Determine the minimal amount of cash sufficient for Bob to
	win the game regardless of the polynomial chosen by Alice.
	
	\textbf{Answer.} 4. 
	
	\textbf{Solution.} 
	To show that Bob cannot win in 3 turns, 
	we first show that for all distinct integers $a, b, c$ there exists polynomial $P$ such that 
	$P(x)=a, P(x)=b, P(x)=c$ has 2, 1, 0 integer solutions, respectively. 
	
	\emph{Case 1}: $b-a\neq 1$. Take 
	\[
	P(x)=(x-1)(x-(b-a))(Kx^2+1)+a, K>\max\{|b-a|, |c-a|, |c-b|\}
	\]
	where $K$ is some large number (determined later). 
	Then $P(x)=a$ has solutions at $x=1, b-a$. 
	$P(x)=b$ has solution at $x=0$. 
	By construction of $K$, we have $|Kx^2+1| > |b-a|$ for $x\neq 0$, 
	so $|(x-1)(x-(b-a))(Kx^2+1)|$ cannot be the same as $b-a$ when $x\neq 0$, 
	hence $P(x)=b$ cannot have other integer root. 
	Similarly, since $|Kx^2+1|>|c-a|$ for $x\neq 0$, 
	we have $|(x-1)(x-(b-a))(Kx^2+1)|>|c-a|$ whenever $x\neq 1, 0, b-a$. 
	So $P(x)=c$ cannot have integer root. 
	
	\emph{Case 2}: $b-a=1$. Then take 
	\[
	P(x)=(x-1)(x+1)(-Kx^2-1)+a, K>\max\{|b-a|, |c-a|, |c-b|\}
	\]
	and here $P(-1)=P(1)=a$, $P(0)=b$, and the rest of the analysis is similar to above. 
	
	Thus by this analysis, regardless of the 3 numbers $a, b, c$ Bob says, Alice can always say 0, 1, 2. 
	
	Now to show that Bob can win in 4 turns, we consider the following lemma: 
	
	\emph{Lemma}. Suppose that $P(x)=a$ has at least 3 integer roots. Then $P(x)=a+1$ and $P(x)=a-1$ have no integer root, each. 
	
	Proof: Write $P(x)=(x-a_1)(x-a_2)(x-a_3)Q(x)+a$, with $a_1, a_2, a_3$ distinct. 
	If $P(x)=a\pm 1$ for some $x$, then $|(x-a_1)(x-a_2)(x-a_3)Q(x)|=1$. 
	This means $|x-a_1|, |x-a_2|, |x-a_3|=1$, so $a_1, a_2, a_3$ are all $x\pm 1$. 
	But this would contradict $a_1, a_2, a_3$ all distinct. 
	
	Now Bob can do the following: query $a=0, 1, 2$. 
	If Alice repeats anywhere along then we're done, so assume all three answers are different. 
	Consider the following scenarios: 
	
	\emph{Case 1.} Alice answers $\ge 3$ at least once. 
	If $P(x)=1$ has at least 3 integer roots then by our  lemma above $P(x)=0$ and $P(x)=2$ would have none, 
	so Alice repeats. 
	If $P(x)=0$ has at least 3 integer roots then $P(x)=-1$ and $P(x)=1$ have none, 
	so Bob can ask $-1$ in his next step. 
	THe case where $P(x)=2$ has 3 integer roots is symmetric where Bob can ask 3 in his next turn. 
	
	\emph{Case 2.} Alice answers 0, 1, 2 in some order. 
	Now at least one of $P(x)=0$ and $P(x)=2$ returns $\ge 1$ integer. 
	If this is $P(x)=0$ then $P(x)=-1$ cannot have $\ge 3$ integer roots $\to$ Bob can query $-1$ in his next turn and Alice will repeat herself. 
	If this is $P(x)=2$ then $P(x)=3$ cannot have $\ge 3$ integer roots $\to$ Bob can query $3$ in his next turn and Alice will repeat herself. 

\end{enumerate}
\end{document}