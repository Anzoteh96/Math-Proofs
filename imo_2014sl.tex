\documentclass[11pt,a4paper]{article}
\usepackage{amsmath, amssymb, fullpage, mathrsfs, bm, pgf, tikz}
\usepackage{mathrsfs}
\usetikzlibrary{arrows}
\setlength{\textheight}{10in}
%\setlength{\topmargin}{0in}
\setlength{\topmargin}{-0.5in}
\setlength{\parskip}{0.1in}
\setlength{\parindent}{0in}

\begin{document}
\newcommand{\la}{\leftarrow}
\newcommand{\lra}{\leftrightarrow}
\newcommand{\bbN}{\mathbb{N}}
\newcommand{\bbZ}{\mathbb{Z}}
\newcommand{\dsum}{\displaystyle\sum}
\newcommand{\dprod}{\displaystyle\prod}

\section*{Algebra}
\begin{enumerate}
	\item [\textbf{A1}] (IMO 1) Let $a_0 < a_1 < a_2 \ldots$ be an infinite sequence of positive integers. Prove that there exists a unique integer $n\geq 1$ such that
	\[a_n < \frac{a_0+a_1+a_2+\cdots+a_n}{n} \leq a_{n+1}.\]
	
	\textbf{Solution.} The condition that $a_n<\dfrac{a_0+a_1+a_2+\cdots+a_n}{n}$ means that $a_0+a_1+\cdots + a_{n-1}-(n-1)a_n>0$ and similarly $\dfrac{a_0+a_1+a_2+\cdots+a_n}{n} \leq a_{n+1}$ means $a_0+\cdots + a_n-na_{n+1}\le 0$. 
	Let $f(n)=a_0+\cdots + a_{n-1}-(n-1)a_n$. We are to prove that there exists a unique $n$ such that $f(n)>0$ but $f(n+1)\le 0$. 
	
	We first claim that $f(n)$ is a decreasing function. To see this, we have 
	\[f(n)-f(n+1) = (a_0+a_1+\cdots + a_{n-1}-(n-1)a_n) - (a_0+\cdots + a_n-na_{n+1})
	=n(a_{n+1}-a_n)
	\]
	and since $\{a_n\}$ is a strictly increasing sequence, we have $n(a_{n+1}-a_n)>0$, and therefore $f(n)-f(n+1)>0$ for all $n$. 
	
	The next observation is that $f(1)=a_0-(1-1)a_1=a_0>0$. Since $f(n)$ decreases by at least 1 as $n$ increases(because $\{a_n\}$) are sequence of integers), we see that there exists an $n_0$ with $f(n_0)\le 0$ (in particular we can choose $f(a_0+1)$). 
	Since $f(n)\le 0$ for all $n\ge n_0$, the set $\{n: f(n)>0\}$ is finite but nonzero since $f(1)>0$. Let $n_1$ be the largest element in this set, and we see that $f(n_1)>0$ but $f(n_1+1)\le 0$, $f(n)>0$ for $n\le n_1$ and $f(n)\le 0$ for all $n>n_1$. Thus this $n_1$ is the only $n$ fitting the criteria. 
	
	\item[\textbf{A2}] Define the function $f:(0,1)\to (0,1)$ by \[\displaystyle f(x) = \left\{ \begin{array}{lr} x+\frac 12 & \text{if}\ \  x < \frac 12\\ x^2 & \text{if}\ \  x \ge \frac 12 \end{array} \right.\] Let $a$ and $b$ be two real numbers such that $0 < a < b < 1$. We define the sequences $a_n$ and $b_n$ by $a_0 = a, b_0 = b$, and $a_n = f( a_{n -1})$, $b_n = f (b_{n -1} )$ for $n > 0$. Show that there exists a positive integer $n$ such that \[(a_n - a_{n-1})(b_n-b_{n-1})<0.\]
	
	\textbf{Solution.} We notice that $f(x)>x$ if $x<\frac 12$, and if $x\ge 12$ then since $x<1$ as well, we have $f(x)<x$. Hence, it suffices to find $n$ such that either $a_n<\frac 12\le b_n$, or $b_n<\frac 12 \le a_n$. 
	
	Suppose otherwise, then for each $n$, we either have $a_n, b_n < 12$, or $a_n, b_n\ge\frac 12$. 
	Within each range $(0, \frac 12)$ and $[\frac 12, 1)$, we also have $f$ monotonous: $f(x)<f(y)$ iff $x<y$. 
	Hence from $a<b$ we also have $a_n<b_n$ for all $n$. We now consider the difference $b_n-a_n$ in terms of $b_{n-1}-a_{n-1}$: 
	\begin{itemize}
		\item If $a_{n-1}, b_{n-1}<\frac 12$ then $b_n-a_n=b_{n-1}-a_{n-1}$. 
		
		\item Otherwise we have $b_n-a_n = (b_{n-1}-a_{n-1})(b_{n-1}+a_{n-1})$, and by $\frac 12\le a_{n-1}<b_{n-1}$ we also have $b_{n-1}-a_{n-1}>1$ so $b_n-a_n>b_{n-1}-a_{n-1}$. 
	\end{itemize}
	Regardless, we always have $b_n-a_n \ge b_{n-1} - a_{n-1}$. 
	
	Now let $c=b-a=b_0-a_0$. We have $b_{n}-a_{n}\ge c$ as always. If $n_1< n_2, \cdots < n_k<n$ are indices such that $a_{n_i}\ge\frac 12$ then $b_{n_i+1}-a_{n_i+1} = (b_{n_i} - a_{n_i})(b_{n_i} + a_{n_i})\ge (1+c)(b_{n_i}-a_{n_i})$. Thus we now have $a_n-b_n\ge (1+c)^{k}c$. 
	Since $0< a_n<b_n<1$, $(1+c)^kc<1$, i.e. $k<-\log_{1+c}(c)$. 
	This also means that regardless of $n$, the number of $m<n$ with $a_m, b_m,\ge \frac 12$ is bounded by $-\log_{1+c}(c)$ which does not depend on $n$. Hence such $m$'s are also finite. 
	This means that there exists an $N$ such that for all $n\ge N$ we have $a_n<\frac 12$. This is impossible, since $a_n<\frac 12$ means that $a_{n+1}=a_n+\frac 12>\frac 12$. The desired contradiction proves our statement. 
	
	\item[\textbf{A4}] Determine all functions $f: \mathbb{Z}\to\mathbb{Z}$ satisfying \[f\big(f(m)+n\big)+f(m)=f(n)+f(3m)+2014\] for all integers $m$ and $n$.
	
	\textbf{Answer.} The only such function is $f(n)=2n+1007$. \\
	\textbf{Solution.} We first consider $m=0$, and let $f(0)=c$. So plugging $m=0$ yield $f(n+c)+c=f(n)+c+2014$, i.e. $f(n+c)-f(n)=2014$. By considering $n, n+c, n+2c, \cdots$ we can generalize this statement to $f(n+kc)-f(n)=2014k$ for all $k\ge 0$. 
	Next, consider $m_1=m+c$, and therefore $f(m_1)=f(m)+2014$ and $f(3m_1)=f(3m+3c)=f(3m)+3(2014)$. 
	Therefore we have the following: 
	\[f(f(m_1)+n)+f(m_1)=f(f(m)+n+2014)+f(m)+2014\]
	and
	\[
	f(n)+f(3m_1)+2014=f(n)+f(3m)+3(2014)+2014=f(n)+f(3m)+4(2014)
	\]
	and therefore $f(f(m)+n+2014)+f(m)=f(n)+f(3m)+3(2014)$. Comparing this with the original statement and subtracting both sides, we have $f(f(m)+n+2014)-f(f(m)+n)=2(2014)$. Since $f(m)+n$ attains all values in $\bbZ$ (by varying $n$), we have $f(n+2014)-f(n)=2(2014)$ for all $n$, or more generally (by following the logic above), $f(n+2014k)-f(n)=2k(2014)$. 
	
	We now have $f(n+kc)-f(n)=2014k$ for all $k\ge 0$ and $f(n+2014\ell)-f(n)=2\ell(2014)$ for all $\ell\ge 0$. Let $k=2014$ and $\ell=c$ we have $2014^2=2c(2014)$, forcing $c=1007$ (and therefore $f(0)=1007$). 
	
	We now consider the next set of formulation: $f(f(m)+n)-f(n)=f(3m)-f(m)+2014$. Denote $d=f(3m)-f(m)+2014$. 
	This means, for each $m$, considering the sequence $n, f(m)+n, 2f(m)+n$, etc, we get $f(kf(m)+n)-f(n)=kd$. Considering, that, when $k=1007$ we have $f(1007f(m)+n)-f(n)=1007d$, but by above $f(1007f(m)+n)-f(n)$ is also $2014f(m)$, yielding $2014f(m)=1007d$, or $d=2f(m)$. This means, $2f(m)=f(3m)-f(m)+2014$, i.e. $3f(m)=f(3m)+2014$, or simply $f(3m)=3f(m)-2014$. Inductively this also means that for all $m\neq 0$ we have $f(3^km)=3^kf(m)-(3^{k-1}+3^{k-2}+\cdots + 1)2014=3^kf(m)-\frac{3^k-1}{2}(2014)$. By Fermat's little theorem, we can choose $k_0$ such that $1007\mid 3^{k_0}-1$. 
	Thus for this $k_0$, we have $f(3^km)=f(m+(3^{k_0}-1)m)=f(m)+2(3^{k_0}-1)m$ (reason being that if $3^{k_0}-1=1007a$ then $f(m+(3^{k_0}-1)m)=f(m+1007am)=f(m)+2014am=f(m)+2(3^{k_0}-1)m$). 
	On the other hand the right hand suggests that this sum is also equal to $3^kf(m)-\frac{3^k-1}{2}(2014)$ and therefore
	\[(3^{k_0}-1)f(m)=2(3^{k_0}-1)m + \frac{3^{k_0}-1}{2}(2014) = (3^{k_0}-1)(2m + 1007)
	\]
	and therefore $f(m)=2m+1007$ (we need to choose $k_0>0$ so that $3^{k_0}-1\neq 0$, but such $k_0$ exists by FLT). 
	
	Finally, such function works: we have the left hand side as $4(2017)+6m+2n$, and same goes to right hand. Q.E.D. 
	
\end{enumerate}

\section*{Combinatorics}
\begin{enumerate}
	\item[\textbf{C1}] Let $n$ points be given inside a rectangle $R$ such that no two of them lie on a line parallel to one of the sides of $R$. The rectangle $R$ is to be dissected into smaller rectangles with sides parallel to the sides of $R$ in such a way that none of these rectangles contains any of the given points in its interior. Prove that we have to dissect $R$ into at least $n + 1$ smaller rectangles.
	
	\textbf{Solution.} We will prove this by induction: for $n=0$, we already have the rectangle $R$, hence one rectangle needed. 
	
	Now consider the set $S$ of $n$ points inside a given rectangle, and a configuration that splits it into $n+1$ rectangles.  Let a point $P$ be arbitrary. By induction hypothesis, the rectangle containing $S\backslash \{P\}$ must be dissected into at least $n$ rectangles. 
	
	By the problem requirement, $P$ must lie on some boundary, which is a segment that we have drawn to split it into the rectangles. We now recover the dissection of $S\backslash \{P\}$ by the following algorithm: 
	\begin{itemize}
		\item Remove the segment $\ell$ containing $P$ entirely. WLOG assume that it's horizontal. 
		
		\item The resulting configuration might not be rectangles, but this can fixed in the following way: let the interior of $\ell$ intersect with vertical lines $\ell_1, \cdots , \ell_k$, then we extend each $\ell_i$'s to meet the next posible horizontal line. 
	\end{itemize}
	Now from the second protocol, we know that $k+2$ rectangles correspond to our line $\ell$. After the removal of $\ell$ and extending the $\ell_i$'s, we have at most $k+1$ rectangles overlapping the space that would otherwise be covered by $\ell$. Thus the number of rectangles decrease by at least 1. By induction principle, we have at least $n$ rectangles after the removal of $\ell$, so we have at least $n+1$ of them in the beginning. 
	
	\item[\textbf{C2}] We have $2^m$ sheets of paper, with the number $1$ written on each of them. We perform the following operation. In every step we choose two distinct sheets; if the numbers on the two sheets are $a$ and $b$, then we erase these numbers and write the number $a + b$ on both sheets. Prove that after $m2^{m -1}$ steps, the sum of the numbers on all the sheets is at least $4^m$ .
	
	\textbf{Solution.} We consider the product on the $2^m$ paper, which is 1 in the beginning. Each time, the product increases by $\frac{(a+b)^2}{ab}$ times, and since $a, b > 0$ and $(a+b)^2-4ab=(a-b)^2\ge 0$, this ratio is at least 4. Thus the product after $m2^{m-1}$ steps is at least $4^{m2^{m-1}}=2^{m2^m}.$ The geometric mean is at least $2^m$ and by the AM-GM inequality, the arithmetic mean is at least $2^m$ too. Therefore the sum is at least $2^m\times 2^m=4^m$. 
	
	\item[\textbf{C3}] (IMO 2)
	Let $n \ge 2$ be an integer. Consider an $n \times n$ chessboard consisting of $n^2$ unit squares. A configuration of $n$ rooks on this board is peaceful if every row and every column contains exactly one rook. Find the greatest positive integer $k$ such that, for each peaceful configuration of $n$ rooks, there is a $k \times k$ square which does not contain a rook on any of its $k^2$ unit squares.
	
	\textbf{Answer.} $k=\lceil\sqrt{n}\rceil -1$\\
	\textbf{Solution.} We show that such $k$ fits if and only if $k^2<n$. 
	Consider, now, any peaceful configuration with $k^2<n$. Consider the first $k^2\times k^2$ squares: squares with coordinates $(i, j)$ with $1\le i, k\le k^2$, and suppose that the condition does not hold true. We now divide the $k^2\times k^2$ squares into $k\times k$ large grids, each large grid being a $k\times k$ square. By our assumption each large grid contains at least a rook, so there are at least $k^2$ rooks in the $k^2\times k^2$ squares. 
	But then these $k^2\times k^2$ squares span only $k^2$ columns (and $k^2$ rows) so there cannot be more than $k^2$ rooks. This means: 
	\begin{itemize}
		\item There are \emph{exactly} $k^2$ rooks in these $k^2\times k^2$ squares. 
		
		\item There is no rook in $(i, j)$ if exactly one of $i$ and $j$ lie in the interval $[1, k^2]$. 
		
		\item Consequently, the remaining $n-k^2$ rooks must be in the grid $(i, j)$ with $k^2+1\le i, j\le n$ (i.e. $m\times m$ square with $m=n-k^2$). 
	\end{itemize}
	In other words, there are $m$ rooks ($m=n-k^2$) in the bottom right corner of $m\times m$ squares. But then by rotational symmetry the same holds for the top left corner, top right corner and bottom left corner, which would contradict that each column / row contains exactly one rook. 
	
	Now we move on to prove that if $n$ and $k$ are such that $n\le k^2$, then there is a peaceful configuration such that any $k\times k$ square contains at least a rook. We first consider the case $n=k^2$, which allows us to consider (like above) $k\times k$ large grid, each of which is a $k\times k$ square. For the $(i, j)$-th large grid, we place a rook at coordinate $(j, i)$: in other words for each $1\le i, j\le k$ the rooks are at $((i-1)k+j, (j-1)k+i)$, and it's not hard to see that there are $k^2$ of them and each of them has different first- and second- coordinates. 
	
	To show that each $k\times k$ square must contain at least a rook, consider any $k\times k$ square. Now there are three cases: 
	\begin{itemize}
		\item This $k\times k$ square coincides with one of the big square before; by definition it contains at least a rook. 
		
		\item It overlaps with two of the big squares, and two opposite boundaries of this $k\times k$ square also coincide with the boundary of the two big squares. W.L.O.G. let the rows to be $(i-1)k+1, \cdots , ik$ (but the columns can be any $k$ consecutive columns). Recall that there are rooks at $((i-1)k+j, (j-1)k+i)$, $j=1, 2, \cdots , k$. Since the columns of the rooks are $(j-1)k+i$ for $j=1, 2, \cdots , k$, exactly one of them match the columns covered by this $k\times k$ square. Since the rows are $(i-1)k+j$, the rows of all rooks match the rows covered by this $k\times k$ square, too. Hence there's a rook in this $k\times k$ square. 
		
		\item Neither of the above holds, so it must overlap with four neighbouring $(2\times 2)$ big grids. Consider Let the four neighbouring big grids be $(i, j), (i, j+1), (i+1, j), (i+1, j+1)$. We also let the top left corner of our $k\times k$ square be $((i-1)k+a, (j-1)k+b)$ with $1\le a, b\le k$. 
		
		Considering $(i, j), (i, j+1)$, with rooks at $((i-1)k+j, (j-1)k+i)$ and $((i-1)k+(j+1), jk+i)$. Since our $k\times k$ square covers columns $(j-1)k+b, \cdots , jk+(b-1)$, either $(j-1)k+i$ or $jk+i$ is covered by these intervals. Hence, if both rooks are not in this grid then either $((i-1)k+j)$ or $((i-1)k+(j+1))$ is not covered as row. 
		Similarly, we have the following: 
		\begin{itemize}
			\item Considering $(i+1, j)$ and $(i+1, j+1)$ means that either $ik+j$ or $ik+(j+1)$ is not covered as row. 
			\item Considering $(i, j)$ and $(i+1, j)$ with rooks at $((i-1)k+j, (j-1)k+i)$ and $(ik+j, (j-1)k+(i+1))$ means either $(j-1)k+i$ or $(j-1)k+(i+1)$ is not covered as column. 
			
			\item Similarly considering $(i, j+1)$ and $(i+1, j+1)$ means either $jk+i$ or $jk+(i+1)$ is not covered as column. 
		\end{itemize}
		Thus considering our top left rows and columns are in the form $(i-1)k+a$ and $(j-1)k+b$ the only choice is $a=j+1$ and $b=i+1$, which covers rows $(i-1)k+(j+1), \cdots , ik+j$ and columns $(j-1)k+(i+1), \cdots jk+i$. 
		However, $(ik+j, (j-1)k+(i+1))$ itself contains a rook, contradiction. 
	\end{itemize}
	These effectively solve for $n=k^2$. 
	
	For $n<k^2$, all we need to do is simply extend the chess board from $n$ to $k^2$, put the rooks in the configuration above, and then remove those that do not lie in the first $n\times n$ grid. Now the condition still holds: each $k\times k$ square has at least a rook, and each of the remaining rooks are in different rows and columns. Then iteratively (one by one), we identify a column and a row without a rook, and place one at the intersection of the column and row until we have all $n$ of them. Each $k\times k$ square will still contain a rook since we only add the rooks, not remove them. 
\end{enumerate}

\section*{Geometry}
\begin{enumerate}
	\item [\textbf{G1}] (IMO 4) Let $P$ and $Q$ be on segment $BC$ of an acute triangle $ABC$ such that $\angle PAB=\angle BCA$ and $\angle CAQ=\angle ABC$. Let $M$ and $N$ be the points on $AP$ and $AQ$, respectively, such that $P$ is the midpoint of $AM$ and $Q$ is the midpoint of $AN$. Prove that the intersection of $BM$ and $CN$ is on the circumference of triangle $ABC$.
	
	\textbf{Solution.} We need to prove that $\angle CBM + \angle BCN = \angle BAC$. 
	Notice that the triangle $ABC$ is similar to both $QAC$ and $PBA$, so we have $\frac{QN}{QC}=\frac{QA}{QC}=\frac{PB}{PA}=\frac{PB}{PM}$. 
	Coupled with the fact that $\angle APB = \angle AQC = \angle BAC$ we have $\angle CQN = \angle BPM$, so the triangles $QNC$ and $PBM$ are similar, meaning that $\angle CBM + \angle BCN = \angle QNC + \angle QCN = 180^{\circ} - \angle CQN = \angle AQC = \angle BAC$, as desired. 
	
	\item [\textbf{G3}] Let $\Omega$ and $O$ be the circumcircle and the circumcentre of an acute-angled triangle $ABC$ with $AB > BC$. The angle bisector of $\angle ABC$ intersects $\Omega$ at $M \ne B$. Let $\Gamma$ be the circle with diameter $BM$. The angle bisectors of $\angle AOB$ and $\angle BOC$ intersect $\Gamma$ at points $P$ and $Q,$ respectively. The point $R$ is chosen on the line $P Q$ so that $BR = MR$. Prove that $BR\parallel AC$.
	(Here we always assume that an angle bisector is a ray.)
	
	\textbf{Solution.} We first notice that $OP$ is the perpendicular bisector of $AB$, and $OQ$ the perpendicular bisector of $BC$. Now extend $OP$ to meet $\Gamma$ again at $P_1$, and definte $Q_1$ similarly. 
	If $N$ is the midpoint of $BM$, then $ON$ is perpendicular to $BM$, and since $BM$ is the internal angle bisector of $\angle ABC$, line $OP$ and line $OQ$ are symmetric in the line passing through $O$ and parallel to $BM$. Thus $OP$ and $OQ$ are also symmetric in $ON$, meaning that $PQP_1Q_1$ is actually isoceles trapezoid with parallel sides $PQ_1$ and $QP_1$. 
	Moreover, if $PQ$ and $P_1Q_1$ were to intersect at $R_1$, the points $O, N, R_1$ are all collinear, and is the perpendicular bisector of $PQ_1$, $QP_1$. Considering that $BM$ is the diameter of the circle and $ON\perp BM$, $ON$ is also the perpendicular bisector of $BM$. Since $R_1$ is on this perpendicular bisector, $BR_1=MR_1$, and since $R_1$ is also on $PQ$, we have $R=R_1$. 
	
	By Brokard's theorem, taking the point of infinity $P_{\infty}$ determined by the parallel lines $PQ_1$ and $QP_1$, we have $R$ as the polar of the line $OP_{\infty}$ (i.e. the line through $O$ parallel to $BM$). Consider, now, the lines $BR$ and $MR$ and let them intersect $\Gamma$ again at $B_1$ and $M_1$. Since $BR=MR$, we also have $BM\parallel B_1M_1$. Also by above (Brokard's theorem again), $MB_1$ and $BM_1$ intersect at $O$. Now $\angle MB_1B=90^{\circ}$, so $BM_1$ (i.e. line $BR$) is perpendicular to $MB_1$, i.e. $MO$. But $AC$ is also perpendicular to $MO$ since $M$ is the midpoint of arc $AC$, so $AC\parallel BR$. Q.E.D. 
	
	\item [\textbf{G5}] (IMO 3) Convex quadrilateral $ABCD$ has $\angle ABC = \angle CDA = 90^{\circ}$. Point $H$ is the foot of the perpendicular from $A$ to $BD$. Points $S$ and $T$ lie on sides $AB$ and $AD$, respectively, such that $H$ lies inside triangle $SCT$ and \[
	\angle CHS - \angle CSB = 90^{\circ}, \quad \angle THC - \angle DTC = 90^{\circ}. \] Prove that line $BD$ is tangent to the circumcircle of triangle $TSH$.
	
	\textbf{Solution.} We first show that $Q$, the circumcenter of $CHS$, lies on the line $AB$. To see this, by the identity of angle of chord subtending at the center vs angle subtending on the arc we have 
	\[2\angle CHS = 360^{\circ} - \angle CQS
	\]
	since $\angle CHS>90^{\circ}$ by the problem definition. 
	Since $CQ=CS$, we have $\angle CSQ = 90^{\circ} - \frac{\angle CQS}{2} = 90^{\circ} - \frac{360^{\circ} - 2\angle CHS}{2} = \angle CHS - 90^{\circ} = \angle CSB$, implying that $Q, S, B$ are collinear and therefore all lie on line $AB$. Similarly, naming $R$ as the circumcenter of $DHT$ we have $R$ on $AD$. 
	
	Now, with the fact that $AH$ is perpendicular to $BD$, the goal reduces to showing that the circumcenter $O$ of $TSH$ lies on $AH$. But then $O$ is on the perpendicular bisector of $SH$, which is in turn the internal angle bisector of $SQH$ (i.e. angle bisector of $AQH$) because $SQ=QH$. Similarly $O$ is on the angle bisector of $ARH$. So all we need to prove is that the angle bisectors of $\angle SQH$ and $\angle ARH$ intersect at $AH$. By the angle bisector theorem this is equivalent to proving the following ratio: 
	\[
	\frac{AQ}{QH} = \frac{AR}{RH}\to \frac{AQ}{AR} = \frac{QH}{RH}
	\]
	the second equivalence being equivalent to the first, hence we will prove the second one. 
	
	Let $N$ be the midpoint of $CH$, the $QR$ is the perpendicular of $CH$, hence perpendicular to $CH$ and passes through $N$. We also have $\angle CBQ=\angle CNQ = 90^{\circ}$ so $C, B, Q, N$ are concyclic. Similarly $C, B, R, N$ is concyclic. We now have the following ratio, thanks to sine rule: 
	\[
	\frac{AQ}{AR} = \frac{\sin\angle QRA}{\sin\angle RQA} = \frac{\sin\angle NCD}{\sin\angle NCB}
	\]
	and since $QH=QC$ and $RH=RC$, 
	\[
	\frac{QH}{RH} = \frac{QC}{RC} = \frac{\sin\angle CRQ}{\sin\angle CQR} = \frac{\sin\angle CDN}{\sin\angle CBN}
	\]
	but then $\frac{\sin\angle NCD}{\sin\angle NCB} = \frac{\sin\angle CDN}{\sin\angle CBN}$ if and only if $\frac{\sin\angle NCD}{\sin\angle CDN} = \frac{\sin\angle NCB}{\sin\angle CBN}$. The first ratio is (sine rule again) the same as $\frac{NB}{NC}$ and the second, $\frac{ND}{NC}$, so all we need now is $NB=ND$. 
	To see why this is true, let $L$ be the midpoint of $AC$. Then since $\angle B=\angle D=90^{\circ}$, $L$ is the circumcenter of the qudrilateral $ABCD$, and hence $LB=LD$. Since $N$ is the midpoint of $CH$ we also have $AH\parallel NL$, meaning that $NL\perp BD$. But since $L$ is on the perpendicular bisector of $BD$, so is $N$, and therefore $NB=ND$. Q.E.D. 
\end{enumerate}

\section*{Number Theory}
\begin{enumerate}
	\item [\textbf{N2}] Determine all pairs $(x, y)$ of positive integers such that \[\sqrt[3]{7x^2-13xy+7y^2}=|x-y|+1.\]
	
	\textbf{Answer.} $\{x, y\}=\{1, 1\}$ and $\{c^3+c^2-2c-1, c^3+2c^2-c-1\}$. \\
	\textbf{Solution.} By symmetry we can assume that $y\ge x$, and let $k$ be such that $y=x+k$. Also, $7x^2-13xy+7y^2=7(x^2-2xy+y^2)+xy=7(x-y)^2+xy=7k^2+x(x+k)$. Therefore we now have 
	\[(k+1)^3 = 7k^2+x(x+k)
	\]
	and notice that this is actually a quadratic equation in terms of $x$, i.e. $x^2+kx+(7k^2-(k+1)^3)=0$. Since we are finding for integer solutions when $x$ and $k$ are both integers, we need the discriminant $k^2-4(7k^2-(k+1)^3)$ to be a perfect square. But notice the following: 
	\[
	k^2-4(7k^2-(k+1)^3)=4k^3-15k^2+12k+4=(k-2)(4k^2-7k-2)=(k-2)^2(4k+1)
	\]
	so either $k=2$ or $4k+1$ is a perfect square. Since $4k+1$ is odd, we have $4k+1=(2c+1)^2$ for some nonnegative integer $c$, leaving $k=c(c+1)$. Hence the following equation obtained by solving the squadratic equation: 
	\[
	x=\frac{-k\pm \sqrt{(k-2)^2(4k+1)}}{2}
	=\frac{-c(c+1)\pm |c(c+1)-2|\cdot (2c+1)}{2}
	\]
	The product of roots is given by $7k^2-(k+1)^3$, which is positive only when $k=2$ and negative for the rest of the cases. For $k=2$ the discriminant turned out to be 0, so we have $x^2+2x+1=0$, or $(x+1)^2=0$, so $x=-1$ which is impossible because $x$ must be positive. Otherwise, there would be one positive and one negative root in the quadratic equation, and therefore we choose the bigger root given by the plus sign, yielding: 
	\[
	x=\frac{-c(c+1)+ |c(c+1)-2|\cdot (2c+1)}{2}\]
	and
	\[
	y=x+k = \frac{-c(c+1)+ |c(c+1)-2|\cdot (2c+1)}{2} + c(c+1) 
	=\frac{c(c+1)+ |c(c+1)-2|\cdot (2c+1)}{2}
	\]
	for all $c=0, 2, 3, 4, 5, \cdots $. ($c=1$ corresponds to $k=2$). Notice for $c=0$ with $k=0$ we actually have $1=x^2$ so $x=y=1$. Otherwise we have $|c(c+1)-2|>0$ so 
	we can remove the modulus to get 
	\[
	x=\frac{-c(c+1)+ (c(c+1)-2)\cdot (2c+1)}{2}=c^3+c^2-2c-1
	\]
	and similarly $y=c^3+2c^2-c-1$. 
	
	\item [\textbf{N3}] (IMO 5) For each positive integer $n$, the Bank of Cape Town issues coins of denomination $\frac1n$. Given a finite collection of such coins (of not necessarily different denominations) with total value at most $99+\frac12$, prove that it is possible to split this collection into $100$ or fewer groups, such that each group has total value at most $1$.
	
	\textbf{Solution.} In fact, we shall prove that for any pile of coins worth at most $n-\frac 12$ we can split them into $n$ groups, each of total value at most 1. 
	
	The key to this problem is to `group' certain coins according to the following algorithm, in the hope to minimize the number of cape town coins and their denominators. To see this, we perform the following operations iteratively: for each $a>1$ and positive integer $b$, if there are $a$ coins of value $\frac {1}{ab}$, then we group them together into a single coin of value $\frac {1}{b}$ (notice that the total value never changes). Since the number of coins decrease each iterations and there are only a finite number fo coins, such operation can only be done finitely many times, hence it must terminate. This means, at the end of the process, for each positive integer $b$ and a prime number $p$ (or simply any $p>1$) dividing $b$, the number of coins of value $\frac{1}{b}$ must be less than $p$. In particular, if $b$ is even, choosing $p=2$ means there is at most one such coin. 
	
	Next, let $m$ be the number of coins with value 1. We place these coins into $m$ separate piles, leaving coins of total value at most $n-m-\frac 12$. Denote $n_1=n_1-\frac 12$ and consider the coins of values $\frac{1}{k}$ for $k=2, 3, \cdots , 2n_1$. 
	We have seen that if $k$ is even then there is at most 1 coin of value $\frac 1k$, and if $k$ is odd, there is at most $k-1$ of them. Consider, also, the fact that $\frac{2i-2}{2i-1}+\frac{1}{2i}<1$ for all $i\ge 2$, we can group all coins of value $\frac{1}{2i-1}$ and $\frac{1}{2i}$ into one pile, for all $i=2, 3, \cdots , n_1$ (i.e. $n_1-1$ of them), and place the coin (if exists; otherwise we just have an empty pile) with value $\frac 12$ into the last pile, making a total of $n_1$ piles right now. Now all piles have value at most 1. 
	
	We are not done yet: there might still be coins of value $\frac{1}{k}$ for $k>2n_1$. However, the coins in the piles have total value at most $n_1-\frac 12$, i.e. an average (w.r.t. pile) of at most $1-\frac{1}{2n_1}$. 
	By pigeonhole principle, there must be a pile of value at most $1-\frac{1}{2n_1}$. Since each leftover coin has value less than $\frac{1}{2n_1}$, we can choose any of the leftover coin and put into this pile of value at most $1-\frac{1}{2n_1}$, and after that this pile's value cannot exceed 1. This invariant holds as long as there is a coin not in any pile yet, so we can repeat this argument, find a pile of value at most $1-\frac{1}{2n_1}$ and put a coin into the pile. Eventually, all coins are in the piles. Q.E.D. 
	
	\item[\textbf{N4}] Let $n > 1$ be a given integer. Prove that infinitely many terms of the sequence $(a_k )_{k\ge 1}$, defined by \[a_k=\left\lfloor\frac{n^k}{k}\right\rfloor,\] are odd. (For a real number $x$, $\lfloor x\rfloor$ denotes the largest integer not exceeding $x$.)
	
	\textbf{Solution.} We settle the easier case first: $n$ is odd. Now for each $k=n^m$, we have $\frac{n^k}{k}=n^{k-m}=n^{n^m-m}$ and for sufficiently large $m$, $n^m\ge m$ ($n\ge 2$ and therefore it's well-known that $n^m\in \Omega(m)$ for each $n>1$ fixed). This means that $k\mid n^k$ and therefore in this case $a_k=\frac{n^k}{k}$. Since $n$ is odd, so is $n^k$ and therefore $\frac{n^k}{k}$ is also odd. 
	
	Now let's see the even case, and we show that there are infinitely many $m$'s such that $k=n^m(n+1)$ will work. To see how this works, consider the following: 
	\[
	\frac{n^k}{k} = \frac{n^{n^m(n+1)}}{n^m(n+1)}
	=\frac{n^{n^m(n+1)-m}}{n+1}
	\]
	we now proceed to a lemma: for $n\ge 2$ even, $\left\lfloor\frac{n^k}{n+1}\right\rfloor$ is even for all $k\ge 1$ odd, and even otherwise. To see why, notice that $\left\lfloor\frac{n^k}{n+1}\right\rfloor$ is even iff $n^k$ is congruent to $0, 1, \cdots , n$ modulo $2n+2$, and odd if congruent to $n+1, \cdots , 2n+1$ modulo $2n+2$. 
	When $k=1$ the congruence is $n$; when $k=2$ we have $n^2=(n+1)(n-1)+1=(n+1)(n-2)+n+2$ and since $n-2$ is even, we have $2n+2\mid (n+1)(n-2)$ and therefore $n^2\equiv n+2\pmod{2n+2}$. 
	Finally, when $k=3$ we have \[n^3\equiv (n^2)\cdot n\equiv (n+2)n\equiv n^2+2n\equiv n+2+2n\equiv 3n+2\equiv n\pmod{2n+2}\]
	so the congruence alternates between $n$ and $n+2$ when $k$ is even or odd, completing the proof for our lemma. 
	
	To finish the proof, we need to find those (infinitely many) $m$ such that $n^m(n+1)-m$ is even. Since $n$ is even, $n^m(n+1)$ is even for all $m>0$, which reduces to finding $m$ is even. Thus all $m\ge 2$ even works, Q.E.D. 
	
	\item[\textbf{N5}] Find all triples $(p, x, y)$ consisting of a prime number $p$ and two positive integers $x$ and $y$ such that $x^{p -1} + y$ and $x + y^ {p -1}$ are both powers of $p$.
	
	\textbf{Answer.} Any triples in the form $(2, x, 2^k-x)$ provided $2^k>x$, and $(3, 2, 5)$ and $(3, 5, 2)$. \\
	\textbf{Solution.} When $p=2$, we are only asked to find the pairs where $x+y$ is a power of 2 which is easily settled above. From now on we focus only on odd primes $p$. 
	
	First, we show that $x$ and $y$ cannot be simultanouesly divisible by $p$. Otherwise, let $v_p(x)\le v_p(y)$, then $v_p(y^{p-1})=(p-1)v_p(y)>v_p(y)\ge v_p(x)$ and therefore $v_p(y^{p-1}+x)=v_p(x)$. If $x=cp^k$ where $k=v_p(x)$ then $y^{p-1}+x=dp^k$ too, with $p\nmid d$. Since this $y^{p-1}+x$ must be a $p$-th power, we have $d=1$, but $dp^k>cp^k$ so $1=d>c\ge 1$, contradiction. 
	
	Therefore we have $p$ not dividing $x$ and $y$. Now let $x\le y$ and let $x^{p-1}+y=p^k$. THen $p^k\mid y^{p-1}+x$, too. We now have $y\equiv -x^{p-1}\pmod{p^k}$ and $0\equiv y^{p-1}+x\equiv (-x^{p-1})^{p-1}+x\equiv x^{(p-1)^2}+x=x(x^{(p-1)^2-1}+1)\pmod{p^k}$. 
	Since $p\nmid x$, we have $p^k\mid x^{(p-1)^2-1}+1$, i.e. $x^{(p-1)^2-1}\equiv -1\pmod{p^k}$. 
	
	Now by Fermat's Little theorem, $1\equiv x^{(p-1)^2}\equiv x\cdot(-1)=-x\pmod{p}$ so $p\mid x+1$. We now let $x=cp^{\ell}-1$ with $p\nmid c$ and $\ell\ge 1$. We now consider the expansion $(cp^{\ell}-1)^{(p-1)^2-1}+1$, with the following observation: 
	\begin{itemize}
		\item The expansion has the form $\dsum_{i=1}^{p(p-2)}\dbinom{p(p-2)}{i}(-1)^{p(p-2)-i}(cp^{\ell})^i$, with $i=0$ ommited since the $-1$ term is offset by the $+1$ term at $(cp^{\ell}-1)^{(p-1)^2-1}+1$. 
		
		\item Since $\dbinom{p(p-2)}{i}$ is divisible by $p$ for all $p\nmid i$ (well-known), this is also true for $i=2$, and since $\ell\ge 1$, $p^{2\ell+1}\mid \dsum_{i=2}^{p(p-2)}\dbinom{p(p-2)}{i}(-1)^{p(p-2)-i}(cp^{\ell})^i$. 
		
		\item If we consider $i=1$ we notice that the term is actually $p(p-2)cp^{\ell}$, with $v_p(p(p-2)cp^{\ell})=\ell+1$. 
	\end{itemize}
	so these points are enough to show that the highest power of $p$ dividing $x^{(p-1)^2-1}+1$ is $\ell+1$. But since this term is also divisible by $p^k$, we have $\ell\ge k-1$. 
	
	In other words, we have $x\ge p^{k-1}-1$ but then $x^{p-1}+y=p^k$ so $x^{p-1}<p^k$, i.e. $(p^{k-1}-1)^{p-1}<p^k$. Since $p\mid x+1$ anyways, we can safely assume that $(p^{k-1}-1)^{p-1}\ge (p-1)^{p-1}$. For $p\ge 5$, $(p-1)^{p-1}>p^2$ so we have $k\ge 3$, but then $(p^{k-2})^{p-1} < (p^{k-1}-1)^{p-1}<p^k$ so $(k-2)(p-1)<k$, which is impossible. Hence we only consider $p=3$, i.e. $(3^{k-1}-1)^2<3^k$. This holds true for $k=2$, but not any $k=3$, and for $k\ge 4$ we can reuse the result $(k-2)(p-1)<k$ i.e. $2(k-2)<k$ to produce a contradiction. 
	Therefore $k=2$, and $x^2+y=3^2=9$. Since $p\mid x+1$ with $x^2<9$, the only choice is $x=2$ and $y=5$, and it turned out that $5^2+2=27=3^3$ works too. 
	
	\item[\textbf{N6}] Let $a_1 < a_2 <  \cdots <a_n$ be pairwise coprime positive integers with $a_1$ being prime and $a_1 \ge n + 2$. On the segment $I = [0, a_1 a_2  \cdots a_n ]$ of the real line, mark all integers that are divisible by at least one of the numbers $a_1 ,   \ldots , a_n$ . These points split $I$ into a number of smaller segments. Prove that the sum of the squares of the lengths of these segments is divisible by $a_1$.
	
	\textbf{Solution.} Might as well give an alias for $a_1$, say $p$. 
	It's not hard to see that each segment must have length at most $p$. We show by fixing $p$ and inducting on $n$ that, if $x_1, \cdots , x_{p-1}$ represents the segments with lengths $1, 2, \cdots , p-1$, then there exists a polynomial $P(x)$ such that: 
	\begin{itemize}
		\item $P(i)=x_i$, $\forall i=1, \cdots , p-1$. 
		\item $P$ has coefficient that are rational numbers such that, when written in simplest form, the denominator is not divisible by $p$. 
		\item $P$ has degree at most $n-2$. 
	\end{itemize}
	We proceed by induction on $n$. When $n=1$ there's nothing to prove: we have a single segment of length $p$, so $x_1=\cdots =x_{p-1}=0$, which actually fits into the zero polynomial: the polynomial with degree $-1$, by our convention here. 
	
	Now suppose that for some $n$, the number of segments of length $p$ is $F(n)$ and let $P_n$ be a polynomial (satisfying our aforementioned conditions) such that for each $1\le i\le p-1$, the number of segments of length $i$ is $F(i)$. Now consider what happens when we add $a_{i+1}$. We have: 
	\begin{itemize}
		\item The segment extends from length $a_1\cdots a_n$ to $a_1\cdots a_{n+1}$. 
		
		\item If we do not mark the points divisible by $a_{n+1}$, we have $F(n)$ segments of length $p$ and $a_{n+1}P_n(i)$ segments of length $i$. 
	\end{itemize}
	Let's see what happens when we mark points divisible by $a_{n+1}$, which might or might not split a segment into two (it cannot split the a segment into three since $a_{n+1}>p$). 
	In the original configuration (where we have length of $I$ as $a_1\cdots a_n$), take $[x, x+i]$ as any segment of length $i$. 
	Consider, now, the $a_{n+1}$ copies of it: $[ja_{n+1}+x, ja_{n+1}+x+i]$ for $j=0, \cdots , a_{n+1}$. 
	For each $k=1, 2, \dots , i-1$, among the numbers $ja_{n+1}+x+k, j=0, \cdots , a_{n+1}$, from the fact that each $a_i$'s are pairwise coprime, exactly one of them is divisible by $a_{i+1}$. This means, $i-1$ of the segments are divided further into segments of length $(1, i-1), (2, i-2), \cdots , (i-1, 1)$, while the rest $a_{n+1}-i+1$ of them remain undivided. 
	
	Now for each $i$, we shall see how many segments are there have length $i$. Those must correspond to segments of length $j\ge i$ when considering only $a_1, \cdots, a_n$ and not $a_{n+1}$ since considering $a_{n+1}$ only splits them up (potentially). 
	By the fact above, the number of segments of length $i$ can be given by 
	\[P_n(i)(a_{n+1}-i+1) + \dsum_{j=i+1}^{p-1}2P_n(j) + 2F(n)
	\]
	We now consider 2 cases. If $n=1$ then $P_n\equiv 0$ so each term is $2F(n)$, i.e. a constant. 
	Otherwise, $P_n$ is assumed to have degree at most $n-2$. We now consider each term separately: 
	\begin{itemize}
		\item $2F(n)$ is constant throughout $i$, and is an integer (hence constant polynomial integer coefficient). 
		\item $P_n(i)(a_{n+1}-i+1) = (a_{n+1}+1)P_n(i) - iP_n(i)$. $(a_{n+1}+1)$ is constant while $\deg(iP_n(i)) = \deg(P_n)+1$ so by induction hypothesis, this $(a_{n+1}+1)P_n(i) - iP_n(i)$. $(a_{n+1}+1)$ has degree at most $n-1$. Moreover the polynomial of rational coefficient with denominator not divisible by $p$ is closed under multiplication and addition: the set $\{a/b: a\in\bbZ, \gcd(b, p)=1\}$ is a ring. 
	\end{itemize}
	We leave the middle term out deliberately: let $Q=\dsum_{j=1}^{p-1}2P_n(j)$, then $\dsum_{j=i+1}^{p-1}2P_n(j) = Q - \dsum_{j=1}^{i}2P_n(j)$. 
	Since $Q$ is a constant, we only need to consider the last term $\dsum_{j=1}^{i}P_n(j)$, which brings us to a lemma which directly addresses this subproblem: 
	
	\emph{Lemma}: let $P$ be a polynomial of degree $k\le p - 1$ with rational coefficients in terms of $\{a/b: a\in\bbZ, \gcd(b, p)=1\}$. Then if $Q(n)=\dsum_{i=1}^n P(i)$, $Q$ also has the same cclass of coefficients but with degree $k+1$. 
	
	Proof: Write $P(n) = \dsum_{i=0}^k a_ix^{k}$. Now, we need to solve for $Q=\dsum_{i=0}^{k+1}q_ix^i$ such that $Q(n)-Q(n-1) = P(n)$. Nevertheless, we also have the following: 
	\begin{flalign*}
		Q(n) - Q(n-1)
		&= \dsum_{i=0}^{k+1}q_i(n^i - (n-1)^i)
		\\&= \dsum_{i=0}^{k+1}q_i \dsum_{j=0^{i-1}}(-1)^{i-j-1}\dbinom{i}{j}n^j
	\end{flalign*}
	where each term $n^i-(n-1)^i$ is a polynomial of degree $i-1$ and leading coefficient $i$. 
	Putting this into the matrix form, this is what we need to solve: 
	\[
	\begin{pmatrix}
		k & -\dbinom{k}{2} & \cdots & (-1)^{k-1}\\
		0 & (k-1) & \cdots & (-1)^{k-2}\\
		\vdots\\
		0 & 0 & \cdots & 1\\
	\end{pmatrix}
	\begin{pmatrix}
		q_{k+1} \\ q_{k} \\ \vdots \\ q_1\\
	\end{pmatrix}
	=
	\begin{pmatrix}
	a_{k} \\ a_{k-1} \\ \vdots \\ a_0\\
	\end{pmatrix}
	\]
	and since this system of equations is upper triangular, it has determinant $k!$ which is relatively prime to $p$ since $k<p$. This means, this equation is solvable in $q_i$'s where our coefficients have denominator not divisible by $p$. Finally, we let $q_0=0$, and this proves the lemma. 
	
	Now we can finish the proof. Let $P_n(i)$ be the number of segments of length $i$, and we need to show that $p$ divides $\dsum_{i=1}^{p-1} i^2P_n(i)$. 
	Let $P_n(x) = \dsum_{i=0}^{n-2} a_ix^i$, then $\dsum_{i=1}^{p-1} i^2P_n(i) = \dsum_{i=0}^{n} a_n (\dsum_{j=1}^{p-1} j^{i+2})$. 
	Since $a_n$ has denominator not devisible by $p$, it suffices to show that $\dsum_{j=1}^{p-1} j^{i+2}$ is disivible by $p$ for all $0\le i\le n-2$. Let $g$ be the primitive root, then we can think $\dsum_{j=1}^{p-1} j^{i+2}$ as $\dsum_{j=0}^{p-2} (g^j)^{i+2}=\dfrac{g^{(p-1)(i+2)}-1}{g^{i+2}-1}$. 
	The numerator is disivible by $p$ by Fermat's Little theorem; the denominator is not because $g$ is a primitive root of $p$ but $i+2\le n<p-1$. The conclusion then follows. 
\end{enumerate}


\end{document}